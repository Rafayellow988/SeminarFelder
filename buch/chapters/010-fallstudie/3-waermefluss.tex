%
% 3-waermefluss.tex
%
% (c) 2024 Prof Dr Andreas Müller
%

%
% Wärmefluss
%
\section{Wärmefluss
\label{buch:fallstudie:waermefluss}}
\kopfrechts{Wärmefluss}
Betrachtet man einen von einer geschlossenen Fläche $S$ berandetes
Volumen in einem Körper, dann bewirken Temperaturunterschiede zwischen
Teilen des Körpers auf eiden Seiten der Fläche $S$, dass Wärme durch
die Fläche fliessen.
Der Gradient
\[
\operatorname{grad} T(x)
=
\nabla T
=
\renewcommand{\arraystretch}{1.9}
\begin{pmatrix}
\displaystyle \frac{\partial T}{\partial x_1}\\
\displaystyle \frac{\partial T}{\partial x_2}\\
\displaystyle \frac{\partial T}{\partial x_3}
\end{pmatrix}
\]
beschreibt die Richtung der grössten Temperaturzunahme.
Unterschieden sich zwei Punkte um den Vektor $\vec{v}$, dann ist der
Temperaturunterschied in erster Näherung gegeben durch das Skalarprodukt
$\vec{v}\cdot \nabla T$.
Sofern der Gradient parallel zur Fläche $S$ ist, kann keine Energie
durch die Fläche fliessen und die Energie im Inneren von $S$
ist erhalten.

Ist $\vec{n}$ in jedem Punkt der Fläche ein Vektor, der senkrecht
auf der Fläche steht, dann ist $\vec{n}\cdot \nabla T$ proportional
zum Wärmefluss.
Um den Energieverlust oder -gewinn durch die Fläche $S$ zu
berechnen, muss ein Integralbegriff konstruiert werden, der
nur vom Gradienten $\nabla T$ und der Oberfläche $S$ abhängt.
Die Parametrisierung der Oberfläche darf dabei genausowenig einen
Einfluss haben wie die Wahl des Koordinatensystems, mit dem der
Gradientvektor berechnet wird.

Ein weiteres Beispiel ist aus der Mechanik bekannt.
Das Gravitationsfeld $\smash{\vec{F}}$ als Vektorfeld kann als
Gradient $\smash{\vec{F}}=\nabla\varphi$ des Gravitationspotenials
$\varphi$ gefunden werden.
Die Arbeit, die zwischen den Punkten $A$ und $B$ geleistet
wird, wird durch das Integral
\[
W
=
\int_A^B \vec{F}\cdot d\vec{s}
=
\int_A^B \nabla \varphi \cdot d\vec{s}
\]
unabhängig vom gewählten Weg berechnet.
Auch in diesem Fall ist eine zentrale physikalische Grösse durch
ein Integral von Funktionen und Ableitungen gegeben, welches über
eine in diesem Fall eindimnensionale Untermannigfaltigkeit 
erstreckt wird.
Sowohl die Ableitungen wie auch die Parametrisierung des Weges
sind koordinatenabhängig.

\begin{aufgabe}
Konstruiere eine koordinatenunabhängige Theorie der Integration
von Funktioenn und Ableitungen auf beliebigen Untermannigfaltigkeiten
des Raumes und Sätze, die Bilanzbeziehungen wie die Energieerhaltung
beim Fluss Wärmeenergiefluss durch eine Oberfläche zu formulieren
erlauben.
\end{aufgabe}

