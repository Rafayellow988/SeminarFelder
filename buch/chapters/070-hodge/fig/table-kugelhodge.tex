%
% table-kugelhodge.tex
%
% (c) 2025 Prof Dr Andreas Müller
%
\begin{table}
\centering
\renewcommand\arraystretch{1.3}
\begin{tabular}{|>{$}c<{$}|>{$}c<{$}||>{$}c<{$}|>{$}c<{$}|}
\hline
\text{1-Form $\omega$} & \ast\omega & \text{2-Form $\alpha$}    & \ast\alpha\\
\hline
dr
& r^2\sin\vartheta\, d\vartheta\wedge d\varphi
& d\vartheta\wedge d\varphi
& \displaystyle \frac{1}{r^2\sin\vartheta}\,dr
\raisebox{9pt}{\mathstrut}
\\[7pt]
d\vartheta
& -\sin\vartheta\, dr\wedge d\varphi
& dr\wedge d\varphi
& \displaystyle -\frac{1}{\sin\vartheta}\,d\vartheta
\\[7pt]
d\varphi
& \displaystyle \frac{1}{\sin\vartheta}\, dr\wedge d\vartheta
& dr\wedge d\vartheta
& \sin\vartheta\, d\varphi
\\[7pt]
\hline
\end{tabular}
\caption{Tabelle der Wirkung des Hodge-Operators auf 1-Formen und 2-Formen
in Kugelkoordinaten.
\label{buch:hodge:skalarprodukt:table:kugelhodge}}
\end{table}
