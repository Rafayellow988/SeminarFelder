%
% chapter.tex -- Hodge-Operator, Vektoroperatoren, Codifferential und Laplace
%
% (c) 2025 Prof Dr Andreas Müller
%
\chapter{Hodge-Operator, Vektoranalysis und Laplace-Operator
\label{chapter:hodge}}
\kopflinks{Hodge-Operator, Vektoranalysis und Laplace-Operator}
In der Theorie der $p$-Formen wurde klar, dass der Vektorraum der
$p$-Formen $\binom{n}{p}$-dimensional ist.
Diese Eigenschaft folgt daraus, dass die Basisvektoren
$dx^{i_1}\wedge\dots\wedge dx^{i_p}$ durch die Auswahl von $p$ Elementen
aus den $n$ möglichen Basis-1-Formen $dx^1,\dots,dx^n$ gebildet
werden.
Der Raum der $n-p$-Formen hat wegen $\binom{n}{n-p}=\binom{n}{p}$
die gleiche Dimension.
Dies ist nicht nur ein Zufall.
Es nämlich eine natürlich invertierbare lineare Abbildung, die $p$-Formen
in $n-p$-Formen umwandelt, den Hodge-Operator, der in
Abschnitt~\ref{buch:hodge:section:hodge} definiert wird.
Die klassischen Operatoren der Vektoranalysis lassen sich als
Kombinationen des Hodge-Operators mit der äusseren Ableitung identifizieren,
was in Abschnitt~\ref{buch:hodge:section:vektoranalysis} durchgeführt
wird.

%
% Der Hodge-Operator
%
\section{Der Hodge-Operator
\label{buch:hodge:section:hodge}}
Der Hodge-Operator bildet $p$-Formen umkehrbar auf $n-p$-Formen ab.
Im folgenden wird er zunächst ad hoc mit Hilfe der Basis-$p$-Formen
konstruiert.

\subsection{Basisformen}
Der Raum der $p$-Formen hat als Basis die $p$-Formen der Form
\[
\alpha = dx^{i_1}\wedge\dots\wedge dx^{i_p}.
\]
Die Indizes $i_1,\dots,i_p$ müssen alle verschieden sein, weil
$dx^i\wedge dx^i=0$ ist.
Zu jeder Auswahl von $p$ Indizes $i_1,\dots,i_p$ aus der Mengen
$[n]=\{1,\dots,n\}$ der Zahlen von 1 bis $n$ gibt es genau eine Basisform.
Die Dimension des Raums der $p$-Formen ist daher die Anzahl der möglichen
Auswahlen von $p$ Zahlen aus den $n$ Zahlen $[n]$.

Die Auswahl von $p$ Zahlen aus $[n]$ lässt aber genau $n-p$ Basisformen
zurück.
Aus den Indizes $n-p$ Zahlen aus $]n]$, die in $i_1,\dots,i_p$ nicht
vorkommen, lässt sich eine Basis $n-p$-Form bilden.
Diese Zuordnung einer Basis-$p$-Form $dx^{i_1}\wedge\dots\wedge dx^{i_p}$
ist offensichtlich umkehrbar.
Es bleibt noch die Freiheit, das Vorzeichen festzulegen.

\begin{definition}[Hodge-Operator]
\label{buch:hodge:hodge:definition:hodge}
Der Hodge-Operator ist der lineare Operator, der der
Basis-$p$-Form
$dx^{i_1}\wedge\dots\wedge dx^{i_p}$ mit $i_1<i_2<\dots<i_p$
die $n-p$-Form
\[
\ast(dx^{i_1}\wedge\dots\wedge dx^{i_p})
=
s\,dx^{j_1}\wedge\dots\wedge dx^{j_{n-p}}
\]
mit $j_1<\dots <j_{n-p}$, $s\in\{\pm1\}$ zuordnet, für die
$\{i_1,\dots,i_p\}\cup\{j_1,\dots,j_{n-p}\}=[n]$ gilt.
Das Vorzeichen $s$ muss so gewählt werden, dass
\[
(dx^{i_1}\wedge\dots\wedge dx^{i_p})
\wedge
(s\,dx^{j_1}\wedge\dots\wedge dx^{j_{n-p}})
=
dx^1\wedge dx^2\wedge\dots\wedge dx^n
\]
gilt.
\end{definition}

Ganz allgemein lässt sich daraus bereits beantworten, wie $0$-Formen und
$n$-Formen durch den Hodge-Operator abgebildet werden.
Die einzige 0-Form ist $\alpha=1$.
Es muss eine $n$-Form $\beta$ gefunden werden, so dass
$\alpha\wedge\beta=dx^1\wedge\dots\wedge dx^n$ ist.
Da es nur die eine $n$-Form $dx^1\wedge\dots\wedge dx^n$ gibt, muss
nur noch das Vorzeichen gefunden werden.
Da aber
\[
\alpha\wedge
s\,dx^1\wedge\dots\wedge dx^n
=
s\,dx^1\wedge\dots\wedge dx^n
=
dx^1\wedge\dots\wedge dx^n
\]
sein soll, muss $s=1$ gewählt werden.
Somit ist $\ast 1=dx^1\wedge\dots\wedge dx^n$.
Auf die gleiche Art kann man folgern, dass
$\ast(dx^1\wedge\dots\wedge dx^n)=1$.

\begin{beispiel}
Hodge-Operator in $\mathbb{R}^2$.
Die Basis-1-Formen sind $dx^1$ und $dx^2$ und die Form höchsten Grades ist
die 2-Form $dx^1\wedge dx^2$.
Aus der Definition lässt sich jetzt sofort die Wirkung des Hodge-Operators
ableiten.
Für 0- und 2-Formen wurde dies im Anschluss an die
Definition~\ref{buch:hodge:hodge:definition:hodge} durchgeführt.

Die 1-Formen $dx^1$ und $dx^2$ werden aufeinander abgebildet, es müssen
aber noch die Vorzeichen ermittelt werden.
Wenn $\*dx^1=s\,dx^2$ ist, dann soll
$dx^1\wedge \*dx^1=dx^1\wedge s\,dx^2=dx^1\wedge dx^2$ sein.
Somit muss $s=1$ gewählt werden und es folgt $\*dx^1=dx^2$.
Wenn $\*dx^2=s\,dx^1$ ist, dann soll
$dx^2\wedge(\*dx^2)=dx^2\wedge(s\,dx^1)=dx^1\wedge dx^2$ sein.
Dies kann erreicht werden, dann man im mittleren Ausdruck der
Gleichungskette $dx^1$ und $dx^2$ vertauscht und dafür $s=-1$ setzt.
Es folgt $\*dx^2=-dx^1$.
\end{beispiel}

\begin{beispiel}
\label{buch:hodge:hodge:beispiel:r3}
Hodge-Operator in $\mathbb{R}^3$.
Im dreidimensionalen Fall ist nur die Abbildung von 1-Formen auf
2-Formen und zurück zu untersuchen.
\end{beispiel}

Da die Indexmengen $\{i_1,\dots,i_p\}$ und
$\{j_1,\dots,j_{n-p}\}$ in $[n]$ komplementär sein müssen,
muss
\(
\ast\ast dx^{i_1}\wedge\dots\wedge dx^{i_p}
=
\pm dx^{i_1}\wedge\dots\wedge dx^{i_p}
\)
sein.
Aus den Beispielen kann man ablesen, dass in den Fällen $n=2$ und $n=3$
der iterierte Hodge-Operator  $\ast\ast$ durch
\[
\left.
\begin{aligned}
\ast{\ast dx^1} &= \phantom{-}{\ast dx^2} = -dx^1\\
\ast{\ast dx^2} &=          - {\ast dx^1} = -dx^2
\end{aligned}
\quad
\right\}
\qquad
\text{bzw.}
\qquad
\left\{
\quad
\begin{aligned}
\ast{\ast dx^1} &= \phantom{-}{\ast(dx^2\wedge dx^3)} = dx^1 \\
\ast{\ast dx^2} &=          - {\ast(dx^1\wedge dx^3)} = dx^2 \\
\ast{\ast dx^3} &= \phantom{-}{\ast(dx^1\wedge dx^2)} = dx^3 
\end{aligned}
\right.
\]
gegeben ist.
Diese Überlegungen können verallgemeinert werden und ergeben den
folgenden Satz.

\begin{satz}
Für jede $p$-Form auf einer $n$-dimensionalen Mannigfaltigkeit gilt
\[
\ast{\ast \omega}
=
(-1)^{p(n-p)}\omega.
\]
\end{satz}

\begin{proof}
Sei $\omega = dx^{i_1}\wedge\dots\wedge dx^{i_p}$ und 
$\ast\omega = s\,dx^{j_1}\wedge\dots\wedge dx^{j_{n-p}}$ mit
$s\in\{\pm 1\}$.
Das Vorzeichen $s$ kommt von den Vertauschungen der Basis-1-Formen,
mit denen man die Standardreihenfolge $dx^1\wedge \dots\wedge dx^n$
erreicht.
Es erfüllt
\begin{equation}
(dx^{i_1}\wedge\dots\wedge dx^{i_p})
\wedge
s(dx^{j_1}\wedge\dots\wedge dx^{j_{n-p}})
=
dx^1\wedge\dots\wedge dx^n.
\label{buch:hodge:hodge:vorz1}
\end{equation}

Jetzt soll $\ast{\ast\,\omega}$ berechnet werden.
Nach Definition gibt es einen Vorzeichenfaktor $t\in\{\pm1\}$ mit
\[
\ast(dx^{j_1}\wedge\dots\wedge dx^{j_{n-p}})
=
t\,dx^{i_1}\wedge\dots\wedge dx^{i_p}.
\]
Er muss so gewählt werden, dass
\begin{align}
(dx^{j_1}\wedge\dots\wedge dx^{j_{n-p}})
\wedge
t(dx^{i_1}\wedge\dots\wedge dx^{i_p})
&=
dx^1\wedge\dots\wedge dx^n
\label{buch:hodge:hodge:vorz2}
\end{align}
gilt.
Für den iterierten Hodge-Operator findet man dann
\begin{align}
\ast{\ast\, \omega}
=
\ast(s\,dx^{j_1}\wedge\dots\wedge dx^{j_{n-p}})
&=
st\, dx^{i_1}\wedge\dots\wedge dx^{i_p}
\notag
\end{align}
Der iterierte Hodge-Operator ist also das Vorzeichen $st\in\{\pm 1\}$.

Durch $p$ Vertauschungen mit den 1-Formen $dx^{j_p}$,
$dx^{j_{p-1}},\dots,dx^{j_1}$ auf der linken Seite von
\eqref{buch:hodge:hodge:vorz2}
kann man $dx^{i_1}$ an den Anfang
der $n$-Form bringen.
Durch Widerholung für die 1-Formen $dx^{i_2}$ bis $dx^{i_p}$ kann
man mit insgesamt $p(n-p)$ Vertauschungen alle $dx^{i_k}$, $k=1,\dots,p$
nach vorne bringen und erhält so
\begin{align}
(-1)^{p(n-p)}
t(dx^{i_1}\wedge\dots\wedge dx^{i_p})
\wedge
(dx^{j_1}\wedge\dots\wedge dx^{j_p})
&=
dx^1\wedge\dots\wedge dx^n.
\end{align}
Durch Vergleich mit 
\eqref{buch:hodge:hodge:vorz1}
kann man ablesen, dass $(-1)^{p(n-p)}t=s$ gelten muss.
Daraus kann man
$st=(-1)^{p(n-p)}$ ablesen, was den Satz beweist.
\end{proof}

Wenn $n$ ungerade ist, dann ist immer einer der beiden Faktoren $p$
oder $n-p$ gerade und damit ist $(-1)^{p(n-p)}=1$.
Für ungerades $n$ ist der Hodge-Operator also eine Involution.
Für gerades $n$ sind entweder beide Zahlen $p$ und $n-p$ ungerade
oder keine von beiden.
Für $p$ ungerade ist auch $n-p$ und damit $p(n-p)$ ungerade
und entsprechend ist $\ast{\ast\omega}=-\omega$ für $p$-Formen.
Der iterierte Hodge-Operator ist also eine Antiinvolution auf
$p$-Formen ungeraden Grades.

\subsection{Hodge-Operator und Wedge-Produkt}

\subsection{Koordinatentransformation}

\subsection{Skalarprodukt und Hodge-Operator}

\subsubsection{Skalarprodukt der 1-Formen}
\subsubsection{Skalarprodukt der $p$-Formen}
\subsubsection{Definition des Hodge-Operators}
\subsubsection{Hodge-Operator in Polarkoordinaten}
\subsubsection{Hodge-Operator in Kugelkoordinaten}

\subsection{Hodge-Operator für $k$-Vektoren}

%
% Hodge-Operator und äussere Ableitung
%
\section{Hodge-Operator und äussere Ableitung}



%
% Operatoren der Vektoranalysis
%
\section{Operatoren der Vektoranalysis
\label{buch:hodge:section:vektoranlysis}}
Die klassische Vektoranalysis definiert dagegen mehr oder weniger
ad hoc eine Reihe von Operatoren auf Vektorfeldern, meistens nur
in kartesischen Koordinaten.
Feldgleichungen für Strömungsfelder oder für das elektromagnetische
Feld können dann mit diesen Operatoren formuliert werden, aber auch
diese nur in kartesischen Koordinaten.
Es entsteht die etwas unbefriedigende Situation, dass man nicht
sicher sein kann, dass die konstruierten Gleichungen Aussagen
darstellen, die nicht Artefakte der Wahl des Koordinatensystems
sind.

Die äussere Ableitung von $p$-Formen wurde als allgemeine Theorie
der offensichtlich koordinatenunabhängig definierbaren
Differentialoperatoren aufgebaut.
Wenn es gelingt, die klassischen Operatoren der Vektoranalysis
durch Operatoren auszudrücken, die koordinatenunabhängig formuliert
werden können, dann weiss man auch, dass die Gleichungen, die man
aus diesen Operatoren konstruiert, unabhängig sind von der Wahl
eines Koordinatensystems.
Gleichzeitig ermöglicht der allgemeine Formalismus, diese Operatoren
für beliebige Koordinatensysteme darzustellen.

Die klassiche Vektoranalysis arbeitet nur in Dimension 3, wir gehen
daher in den nachfolgenden Ausführungen normalerweise von einem
dreidimensionalen Raum aus.
Einige der Konstruktionen funktionieren allerdings auch in $n$.

%
% Gradient
%
\subsection{Gradient}
Die Richtungsableitung einer Funktion $f\colon M\to\mathbb{R}$ in 
\index{Richtungsableitung}%
Richtung des Tangentialvektors $X$ im Punkt $p\in M$ ist gegeben
durch das Differential
\[
X\cdot f(p) = \langle df, X\rangle.
\]
In einer Karte ist
\[
\langle df,X\rangle
=
\frac{\partial f}{\partial x^k}(p)
\cdot X^k
\]
Da dies wie ein Skalarprodukt aussieht, werden in der klassischen
Vektoranalysis die Koordinaten des Differentials mit den Komponenten
eines Vektors identifiziert.
Wir verwenden daher in den nachfolgenden jeweils die Identifikation
\[
V\colon
a_i\,dx^i \mapsto \begin{pmatrix} a_1\\a_2\\a_3\end{pmatrix}.
\]
In dieser Identifikation ist
\[
V(df)
=
V\biggl(
\frac{\partial f}{\partial x^1}\,dx^1
+
\frac{\partial f}{\partial x^2}\,dx^2
+
\frac{\partial f}{\partial x^3}\,dx^3
\biggr)
=
\renewcommand{\arraystretch}{1.8}
\begin{pmatrix}
\displaystyle\frac{\partial f}{\partial x^1}\\
\displaystyle\frac{\partial f}{\partial x^2}\\
\displaystyle\frac{\partial f}{\partial x^3}
\end{pmatrix}
\]
Die Identifikation von Vektoren mit 1-Formen ist nicht nur in 
3 Dimensionen möglich sondern auch für Formen auf $\mathbb{R}^n$.
Der {\em Gradient} als Differentialoperator ist also auch auf $\mathbb{R}^n$
\index{Gradient}%
definiert.

%
% Rotation
%
\subsection{Rotation}
Wir betrachten jetzt die äussere Ableitung der 1-Form
$\omega=f_i(x)\,dx^i$.
Sie ist definiert durch
\[
d\omega
=
\frac{\partial f_i}{\partial x^k}\,dx^k\wedge dx^i
\in
\Omega^3(\mathbb{R}^3).
\]
In drei Dimensionen ausgeschrieben ist dies
\begin{align*}
d\omega
&=\phantom{+}
\frac{\partial f_1}{\partial x^1}\,dx^1\wedge dx^1
+
\frac{\partial f_1}{\partial x^2}\,dx^2\wedge dx^1
+
\frac{\partial f_1}{\partial x^3}\,dx^3\wedge dx^1
\\
&
\phantom{=}+
\frac{\partial f_2}{\partial x^1}\,dx^1\wedge dx^2
+
\frac{\partial f_2}{\partial x^2}\,dx^2\wedge dx^2
+
\frac{\partial f_2}{\partial x^3}\,dx^3\wedge dx^2
\\
&
\phantom{=}+
\frac{\partial f_3}{\partial x^1}\,dx^1\wedge dx^3
+
\frac{\partial f_3}{\partial x^2}\,dx^2\wedge dx^3
+
\frac{\partial f_3}{\partial x^3}\,dx^3\wedge dx^3.
\intertext{Die 2-Formen $dx^i\wedge dx^i$ verschwinden alle und die
die 2-Formen mit verschiedenen Faktoren sind zum Teil noch nicht in
der Standardreihenfolge aufsteigender Indizes, wodurch zusätzlich
Vorzeichen auftreten:}
&=-
\frac{\partial f_1}{\partial x^2}\,dx^1\wedge dx^2
-
\frac{\partial f_1}{\partial x^3}\,dx^1\wedge dx^3
\\
&
\phantom{=}+
\frac{\partial f_2}{\partial x^1}\,dx^1\wedge dx^2
-
\frac{\partial f_2}{\partial x^3}\,dx^2\wedge dx^3
\\
&
\phantom{=}+
\frac{\partial f_3}{\partial x^1}\,dx^1\wedge dx^3
+
\frac{\partial f_3}{\partial x^2}\,dx^2\wedge dx^3.
\intertext{Durch Zusammenfassen gleicher Basis-2-Formen erhält man}
&=
\biggl(
\frac{\partial f_3}{\partial x^2}
-
\frac{\partial f_2}{\partial x^3}
\biggr)\,dx^2\wedge dx^3
+
\biggl(
\frac{\partial f_3}{\partial x^1}
-
\frac{\partial f_1}{\partial x^3}
\biggr)\,dx^1\wedge dx^3
+
\biggl(
\frac{\partial f_2}{\partial x^1}
-
\frac{\partial f_1}{\partial x^2}
\biggr)\,dx^1\wedge dx^2.
\end{align*}

$d\omega$ ist eine 2-Form ist, die nicht direkt mit einem Vektor identifiziert
werden kann.
Durch Anwendung des Hodge-Operators kann die 2-Form $d\omega$ in eine 1-Form
verwendelt werden, ohne dass dabei Information verloren geht.
Der Hodge-Operator auf $\mathbb{R}^3$ wurde in
Beispiel~\ref{buch:hodge:hodge:beispiel:r3} berechnet.
Angewendet auf $d\omega$ ergibt sich
\begin{align*}
\ast d\omega
&=
\biggl(
\frac{\partial f_3}{\partial x^2}
-
\frac{\partial f_2}{\partial x^3}
\biggr)\, dx^1
+
\biggl(
\frac{\partial f_1}{\partial x^3}
-
\frac{\partial f_3}{\partial x^1}
\biggr)\,dx^3
+
\biggl(
\frac{\partial f_2}{\partial x^1}
-
\frac{\partial f_1}{\partial x^2}
\biggr)\,dx^3.
\end{align*}
Man beachte den Vorzeichenwechsel beim mittleren Teil.
Diese 1-Form wird durch die Abbildung $V$ mit dem Vektor
\[
V(\ast d\omega)
=
\bgroup
\renewcommand\arraystretch{2.0}
\begin{pmatrix}
\displaystyle
\frac{\partial f_3}{\partial x^2}
-
\frac{\partial f_2}{\partial x^3}
\\
\displaystyle
\frac{\partial f_1}{\partial x^3}
-
\frac{\partial f_3}{\partial x^1}
\\
\displaystyle
\frac{\partial f_2}{\partial x^1}
-
\frac{\partial f_1}{\partial x^2}
\end{pmatrix}
\egroup
=
\operatorname{rot}
\begin{pmatrix}
f_1\\
f_2\\
f_3
\end{pmatrix}
=
\operatorname{rot}\vec{f}
\]
identifiziert.
Dies ist die {\em Rotation} des Vektorfeldes $\vec{f}$.
\index{Rotation}%
In der englischsprachigen Literatur wird die Rotation auch als
$\operatorname{curl}\vec{f}$ bezeichnet.
\index{curl}%
Sie kann formal auch als das Vektorprodukt mit dem Nabla-Operator
\[
\nabla\times\vec{f}
=
\operatorname{rot}\vec{f}
\]
geschrieben werden.

%
% Divergenz
%
\subsection{Divergenz}
Die Rotation wurde durch äussere Ableitung einer 1-Form erhalten.
Es war nötig, die Ableitung durch Anwendung des Hodge-Operators in eine
1-Form zu verwandeln, die dann wieder mit einem Vektor identifiziert
werden konnte.

Um die Ableitung einer 2-Form mit einer Vektoranalysis-Operation
zu identifizieren, muss erst eine 2-Form gewonnen werden.
Ein Vektor kann mit einer 1-Form identifiziert werden, die durch
den Hodge-Operator in eine 2-Form umgewandelt werden kann.
Wir beginnen daher mit dem Vektor $\vec{f}=V(f_i\,dx^i)=V(\omega)$
und wenden den Hodge-Operator darauf an:
\begin{align*}
\ast(f_i\,dx^i)
&=
f_1\, dx^2\wedge dx^3
-
f_2\, dx^1\wedge dx^3
+
f_3\, dx^1\wedge dx^2
\end{align*}
Bei der Berechnung der äusseren Ableitung kommen zu den Basis-2-Formen
weitere Basis-1-Formen hinzu, jedoch gibt es in jedem Term immer nur eine
einzige Basis-1-Form, die das Wedge-Produkt nicht zu 0 macht.
Daher ist die äusser Ableitung
\begin{align*}
d{\ast\omega}
&=
\frac{\partial f_1}{\partial x^1}dx^1\wedge dx^2\wedge dx^3
-
\frac{\partial f_2}{\partial x^2}dx^2\wedge dx^1\wedge dx^3
+
\frac{\partial f_3}{\partial x^3}dx^3\wedge dx^1\wedge dx^2
\intertext{Die 3-Formen auf der rechten Seite müssen in die
Standardreihenfolge gebracht werden und nehmen bei einer ungeraden
Anzahl Vertauschungen ein negatives Vorzeichen auf:}
&=
\biggl(
\frac{\partial f_1}{\partial x_1}
+
\frac{\partial f_2}{\partial x_2}
+
\frac{\partial f_3}{\partial x_3}
\biggr)
\,
dx^1\wedge dx^2\wedge dx^3.
\intertext{Dies ist eine 3-Form, in der klassischen Vektoranalysis gibt
es aber nur Funktionen und Vektoren.
Daher wird jetzt nochmals der Hodge-Operator angewendet, um die 3-Form
in eine Funktion zu verwandeln.
So entsteht die Gleichung}
\ast{d{\ast\omega}}
&=
\frac{\partial f_1}{\partial x_1}
+
\frac{\partial f_2}{\partial x_2}
+
\frac{\partial f_3}{\partial x_3}.
\end{align*}
Diese Funktion ist auch als die {\em Divergenz}
\index{Divergenz}%
\begin{equation}
\operatorname{div}\vec{f}
=
\sum_{i=1}^n \frac{\partial f_i}{\partial x^i}
\label{buch:hodge:hodge:divergenz:eqn:divdef}
\end{equation}
bekannt.
Formal kann sie auch als Skalarprodukt
\[
\operatorname{div}\vec{f}
=
\nabla\cdot\vec{f}
\]
mit dem Nabla-Operator geschrieben werden.

Die Divergenz eines Vektorfeldes ist also die Funktion, die durch
Anwendung des Operators $*{d}*$ auf die zugehörige 1-Form entsteht.
Dieser Operator funktioniert aber auch auf 1-Formen in $n$ Dimensionen.
Der Hodge-Operator macht aus einer 1-Form auf $\mathbb{R}^n$ eine
$n-1$-Form.
Die äussere Ableitung davon ist eine $n$-Form, die der Hodge-Operator
wieder zu einer Funktion macht.
Auch die Formel \eqref{buch:hodge:hodge:divergenz:eqn:divdef} gilt
für bliebige $n$.
%
% table-operatoren.tex
%
% (c) 2025 Prof Dr Andreas Müller
%
\begin{table}
\centering
\begin{tabular}{|>{$}c<{$}|>{$}c<{$}>{$}c<{$}|>{$}c<{$}|}
\hline
\text{Vektoranalsis} & p & \text{$p$-Formen} & \text{$\nabla$-Schreibweise}
\\
\hline
\operatorname{grad}  & 0 &  d                & \nabla 
\\
\operatorname{rot}   & 1 & *d                & \nabla\times\mathstrut
\\
\operatorname{div}   & 1 & *d*               & \nabla\cdot\mathstrut
\\
\hline
\end{tabular}
\caption{Korrespondenz wischen Operatoren der klassischen Vektoranalysis,
den Kombinationen von Hodge-Operator und äusserer Ableitung und der
Schreibweise der Operatoren der Vektoranalysis mit dem Nabla-Operator.
\label{buch:hodge:hodge:table:operatoren}}
\end{table}
%

%
% Operatorrelationen
%
\subsection{Operatorrelationen}
Die äussere Ableitung ist nilpotent vom Grad 2, d.~h.~die iterierte
äussere Ableitung $d^2=0$ verschwindet.
Zusammen mit den Identifikationen mit Vektorfeldern ergeben sich
daraus Kombinationen von Operatoren der Vektoranalysis, die
0 ergeben.

\subsubsection{Rotation eines Gradienten}
Der Gradient einer Funktion $f$ ist
\[
\operatorname{rot}\operatorname{grad}f
=
*{d}df
=
*d^2f
=
0
\]
verschwindet, weil der äussere Differentialoperator nilpotent mit
Ordnung $2$ ist.

\subsubsection{Divergenz einer Rotation}
Die Divergenz der Rotation eines Vektorfeldes ist
\[
\operatorname{div}\operatorname{rot} \vec{f}
=
*{d}{*}{*}d V^{-1}(\vec{f})
=
*d^2 V^{-1}(\vec{f})
=
0,
\]
wobei beim zweiten Gleichheitszeichen die Involutionseigenschaft
des Hodge-Operators und beim dritten die Nilpotenz der äusseren Ableitung
verwendet wird.

\subsubsection{Poincaré-Lemma in Vektoranalysis-Schreibeweise}
Das Poincaré-Lemma besagt, dass geschlossene Differentialformen auf
$\mathbb{R}^n$ exakt ist.
Eine geschlossene 0-Form ist eine Funktion $f$, deren Differential
$df=0$ verschwindet.
In einer Karte verschwinden alle partiellen Ableitungen, die Funktion
ist konstant.

Für $n=3$ gibt es zwei interessante Fälle, nämlich geschlossene 1-Formen
und geschlossene 2-Formen.
Wir übersetzen die Aussagen des Poincaré-Lemmas für diese beiden
Fälle in die Schreibweise der Vektoranalsis.

Eine geschlossene 1-Form $\omega$ entspricht einem Vektorfeld.
Geschlossen bedeutet, dass die Rotation dieses Vektorfeldes verschwindet.
Nach dem Poincaré-Lemma gibt es eine 0-Form $f$ derart dass
$df=\omega$ ist.
Die Identifikation von 1-Formen mit Vektorfeldern besagt also, dass
ein Vektorfeld, dessen Rotation verschwindet, ein Gradientfeld
einer Funktion $f$ ist.

Eine geschlossene 2-Form $\omega$ entspricht einer 1-Form $\ast\omega$,
auf die mit Hilfe des Hodge-Operators angewendet worden ist.
$d\omega=0$ bedeutet, dass die Divergenz des zu $\ast\omega$
gehörigen Vektorfeldes verschwindet.
Da $d\omega=0$ ist, gibt es nach dem Poincaré-Lemma eine 1-Form $\alpha$,
deren äussere Ableitung $d\alpha=\omega$ ist.
Folglich ist 
\[
d\alpha = \omega
\qquad\Rightarrow\qquad
\ast d\alpha = \ast\omega.
\]
Der 1-Form $\beta$ entspricht ein Vektorfeld, und $\ast d\beta$ entspricht
der Rotation dieses Vektorfelds.
Es folgt, dass ein Vektorfeld, dessen Divergenz verschwindet, als
Rotation eines anderen Vektorfeldes geschrieben werden kann.

\begin{satz}[Poincaré-Lemma der Vektoranalsis]
Wenn die Rotation eines Vektorfeldes $\vec{v}$ auf $\mathbb{R}^3$
verschwindet, dann gibt es eine Funktion $f$ mit
$\vec{v}=\operatorname{grad}f$.
Wenn die Divergenz eines Vektorfeldes $\vec{v}$ verschwindet,
dann gibt es ein Vektorveld $\vec{a}$, dessen Rotation
$\operatorname{rot}\vec{a}=\vec{v}$ das Vektorfeld ist.
\end{satz}

%
% Kodifferential
%
\section{Kodifferential}
Der $d$-Operator erhöht den Grad einer Differentialform um $1$.
Aus der Zusammensetzung mit dem Hodge-Operator entsteht ein neuer
Operator, das Kodifferential $\delta$, der den Grad erniedrigt.
Zusammen mit $d$ entstehen damit weitere Möglichkeiten, koordinatenunabhängige
Operatoren zu definieren.

\subsection{Definition}

\subsection{Vektoranalysis und Kodifferential}

\subsection{Poincaré-Lemma für das Kodifferential}

%
% Laplace-Operator
%
\section{Laplace-Operator}



