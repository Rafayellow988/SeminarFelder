%
% kurve.tex -- Kurve 
%
% (c) 2021 Prof Dr Andreas Müller, OST Ostschweizer Fachhochschule
%
\documentclass[tikz]{standalone}
\usepackage{amsmath}
\usepackage{times}
\usepackage{txfonts}
\usepackage{pgfplots}
\usepackage{csvsimple}
\usetikzlibrary{arrows,intersections,math,calc}
\begin{document}
\def\skala{1}
\begin{tikzpicture}[>=latex,thick,scale=\skala,
declare function={
	X(\t) = 0.04*cos(\t)*\t;
	Y(\t) = sin(\t);
}]

\begin{scope}[xshift=-5cm]
	\begin{scope}[yshift=-3.5cm]
		\draw[->] (-1.5,0) -- (1.5,0);
		\draw[line width=1.2pt] (-0.6,0) -- (0.6,0);
		\node at (-1.5,0) [above] {$\mathbb{R}$};
		\node at (1.5,0) [above] {$x$};
		\node at (-0.5,0) [above] {$U_1$};
		\fill (0,0) circle[radius=0.05];
		\node at (0,0) [below] {$x_1$};
	\end{scope}

	\draw[line width=1.4pt]
		plot[domain=-40:80,samples=100] ({X(\x)},{Y(\x)});
	\fill ({X(10)},{Y(10)}) circle[radius=0.05];
	\node at ({X(10)},{Y(10)}) [below right] {$x_0$};
	\node at ({X(80)},{Y(80)}) [left] {$X$};

	\draw[->,shorten <= 0.2cm] ({X(10)},{Y(10)}) -- ++(-0.4,-3.5);
	\node at ($({X(10)},{Y(10)})+(-0.16,-0.65)$) [left] {$\psi_2$};
\end{scope}

\begin{scope}[yshift=0.25cm]
	\fill[color=gray!40,rounded corners=0.5cm] (-2,-1.1) rectangle ++(4,2.2);
	\draw[->] (0,-1.2) -- ++(0,-1);
	\begin{scope}[xshift=-0.9cm,yshift=-3.6cm]
		\draw[->] (-0.6,0) -- (2.4,0) coordinate[label={$y^1$}];
		\draw[->] (0,-0.6) -- (0,1.4) coordinate[label={left:$y^n$}];
	\end{scope}
\end{scope}

\begin{scope}[xshift=5cm]
	\begin{scope}[yshift=2cm]
		\draw[->] (-2,0) -- (2,0);
		\draw[line width=1.2pt] (-1.0,0) -- (1.0,0);
		\node at (-2,0) [above] {$\mathbb{R}$};
		\node at (2,0) [above] {$x$};
		\node at (-0.5,0) [above] {$U_1$};
		\fill (0,0) circle[radius=0.05];
		\node at (0,0) [below] {$x_1$};
	\end{scope}

	\begin{scope}[yshift=-1.5cm]
		\draw[->] (-2,0) -- (2,0);
		\draw[line width=1.2pt] (-1.0,0) -- (1.0,0);
		\node at (-2,0) [above] {$\mathbb{R}$};
		\node at (2,0) [above] {$x$};
		\node at (-0.5,0) [above] {$U_2$};
		\fill (0,0) circle[radius=0.05];
		\node at (0,0) [below] {$x_2$};
	\end{scope}

	\draw[line width=1.4pt]
		plot[domain=-40:80,samples=100] ({X(\x)},{Y(\x)});
	\fill ({X(10)},{Y(10)}) circle[radius=0.05];
	\node at ({X(10)},{Y(10)}) [below right] {$x_0$};

	\draw[->,shorten <= 0.2cm] ({X(10)},{Y(10)}) -- ++(-0.32,1.3);
	\node at ($({X(10)},{Y(10)})+(-0.16,0.65)$) [left] {$\psi_1$};
	\draw[->,shorten <= 0.2cm] ({X(10)},{Y(10)}) -- ++(-0.32,-1.3);
	\node at ($({X(10)},{Y(10)})+(-0.16,-0.65)$) [left] {$\psi_2$};
\end{scope}

\end{tikzpicture}
\end{document}

