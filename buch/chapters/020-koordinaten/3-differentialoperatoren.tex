%
% 3-differentialoperatoren.tex -- Differentialoperatoren
%
% (c) 2024 Prof Dr Andreas Müller
%
\section{Differentialoperatoren
\label{buch:koordinaten:section:differentialoperatoren}}
\kopfrechts{Differentialoperatoren}
Der Tangentialvektor ist in
Definition~\ref{buch:koordinaten:tangentialvektoren:def:tangentialvektor}
sehr abstrakt definiert worden.
Tangentialvektoren sind das, was tangentialen Kurven gemeinsam ist,
also der Berührpunkt, die ``Richtung'' und ``Geschwindigkeit''.
Es ist aber nur indirekt mit Hilfe eines Koordinatensystems möglich, sich 
einen Tangentialvektor als einen ``Pfeil'' vorzustellen.
Richtung und Geschwindigkeit kann man aber auch dadurch detektieren, dass
man bestimmt, wie schnell sich eine Messgrösse, die von den Koordinaten
abhängit, ändert.
Die instantane Änderung einer Funktion ein einem Punkt ist so etwas
wie eine Ableitung.
In diesem Abschnitt soll daher gezeigt werden, dass man sich
Tagentialvektoren auch als Differentialoperatoren vorstellen
kann, die Funktionen ableiten können.

%
% Ableitung entlang einer Kurve
%
\subsection{Ableitung entlang einer Kurve}
Sei $X$ eine Menge versehen mit einem Koordinatensystem
$\varphi\colon X\to \mathbb{R}^n$ und sei ausserdem 
$f\colon X\to\mathbb{R}$ eine reellwertige Funktion definiert
auf $Y$.
Mithilfe des Koordinatensystems ist es möglich, die Funktion $f$
durch die Koordinaten $x^1,\dots,x^n$ als
\[
f(x^1,\dots,x^n) = f\circ\varphi^{-1} (x^1,\dots,x^n)
\]
auszudrücken.

Bereits in
Abschnitt~\ref{buch:koordinaten:koordinaten:subsection:koordinatenwechsel}
wurde erklärt, was es heisst, dass die Funktion $f$ stetig differenzierbar
ist.
Die Ableitungen der Funktion $f\circ\varphi^{-1}$ nach allen Koordinaten
$x^k$ müssen stetig sein.
Wir nehmen im folgenden an, dass die Funktion $f$ stetig differenzierbar
ist.

Sei jetzt eine differenzierbare Kurve in $X$ gegeben, die wir
der Einfachheit halber als eine differenzierbare Abbildung
\[
\gamma\colon (-\varepsilon,\varepsilon) \to X
\]
schreiben wollen und damit den Komplikationen durch verschiedene
Koordinatensysteme auf dem eindimensionalen Definitionsbereich der
Kurve für die folgende Untersuchung ignorieren.
Differenzierbarkeit bedeutet, dass die Zusammensetzung von $\gamma$
mit der Koordinatenabbildung $\varphi$ stetig differenzierbar ist.
Wir schreiben auch $x^i(t) = \varphi^i\circ\gamma(t)$ für die Koordinaten
eines Punktes, der sich auf der Kurve bewegt.

Die Zusammensetzung der Funktion $f$ mit der Kurve $\gamma$ ist eine
differenzierbare Funktion
\[
f\circ\gamma
\colon
(-\varepsilon,\varepsilon) \to \mathbb{R}
:
t
\mapsto
f(x^1(t),\dots,x^n(t)).
\]
Die Ableitung
\[
\frac{d}{dt} f(\gamma(t))
\bigg|_{t=0}
=
\frac{d}{dt} f(x^1(t),\dots,x^n(t))
\bigg|_{t=0}
\]
an der Stelle $t=0$ heisst die Ableitung von $f$ entlang der Kurve
an der Stelle $\gamma(0)$.
Sie kann mit der Kettenregel als
\begin{equation}
\frac{d}{dt} f(\gamma(t))
\bigg|_{t=0}
=
\frac{\partial f}{\partial x_k}(x^1(0),\dots,x^n(0))
\cdot
\frac{dx^k}{dt}(0)
\label{buch:koordinaten:differentialoperatoren:eqn:kurvenableitung}
\end{equation}
berechnet werden (man beachte die einsteinsche Summenkonvention).
Aus der linken Seite der Formel ist auch bereits klar, dass diese
Ableitung unabhängig ist vom gewählten Koordinatensystem auf $X$.

%
% Tangentialvektoren als Differentialoperatoren
%
\subsection{Tangentialvektoren als Differentialoperatoren}
Man erwartet, dass die Ableitung entlang der Kurve an der Stelle
$\gamma(0)$ eine lokale Eigenschaft ist, dass also nur der Verlauf
der Kurve in einer beliebig kleinen Umgebung des Punktes $\gamma(0)$
eine Rolle spielt.
Eine andere Kurve $\tilde{\gamma}:\colon(-\varepsilon,\varepsilon)\to X$,
die in $0$ tangential ist an die Kurve $\gamma$, sollte zur gleichen
Ableitung führen.
Wir schreiben sie auch $\tilde{\gamma}^i(t) = \tilde{x}^i(t)$.
Tatsächlich besagt die rechte Seite von
\eqref{buch:koordinaten:differentialoperatoren:eqn:kurvenableitung},
dass die Ableitung nur von den Ableitungen
\[
\dot{x}^i(0)
=
\frac{d\gamma^i}{dt}(0)
=
\frac{d\tilde{\gamma}^i}{dt}(0)
=
\dot{\tilde{x}}^i(0)
\]
abhängt.
Dies sind aber die Komponenten des Tangentialvektors.
Die Ableitung einer Funktion entlang einer Kurve verwendet von
der Kurve nur den Tangentialvektor.

\begin{definition}[Tangentialvektor als Differentialoperator]
Sei $f\colon X\to\mathbb{R}$ eine differenzierbare Funktion und
$V$ ein Tangentialvektor im Punkt $x_0\in X$.
Dann ist
\[
X\cdot f(x_0)
=
\frac{d}{dt} f(\gamma(t))\bigg|_{t=0}
\]
für jede Kurve mit dem Tangentialvektor $V$ an der Stelle $t=0$.
\end{definition}

Tangentialvektoren können also als Differentialoperatoren betrachtet
werden.

%
% Partielle Ableitungensoperatoren
%
\subsection{Partielle Ableitungsoperatoren}
Die Koordinatenlinien von
Abschnitt~\ref{buch:koordinaten:tangentialvektoren:subsection:koordinatenlinien}
sind Kurven in $X$.
Die Kurve
\[
\varphi\circ\gamma
\colon
(-\varepsilon,\varepsilon)
\to
\mathbb{R}^n
:
t\mapsto (x_0^1,\dots,x_0^k+t,\dots,x_0^n)
\]
durch den Punkt $x_0$ mit den Koordinaten $(x_0^1,\dots,x_0^n)$
hat als Tangentialvektor den $k$-ten Standardbasisvektor.
Die Ableitung einer Funktion $f$ entlang dieser Kurve ist
\begin{align*}
\frac{d}{dt}
f(x_0^1,\dots,x_0^k+t,\ddots,x_0^n)
\bigg|_{t=0}
&=
\frac{\partial f}{\partial y^k}(x_0^1,\dots,x_0^n).
\end{align*}
Man kann daher den $k$-ten Standardbasisvektor im Tangentialraum
des Punktes $y_0$ mit der partiellen Ableitung nach der Koordinate

\begin{satz}
In einem Koordinatensystem $(X,\varphi)$ auf $X$ bilden 
die partiellen Ableitungsoperatoren 
\[
e_k
\mapsto
\partial_k = \frac{\partial}{\partial x^k}
\]
nach den Koordinaten eine Basis.
\end{satz}

Dem Tangentialvektor mit den Komponenten $u^i$ im gegebenen
Koordinatensystem entspricht also der Differentialoperator
\[
u^k\partial_i = u^k\frac{\partial}{\partial x^k},
\]
also eine Linearkombination der partiellen Ableitungsoperatoren
(einsteinsche Summenkonvention).


