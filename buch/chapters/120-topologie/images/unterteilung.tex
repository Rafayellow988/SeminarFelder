%
% unterteilung.tex -- Unterteilung von Dreiecken einer Triangulation
%
% (c) 2021 Prof Dr Andreas Müller, OST Ostschweizer Fachhochschule
%
\documentclass[tikz]{standalone}
\usepackage{amsmath}
\usepackage{times}
\usepackage{txfonts}
\usepackage{pgfplots}
\usepackage{csvsimple}
\usetikzlibrary{arrows,intersections,math,calc}
\begin{document}
\definecolor{darkred}{rgb}{0.8,0,0}
\definecolor{darkgreen}{rgb}{0,0.6,0}
\def\skala{1}
\def\punkt#1#2{
	\fill[color=white] #1 circle[radius=0.05];
	\draw[color=#2] #1 circle[radius=0.05];
}
\begin{tikzpicture}[>=latex,thick,scale=\skala]

\begin{scope}[xshift=-3cm]
	\coordinate (A) at (-2,-2);
	\coordinate (B) at (2.0,-1.0);
	\coordinate (C) at (0,2.5);
	\coordinate (S) at (0.2,-0.4);
	\fill[color=darkred!10] (S) -- (A) -- (B);
	\fill[color=blue!10] (S) -- (B) -- (C);
	\fill[color=darkgreen!10] (S) -- (C) -- (A);
	\draw (A) -- (B) -- (C) -- cycle;
	\draw[color=darkred] (S) -- (A);
	\draw[color=darkred] (S) -- (B);
	\draw[color=darkred] (S) -- (C);
	\punkt{(A)}{black}
	\punkt{(B)}{black}
	\punkt{(C)}{black}
	\punkt{(S)}{darkred}
	\node at (S) [above right] {$S$};
	\fill[color=white,opacity=0.7] ($0.333*(A)+0.333*(S)+0.333*(B)$)
		circle[radius=0.16];
	\fill[color=white,opacity=0.7] ($0.333*(C)+0.333*(S)+0.333*(B)$)
		circle[radius=0.16];
	\fill[color=white,opacity=0.7] ($0.333*(A)+0.333*(S)+0.333*(C)$)
		circle[radius=0.16];
	\node[color=darkred] at ($0.333*(A)+0.333*(S)+0.333*(B)$) {$\scriptstyle 1$};
	\node[color=blue] at ($0.333*(C)+0.333*(S)+0.333*(B)$) {$\scriptstyle 2$};
	\node[color=darkgreen] at ($0.333*(A)+0.333*(S)+0.333*(C)$) {$\scriptstyle 3$};
	\node[color=darkred] at ($0.6*(A)+0.4*(S)$) [above] {$\scriptstyle 1$};
	\node[color=darkred] at ($0.6*(B)+0.4*(S)$) [above] {$\scriptstyle 2$};
	\node[color=darkred] at ($0.6*(C)+0.4*(S)$) [left] {$\scriptstyle 3$};
\end{scope}

\begin{scope}[xshift=3cm]
	\coordinate (A) at (-2.5,0);
	\coordinate (B) at (1.5,-0.5);
	\coordinate (C) at (-1,2.5);
	\coordinate (D) at (0,-2.0);
	\coordinate (M) at (-0.5,-0.25);
	\fill[color=darkred!10] (A) -- (B) -- (C) -- cycle;
	\fill[color=darkred!20] (M) -- (B) -- (C) -- cycle;
	\fill[color=blue!10] (A) -- (B) -- (D) -- cycle;
	\fill[color=blue!20] (M) -- (B) -- (D) -- cycle;
	\draw (A) -- (D) -- (B) -- (C) -- cycle;
	\draw[color=gray] (A) -- (B);
	\draw[color=darkred,dashed] (A) -- (B);
	\draw[color=darkred] (C) -- (M) -- (D);
	\punkt{(A)}{black}
	\punkt{(B)}{black}
	\punkt{(C)}{black}
	\punkt{(D)}{black}
	\punkt{(M)}{darkred}
	\node at (M) [above right] {$M$};
	\fill[color=white,opacity=0.7]
		($0.333*(A)+0.333*(M)+0.333*(C)$)
		circle[radius=0.16];
	\fill[color=white,opacity=0.7]
		($0.333*(B)+0.333*(M)+0.333*(C)$)
		circle[radius=0.16];
	\fill[color=white,opacity=0.7]
		($0.333*(A)+0.333*(M)+0.333*(D)$)
		circle[radius=0.16];
	\fill[color=white,opacity=0.7]
		($0.333*(B)+0.333*(M)+0.333*(D)$)
		circle[radius=0.16];
	\node[color=darkred] at ($0.333*(A)+0.333*(M)+0.333*(C)$) {$\scriptstyle 1$};
	\node[color=darkred] at ($0.333*(B)+0.333*(M)+0.333*(C)$) {$\scriptstyle 2$};
	\node[color=blue] at ($0.333*(A)+0.333*(M)+0.333*(D)$) {$\scriptstyle 3$};
	\node[color=blue] at ($0.333*(B)+0.333*(M)+0.333*(D)$) {$\scriptstyle 4$};
	\node[color=darkred] at ($0.7*(A)+0.3*(M)$) [above] {$\scriptstyle 1$};
	\node[color=darkred] at ($0.7*(D)+0.3*(M)$) [left] {$\scriptstyle 2$};
	\node[color=darkred] at ($0.7*(B)+0.3*(M)$) [above] {$\scriptstyle 3$};
	\node[color=darkred] at ($0.7*(C)+0.3*(M)$) [right] {$\scriptstyle 4$};
\end{scope}

\end{tikzpicture}
\end{document}

