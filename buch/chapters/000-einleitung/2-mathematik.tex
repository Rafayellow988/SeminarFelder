%
% 2-mathematik.tex
%
% (c) 2025 Prof Dr Andreas Müller
%
\section{Feldtheorie als mathematische Disziplin}
\kopfrechts{Feldtheorie als mathematische Disziplin}
Die newtonsche Infinitesimalrechnung war auf die Mechanik von Massepunkten
ausgerichtet.
Sie berechnet den Bewegungszustand aus den Beschleunigungen, die durch
ortsabhängige Kräfte hervorgerufen werden.
In moderner Schreibweise verwendet sie gewöhnliche Differentialgleichungen,
die als Lösung Funktionen einer einzigen Variable, nämlich der Zeit, haben.
Schreibt man die Funktion als $x(t)$, dann ist eine
gewöhnliche Differentialgleichung eine Beziehung zwischen den 
Ableitungen $\dot{x}(t)$, $\ddot{x}(t)$ und eventuell weiteren.
Im besten Fall kann eine solche Gleichung in expliziter Form als
\[
\ddot{x} = F(t, x, \dot{x})
\]
geschrieben werden.
Zuverlässige numerische Verfahren gestatten, genaue Lösungen zu finden.

Diese Mathematik ist ganz offensichtlich nicht geeignet für die
Beschreibung eines im Raum ausgedehnten Feldes, wo die relevanten
Variablen an jedem Punkt des Raumes anders sind.
Ein Feld wird notwendigerweise durch Funktionen beschrieben, die nicht
nur von der Zeit, sondern auch vom Ort abhängig sind.
Die Feldgleichungen sind daher partielle Differentialgleichungen.
\index{partielle Differentialgleichung}%
\index{Differentialgleichung!partiell}%
Doch führt nicht jede mögliche Kombination von Differentialoperatoren
auch auf eine physikalisch sinnvolle Differentialgleichung.
Die Feldtheorie befasst sich daher nicht nur mit der Frage, unter
welchen Voraussetzungen die partiellen Differentialgleichungen
wohlbestimmte Lösungen haben, sondern auch damit, welche Arten von
Differentialoperatoren in realistischen Feldgleichungen tatsächlich
vorkommen können.

