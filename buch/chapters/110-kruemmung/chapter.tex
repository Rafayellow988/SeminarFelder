%
% chapter.tex -- Krümmung
%
% (c) 2025 Prof Dr Andreas Müller
%
\chapter{Krümmung
\label{chapter:kruemmung}}
\kopflinks{Krümmung}
In Abbildung~\ref{buch:kruemmung:fig:exzess}
%
% fig-exzess.tex
%
% (c) 2025 Prof Dr Andreas Müller
%
\begin{figure}
\centering
\includegraphics{chapters/110-kruemmung/images/exzess.pdf}
\caption{Der Paralleltransport eines Tangentialvektors um ein 
sphärisches Dreieck führt wegen der Krümmung der Kugeloberfläche
zu einer Drehung des Vektors um den orangen Winkel beim Punkt $A$.
Für kleine Dreiecke bzw.~grosse Kugelradien nähert sich das
Dreieck einem ebenen Dreieck an und die Drehung wird sehr klein.
\label{buch:kruemmung:fig:exzess}}
\end{figure}

%
% fig-drehung.tex
%
% (c) 2025 Prof Dr Andreas Müller
%
\begin{figure}
\centering
\includegraphics{chapters/110-kruemmung/images/drehung.pdf}
\caption{Die Drehung eines Vektors beim Paralleltransport eines
Tangentialvektors ausgehend vom Punkt $A$ entlang der Seiten eines
sphärischen Dreiecks entlang der Seiten des Dreiecks resultiert
in einer Drehung um den blauen Winkel.
Er ist gleich gross wie die Summe der in jeder Ecke eingezeichneten
kleinen blauen Winkel, die anzeigen, wieviel grösser die Winkel
des sphärischen Dreiecks sind als die des ebenen Dreiecks
$\triangle ABC$.
Ihre Summe heisst daher der sphärische Exzess
$\varepsilon = \alpha+\beta+\gamma-\pi$.
\label{buch:kruemmung:fig:drehung}}
\end{figure}

sind gleichseitige sphärische Dreiecke zu unterschiedlichen
Kugelradien dargestellt.
Die Seiten $a=b=c$ sind gleich, ebenso die Winkel $\alpha=\beta=\gamma$.
Das Dreieck ganz rechts ist im Vergleich zum Kugelradius so klein,
dass die Krümmung kaum mehr sichtbar ist.
Nach dem Seitenkosinussatz
\[
\cos c = \cos a\cos b + \sin a \sin b \cos\alpha
\]
der sphärischen Trigonometrie folgt für das gleichseitige Dreieck
\begin{align*}
\cos a-\cos^2a
&=
\sin^2 a \cos\alpha
=
(1-\cos^2 a)\cos\alpha
\\
\cos a(1-\cos a)
&=
(1+\cos a)(1-\cos a)\cos\alpha
\intertext{und nach Division durch $1-\cos a$}
\frac{ \cos a }{ 1+\cos a }
&=
\cos\alpha.
\end{align*}
Im Grenzwert $a\to 0$ strebt $\cos a$ gegen 1.
Der Grenzwert für den Winkel $\alpha$ ergibt sich daher aus
$\cos\alpha=\frac12$ als $\alpha=60^\circ$.

Der Transport des Tangentialvektors entlang der Seite $AB$ eines
beliebigen sphärischen Dreiecks wie in
Abbildung~\ref{buch:kruemmung:fig:drehung}
liefert den Tangentialvektor an $AB$ im Punkt $B$.
Dieser schliesst mit der Seite $BC$ den Winkel $\beta$ ein.
Der Paralleltransport dieses Vektors ergibt einen Vektor im Punkt $C$,
der mit der Seite $CA$ einen Winkel $\beta+\gamma$ einschliesst.
Durch Paralleltransport zum Punkt $A$ ergibt sich eine Drehung um
insgesamt $\varepsilon= \alpha+\beta+\gamma -\pi$.
Die Grösse $\varepsilon$ ist bekannt als der {\em sphärische Exzess}
\index{Exzess, sphärisch}%
\index{sphärischer Exzess}%
Bei einem ebenen Dreieck verschwindet der sphärische Exzess,
beim Paralleltransport eines Vektors ist keine Drehung feststellbar.

Beim gleichseitigen sphärischen Dreieck von
Abbildung~\ref{buch:kruemmung:fig:exzess}
sind alle Winkel gleich gross, der sphärische Exzess und damit die
Drehung des Tangentialvektors ist daher $\varepsilon=3\alpha-\pi$. 
Er erreicht für ein Dreieck mit drei rechten Winkeln und drei
Seiten der Längen $\frac{\pi}2$ den Wert $\frac{\pi}2$.

Der sphärische Exzess eines sphärischen Dreiecks ist auch ein Mass für
den Flächeninhalt $\Delta F = (\alpha+\beta+\gamma-\pi)R^2$.
Je grösser der Flächeninhalt in Relation zur Oberfläche der Kugel,
desto grösser die Drehung beim Paralleltransport eines Vektors um
ein Dreieck.
Oder je grösser der Drehwinkel, desto grösser die Krümmung $1/R^2$
der Kugeloberfläche.
Die Krümmung einer Fläche kann also verallgemeinert werden als
der Grenzwert des Verhältnisses des Drehwinkels eines Vektors um einen
geschlossenen polygonalen Weg zum Flächeninhalt des Polygons für
sehr kleine Polygone.

In Abschnitt~\ref{buch:kruemmung:section:riemann} dieses Kapitels
wird der Riemannsche Krümmungstensor als eine lineare Abbildung,
die aus einem 2-Vektor eine Drehmatrix berechnet.
Zur Charakterisierung kann man aber auch nur einzelne Invarianten
der Drehmatrix verwenden.
Zum Beispiel lässt sich der Drehwinkel aus der Spur der Drehmatrix
berechnen.
Die Spur des riemannschen Krümmungstensors wird daher in 
Abschnitt~\ref{buch:kruemmung:section:riemann} als der Ricci-Tensor
eingeführt.
Aus dem Ricci-Tensor lassen sich die Einsteinschen Feldgleichungen
konstruieren, wie Abschnitt~\ref{buch:kruemmung:section:gravitation}
gezeigt wird.
Damit werden dann Aussagen über schwarze Löcher
(Abschnitt~\ref{buch:kruemmung:section:schwarzesloch})
und die Geschichte des Universums
(Abschnitt~\ref{buch:kruemmung:section:friedmann})
ableiten.

%
% Der riemannsche Krümmungstensor
%
\section{Der Krümmung
\label{buch:kruemmung:section:riemann}}

\subsection{Der riemannsche Krümmungstensor}
Das Einführungsbeispiel der Winkeldrehung eines paralleltransportieren
Vektors motiviert die Definition der Krümmung als die Abbildung 
der Tangentialvektoren beim Paralleltransport.
Ein Koordinatenparallelgramm mit den Ecken
\begin{center}
\begin{tikzpicture}[>=latex,thick]
\begin{scope}[xshift=-6cm,yshift=-0cm]
\node at (0,0) {$\displaystyle
\begin{aligned}
A &= x,\\
B &= x+\vec{h},\\
C &= x+\vec{h}+\vec{k}\\
\text{und}\quad
D &= x+\vec{k}
\end{aligned}$};
\end{scope}
\begin{scope}
\coordinate (A) at (-1.5,-1);
\coordinate (B) at (1.0,-0.5);
\coordinate (C) at (1.5,1);
\coordinate (D) at (-1.0,0.5);
\fill[color=gray!20] (A) -- (B) -- (C) -- (D) -- cycle;
\node at ($0.25*(A)+0.25*(B)+0.25*(C)+0.25*(D)$) {$\vec{h}\wedge\vec{k}$};
\draw[->] (A) -- (B);
\draw[->] (B) -- (C);
\draw[->] (C) -- (D);
\draw[->] (D) -- (A);
\fill (A) circle[radius=0.05];
\fill (B) circle[radius=0.05];
\fill (C) circle[radius=0.05];
\fill (D) circle[radius=0.05];
\node at (A) [below left] {$A$};
\node at (B) [below right] {$B$};
\node at (C) [above right] {$C$};
\node at (D) [above left] {$D$};
\node at ($0.5*(A)+0.5*(B)$) [below] {$\vec{h}$};
\node at ($0.5*(A)+0.5*(D)$) [left] {$-\vec{k}$};
\node at ($0.5*(B)+0.5*(C)$) [right] {$\vec{k}$};
\node at ($0.5*(D)+0.5*(C)$) [above] {$-\vec{h}$};
\end{scope}
\end{tikzpicture}
\end{center}
hat als Rand einen geschlossen Pfad, entlang dem ein Vektor parallel
transportiert werden soll.
Ein Vektor $A$ mit den Komponenten $a^i$ wird beim Transport entlang
des Weges linear abgebildet auf einen Vektor
$R(\vec{h}\wedge\vec{k})\cdot A$ abgebildet, der in linearer Näherung
nur vom 2-Vektor $\vec{h}\wedge\vec{k}$ und vom Vektor $A$ abhängt.
Der {\em riemannsche Krümmungstensor} ist daher die lineare Abbildung
\index{riemannscher Krümmungstensor}%
$R(\vec{h}\wedge\vec{k})$.

\subsubsection{Tangentialvektoren der Drehgruppen}
Der Krümmungstensor hängt antisymmetrisch von den Komponenten von
$\vec{h}$ und $\vec{k}$ ab und liefert eine Matrix, die von einer
Drehung herrührt.
Sei also $D(t)\in\operatorname{SO}(n)$ ein Weg in der Gruppe
der $n$-dimensionalen Drehmatrizen, die beim Parameterwert $t=0$
durch die Einheitsmatrix $I\in \operatorname{SO}(n)$ geht.
Er erfüllt
\begin{align*}
D(t)^tD(t) &= I
&&\text{und}&
\det D(t) &= 1.
\intertext{Die Ableitung nach $t$ an der Stelle $t=0$ ist}
\bigl(
\dot{D}(t)^tD(t)
+
D(t)^t\dot{D}^t
\bigr)\Bigl|_{t=0}
&=
0
&&\text{und}&
\operatorname{Spur}\dot{D}(0)
&=
0.
\end{align*}
Aus der linken Gleichung wird mit $D(0)=I$ die Bedingung
\[
\dot{D}(0) + \dot{D}(0)^t
\qquad\Rightarrow\qquad
\dot{D}(0)^t
=
-\dot{D}(0).
\]
Die Tangentialvektoren an die Drehgruppe sind als antisymmetrische
Matrizen mit verschwindender Spur.

\begin{beispiel}
Die Matrizen in der zweidimensionalen Drehmatrix sind
\[
D(t)
=
\begin{pmatrix}
\cos kt &          - \sin kt\\
\sin kt & \phantom{-}\cos kt
\end{pmatrix}
\]
mit der Ableitung
\[
\dot{D}(0)
=
\begin{pmatrix}
0 & -k\\
k &\phantom{-}0
\end{pmatrix}.
\]
\end{beispiel}

Die oben durchgeführten Rechnungen gelten nur in $\operatorname{SO}(n)$,
die mit dem Standardskalarprodukt definiert wird.
Auf einer beliebigen riemannschen Mannigfaltigkeit sind Drehmatrizen $D$
durch die Eigenschaft gegeben, dass
\[
D^t(t)GD(t)
=
G
\]
gilt.
Durch Ableitung nach $t$ an der Stelle $t=0$ folgt
\[
\dot{D}(0)^tG + G\dot{D}(0)
=
0
\qquad\Rightarrow\qquad
G\dot{D}(0)
=
-\dot{D}(0)^tG.
\]
Schreibt man die Komponenten von $\dot{D}(0)$ als $D^i\mathstrut_l$,
dann bedeutet dies
\[
g_{uv}D^{v}\mathstrut_l
=
-
\sum_{v}
D^{u}\mathstrut_v
g_{vl}
\]
Zieht man den oberen Index von $D^u\mathstrut_v$ mit herunter,
entsteht ein symmetrischer Tensor
\[
D_{uv}
=
g_{ui} D^i\mathstrut_v
\qquad\Rightarrow\qquad
D_{uv}
=
-D_{vu}.
\]

\subsubsection{Symmetrien des Krümmungstensors}
Der Krümmungstensor $R(\vec{h}\wedge\vec{k})$ hat als Wert eine Matrix,
deren Komponenten wir mit $R^i\mathstrut_l$ bezeichnen können.
Sie hängt antisymmetrisch von den Komponenten der beiden Vektoren
$\vec{h}\wedge\vec{k}$ ab.
In Komponenten ausgedrückt ist also
\[
R^i\mathstrut_l(\vec{h}\wedge\vec{k})
=
R^i\mathstrut_{luv} h^uk^v,
\]
wobei wie üblich nach der einsteinschen Summationskonventen über $u$
$v$ summiert werden muss.
Falls der metrische Tensor die Einheitsmatrix ist, muss $R^i\mathstrut_l$
antisymmetrisch sein, doch dies ist keine Eigenschaft, die allgemein
kovariant definiert ist.
Aus den Resultaten des letzten Abschnitts folgt aber, dass nach
Herunterziehen des oberen Index mithilfe von $g_{uv}$ eine
antismmetrische Matrix entsteht.
Für beliebige Vektoren $\vec{h}$ und $\vec{v}$ muss daher
$g_{li}R^i\mathstrut_l(\vec{h}\wedge\vec{k})=R_{il}(\vec{h}\wedge\vec{k})$
ein antisymmetrischer Tensor zweiter Stufe sein.
In Komponenten muss für beliebige Indizes $u$ und $v$ muss
$R_{jluv}=g_{ji}R^i\mathstrut_{luv}$ antisymmetrisch in den
Indizes $j$ und $l$ sein.

\subsubsection{Kovariante Ableitung und Krümmungstensor}
Wir berechnen jetzt den Paralleltransport des Vektors mit den
Komponenten $a^i$ entlang der Kantenpfade $ABC$ und $ADC$ 
des von $\vec{h}$ und $\vec{k}$ aufgespannten Parallelogramms.
Der Transport von $A$ nach $B$ ändert den Vektor um die Komponenten
\[
-\Gamma^l_{uv} a^u h^v 
\]
Der neue Vektor muss jetzt um den Vektor $\vec{k}$ transportiert
werden, der transportierte Vektor ist
\[
b^l
=
a^l-\Gamma^l_{uv} a^u h^v.
\]
Dies ergibt
\begin{align*}
\biggl(
\frac{\partial b^s}{\partial x^m}
+
\Gamma^s_{nm} b^n
\biggr)
k^m
&=
\biggl(
\frac{\partial \Gamma^s_{uv}}{\partial x^m}
a^u
h^v
-
\Gamma^s_{nm}
a^n
+
\Gamma^s_{nm} \Gamma^n_{uv}
a^u
h^v
\biggr)
k^m
\\
&=
\biggl(
\frac{\partial \Gamma^s_{uv}}{\partial x^m}
+
\Gamma^s_{nm}\Gamma^n_{uv}
\biggr)
a^u
h^v k^m
-
\Gamma^s_{nm}a^nk^m.
\end{align*}
Vertauschung von $v$ und $m$ führt auf den entsprechenden
Ausdruck für den Weg $ADC$.
Die Differenz ist der gesucht Krümmungstensor
\begin{align*}
R^s\mathstrut_{uvm}
&=
\frac{\partial\Gamma^s_{uv}}{\partial x^m}
-
\frac{\partial\Gamma^s_{um}}{\partial x^v}
-
\Gamma^s_{nm}\Gamma^n_{uv}
+
\Gamma^s_{nv}\Gamma^n_{um}
\end{align*}
Dies sind die Komponenten des riemannschen Krümmungstensors.
Durch herunterziehen des ersten Index erhält man die Komponenten
\[
R_{ruvm}
=
g_{rs}
\biggl(
\frac{\partial\Gamma^s_{uv}}{\partial x^m}
-
\frac{\partial\Gamma^s_{um}}{\partial x^v}
-
\Gamma^s_{nm}\Gamma^n_{uv}
+
\Gamma^s_{nv}\Gamma^n_{um}
\biggr)
\]
des kovarianten Riemannschen Tensors.
Er ist antisymmetrisch in den ersten zwei und den letzten zwei
Indizes.

\begin{beispiel}
Der riemannsche Krümmungstensor für die Ebene in Polarkoordinaten.
\end{beispiel}

\begin{beispiel}
Der riemannsche Krümmungstensor für die Kugeloberfläche.
\end{beispiel}

\subsubsection{Herleitung mit Hilfe des Satzes von Green}
Jeder dieser Wege kann für den Paralleltransport eines Vektors
verwendet werden, der Tangentialvektoren $T_AM$ im Punkt $A$ auf
Tangentialvektoren in $T_CM$ abbildet.

Sei also $g_{ik}$ die Metrik und $\Gamma^i_{jk}$ die
Zusammenhangskomponeten.
Die kovariante Ableitung ermöglicht, die Veränderung eines Vektors in
Koordinaten zu berechnen, wenn er in Richtung eines Vektors parallel
transportiert wird.
Der Vektor $Z\in T_AM$ mit den Komponenten $z^i$ wird beim Transport
entlang des Randes 

%
% Ricci-Krümmung
%
\section{Ricci-Krümmung
\label{buch:kruemmung:section:ricci}}

%
% Energie-Impuls-Tensor und die Feldgleichungen der Graviation
%
\section{Energie-Impuls-Tensor und die Feldgleichungen der Gravitation
\label{buch:kruemmung:section:gravitation}}

%
% Schwarze Löcher
%
\section{Schwarze Löcher
\label{buch:kruemmung:section:schwarzesloch}}

%
% Die Friedmann-Gleichung und die Geschichte des Universums
%
\section{Die Friedmann-Gleichung und die Geschichte des Universums
\label{buch:kruemmung:section:friedmann}}

