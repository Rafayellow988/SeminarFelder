%
% Abbildung von p-Formen
%
\section{Abbildung von $p$-Formen
\label{buch:pformen:section:abbildung}}
\kopfrechts{Abbildung von $p$-Formen}%
Beim Studium des Satzes von Stokes hat sich gezeigt, dass eine
Einbettung einer 2-dimensionalen Mannigfaltigkeit $N$ in eine beliebige
$n$-dimensionale Mannigfaltigkeit $M$ Tangentialvektoren von $N$ in
Tangentialvektoren von $M$ abgebildet werden, dass aber auch Linearformen
auf $M$ auf Linearformen in auf $N$ zurückgezogen werden können.
In diesem Abschnitt soll gezeigt werden, dass die Dimension 2
kein Spezialfall ist, sondern dass diese Konstruktion immer möglich ist

%
% Abbildungen zwischen Mannigfaltigkeiten
%
\subsection{Abbildungen zwischen Mannigfaltigkeiten}
In diesem Abschnitt sind $N$ und $M$ Mannigfaltigkeiten der Dimension
$n$ bzw.~$m$ und $f\colon N\to M$ ist eine beliebig oft stetig
differenzierbare Abbildung von $N$ nach $M$.
Eine Funktion $h\colon M\to\mathbb{R}$ auf der Mannigfaltigkeit gibt
unmittelbar Anlass zu einer Funktion $f^*h\colon N\to\mathbb{R}$,
die sich durch Zusammensetzung $f^*h=h\circ f$ ergibt.

%
% Abbildung von Vektoren
%
\subsubsection{Abbildung von Vektoren}
Die Ableitung von $f$ definiert eine lineare Abblidung der
Tangentialvektoren $T_pN$ von $N$ in einem Punkt $p$ in die
Tangentialvektoren $T_{f(p)}M$ von $M$ im Punkt $f(p)$.
Wir betrachten eine Karte $(V,\varphi)$ von $N$ in der Umgebung des Punktes
$p$ und eine Karte $(U,\psi)$ von $M$ in der Umgebung von $f(p)$.
Der Einfachheit halber nehmen wir an, dass $f(V)\subset U$ ist, was wird
nötigenfalls durch Verkleinern von $V$ immer erreichen können.
In den Karten $(V,\varphi)$ und $(U,\psi$ wird die Abbildung
$f$ zu einer glatten Abbildung
\[
\tilde{f}
=
\psi\circ f\circ\varphi^{-1}
\colon
\varphi(V)\to\psi(U)
:
q\mapsto \psi\circ f\circ \varphi^{-1}(q).
\]
Ist $\gamma\colon I\to \varphi(V)$ eine Kurve durch den Punkt $\varphi(p)$
in $\varphi(V)$, dann ist die Zusammensetzung $g\circ \gamma$
eine Kurve durch den Punkt $\psi(f(p))$ in $\psi(U)$.
In Koordinaten wird die Abbildung durch die Matrix der partiellen
Ableitungen
\[
Dg
=
\renewcommand\arraystretch{2.0}
\begin{pmatrix}
 \displaystyle\frac{\partial g^1}{\partial x^1}
&\displaystyle\frac{\partial g^1}{\partial x^2}
&\dots&
 \displaystyle\frac{\partial g^1}{\partial x^n}
\\
 \displaystyle\frac{\partial g^2}{\partial x^1}
&\displaystyle\frac{\partial g^2}{\partial x^2}
&\dots&
 \displaystyle\frac{\partial g^2}{\partial x^n}
\\[-1pt]
\vdots&\vdots&\ddots&\vdots
\\[-3pt]
 \displaystyle\frac{\partial g^m}{\partial x^1}
&\displaystyle\frac{\partial g^m}{\partial x^2}
&\dots&
 \displaystyle\frac{\partial g^m}{\partial x^n}
\end{pmatrix}
\]
gegegben.
Die Ableitung $D\tilde{f}$ bildet damit Tangentialvektoren von $N$
auf Tangentialvektoren von $M$ ab.

%
% Abbildung von p-Vektoren
%
\subsubsection{Abbildung von $p$-Vektoren}
Die Abbildung $Tf$ bildet kann auch auf Tensoren beliebiger Stufe und
insbesondere auf $p$-Vektoren ausgedehnt werden.
Die Abbildung $Tf^{\otimes p}$ bildet einen kontravarianten Tensor
$p$-ter Stufe gemäss
\[
\renewcommand{\arraycolsep}{1.5pt}
\begin{array}{rcccc}
Tf^{\otimes p}
=
Tf\otimes\dots\otimes Tf
&:&
\underbrace{TN\otimes\dots\otimes TN}_{\bigotimes TM}
&\to&
\underbrace{TM\otimes\dots\otimes TM}_{\bigotimes TM}
\\
&:&
X_1\otimes\dots\otimes X_p
&\mapsto&
Tf(X_1)\otimes\dots\otimes TF(X_p)
\end{array}
\]
ab.
Da $Tf^{\otimes p}$ linear ist, gilt auch
\begin{align*}
Tf^{\otimes p}(X_1\wedge\dots\wedge X_p)
&=
Tf^{\otimes p}\biggl(
\sum_{i_1,\dots,i_p=1}^{p}
\varepsilon_{i_1\dots i_p}
X_{i_1} \otimes\dots\otimes X_{i_p}
\biggr)
\\
&=
\sum_{i_1,\dots,i_p=1}^{p}
\varepsilon_{i_1\dots i_p}
Tf^{\otimes p}(X_{i_1}\otimes\dots\otimes X_{i_p})
\\
&=
\sum_{i_1,\dots,i_p=1}^{p}
\varepsilon_{i_1\dots i_p}
Tf(X_{i_1})\otimes\dots\otimes Tf(X_{i_p})
\\
&=
Tf(X_1)\wedge\dots\wedge Tf(X_p).
\end{align*}
Diese Abbildung  von $p$-Vektoren wird manchmal auch als
\[
Tf_*
=
{\textstyle\bigwedge^p}Tf
=
Tf^{\bigwedge p}
=
\underbrace{Tf\wedge\dots\wedge Tf}_{\displaystyle \text{$p$ Faktoren}}
\]
oder auch nur als $Tf$ bezeichnet.

%
% Abbildung von 1-Formen
%
\subsection{Abbildung von 1-Formen}
Eine 1-Form ist durch die Werte auf Tangentialvektoren
gegeben.
Falls die Dimension $n$ von $N$ kleiner ist als die Dimension $m$ von $M$,
bilden die Bildvektoren $Tf(X)$ von Vektoren $X\in T_pN$ einen höchstens
$n$-dimensionalen Unterraum.
Es ist daher unmöglich, die Werte einer Linearform auf den restlichen
Richtungen in $T_{f(p)}M$ festzulegen.
Man kann also nicht erwarten, dass es eine Abbildung von Linearformen
von $N$ auf Linearformen von $M$ gibt.

Umgekehrt sei jetzt $\alpha$ eine 1-Form auf $M$, also eine lineare
Abbildung $TM\to \mathbb{R}$.
Daraus lässt sich eine lineare Abbildung
$Tf^*(\alpha)\colon TN\to\mathbb{R}$ konstruieren.
Dazu muss für jeden Vektor $X\in T_pN$ der Wert auf dem Vektor $X$
festgelegt werden.
\[
\langle Tf^*(\alpha),X\rangle
=
\langle \alpha, Tf(X)\rangle.
\]
Da $Tf(X)\in T_{f(p)}M$ im Definitionsbereich von $\alpha$ ist, ist
die rechte Seite ist wohldefiniert.

In einer Karte ist die Abbildung besonders leicht durchzuführen.
Sei dazu in geeigneten Karten die Abbildung $f$ durch die Abbildung
\[
y
\colon
\varphi(U) \to \mathbb{R}^n 
:
x\mapsto y(x)
\]
gegeben.
Die Komponente $y^k(x)$ ist die $k$-te Koordinaten in der Karte von $M$.
Die Linearformen $\alpha$ auf $TM$ können in der Basis der 1-Formen $dy^k$ als
\[
\alpha
=
\sum_{k=1}^n a_k(x) \,dy^k
\]
geschrieben werden.
Die Differentiale von $y^k(x)$ sind
\[
dy^k
=
\sum_{i=1}^m
\frac{\partial y^k}{\partial x^i}\,dx^i
\]
und damit folgt
\begin{align*}
Tf^*(\alpha)
&=
\sum_{k=1}^n
a_k(x)
\sum_{i=1}^m
\frac{\partial y^k}{\partial x^i}\,dx^i
\\
&=
\sum_{i=1}^m
\biggl(
\sum_{k=1}^n
a_k(x)
\frac{\partial y^k}{\partial x^i}
\biggr)\,dx^i.
\end{align*}
Die Koeffizienten der 1-Form $Tf^*(\alpha)$ in der Basis der
1-Formen $dx^i$ ist daher durch das Matrizenprodukt
\[
\biggl(\frac{\partial(y^1,\dots,y^m)}{\partial(x^1,\dots,x^n)}\biggr)^t
\begin{pmatrix}
a_1(x)\\[-3pt]
\vdots\\[-1pt]
a_n(x)
\end{pmatrix}
\]
gegeben.

%
% Abbildung von p-Formen
%
\subsection{Abbildung $p$-Formen}
Die Abbildung $Tf^*$ von 1-Formen ist linear, daher kann sie ebenso
wie die Abbildung $Tf$ der Vektoren auch auf Tensorprodukte und
Wedge-Produkte von 1-Formen ausgedehnt werden.
Die Abbildung für das Tensorprodukt ist
\begin{align*}
Tf^*(dx^{i_1}\otimes\dots\otimes dx^{i_p})
&=
Tf^*(dx^{i_1}) \otimes\dots\otimes Tf^*(dx^{i_p})
\intertext{und für das Wedge-Produkt}
Tf^*(dx^{i_1}\wedge\dots\wedge dx^{i_p})
&=
Tf^*(dx^{i_1}) \wedge\dots\wedge Tf^*(dx^{i_p}).
\end{align*}

Falls $N\subset M$ eine Untermannigfaltigkeit ist, dann lässt sich
in der Umgebung eines Punktes $p\in N$ jeweils ein Koordinatensystem
von $M$ finden derart, dass $N$ in Koordinaten durch
$y^{n+1}=\dots=y^{m}=0$ gegeben ist.
Die Koordinaten $x^1=y^1,\dots,x^n=y^n$ sind ein lokales Koordinatensystem
von $N$.
Die Einbettungsabbildung ist
$i\colon (x^1,\dots,x^n)\mapsto(x^1,\dots,x^n,0,\dots,0)$
hat die Funktionalmatrix
\[
Di
=
\frac{\partial (y^1,\dots,y^n,y^{n+1},\dots,y^m)}{\partial(x^1,\dots,x^n)}
=
\begin{pmatrix}
1&0&\dots&0&0&\dots&0\\
0&1&\dots&0&0&\dots&0\\[-3pt]
\vdots&\vdots&\ddots&\vdots&\vdots&\ddots&\vdots\\
0&0&\dots&1&0&\dots&0
\end{pmatrix}.
\]
Die Transponierte von $Di$ bildet die Basis-1-Formen gemäss
\[
dy^i
\mapsto
\begin{cases}
dx^i&\qquad \text{für $1\le i\le n$}\\
0   &\qquad \text{für $i>n$}
\end{cases}
\]
ab.

