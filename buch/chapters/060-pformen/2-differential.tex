%
% Die äussere Ableitung
%
\section{Die äussere Ableitung
\label{buch:pformen:section:aeussereableitung}}
\kopfrechts{Die äussere Ableitung}
Für 1-Formen, 2-Formen und $n-1$-Formen wurde die äussere
Ableitung bereits definiert.
Auf Basisdifferentialformen ist die äussere Ableitung
durch die ausreichend verschiedenen Formeln
\begin{align*}
d(
f\,dx^i
)
&=
\frac{\partial f}{\partial x^k}\, dx^k\wedge dx^i
\\
d(g\,dx^i\wedge dx^k)
&=
\sum_{l=1}^n
\frac{\partial g}{\partial x^l}\, dx^l\wedge dx^i\wedge dx^k
\\
d(h\,
dx^1\wedge\dots\wedge \widehat{dx^i}\wedge\dots\wedge dx^n
)
&=
\frac{\partial h}{\partial x^i}
\,
dx^i\wedge
(dx^1\wedge\dots\wedge \widehat{dx^i}\wedge\dots\wedge dx^n)
\\
&=
(-1)^{n-1}
\frac{\partial h}{\partial x^i}
\,
dx^1\wedge\dots\wedge dx^n.
\end{align*}
gegeben.
Für die weitere Entwicklung brauchen wir eine Definition, die in allen
Fällen anwendbar ist.

%
% Definition der äusseren Ableitung
%
\subsection{Definition der äusseren Ableitung}
Die äussere Ableitung ist linear, es reicht daher, sie auf
Basisdifferentialformen zu definieren.

\begin{definition}[Äussere Ableitung]
\label{buch:pformen:def:aeussereableitung}
Die äussere Ableitung der Basis-$p$-Form 
$\alpha=f(x)\, dx^{i_1}\wedge\dots\wedge dx^{i_p}$
ist
\[
d\alpha
=
\sum_{k=1}^n
\frac{\partial f}{\partial x^k}
\,dx^k\wedge
dx^{i_1}\wedge\dots\wedge dx^{i_p}.
\]
\end{definition}

Für eine beliebige $p$-Form
\[
\omega
=
\sum_{i_1<\dots <i_p}
f_{i_1\dots i_p}\, dx^{i_1}\wedge\dots\wedge dx^{i_p}
\]
kann die äussere Ableitung daher als
\[
d\omega
=
\sum_{k=1}^n
\sum_{i_1<\dots <i_p}
\frac{\partial f_{i_1\dots i_p}}{\partial x^k}
\,dx^k
\wedge
dx^{i_1}
\wedge
\dots
\wedge
dx^{i_p}
\]
geschrieben werden.
Verwendet man vollständig antisymmetrische Koeffizienten wie in
\eqref{buch:pformen:pformen:eqn:summenformel}, bekommt die
äussere Ableitung die Form
\[
d\omega
=
\frac{1}{p!}
\sum_{i_1,\dots,i_p=1}^n
\frac{\partial f_{i_1\dots i_p}}{\partial x^k}
\,
dx^k\wedge dx^{i_1}\wedge\dots\wedge dx^{i_p}.
\]

%
% Äussere Ableitung und Wedge-Produkt
%
\subsection{Äussere Ableitung und Wedge-Produkt}
Wir berechnen die äussere Ableitung des Wedge-Produkts zweier
Differentialformen.
Seien also $\alpha\in\Omega^p(M)$ und $\beta\in\Omega^q(M)$
Differentialformen auf der Mannigfaltigkeit.
Um die äussere Ableitung des Produktes $\alpha\wedge\beta$ zu
berechnen, gehen wir von einfachen Monomen
\[
\alpha
=
f\,dx^{i_1}\wedge\dots\wedge dx^{i_p}
\qquad\text{und}\qquad
\beta
=
g\,dx^{i_{p+1}}\wedge\dots\wedge dx^{i_{p+q}}
\]
aus.
Das Produkt ist
\[
\alpha\wedge\beta
=
f(x)\,g(x)\,
dx^{i_1}\wedge\dots\wedge dx^{i_p}
\wedge
dx^{i_{p+1}}\wedge\dots\wedge d^{i_{p+q}}.
\]
Die äussere Ableitung ist
\begin{align*}
d(\alpha\wedge\beta)
&=
\sum_{k=1}^n
\frac{\partial(f\cdot g)}{\partial x^k}
dx^k
\wedge
dx^{i_1}\wedge\dots\wedge dx^{i_p}
\wedge
dx^{i_{p+1}}\wedge\dots\wedge d^{i_{p+q}}
\\
&=
\sum_{k=1}^n
\biggl(
\frac{\partial f}{\partial x^k}(x)
g(x)
+
f(x)
\frac{\partial g}{\partial x^k}(x)
\biggr)\,
dx^k
\wedge
dx^{i_1}\wedge\dots\wedge dx^{i_p}
\wedge
dx^{i_{p+1}}\wedge\dots\wedge d^{i_{p+q}}
\\
&=
\sum_{k=1}^n
\frac{\partial f}{\partial x^k}(x)\,
dx^k
\wedge
dx^{i_1}\wedge\dots\wedge dx^{i_p}
\wedge
g(x)\,
dx^{i_{p+1}}\wedge\dots\wedge d^{i_{p+q}}
\\
&\qquad +
\sum_{k=1}^n
f(x)\,
dx^k
\wedge
dx^{i_1}\wedge\dots\wedge dx^{i_p}
\wedge
\frac{\partial g}{\partial x^k}(x)\,
dx^{i_{p+1}}\wedge\dots\wedge d^{i_{p+q}}.
\intertext{Um den Faktor $dx^k$ im zweiten Summanden neben
die partielle Ableitung von $g$ zu bringen, sind $p$ Vertauschungen
nötig, die Ableitung wird dann}
&=
\biggl(
\sum_{k=1}^n
\frac{\partial f}{\partial x^k}(x)\,
dx^k
\wedge
dx^{i_1}\wedge\dots\wedge dx^{i_p}
\biggr)
\wedge
g(x)\,
dx^{i_{p+1}}\wedge\dots\wedge d^{i_{p+q}}
\\
&\qquad +
f(x)\,
dx^{i_1}\wedge\dots\wedge dx^{i_p}
\wedge
(-1)^p
\sum_{k=1}^n
\frac{\partial g}{\partial x^k}(x)\,
dx^k
\wedge
dx^{i_{p+1}}\wedge\dots\wedge d^{i_{p+q}}
\\
&=
d\alpha \wedge \beta
+
(-1)^p
\alpha\wedge d\beta.
\end{align*}
Abgesehen vom zusätzlichen Vorzeichen im zweiten Term ist dies
die Produktregel.

\begin{definition}
Eine {\em Antiderivation} $d$ einer graduierten Algebra $A$ ist eine
\index{Antiderivation}%
lineare Abbildung mit
\begin{equation}
d(\alpha\cdot\beta)
=
(d\alpha)\cdot\beta
+
(-1)^p \alpha\cdot(d\beta)
\label{buch:pformen:pformen:eqn:antiderivation}
\end{equation}
für $\alpha\in A^p$ und $\beta\in A^q$.
\end{definition}

Die äussere Ableitung ist also eine Antiderivation in der graduierten
Algebra der Differentialformen auf einer Mannigfaltigkeit.
Für eine Funktion $f\in\Omega^0(M)$, $p=0$, und $\beta\in\Omega^q(M)$ 
wird aus
\eqref{buch:pformen:pformen:eqn:antiderivation}
die bekannte Produktregel
\[
d(f\beta)
=
(df)\wedge \beta
+
f\wedge d\beta.
\]

%
% Geschlossene und exakte Formen
%
\subsection{Geschlossene und exakte Formen}
Als Anwendung des Satzes von Green wurde gezeigt, dass Wegintegrale
einer 1-Form $\alpha$, deren äussere Ableitung $d\alpha$ verschwindet,
vom Weg unabhängig sind.
Es gibt dann eine Funktion $f$, deren Differential $df=\alpha$ ist.
Die Idee kann auch auf $p$-Formen ausgedehnt werden.

\begin{definition}
Eine $p$-Form $\omega$ heisst {\em geschlossen}, wenn $d\omega=0$ ist.
\index{geschlossene $p$-Form}%
Eine $p$-Form heisst {\em exakt}, wenn es eine $p-1$-Form $\alpha$ gibt,
sodass $\omega = d\alpha$.
\index{exakte $p$-Form}%
\end{definition}


%
% Die zweite äussere Ableitung
%
\subsubsection{Die zweite äussere Ableitung}
Ist $\alpha\in\Omega^p(M)$ eine $p$-Form, dann ist $d\alpha\in\Omega^{p+1}(M)$
eine $p+1$-Form.
Sie kann natürlich noch einmal mit der äusseren Ableitung differenziert
werden.

\begin{satz}
\label{buch:section:pformen:satz:zweiteableitung}
Für eine $p$-Form $\alpha\in\Omega^{p}(M)$ gilt $dd\alpha=0$.
\end{satz}

\begin{proof}
Wegen der Linearität der äusseren Ableitung genügt es, die Identität für 
eine $p$-Form der Form
\[
\alpha
=
f(x^1,\dots,x^n)\, dx^{i_1}\wedge\dots\wedge dx^{i_p}
\]
nachzurechnen.
Ausserdem dürfen wir annehmen, dass $i_1<\dots<i_p$ ist.

Die äussere Ableitung von $\alpha$ ist nach
Definition~\ref{buch:pformen:def:aeussereableitung}
\[
d\alpha
=
\sum_{k=1}^n
\frac{\partial f}{\partial x^k}
\,
dx^k\wedge
dx^{i_1}\wedge\dots\wedge dx^{i_p}.
\]
Die $p+1$-Form muss jetzt ein weiteres Mal abgeleitet werden.
Wir schreiben den neuen Index für die Koordinate, nach der abgeleitet werden
soll, mit $l$ und erhalten
\begin{align}
dd\alpha
&=
\sum_{l=1}^n 
\sum_{k=1}^n
\frac{\partial f}{\partial x^l\,\partial x^k}
\,
dx^l\wedge
dx^k\wedge
dx^{i_1}\wedge\dots\wedge dx^{i_p}.
\\
&=
\sum_{l,k=1}^n
\frac{\partial f}{\partial x^l\,\partial x^k}
\,
dx^l\wedge
dx^k\wedge
dx^{i_1}\wedge\dots\wedge dx^{i_p}.
\label{buch:pformen:ddbeweis:summe}
\intertext{Natürlich verschwinden alle Terme, in denen $l$ oder $k$ mit einem
der Indizes $i_1,\dots,i_p$ übereinstimmt.
Ebenso verschwinden die Terme mit $l=k$.
Es gibt keine Einschränkung, welcher der beiden Indizes $l$ und $k$
in~\eqref{buch:pformen:ddbeweis:summe}
grösser ist, jeder kann grösser sein.
Unterscheiden wir die beiden Fälle, erhalten wir
}
&=
\sum_{l<k}
\frac{\partial f}{\partial x^l\,\partial x^k}
\,
dx^l\wedge
dx^k\wedge
dx^{i_1}\wedge\dots\wedge dx^{i_p}.
\notag
\\
&\qquad
+
\sum_{l>k}
\frac{\partial f}{\partial x^l\,\partial x^k}
\,
dx^l\wedge
dx^k\wedge
dx^{i_1}\wedge\dots\wedge dx^{i_p}.
\notag
\intertext{Wir können die beiden Summen in eine einzige vereinen,
indem wir in der zweiten Systeme die Namen der Indizes austauschen:}
&=
\sum_{l<k}
\frac{\partial f}{\partial x^l\,\partial x^k}
\,
dx^l\wedge
dx^k\wedge
dx^{i_1}\wedge\dots\wedge dx^{i_p}.
\notag
\\
&\qquad
+
\sum_{k>l}
\frac{\partial f}{\partial x^k\,\partial x^l}
\,
dx^k\wedge
dx^l\wedge
dx^{i_1}\wedge\dots\wedge dx^{i_p}.
\notag
\intertext{Beide Summen erstrecken sich über die Paare von Indizes mit
$l<k$, wir können sie daher als eine Summe}
&=
\sum_{l<k}
\biggl(
\frac{\partial f}{\partial x^l\,\partial x^k}
dx^l\wedge dx^k
+
\frac{\partial f}{\partial x^k\,\partial x^l}
\,
dx^k\wedge
dx^l
\biggr)
\wedge
dx^{i_1}\wedge\dots\wedge dx^{i_p}.
\notag
\intertext{schreiben.
Damit die 2-Formen in der Klammer aus der Klammer gezogen werden können,
müssen wir die Faktoren im zweiten Term in der Klammer vertauschen,
wobei das Vorzeichen wechselt.
Die Summe vereinfacht sich dann zu
}
&=
\sum_{l<k}
\biggl(
\frac{\partial f}{\partial x^l\,\partial x^k}
-
\frac{\partial f}{\partial x^k\,\partial x^l}
\biggr)
\,
dx^l\wedge dx^k
\wedge
dx^{i_1}\wedge\dots\wedge dx^{i_p}.
\notag
\end{align}
Nach dem Satz von Schwarz dürfen die partiellen Ableitungen
von $f$ nach $x^k$ und $x^l$ vertauscht werden.
Die beiden Terme in der Klammer stimmen daher überein, die Klammer
verschwindet.
Daraus folgt auch, das $dd\alpha=0$ ist.
\end{proof}

%
% Äussere Ableitungen sind geschlossen und exakt
%
\subsubsection{Äussere Ableitungen sind geschlossen und exakt}
Aus dem Satz~\ref{buch:section:pformen:satz:zweiteableitung} folgt,
dass die äussere Ableitung einer beliebigen $p$-Form eine geschlossene
Form ist.
Eine äussere Ableitung einer $p$-Form ist von der Form $\omega=d\alpha$,
sie ist also nach Definition auch exakt.

Etwas schwieriger ist die Frage, ob eine geschlossene $p+1$-Form $\omega$
exakt ist.
Dazu muss eine $p$-Form $\alpha$ mit $\omega=d\alpha$ gefunden werden.
Es ist nicht offensichtlich, wie dies möglich sein soll.
Nach den Erkenntnissen von Abschnitt~\ref{buch:green:section:geschlossen}
ist eine geschlossene 1-Form auf einem zusammenhängenden Gebiet exakt.
In Abschnitt~\ref{buch:pformen:geschlossen:subsection:exakt}
wird das Poincaré-Lemma~\ref{buch:pformen:geschlossen:satz:poincare-lemma}
bewiesen, welches zeigt, dass auf einem sternförmigen Gebiet eine
geschlossene $p$-Form exakt ist.
