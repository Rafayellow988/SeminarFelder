%
% greenrand2d.tex -- 2D-Darstellung der Rand-Situation
%
% (c) 2021 Prof Dr Andreas Müller, OST Ostschweizer Fachhochschule
%
\documentclass[tikz]{standalone}
\usepackage{amsmath}
\usepackage{times}
\usepackage{txfonts}
\usepackage{pgfplots}
\usepackage{csvsimple}
\usetikzlibrary{arrows,intersections,math,calc}
\definecolor{darkred}{rgb}{0.8,0,0}
\definecolor{darkgreen}{rgb}{0,0.6,0}
\begin{document}
\def\skala{1}
\begin{tikzpicture}[>=latex,thick,scale=\skala]

\begin{scope}[xshift=-3.3cm]
	\coordinate (P) at (-0.2,0);
	\begin{scope}
		\clip (-2.5,0) rectangle (2.5,1.5);
		\foreach \p in {1,...,40}{
			\pgfmathparse{sqrt(100-2*\p)/10}
			\xdef\faktor{\pgfmathresult}
			\fill[color=blue!\p] (0,0)
				ellipse({2*\faktor} and {1.4*\faktor});
		}
	\end{scope}
	\foreach \x in {-2.5,-2,...,2.5}{
		\draw[color=gray!30] (\x,0) -- ++(0,3.7);
	}
	\foreach \y in {0.5,1,...,3.5}{
		\draw[color=gray!30] (-2.6,\y) -- (2.6,\y);
	}
	\draw[->,color=violet,line width=1.6pt] (P) -- ++(0,1.5)
		coordinate[label={right:$\displaystyle\frac{\partial f_1}{\partial x^2}\,dx^1\wedge dx^2$}];

	\draw[->] (-2.6,0) -- (2.9,0) coordinate[label={$x^1$}];
	\draw[color=darkred,line width=1.4pt] (-2.6,0) -- (2.6,0);
	\draw[->] (-1,-0.1) -- (-1,4) coordinate[label={right:$x^2$}];
	\fill[color=darkgreen] (P) circle[radius=0.08];
	\draw[->,color=darkgreen,line width=1.5pt] (P) -- ++(2,0);
	\node[color=darkgreen] at ($(P)+(2,0)$) [above] {$Y$};
	\node[color=darkgreen] at (P) [below right] {$f_1\,dx^1$};
	\node at (-2.5,3.53) [below right] {$\varphi(U)$};
\end{scope}

\def\xk{-3}
\def\yk{5}

\pgfmathparse{sqrt((\xk)*(\xk)+(\yk)*(\yk))}
\xdef\r{\pgfmathresult}

\pgfmathparse{atan(-\xk/\yk)}
\xdef\w{\pgfmathresult}

\pgfmathparse{\w-90-32}
\xdef\wmin{\pgfmathresult}

\pgfmathparse{\w-90+32}
\xdef\wmax{\pgfmathresult}

\def\punkt#1#2{ ({\xk+(\r-#2)*cos(#1)},{\yk+(\r-#2)*sin(#1)}) }

\begin{scope}[xshift=3.3cm,yshift=0.5cm]
	\begin{scope}
		\clip (-3,-1.1) rectangle (3,3.7);
		\begin{scope}
			\clip (\xk,\yk) circle[radius=\r];
			\foreach \p in {1,...,40}{
				\pgfmathparse{sqrt(100-2*\p)/10}
				\xdef\faktor{\pgfmathresult}
				\fill[color=blue!\p,rotate=\w] (0,0.2)
					ellipse({2*\faktor} and {1.4*\faktor});
			}
		\end{scope}
		\foreach \x in {-2.5,-2,...,2.5}{
			\draw[color=gray!30] (\x,-1.1) -- (\x,3.1);
		}
		\foreach \y in {-1,-0.5,...,3}{
			\draw[color=gray!30] (-2.6,\y) -- (2.6,\y);
		}
		\draw[->] (-2.6,-0.5) -- (2.8,-0.5) coordinate[label={$y^1$}];
		\draw[->] (-2,-1.1) -- (-2,3.5) coordinate[label={right:$y^2$}];
		\draw[color=darkred,line width=1.4pt]
			({\xk+\r*cos(\wmin)},{\yk+\r*sin(\wmin)})
				arc(\wmin:{\wmax+10}:\r);
	\end{scope}
	%\fill (0,0) circle[radius=0.05];
	\pgfmathparse{\w-90+5}
	\xdef\winkel{\pgfmathresult}
	\draw[->,color=violet,line width=1.6pt]
		\punkt{\winkel}{0} -- ++({\winkel+180}:1.5);
	\begin{scope}[xshift=0.2cm]
		\node[color=violet] at \punkt{\winkel}{1.5} [above]
		{$\displaystyle\biggl(\frac{\partial f_1}{\partial y^2}-\frac{\partial f_2}{\partial y^1}\biggr)\,dy^1\wedge dy^2$};
	\end{scope}
	\fill[color=darkgreen] \punkt{\winkel}{0} circle[radius=0.08];
	\draw[->,color=darkgreen,line width=1.5pt]
		\punkt{\winkel}{0} -- ++({\winkel+90}:2);
	\node[color=darkgreen] at (2.2,1.5) {$Y$};
	\node[color=darkgreen] at \punkt{\winkel}{0}
		[below,rotate={\winkel+90}]
		{\hspace*{1.2cm}$f_1\,dy^1+f_2\,dy^2$};
	\node at (2.35,3.03) [below left] {$\psi(V)$};
	\node[color=darkred] at (-1.25,-0.8) {$\gamma$};
\end{scope}

\end{tikzpicture}
\end{document}

