%
% chapter.tex -- Kapitel 3: 1-Formen und Kurvenintegrale
%
% (c) 2024 Prof Dr Andreas Müller
%
\chapter{Differentialformen und Kurvenintegrale
\label{chapter:kurvenintegral}}
\kopflinks{Differentialformen und Kurvenintegrale}

\noindent
Ein Flug zum Mars muss Energie aufwenden, um gegen die Schwerkraft der
Sonne von der Erdbahn aus zu der 
Entfernung von der Sonne zu gewinnen.
In der Praxis wird dies dadurch erreicht, dass man dem Raumfahrzeug
mithilfe von Raketentriebwerken genügend kinetische Energie erteilt.
Es folgt dann einer elliptischen Bahn, die ihren sonnenfernsten Punkt
bei der Marsbahn hat.
Während des Fluges wird laufend kinetische Energie des Raumschiffs in
potenzielle Energie umgewandelt werden.
Mit jedem kleinen Teilstück der Flugbahn verliert das Raumschiff 
kinetische Energie, die proportional ist zur Kraftkomponente parallel
zur Flugbahn. 
Das potentielle Energiepaket das während des Teilstücks gewonnen wird,
ist linear vom Tangentialvektor der Bahn abhängig.
Die gesamte potentielle Energie, die zwischen Erde und Marsbahn
gewonnen wird, ist die Summe dieser Teilstücke.
Mathematisch wird es durch eine Art Integral einer Funktion dargestellt,
welches sowohl linear von der Bahntangente wie auch vom Pfad des
Raumschiffs abhängt.
Diese Integralkonstruktion muss aber so erfolgen, dass Sie nicht von der
Wahl von Koordinatensystemen und Parametrisierungen abhängt, was in
diesem Kapitel durchgeführt werden soll.
Sie muss ausserdem so funktionieren, dass sie sich über mehrer
Koordinatensysteme hinweg zusammensetzen lässt, wie dies bei einer
Bahn auf einer Mannifaltigkeit unvermeidlich wird.

\input{chapters/030-kurvenintegral/1-1formen.tex}
%
% Kurvenintegral einer 1-Form
%
\section{Kurvenintegral einer 1-Form}
Eine Kurve ist eine eindimensionale Untermannigfaltigkeit in
einer $n$-dimensionalen Mannigfaltigkeit $M$.
Eine $1$-Form auf der Mannigfaltigkeit $M$ führt zu einer 
$1$-Form auf der Kurve und damit wird es möglich, das Integral
dieser 1-Form auf der Kurve zu berechnen.
Dies soll in diesem Abschnitt schrittweise durchgeführt werden.

\subsection{$1$-Form auf einer Mannigfaltigkeit}
Eine $1$-Form auf einer $n$-dimensionalen Mannigfaltigkeit ist eine
Linearform, die auf einen Tangentialvektor angewendet werden kann
und einen Zahlenwert ergibt.
In einem Koordinatensystem mit den Koordinaten $x^1,\dots,x^n$ bilden
die Tangentialvektoren an die Koordinatenlinien eine Basis des
Tangentialraumes.
Die einzelnen Basisvektoren sind früher auch schon mit den
Ableitungsoperatoren $\partial/\partial x^i$ identifiziert worden.
Die Koordinaten-1-Formen $dx^1,\dots,dx^n$ ermitteln, wie schnell sich
die entsprechende Koordinate entlang eines Tangentialvektors ändert.
Da sich entlang einer Koordinatenlinie immer nur eine einzige
Koordinate ändert, gilt
\[
\biggl\langle dx^i,\frac{\partial}{\partial x^k}\biggr\rangle
=
\delta_{ik}
=
\begin{cases}
1&\qquad \text{falls $i=k$}\\
0&\qquad \text{sonst}.
\end{cases}
\]
Man nennt diese Basis der Linearformen auch die zur Basis der
Ableitungsoperatoren {\em duale Basis}.

Eine $1$-Form auf der Mannigfaltigkeit hat daher im Koordinatensystem
mit den Koordinaten $x^i$ die Form
\[
\alpha
=
\alpha_i(x)\, dx^i,
\]
wobei wieder die Summationskonvention gilt.

\begin{beispiel}
Die $1$-Form
\[
\beta
=
-
\frac{GMx^i}{r^3}
dx^i
\]
soll auf dem Impulsvektor eines Teilchens berechnet werden.
Wird die Bahnkurve mit der Zeit parametrisiert, hat der Geschwindigkeitsvektor
die Komponenten $\dot{x}^i$.
Der Impuls $p$ des Teilchens ist der Vektor mit den Komponenten
$p^i=m\dot{x}^i$.
Wendet man die Linearform $\beta$ darauf an, ergibt sich
\begin{align*}
\langle \alpha, p\rangle
&=
-
\frac{GMm}{r^3}
\biggl\langle \sum_i x^i\,dx^i, \sum_k\dot{x}^k\frac{\partial}{\partial x_k}\biggr\rangle
=
-
\frac{GMm}{r^3}
\sum_{i,k} x^i \biggl\langle dx^i,\dot{x}^k\frac{\partial}{\partial x_k}\biggr\rangle
\notag
\\
&=
-
\frac{GMm}{r^3}
\sum_{i,k}
x^i \delta_{ik} \dot{x}^k
=
-\frac{GMm}{r^3}
\sum_i x^i\dot{x}^i
\label{buch:kurvenintegral:kurvenintegral:eqn:gravimpuls}
\end{align*}
Dies ist die Leistung, die das Teilchen entgegen dem Gravitationsfeld 
arbeitet.
Eine $1$-Form ist also dazu geeignet, ein wichtiges physikalisches Konzept
des Gravitationsfeldes auszudrücken.

Die Form
\eqref{buch:kurvenintegral:kurvenintegral:eqn:gravimpuls}
ist allerdings noch nicht wirklich geeignet.
In der Formel wirden kontravariante Komponenten miteinander multipliziert
und summiert, dies führt nicht eine koordinatensystemunabhängige
Grösse.
Die Schwierigkeit rührt von der Darstellung des Impulses her, die später
noch modifiziert werden muss.
% XXX wo wird das geflickt?
\end{beispiel}

%
% Abbildung auf einer $1$-Form
%
\subsection{Abbildung einer $1$-Form}
Sei wieder eine $1$-Form $\alpha$ auf einer $n$-dimensionalen Mannigfaltigkeit
durch die Koeffizienten $\alpha_i(x)$ gegeben.
Sei ausserdem $f\colon N\to M$ eine differenzierbare Abbildung einer
$m$-dimensionalen Mannigfaltikgiet $N$.
Ein Tangentialvektor von $N$ im Punkt $p$ wird dargestellt durch eine
Kurve durch den Punkte $p$ in $N$.
Durch die Abbildung $f$ wird die Kurve auf eine Kurve durch den Bildpunkt
$q=f(p)$ abgebildet.
Da die Abbildung $f$ differenzierbar ist, entspricht der Bildkurve 
ein Tangentialvektor im Punkt $f(p)$.
Dies ist die tangentialabbildung $Tf\colon TN\to TM$, die Tangentialvektoren
von $N$ auf Tangentialvektoren von $M$ abbildet.

Eine $1$-Form berechnet aus einem Tangentialvektor einen Zahlenwert.
Die $1$-Form $\alpha$ auf $M$ tut dies für Tangentialvektoren $X$ von
$M$, nicht aber für Tangentialvektoren von $N$.
Dazu muss erst die Abbildung $Tf$ angewendet werden, die aus dem
Tangentialvektor $Y\in T_pN$ einen Tangentialvektor $T_pf(Y)=X\in T_qM$
macht.
Die Abbildung
\[
Y\mapsto \langle \alpha, T_pf(Y)\rangle \in \mathbb{R}
\]
ist eine lineare Abbildung, denn die Abbildung $T_pf$ ist linear und
Auswertung $X\mapsto \langle\alpha,X\rangle$ ist ebenfalls linear.
Die $1$-Form $\alpha$ auf $M$ definiert damit auch eine $1$-Form auf
$N$, die wir mit $T_pf^*\alpha$ bezeichnen.

\begin{beispiel}
\label{buch:kurvenintegral:kurvenintegral:beispiel:loxodrome}
%
% fig-loxodrome.tex
%
% (c) 2025 Prof Dr Andreas Müller
%
\begin{figure}
\centering
\includegraphics{chapters/030-kurvenintegral/images/loxodrome.pdf}
\caption{Die Loxodrome auf einer Kugeloberfläche schneidet die Längen- und
Breitenkreise unter konstantem Winkel.
Als Abbildung $\mathbb{R}\to S^2$ transportiert die Loxodrome die
$1$-Form $\sin^2\vartheta\,d\varphi$ auf der Kugeloberfläche auf die
reelle Achse und macht daraus die $1$-Form $(1-\tanh^2 kt)\,dt$.
\label{buch:kurvenintegral:fig:loxodrome}}
\end{figure}
%
Sei $N=\mathbb{R}$ eindimensional und $M=S^2$ eine Kugel.
Die Abbildung 
\begin{equation}
f
\colon
\mathbb{R}\to S^2
:
t
\mapsto 
\frac{1}{\cosh kt}
\begin{pmatrix}
\cos t\\
\sin t\\
\sinh kt
\end{pmatrix}
\label{buch:kurvenintegral:kurvenintegral:eqn:kugel}
\end{equation}
beschreibt eine Kurve auf der Kugeloberfläche, die als die
{\em Loxodrome} bekannt ist.
Sie schneidet die Längenkreise unter konstantem Winkel $\arctan k$
(Abbildung~\ref{buch:kurvenintegral:fig:loxodrome}).
Wir betrachten die $1$-Form $\sin^2\vartheta\cdot d\varphi$, wobei
$(\varphi,\vartheta)$ die Längen- und Breitenkoordinaten auf der
Kugeloberfläche sind.

Um den Tangentialvektor in den $(\varphi,\vartheta)$-Koordinaten zu
bestimmen, muss als erstes der Bildpunkt der Abbildung $f$ in Länge
und Breite umgerechnet werden.
Zunächst ist aus der Darstellung
\eqref{buch:kurvenintegral:kurvenintegral:eqn:kugel}
kann man sofort ablesen, dass $t=\varphi$ die geographische Länge ist,
es bleibt also nur noch die geographische Breite abzulesen.
Diese lässt sich aber sofort aus der $z$-Komponente ablesen:
\[
\cos\vartheta
=
\tanh kt
\qquad\Rightarrow\qquad
\vartheta = \arccos\tanh kt.
\]
Damit kann jetzt der Tangentialvektor berechnet werden, werden:
\begin{align*}
\frac{d}{dt} \bigl(\varphi(t),\vartheta(t)\bigr)
&=
\biggl(
\frac{d\varphi(t)}{dt},
\frac{d\vartheta(t)}{dt}
\biggr)
\\
&=
\biggl(1,
-\frac{k\operatorname{sech}^2(kt)}{\sqrt{1-\tanh^2(kt)}}
\biggr)
\\
&=
\frac{\partial}{\partial \varphi}
-
\frac{k\operatorname{sech}^2(kt)}{\sqrt{1-\tanh^2(kt)}}
\frac{\partial}{\partial\vartheta}
.
\end{align*}
Damit sind die Komponenten des Tangentialvektors in
$(\varphi,\vartheta)$-Koordinaten gefinden.

Wir wenden jetzt die $1$-Form $\sin^2\vartheta\, d\varphi$ darauf
an und erhalten:
\begin{align*}
\biggl\langle
\sin^2\vartheta\,d\varphi,
\biggl(1,
-\frac{k\operatorname{sech}^2(kt)}{\sqrt{1-\tanh^2(kt)}}
\biggr)
\biggr\rangle
&=
\sin^2\vartheta
\underbrace{
\biggl\langle
d\varphi,\frac{\partial}{\partial \varphi}
\biggr\rangle
}_{\displaystyle = 1}
-
\frac{k\operatorname{sech}^2(kt)}{\sqrt{1-\tanh^2(kt)}}
\underbrace{
\biggl\langle
d\varphi,\frac{\partial}{\partial\vartheta}
\biggr\rangle
}_{\displaystyle = 0}
\\
&=
\sin^2\vartheta(t)
\\
&=
\sin^2 \arccos \tanh kt
\\
&=
1-\tanh^2 kt
\end{align*}
Die $1$-Form $Tf^*\alpha$ auf $\mathbb{R}$ hat daher die Form 
$Tf^*(\sin^2\vartheta\,d\varphi)=(1-\tanh^2 kt)\,dt$.
\end{beispiel}


%
% Integral entlang einer Kurve
%
\subsection{Integral entlang einer Kurve}
Da sich jede $1$-Form auf einer Mannigfaltigkeit $M$ mit Hilfe einer
Kurve auf das Definitionsgebiet der Kurve transportieren lässt, lässt
sich das in Abschnitt~\ref{buch:1formen:subsection:integral} auf einem
eindimensionalen Definitionsgebiet definierte Integral auf ein
Kurvenintegral transportieren.

\begin{definition}
Ist $f\colon [a,b]\to M$ eine Kurve auf der Mannigfaltigkeit zwischen 
den Punkten $A=f(a)$ und $B=f(b)$ und $\alpha$ eine $1$-Form auf $M$,
dann ist das Kurvenintegral von $\alpha$ entlang der Kurve gegeben
durch
\[
\int_{AB} \alpha
=
\int_{[a,b]}
(Tf)^*(\alpha).
\]
In einer Karte mit Koordinaten $x^i$ wird die $1$-Form durch Koordinaten
$\alpha_i\,dx^i$ und die Kurve durch die
Parametrisierung $t\mapsto x^i(t)$ beschrieben.
Darin wird das Integral
\[
\int_{AB} \alpha
=
\int_a^b \alpha_i(f(t)) dx^i
=
\int_a^b \alpha_i(f(t))\,\dot{x}^{i}(t)\,dt.
\]
\end{definition}

\begin{beispiel}
Man berechne das Kurvenintegral der $1$-Form $\alpha=-x^2\,dx^1+x^1\,dx^2$ auf
der zweidimensionalen Ebene mit kartesischen Koordinaten $(x^1,x^2)$
entlang des Einheitskreises $S^1\subset\mathbb{R}^2$.
\smallskip

\noindent
Als Parametrisierung des Einheitskreises verwenden wir die Abbildung
\[
f
\colon
[0,2\pi]
\to
\mathbb{R}^2
:
t
\mapsto
\begin{pmatrix} x^1(t)\\x^2(t)\end{pmatrix}
=
\begin{pmatrix} \cos t\\\sin t\end{pmatrix}.
\]
Die 1-Formen werden durch $Tf^*$ auf die Formen
\[
\left.
\begin{aligned}
(Tf)^*(dx^1) &= -\sin t\,dt           &&\Rightarrow& (Tf)^*(-x^2\,dx^1) &= \sin^2 t\,dt\\
(Tf)^*(dx^2) &= \phantom{-}\cos t\,dt &&\Rightarrow& (Tf)^*(\phantom{-} x^1\,dx^2) &= \cos^2 t\,dt
\end{aligned}
\;
\right\}
\quad\Rightarrow\quad
(Tf)^*(\alpha)
%=
%(\sin^2 t+\cos^2 t)\,dt
=
dt
\]
abgebildet.
Das Kurvenintegral ist daher
\[
\oint_{S^1} \alpha
=
\int_0^{2\pi} dt
=
\bigl[
t
\bigr]_0^{2\pi}
=
2\pi.
\qedhere
\]
\end{beispiel}

\begin{beispiel}
Man berechne das Integral der $1$-Form $\in^2\vartheta\,d\varphi$ auf
der Kugeloberfläche entlang der Loxodrome $L(-a,a)$ zwischen den Punkten
mit dem Parameter $-a$ und $+a$.
\smallskip

\noindent
In Beispiel~\ref{buch:kurvenintegral:kurvenintegral:beispiel:loxodrome}
wurde bereits die $1$-Form auf $\mathbb{R}$ transportiert und dafür
\[
(1-\tanh^2 kt)\,dt
\]
gefunden.
Wegen
\[
\tanh' t
=
1-\tanh^2 t
=
\frac{1}{\cosh^2 t}
\qquad\Rightarrow\qquad
\frac{1}{k}
\frac{d}{dt}
\tanh(kt)
=
1-\tanh^2 kt
\]
ist das gesuchte Integral also
\begin{align*}
\int_{L(-a,a)} \sin^2\vartheta\,d\varphi
&=
\int_{-a}^a
(1-\tanh^2 kt)
\,dt
=
\biggl[
\frac{1}k
\tanh kt
\biggr]_{-a}^a
=\frac{2}{k} \tanh ka.
\end{align*}
Im letzten Schritt wurde benutzt, dass die Funktion $t\mapsto \tanh t$
ungerade ist und dass für eine ungerade Funktion $g(x)$ gilt
$g(a)-g(-a)= g(a)-(-g(a))=2g(a)$.
\end{beispiel}

%
% Kurvenintegral eines Vektorfeldes
%
\subsection{Kurvenintegral eines Vektorfeldes
\label{buch:kurvenintegral:subsection:kurvenintegralvektorfeld}}
In der Physik wird die Arbeit $W$, die gegen eine Kraft $F$ geleistet wird,
durch das Produkt
\[
W = F\cdot s
\]
definiert, wobei $s$ der Weg ist, der gegen den Widerstand der Kraft
zurückgelegt wird.
Dabei spielt nur die Kraftkomponente parallel zur Verschiebung eine Rolle.
Ist $\vec{F}$ die Kraft und $\vec{s}$ der zurückgelegte Weg, dann kann 
die Arbeit auch als das Skalarprodukt $W=\vec{F}\cdot\vec{s}$ geschrieben
werden.

Entlang eines Weges $\gamma$ vom Punkt $A=\gamma(a)$ zum Punkt
$B=\gamma(b)$ kann sich die Kraft verändern, so dass $\vec{F}$
als Kraftfeld betrachtet werden sollte.
Der Kraftvektor wird daher als Funktion $\vec{F}(x)$ beschrieben.

Der Weg $x(t)$ parametrisiert durch den Parameter $t$ setzt sich aus
kleinen Segmenten $d\vec{s} = x(t+dt)-x(t)$ zusammen, die in einem
kleinen Zeitinterval $dt$ durchlaufen werden.
Jedes einzelne trägt den Betrag $\vec{F}(x(t))\cdot d\vec{s}$ zur
Arbeit bei.
Diese Beiträge müssen aufsummiert werden, was uns auf etwas informelle
Art auf die Integralformel
\[
W
=
\int_a^b \vec{F}(x(t))\cdot d\vec{s}(t)
\]
führt.

Etwas formeller ist die Ableitung von $x(t)$ nach der Zeit
der Geschwindigkeitsvektor $\vec{v}(t)=\dot{x}(t)$.
Das Skalarprodukt $\vec{F}(x(t))\cdot \vec{v}(t)$ ist die
instantane Leistung gegen die Kraft.
Durch Integration der Leistung über die Zeit wird die Arbeit
\begin{equation}
W
=
\int_a^b \vec{F}(x(t))\cdot \vec{v}(t)\,dt
=
\int_a^b \vec{F}(x(t))\cdot \dot{x}(t)\,dt
\label{buch:kurvenintegral:kurvenintegral:eqn:arbeit}
\end{equation}
berechnet.

Die Formulierung \eqref{buch:kurvenintegral:kurvenintegral:eqn:arbeit}
ist abhängig von der Wahl der Koordinaten und daher kein allgemein 
kovariantes Naturgesetz.
Dazu muss das Integral in das Integral über eine $1$-Form umgewandelt
werden.
Dazu schreiben wir die Kraftkomponenten als $F^i$ und die Koordinaten
als $x^k(t)$.
Die Geschwindigkeitskomponenten sind durch $\dot{x}^k(t)$ gegeben.
Das Integral \eqref{buch:kurvenintegral:kurvenintegral:eqn:arbeit}
wird damit zu
\[
W
=
\int_a^b F^i(x(t))\,g_{ik}(x(t))\, \dot{x}^k(t)\,dt.
\]
Die $g_{ik}$ sind die Komponenten des metrischen Tensors, mit dem
das Skalarprodukt definiert wird.
Daraus kann jetzt die $1$-Form abgelesen werden, die über die
Kurve $\gamma$ integriert worden, sie ist
\[
\omega
=
F^ig_{ik}\,dx^k.
\]

%
% Der Fluss eines Vektorfeldes durch eine Kurve
%
\subsection{Der Fluss eines Vektorfeldes durch eine Kurve}
Wir betrachten die Strömung eines Mediums in der Ebene, die durch
den zweidimensionalen Geschwindigkeitsvektor $\vec{v}(x)$ an jeder
Stelle gegeben ist.
Wir möchten berechnen, wieviel des Mediums durch die Randkurve
eines Gebietes fliesst.
Die Menge des Mediums ist proportional zur Komponente der
Strömungsgeschwindigkeit orthogonal zur Randkurve.
Wir beschreiben ein Randsegment durch einen nach aussen
zeigenden Vektor $\vec{n}$.

Wir parametrisieren die Randkurve mit dem Parameter $s\mapsto x(s)$.
Das Segment der Randkurve zwischen den Punkten $x(s)$ und $x(s+ds)$
hat Richtung nahe bei $\dot{x}(s)$ und Länge $|\dot{x}(s)|\,ds$.
Wir nehmen an, dass die Kurve so parametrisiert ist, dass das Gebiet
links liegt.
Der Normalenvektor $\vec{n}$ kann man dann durch Rotation $R$ um
$90^\circ$ im Uhrzeigersinn erhalten.
Die Menge des durch dieses Segment fliessendenb Mediums ist
\[
\vec{v(x(s))}\cdot \vec{n}(x(s))
=
\vec{v}(x(s)) R\dot{x}(s).
\]
Das Skalarprodukt ist invariant unter Drehungen, daher bleibt der
Fluss der gleiche, wenn man beide Vektoren mit $R^{-1}$ multipliziert.
Damit entsteht die Formel
\[
V
=
\int_a^b 
R^{-1}\vec{v}(x(s))\cdot \dot{x}(s)
\,ds
\]
für das Volumen des Mediums.
Das Flussintegral ist daher das Kurvenintegral wie in
Abschnitt~\ref{buch:kurvenintegral:subsection:kurvenintegralvektorfeld}
eines Vektor um $90^\circ$ gegen den Uhrzeigersinn gedrehten
Vektorfeldes.

In kartesischen Koordinaten ist die Drehung besonders einfach durch
die Matrix 
\[
R^{-1}
=
\begin{pmatrix}
0&-1\\
1&\phantom{-}0
\end{pmatrix}
\]
beschreiben.
Sie vertauscht die beden Koordinaten und kehrt das Vorzeichen
der ersten Komponente.

\begin{beispiel}
XXX Fluss eines Feldes durch einen Kreis
\end{beispiel}


%
% Differential einer Funktion
%
\section{Differential einer Funktion}
Wir betrachten eine reellwertige, differenzierbare Funktion
$f\colon M\to\mathbb{R}$ auf einer $n$-dimensionalen 
Mannigfaltigkeit $M$.
In einer Koordinaten kann man sie als Funktion 
$f(x^1,\dots,x^n)$ der Koordinaten $x^i$ schreiben.

Ausserdem sei jetzt auch noch ein Tangentialvektor $X$ im Punkt
$p$ der Mannigfaltigkeit gegeben.
Er kann durch eine Kurve $\gamma(t)$ beschrieben werden, die
für $t=0$ durch den Punkt $p$ geht.
Die Ableitung nach $t$ ist in der Karte gegeben durch die
lineare Funktion
\[
\frac{d}{dt}
f(\gamma(t))
\bigg|_{t=0}
=
\frac{\partial f}{\partial x^i}(\gamma(t))\,\dot{x}^i(t)\bigg|_{t=0}
=
\frac{\partial f}{\partial x^i}(p)\dot{x}^i(0)
\]
der Komponenten $\dot{x}^i(0)$ des Tangentialvektors im Punkt $p$.

Die Ableitung von $f$ entlang der Kurve ist eine lineare 
Funktion des Tangentialvektors, die in einer Karte die
partiellen Ableitungen nach den Koordinaten als Koeffizienten
hat.

\begin{definition}[Differential einer Funktion]
\label{buch:kurvenintegral:differential:def:differential}
Das Differential $df$ einer Funktion an der Stelle $x$ einer
$n$-dimensionalen Mannigfaltigkeit ist die $1$-Form, die in einer
Karte als Komponenten die partiellen Ableitungen von $f$ nach den
Koordinaten hat.
Es ist
\[
df
=
\frac{\partial f}{\partial x^i}\,dx^i.
\]
\end{definition}

%
% Rechenregeln für den $d$-Operator
%
\subsection{Rechenregeln für den $d$-Operator}
Die Definition~\ref{buch:kurvenintegral:differential:def:differential}
ermöglicht, die bekannten Rechenregeln für Funktionen mehrere Variablen
auf das Differential von Funktionen auf einer Mannigfaltigkeit zu
übertragen.
Seien also im folgenden $f$ und $g$ Funktionen auf einer differenzierbaren
Mannigfaltigkeit.
Dann sind auch $f+g$ und $fg$, definiert durch
\begin{align*}
(f+g)(x) &= f(x)+g(x)
&
(fg)(x) &= f(x)\,g(x),
\end{align*}
differenzierbare Funktionen auf der Mannigfaltigkeit.
In einer Karte können die Differentiale wie folgt berechnet werden.
\begin{align*}
d(f+g)(x)
&=
\frac{\partial(f+g)}{\partial x^i}(x)\,dx^i
\\
&=
\frac{\partial f}{\partial x^i}(x)\,dx^i
+
\frac{\partial g}{\partial x^i}(x)\,dx^i
=
df(x) + dg(x),
\\
d(fg)(x)
&=
\frac{\partial (fg)}{\partial x^i}(x)\,dx^i
\\
&=
\frac{\partial f}{\partial x^i}(x)\,g(x)\,dx^i
+
f(x)\,\frac{\partial g}{\partial x^i}(x)\,dx^i
=
f(x)\,dg(x) + g(x)\,df(x),
\\
d\biggl(\frac{f}{g}\biggr)(x)
&=\frac{\partial}{\partial x^i}\biggl(\frac{f}{g}\biggr)\,dx^i
\\
&=
\frac{\displaystyle
\frac{\partial f}{\partial x^i}(x)\,g(x)-f(x)\,\frac{\partial g}{\partial x^i}(x)
}{
g(x)^2
}\,dx^i
\\
&=
\frac{\displaystyle
g(x)\,
\frac{\partial f}{\partial x^i}\,dx^i
-
f(x)\,\frac{\partial g}{\partial x^i}\,dx^i
}{g(x)^2}
=
\frac{g(x)\,df(x)-f(x)\,dg(x)}{g(x)^2}.
\end{align*}
Wir fassen die Resultate im folgenden Satz zusammen.

\begin{satz}[Rechenregeln für $d$]
Das Differential auf differenzierbaren Funktionen auf einer
Mannigfaltigkeit erfüllt die
\begin{align*}
\text{Summenregel:}& & d(f+g)&= df + dg, \\[6pt]
\text{Produktregel:}& & d(fg) &= f\,dg + g\,df,\\
\text{Quotientenregel:}& & d\biggl(\frac{f}{g}\biggr) &= \frac{g\,df - f\,dg}{g^2}.
\end{align*}
Eine Abbildung, die die Summen- und Produktregel erfüllt, heisst eine
{\em Derivation}.
\index{Derivation}%
\end{satz}
Wenn der Faktor $g$ konstant ist, dann ist $dg=0$ und es folgt
$d(fg) = g\,df$.
Der Operator $d$ ist also insbesondere linear.

%
% Integral eines Differentials
%
\subsection{Integral eines Differentials}
Sei $f$ eine differenzierbare Funktion auf einer $n$-dimensionalen
Mannigfaltigkeit $M$.
Das Differential $df$ ist eine $1$-Form auf der Mannigfaltigkeit.
In einer Karte kann sie als
\[
df
=
\frac{\partial f}{\partial x^i}\,dx^i
\]
geschrieben werden.

Eine Kurve zwischen zwei Punkten $A$ und $B$ sei durch die 
Parametrisierung $[a,b]\to\mathbb{R}^n:t\mapsto x^i(t)$ gegeben,
dann kann das Integral sofort durch
\[
\int_{\gamma} df
=
\int_{\gamma} \frac{\partial f}{\partial x^i} dx^i
=
\int_a^b \frac{\partial f}{\partial x^i}(x(t))\,\dot{x}^i(t)\,dt
\]
ausgedrückt werden.
Der Integrand des letzten Integrals ist die Ableitung
\[
\frac{d}{dt} f(x(t))
=
\frac{\partial f}{\partial x^i}(x(t))\,\dot{x}^i(t).
\]
Die Funktion $t\mapsto f(x(t))$ ist daher eine Stammfunktion und
das Integral kann damit als
\[
\int_{\gamma} df
=
\bigl[ f(x(t)) \bigr]_a^b
=
f(x(b)) - f(x(a))
\]
berechnet werden.
Dies beweist den folgenden Satz.

\begin{satz}
Ist $\gamma\colon[a,b]\to M$ eine Kurve auf $M$, die die Punkte $A=\gamma(a)$
und $B=\gamma(b)$ verbindet, dann ist das Integral
\[
\int_\gamma df
=
f(A) - f(B)
\]
unabhängig von der Wahl des Weges zwischen $A$ und $B$.
\end{satz}

Der Satz formuliert den Hauptsatz der Infinitesimalrechnung mit Hilfe
von Differentialen und Wegintegralen.
Das folgende Beispiel zeigt, dass es 1-Formen gibt, deren Integral
von der Kurve abhängt.
Der Satz bedeutet also, dass eine solche 1-Form nicht das Differential
einer Funktion sein kann.

\begin{beispiel}
%
% fig-wege.tex
%
% (c) 2025 Prof Dr Andreas Müller
%
\begin{figure}
\centering
\includegraphics{chapters/030-kurvenintegral/images/wege.pdf}
\caption{Verschiedene Wege zur Berechnung des Integrals der 1-Form
$r\,d\varphi$.
Die Wege, die nahe am Nullpunkt vorbeigehen, ergeben nur sehr kleine
Werte für das Integral, während der Kreisbogen $\gamma$ des
Einheitskreises den Wert $\pi$ ergibt.
\label{buch:kurvenintegral:differential:fig:wege}}
\end{figure}

Wir betrachten das Gebiet $M=\mathbb{R}^2\setminus\{0\}$, also die
Ebene ohne den Nullpunkt.
Statt in $x$-$y$-Koordinaten kann sie auch in Polarkoordinaten
beschrieben werden.
Die Umrechnung ist durch die Abbildung
\[
(x,y) \mapsto (r,\varphi)=\biggl(\sqrt{x^2+y^2},\arctan\frac{y}{x}\biggr)
\]
gegeben.

Auf diesem Gebiet soll jetzt die 1-Form $r\,d\varphi$ betrachtet und
zwischen den beiden Punkten $A=(0,-1)$ und $B=(0,1)$ integriert werden.
In Polarkoordinaten haben die beiden Punkte beide den Radius $r=1$,
aber der Polarwinkel ist $\varphi(A) = -\frac{\pi}2$ und
$\varphi(B)=\frac{\pi}2$.
Als Weg $\gamma$ wird der Abschnitt des Einheitskreises verwendet, der in
Polarkoordinaten durch $[-\frac{\pi}2,\frac{\pi}2]\to M:t\mapsto (1,t)$ 
gegeben ist.
Das Integral von $r\,d\varphi$ ist
\begin{equation}
\int_{\gamma} r\,d\varphi
=
\int_{-\frac{\pi}2}^{\frac{\pi}2}
1\cdot \dot{\varphi}\,dt
=
\int_{-\frac{\pi}2}^{\frac{\pi}2}\,dt
=
\bigl[t\bigr]_{-\frac{\pi}2}^{\frac{\pi}2}
=
\pi.
\label{buch:kurvenintegral:differential:eqn:halbkreis}
\end{equation}
Der Wert ändert sich jedoch, wenn ein anderer Weg gewählt wird.
Der Weg $\gamma_r$ setzt sich zusammen aus einem vertikalen
Streckensegment von $A$ zum Punkt $C$, einem Halbkreisbogen mit
Radius $r$ von $C$ zu $D$ und einem weiteren vertikalen Streckensegment
von $D$ zu $B$.
Da sich der Polarwinkel entlang der Streckensegmente nicht ändert,
verschwindet das Integral von $r\,df$ über die Streckensegment.
Das Integral über den Kreisbogen kann wie in 
\eqref{buch:kurvenintegral:differential:eqn:halbkreis}
berechnet werden, mit dem einzigen Unterschied, dass der Faktor $r$
jetzt nicht $1$ ist.
Das Integral hat daher den Wert $r\pi$ und es folgt
\[
\int_{\gamma_r} r\,d\varphi
=
\underbrace{\int_{AC} r\,d\varphi}_{\displaystyle = 0}
+
\underbrace{\int_{CD} r\,d\varphi}_{\displaystyle = r\pi}
+
\underbrace{\int_{DB} r\,d\varphi}_{\displaystyle = 0}
=
r\pi.
\]
Indem man den Radius $r$ des Kreisbogens klein macht, kann das Integral
also beliebig klein gemacht werden.

Gegen das Argument kann eingewendet werden, dass die verwendete 
Kurve $\gamma_r$ nicht differenzierbar ist.
Die entwickelte Theorie ist daher gar nicht anwendbar.
Man könnte aber auch einen Parabelbogen, der durch die Punkte 
$A$, $E$ und $B$ geht differenzierbar deformieren.
Der Weg 
\[
\gamma_s
\colon
[-1,1]
\to
\mathbb{R}^2
:
t\mapsto (s(1-t)(1+t), t)=(s(1-t^2),t)
\]
geht durch die Punkte $A$, $B$ und $(s,0)$ in karteischen
Koordinaten.
Insbesondere geht er im Abstand $s$ am Nullpunkt vorbei.
Da $\gamma_s$ in kartesischen Koordinaten gegeben ist, muss
dazu die 1-Form $r\,d\varphi$ erst in die $x$-$y$-Karte umgerechnet
werden.

Das Differential $d\varphi$ in $x$-$y$-Koordinaten ist
\begin{align*}
d\varphi
&=
d\arctan\frac{y}{x}
=
\frac{\partial}{\partial x}\arctan\frac{y}{x}\,dx
+
\frac{\partial}{\partial y}\arctan\frac{y}{x}\,dy
\\
&=
-\frac{y}{x^2+y^2}\,dx
+
\frac{x}{x^2+y^2}\,dy
=
-\frac{y}{r^2}\,dx
+
\frac{x}{r^2}\,dy
\end{align*}
Die 1-Form
\[
r\,d\varphi
=
-\frac{y}{\sqrt{x^2+y^2}}\,dx
+
\frac{x}{\sqrt{x^2+y^2}}\,dy
=
-\frac{y}{r}\,dx
+
\frac{x}{r}\,dy
\]
ist damit in $x$-$y$-Koordinaten ausgedrückt.

Für die Berechnung des Integrals entlang des Weges $\gamma_s$ werden
die Ableitungen der Koordinaten nach dem Parameter $t$ benötigt, sie
sind
\begin{align*}
\dot{x}(t) &= \frac{d}{dt} s(1-t^2) = -2st \\
\dot{y}(t) &= \frac{d}{dt} t = 1.
\end{align*}
Das Integral entlang des Weges $\gamma_s$ muss jetzt als das Integral
\begin{align*}
\int_{\gamma_s} r\,d\varphi
&=
\int_{-1}^1
-\frac{t}{\sqrt{s^2(1-t^2)^2+1}} \dot{x}(t)
+\frac{s(1-t^2)}{\sqrt{(1-t^2)^2+1}} \dot{y}(t)
\,dt
\\
&=
\int_{-1}^1
\frac{2st^2}{\sqrt{s^2(1-t^2)^2+1}} 
+\frac{s(1-t^2)}{\sqrt{s^2(1-t^2)^2+1}}
\,dt
\\
&=
s\int_{-1}^1
\frac{2t^2+1-t^2}{\sqrt{s^2(1-t^2)^2+1}}
\,dt
\end{align*}
Der Integrand
\[
f(t,s)
=
\frac{s(t^2+1)}{\sqrt{s^2(1-t^2)^2+1}}
\]
nimmt, wie man der
%
% fig-integrand.tex
%
% (c) 2024 Prof Dr Andreas Müller
%
\begin{figure}
\centering
\includegraphics{chapters/030-kurvenintegral/images/integrand.pdf}
\caption{Der Integrand des Wegintegrals entlang des Weges $\gamma_s$ nimmt
das Maximum $s/\sqrt{s^2+1}$ bei $t=0$ an.
\label{buch:kurvenintegral:differential:fig:integrand}}
\end{figure}

Abbildung~\ref{buch:kurvenintegral:differential:fig:integrand}
entnehmen kann, sein Maximum bei $t=0$ an, wo der Funktionswert
$s/\sqrt{s^2+1}$ ist.
Das Integral kann damit nach oben durch den grössten
Wert des Integranden als
\[
\int_{\gamma_s} r\,d\varphi
=
\int_{-1}^1 f(t,s)\,dt
=
2 \max_{t\in[-1,1]} f(t,s)
=
\frac{2s}{\sqrt{s^2+1}}
\le
2s
\]
abgeschätzt werden.
Je kleiner $s$ wird, desto kleiner wird daher auch das Wegintegral.
Dies zeigt erneut, dass durch einen Weg, der nahe am Nullpunkt vorbeiführt,
das Wegintegral beliebig klein gemacht werden kann und insbesondere
sehr stark von der Wahl des Weges abhängig ist.
\end{beispiel}

%
% Richtungsableitung
%
\subsection{Richtungsableitung}
Die Richtungsableitung einer Funktion 
\[
f\colon
\mathbb{R}^n\to\mathbb{R}
:
(x^1,\dots,x^n)\mapsto f(x^1,\dots,x^n)
\]
von $n$ Variablen an der Stelle $x\in\mathbb{R}^n$ in Richtung
eines Vektors $\vec{v}\in\mathbb{R}^n$ ist gegeben durch
\begin{align}
D_{\vec{v}}f(x)
&=
\frac{d}{dt}
f(x+t\vec{v})
\bigg|_{t=0}
=
\frac{d}{dt}
f(x^1+tv^1,\dots,x^n+tv^n)\bigg|_{t=0}
\notag
\\
&=
\frac{\partial f}{\partial x^1}(x)v^1
+
\dots
+
\frac{\partial f}{\partial x^n}(x)v^n
=
\frac{\partial f}{\partial x^i}(x)\,v^i
\label{buch:kurvenintegral:differential:eqn:ricthungsableitung}
\end{align}
(man beachte die Summationskonvention).
Die Richtungsableitung ist also nur ein Spezialfall der Ableitung
einer Funktion entlang einer Kurve für eine speziell gewählte
Kurve.
In der Karte mit den $x^i$-Koordinaten ist die Kurve die Gerade
\[
x^i(t) = x^i(0) + tv^i
\]
mit dem Richtungsvektor mit Komponenten $v^i$.

Insbesondere kann man die Richtungsableitung mit dem Differential
als
\[
D_{\vec{v}}f(x)
=
\langle
df(x),
V
\rangle
\]
schreiben, wobei $V$ der Tangentialvektor ist, der in der Karte
der $x^i$-Koordinaten die Komponenten $v^i$ hat.
Das Differential $df$ verallgemeinert also die Theorie der 
Richtungsableitung aof beliebige Koordinantensysteme.


%
% Gradient
%
\subsection{Gradient}
In der klassischen Vektoranalysis schreibt man den Ausdruck
\eqref{buch:kurvenintegral:differential:eqn:ricthungsableitung}
für die Richtungsableitung als Skalarprodukt
\[
D_{\vec{v}}f(x)
=
\renewcommand{\arraystretch}{1.5}
\begin{pmatrix}
\displaystyle
\frac{\partial f}{\partial x^1}(x)\\
\vdots\\
\displaystyle
\frac{\partial f}{\partial x^n}(x)
\end{pmatrix}
\cdot
\vec{v}
\]
und nennt den Vektor
\[
\renewcommand{\arraystretch}{1.5}
\begin{pmatrix}
\displaystyle
\frac{\partial f}{\partial x^1}(x)\\
\vdots\\
\displaystyle
\frac{\partial f}{\partial x^n}(x)
\end{pmatrix}
=
\operatorname{grad}f(x)
=
\nabla f(x)
\]
der partiellen Ableitungen von $f$ nach den
Variablen auch den {\em Gradienten}
\index{Gradient}%
der Funktion $f$.
Der Operator $\nabla$ heisst auch der Nabla-Operator\footnote{Das Zeichen
$\nabla$ wurde von William Rowan Hamilton in der Quaternionenanalysis
als auf den Kopf gestelltes $\Delta$ eingeführt.
Der Theologe William Robertson Smith nannte es Nabla, weil
es ihn an die Form eines so genannten Saiteninstruments aus dem
alten Israel erinnerte.}.

Unter allen Richtungsvektoren $\vec{v}$ mit Länge $1$ wird die
Richtungsableitung
\[
D_{\vec{v}}f(x)
=
\operatorname{grad}f(x)\cdot \vec{v}
=
|\operatorname{grad}f(x)|\;|\vec{v}|\;\cos\alpha
\]
genau dann maximal, wenn der Zwischenwinkel $\alpha$ zwischen
dem Gradienten $\operatorname{grad}f(x)$ und der Richtung $\vec{v}$
verschwindet.
Der Gradient zeigt daher in die Richtung, in der die Funktion am
schnellsten zunimmt.

%
% Der inverse metrische Tensor
%
\subsubsection{Der inverse metrische Tensor}
Die Interpretation des Wertes der 1-Form $df$ auf dem Tangentialvektor $V$ 
als Skalarprodukt des Gradientenvektors $\nabla f(x)$ mit dem
Richtungsvektor $\vec{v}$ verletzt die Regel, dass die Gesetzmässigkeiten
unabhängig von den Koordinatensystemen formuliert werden sollen.
Die Interpretation von
\eqref{buch:kurvenintegral:differential:eqn:ricthungsableitung}
als Skalarprodukt nimmt an, dass an der Stelle $x$ die Koordinatenrichtungen
orthonormiert sind.
Koordinatenunabhängig kann dies nur formuliert werden, wenn ein metrischer
Tensor $g_{ik}$ zur Verfügung steht.
Dann kann man nach einem Vektor $r$ mit den Komponenten $r^k$ fragen,
der über die Formel
\[
\langle df(x),v\rangle
=
\frac{\partial f}{\partial x^i}(x)\,v^i
=
g_{ik} r^k\,v^i
\]
die Richtungsableitung reproduziert.
Da dies für jeden Vektor $v^i$ gilt, muss für jedes $i$ die Gleichung
\begin{equation}
g_{ik}r^k = \frac{\partial f}{\partial x^i}(x)
\label{buch:kurvenintegral:differential:eqn:ggleichung}
\end{equation}
gelten.
Dies ist ein lineares Gleichungssystem mit der symmetrischen
Koeffizientenmatrix $g_{ik}$.
Da der metrische Tensor positiv definit ist, gibt es nur einen
Vektor $r^k$, der das Gleichungssystem löst.

Das Gleichungssystem~\eqref{buch:kurvenintegral:differential:eqn:ggleichung}
kann mit der inversen Matrix der Koeffizientenmatrix gelöst
werden.
Ist $G$ die Matrix mit den Einträgen $g_{ik}$, dann schreiben wir
die inverse Matrix $G^{-1}$ von
\[
G
=
\begin{pmatrix}
g_{11}&\dots &g_{1n}\\
\vdots&\ddots&\vdots\\
g_{n1}&\dots &g_{nn}
\end{pmatrix}
\qquad
\text{mit den Einträgen}
\qquad
G^{-1}
= 
\begin{pmatrix}
g^{11}&\dots &g^{1n}\\
\vdots&\ddots&\vdots\\
g^{n1}&\dots &g^{nn}
\end{pmatrix}.
\]
Die Matrix $G^{-1}$ ist ebenfalls symmetrisch.
Die Bedingung $G^{-1}G=I$ wird in Komponenten durch
\[
g^{ki}g_{kl}
=
\delta^k\mathstrut_i
\]
ausgedrückt.

Die Lösung des
Gleichungssystems~\eqref{buch:kurvenintegral:differential:eqn:ggleichung}
ist jetzt durch das Produkt $G^{-1}r$ mit den Komponenten
\[
r^k = g^{ki}\frac{\partial f}{\partial x^i}
\]
gegeben.
Die Rolle des Gradienten wird also vom kovarianten Tensor
\[
g^{ki}\frac{\partial f}{\partial x^i}
\]
übernommen.

Für eine orthonormierte Basis ist der metrische Tensor $g_{ik}=\delta_{ik}$
besonderes einfach, die Matrix $G$ ist die Einheitsmatrix und die 
Inverse ist $G^{-1}=I^{-1}$ ist ebenfalls die Einheitsmatrix.
In diesem Fall ist der Gradient also durch
\[
g^{ki}\frac{\partial f}{\partial x^i}(x)
=
\delta^{ki}\frac{\partial f}{\partial x^i}(x)
=
\frac{\partial f}{\partial x^k}(x)
\]
gegeben, die der klassischen Definition des Gradienten entspricht.

%
% Kontravarianz des inversen metrischen Tensors
%
\subsubsection{Kontravarianz des inversen metrischen Tensors}
Bei einem Koordinatenwechsel vom $(x^1,\dots,x^n)$-Koordinatensystem
mit metrischem Tensor $g_{ik}$ in das $(y^1,\dots,y^n)$-Koordinatensystem
mit dem metrischen Tensor $h_{ik}$ mit den partiellen Ableitungen
der $y$-Koordinaten nach den $x$-Koordinaten umgerechnet.
Aus der Rechnung
\begin{align*}
\sum_{i,k=1}^n
h_{ik} \, dy^i \otimes dy^k
&=
\sum_{i,k=1}^n
h_{ik}
\biggl(\sum_{l=1}^n \frac{\partial y^i}{\partial x^l}\, dx^l \biggr)
\otimes
\biggl(\sum_{m=1}^n \frac{\partial y^k}{\partial x^m}\, dx^m \biggr)
\\
&=
\sum_{l,m=1}^n
\biggl(
\underbrace{
\sum_{i,k=1}^n h_{ik}
\frac{\partial y^i}{\partial x^l}
\frac{\partial y^k}{\partial x^m}
}_{\displaystyle = g_{lm}}
\biggr)
\,
dx^l\otimes dx^m
\end{align*}
folgt daher, dass
\begin{equation}
g_{ik}
=
\sum_{i,k=1}^n h_{ik}
\frac{\partial y^i}{\partial x^l}
\frac{\partial y^k}{\partial x^m}.
\label{buch:kurvenintegral:differential:eqn:gkovarianz}
\end{equation}
Dies ist das erwartete Resultat für einen Tensor mit unteren und daher
kovarianten Indizes.

Für die metrischen Tensoren für 1-Formen kann eine analoge Rechnung
durchgeführt werden.
Dabei wird verwendet, dass
\[
\frac{\partial}{\partial y^i}
=
\sum_{k=1}^n
\frac{\partial x^k}{\partial y^i}
\,
\frac{\partial}{\partial x^k}
\qquad\text{und}\qquad
\frac{\partial}{\partial x^k}
=
\sum_{i=1}^n
\frac{\partial y^i}{\partial x^k}
\,
\frac{\partial}{\partial x^i}.
\]
Eingesetzt in die Transformationsformel für Metrik von Formen folgt
\begin{align*}
\sum_{k,i=1}^n
g^{ik} \frac{\partial}{\partial x^i}\otimes \frac{\partial}{\partial x^k}
&=
\sum_{i,k=1}^n
g^{ik}
\biggl(
\sum_{l=1}^n \frac{\partial y^l}{\partial x^i}\frac{\partial}{\partial y^l}
\biggr)
\otimes
\biggl(
\sum_{m=1}^n \frac{\partial y^m}{\partial x^k}\frac{\partial}{\partial y^m}
\biggr)
\\
&=
\sum_{l,m=1}^n
\biggl(
\sum_{i,k=1}^n
g^{ik}
\frac{\partial y^l}{\partial x^i}
\frac{\partial y^m}{\partial x^k}
\biggr)
\,
\frac{\partial}{\partial y^l}
\otimes
\frac{\partial}{\partial y^m}
\intertext{und daher}
h^{lm}
&=
\sum_{i,k=1}^n
g^{ik}
\frac{\partial y^l}{\partial x^i}
\frac{\partial y^m}{\partial x^k}.
\end{align*}
Die Transformation erfolgt also im Gegensatz zu den Komponenten
$g_{ik}$ in der umgekehrten Richtung, was zeigt, dass die $g^{ik}$
kontravariant sind.

%
% Indizes anheben und absenken
%
\subsubsection{Indizes anheben und absenken}
Der Tensor $g_{ik}$ hat schon in
Abschnitt~\ref{buch:kurvenintegral:1formen:subsection:linearformen}
ermöglicht, aus einem Vektor, also einem kontravarianten Tensor,
eine Linearform, also einen kovarianten Tensor zu machen.
Mit dem Tensor $g^{ik}$ wird jetzt auch das, aus einem kontravarianten
Index einen kovarianten zu machen.
Diese Operationen sind auf jeden beliebigen Index eines Tensors
anwendbar.

\begin{definition}[Anheben und Absenken von Indizes]
Ein kontravarianter Index $i$ eines Tensors $A^i$ wird durch
$g_{ki}A^i$ zu einem kovarianten Index $k$ {\em abgesenkt}.
\index{absenken, Index}%
Ein kovarianter Index $i$ eines Tensors $B_i$ wird durch
$g^{il}B_i$ zu einem kontravarianten Index {\em angehoben}.
\index{anheben, Index}%
\end{definition}

%
% Konforme Abbildungen
%
\subsubsection{Konforme Abbildungen}
Eine Abbildung heisst {\em konform}, wenn Sie winkeltreu ist.
Die Länge von Vektoren darf also ändern, nicht aber der Winkel.
Da Winkel mit der Skalarproduktformel
\[
\cos \angle(X,Y)
=
\frac{
\langle X,Y \rangle
}{
\!\sqrt{\langle X, X\rangle}\cdot\!\sqrt{\langle Y,Y\rangle}
}
\]
berechnet werden können, darf sich das Skalarprodukt unter einer
konformen Abbildung nur um einen gemeinsamen Faktor ändern.
Mit der Notation von
\eqref{buch:kurvenintegral:differential:eqn:gkovarianz}
folgt daher, dass
\[
g_{ik} = s\,h_{ik}
\quad\Rightarrow\quad
g_{ik}
=
\sum_{l,m=1}^n
\frac{\partial y^l}{\partial x^i}
\frac{\partial y^m}{\partial x^k}
g_{lm}
\]
als Bedingung dafür, dass die Koordinatenwechselabbildung
konform ist.

\begin{beispiel}
Die stereographische Projektion von Beispiel 
\ref{buch:koordinaten:diffmannig:beispiel:stereographisch}
ist konform, wie im Folgenden nachgerechnet werden soll.
\end{beispiel}

\subsubsection{Die Mercator-Projektion}
Die Mercator-Projektion bildet Punkte $(\vartheta,\beta)$ auf
der Oberfläche der Einheitskugel auf einen Zylinder mit Koordinaten
$(\varphi,z)$ ab, der die Kugel im Äquator berührt.
Wir verwenden die vom Äquator gemessene geographsiche Breite $\beta$
anstelle des bei Kugelkoordinaten üblichen Polabstandes $\vartheta$.
Sie ist aber nicht eine geometrische Projektion, vielmehr wird die
$z$-Koordinate durch eine nichtlineare Funktion $z(\beta)$
gegeben.

Zunächst muss der metrische Tensor für die Kugeloberfläche
berechnet werden.
Die Kugelkoordinaten sind orthogonal, der metrische Tensor wird
eine Diagonalmatrix, genauer
\[
g
=
\begin{pmatrix}
 1 & 0           \\
 0 & \cos^2\beta
\end{pmatrix}.
\]
Da der Zylinder nur eine ``aufgerollte Ebene'' ist, ist die Metrik
auf dem Zylinder durch die Einheitsmatrix
\[
h = \begin{pmatrix} 1&0\\0&1\end{pmatrix}
\]
gegeben.

Die Koordinatentransformation ist
\[
S^2\to\mathbb{R}^2
:
(\varphi,\beta) \mapsto (\varphi, z(\beta)).
\]
Die zugehörige Jacobimatrix ist
\[
J
=
\frac{\partial(\varphi,z)}{\partial(\varphi,\beta)}
=
\begin{pmatrix}
 1 & 0 \\
 0 & f'(\beta)
\end{pmatrix}.
\]
Transformiert man $g$ mit dieser Matrix, muss ein Vielfaches der
Einheitsmatrix entstehen.
Die Transformation liefert
\begin{align*}
J^tgJ
&=
\begin{pmatrix}
 1 & 0 \\
 0 & f'(\beta)
\end{pmatrix}
\begin{pmatrix}
 1 & 0           \\
 0 & \cos^2\beta
\end{pmatrix}
\begin{pmatrix}
 1 & 0 \\
 0 & f'(\beta)
\end{pmatrix}
\\
&=
\begin{pmatrix}
1&0\\
0&f'(\beta)^2\cos^2\beta
\end{pmatrix}
=
I
\qquad\Rightarrow\qquad
f'(\beta)^2\cos^2\beta = 1.
\end{align*}
Daraus ergibt sich die Differentialgleichung
\[
f'(\beta)=\frac{1}{\cos \beta}
\qquad\text{mit der Anfangsbedingung}\qquad
f(0)=0.
\]
Sie kann durch Integration gelöst werden, die Lösung ist
\[
f(\beta)
=
\int_0^\beta \frac{dt}{\cos t}
=
\frac12 \log\biggl(
\frac{1+\sin\beta}{1-\sin\beta}
\biggr).
\]
Die Mercator-Projektion ist daher gegeben durch
\begin{equation}
(\varphi,\beta) \mapsto \biggl(
\varphi,\frac12\log\biggl(\frac{1+\sin\beta}{1-\sin\beta}\biggr)
\biggr).
\label{buch:kurvenintegral:differential:eqn:mercatorprojektion}
\end{equation}
%
% fig-mercator.tex
%
% (c) 2025 Prof Dr Andreas Müller
%
\begin{figure}
\centering
\includegraphics{chapters/030-kurvenintegral/images/mercator.pdf}
\caption{Die Funktion $f(\beta)$, die die Mercator-Projektion definiert,
dargestellt als die {\color{darkred}rote} Kurve.
Zum Vergleich ist auch die Funktion $\tan\beta$ darstellt, die zur
Zylinderprojektion führt.
\label{buch:kurvenintegral:fig:mercator}}
\end{figure}
%
Die Funktion $f(\beta)$ ist auch in
Abbildung~\ref{buch:kurvenintegral:fig:mercator} dargestellt.
 
%
% Differentialgleichungen
%
\subsection{Differentialgleichungen}
Eine $1$-Form $\alpha = \alpha_i\,dx^i$ auf einer zweidimensionalen
Mannigfaltigkeit ist in jedem Punkt der Mannigfaltigkeit eine lineare
Funktion, die einem Tangentialvektor einen Zahlenwert zuordnet.
Falls die Linearform nicht verschwindet, kann sie nur Rang $1$
haben.
Es gibt daher einen eindimensionalen Unterraum von Vektoren $X$, die
von der $1$-Form zu $0$ gemacht werden.

Ein Tangentialvektor kann durch eine parametrisierte Kurve repräsentiert
werden.
In Koordinaten sind die Komponenten des Tangentialvektors durch
die Ableitungen der $\dot{x}^i$ gegeben.
Ein Tangentialvektor, der die 1-Form zum Verschwinden bringt, erfüllt
daher
\begin{align*}
0
&=
\langle\alpha, X\rangle
=
\langle
\alpha_i\,dx^i,
X
\rangle
\\
&=
\biggl\langle
\alpha_i(x(t))\,dx^i,
\dot{x}^k(t)\frac{\partial}{\partial x^k}
\biggr\rangle
\\
&=
\sum_{i=1}^n
\alpha_i(x(t))
\biggl\langle
dx^i,\frac{\partial}{\partial x^k}
\biggr\rangle
\dot{x}^k(t)
=
\alpha_i(x(t))\delta_{ik}\dot{x}^k(t)
\\
&=
\alpha_i(x(t))\dot{x}^i(t).
\end{align*}
Falls $\alpha_2(x(t))\ne 0$ ist, kann die Gleichung
\begin{equation}
\alpha_1(x(t))\, \dot{x}^1(t) + \alpha_2(x(t))\,\dot{x}^2(t) = 0
\label{buch:kurvenintegral:differential:eqn:1formsteigung}
\end{equation}
aufgelöst werden nach
\[
\frac{\dot{x}^2}{\dot{x}^1(t)}
=
\frac{\alpha_1(x(t))}{\alpha_2(x(t))}.
\]
Dies ist die Steigung der Kurve in einem 
$(x^1,x^2)$-Koordinatensystem.
Falls $\alpha_2(x(t))=0$ folgt aus
\eqref{buch:kurvenintegral:differential:eqn:1formsteigung}, dass
$\dot{x}^1(t)=0$ sein muss, dass die Kurve also eine Tangentenrichtung
parallel zur $x^2$-Richtung hat.

Die Gleichung $\langle\alpha,X\rangle=0$ legt daher die Richtung
einer Lösungskurve vor.
Eine 1-Form definiert somit ein Richtungsfeld auf einer zweidimensionalen
Mannigfaltigkeit genauso wie eine gewöhnliche Differentialgleichung
der Form $y'=f(x,y)$ ein Richtungsfeld auf der $x$-$y$-Ebene festlegt.

%
% Separation der Variablen
%
\subsubsection{Separation der Variablen}
Wenn die Koeffizienten $\alpha_i(x)$ der 1-Form nur von jeweils der einen
Variable $x^i$ abhängen, dann kann man die beiden 1-Formen
$\alpha_1\,dx^1$ und $\alpha_2\,dx^2$ entlang der Lösungskurve
integrieren, man kann also schreiben
\begin{equation}
\alpha_1\,dx^1
+
\alpha_2\,dx^2
=0
\quad\Rightarrow\quad
\int \alpha_1(x^1)\,dx^1 = -\int \alpha_2(x^2)\,dx^2.
\label{buch:kurvenintegral:differential:eqn:separation}
\end{equation}
Da in $\alpha_1$ nur die Variable $x^1$ vorkommt und in $\alpha_2$
nur die Variable $x^2$, lassen sich die beiden Integrale auf den
beiden Seiten \eqref{buch:kurvenintegral:differential:eqn:separation}
als gewöhnliche Integrale ausführen und liefern eine Gleichung,
die $x^1$ und $x^2$ erfüllen müssen.
Diese kann man jetzt nach $x^2$ auflösen und damit eine Lösung der
Differentialgleichung in der Form einer Funktion $x^2(x^1)$ finden.

Die eben beschriebene Vorgehensweise erinnert nicht nur oberflächlich
an die Methode der Separation der Variablen, die man in einer
Anfängervorlesung über gewöhnliche Differentialgleichung typischerweise
kennenlernt.
In der Theorie der Ableitung und des Riemann-Integrals sind die Zeichen
$dx^1$ und $dx^2$ nur Formelteile, die für sich allein keine
wohldefinierte mathematische Bedeutung haben.
Die Vorgehensweise in der Methode der Separation der Variablen ist
daher nur formal und nur dadurch gerechtfertigt, dass es am Schluss
``aufgeht''.
Die Theorie der 1-Formen und der Integration von 1-Formen zeigt,
wie diese Lösungsmethode mathematisch widerspruchsfrei formuliert
werden kann.

\subsubsection{Die Differentialgleichung der Loxodrome}
Die Loxodrome auf der Kugeloberfläche wird in einer Mercator-Projektion
zu einer Geraden.
Betrachten wir die Mercator-Projektion als Karte der Kugeloberfläche
mit den Koordinaten $(\varphi,z)$, dann erfüllen die Geraden mit
Steigung $k$ in Parameterdarstellung die Gleichung
\[
\frac{\dot{z}(t)}{\dot{\varphi}(t)}
=
k.
\]
Die Differentialgleichung kann man als $1$-Form 
\begin{equation}
\alpha
=
k\,d\varphi - dz
\label{buch:kurvenintegral:differential:mercatorform}
\end{equation}
geschrieben werden.

Die Mercator-Projektion ist in geographischen Koordinaten
auf der Kugelaberfläche die Abbildung
\[
f\colon
\mathbb{R}^2\to\mathbb{R}^2
:
(\varphi,\beta)
\mapsto
(\varphi,z)
=
(\varphi,f(\beta))
=
\biggl(\varphi,\frac12\log\frac{1+\sin\beta}{1-\sin\beta}\biggr).
\]
Durch Transport der 1-Form
\eqref{buch:kurvenintegral:differential:mercatorform}
mithilfe der Mercator-Projektion
von~\eqref{buch:kurvenintegral:differential:eqn:mercatorprojektion}
bekommt man aus
\begin{equation}
dz
=
f'(\beta)\,d\beta
=
\frac{1}{\cos\beta}\,d\beta
\qquad\Rightarrow\qquad
\alpha
=
k\,d\varphi-\frac{1}{\cos\beta}\,d\beta
\label{buch:kurvenintegral:differential:mercatorsepariert}
\end{equation}

Wir suchen eine Lösung der Differentialgleichung $\alpha=0$
als Funktion $\vartheta(\varphi)$.
Die $1$-Form
\eqref{buch:kurvenintegral:differential:mercatorsepariert}
hat die Eigenschaft, dass die Koeffizienten nur von jeweils
einer Variablen abhängen.
Somit sind die Voraussetzungen für die Anwendung der Methode
der Separation der Variablen gegeben.
Die Differentialgleichung kann damit durch Integration
\[
k\,d\varphi=\frac{1}{\cos\beta}\,d\beta
\quad
\Rightarrow
\quad
k\int d\varphi
=-
\int \frac{d\beta}{\cos\beta}
\quad
\Rightarrow
\quad
k\varphi + C
=
\frac12
\log\frac{1+\sin\beta}{1-\sin\beta}
\]
gelöst werden.
Die Funktion auf der rechten Seite ist auch der $\operatorname{artanh}$
von $\sin\varphi$, daher kann man die Lösung für die Loxodrome auch
als
\[
\sin\beta
=
\tanh(k\varphi+C)
\quad
\Rightarrow
\quad
\beta = \arcsin\tanh(k\varphi+C)
\]
schreiben.


%
% Differenzierbare Zerlegungen der Einheit
%
\section{Differenzierbare Zerlegung der Einheit
\label{buch:kurvenintegral:section:zerlegung}}
Ein Wegintegral kann sich entlang eines Pfades erstrecken, der
mehrere Kartengebiete einer Mannigfaltigkeit durchauert.
Das Kurvenintegral ist mithilfe eines Koordinatensystems definiert,
kann also immer nur innerhalb eines Kartengebietes berechnet werden.
Es muss also ein Technik gefunden werden, mit der Teilintegrale in
einzelnen Kartengebieten zu einem Integral über die ganze Kurve
zusammengesetzt werden.
Die Konstruktion muss von der Wahl der Koordinatensysteme enthlange
des Pfades unabhängig sein.
Dies wird erreicht mit einer differenzierbaren Zerlegung der Einheit
und dank der Tatsache, dass das Kurvenintegral linear in der 1-Form
ist.

%
% Glatte Funktionen mit Träger in einem Interval
%
\subsection{Glatte Funktionen mit Träger in einem Interval}
Wir müssen eine beliebig oft differenzierbare Funktion konstruieren,
die genau auf einem vorgegebenen offenen Intervall von Null
verschieden ist.

%
% Der Träger einer Funktion
%
\subsubsection{Der Träger einer Funktion}

\begin{definition}[Träger einer Funktion]
Sei $f\colon M\to\mathbb{R}$ eine differenzierbare Funktion.
Der {\em Träger} von $f$ ist die Menge
\index{Träger}%
\[
\operatorname{supp}(f)
=
\overline{
\{x\in M\mid f(x)\ne 0\}
},
\]
der Abschluss der Menge der Punkte, in denen $f$ von $0$ verschieden
ist.
\end{definition}

%
% Eine glatte Funktion mit Träger $\mathbb{R}_{\ge 0}$
%
\subsubsection{Eine glatte Funktion mit Träger $\mathbb{R}_{\ge 0}$}
Die Funktion
\[
g(x)
=
\begin{cases}
e^{-1/x}&\qquad \text{für $x>0$}\\
0       &\qquad \text{sonst.}
\end{cases}
\]
ist sicher beliebig oft differenzierbar ausserhalb des Punktes $x=0$.
Die linksseitigen Ableitungen der Funktion im Punkt $0$ verschwinden
alle.
Es ist also zu überprüfen, ob auch die rechtsseitigen Ableitungen
verschwinden.
Dazu berechnen wir die Ableitungen von $e^{-1/x}$ mit Hilfe der
Kettenregel:
\begin{align*}
\frac{d}{dx}e^{-1/x}
&=
e^{-1/x}\frac{d}{dx}\biggl(-\frac1x\biggr)
=
e^{-1/x}\frac{1}{x^2}
\\
\end{align*}

\begin{satz}
Die $k$-te Ableitung der Funktion $f(x)=e^{-1/x}$ ist von der Form
\[
\frac{d^n}{dx^n}f(x)
=
f^{(n)}(x)
=
\frac{P_n(x)}{x^{2n}}e^{-1/x},
\]
wobei $P_n(x)$ ein Polynom vom Grad $n-1$ ist, welches die
Rekursionsformel
\[
P_{n+1}(x)
=
x^2P'_n(x) + (1-2nx)P_n(x)
\]
erfüllt.
\end{satz}

\begin{proof}
Die Aussage kann mithilfe von vollständiger Induktion beweisen
werden.
Sie ist offensichtlich für die $0$-te Ableitung, also für die Funktion
$f(x)$ selbst war, das Polynom $P_0(x)=1$ ist als Konstante vom Grad 0.

Um den Induktionsschritt durchzuführen, nehmen wir an, dass $f^{(n)}(x)$
die behauptete Form hat, und berechnen die Ableitung
\begin{align}
f^{(n+1)}(x)
&=
\frac{d}{dx}
f^{(n)}(x)
=
\frac{d}{dx}
\frac{P_n(x) e^{-1/x}}{x^{2n}}
\notag
\\
&=
\frac{P'_n(x)e^{-1/x}}{x^{2n}}
+
\frac{P_n(x)e^{-1/x}}{x^{2n}}\frac{d}{dx}\biggl(-\frac1x\biggr)
+
P_n(x)e^{-1/x}\frac{d}{dx}\frac{1}{x^{2n}}
\notag
\\
&=
\biggl(
x^2 P'_n(x)
+
P_n(x)
-
2nx P_n(x)
\biggr)\frac{e^{-1/x}}{x^{2(n+1)}}
\notag
\\
&=
\bigl(x^2P'_n(x)+(1-2nx)P_n(x)\bigr) \frac{e^{-1/x}}{x^{2(n+1)}}.
\label{buch:kurvenintegral:zerlegung:eqn:rekursion}
\end{align}
Die Ableitung $P'_n(x)$ ist ein Polynom vom Grad $n-2$, also ist
$x^2P'_n(x)$ vom Grad $n$.
Ebenso ist der zweite Summand $(1-2nx)P_n(x)$ in
\eqref{buch:kurvenintegral:zerlegung:eqn:rekursion}
ein Polynom vom Grad $n$.
Das Polynom
\[
P_{n+1}(x)
=
x^2P'_{n}(x)+(1-2nx)P_n(x)
\]
ist das gesuchte Polynom.
\end{proof}

\begin{satz}
\label{buch:kurvenintegral:zerlegung:satz:g}
Die Funktion
\[
g(x)
=
\begin{cases}
e^{-1/x}&\qquad\text{für $x>0$}\\
0&\qquad\text{sonst}
\end{cases}
\]
ist beliebig oft stetig differenzierbar.
\end{satz}

\begin{proof}
Es ist bereits bekannt, dass die Funktion stetig differenzierbar ist
ausserhalb des Punktes $x=0$.
Es muss also nur noch gezeigt werden, dass alle Ableitungen von $f$
für $x\to 0$ gegen $0$ konvergieren.
Dazu berechnet man
\begin{align}
\lim_{x\to 0+} f^{(n)}(x)
&=
\lim_{x\to 0+} P_n(x) \frac{e^{-1/x}}{x^{2n}}.
\notag
\intertext{Als Polynom ist $P_n(x)$ stetig an der Stelle $0$,
sein Grenzwert für $x\to 0$ ist $P_n(0)$.
Der Grenzwert der Ableitung kann damit vereinfacht werden zu}
&=
P_n(0) \lim_{x\to 0+}\frac{e^{-1/x}}{x^{2n}}.
\notag
\intertext{Schreibt man $x=1/t$, wird daraus der Grenzwert}
&=P_n(0)\lim_{t\to\infty}t^{2n}e^{-t}
\label{buch:kurvenintegral:zerlegung:eqn:bruch}
\end{align}
für $t\to\infty$.
Der Kehrwert des Bruchs in

\eqref{buch:kurvenintegral:zerlegung:eqn:bruch}
ist
\begin{align*}
\lim_{t\to\infty}
\frac{1}{t^{2n}e^{-t}}
&=
\lim_{t\to\infty}
\frac{u(t)}{v(t)}
=
\lim_{t\to\infty}
\frac{e^t}{t^{2n}}
=
\lim_{t\to\infty},
\intertext{auf den die Regel von de l'Hospital für den Bruch $u(t)/v(t)$
mit $u(t)=e^t$ und $v(t)=t^{2n}$ wiederholt angewendet werden kann.
Die Ableitungen von $u$ sind $u^{(k)}(t)=e^t$ und
$v^{(k)}=2n(2n-1)\dots(2n-k+1)x^{2n-k}$.
$2n$-malige Anwendung ergibt daher}
\lim_{t\to\infty}
\frac{1}{t^{2n}e^{-t}}
&=
\lim_{t\to\infty}\frac{u^{(2n)}(t)}{v^{(2n)}(t)}
=
\lim_{t\to\infty}\frac{e^t}{(2n)!}=\infty.
\end{align*}
Dies zeigt, dass der Grenzwert
\eqref{buch:kurvenintegral:zerlegung:eqn:bruch}
verschwindet.
\end{proof}

%
% Eine glatte Funktion mit Trägerr [a,b]
%
\subsubsection{Eine glatte Funktion mit Träger $[a,b]$}
Mit der Funktion $g(x)$ von Satz~\ref{buch:kurvenintegral:zerlegung:satz:g}
lässt sich jetzt eine beliebig oft stetig differnzierbare
Funktion mit Träger im Intervall $[a,b]$ konstruieren.

\begin{satz}
Für $a,b\in\mathbb{R}$ mit $a<b$ ist
die Funktion
\[
g
\colon
\mathbb{R}\to\mathbb{R}
:
x\mapsto
g_{a,b}(x)
=
g(x-a)\cdot g(b-x)
\]
beliebig oft stetig differenzierbar und hat Träger $[a,b]$.
\end{satz}

\begin{proof}
Für $x<a$ ist $x-a<0$ und daher $g(x-a)=0$.
Für $x>b$ ist $b-x<0$ und daher $g(b-x)=0$.

Für $x$ im offenen Intervall $(a,b)$ ist $x-a>0$ und $b-x>0$ ist
$g(x-a)>0$ und $g(b-x)>0$ und damit
auch $g_{a,b}(x)=g(x-a)g(b-x)>0$.
Damit ist gezeigt, dass $g_{a,b}(x)\ne 0$ für $x\in(a,b)$ und damit
\[
\operatorname{supp} g_{a,b}(x)
=
[a,b],
\]
wie behauptet.
\end{proof}

%
% Überdeckung mit offenen Intervallen
%
\subsection{Überdeckung mit offenen Intervallen}
Eine Kurve auf einer $n$-dimensionalen Mannigfaltigkeit kann sich
nacheinander durch mehrere Kartengebiet winden.
In jedem Kartengebiet lässt sich eine auf der Mannigfaltigkeit
definierte $1$-Form $\alpha$ in den Koordinaten ausdrücken und das
Integral kann mit den Methoden der klassischen Integralrechnung
berechnet werden.
Damit wird aber der Beitrag zum Integral ausserhalb der Karte
ignoriert und es ist unklar, was am Rande des Kartengebietes
passiert.

Sei $[a,b]$ das Parametergebiet der Kurve $\gamma\colon\mathbb{R}\to M$,
welches in eine einzige Karte abgebildet wird.
Die 1-Form $g_{a,v}(x)(T\gamma)_*\alpha(x)$ verswindet ausserhalb
des Intervalls $[a,b]$, so dass das Integral über die Kurve nur
vom Teil im Inneren der Karte abhängt.
Damit besteht die Möglichkeit, das Integral von $\alpha$ über
die ganze Kurve zusammenzusetzen aus Summanden, die jeweils nur
innerhalb eines Kartengebietes von $0$ verschieden sind.
Zusätzlich sollten jeweils nur endlich viele Summanden gleichzeitig
von 0 verschieden sein, da sonst auch noch ein Begriff des Grenzwertes
einer unendlichen Summe von 1-Formen definiert werden müsste.
Es muss also untersucht werden, ob es möglich ist, aus einer
Überdeckung des Definitionsbereichs der Kurve durch offene Intervalle
eine Überdeckung auszuwählen, die jeden Punkt des Definitionsgebietes
nur in endlich vielen Mengen enthält.

\begin{satz}
\label{buch:kurvenintegral:zerlegung:satz:kompakt}
Sei $U_\alpha \subset [a,b]$, $\alpha\in I$, eine Familie offener
Intervalle in $\mathbb{R}$, die die ganze Menge $[a,b]$ überdecken,
also
\[
\bigcup_{\alpha\in I} U_\alpha \supset [a,b].
\]
Dann gibt es eine endliche Teilmenge $J\subset I$ derart, dass
\[
\bigcup_{\alpha\in J}U_\alpha\supset [a,b].
\]
\end{satz}

\begin{proof}
Wir beweisen die Aussage mithilfe eines Widerspruchs.
Wir nehmen also an, dass in jeder Vereinigung von endlich vielen
der Mengen $U_\alpha$ mindestens ein Punkt von $[a,b]$ nicht
enthalten ist.
Wir führen dies zu einem Widerspruch.

Als ersten Schritt dürfen wir annehmen, dass die Familie $I$ abzählbar
unendlich ist.
Die Menge der rationalen Zahlen im Intervall $[a,b]$ ist nämlich
abzählbar.
Indem wir zu jeder rationalen Zahl in $[a,b]$ eine der Mengen $U_\alpha$
auswählen, die die Zahl enthält, erhalten wir eine abzählbare Familie
von offenen Intervallen, die alle rationalen Zahlen in $[a,b]$
enthalten.
Da die rationalen Zahlen dicht sind im Intervall $[a,b]$, muss die
Vereinigung der Intervalle ganz $[a,b]$ enthalten.
Wir nehmen daher im folgenden an, dass $I=\mathbb{N}$ ist.

Nach Voraussetzung gibt es zu jeder natürlichen Zahl $n$ ein
\begin{equation}
x_n \in [a,b] \setminus \bigcup_{i=1}^n U_i.
\label{buch:kurvenintegral:zerlegung:eqn:xn}
\end{equation}
Die Folge $x_n$ ist beschränkt und enthält daher eine konvergente
Teilfolge, die gegen $x=\lim_{n\to\infty}x_n\in[a,b]$ konvergiert.
Da die Gesamtheit der $U_i$ das Intervall $[a,b]$ überdecken, gibt
es ein $n$ derart, dass $x\in U_n$.
Da $x$ der Grenzwert einer Teilfolge von $x_n$ ist, müssen unendlich
viele Punkte der Folge in $U_n$ sein, was der Konstruktion
\eqref{buch:kurvenintegral:zerlegung:eqn:xn}
von $x_n$ widerspricht.
Der Widerspruch zeigt, dass sich $[a,b]$ mit endlich vielen
Mengen der Familie überdecken lässt.
\end{proof}

Die in Satz~\ref{buch:kurvenintegral:zerlegung:satz:kompakt}
formulierte Eigenschaft ist als {\em Kompaktheit} bekannt.

\begin{definition}[Kompaktheit]
Ein topologischer Raum heisst {\em kompakt}, wenn jede Familie
\index{kompakt}%
offener Teilmenge, die den Raum überdecken, eine endliche Teilfamilie
enthält, die den Raum ebenfalls überdecken.
\end{definition}

Kompaktheit ist also genau die Endlichkeitsbedingung, die wir benötigen
um sicherzustellen, dass wir nur jeweils endlich viele Summanden
zusammenfügen müssen.

%
% Differenzierbre Zerlegung der Einheit
%
\subsection{Differenzierbare Zerlegung der Einheit
\label{buch:kurvenintegral:zerlegug:subsection:dze}}
Sei $M$ eine $n$-dimensionale Mannigfaltigkeit mit einem Atlas mit
den Kartengebieten $U_\alpha$, $\alpha\in I$.
Sei ausserdem $\gamma\colon [a,b]\to M$ eine differenzierbare
Kurve in $M$.
Die Mengen
\[
V_\alpha
=
\gamma^{-1}(U_\alpha)
=
\{
t\in[a,b]
\mid
\gamma(t)\in U_\alpha
\}
\]
sind offene Teilmengen von $[a,b]$.
Sie müssen nicht notwendiger zusammenhängend sein, die Kurve kann
das Gebiet $U_\alpha$ mehrmals verlassen und wieder neu betreten.
Jedes $V_\alpha$ kann zerlegt werden in endlich viele Intervalle.
Damit ist eine Familie von offenen Intervallen gefunden, die ganz
$[a,b]$ abdecken.

Die Kompaktheit von $[a,b]$ bedeutet, dass es auch eine endliche
Familie von Mengen $V_i$, $i=1,\dots,N$ gibt mit der Eigenschaft,
dass
\[
\bigcup_{i=1}^N V_i = [a,b].
\]
Jede der Mengen $V_i$ ist ein Intervall $V_i=(a_i,b_i)$.
Die Funktionen $g_i(x) = g_{a_i,b_i}(x)$ ist genau in $V_i$
von $0$ verschieden.
Da die Intervalle $V_i$ ganz $[a,b]$ überdecken, ist die
Summe
\begin{equation}
G(x) = \sum_{i=1}^N g_i(x) \ne 0
\label{buch:kurvenintegral:zerlegung:eqn:sum}
\end{equation}
für alle $x\in[a,b]$.
Da es auserdem nur endlich viele Funktionen gibt, hat die Summe
\eqref{buch:kurvenintegral:zerlegung:eqn:sum}
nur endlich viele Summanden, es gibt keine offenen Fragen der
Konvergenz.

Da die Funktion $G(x)$ beliebig oft stetig differenzierbar ist
und nirgends verschwindet, ist auch $1/G(x)$ beliebig oft stetig
differenzierbar.
%
% fig-zerlegung.tex
%
% (c) 2025 Prof Dr Andreas Müller
%
\begin{figure}
\centering
\includegraphics{chapters/030-kurvenintegral/images/zerlegung.pdf}
\caption{Zerlegung der Einheit mit sechs Intervallen $[a_i,b_i]$,
ermittelt mit der Methode dieses Kapitels.
Zunächst wurden die Funktionen $g_i = g_{a_i,b_i}$ konstruiert, dann
wurden die Funktionen $h_i$ durch Normierung mit der Summe der $g_i$
gewonnen.
Die unterste Graphik zeigt, wie die funktionen zusammen wieder den
Wert $1$ ergeben.
\label{buch:kurvenintegral:fig:zerlegung}}
\end{figure}
%
Damit können die neuen Funktion
\[
h_i(x)
=
\frac{1}{G(x)}\,g_i(x)
=
\frac{1}{G(x)}\,g_{a_i,b_i}(x)
\]
definiert werden.
Jede Funktion $h_i(x)$ ist beliebig oft stetig differenzierbar und
es gilt
\[
\sum_{i=1}^N h_i(x)
=
\frac{1}{G(x)}
\underbrace{\sum_{i=1}^N g_i(x)}_{\displaystyle=G(x)}
=
1.
\]
Die Funktionen $h_i(x)$ bilden was man eine {\em Zerlegung der Einheit}
nennt.
\index{Zerlegung der Einheit}%
Dieses Vorgehen zur Gewinnung einer Zerlegung der Einheit ist auch
in Abbildung~\ref{buch:kurvenintegral:fig:zerlegung} illustriert.

%
% Zerlegung eines Kurvenintegrals
%
\subsection{Zerlegung eines Kurvenintegrals}
Sei $\omega$ eine 1-Form auf der $n$-dimensionalen Mannigfaltigkeit $M$,
$\gamma\colon[a,b]\to M$ eine Kurve in $M$
und seien die Funktionen $h_i(x)$ eine Zerlegung der Einheit wie in
Abschnitt~\ref{buch:kurvenintegral:zerlegug:subsection:dze} konstruiert.
Auf jeder Menge $V_i$ lässt sich die 1-Form $(T\gamma)_*(\omega)$ mit
Hilfe einer Karte berechnen.
Die 1-Form
\[
\omega_i
=
h_i\, (T\gamma)^*(\omega)
\]
verschwindet ausserhalb des Intervalls $V_i=[a_i,b_i]$.
Die Summe ist
\[
\sum_{i=1}^N \omega_i
=
\]
Das Integral entlang der Kurve ist daher
\[
\int_\gamma \omega_i
=
\int_{a_i}^{b_i} h_i(x(t)) \langle \omega(x), \dot{x}(t)\rangle\,dt.
\]
Das Integral von $\omega$ kann jetzt in eine Summe von Integralen
zerlegt werden, die einzeln in einer Karte berechnet werden können:
\[
\int_\gamma\omega
=
\sum_{i=1}^N \int_{\gamma} \omega_i.
\]
Eine differenzierbare Zerlegung der Einheit ermöglich also, jedes
Kurvenintegral auf differenzierbare Art in Integrale innerhalb
einer Karte zu zerlegen.
Damit ist sichergestellt, dass das Integral einer $1$-Form
auf einer Kurve einen Sinn unabhängig davon hat, wie die Mannigfaltigkeit
mit Karten überdeckt wird.



\uebungsabschnitt

\aufgabetoplevel{chapters/030-kurvenintegral/uebungsaufgaben}
\begin{uebungsaufgaben}
\uebungsaufgabe{301}
\uebungsaufgabe{302}
\end{uebungsaufgaben}
\enduebungsabschnitt

