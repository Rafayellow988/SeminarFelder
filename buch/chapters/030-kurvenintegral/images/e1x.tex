%
% e1x.tex -- 
%
% (c) 2021 Prof Dr Andreas Müller, OST Ostschweizer Fachhochschule
%
\documentclass[tikz]{standalone}
\usepackage{amsmath}
\usepackage{times}
\usepackage{txfonts}
\usepackage{pgfplots}
\usepackage{csvsimple}
\usetikzlibrary{arrows,intersections,math}
\begin{document}
\def\skala{1}
\definecolor{darkred}{rgb}{0.8,0,0}
\begin{tikzpicture}[>=latex,thick,scale=\skala]

\draw[->] (-3.1,0) -- (9.5,0) coordinate[label={$x$}];
\draw[->] (0,-0.1) -- (0,3.4) coordinate[label={right:$y=g(x)$}];

\draw[color=darkred,line width=1.4pt] 
	plot[domain=3:0.1,samples=100]  ({3*\x},{3*exp(-1/\x)})
	--
	plot[domain=10:100,samples=100] ({3/\x},{3*exp(-\x)})
	--
	(0,0) -- (-3,0);

\draw (-3,-0.05) -- ++(0,0.1);
\node at (-3,0) [below] {$-1$};
\draw (3,-0.05) -- ++(0,0.1);
\node at (3,0) [below] {$1$};
\draw (6,-0.05) -- ++(0,0.1);
\node at (6,0) [below] {$2$};
\draw (9,-0.05) -- ++(0,0.1);
\node at (9,0) [below] {$3$};

\node[color=darkred] at ({1.5*3},{3*exp(-1/1.5)})
		[below right] {$g(x)=e^{-1/x}$};
\node[color=darkred] at (-1.5,0) [above] {$g(x)=0$};

\end{tikzpicture}
\end{document}

