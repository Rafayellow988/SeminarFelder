%
% intro.tex -- Einleitung zum Thema
%
% !TEX root = ../../buch.tex
% !TEX encoding = UTF-8
%
\section{Einleitung}

Wirbelringe sind ein interessantes Naturphänomen welchem man schneller begegnet als man denkt. 
In diesem Paper schauen wir uns Wirbelringe etwas genauer an und begründen einige Eigenschaften. 
Des Weiteren untersuchen wir eine weitverbreitete praktische (ungewollte) Anwendung und dessen Auswirkung. 
Zuletzt formulieren wir ein Modell womit man zumindest angenähert selbst gezielt Wirbelringe berechnen und erzeugen kann.

In diesem Paper werden, wo nicht anders angegeben, nur ideale Fluide betrachtet.

\begin{quote}
    The theory of inviscid fluids is the study of «dry water».

    - John von Neumann \cite{Wirbelringe:feynman1964lectures}
\end{quote}

\subsection{Wirbel}

\begin{figure}
\centering
\begin{tikzpicture}
\clip (-6.3,-2.7) rectangle (6.3,2.7);
\node at (-0.07,-0.07) {\includegraphics[width=1.08\textwidth]{papers/wirbelringe/fig/flacher_wirbel.pdf}};
%\draw[color=red] (-6.3,-2.7) rectangle (6.3,2.7);
\end{tikzpicture}
\caption{Darstellung eines 2-dimensionalen Wirbels mit Wirbelvektor
\(\vec{\omega}\).
\label{Wirbelringe:fig:flacher_wirbel}}
\end{figure}


Für die Betrachtung von Wirbelringen starten wir zunächst mit einzelnen Wirbeln.
Ein Wirbel ist eine Formation von Teilchen, welche um einen Mittelpunkt rotieren.
In Abbildung \ref{Wirbelringe:fig:flacher_wirbel} ist ein Wirbel abgebildet.
Die eingezeichneten Pfeile stellen die Geschwindigkeitsvektoren \( \vec{v} \) von den Teilchen dar, welche Teil des Wirbels sind.
Nicht explizit eingezeichnet sind die Bahnen, auf welchen sich die Teilchen bewegen.
Die Bahnen bilden konzentrische Kreise.
Auch ist in der Abbildung \ref{Wirbelringe:fig:flacher_wirbel} der Wirbelstärkevektor \(\vec{\omega}\) eingezeichnet.
Wir kommen später genauer auf \(\vec{\omega}\) zu sprechen.

\subsubsection*{Wirbellinien \label{Wirbelringe:Wirbellinien}}

Um Wirbel besser zu beschreiben, führen wir hier den Begriff {\em Wirbellinie} ein.
Wirbellinien sind die „Mittelachsen“ eines Wirbels. 
Um diese Achse rotieren die Teile, welche Teil eines Wirbels sind. 
Diese hat an sich kein Volumen, allerdings kann es sein, dass Teilchen auf dieser Achse zu liegen kommen. 
\(\vec{\omega}\) steht tangential zu dieser Wirbellinie.  
In der Praxis ist eine Wirbellinie nicht gerade, sondern gekrümmt oder sogar spiralenähnlich. 
Des Weiteren ist eine sehr wichtige Eigenschaft, dass Wirbellinien nur auf einer Grenzfläche enden können, 
wie wir in Abschnitt \ref{Wirbelringe:Grenzflaechen} sehen werden.

\subsection{Stabilität}

Um die Stabilität von Wirbeln zu beurteilen, können wir betrachten wie sich die Menge an Teilchen verhaltet, ob sie wächst oder schrumpft. 
Also die Divergenz der Teilchen, die Teil eines Wirbels sind.
Mit dieser Frage kommt man auf die Rechnung
\[
\operatorname{div} ( \operatorname{rot} ( \vec{A} ) )
= % \overset{!}{=}
?,
\]
wobei wir für \(\vec{A}\) das Vektorfeld aus Abbildung \ref{Wirbelringe:fig:flacher_wirbel} verwenden.
Nehmen wir an das \(\vec{A}\) zweimal stetig differenzierbar ist, so erhalten wir
\begin{align*}
\operatorname{div} ( \operatorname{rot} ( \vec{A} ) )
&=
\operatorname{div}      
    \begin{pmatrix} 
        \frac{\partial A_z}{\partial y} - \frac{\partial A_y}{\partial z} \\ 
        \frac{\partial A_x}{\partial z} - \frac{\partial A_z}{\partial x} \\ 
        \frac{\partial A_y}{\partial x} - \frac{\partial A_x}{\partial y} \\ 
    \end{pmatrix} \\
&=
\frac{\partial^2 A_z}{\partial x \partial y} - \frac{\partial^2 A_y}{\partial x \partial z} + 
\frac{\partial^2 A_x}{\partial y \partial z} - \frac{\partial^2 A_z}{\partial y \partial x} +
\frac{\partial^2 A_y}{\partial z \partial x} - \frac{\partial^2 A_x}{\partial z \partial y}
\\
&=
\frac{\partial^2 A_z}{\partial x \partial y} - \frac{\partial^2 A_z}{\partial y \partial x} + 
\frac{\partial^2 A_x}{\partial y \partial z} - \frac{\partial^2 A_x}{\partial z \partial y} +
\frac{\partial^2 A_y}{\partial z \partial x} - \frac{\partial^2 A_y}{\partial x \partial z}
\quad \text{(Satz von Schwarz)}\\
&=
\overbrace{\frac{\partial^2 A_z}{\partial x \partial y} - \frac{\partial^2 A_z}{\partial x \partial y}}^0 + 
\overbrace{\frac{\partial^2 A_x}{\partial y \partial z} - \frac{\partial^2 A_x}{\partial y \partial z}}^0 +
\overbrace{\frac{\partial^2 A_y}{\partial x \partial z} - \frac{\partial^2 A_y}{\partial x \partial z}}^0
\\
&=
0
\end{align*}

Aus dem Resultat
\begin{equation}
    \label{Wirbelringe:eq:wIdent}
\operatorname{div} ( \operatorname{rot} ( \vec{A} ) ) 
= 
0
\end{equation}
lässt sich schliessen, dass die Anzahl der Teilchen in einem Rotationsfeld konstant bleibt. 
Das bedeutet, dass sie immer weiter rotieren. 
Aus diesem Grund sind Wirbel sehr stabil.

\subsection{Vom Wirbel zum Wirbelring}

Um aus Wirbeln ein Wirbelring zu machen brauchen wir noch eine Definition.

\subsubsection*{Wirbelfäden}

Ein {\em Wirbelfaden} ist ein Zylinder, welcher eine Wirbellinie als Zentrum des Zylinders hat.
Schneidet man nun diesen Zylinder senkrecht zu der Wirbellinie, ergibt sich ein einzelner Wirbel.
Wirbelfäden werden auch {\em Wirbelröhren} genannt.

Um aus einem Wirbelfaden ein Wirbelring zu machen, schliessen wir die offenen Enden (wir betrachten später, wie man mit diesen umzugehen hat) zusammen.
Die Wirbellinie formt ein Kreis.
Jetzt ist es ein Wirbelring. 
Solch ein Wirbelring ist in Abbildung \ref{Wirbelringe:fig:generell} dargestellt.

\begin{figure}
\centering
\includegraphics[width=1\textwidth]{papers/wirbelringe/images/wirbelring.png}
\caption{Typischer idealer Wirbelring.
Dargestellt durch momentane Bewegungsvektoren unterschiedlicher Teilchen in regelmässigem Abstand.
Zur besseren Übersicht sind Teilchen eines Wirbels mit derselben Farbe markiert.
Unterschiedliche, benachbarte Wirbel haben unterschiedliche Farben.
Die Wirbellinie ist als silbrige Linie eingezeichnet.
Ein einzelner Wirbelvektor ist violett in der Aussparung eingezeichnet.\label{Wirbelringe:fig:generell}}
\end{figure}
