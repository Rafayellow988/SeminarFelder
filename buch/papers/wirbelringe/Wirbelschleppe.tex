\section{Wirbelschleppe}
An einem nebligen Tag kann man sie an einem Flughafen gut beobachten. 
Die Rede ist von den Wirbelschleppen hinter den Enden der Tragflächen eines Flugzeugs.
Aber wieso entstehen sie eigentlich? 
Wo genau starten sie und wo ist deren Ende?
Wieso schauert es Kleinflugzeug-Piloten, wenn sie diesen Begriff hören?
All das und mehr klären wir in diesem Abschnitt mit ein wenig Mathematik.

\subsection{Entstehung}
Wirbelschleppen bilden sich genau am Ende der Tragfläche aufgrund des Druckunterschieds, welcher das Flugzeug in erster Linie Fliegen lässt.
Um deren Entstehung genauer anschauen zu können gibt es nun einen kleinen Theorieausschnitt aus der Aviatik:
Setzt sich ein Flugzeug in Bewegung, so entsteht um die Flügeltragflächen ein Luftstrom.
Ein weitverbreiteter Irrglaube besagt, dass die Luft, welche über den Flügel geht "länger" hat und deshalb schneller sein muss.
Dies ist jedoch nicht ganz korrekt.
Es ist vielmehr die \textbf{Fläche} eines einzelnen, gedachten Luftstroms, welche auf dem Flügel verkleinert wird.% TODO: Genauere Erklärung noch anfügen!
Wenn man dann mittels Kontinuitätsgleichung
\[v_{oben}A_{oben} = v_{unten}A_{unten}\]
die Geschwindigkeiten der beiden gedachten Luftströme vergleicht, muss der Luftstrom mit Geschwindigkeit $v_{oben}$ schneller sein als $v_{unten}$.
Dies hat jetzt zur Folge, dass laut dem Satz von Bernoulli 
\[p+\frac{1}{2}\rho v^2+\rho gh=konstant\]
der Druck $p_{oben}$ kleiner sein muss als der Druck $p_{unten}$.
Denn zwei punkte, einer auf der Oberseite des Flügels und einer auf der Unterseite können folgendermassen gleichgesetzt werden
\[p_{oben}+\frac{1}{2}\rho v^2_{oben} + \rho gh_1 = p_{unten}+\frac{1}{2}\rho v^2_{unten}+\rho gh_2.\]
Jetzt kann angenommen werden, dass der Höhenunterschied zwischen den beiden Punkten ob/unter dem Flügel vernachlässigbar klein ist also \(h_1\approx h_2\).
Zudem ist die Dichte der Luft an beiden Punkten ungefähr gleich weshalb sich der Term zu folgendem vereinfacht
\[p_{oben}+\frac{1}{2}\rho v^2_{oben} = p_{unten}+\frac{1}{2}\rho v^2_{unten}.\]
Dann noch umgestellt nach $p_{oben}$ und $p_{unten}$
\[p_{oben}-p_{unten} = \frac{1}{2}\rho( v^2_{unten}-v^2_{oben})\]
und man kann erkennen, dass wenn $v_{unten} < v_{oben}$ muss $p_{unten} > p_{oben}$ sein.