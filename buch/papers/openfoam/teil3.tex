%
% teil3.tex -- Beispiel-File für Teil 3
%
% (c) 2020 Prof Dr Andreas Müller, Hochschule Rapperswil
%
% !TEX root = ../../buch.tex
% !TEX encoding = UTF-8
%
\section{Anwendungsbeispiel
\label{openfoam:section:Anwendungsbeispiel}}
\kopfrechts{Anwendungsbeispiel}
Wie wir gesehen haben, kann OpenFOAM, mit genügend Geduld, alles. 
Wir werden jetzt an einem praktischen Beispiel zeigen, wie man eine eigene Simulation erstellt.

Für ein anschauliches Beispiel haben wir ein Objekt gewählt, dass alle Studenten der OST-Rapperswil kennen: den Campus selbst.
Das Ziel der Simulation ist, zu sehen, wie der Wind um die Gebäude strömt und mit ihnen interagiert.

Als erstes brauchen wir ein geeignetes 3d Objekt, in unserem Fall wurden die Gebäude in Onshape erstellt.
%hint, open foam arbeite in Meter, wenn das Objekt in mm exportiert wurde, faktor 1000 zu gross
Dieses 3D file bildet die Grundlage unsere Simulation.

%Bild Ordnerstruktur_Leer
%\subsubsection{Mesh generating}
\subsubsection{Mesh generating\label{openfoam:section:Mesh generating}}
mesh generating ist der Prozess, bei dem das 3D Objekt in ein Netz aus Zellen unterteilt wird, die von OpenFOAM verarbeitet werden können.


%Bild Snappy fein und grob 
%\subsubsection{Parameter}
\subsubsection{Parameter\label{openfoam:section:Parameter}}
Mit den Parameter werden alle Simulationsspezifischen Einstellungen vorgenommen.

%\subsubsection{Numerische Parameter}
\subsubsection{Numerische Parameter\label{openfoam:section:Numerische Parameter}}
Zeitschritte, Iterationen, Toleranzen   
Zeitschritte usw "system"

%\subsubsection{Physikalische Parameter}
\subsubsection{Physikalische Parameter\label{openfoam:section:Physikalische Parameter}}
iskosität, dichte ect "constant"

%\subsubsection{Anfangsbedingungen Parameter}
\subsubsection{Anfangsbedingungen Parameter\label{openfoam:section:Anfangsbedingungen Parameter}}
Strömungsrichtung, geschwinigkeiten usw, "0" Ordner

%\subsubsection{Simulation}
\subsubsection{Simulation\label{openfoam:section:Simulation}}
simulation


%Bild Ordnerstruktur_Simuliert
%Bild vorschlag_Wind_Westen_10m_blocky?

\subsubsection{Postprocessing}
%\subsection{Postprocessing\label{openfoam:section:Postprocessing}}
Postprocessing ist der Prozess, bei dem die Simulationsergebnisse visualisiert und analysiert werden. 
%Bild vorschlag_Wind_Osten_5m oder Westen, evt interessant mehrere 5/10/15m

%https://doc.openfoam.com/2306/quickstart/
%https://www.openfoam.com/documentation/user-guide Dokumentation

\subsubsection{Ablauf\label{openfoam:section:Ablauf}}

surfaceFeatureExtract, erstellt .eMesh datei und extrahiert alle features des Objekts

blockMesh, grösse der Zellen definieren wie genau und auswendig die Simulation wird

snappyHexMesh, im dict den ort des 3d Obj und eMesh files angeben, dauert lange

simpleFoam, vor dem start die startwerte von 0.orig in 1 und 2 kopieren

abschluss für nicht windows user

foamToVTK
paraFoam -touch, erstellt .OpenFOAM datei, welche dann mit Paraview geöffnet wird

darstellung
Paraview, navigiere zum Simulations ordner, öffne .openFOAM datei