%
% teil3.tex -- Beispiel-File für Teil 3
%
% (c) 2020 Prof Dr Andreas Müller, Hochschule Rapperswil
%
% !TEX root = ../../buch.tex
% !TEX encoding = UTF-8
%
\section{Anwendungsbeispiel
\label{openfoam:section:Anwendungsbeispiel}}
\kopfrechts{Anwendungsbeispiel}
Wie wir gesehen haben, kann OpenFOAM, mit genügend Geduld, alles. 
Wir werden jetzt an einem praktischen Beispiel zeigen, wie man eine eigene Simulation erstellt.

Für ein anschauliches Beispiel haben wir ein Objekt gewählt, dass alle Studenten der OST-Rapperswil kennen: den Campus selbst.
Das Ziel der Simulation ist, zu sehen, wie der Wind um die Gebäude strömt und mit ihnen interagiert.

Als erstes brauchen wir ein geeignetes 3d Objekt, in unserem Fall wurden die Gebäude in Onshape erstellt.
%hint, open foam arbeite in Meter, wenn das Objekt in mm exportiert wurde, faktor 1000 zu gross
Dieses 3D file bildet die Grundlage unsere Simulation.

%Bild Ordnerstruktur_Leer

%Bild Snappy fein und grob 

\subsubsection{Mesh generating\label{openfoam:section:Mesh generating}}
Der grundlegende Schritt einer jeder Simulation ist, die reale, analoge Welt in eine diskrete Welt zu überführen.
Mithilfe der \textbf{snappyHexMesh} funktion wird das 3d-Objekt in disktrete Zellen unterteilt. Je mehr Zellen desto genauer
wird die Simulation, desto grösser jedoch der Aufwand für die Simulation. Diese Abwägung, Qualität vs. Aufwand, muss vom Auftraggeber definiert werden.
Die Zellen können eine freie Form annehmen, oft wird in einem ersten Durchlauf der Simulationsraum grob in Quader aufgeteilt (\ref{fig:snappygrobbild}),
in folgenden Schritten wird die Annäherung and das Grundmodell immer besser (\ref{fig:snappyfeinbild}).


\begin{figure}
    \centering
    \includegraphics[width=0.55\textwidth]{papers/openfoam/Bilder/Snappy_grob.png}
    \caption{Mesh nach dem ersten Durchlauf von snappyHexMesh}
    \label{fig:snappygrobbild}
\end{figure}

\begin{figure}
    \centering
    \includegraphics[width=0.55\textwidth]{papers/openfoam/Bilder/Snappy_fein.png}
    \caption{Mesh nach dem zweiten Durchlauf von snappyHexMesh}
    \label{fig:snappyfeinbild}
\end{figure}
%-------------------------------------------------------------------------------
\subsubsection{Parameter\label{openfoam:section:Parameter}}
Im folgenden einen kurzen Überblick über die wichtigsten Parameter einer OpenFOAM Simulation.
%TODO, die einzelnen paramter als liste in liste, falls das möglich ist

\subsubsection{Numerische Parameter\label{openfoam:section:Numerische Parameter}}
Mit den numerischen Paramter im File "controlDict" wird festgelegt
\begin{numpam}
    \item \textbf{Zeitschrittgrößen (deltaT):} Bestimmt, wie fein der zeitliche Ablauf der Simulation aufgelöst wird. Zu große Zeitschritte können zu Instabilität führen, zu kleine verlängern die Rechenzeit unnötig.
    \item \textbf{Endzeit:} Gibt an, bis zu welchem Zeitpunkt simuliert werden soll.
    \item \textbf{Solver-Einstellungen:} Hier wird definiert, welche numerischen Methoden verwendet werden (z.B. Diskretisierungsschemata, Iterationsverfahren) und welche Konvergenzkriterien gelten.
    \item \textbf{Schreibintervall:} Legt fest, wie oft während der Simulation Ergebnisse gespeichert werden.
\end{numpam}

\subsubsection{Physikalische Parameter\label{openfoam:section:Physikalische Parameter}}
Viskosität, dichte ect "constant"
\begin{physpam}
    \item \textbf{Dichte ($\rho$):} Relevant für alle kraftbezogenen Berechnungen, z.B. Auftrieb, Trägheit oder Druckunterschiede.
    \item \textbf{Viskosität ($\mu$):} Bestimmt, wie stark die innere Reibung im Fluid die Strömung hemmt. Besonders wichtig für laminare oder turbulente Strömungen.
    \item \textbf{Turbulenzmodell:} Hier wird festgelegt, ob und wie Turbulenzen berücksichtigt werden (z.B. RANS, LES oder laminar).
\end{physpam}

\subsubsection{Anfangsbedingungen Parameter\label{openfoam:section:Anfangs und Randbedingungen}}
Strömungsrichtung, geschwinigkeiten usw, "0" Ordner
\begin{anfrandbed}
    \item \textbf{Strömungsrichtung und Geschwindigkeit:} Gibt an, aus welcher Richtung der Wind weht und mit welcher Geschwindigkeit.
    \item \textbf{Druckverhältnisse:} Setzt Referenzwerte für den statischen Druck, z.B. atmosphärisch.
    \item \textbf{Wandbedingungen (no-slip, roughness):} Definiert, ob an Oberflächen die Strömung haften bleibt oder rutschen kann.
    \item \textbf{Initialfelder:} Erste Näherung für Geschwindigkeit und Druck, um die Berechnung stabil zu starten.
\end{anfrandbed}



%-------------------------------------------------------------------------------
\subsubsection{Simulation\label{openfoam:section:Simulation}}
In der Simulation wird nun auf jede Zelle die im Mesh generating die für die Simulation relevanten Gleichungen angewandt. 

%Bild Ordnerstruktur_Simuliert
%Bild vorschlag_Wind_Westen_10m_blocky?


%https://doc.openfoam.com/2306/quickstart/
%https://www.openfoam.com/documentation/user-guide Dokumentation

\subsubsection{Ablauf\label{openfoam:section:Ablauf}}
surfaceFeatureExtract, erstellt .eMesh datei und extrahiert alle features des Objekts

blockMesh, grösse der Zellen definieren wie genau und auswendig die Simulation wird

snappyHexMesh, im dict den ort des 3d Obj und eMesh files angeben, dauert lange

simpleFoam, vor dem start die startwerte von 0.orig in 1 und 2 kopieren

abschluss für nicht windows user

\subsubsection{Postprocessing}
%\subsection{Postprocessing\label{openfoam:section:Postprocessing}}
Postprocessing ist der Prozess, bei dem die Simulationsergebnisse visualisiert und analysiert werden. 
%Bild vorschlag_Wind_Osten_5m oder Westen, evt interessant mehrere 5/10/15m



foamToVTK
paraFoam -touch, erstellt .OpenFOAM datei, welche dann mit Paraview geöffnet wird

darstellung
Paraview, navigiere zum Simulations ordner, öffne .openFOAM datei