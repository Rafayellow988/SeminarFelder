%
% teil3.tex -- Beispiel-File für Teil 3
%
% (c) 2020 Prof Dr Andreas Müller, Hochschule Rapperswil
%
% !TEX root = ../../buch.tex
% !TEX encoding = UTF-8
%
\section{Anwendungsbeispiel
\label{openfoam:section:teil3}}
\kopfrechts{Anwendungsbeispiel}
Wie wir gesehen haben, kann OpenFOAM, mit genügend Geduld, mehr oder weniger alles. 
Wir werden jetzt an einem praktischen Beispiel zeigen, wie man eine eigene Simulation erstellt.

Für ein anschauliches Beispiel haben wir ein Objekt gewählt, dass alle Studenten der OST-Rapperswil kennen: den Campus selbst.
Das Ziel der Simulation ist, zu sehen, wie der Wind um die Gebäude strömt und mit ihnen interagiert.

Als erstes brauchen wir ein geeignetes 3d Objekt, in unserem Fall wurden die Gebäude in Onshape erstellt, und mit einer (dünnen) Bodenplatte verbunden, damit alle Gebäude in einem File zusammengefasst sind.
%hint, open foam arbeite in Meter, wenn das Objekt in mm exportiert wurde, faktor 1000 zu gross
Dieses 3D file bildet die Grundlage unsere Simulation.

%beispiel bild ordnerstruktur

\subsection{Mesh generating\label{openfoam:section:Mesh generating}}
\subsection{Parameter\label{openfoam:section:Parameter}}
\subsection{Numerische Parameter\label{openfoam:section:Numerische Parameter}}
\subsection{Physikalische Parameter\label{openfoam:section:Physikalische Parameter}}
\subsection{Anfangsbedingungen Parameter\label{openfoam:section:Anfangsbedingungen Parameter}}
\subsection{Simulation\label{openfoam:section:Simulation}}
\subsection{Postprocessing\label{openfoam:section:Postprocessing}}


