
\section{Wedgeprodukt
\label{geoalgebra:section:wedgeproduct}}
\kopfrechts{Wedgeprodukt}

Das \emph{Wedgeprodukt} bietet die grundlegende Operation der geometrischen Algebra. Angenommen, wir haben die
zwei Vektoren $\mathbf{v_1}, \mathbf{v_2}$ und bilden das Wedgeprodukt
\begin{equation}
  B = \mathbf{v_1} \wedge \mathbf{v_2}
\end{equation}}
erhalten wir ein neues mathematisches Konzept, den \emph{Bivektor} $B$.

\subsection{Definition}
Um das Wedgeprodukt definieren zu können, benötigen wir einige wenige Axiome, d.h. grundlegende Annahmen, die wir treffen,
um uns eine Algebra darauf aufzubauen. Somit definieren wir

\begin{satz}
e_i \wedge e_i = 0
\end{satz}

\begin{satz}
e_i \wedge e_j = -e_j \wedge e_i
\end{satz}
Das Wedgeprodukt ist also \emph{antikommutativ}.
