\section{Rotationen}
\renewcommand{\u}{\hat{\mathbf{u}}}
\subsection{Rotationen sind zwei aufeinanderfolgende Spiegelungen}
Eine Rotation kann durch zwei aufeinanderfolgende Spiegelungen ausgedrückt werden (siehe
\autoref{geoalgebra:fig:rotation-as-two-reflections}).
Die Rotation wird am Drehpunkt $O$ durchgeführt, dem Schnittpunkt der beiden Vektoren $\u, \v$.
\begin{figure}
  \begin{center}
\begin{tikzpicture}[>=latex, scale=2, thick]
\fill[darkgreen!30] (-0.5, 0.5) -- (-0.5, 2) -- (-0.1, 0.9) -- cycle;
\draw[thick, ->] (0, 0) -- (0.5, 2) node[left]{$\v$};
\draw[thick, ->] (0, 0) -- (2, 0.5) node[above]{$\u$};
\fill[darkgreen!30] (0.68, 0.21) -- (1.38, 1.53) -- (0.51, 0.75) -- cycle;
\fill[darkgreen!30] (0.69, 0.13) -- (1.94, -0.70) -- (0.80, -0.42) -- cycle;
\draw[thick, blue] (-0.5, 0.5) -- (0.68, 0.21);
\draw[thick, blue] (-0.5, 2) -- (1.38, 1.53);
\draw[thick, blue] (-0.1, 0.9) -- (0.51, 0.75);

\node[below left] at (0,0) {$O$};

\draw[thick, darkred] (0.68, 0.21) -- (0.69, 0.13);
\draw[thick, darkred] (1.38, 1.53) -- (1.94, -0.70);
\draw[thick, darkred] (0.51, 0.75) -- (0.80, -0.42);
\end{tikzpicture}

  \end{center}
  \caption{Eine Rotation kann durch zwei aufeinanderfolgende Spiegelungen abgebildet werden}
\label{geoalgebra:fig:rotation-as-two-reflections}
\end{figure}

Dasselbe funktioniert auch in höheren Dimensionen. In drei Dimensionen hätten wir es mit einem
Rotationsvektor zu tun. Dieser Rotationsvektor ist orthogonal zu $\u$ und $\v$, zum Beispiel $\u \times \v$,
wobei die Länge irrelevant ist. Relevant ist ausserdem noch eine fixe Position des Vektors, in diesem Fall
$O$.

\subsection{Rotation eines Vektors}
Man betrachte den Vektor $\mathbf{x}$, der zunächst an $\v$ und danach an $\u$ mit Zwischenwinkel $\theta$ gespiegelt wird (siehe \autoref{geoalgebra:fig:rotation}).
Die Spiegelung $\mathbf{x}'$ von $\mathbf{x}$ an $\v$ ergibt sich durch
\begin{equation}
\mathbf{x}' = \v \mathbf{x} \v.
\end{equation}
Analog dazu kann der neu entstandene Vektor $\mathbf{x}'$ an $\u$ gespiegelt werden
\begin{equation}
\mathbf{x}'' = \u \mathbf{x}' \u.
\end{equation}
Die ganze Rotation wird somit durch
\begin{equation}
\mathbf{x}'' = \u \v \mathbf{x} \v \u
  \label{geoalgebra:eq:rotation}
\end{equation}
berechnet. Der Rotationswinkel beträgt dabei genau $2\theta$.
\begin{figure}
  \begin{center}
\begin{tikzpicture}[>=latex, scale=2]
\draw[thick, darkgreen, ->] (0, 0) -- (-0.5, 2) node[left]{$\mathbf{x}$};
\draw[thick, ->] (0, 0) -- (0.5, 2) node[left]{$\v$};
\draw[thick, ->] (0, 0) -- (2, 0.5) node[above]{$\u$};;
\draw[thick, darkgreen!30, ->] (0, 0) -- (1.38, 1.53) node[left]{$\mathbf{x}'$};
\draw[thick, darkgreen!30, ->] (0, 0) -- (1.94, -0.70) node[below]{$\mathbf{x}''$};
\draw[thick, blue] (-0.5, 2) -- (1.38, 1.53);
\draw[thick, darkred] (1.38, 1.53) -- (1.94, -0.70);
\end{tikzpicture}


  \end{center}
  \caption{Rotation eines Vektors}
\label{geoalgebra:fig:rotation}
\end{figure}
Verglichen mit der Drehmatrix in zwei Dimensionen
\begin{equation}
  R_{2\theta} = \begin{pmatrix}
    \cos{2\theta} & -\sin{2\theta} \\
    \sin{2\theta} & \cos{2\theta}
  \end{pmatrix}
\end{equation}
ist dieses Resultat sehr elegant und dazu gültig in \emph{beliebigen Dimensionen}!

\subsection{Rotoren}
Bei den bisherigen Rotationen ergibt sich das Problem, dass zwei Vektoren
in bestimmten Richtungen nötig sind, die die Rotation bestimmen. Möchten wir
jedoch um einen beliebigen Vektor $\theta$ rotieren, wird ein weiteres Konzept
benötigt: \emph{Rotoren}.

Ein Rotor $R$ ist definiert als
\begin{equation}
  R = \u \v,
\end{equation}
wobei $\u$ und $\v$ Einheitsvektoren sind.
Das Gegenstück zu $R$ ist
\begin{equation}
  \tilde{R} = \v \u.
\end{equation}
Der Ausdruck \eqref{geoalgebra:eq:rotation} kann mithilfe des Rotors $R = \u \v$ somit als
\begin{equation}
  \mathbf{x}'' = R \mathbf{x} \tilde{R}
\end{equation}
dargestellt werden.
Es gilt also
\begin{align}
  R &= \u \v = \u \cdot \v + \u \wedge \v \\
  \tilde{R} &= \v \u = \u \cdot \v - \u \wedge \v.
\end{align}
Das Skalarprodukt von zwei Einheitsvektoren $\u$ und $\v$ ist äquivalent zum Kosinus des Winkels dazwischen:
\begin{equation}
  \u \cdot \v = \cos{\theta}
\end{equation}
