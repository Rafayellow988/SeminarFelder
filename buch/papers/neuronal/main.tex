%
% main.tex -- Paper zum Thema <neuronal>
%
% (c) 2020 Autor, OST Ostschweizer Fachhochschule
%
% !TEX root = ../../buch.tex
% !TEX encoding = UTF-8
%
\chapter{Lösung von Feldgleichungen mit neuronalen Netzen\label{chapter:neuronal}}
\kopflinks{Feldgleichungen mit neuronalen Netzen}
\begin{refsection}
\chapterauthor{Nicola Dall'Acqua und Roman Cvijanovic}
\index{Nicola Dall'Acqua}%
\index{Dall'Acqua, Nicola}%
\index{Roman Cvijanovic}%
\index{Cvijanovic, Roman}%

\noindent
Die Dynamik von Feldern --- die raum-zeitliche Änderung der Feldgröße --- wird mittels Feldgleichungen beschrieben.
\index{partielle Differentialgleichung}%
Feldgleichungen sind partielle Differentialgleichungen und deren Lösungen sind Funktionen.
Das Lösen dieser Gleichungen ist von hoher Bedeutung in der Feldtheorie.

Es ist nicht immer möglich, eine explizite Lösung für partielle Differentialgleichungen zu finden, in diesen Fällen müssen Lösungen approximiert werden.
Hierzu gibt es diverse Approximationsverfahren.
Innerhalb dieses Papers soll eine neuere Methode --- Lösungen mittels neuronaler Netze --- vorgestellt werden.
\index{neuronales Netz}%

Zunächst wird die Methode theoretisch hergeleitet.
Anschliessend wird die konkrete Lösung für die Wellengleichung und die Burgers-Gleichung vorgestellt.
\index{Burgers-Gleichung}%
Zum Schluss werden die Methode sowie weitere Anwendungsmöglichkeiten diskutiert.

%% Ein paar Hinweise für die korrekte Formatierung des Textes
%% \begin{itemize}
%% \item
%% Absätze werden gebildet, indem man eine Leerzeile einfügt.
%% Die Verwendung von \verb+\\+ ist nur in Tabellen und Arrays gestattet.
%% \item
%% Die explizite Platzierung von Bildern ist nicht erlaubt, entsprechende
%% Optionen werden gelöscht. 
%% Verwenden Sie Labels und Verweise, um auf Bilder hinzuweisen.
%% \item
%% Beginnen Sie jeden Satz auf einer neuen Zeile. 
%% Damit ermöglichen Sie dem Versionsverwaltungssysteme, Änderungen
%% in verschiedenen Sätzen von verschiedenen Autoren ohne Konflikt 
%% anzuwenden.
%% \item 
%% Bilden Sie auch für Formeln kurze Zeilen, einerseits der besseren
%% Übersicht wegen, aber auch um GIT die Arbeit zu erleichtern.
%% \end{itemize}

%
% 1_herleitung.tex -- Herleitung der Methode
%
% (c) 2025 Roman Cvijanovic & Nicola Dall'Acqua, Hochschule Rapperswil
%
% !TEX root = ../../buch.tex
% !TEX encoding = UTF-8
%

\section{Herleitung der Methode\label{neuronal:section:herleitung}}
\kopfrechts{Herleitung der Methode}

Im Folgenden wird die Methode zum Lösen von Feldgleichungen mittels neuronaler Netze theoretisch hergeleitet.
Dies wird anhand des Beispiels der Wellengleichung gemacht
\begin{equation}
    \frac{\partial^2 u}{\partial t^2} = c^2 \left( \frac{\partial^2 u}{\partial x^2} + \frac{\partial^2 u}{\partial y^2} \right)
\end{equation}


Wobei \( u(x, y, t) \) die z-Koordinate am Punkt \( (x, y) \) zum Zeitpunkt \( t \) darstellt.
Anders ausgedrückt ist \( u \) eine gewellte Oberfläche.

Mittels des neuronalen Netzes \( \hat{u}(x, y, t; \theta) \) kann diese Funktion approximiert werden.\newline



%
% 2_beispiel.tex -- Wellengleichung tatsächlich lösen mit der Methode
%
% (c) 2025 Roman Cvijanovic & Nicola Dall'Acqua, Hochschule Rapperswil
%
% !TEX root = ../../buch.tex
% !TEX encoding = UTF-8
%

\section{Rechenbeispiele}\label{neuronal:section:rechenbeispiel}
\kopfrechts{Rechenbeispiele}

In diesem Abschnitt wird die zuvor vorgestellte Methode auf die Wellengleichung in zwei Dimensionen und die Burgers-Gleichung in einer Dimension angewendet.
Der Ablauf orientiert sich an den Schritten aus Abschnitt \ref{neuronal:section:herleitung}:
\begin{enumerate}
    \item Definition eines neuronalen Netzwerks
    \item Diskretisierung der Definitionsbereiche
    \item Aufbau der Funktion $L(\vartheta)$
    \item Minimierung von $L(\vartheta)$
    \item Qualitätsbewertung anhand von $L(\vartheta)$ und $L^1(\vartheta)$
\end{enumerate}

\subsection{Wellengleichung in zwei Dimensionen}\label{neuronal:subsection:wellengleichung}
Die zu lösende Gleichung lautet
\begin{equation}
    \frac{\partial^2 u}{\partial t^2} = c^2 \left( \frac{\partial^2 u}{\partial x^2} + \frac{\partial^2 u}{\partial y^2} \right).
    \label{neuronal:wellengleichung}
\end{equation}
Die Konstante \( c \) ist die Ausbreitungsgeschwindigkeit der Welle. Der Einfachheit halber wird \( c = 1 \) festgelegt.
Zusätzlich werden die folgenden Anfangsbedingungen
\begin{equation}
    \begin{aligned}
        u(x, y, 0) &= \sin(\pi x) \sin(\pi y)\\
        \frac{\partial u(x, y, 0)}{\partial t} &= 0
    \end{aligned}
    \label{neuronal:wellen_anfangs}
\end{equation}
sowie die Randbedingungen
\begin{equation}
    \begin{aligned}
        u(-2, y, t) &= 0\\
        u(2, y, t) &= 0\\
        u(x, -2, t) &= 0\\
        u(x, 2, t) &= 0
    \end{aligned}
    \label{neuronal:wellen_rand}
\end{equation}
verwendet.
Die Bereiche sind \( x, y \in [-2,2] \) und \( t \in [0,2] \).

 Das neuronale Netzwerk ist \ldots

 Die Datensätze sind \ldots

 Das Resultat ist \ldots

Die Lösung der Wellengleichung ist periodisch, und neuronale Netzwerke haben Schwierigkeiten beim Approximieren von periodischen Funktionen.
Dieses Problem kann durch die Verwendung von \emph{Fourier Features} gelöst werden \cite{neuronal:fourier_features}.
Die grundlegende Idee ist es, die Datenpunkte der Diskretisierung mit Sinus- und Cosinus-Funktionen zu transformieren, um periodische Strukturen besser erfassen zu können. Eine ausführlichere Erklärung hierzu ist im Kapitel über \emph{Fourier-Transformation und Feldtheorie} zu finden (siehe Kapitel~\ref{chapter:fourier}).

\subsection{Burgers-Gleichung}\label{neuronal:subsection:burgers_gleichung}
Die Burgers-Gleichung ist gegeben als
\begin{equation}
    \frac{\partial u}{\partial t} + u \frac{\partial u}{\partial x} = \nu \frac{\partial^2 u}{\partial x^2}.
    \label{neuronal:burgers}
\end{equation}
Der Diffusionskoeffizient \( \nu \) wird auf \( \nu = \frac{0.01}{\pi} \) festgelegt.
Die Anfangsbedingung
\begin{equation}
    u(0, x) = - \sin(\pi x)
    \label{neuronal:burgers_anfang}
\end{equation}
und die Randbedingung
\begin{equation}
    u(t, -1) = u(t, 1) = 0.
    \label{neuronal:burgers_rand}
\end{equation}
werden verwendet.
Die Bereiche sind \( x \in [-1,1] \) und \( t \in [0,1] \).

Das neuronale Netzwerk zur Lösung der Burgers-Gleichung ist folgendermassen aufgebaut:
\begin{itemize}
    \item 10 Teilfunktionen
    \item \( f_1 \): \( \mathbb{R}^2 \longrightarrow \mathbb{R}^{20} \) 
    \item \( f_{10} \): \( \mathbb{R}^{20} \longrightarrow \mathbb{R} \)
    \item Alle anderen Teilfunktionen: \( \mathbb{R}^{20} \longrightarrow \mathbb{R}^{20} \)
    \item Als Aktivierungsfunktion wird der hyperbolische Tangens verwendet
\end{itemize}
Das Netzwerk verfügt somit über 3441 Parameter.
In der letzten Teilfunktion \( f_{10} \) wird keine Aktivierungsfunktion verwendet.
Grund dafür ist das der hyperbolische Tangens den Wertebereich \((-1, 1)\) hat, das Netzwerk aber zur Approximation der Burgers-Gleichung den Wertebereich \( \mathbb{R} \) haben soll.

Wie im Abschnitt \ref{neuronal:subsection:diskretierung} beschrieben, werden insgesamt drei Datensätze verwendet.
Der Datensatz \( F \), in dem die Burgers-Gleichung gilt, besteht aus 5000 Datenpunkten.
Die Datensätze \( A \) und \( B \), in denen die Anfangsbedingungen bzw. die Randbedingungen gilten, bestehen jeweils aus 2000 Datenpunkten.
Je ein Fünftel der Datenpunkte wurde für die Funktion \( L^1(\vartheta) \) abgetrennt und nicht im Optimierungsalgorithmus verwendet (siehe Abschnitt \ref{neuronal:subsection:qualitätsbewertung}).

Der Optimierungsalgorithmus \ref{neuronal:gradient_descent} durchlief 15.000 Iterationen, um geeignete Parameter für die Approximation zu finden.
Die Werte von \( L(\vartheta) \) und \( L^1(\vartheta) \) am Ende der Optimierung sind 0.003328 bzw. 0.003449.
Somit sind die mittleren Approximationsfehler des Netzwerks sehr gering.
Der Verlauf des Approximationsfehlers während der Optimierung ist in Abbildung \ref{fig:fehler_burgers} dargestellt.
\begin{figure}
    \centering
    \hspace*{-0.1\textwidth}
    \includegraphics[width=0.7\textwidth]{papers/neuronal/images/approximation_error_burgers.png}
    \caption{Verlauf des Approximationsfehlers der Burgers-Gleichung}
    \label{fig:fehler_burgers}
\end{figure}

Wertet man das neuronale Netzwerk über die Bereiche von \( x \) und \( t \) aus, ergibt sich ein Plot der Lösung des neuronalen Netzwerks (siehe Abbildung \ref{fig:loesung_burgers}).
\begin{figure}
    \centering
    \includegraphics[width=0.8\textwidth]{papers/neuronal/images/prediction_burgers_net.png}
    \caption{Lösungs-Plot der Burgers-Gleichung}
    \label{fig:loesung_burgers}
\end{figure}

Der gesamte Code zur Umsetzung ist im GitHub-Repository des Seminars abgelegt \cite{neuronal:github_source_code}.


%
% 3_weiteres.tex -- Diskussion & weitere Anwendungsmöglichkeiten
%
% (c) 2025 Roman Cvijanovic & Nicola Dall'Acqua, Hochschule Rapperswil
%
% !TEX root = ../../buch.tex
% !TEX encoding = UTF-8
%

\section{Diskussion\label{neuronal:section:diskussion}}
\kopfrechts{Diskussion}

\begin{itemize}
    \item Auf weitere Anwendungsmöglichkeiten eingehen
    \item Diskussion der Methode
\end{itemize}




\printbibliography[heading=subbibliography]
\end{refsection}
