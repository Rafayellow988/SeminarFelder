%
% 3_weiteres.tex -- Diskussion & weitere Anwendungsmöglichkeiten
%
% (c) 2025 Roman Cvijanovic & Nicola Dall'Acqua, Hochschule Rapperswil
%
% !TEX root = ../../buch.tex
% !TEX encoding = UTF-8
%

\section{Diskussion\label{neuronal:section:diskussion}}
\kopfrechts{Diskussion}

Der grosser Vorteil dieser Methode ist ihre breite Anwendbarkeit.
Sie kann grundsätzlich auf beliebige Feldgleichungen angewendet werden, unabhängig davon, wie komplex diese sind.
Klassische Verfahren wie die Finite-Differenzen- oder Finite-Elemente-Methode sind oft auf bestimmte Gleichungstypen beschränkt und stoßen insbesondere bei stark nichtlinearen oder hochdimensionalen Problemen an ihre Grenzen.

Neuronale Netzwerke bieten durch das \emph{Universal Approximation Theorem} die theoretische Grundlage, beliebige stetige Funktionen mit beliebiger Genauigkeit zu approximieren \cite{neuronal:universal_approximation_theorem}. 
Damit sind sie theoretisch in der Lage, Lösungen für jede beliebige Feldgleichung zu approximieren, unabhängig von deren Komplexität.

In der Praxis zeigen neuronale Netzwerke vielversprechende Ergebnisse bei der Lösung von Feldgleichungen \cite{neuronal:pinns}. 
Dennoch steht die Methode noch am Anfang ihrer Entwicklung. 
Es gibt noch viele offene Fragen, etwa zur Effizienz gegenüber klassischen Verfahren.
Die bisher veröffentlichten Studien zeigen jedoch, dass diese Methode nicht unversprechend ist.
Für eine abschließende Bewertung der Methode ist es aber noch zu früh. 

Abgesehen von dieser generellen Einschätzung der Methode noch ein Punkt zur Wellengleichung.
Die Lösung der Wellengleichung ist periodisch und neuronale Netzwerke haben Schwierigkeiten beim Approximieren von periodischen Funktionen.
Dieses Problem lässt sich mit \emph{Fourier Features} lösen \cite{neuronal:fourier_features}.
Die Idee ist, die Datenpunkte der Diskretierung mit Sinus- und Cosinus-Funktionen zu transformieren, um periodische Strukturen besser erfassen zu können.
