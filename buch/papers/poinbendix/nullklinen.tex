\section{Verhalten von Bahnkurven in 2D und Nullklinen} \label{poinbendix:section:nullklinen}

%TODO Nullklinen erklären anhand Beispie
%TODO Verhalten von Bahnkurven anhand Nullklinen analysieren
\subsection{Nullklinen} \label{poinbendix:subsection:nullklinen}
TODO


\begin{figure}
    \centering
    \begin{tikzpicture}[>=stealth, scale=1.5]
  
  % Draw axes
  \draw[thick, black, ->] (-2,0) -- (2,0); % Horizontal axis
  \draw[thick, black, ->] (0,-2) -- (0,2); % Vertical axis
  
  % Axis labels
  \node[above] at (0,2) {$\dot{y} = 0$};
  \node[right] at (2,0) {$\dot{x} = 0$};
  
  % Define colors for each quadrant
  \colorlet{q1color}{red}
  \colorlet{q2color}{blue}
  \colorlet{q3color}{green}
  \colorlet{q4color}{orange}

  % Clockwise arrows in each quadrant, at 45° angles with unique colors
  % First quadrant: down-right (southeast)
  \draw[q1color, ->] (0.5,1.5) -- (1.5,0.5);
  \node[above, right, q1color] at (1,1) {$\dot{x} > 0, \dot{y} < 0$};
  
  % Second quadrant: down-left (southwest)
  \draw[q2color, ->] (1.5,-0.5) -- (0.5,-1.5);
  \node[below, right, q2color] at (1,-1) {$\dot{x} < 0, \dot{y} < 0$};
  
  % Third quadrant: up-left (northwest)
  \draw[q3color, ->] (-0.5,-1.5) -- (-1.5,-0.5);
  \node[below, left, q3color] at (-1,-1) {$\dot{x} < 0, \dot{y} > 0$};
  
  % Fourth quadrant: up-right (northeast)
  \draw[q4color, ->] (-1.5,0.5) -- (-0.5,1.5);
  \node[above, left, q4color] at (-1,1) {$\dot{x} > 0, \dot{y} > 0$};
  
\end{tikzpicture}

    \caption{Die Nullklinen unterteilen die Fläche in 4 Quadranten mit unterschiedlicher Flussrichtung.}
    \label{poinbendix:fig:nullklinen}
\end{figure}

\subsection{Limesmengen} \label{poinbendix:subsection:limesmengen}

Limesmengen beinhalten alle Punkte, welche von Bahnkurven nach unendlicher Zeit erreicht werden.
Dies können einzelne Punkte sein (Nullstellen), periodische oder nicht periodische Bahnen oder Kombinationen davon.
Dabei startet man bei einem beliebigen Punkt $p$ und schaut sich das Verhalten in positiver und negativer Zeit an.

\begin{definition}[Limesmengen]
Die Alpha- und Omega-Limesmengen
\label{poinbendix:def:limesmengen}
\begin{align*}
    \alpha(p) &= \lim_{t\to-\infty} \Phi_t(p) \\
    \omega(p) &= \lim_{t\to\infty} \Phi_t(p)
\end{align*}
beschreiben das Verhalten eines dynamischen Systems $\Phi_t(p)$ nach unendlicher Zeit $t$.
\end{definition}

Vereinfacht gesagt, beschreibt $\alpha(p)$ von wo aus die Bahnkurve den Punkt $p$ erreicht und $\omega(p)$ wohin sich die Bahnkurve danach fortbewegt.
\footnote{Alpha und Omega sind der erste, respektive letzte Buchstabe im griechischen Alphabet, weshalb die Benennung der Limesmengen intuitiv Sinn ergibt.}

\begin{beispiel} \label{poinbendix:beispiel:1dlimesmengen}
Schauen wir uns die einfache, eindimensionale Differenzialgleichung
\begin{equation*}
    \dot{x} = \sin x,
\end{equation*}
mit einem Startpunkt von $p = \frac{\pi}{2}$ an.
Am Punkt $p$ haben wir eine positive Ableitung ($\dot{p} = 1$), und der Punkt wird eingeschlossen von zwei Nullstellen $0$ und $\pi$.
Daraus folgt, dass beide Limesmengen auf die Nullpunkte fallen $\alpha(p) = 0$ und $\omega(p) = \pi$.
\end{beispiel}
