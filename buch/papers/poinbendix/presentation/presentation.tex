\documentclass[ngerman, aspectratio=169]{beamer}

%style
\mode<presentation>{
	\usetheme{Frankfurt}
}
%packages
\usepackage[utf8]{inputenc}
\usepackage[english]{babel}
\usepackage{graphicx}
\usepackage{array}

\newcolumntype{L}[1]{>{\raggedright\let\newline\\\arraybackslash\hspace{0pt}}m{#1}}
\usepackage{ragged2e}

\usepackage{bm} % bold math
\usepackage{amsfonts}
\usepackage{amssymb}
\usepackage{mathtools}
\usepackage{amsmath}
\usepackage{multirow} % multi row in tables
\usepackage{scrextend}

\usepackage{tikz}
\usepackage{pgf}

\usepackage{algorithmic}

%\usepackage{algorithm} % http://ctan.org/pkg/algorithm
%\usepackage{algpseudocode} % http://ctan.org/pkg/algorithmicx

%\usepackage{algorithmicx}


%citations
\usepackage[style=verbose,backend=biber]{biblatex}
\addbibresource{references.bib}



\usefonttheme[onlymath]{serif}

%Beamer Template modifications
%\definecolor{mainColor}{HTML}{0065A3} % HSR blue
\definecolor{mainColor}{HTML}{D72864} % OST pink
\definecolor{invColor}{HTML}{28d79b} % OST pink
\definecolor{dgreen}{HTML}{38ad36} % Dark green

%\definecolor{mainColor}{HTML}{000000} % HSR blue
\setbeamercolor{palette primary}{bg=white,fg=mainColor}
\setbeamercolor{palette secondary}{bg=orange,fg=mainColor}
\setbeamercolor{palette tertiary}{bg=yellow,fg=red}
\setbeamercolor{palette quaternary}{bg=mainColor,fg=white} %bg = Top bar, fg = active top bar topic
\setbeamercolor{structure}{fg=black} % itemize, enumerate, etc (bullet points)
\setbeamercolor{section in toc}{fg=black} % TOC sections
\setbeamertemplate{section in toc}[sections numbered]
\setbeamertemplate{subsection in toc}{%
	\hspace{1.2em}{$\bullet$}~\inserttocsubsection\par}

\setbeamertemplate{itemize items}[circle]
\setbeamertemplate{description item}[circle]
\setbeamertemplate{title page}[default][colsep=-4bp,rounded=true]
\beamertemplatenavigationsymbolsempty

\setbeamercolor{footline}{fg=gray}
\setbeamertemplate{footline}{%
	\hfill\usebeamertemplate***{navigation symbols}
	\hspace{0.5cm}
	\insertframenumber{}\hspace{0.2cm}\vspace{0.2cm}
}

\usepackage{caption}
\captionsetup{labelformat=empty}

%Title Page
\title{Der Satz von Poincaré-Bendixson}
\author{Raphael Unterer}
\institute{Mathematisches Seminar 2025: Felder}

\newcommand*{\HL}{\textcolor{mainColor}}
\newcommand*{\RD}{\textcolor{red}}
\newcommand*{\BL}{\textcolor{blue}}
\newcommand*{\GN}{\textcolor{dgreen}}
\newcommand*{\YE}{\textcolor{violet}}




\makeatletter
\newcount\my@repeat@count
\newcommand{\myrepeat}[2]{%
	\begingroup
	\my@repeat@count=\z@
	\@whilenum\my@repeat@count<#1\do{#2\advance\my@repeat@count\@ne}%
	\endgroup
}
\makeatother




\usetikzlibrary{automata,arrows,positioning,calc}


\begin{document}

	%Titelseite
	\begin{frame}
		\titlepage
	\end{frame}

	\section{Motivation}
    \begin{frame}
        \frametitle{El Niño Southern Oscillation (ENSO)}
        \begin{center}
            \includegraphics[width=0.7\textwidth]{../images/iconic_ENSO_elNino_lrg.jpg}
        \end{center}
    \end{frame}

    \begin{frame}
        \frametitle{El Niño Southern Oscillation (ENSO)}
        \begin{center}
            \includegraphics[width=0.7\textwidth]{../images/iconic_ENSO_laNina_lrg.jpg}
        \end{center}
    \end{frame}

    \begin{frame}
        \frametitle{El Niño Southern Oscillation (ENSO)}
        \begin{center}
            \includegraphics[width=\textwidth]{../images/elnino_data.png}
        \end{center}
    \end{frame}

    \begin{frame}
        \frametitle{Modellierung als dynamisches System}
        Recharge Oscillator Modell:
        \begin{align*}
            \frac{dT_E}{dt} &= -cT_E + \gamma \left(bT_E + h_W\right) - \epsilon \left(bT_E + h_W\right)^3, \\
            \frac{dh_W}{dt} &= -rh_W - \alpha b T_E.
        \end{align*}
        \pause
        Vereinfachen:
        \begin{align*}
            \dot{x} &= -x + \gamma \left(bx + y\right) - \epsilon \left(bx + y\right)^3, \\
            \dot{y} &= -ry - \alpha b x.
        \end{align*}
    \end{frame}

    \begin{frame}
    \frametitle{Gewünschte Eigenschaften}
        Das Recharge Oscillator ENSO Modell soll folgende Eigenschaften haben:
        \begin{itemize}
            \item Oszillierend
            \item Kleine Änderungen der Parameter ändert nicht viel an der Lösung
            \item Anfangsbedingungen spielen keine grosse Rolle
        \end{itemize}
    \end{frame}

	\section{Poincaré-Bendixson}

    \begin{frame}
    \frametitle{Nullklinen}
        Wir betrachten das System

        \begin{align*}
            \dot{x} &= y - x^2 \\
            \dot{y} &= x - 2
        \end{align*}
        \pause
        $x$-Nullkline:
        \begin{equation*}
            y = x^2
        \end{equation*}
        \pause
        $y$-Nullkline:
        \begin{equation*}
            x = 2
        \end{equation*}
    \end{frame}
    \begin{frame}
    \frametitle{Nullklinen: Unterteilung in Sektoren}
        \begin{center}
            \includegraphics[width=0.7\textwidth]{../images/nullklinen.pdf}
        \end{center}
    \end{frame}

    \begin{frame}
    \frametitle{Limesmengen}
        Dynamisches System $\Phi_t(p)$ hat die Alpha- und Omega-Limesmenge:

        \begin{align*}
            \alpha(p) &= \lim_{t\to-\infty} \Phi_t(p) \\
            \omega(p) &= \lim_{t\to\infty} \Phi_t(p)
        \end{align*}
    \end{frame}
    \begin{frame}
    \frametitle{Poincaré-Bendixson: Fall 1}
        \begin{enumerate}
            \item $\omega(p)$ ist eine Singularität
        \end{enumerate}
        \begin{center}
            %% Creator: Matplotlib, PGF backend
%%
%% To include the figure in your LaTeX document, write
%%   \input{<filename>.pgf}
%%
%% Make sure the required packages are loaded in your preamble
%%   \usepackage{pgf}
%%
%% Also ensure that all the required font packages are loaded; for instance,
%% the lmodern package is sometimes necessary when using math font.
%%   \usepackage{lmodern}
%%
%% Figures using additional raster images can only be included by \input if
%% they are in the same directory as the main LaTeX file. For loading figures
%% from other directories you can use the `import` package
%%   \usepackage{import}
%%
%% and then include the figures with
%%   \import{<path to file>}{<filename>.pgf}
%%
%% Matplotlib used the following preamble
%%   \usepackage{bm}
%%   \usepackage{amsmath}
%%   \usepackage{xcolor}
%%   \usepackage{tgtermes}
%%   \makeatletter\@ifpackageloaded{underscore}{}{\usepackage[strings]{underscore}}\makeatother
%%
\begingroup%
\makeatletter%
\begin{pgfpicture}%
\pgfpathrectangle{\pgfpointorigin}{\pgfqpoint{4.500000in}{2.500000in}}%
\pgfusepath{use as bounding box, clip}%
\begin{pgfscope}%
\pgfsetbuttcap%
\pgfsetmiterjoin%
\definecolor{currentfill}{rgb}{1.000000,1.000000,1.000000}%
\pgfsetfillcolor{currentfill}%
\pgfsetlinewidth{0.000000pt}%
\definecolor{currentstroke}{rgb}{1.000000,1.000000,1.000000}%
\pgfsetstrokecolor{currentstroke}%
\pgfsetdash{}{0pt}%
\pgfpathmoveto{\pgfqpoint{0.000000in}{0.000000in}}%
\pgfpathlineto{\pgfqpoint{4.500000in}{0.000000in}}%
\pgfpathlineto{\pgfqpoint{4.500000in}{2.500000in}}%
\pgfpathlineto{\pgfqpoint{0.000000in}{2.500000in}}%
\pgfpathlineto{\pgfqpoint{0.000000in}{0.000000in}}%
\pgfpathclose%
\pgfusepath{fill}%
\end{pgfscope}%
\begin{pgfscope}%
\pgfsetbuttcap%
\pgfsetmiterjoin%
\definecolor{currentfill}{rgb}{1.000000,1.000000,1.000000}%
\pgfsetfillcolor{currentfill}%
\pgfsetlinewidth{0.000000pt}%
\definecolor{currentstroke}{rgb}{0.000000,0.000000,0.000000}%
\pgfsetstrokecolor{currentstroke}%
\pgfsetstrokeopacity{0.000000}%
\pgfsetdash{}{0pt}%
\pgfpathmoveto{\pgfqpoint{0.562500in}{0.275000in}}%
\pgfpathlineto{\pgfqpoint{4.050000in}{0.275000in}}%
\pgfpathlineto{\pgfqpoint{4.050000in}{2.200000in}}%
\pgfpathlineto{\pgfqpoint{0.562500in}{2.200000in}}%
\pgfpathlineto{\pgfqpoint{0.562500in}{0.275000in}}%
\pgfpathclose%
\pgfusepath{fill}%
\end{pgfscope}%
\begin{pgfscope}%
\pgfpathrectangle{\pgfqpoint{0.562500in}{0.275000in}}{\pgfqpoint{3.487500in}{1.925000in}}%
\pgfusepath{clip}%
\pgfsetbuttcap%
\pgfsetroundjoin%
\pgfsetlinewidth{0.250937pt}%
\definecolor{currentstroke}{rgb}{0.501961,0.501961,0.501961}%
\pgfsetstrokecolor{currentstroke}%
\pgfsetdash{{0.250000pt}{0.412500pt}}{0.000000pt}%
\pgfpathmoveto{\pgfqpoint{0.800960in}{0.275000in}}%
\pgfpathlineto{\pgfqpoint{0.800960in}{2.200000in}}%
\pgfusepath{stroke}%
\end{pgfscope}%
\begin{pgfscope}%
\pgfsetbuttcap%
\pgfsetroundjoin%
\definecolor{currentfill}{rgb}{0.000000,0.000000,0.000000}%
\pgfsetfillcolor{currentfill}%
\pgfsetlinewidth{0.501875pt}%
\definecolor{currentstroke}{rgb}{0.000000,0.000000,0.000000}%
\pgfsetstrokecolor{currentstroke}%
\pgfsetdash{}{0pt}%
\pgfsys@defobject{currentmarker}{\pgfqpoint{0.000000in}{-0.041667in}}{\pgfqpoint{0.000000in}{0.000000in}}{%
\pgfpathmoveto{\pgfqpoint{0.000000in}{0.000000in}}%
\pgfpathlineto{\pgfqpoint{0.000000in}{-0.041667in}}%
\pgfusepath{stroke,fill}%
}%
\begin{pgfscope}%
\pgfsys@transformshift{0.800960in}{0.275000in}%
\pgfsys@useobject{currentmarker}{}%
\end{pgfscope}%
\end{pgfscope}%
\begin{pgfscope}%
\definecolor{textcolor}{rgb}{0.000000,0.000000,0.000000}%
\pgfsetstrokecolor{textcolor}%
\pgfsetfillcolor{textcolor}%
\pgftext[x=0.800960in,y=0.184722in,,top]{\color{textcolor}\rmfamily\fontsize{10.000000}{12.000000}\selectfont \(\displaystyle {-1.0}\)}%
\end{pgfscope}%
\begin{pgfscope}%
\pgfpathrectangle{\pgfqpoint{0.562500in}{0.275000in}}{\pgfqpoint{3.487500in}{1.925000in}}%
\pgfusepath{clip}%
\pgfsetbuttcap%
\pgfsetroundjoin%
\pgfsetlinewidth{0.250937pt}%
\definecolor{currentstroke}{rgb}{0.501961,0.501961,0.501961}%
\pgfsetstrokecolor{currentstroke}%
\pgfsetdash{{0.250000pt}{0.412500pt}}{0.000000pt}%
\pgfpathmoveto{\pgfqpoint{1.556419in}{0.275000in}}%
\pgfpathlineto{\pgfqpoint{1.556419in}{2.200000in}}%
\pgfusepath{stroke}%
\end{pgfscope}%
\begin{pgfscope}%
\pgfsetbuttcap%
\pgfsetroundjoin%
\definecolor{currentfill}{rgb}{0.000000,0.000000,0.000000}%
\pgfsetfillcolor{currentfill}%
\pgfsetlinewidth{0.501875pt}%
\definecolor{currentstroke}{rgb}{0.000000,0.000000,0.000000}%
\pgfsetstrokecolor{currentstroke}%
\pgfsetdash{}{0pt}%
\pgfsys@defobject{currentmarker}{\pgfqpoint{0.000000in}{-0.041667in}}{\pgfqpoint{0.000000in}{0.000000in}}{%
\pgfpathmoveto{\pgfqpoint{0.000000in}{0.000000in}}%
\pgfpathlineto{\pgfqpoint{0.000000in}{-0.041667in}}%
\pgfusepath{stroke,fill}%
}%
\begin{pgfscope}%
\pgfsys@transformshift{1.556419in}{0.275000in}%
\pgfsys@useobject{currentmarker}{}%
\end{pgfscope}%
\end{pgfscope}%
\begin{pgfscope}%
\definecolor{textcolor}{rgb}{0.000000,0.000000,0.000000}%
\pgfsetstrokecolor{textcolor}%
\pgfsetfillcolor{textcolor}%
\pgftext[x=1.556419in,y=0.184722in,,top]{\color{textcolor}\rmfamily\fontsize{10.000000}{12.000000}\selectfont \(\displaystyle {-0.5}\)}%
\end{pgfscope}%
\begin{pgfscope}%
\pgfpathrectangle{\pgfqpoint{0.562500in}{0.275000in}}{\pgfqpoint{3.487500in}{1.925000in}}%
\pgfusepath{clip}%
\pgfsetbuttcap%
\pgfsetroundjoin%
\pgfsetlinewidth{0.250937pt}%
\definecolor{currentstroke}{rgb}{0.501961,0.501961,0.501961}%
\pgfsetstrokecolor{currentstroke}%
\pgfsetdash{{0.250000pt}{0.412500pt}}{0.000000pt}%
\pgfpathmoveto{\pgfqpoint{2.311879in}{0.275000in}}%
\pgfpathlineto{\pgfqpoint{2.311879in}{2.200000in}}%
\pgfusepath{stroke}%
\end{pgfscope}%
\begin{pgfscope}%
\pgfsetbuttcap%
\pgfsetroundjoin%
\definecolor{currentfill}{rgb}{0.000000,0.000000,0.000000}%
\pgfsetfillcolor{currentfill}%
\pgfsetlinewidth{0.501875pt}%
\definecolor{currentstroke}{rgb}{0.000000,0.000000,0.000000}%
\pgfsetstrokecolor{currentstroke}%
\pgfsetdash{}{0pt}%
\pgfsys@defobject{currentmarker}{\pgfqpoint{0.000000in}{-0.041667in}}{\pgfqpoint{0.000000in}{0.000000in}}{%
\pgfpathmoveto{\pgfqpoint{0.000000in}{0.000000in}}%
\pgfpathlineto{\pgfqpoint{0.000000in}{-0.041667in}}%
\pgfusepath{stroke,fill}%
}%
\begin{pgfscope}%
\pgfsys@transformshift{2.311879in}{0.275000in}%
\pgfsys@useobject{currentmarker}{}%
\end{pgfscope}%
\end{pgfscope}%
\begin{pgfscope}%
\definecolor{textcolor}{rgb}{0.000000,0.000000,0.000000}%
\pgfsetstrokecolor{textcolor}%
\pgfsetfillcolor{textcolor}%
\pgftext[x=2.311879in,y=0.184722in,,top]{\color{textcolor}\rmfamily\fontsize{10.000000}{12.000000}\selectfont \(\displaystyle {0.0}\)}%
\end{pgfscope}%
\begin{pgfscope}%
\pgfpathrectangle{\pgfqpoint{0.562500in}{0.275000in}}{\pgfqpoint{3.487500in}{1.925000in}}%
\pgfusepath{clip}%
\pgfsetbuttcap%
\pgfsetroundjoin%
\pgfsetlinewidth{0.250937pt}%
\definecolor{currentstroke}{rgb}{0.501961,0.501961,0.501961}%
\pgfsetstrokecolor{currentstroke}%
\pgfsetdash{{0.250000pt}{0.412500pt}}{0.000000pt}%
\pgfpathmoveto{\pgfqpoint{3.067339in}{0.275000in}}%
\pgfpathlineto{\pgfqpoint{3.067339in}{2.200000in}}%
\pgfusepath{stroke}%
\end{pgfscope}%
\begin{pgfscope}%
\pgfsetbuttcap%
\pgfsetroundjoin%
\definecolor{currentfill}{rgb}{0.000000,0.000000,0.000000}%
\pgfsetfillcolor{currentfill}%
\pgfsetlinewidth{0.501875pt}%
\definecolor{currentstroke}{rgb}{0.000000,0.000000,0.000000}%
\pgfsetstrokecolor{currentstroke}%
\pgfsetdash{}{0pt}%
\pgfsys@defobject{currentmarker}{\pgfqpoint{0.000000in}{-0.041667in}}{\pgfqpoint{0.000000in}{0.000000in}}{%
\pgfpathmoveto{\pgfqpoint{0.000000in}{0.000000in}}%
\pgfpathlineto{\pgfqpoint{0.000000in}{-0.041667in}}%
\pgfusepath{stroke,fill}%
}%
\begin{pgfscope}%
\pgfsys@transformshift{3.067339in}{0.275000in}%
\pgfsys@useobject{currentmarker}{}%
\end{pgfscope}%
\end{pgfscope}%
\begin{pgfscope}%
\definecolor{textcolor}{rgb}{0.000000,0.000000,0.000000}%
\pgfsetstrokecolor{textcolor}%
\pgfsetfillcolor{textcolor}%
\pgftext[x=3.067339in,y=0.184722in,,top]{\color{textcolor}\rmfamily\fontsize{10.000000}{12.000000}\selectfont \(\displaystyle {0.5}\)}%
\end{pgfscope}%
\begin{pgfscope}%
\pgfpathrectangle{\pgfqpoint{0.562500in}{0.275000in}}{\pgfqpoint{3.487500in}{1.925000in}}%
\pgfusepath{clip}%
\pgfsetbuttcap%
\pgfsetroundjoin%
\pgfsetlinewidth{0.250937pt}%
\definecolor{currentstroke}{rgb}{0.501961,0.501961,0.501961}%
\pgfsetstrokecolor{currentstroke}%
\pgfsetdash{{0.250000pt}{0.412500pt}}{0.000000pt}%
\pgfpathmoveto{\pgfqpoint{3.822799in}{0.275000in}}%
\pgfpathlineto{\pgfqpoint{3.822799in}{2.200000in}}%
\pgfusepath{stroke}%
\end{pgfscope}%
\begin{pgfscope}%
\pgfsetbuttcap%
\pgfsetroundjoin%
\definecolor{currentfill}{rgb}{0.000000,0.000000,0.000000}%
\pgfsetfillcolor{currentfill}%
\pgfsetlinewidth{0.501875pt}%
\definecolor{currentstroke}{rgb}{0.000000,0.000000,0.000000}%
\pgfsetstrokecolor{currentstroke}%
\pgfsetdash{}{0pt}%
\pgfsys@defobject{currentmarker}{\pgfqpoint{0.000000in}{-0.041667in}}{\pgfqpoint{0.000000in}{0.000000in}}{%
\pgfpathmoveto{\pgfqpoint{0.000000in}{0.000000in}}%
\pgfpathlineto{\pgfqpoint{0.000000in}{-0.041667in}}%
\pgfusepath{stroke,fill}%
}%
\begin{pgfscope}%
\pgfsys@transformshift{3.822799in}{0.275000in}%
\pgfsys@useobject{currentmarker}{}%
\end{pgfscope}%
\end{pgfscope}%
\begin{pgfscope}%
\definecolor{textcolor}{rgb}{0.000000,0.000000,0.000000}%
\pgfsetstrokecolor{textcolor}%
\pgfsetfillcolor{textcolor}%
\pgftext[x=3.822799in,y=0.184722in,,top]{\color{textcolor}\rmfamily\fontsize{10.000000}{12.000000}\selectfont \(\displaystyle {1.0}\)}%
\end{pgfscope}%
\begin{pgfscope}%
\pgfpathrectangle{\pgfqpoint{0.562500in}{0.275000in}}{\pgfqpoint{3.487500in}{1.925000in}}%
\pgfusepath{clip}%
\pgfsetbuttcap%
\pgfsetroundjoin%
\pgfsetlinewidth{0.250937pt}%
\definecolor{currentstroke}{rgb}{0.501961,0.501961,0.501961}%
\pgfsetstrokecolor{currentstroke}%
\pgfsetdash{{0.250000pt}{0.412500pt}}{0.000000pt}%
\pgfpathmoveto{\pgfqpoint{0.562500in}{0.388145in}}%
\pgfpathlineto{\pgfqpoint{4.050000in}{0.388145in}}%
\pgfusepath{stroke}%
\end{pgfscope}%
\begin{pgfscope}%
\pgfsetbuttcap%
\pgfsetroundjoin%
\definecolor{currentfill}{rgb}{0.000000,0.000000,0.000000}%
\pgfsetfillcolor{currentfill}%
\pgfsetlinewidth{0.501875pt}%
\definecolor{currentstroke}{rgb}{0.000000,0.000000,0.000000}%
\pgfsetstrokecolor{currentstroke}%
\pgfsetdash{}{0pt}%
\pgfsys@defobject{currentmarker}{\pgfqpoint{-0.041667in}{0.000000in}}{\pgfqpoint{-0.000000in}{0.000000in}}{%
\pgfpathmoveto{\pgfqpoint{-0.000000in}{0.000000in}}%
\pgfpathlineto{\pgfqpoint{-0.041667in}{0.000000in}}%
\pgfusepath{stroke,fill}%
}%
\begin{pgfscope}%
\pgfsys@transformshift{0.562500in}{0.388145in}%
\pgfsys@useobject{currentmarker}{}%
\end{pgfscope}%
\end{pgfscope}%
\begin{pgfscope}%
\definecolor{textcolor}{rgb}{0.000000,0.000000,0.000000}%
\pgfsetstrokecolor{textcolor}%
\pgfsetfillcolor{textcolor}%
\pgftext[x=0.294752in, y=0.341444in, left, base]{\color{textcolor}\rmfamily\fontsize{10.000000}{12.000000}\selectfont \(\displaystyle {-1}\)}%
\end{pgfscope}%
\begin{pgfscope}%
\pgfpathrectangle{\pgfqpoint{0.562500in}{0.275000in}}{\pgfqpoint{3.487500in}{1.925000in}}%
\pgfusepath{clip}%
\pgfsetbuttcap%
\pgfsetroundjoin%
\pgfsetlinewidth{0.250937pt}%
\definecolor{currentstroke}{rgb}{0.501961,0.501961,0.501961}%
\pgfsetstrokecolor{currentstroke}%
\pgfsetdash{{0.250000pt}{0.412500pt}}{0.000000pt}%
\pgfpathmoveto{\pgfqpoint{0.562500in}{0.952344in}}%
\pgfpathlineto{\pgfqpoint{4.050000in}{0.952344in}}%
\pgfusepath{stroke}%
\end{pgfscope}%
\begin{pgfscope}%
\pgfsetbuttcap%
\pgfsetroundjoin%
\definecolor{currentfill}{rgb}{0.000000,0.000000,0.000000}%
\pgfsetfillcolor{currentfill}%
\pgfsetlinewidth{0.501875pt}%
\definecolor{currentstroke}{rgb}{0.000000,0.000000,0.000000}%
\pgfsetstrokecolor{currentstroke}%
\pgfsetdash{}{0pt}%
\pgfsys@defobject{currentmarker}{\pgfqpoint{-0.041667in}{0.000000in}}{\pgfqpoint{-0.000000in}{0.000000in}}{%
\pgfpathmoveto{\pgfqpoint{-0.000000in}{0.000000in}}%
\pgfpathlineto{\pgfqpoint{-0.041667in}{0.000000in}}%
\pgfusepath{stroke,fill}%
}%
\begin{pgfscope}%
\pgfsys@transformshift{0.562500in}{0.952344in}%
\pgfsys@useobject{currentmarker}{}%
\end{pgfscope}%
\end{pgfscope}%
\begin{pgfscope}%
\definecolor{textcolor}{rgb}{0.000000,0.000000,0.000000}%
\pgfsetstrokecolor{textcolor}%
\pgfsetfillcolor{textcolor}%
\pgftext[x=0.402777in, y=0.905642in, left, base]{\color{textcolor}\rmfamily\fontsize{10.000000}{12.000000}\selectfont \(\displaystyle {0}\)}%
\end{pgfscope}%
\begin{pgfscope}%
\pgfpathrectangle{\pgfqpoint{0.562500in}{0.275000in}}{\pgfqpoint{3.487500in}{1.925000in}}%
\pgfusepath{clip}%
\pgfsetbuttcap%
\pgfsetroundjoin%
\pgfsetlinewidth{0.250937pt}%
\definecolor{currentstroke}{rgb}{0.501961,0.501961,0.501961}%
\pgfsetstrokecolor{currentstroke}%
\pgfsetdash{{0.250000pt}{0.412500pt}}{0.000000pt}%
\pgfpathmoveto{\pgfqpoint{0.562500in}{1.516542in}}%
\pgfpathlineto{\pgfqpoint{4.050000in}{1.516542in}}%
\pgfusepath{stroke}%
\end{pgfscope}%
\begin{pgfscope}%
\pgfsetbuttcap%
\pgfsetroundjoin%
\definecolor{currentfill}{rgb}{0.000000,0.000000,0.000000}%
\pgfsetfillcolor{currentfill}%
\pgfsetlinewidth{0.501875pt}%
\definecolor{currentstroke}{rgb}{0.000000,0.000000,0.000000}%
\pgfsetstrokecolor{currentstroke}%
\pgfsetdash{}{0pt}%
\pgfsys@defobject{currentmarker}{\pgfqpoint{-0.041667in}{0.000000in}}{\pgfqpoint{-0.000000in}{0.000000in}}{%
\pgfpathmoveto{\pgfqpoint{-0.000000in}{0.000000in}}%
\pgfpathlineto{\pgfqpoint{-0.041667in}{0.000000in}}%
\pgfusepath{stroke,fill}%
}%
\begin{pgfscope}%
\pgfsys@transformshift{0.562500in}{1.516542in}%
\pgfsys@useobject{currentmarker}{}%
\end{pgfscope}%
\end{pgfscope}%
\begin{pgfscope}%
\definecolor{textcolor}{rgb}{0.000000,0.000000,0.000000}%
\pgfsetstrokecolor{textcolor}%
\pgfsetfillcolor{textcolor}%
\pgftext[x=0.402777in, y=1.469841in, left, base]{\color{textcolor}\rmfamily\fontsize{10.000000}{12.000000}\selectfont \(\displaystyle {1}\)}%
\end{pgfscope}%
\begin{pgfscope}%
\pgfpathrectangle{\pgfqpoint{0.562500in}{0.275000in}}{\pgfqpoint{3.487500in}{1.925000in}}%
\pgfusepath{clip}%
\pgfsetbuttcap%
\pgfsetroundjoin%
\pgfsetlinewidth{0.250937pt}%
\definecolor{currentstroke}{rgb}{0.501961,0.501961,0.501961}%
\pgfsetstrokecolor{currentstroke}%
\pgfsetdash{{0.250000pt}{0.412500pt}}{0.000000pt}%
\pgfpathmoveto{\pgfqpoint{0.562500in}{2.080740in}}%
\pgfpathlineto{\pgfqpoint{4.050000in}{2.080740in}}%
\pgfusepath{stroke}%
\end{pgfscope}%
\begin{pgfscope}%
\pgfsetbuttcap%
\pgfsetroundjoin%
\definecolor{currentfill}{rgb}{0.000000,0.000000,0.000000}%
\pgfsetfillcolor{currentfill}%
\pgfsetlinewidth{0.501875pt}%
\definecolor{currentstroke}{rgb}{0.000000,0.000000,0.000000}%
\pgfsetstrokecolor{currentstroke}%
\pgfsetdash{}{0pt}%
\pgfsys@defobject{currentmarker}{\pgfqpoint{-0.041667in}{0.000000in}}{\pgfqpoint{-0.000000in}{0.000000in}}{%
\pgfpathmoveto{\pgfqpoint{-0.000000in}{0.000000in}}%
\pgfpathlineto{\pgfqpoint{-0.041667in}{0.000000in}}%
\pgfusepath{stroke,fill}%
}%
\begin{pgfscope}%
\pgfsys@transformshift{0.562500in}{2.080740in}%
\pgfsys@useobject{currentmarker}{}%
\end{pgfscope}%
\end{pgfscope}%
\begin{pgfscope}%
\definecolor{textcolor}{rgb}{0.000000,0.000000,0.000000}%
\pgfsetstrokecolor{textcolor}%
\pgfsetfillcolor{textcolor}%
\pgftext[x=0.402777in, y=2.034039in, left, base]{\color{textcolor}\rmfamily\fontsize{10.000000}{12.000000}\selectfont \(\displaystyle {2}\)}%
\end{pgfscope}%
\begin{pgfscope}%
\pgfpathrectangle{\pgfqpoint{0.562500in}{0.275000in}}{\pgfqpoint{3.487500in}{1.925000in}}%
\pgfusepath{clip}%
\pgfsetbuttcap%
\pgfsetroundjoin%
\definecolor{currentfill}{rgb}{0.121569,0.466667,0.705882}%
\pgfsetfillcolor{currentfill}%
\pgfsetlinewidth{1.003750pt}%
\definecolor{currentstroke}{rgb}{0.121569,0.466667,0.705882}%
\pgfsetstrokecolor{currentstroke}%
\pgfsetdash{}{0pt}%
\pgfsys@defobject{currentmarker}{\pgfqpoint{-0.020833in}{-0.020833in}}{\pgfqpoint{0.020833in}{0.020833in}}{%
\pgfpathmoveto{\pgfqpoint{0.000000in}{-0.020833in}}%
\pgfpathcurveto{\pgfqpoint{0.005525in}{-0.020833in}}{\pgfqpoint{0.010825in}{-0.018638in}}{\pgfqpoint{0.014731in}{-0.014731in}}%
\pgfpathcurveto{\pgfqpoint{0.018638in}{-0.010825in}}{\pgfqpoint{0.020833in}{-0.005525in}}{\pgfqpoint{0.020833in}{0.000000in}}%
\pgfpathcurveto{\pgfqpoint{0.020833in}{0.005525in}}{\pgfqpoint{0.018638in}{0.010825in}}{\pgfqpoint{0.014731in}{0.014731in}}%
\pgfpathcurveto{\pgfqpoint{0.010825in}{0.018638in}}{\pgfqpoint{0.005525in}{0.020833in}}{\pgfqpoint{0.000000in}{0.020833in}}%
\pgfpathcurveto{\pgfqpoint{-0.005525in}{0.020833in}}{\pgfqpoint{-0.010825in}{0.018638in}}{\pgfqpoint{-0.014731in}{0.014731in}}%
\pgfpathcurveto{\pgfqpoint{-0.018638in}{0.010825in}}{\pgfqpoint{-0.020833in}{0.005525in}}{\pgfqpoint{-0.020833in}{0.000000in}}%
\pgfpathcurveto{\pgfqpoint{-0.020833in}{-0.005525in}}{\pgfqpoint{-0.018638in}{-0.010825in}}{\pgfqpoint{-0.014731in}{-0.014731in}}%
\pgfpathcurveto{\pgfqpoint{-0.010825in}{-0.018638in}}{\pgfqpoint{-0.005525in}{-0.020833in}}{\pgfqpoint{0.000000in}{-0.020833in}}%
\pgfpathlineto{\pgfqpoint{0.000000in}{-0.020833in}}%
\pgfpathclose%
\pgfusepath{stroke,fill}%
}%
\begin{pgfscope}%
\pgfsys@transformshift{2.311879in}{0.952344in}%
\pgfsys@useobject{currentmarker}{}%
\end{pgfscope}%
\begin{pgfscope}%
\pgfsys@transformshift{2.311879in}{0.952344in}%
\pgfsys@useobject{currentmarker}{}%
\end{pgfscope}%
\begin{pgfscope}%
\pgfsys@transformshift{2.311879in}{0.952344in}%
\pgfsys@useobject{currentmarker}{}%
\end{pgfscope}%
\begin{pgfscope}%
\pgfsys@transformshift{2.311879in}{0.952344in}%
\pgfsys@useobject{currentmarker}{}%
\end{pgfscope}%
\begin{pgfscope}%
\pgfsys@transformshift{2.311879in}{0.952344in}%
\pgfsys@useobject{currentmarker}{}%
\end{pgfscope}%
\begin{pgfscope}%
\pgfsys@transformshift{2.311879in}{0.952344in}%
\pgfsys@useobject{currentmarker}{}%
\end{pgfscope}%
\begin{pgfscope}%
\pgfsys@transformshift{2.311879in}{0.952344in}%
\pgfsys@useobject{currentmarker}{}%
\end{pgfscope}%
\begin{pgfscope}%
\pgfsys@transformshift{2.311879in}{0.952344in}%
\pgfsys@useobject{currentmarker}{}%
\end{pgfscope}%
\begin{pgfscope}%
\pgfsys@transformshift{2.311879in}{0.952344in}%
\pgfsys@useobject{currentmarker}{}%
\end{pgfscope}%
\begin{pgfscope}%
\pgfsys@transformshift{2.311879in}{0.952344in}%
\pgfsys@useobject{currentmarker}{}%
\end{pgfscope}%
\begin{pgfscope}%
\pgfsys@transformshift{2.311879in}{0.952344in}%
\pgfsys@useobject{currentmarker}{}%
\end{pgfscope}%
\begin{pgfscope}%
\pgfsys@transformshift{2.311879in}{0.952344in}%
\pgfsys@useobject{currentmarker}{}%
\end{pgfscope}%
\begin{pgfscope}%
\pgfsys@transformshift{2.311879in}{0.952344in}%
\pgfsys@useobject{currentmarker}{}%
\end{pgfscope}%
\begin{pgfscope}%
\pgfsys@transformshift{2.311879in}{0.952344in}%
\pgfsys@useobject{currentmarker}{}%
\end{pgfscope}%
\begin{pgfscope}%
\pgfsys@transformshift{2.311879in}{0.952344in}%
\pgfsys@useobject{currentmarker}{}%
\end{pgfscope}%
\begin{pgfscope}%
\pgfsys@transformshift{2.311879in}{0.952344in}%
\pgfsys@useobject{currentmarker}{}%
\end{pgfscope}%
\begin{pgfscope}%
\pgfsys@transformshift{2.311879in}{0.952344in}%
\pgfsys@useobject{currentmarker}{}%
\end{pgfscope}%
\begin{pgfscope}%
\pgfsys@transformshift{2.311879in}{0.952344in}%
\pgfsys@useobject{currentmarker}{}%
\end{pgfscope}%
\begin{pgfscope}%
\pgfsys@transformshift{2.311879in}{0.952344in}%
\pgfsys@useobject{currentmarker}{}%
\end{pgfscope}%
\begin{pgfscope}%
\pgfsys@transformshift{2.311879in}{0.952344in}%
\pgfsys@useobject{currentmarker}{}%
\end{pgfscope}%
\begin{pgfscope}%
\pgfsys@transformshift{2.311879in}{0.952344in}%
\pgfsys@useobject{currentmarker}{}%
\end{pgfscope}%
\begin{pgfscope}%
\pgfsys@transformshift{2.311879in}{0.952344in}%
\pgfsys@useobject{currentmarker}{}%
\end{pgfscope}%
\begin{pgfscope}%
\pgfsys@transformshift{2.311879in}{0.952344in}%
\pgfsys@useobject{currentmarker}{}%
\end{pgfscope}%
\begin{pgfscope}%
\pgfsys@transformshift{2.311879in}{0.952344in}%
\pgfsys@useobject{currentmarker}{}%
\end{pgfscope}%
\begin{pgfscope}%
\pgfsys@transformshift{2.311879in}{0.952344in}%
\pgfsys@useobject{currentmarker}{}%
\end{pgfscope}%
\begin{pgfscope}%
\pgfsys@transformshift{2.311879in}{0.952344in}%
\pgfsys@useobject{currentmarker}{}%
\end{pgfscope}%
\begin{pgfscope}%
\pgfsys@transformshift{2.311879in}{0.952344in}%
\pgfsys@useobject{currentmarker}{}%
\end{pgfscope}%
\begin{pgfscope}%
\pgfsys@transformshift{2.311879in}{0.952344in}%
\pgfsys@useobject{currentmarker}{}%
\end{pgfscope}%
\begin{pgfscope}%
\pgfsys@transformshift{2.311879in}{0.952344in}%
\pgfsys@useobject{currentmarker}{}%
\end{pgfscope}%
\begin{pgfscope}%
\pgfsys@transformshift{2.311879in}{0.952344in}%
\pgfsys@useobject{currentmarker}{}%
\end{pgfscope}%
\begin{pgfscope}%
\pgfsys@transformshift{2.311879in}{0.952344in}%
\pgfsys@useobject{currentmarker}{}%
\end{pgfscope}%
\begin{pgfscope}%
\pgfsys@transformshift{2.311879in}{0.952344in}%
\pgfsys@useobject{currentmarker}{}%
\end{pgfscope}%
\begin{pgfscope}%
\pgfsys@transformshift{2.311879in}{0.952344in}%
\pgfsys@useobject{currentmarker}{}%
\end{pgfscope}%
\begin{pgfscope}%
\pgfsys@transformshift{2.311879in}{0.952344in}%
\pgfsys@useobject{currentmarker}{}%
\end{pgfscope}%
\begin{pgfscope}%
\pgfsys@transformshift{2.311879in}{0.952344in}%
\pgfsys@useobject{currentmarker}{}%
\end{pgfscope}%
\begin{pgfscope}%
\pgfsys@transformshift{2.311879in}{0.952344in}%
\pgfsys@useobject{currentmarker}{}%
\end{pgfscope}%
\begin{pgfscope}%
\pgfsys@transformshift{2.311879in}{0.952344in}%
\pgfsys@useobject{currentmarker}{}%
\end{pgfscope}%
\begin{pgfscope}%
\pgfsys@transformshift{2.311879in}{0.952344in}%
\pgfsys@useobject{currentmarker}{}%
\end{pgfscope}%
\begin{pgfscope}%
\pgfsys@transformshift{2.311879in}{0.952344in}%
\pgfsys@useobject{currentmarker}{}%
\end{pgfscope}%
\begin{pgfscope}%
\pgfsys@transformshift{2.311879in}{0.952344in}%
\pgfsys@useobject{currentmarker}{}%
\end{pgfscope}%
\begin{pgfscope}%
\pgfsys@transformshift{2.311879in}{0.952344in}%
\pgfsys@useobject{currentmarker}{}%
\end{pgfscope}%
\begin{pgfscope}%
\pgfsys@transformshift{2.311879in}{0.952344in}%
\pgfsys@useobject{currentmarker}{}%
\end{pgfscope}%
\begin{pgfscope}%
\pgfsys@transformshift{2.311879in}{0.952344in}%
\pgfsys@useobject{currentmarker}{}%
\end{pgfscope}%
\begin{pgfscope}%
\pgfsys@transformshift{2.311879in}{0.952344in}%
\pgfsys@useobject{currentmarker}{}%
\end{pgfscope}%
\begin{pgfscope}%
\pgfsys@transformshift{2.311879in}{0.952344in}%
\pgfsys@useobject{currentmarker}{}%
\end{pgfscope}%
\begin{pgfscope}%
\pgfsys@transformshift{2.311879in}{0.952344in}%
\pgfsys@useobject{currentmarker}{}%
\end{pgfscope}%
\begin{pgfscope}%
\pgfsys@transformshift{2.311879in}{0.952344in}%
\pgfsys@useobject{currentmarker}{}%
\end{pgfscope}%
\begin{pgfscope}%
\pgfsys@transformshift{2.311879in}{0.952344in}%
\pgfsys@useobject{currentmarker}{}%
\end{pgfscope}%
\begin{pgfscope}%
\pgfsys@transformshift{2.311879in}{0.952344in}%
\pgfsys@useobject{currentmarker}{}%
\end{pgfscope}%
\begin{pgfscope}%
\pgfsys@transformshift{2.311879in}{0.952344in}%
\pgfsys@useobject{currentmarker}{}%
\end{pgfscope}%
\begin{pgfscope}%
\pgfsys@transformshift{2.311879in}{0.952344in}%
\pgfsys@useobject{currentmarker}{}%
\end{pgfscope}%
\begin{pgfscope}%
\pgfsys@transformshift{2.311879in}{0.952344in}%
\pgfsys@useobject{currentmarker}{}%
\end{pgfscope}%
\begin{pgfscope}%
\pgfsys@transformshift{2.311879in}{0.952344in}%
\pgfsys@useobject{currentmarker}{}%
\end{pgfscope}%
\begin{pgfscope}%
\pgfsys@transformshift{2.311879in}{0.952344in}%
\pgfsys@useobject{currentmarker}{}%
\end{pgfscope}%
\begin{pgfscope}%
\pgfsys@transformshift{2.311879in}{0.952344in}%
\pgfsys@useobject{currentmarker}{}%
\end{pgfscope}%
\begin{pgfscope}%
\pgfsys@transformshift{2.311879in}{0.952344in}%
\pgfsys@useobject{currentmarker}{}%
\end{pgfscope}%
\begin{pgfscope}%
\pgfsys@transformshift{2.311879in}{0.952344in}%
\pgfsys@useobject{currentmarker}{}%
\end{pgfscope}%
\begin{pgfscope}%
\pgfsys@transformshift{2.311879in}{0.952344in}%
\pgfsys@useobject{currentmarker}{}%
\end{pgfscope}%
\begin{pgfscope}%
\pgfsys@transformshift{2.311879in}{0.952344in}%
\pgfsys@useobject{currentmarker}{}%
\end{pgfscope}%
\begin{pgfscope}%
\pgfsys@transformshift{2.311879in}{0.952344in}%
\pgfsys@useobject{currentmarker}{}%
\end{pgfscope}%
\begin{pgfscope}%
\pgfsys@transformshift{2.311879in}{0.952344in}%
\pgfsys@useobject{currentmarker}{}%
\end{pgfscope}%
\begin{pgfscope}%
\pgfsys@transformshift{2.311879in}{0.952344in}%
\pgfsys@useobject{currentmarker}{}%
\end{pgfscope}%
\begin{pgfscope}%
\pgfsys@transformshift{2.311879in}{0.952344in}%
\pgfsys@useobject{currentmarker}{}%
\end{pgfscope}%
\begin{pgfscope}%
\pgfsys@transformshift{2.311879in}{0.952344in}%
\pgfsys@useobject{currentmarker}{}%
\end{pgfscope}%
\begin{pgfscope}%
\pgfsys@transformshift{2.311879in}{0.952344in}%
\pgfsys@useobject{currentmarker}{}%
\end{pgfscope}%
\begin{pgfscope}%
\pgfsys@transformshift{2.311879in}{0.952344in}%
\pgfsys@useobject{currentmarker}{}%
\end{pgfscope}%
\begin{pgfscope}%
\pgfsys@transformshift{2.311879in}{0.952344in}%
\pgfsys@useobject{currentmarker}{}%
\end{pgfscope}%
\begin{pgfscope}%
\pgfsys@transformshift{2.311879in}{0.952344in}%
\pgfsys@useobject{currentmarker}{}%
\end{pgfscope}%
\begin{pgfscope}%
\pgfsys@transformshift{2.311879in}{0.952344in}%
\pgfsys@useobject{currentmarker}{}%
\end{pgfscope}%
\begin{pgfscope}%
\pgfsys@transformshift{2.311879in}{0.952344in}%
\pgfsys@useobject{currentmarker}{}%
\end{pgfscope}%
\begin{pgfscope}%
\pgfsys@transformshift{2.311879in}{0.952344in}%
\pgfsys@useobject{currentmarker}{}%
\end{pgfscope}%
\begin{pgfscope}%
\pgfsys@transformshift{2.311879in}{0.952344in}%
\pgfsys@useobject{currentmarker}{}%
\end{pgfscope}%
\begin{pgfscope}%
\pgfsys@transformshift{2.311879in}{0.952344in}%
\pgfsys@useobject{currentmarker}{}%
\end{pgfscope}%
\begin{pgfscope}%
\pgfsys@transformshift{2.311879in}{0.952344in}%
\pgfsys@useobject{currentmarker}{}%
\end{pgfscope}%
\begin{pgfscope}%
\pgfsys@transformshift{2.311879in}{0.952344in}%
\pgfsys@useobject{currentmarker}{}%
\end{pgfscope}%
\begin{pgfscope}%
\pgfsys@transformshift{2.311879in}{0.952344in}%
\pgfsys@useobject{currentmarker}{}%
\end{pgfscope}%
\begin{pgfscope}%
\pgfsys@transformshift{2.311879in}{0.952344in}%
\pgfsys@useobject{currentmarker}{}%
\end{pgfscope}%
\begin{pgfscope}%
\pgfsys@transformshift{2.311879in}{0.952344in}%
\pgfsys@useobject{currentmarker}{}%
\end{pgfscope}%
\begin{pgfscope}%
\pgfsys@transformshift{2.311879in}{0.952344in}%
\pgfsys@useobject{currentmarker}{}%
\end{pgfscope}%
\begin{pgfscope}%
\pgfsys@transformshift{2.311879in}{0.952344in}%
\pgfsys@useobject{currentmarker}{}%
\end{pgfscope}%
\begin{pgfscope}%
\pgfsys@transformshift{2.311879in}{0.952344in}%
\pgfsys@useobject{currentmarker}{}%
\end{pgfscope}%
\begin{pgfscope}%
\pgfsys@transformshift{2.311879in}{0.952344in}%
\pgfsys@useobject{currentmarker}{}%
\end{pgfscope}%
\begin{pgfscope}%
\pgfsys@transformshift{2.311879in}{0.952344in}%
\pgfsys@useobject{currentmarker}{}%
\end{pgfscope}%
\begin{pgfscope}%
\pgfsys@transformshift{2.311879in}{0.952344in}%
\pgfsys@useobject{currentmarker}{}%
\end{pgfscope}%
\begin{pgfscope}%
\pgfsys@transformshift{2.311879in}{0.952344in}%
\pgfsys@useobject{currentmarker}{}%
\end{pgfscope}%
\begin{pgfscope}%
\pgfsys@transformshift{2.311879in}{0.952344in}%
\pgfsys@useobject{currentmarker}{}%
\end{pgfscope}%
\begin{pgfscope}%
\pgfsys@transformshift{2.311879in}{0.952344in}%
\pgfsys@useobject{currentmarker}{}%
\end{pgfscope}%
\begin{pgfscope}%
\pgfsys@transformshift{2.311879in}{0.952344in}%
\pgfsys@useobject{currentmarker}{}%
\end{pgfscope}%
\begin{pgfscope}%
\pgfsys@transformshift{2.311879in}{0.952344in}%
\pgfsys@useobject{currentmarker}{}%
\end{pgfscope}%
\begin{pgfscope}%
\pgfsys@transformshift{2.311879in}{0.952344in}%
\pgfsys@useobject{currentmarker}{}%
\end{pgfscope}%
\begin{pgfscope}%
\pgfsys@transformshift{2.311879in}{0.952344in}%
\pgfsys@useobject{currentmarker}{}%
\end{pgfscope}%
\begin{pgfscope}%
\pgfsys@transformshift{2.311879in}{0.952344in}%
\pgfsys@useobject{currentmarker}{}%
\end{pgfscope}%
\begin{pgfscope}%
\pgfsys@transformshift{2.311879in}{0.952344in}%
\pgfsys@useobject{currentmarker}{}%
\end{pgfscope}%
\begin{pgfscope}%
\pgfsys@transformshift{2.311879in}{0.952344in}%
\pgfsys@useobject{currentmarker}{}%
\end{pgfscope}%
\begin{pgfscope}%
\pgfsys@transformshift{2.311879in}{0.952344in}%
\pgfsys@useobject{currentmarker}{}%
\end{pgfscope}%
\begin{pgfscope}%
\pgfsys@transformshift{2.311879in}{0.952344in}%
\pgfsys@useobject{currentmarker}{}%
\end{pgfscope}%
\begin{pgfscope}%
\pgfsys@transformshift{2.311879in}{0.952344in}%
\pgfsys@useobject{currentmarker}{}%
\end{pgfscope}%
\begin{pgfscope}%
\pgfsys@transformshift{2.311879in}{0.952344in}%
\pgfsys@useobject{currentmarker}{}%
\end{pgfscope}%
\begin{pgfscope}%
\pgfsys@transformshift{2.311879in}{0.952344in}%
\pgfsys@useobject{currentmarker}{}%
\end{pgfscope}%
\begin{pgfscope}%
\pgfsys@transformshift{2.311879in}{0.952344in}%
\pgfsys@useobject{currentmarker}{}%
\end{pgfscope}%
\begin{pgfscope}%
\pgfsys@transformshift{2.311879in}{0.952344in}%
\pgfsys@useobject{currentmarker}{}%
\end{pgfscope}%
\begin{pgfscope}%
\pgfsys@transformshift{2.311879in}{0.952344in}%
\pgfsys@useobject{currentmarker}{}%
\end{pgfscope}%
\begin{pgfscope}%
\pgfsys@transformshift{2.311879in}{0.952344in}%
\pgfsys@useobject{currentmarker}{}%
\end{pgfscope}%
\begin{pgfscope}%
\pgfsys@transformshift{2.311879in}{0.952344in}%
\pgfsys@useobject{currentmarker}{}%
\end{pgfscope}%
\begin{pgfscope}%
\pgfsys@transformshift{2.311879in}{0.952344in}%
\pgfsys@useobject{currentmarker}{}%
\end{pgfscope}%
\begin{pgfscope}%
\pgfsys@transformshift{2.311879in}{0.952344in}%
\pgfsys@useobject{currentmarker}{}%
\end{pgfscope}%
\begin{pgfscope}%
\pgfsys@transformshift{2.311879in}{0.952344in}%
\pgfsys@useobject{currentmarker}{}%
\end{pgfscope}%
\begin{pgfscope}%
\pgfsys@transformshift{2.311879in}{0.952344in}%
\pgfsys@useobject{currentmarker}{}%
\end{pgfscope}%
\begin{pgfscope}%
\pgfsys@transformshift{2.311879in}{0.952344in}%
\pgfsys@useobject{currentmarker}{}%
\end{pgfscope}%
\begin{pgfscope}%
\pgfsys@transformshift{2.311879in}{0.952344in}%
\pgfsys@useobject{currentmarker}{}%
\end{pgfscope}%
\begin{pgfscope}%
\pgfsys@transformshift{2.311879in}{0.952344in}%
\pgfsys@useobject{currentmarker}{}%
\end{pgfscope}%
\begin{pgfscope}%
\pgfsys@transformshift{2.311879in}{0.952344in}%
\pgfsys@useobject{currentmarker}{}%
\end{pgfscope}%
\begin{pgfscope}%
\pgfsys@transformshift{2.311879in}{0.952344in}%
\pgfsys@useobject{currentmarker}{}%
\end{pgfscope}%
\begin{pgfscope}%
\pgfsys@transformshift{2.311879in}{0.952344in}%
\pgfsys@useobject{currentmarker}{}%
\end{pgfscope}%
\begin{pgfscope}%
\pgfsys@transformshift{2.311879in}{0.952344in}%
\pgfsys@useobject{currentmarker}{}%
\end{pgfscope}%
\begin{pgfscope}%
\pgfsys@transformshift{2.311879in}{0.952344in}%
\pgfsys@useobject{currentmarker}{}%
\end{pgfscope}%
\begin{pgfscope}%
\pgfsys@transformshift{2.311879in}{0.952344in}%
\pgfsys@useobject{currentmarker}{}%
\end{pgfscope}%
\begin{pgfscope}%
\pgfsys@transformshift{2.311879in}{0.952344in}%
\pgfsys@useobject{currentmarker}{}%
\end{pgfscope}%
\begin{pgfscope}%
\pgfsys@transformshift{2.311879in}{0.952344in}%
\pgfsys@useobject{currentmarker}{}%
\end{pgfscope}%
\begin{pgfscope}%
\pgfsys@transformshift{2.311879in}{0.952344in}%
\pgfsys@useobject{currentmarker}{}%
\end{pgfscope}%
\begin{pgfscope}%
\pgfsys@transformshift{2.311879in}{0.952344in}%
\pgfsys@useobject{currentmarker}{}%
\end{pgfscope}%
\begin{pgfscope}%
\pgfsys@transformshift{2.311879in}{0.952344in}%
\pgfsys@useobject{currentmarker}{}%
\end{pgfscope}%
\begin{pgfscope}%
\pgfsys@transformshift{2.311879in}{0.952344in}%
\pgfsys@useobject{currentmarker}{}%
\end{pgfscope}%
\begin{pgfscope}%
\pgfsys@transformshift{2.311879in}{0.952344in}%
\pgfsys@useobject{currentmarker}{}%
\end{pgfscope}%
\begin{pgfscope}%
\pgfsys@transformshift{2.311879in}{0.952344in}%
\pgfsys@useobject{currentmarker}{}%
\end{pgfscope}%
\begin{pgfscope}%
\pgfsys@transformshift{2.311879in}{0.952344in}%
\pgfsys@useobject{currentmarker}{}%
\end{pgfscope}%
\begin{pgfscope}%
\pgfsys@transformshift{2.311879in}{0.952344in}%
\pgfsys@useobject{currentmarker}{}%
\end{pgfscope}%
\begin{pgfscope}%
\pgfsys@transformshift{2.311879in}{0.952344in}%
\pgfsys@useobject{currentmarker}{}%
\end{pgfscope}%
\begin{pgfscope}%
\pgfsys@transformshift{2.311879in}{0.952344in}%
\pgfsys@useobject{currentmarker}{}%
\end{pgfscope}%
\begin{pgfscope}%
\pgfsys@transformshift{2.311879in}{0.952344in}%
\pgfsys@useobject{currentmarker}{}%
\end{pgfscope}%
\begin{pgfscope}%
\pgfsys@transformshift{2.311879in}{0.952344in}%
\pgfsys@useobject{currentmarker}{}%
\end{pgfscope}%
\begin{pgfscope}%
\pgfsys@transformshift{2.311879in}{0.952344in}%
\pgfsys@useobject{currentmarker}{}%
\end{pgfscope}%
\begin{pgfscope}%
\pgfsys@transformshift{2.311879in}{0.952344in}%
\pgfsys@useobject{currentmarker}{}%
\end{pgfscope}%
\begin{pgfscope}%
\pgfsys@transformshift{2.311879in}{0.952344in}%
\pgfsys@useobject{currentmarker}{}%
\end{pgfscope}%
\begin{pgfscope}%
\pgfsys@transformshift{2.311879in}{0.952344in}%
\pgfsys@useobject{currentmarker}{}%
\end{pgfscope}%
\begin{pgfscope}%
\pgfsys@transformshift{2.311879in}{0.952344in}%
\pgfsys@useobject{currentmarker}{}%
\end{pgfscope}%
\begin{pgfscope}%
\pgfsys@transformshift{2.311879in}{0.952344in}%
\pgfsys@useobject{currentmarker}{}%
\end{pgfscope}%
\begin{pgfscope}%
\pgfsys@transformshift{2.311879in}{0.952344in}%
\pgfsys@useobject{currentmarker}{}%
\end{pgfscope}%
\begin{pgfscope}%
\pgfsys@transformshift{2.311879in}{0.952344in}%
\pgfsys@useobject{currentmarker}{}%
\end{pgfscope}%
\begin{pgfscope}%
\pgfsys@transformshift{2.311879in}{0.952344in}%
\pgfsys@useobject{currentmarker}{}%
\end{pgfscope}%
\begin{pgfscope}%
\pgfsys@transformshift{2.311879in}{0.952344in}%
\pgfsys@useobject{currentmarker}{}%
\end{pgfscope}%
\begin{pgfscope}%
\pgfsys@transformshift{2.311879in}{0.952344in}%
\pgfsys@useobject{currentmarker}{}%
\end{pgfscope}%
\begin{pgfscope}%
\pgfsys@transformshift{2.311879in}{0.952344in}%
\pgfsys@useobject{currentmarker}{}%
\end{pgfscope}%
\begin{pgfscope}%
\pgfsys@transformshift{2.311879in}{0.952344in}%
\pgfsys@useobject{currentmarker}{}%
\end{pgfscope}%
\begin{pgfscope}%
\pgfsys@transformshift{2.311879in}{0.952344in}%
\pgfsys@useobject{currentmarker}{}%
\end{pgfscope}%
\begin{pgfscope}%
\pgfsys@transformshift{2.311879in}{0.952344in}%
\pgfsys@useobject{currentmarker}{}%
\end{pgfscope}%
\begin{pgfscope}%
\pgfsys@transformshift{2.311879in}{0.952344in}%
\pgfsys@useobject{currentmarker}{}%
\end{pgfscope}%
\begin{pgfscope}%
\pgfsys@transformshift{2.311879in}{0.952344in}%
\pgfsys@useobject{currentmarker}{}%
\end{pgfscope}%
\begin{pgfscope}%
\pgfsys@transformshift{2.311879in}{0.952344in}%
\pgfsys@useobject{currentmarker}{}%
\end{pgfscope}%
\begin{pgfscope}%
\pgfsys@transformshift{2.311879in}{0.952344in}%
\pgfsys@useobject{currentmarker}{}%
\end{pgfscope}%
\begin{pgfscope}%
\pgfsys@transformshift{2.311879in}{0.952344in}%
\pgfsys@useobject{currentmarker}{}%
\end{pgfscope}%
\begin{pgfscope}%
\pgfsys@transformshift{2.311879in}{0.952344in}%
\pgfsys@useobject{currentmarker}{}%
\end{pgfscope}%
\begin{pgfscope}%
\pgfsys@transformshift{2.311879in}{0.952344in}%
\pgfsys@useobject{currentmarker}{}%
\end{pgfscope}%
\begin{pgfscope}%
\pgfsys@transformshift{2.311879in}{0.952344in}%
\pgfsys@useobject{currentmarker}{}%
\end{pgfscope}%
\begin{pgfscope}%
\pgfsys@transformshift{2.311879in}{0.952344in}%
\pgfsys@useobject{currentmarker}{}%
\end{pgfscope}%
\begin{pgfscope}%
\pgfsys@transformshift{2.311879in}{0.952344in}%
\pgfsys@useobject{currentmarker}{}%
\end{pgfscope}%
\begin{pgfscope}%
\pgfsys@transformshift{2.311879in}{0.952344in}%
\pgfsys@useobject{currentmarker}{}%
\end{pgfscope}%
\begin{pgfscope}%
\pgfsys@transformshift{2.311879in}{0.952344in}%
\pgfsys@useobject{currentmarker}{}%
\end{pgfscope}%
\begin{pgfscope}%
\pgfsys@transformshift{2.311879in}{0.952344in}%
\pgfsys@useobject{currentmarker}{}%
\end{pgfscope}%
\begin{pgfscope}%
\pgfsys@transformshift{2.311879in}{0.952344in}%
\pgfsys@useobject{currentmarker}{}%
\end{pgfscope}%
\begin{pgfscope}%
\pgfsys@transformshift{2.311879in}{0.952344in}%
\pgfsys@useobject{currentmarker}{}%
\end{pgfscope}%
\begin{pgfscope}%
\pgfsys@transformshift{2.311879in}{0.952344in}%
\pgfsys@useobject{currentmarker}{}%
\end{pgfscope}%
\begin{pgfscope}%
\pgfsys@transformshift{2.311879in}{0.952344in}%
\pgfsys@useobject{currentmarker}{}%
\end{pgfscope}%
\begin{pgfscope}%
\pgfsys@transformshift{2.311879in}{0.952344in}%
\pgfsys@useobject{currentmarker}{}%
\end{pgfscope}%
\begin{pgfscope}%
\pgfsys@transformshift{2.311879in}{0.952344in}%
\pgfsys@useobject{currentmarker}{}%
\end{pgfscope}%
\begin{pgfscope}%
\pgfsys@transformshift{2.311879in}{0.952344in}%
\pgfsys@useobject{currentmarker}{}%
\end{pgfscope}%
\begin{pgfscope}%
\pgfsys@transformshift{2.311879in}{0.952344in}%
\pgfsys@useobject{currentmarker}{}%
\end{pgfscope}%
\begin{pgfscope}%
\pgfsys@transformshift{2.311879in}{0.952344in}%
\pgfsys@useobject{currentmarker}{}%
\end{pgfscope}%
\begin{pgfscope}%
\pgfsys@transformshift{2.311879in}{0.952344in}%
\pgfsys@useobject{currentmarker}{}%
\end{pgfscope}%
\begin{pgfscope}%
\pgfsys@transformshift{2.311879in}{0.952344in}%
\pgfsys@useobject{currentmarker}{}%
\end{pgfscope}%
\begin{pgfscope}%
\pgfsys@transformshift{2.311879in}{0.952344in}%
\pgfsys@useobject{currentmarker}{}%
\end{pgfscope}%
\begin{pgfscope}%
\pgfsys@transformshift{2.311879in}{0.952344in}%
\pgfsys@useobject{currentmarker}{}%
\end{pgfscope}%
\begin{pgfscope}%
\pgfsys@transformshift{2.311879in}{0.952344in}%
\pgfsys@useobject{currentmarker}{}%
\end{pgfscope}%
\begin{pgfscope}%
\pgfsys@transformshift{2.311879in}{0.952344in}%
\pgfsys@useobject{currentmarker}{}%
\end{pgfscope}%
\begin{pgfscope}%
\pgfsys@transformshift{2.311879in}{0.952344in}%
\pgfsys@useobject{currentmarker}{}%
\end{pgfscope}%
\begin{pgfscope}%
\pgfsys@transformshift{2.311879in}{0.952344in}%
\pgfsys@useobject{currentmarker}{}%
\end{pgfscope}%
\begin{pgfscope}%
\pgfsys@transformshift{2.311879in}{0.952344in}%
\pgfsys@useobject{currentmarker}{}%
\end{pgfscope}%
\begin{pgfscope}%
\pgfsys@transformshift{2.311879in}{0.952344in}%
\pgfsys@useobject{currentmarker}{}%
\end{pgfscope}%
\begin{pgfscope}%
\pgfsys@transformshift{2.311879in}{0.952344in}%
\pgfsys@useobject{currentmarker}{}%
\end{pgfscope}%
\begin{pgfscope}%
\pgfsys@transformshift{2.311879in}{0.952344in}%
\pgfsys@useobject{currentmarker}{}%
\end{pgfscope}%
\begin{pgfscope}%
\pgfsys@transformshift{2.311879in}{0.952344in}%
\pgfsys@useobject{currentmarker}{}%
\end{pgfscope}%
\begin{pgfscope}%
\pgfsys@transformshift{2.311879in}{0.952344in}%
\pgfsys@useobject{currentmarker}{}%
\end{pgfscope}%
\begin{pgfscope}%
\pgfsys@transformshift{2.311879in}{0.952344in}%
\pgfsys@useobject{currentmarker}{}%
\end{pgfscope}%
\begin{pgfscope}%
\pgfsys@transformshift{2.311879in}{0.952344in}%
\pgfsys@useobject{currentmarker}{}%
\end{pgfscope}%
\begin{pgfscope}%
\pgfsys@transformshift{2.311879in}{0.952344in}%
\pgfsys@useobject{currentmarker}{}%
\end{pgfscope}%
\begin{pgfscope}%
\pgfsys@transformshift{2.311879in}{0.952344in}%
\pgfsys@useobject{currentmarker}{}%
\end{pgfscope}%
\begin{pgfscope}%
\pgfsys@transformshift{2.311879in}{0.952344in}%
\pgfsys@useobject{currentmarker}{}%
\end{pgfscope}%
\begin{pgfscope}%
\pgfsys@transformshift{2.311879in}{0.952344in}%
\pgfsys@useobject{currentmarker}{}%
\end{pgfscope}%
\begin{pgfscope}%
\pgfsys@transformshift{2.311879in}{0.952344in}%
\pgfsys@useobject{currentmarker}{}%
\end{pgfscope}%
\begin{pgfscope}%
\pgfsys@transformshift{2.311879in}{0.952344in}%
\pgfsys@useobject{currentmarker}{}%
\end{pgfscope}%
\begin{pgfscope}%
\pgfsys@transformshift{2.311879in}{0.952344in}%
\pgfsys@useobject{currentmarker}{}%
\end{pgfscope}%
\begin{pgfscope}%
\pgfsys@transformshift{2.311879in}{0.952344in}%
\pgfsys@useobject{currentmarker}{}%
\end{pgfscope}%
\begin{pgfscope}%
\pgfsys@transformshift{2.311879in}{0.952344in}%
\pgfsys@useobject{currentmarker}{}%
\end{pgfscope}%
\begin{pgfscope}%
\pgfsys@transformshift{2.311879in}{0.952344in}%
\pgfsys@useobject{currentmarker}{}%
\end{pgfscope}%
\begin{pgfscope}%
\pgfsys@transformshift{2.311879in}{0.952344in}%
\pgfsys@useobject{currentmarker}{}%
\end{pgfscope}%
\begin{pgfscope}%
\pgfsys@transformshift{2.311879in}{0.952344in}%
\pgfsys@useobject{currentmarker}{}%
\end{pgfscope}%
\begin{pgfscope}%
\pgfsys@transformshift{2.311879in}{0.952344in}%
\pgfsys@useobject{currentmarker}{}%
\end{pgfscope}%
\begin{pgfscope}%
\pgfsys@transformshift{2.311879in}{0.952344in}%
\pgfsys@useobject{currentmarker}{}%
\end{pgfscope}%
\begin{pgfscope}%
\pgfsys@transformshift{2.311879in}{0.952344in}%
\pgfsys@useobject{currentmarker}{}%
\end{pgfscope}%
\begin{pgfscope}%
\pgfsys@transformshift{2.311879in}{0.952344in}%
\pgfsys@useobject{currentmarker}{}%
\end{pgfscope}%
\begin{pgfscope}%
\pgfsys@transformshift{2.311879in}{0.952344in}%
\pgfsys@useobject{currentmarker}{}%
\end{pgfscope}%
\begin{pgfscope}%
\pgfsys@transformshift{2.311879in}{0.952344in}%
\pgfsys@useobject{currentmarker}{}%
\end{pgfscope}%
\begin{pgfscope}%
\pgfsys@transformshift{2.311879in}{0.952344in}%
\pgfsys@useobject{currentmarker}{}%
\end{pgfscope}%
\begin{pgfscope}%
\pgfsys@transformshift{2.311879in}{0.952344in}%
\pgfsys@useobject{currentmarker}{}%
\end{pgfscope}%
\begin{pgfscope}%
\pgfsys@transformshift{2.311879in}{0.952344in}%
\pgfsys@useobject{currentmarker}{}%
\end{pgfscope}%
\begin{pgfscope}%
\pgfsys@transformshift{2.311879in}{0.952344in}%
\pgfsys@useobject{currentmarker}{}%
\end{pgfscope}%
\begin{pgfscope}%
\pgfsys@transformshift{2.311879in}{0.952344in}%
\pgfsys@useobject{currentmarker}{}%
\end{pgfscope}%
\begin{pgfscope}%
\pgfsys@transformshift{2.311879in}{0.952344in}%
\pgfsys@useobject{currentmarker}{}%
\end{pgfscope}%
\begin{pgfscope}%
\pgfsys@transformshift{2.311879in}{0.952344in}%
\pgfsys@useobject{currentmarker}{}%
\end{pgfscope}%
\begin{pgfscope}%
\pgfsys@transformshift{2.311879in}{0.952344in}%
\pgfsys@useobject{currentmarker}{}%
\end{pgfscope}%
\begin{pgfscope}%
\pgfsys@transformshift{2.311879in}{0.952344in}%
\pgfsys@useobject{currentmarker}{}%
\end{pgfscope}%
\begin{pgfscope}%
\pgfsys@transformshift{2.311879in}{0.952344in}%
\pgfsys@useobject{currentmarker}{}%
\end{pgfscope}%
\begin{pgfscope}%
\pgfsys@transformshift{2.311879in}{0.952344in}%
\pgfsys@useobject{currentmarker}{}%
\end{pgfscope}%
\begin{pgfscope}%
\pgfsys@transformshift{2.311879in}{0.952344in}%
\pgfsys@useobject{currentmarker}{}%
\end{pgfscope}%
\begin{pgfscope}%
\pgfsys@transformshift{2.311879in}{0.952344in}%
\pgfsys@useobject{currentmarker}{}%
\end{pgfscope}%
\begin{pgfscope}%
\pgfsys@transformshift{2.311879in}{0.952344in}%
\pgfsys@useobject{currentmarker}{}%
\end{pgfscope}%
\begin{pgfscope}%
\pgfsys@transformshift{2.311879in}{0.952344in}%
\pgfsys@useobject{currentmarker}{}%
\end{pgfscope}%
\begin{pgfscope}%
\pgfsys@transformshift{2.311879in}{0.952344in}%
\pgfsys@useobject{currentmarker}{}%
\end{pgfscope}%
\begin{pgfscope}%
\pgfsys@transformshift{2.311879in}{0.952344in}%
\pgfsys@useobject{currentmarker}{}%
\end{pgfscope}%
\begin{pgfscope}%
\pgfsys@transformshift{2.311879in}{0.952344in}%
\pgfsys@useobject{currentmarker}{}%
\end{pgfscope}%
\begin{pgfscope}%
\pgfsys@transformshift{2.311879in}{0.952344in}%
\pgfsys@useobject{currentmarker}{}%
\end{pgfscope}%
\begin{pgfscope}%
\pgfsys@transformshift{2.311879in}{0.952344in}%
\pgfsys@useobject{currentmarker}{}%
\end{pgfscope}%
\begin{pgfscope}%
\pgfsys@transformshift{2.311879in}{0.952344in}%
\pgfsys@useobject{currentmarker}{}%
\end{pgfscope}%
\begin{pgfscope}%
\pgfsys@transformshift{2.311879in}{0.952344in}%
\pgfsys@useobject{currentmarker}{}%
\end{pgfscope}%
\begin{pgfscope}%
\pgfsys@transformshift{2.311879in}{0.952344in}%
\pgfsys@useobject{currentmarker}{}%
\end{pgfscope}%
\begin{pgfscope}%
\pgfsys@transformshift{2.311879in}{0.952344in}%
\pgfsys@useobject{currentmarker}{}%
\end{pgfscope}%
\begin{pgfscope}%
\pgfsys@transformshift{2.311879in}{0.952344in}%
\pgfsys@useobject{currentmarker}{}%
\end{pgfscope}%
\begin{pgfscope}%
\pgfsys@transformshift{2.311879in}{0.952344in}%
\pgfsys@useobject{currentmarker}{}%
\end{pgfscope}%
\begin{pgfscope}%
\pgfsys@transformshift{2.311879in}{0.952344in}%
\pgfsys@useobject{currentmarker}{}%
\end{pgfscope}%
\begin{pgfscope}%
\pgfsys@transformshift{2.311879in}{0.952344in}%
\pgfsys@useobject{currentmarker}{}%
\end{pgfscope}%
\begin{pgfscope}%
\pgfsys@transformshift{2.311879in}{0.952344in}%
\pgfsys@useobject{currentmarker}{}%
\end{pgfscope}%
\begin{pgfscope}%
\pgfsys@transformshift{2.311879in}{0.952344in}%
\pgfsys@useobject{currentmarker}{}%
\end{pgfscope}%
\begin{pgfscope}%
\pgfsys@transformshift{2.311879in}{0.952344in}%
\pgfsys@useobject{currentmarker}{}%
\end{pgfscope}%
\begin{pgfscope}%
\pgfsys@transformshift{2.311879in}{0.952344in}%
\pgfsys@useobject{currentmarker}{}%
\end{pgfscope}%
\begin{pgfscope}%
\pgfsys@transformshift{2.311879in}{0.952344in}%
\pgfsys@useobject{currentmarker}{}%
\end{pgfscope}%
\begin{pgfscope}%
\pgfsys@transformshift{2.311879in}{0.952344in}%
\pgfsys@useobject{currentmarker}{}%
\end{pgfscope}%
\begin{pgfscope}%
\pgfsys@transformshift{2.311879in}{0.952344in}%
\pgfsys@useobject{currentmarker}{}%
\end{pgfscope}%
\begin{pgfscope}%
\pgfsys@transformshift{2.311879in}{0.952344in}%
\pgfsys@useobject{currentmarker}{}%
\end{pgfscope}%
\begin{pgfscope}%
\pgfsys@transformshift{2.311879in}{0.952344in}%
\pgfsys@useobject{currentmarker}{}%
\end{pgfscope}%
\begin{pgfscope}%
\pgfsys@transformshift{2.311879in}{0.952344in}%
\pgfsys@useobject{currentmarker}{}%
\end{pgfscope}%
\begin{pgfscope}%
\pgfsys@transformshift{2.311879in}{0.952344in}%
\pgfsys@useobject{currentmarker}{}%
\end{pgfscope}%
\begin{pgfscope}%
\pgfsys@transformshift{2.311879in}{0.952344in}%
\pgfsys@useobject{currentmarker}{}%
\end{pgfscope}%
\begin{pgfscope}%
\pgfsys@transformshift{2.311879in}{0.952344in}%
\pgfsys@useobject{currentmarker}{}%
\end{pgfscope}%
\begin{pgfscope}%
\pgfsys@transformshift{2.311879in}{0.952344in}%
\pgfsys@useobject{currentmarker}{}%
\end{pgfscope}%
\begin{pgfscope}%
\pgfsys@transformshift{2.311879in}{0.952344in}%
\pgfsys@useobject{currentmarker}{}%
\end{pgfscope}%
\begin{pgfscope}%
\pgfsys@transformshift{2.311879in}{0.952344in}%
\pgfsys@useobject{currentmarker}{}%
\end{pgfscope}%
\begin{pgfscope}%
\pgfsys@transformshift{2.311879in}{0.952344in}%
\pgfsys@useobject{currentmarker}{}%
\end{pgfscope}%
\begin{pgfscope}%
\pgfsys@transformshift{2.311879in}{0.952344in}%
\pgfsys@useobject{currentmarker}{}%
\end{pgfscope}%
\begin{pgfscope}%
\pgfsys@transformshift{2.311879in}{0.952344in}%
\pgfsys@useobject{currentmarker}{}%
\end{pgfscope}%
\begin{pgfscope}%
\pgfsys@transformshift{2.311879in}{0.952344in}%
\pgfsys@useobject{currentmarker}{}%
\end{pgfscope}%
\begin{pgfscope}%
\pgfsys@transformshift{2.311879in}{0.952344in}%
\pgfsys@useobject{currentmarker}{}%
\end{pgfscope}%
\begin{pgfscope}%
\pgfsys@transformshift{2.311879in}{0.952344in}%
\pgfsys@useobject{currentmarker}{}%
\end{pgfscope}%
\begin{pgfscope}%
\pgfsys@transformshift{2.311879in}{0.952344in}%
\pgfsys@useobject{currentmarker}{}%
\end{pgfscope}%
\begin{pgfscope}%
\pgfsys@transformshift{2.311879in}{0.952344in}%
\pgfsys@useobject{currentmarker}{}%
\end{pgfscope}%
\begin{pgfscope}%
\pgfsys@transformshift{2.311879in}{0.952344in}%
\pgfsys@useobject{currentmarker}{}%
\end{pgfscope}%
\begin{pgfscope}%
\pgfsys@transformshift{2.311879in}{0.952344in}%
\pgfsys@useobject{currentmarker}{}%
\end{pgfscope}%
\begin{pgfscope}%
\pgfsys@transformshift{2.311879in}{0.952344in}%
\pgfsys@useobject{currentmarker}{}%
\end{pgfscope}%
\begin{pgfscope}%
\pgfsys@transformshift{2.311879in}{0.952344in}%
\pgfsys@useobject{currentmarker}{}%
\end{pgfscope}%
\begin{pgfscope}%
\pgfsys@transformshift{2.311879in}{0.952344in}%
\pgfsys@useobject{currentmarker}{}%
\end{pgfscope}%
\begin{pgfscope}%
\pgfsys@transformshift{2.311879in}{0.952344in}%
\pgfsys@useobject{currentmarker}{}%
\end{pgfscope}%
\begin{pgfscope}%
\pgfsys@transformshift{2.311879in}{0.952344in}%
\pgfsys@useobject{currentmarker}{}%
\end{pgfscope}%
\begin{pgfscope}%
\pgfsys@transformshift{2.311879in}{0.952344in}%
\pgfsys@useobject{currentmarker}{}%
\end{pgfscope}%
\begin{pgfscope}%
\pgfsys@transformshift{2.311879in}{0.952344in}%
\pgfsys@useobject{currentmarker}{}%
\end{pgfscope}%
\begin{pgfscope}%
\pgfsys@transformshift{2.311879in}{0.952344in}%
\pgfsys@useobject{currentmarker}{}%
\end{pgfscope}%
\begin{pgfscope}%
\pgfsys@transformshift{2.311879in}{0.952344in}%
\pgfsys@useobject{currentmarker}{}%
\end{pgfscope}%
\begin{pgfscope}%
\pgfsys@transformshift{2.311879in}{0.952344in}%
\pgfsys@useobject{currentmarker}{}%
\end{pgfscope}%
\begin{pgfscope}%
\pgfsys@transformshift{2.311879in}{0.952344in}%
\pgfsys@useobject{currentmarker}{}%
\end{pgfscope}%
\begin{pgfscope}%
\pgfsys@transformshift{2.311879in}{0.952344in}%
\pgfsys@useobject{currentmarker}{}%
\end{pgfscope}%
\begin{pgfscope}%
\pgfsys@transformshift{2.311879in}{0.952344in}%
\pgfsys@useobject{currentmarker}{}%
\end{pgfscope}%
\begin{pgfscope}%
\pgfsys@transformshift{2.311879in}{0.952344in}%
\pgfsys@useobject{currentmarker}{}%
\end{pgfscope}%
\begin{pgfscope}%
\pgfsys@transformshift{2.311879in}{0.952344in}%
\pgfsys@useobject{currentmarker}{}%
\end{pgfscope}%
\begin{pgfscope}%
\pgfsys@transformshift{2.311879in}{0.952344in}%
\pgfsys@useobject{currentmarker}{}%
\end{pgfscope}%
\begin{pgfscope}%
\pgfsys@transformshift{2.311879in}{0.952344in}%
\pgfsys@useobject{currentmarker}{}%
\end{pgfscope}%
\begin{pgfscope}%
\pgfsys@transformshift{2.311879in}{0.952344in}%
\pgfsys@useobject{currentmarker}{}%
\end{pgfscope}%
\begin{pgfscope}%
\pgfsys@transformshift{2.311879in}{0.952344in}%
\pgfsys@useobject{currentmarker}{}%
\end{pgfscope}%
\begin{pgfscope}%
\pgfsys@transformshift{2.311879in}{0.952344in}%
\pgfsys@useobject{currentmarker}{}%
\end{pgfscope}%
\begin{pgfscope}%
\pgfsys@transformshift{2.311879in}{0.952344in}%
\pgfsys@useobject{currentmarker}{}%
\end{pgfscope}%
\begin{pgfscope}%
\pgfsys@transformshift{2.311879in}{0.952344in}%
\pgfsys@useobject{currentmarker}{}%
\end{pgfscope}%
\begin{pgfscope}%
\pgfsys@transformshift{2.311879in}{0.952344in}%
\pgfsys@useobject{currentmarker}{}%
\end{pgfscope}%
\begin{pgfscope}%
\pgfsys@transformshift{2.311879in}{0.952344in}%
\pgfsys@useobject{currentmarker}{}%
\end{pgfscope}%
\begin{pgfscope}%
\pgfsys@transformshift{2.311879in}{0.952344in}%
\pgfsys@useobject{currentmarker}{}%
\end{pgfscope}%
\begin{pgfscope}%
\pgfsys@transformshift{2.311879in}{0.952344in}%
\pgfsys@useobject{currentmarker}{}%
\end{pgfscope}%
\begin{pgfscope}%
\pgfsys@transformshift{2.311879in}{0.952344in}%
\pgfsys@useobject{currentmarker}{}%
\end{pgfscope}%
\begin{pgfscope}%
\pgfsys@transformshift{2.311879in}{0.952344in}%
\pgfsys@useobject{currentmarker}{}%
\end{pgfscope}%
\begin{pgfscope}%
\pgfsys@transformshift{2.311879in}{0.952344in}%
\pgfsys@useobject{currentmarker}{}%
\end{pgfscope}%
\begin{pgfscope}%
\pgfsys@transformshift{2.311879in}{0.952344in}%
\pgfsys@useobject{currentmarker}{}%
\end{pgfscope}%
\begin{pgfscope}%
\pgfsys@transformshift{2.311879in}{0.952344in}%
\pgfsys@useobject{currentmarker}{}%
\end{pgfscope}%
\begin{pgfscope}%
\pgfsys@transformshift{2.311879in}{0.952344in}%
\pgfsys@useobject{currentmarker}{}%
\end{pgfscope}%
\begin{pgfscope}%
\pgfsys@transformshift{2.311879in}{0.952344in}%
\pgfsys@useobject{currentmarker}{}%
\end{pgfscope}%
\begin{pgfscope}%
\pgfsys@transformshift{2.311879in}{0.952344in}%
\pgfsys@useobject{currentmarker}{}%
\end{pgfscope}%
\begin{pgfscope}%
\pgfsys@transformshift{2.311879in}{0.952344in}%
\pgfsys@useobject{currentmarker}{}%
\end{pgfscope}%
\begin{pgfscope}%
\pgfsys@transformshift{2.311879in}{0.952344in}%
\pgfsys@useobject{currentmarker}{}%
\end{pgfscope}%
\begin{pgfscope}%
\pgfsys@transformshift{2.311879in}{0.952344in}%
\pgfsys@useobject{currentmarker}{}%
\end{pgfscope}%
\begin{pgfscope}%
\pgfsys@transformshift{2.311879in}{0.952344in}%
\pgfsys@useobject{currentmarker}{}%
\end{pgfscope}%
\begin{pgfscope}%
\pgfsys@transformshift{2.311879in}{0.952344in}%
\pgfsys@useobject{currentmarker}{}%
\end{pgfscope}%
\begin{pgfscope}%
\pgfsys@transformshift{2.311879in}{0.952344in}%
\pgfsys@useobject{currentmarker}{}%
\end{pgfscope}%
\begin{pgfscope}%
\pgfsys@transformshift{2.311879in}{0.952344in}%
\pgfsys@useobject{currentmarker}{}%
\end{pgfscope}%
\begin{pgfscope}%
\pgfsys@transformshift{2.311879in}{0.952344in}%
\pgfsys@useobject{currentmarker}{}%
\end{pgfscope}%
\begin{pgfscope}%
\pgfsys@transformshift{2.311879in}{0.952344in}%
\pgfsys@useobject{currentmarker}{}%
\end{pgfscope}%
\begin{pgfscope}%
\pgfsys@transformshift{2.311879in}{0.952344in}%
\pgfsys@useobject{currentmarker}{}%
\end{pgfscope}%
\begin{pgfscope}%
\pgfsys@transformshift{2.311879in}{0.952344in}%
\pgfsys@useobject{currentmarker}{}%
\end{pgfscope}%
\begin{pgfscope}%
\pgfsys@transformshift{2.311879in}{0.952344in}%
\pgfsys@useobject{currentmarker}{}%
\end{pgfscope}%
\begin{pgfscope}%
\pgfsys@transformshift{2.311879in}{0.952344in}%
\pgfsys@useobject{currentmarker}{}%
\end{pgfscope}%
\begin{pgfscope}%
\pgfsys@transformshift{2.311879in}{0.952344in}%
\pgfsys@useobject{currentmarker}{}%
\end{pgfscope}%
\begin{pgfscope}%
\pgfsys@transformshift{2.311879in}{0.952344in}%
\pgfsys@useobject{currentmarker}{}%
\end{pgfscope}%
\begin{pgfscope}%
\pgfsys@transformshift{2.311879in}{0.952344in}%
\pgfsys@useobject{currentmarker}{}%
\end{pgfscope}%
\begin{pgfscope}%
\pgfsys@transformshift{2.311879in}{0.952344in}%
\pgfsys@useobject{currentmarker}{}%
\end{pgfscope}%
\begin{pgfscope}%
\pgfsys@transformshift{2.311879in}{0.952344in}%
\pgfsys@useobject{currentmarker}{}%
\end{pgfscope}%
\begin{pgfscope}%
\pgfsys@transformshift{2.311879in}{0.952344in}%
\pgfsys@useobject{currentmarker}{}%
\end{pgfscope}%
\begin{pgfscope}%
\pgfsys@transformshift{2.311879in}{0.952344in}%
\pgfsys@useobject{currentmarker}{}%
\end{pgfscope}%
\begin{pgfscope}%
\pgfsys@transformshift{2.311879in}{0.952344in}%
\pgfsys@useobject{currentmarker}{}%
\end{pgfscope}%
\begin{pgfscope}%
\pgfsys@transformshift{2.311879in}{0.952344in}%
\pgfsys@useobject{currentmarker}{}%
\end{pgfscope}%
\begin{pgfscope}%
\pgfsys@transformshift{2.311879in}{0.952344in}%
\pgfsys@useobject{currentmarker}{}%
\end{pgfscope}%
\begin{pgfscope}%
\pgfsys@transformshift{2.311879in}{0.952344in}%
\pgfsys@useobject{currentmarker}{}%
\end{pgfscope}%
\begin{pgfscope}%
\pgfsys@transformshift{2.311879in}{0.952344in}%
\pgfsys@useobject{currentmarker}{}%
\end{pgfscope}%
\begin{pgfscope}%
\pgfsys@transformshift{2.311879in}{0.952344in}%
\pgfsys@useobject{currentmarker}{}%
\end{pgfscope}%
\begin{pgfscope}%
\pgfsys@transformshift{2.311879in}{0.952344in}%
\pgfsys@useobject{currentmarker}{}%
\end{pgfscope}%
\begin{pgfscope}%
\pgfsys@transformshift{2.311879in}{0.952344in}%
\pgfsys@useobject{currentmarker}{}%
\end{pgfscope}%
\begin{pgfscope}%
\pgfsys@transformshift{2.311879in}{0.952344in}%
\pgfsys@useobject{currentmarker}{}%
\end{pgfscope}%
\begin{pgfscope}%
\pgfsys@transformshift{2.311879in}{0.952344in}%
\pgfsys@useobject{currentmarker}{}%
\end{pgfscope}%
\begin{pgfscope}%
\pgfsys@transformshift{2.311879in}{0.952344in}%
\pgfsys@useobject{currentmarker}{}%
\end{pgfscope}%
\begin{pgfscope}%
\pgfsys@transformshift{2.311879in}{0.952344in}%
\pgfsys@useobject{currentmarker}{}%
\end{pgfscope}%
\begin{pgfscope}%
\pgfsys@transformshift{2.311879in}{0.952344in}%
\pgfsys@useobject{currentmarker}{}%
\end{pgfscope}%
\begin{pgfscope}%
\pgfsys@transformshift{2.311879in}{0.952344in}%
\pgfsys@useobject{currentmarker}{}%
\end{pgfscope}%
\begin{pgfscope}%
\pgfsys@transformshift{2.311879in}{0.952344in}%
\pgfsys@useobject{currentmarker}{}%
\end{pgfscope}%
\begin{pgfscope}%
\pgfsys@transformshift{2.311879in}{0.952344in}%
\pgfsys@useobject{currentmarker}{}%
\end{pgfscope}%
\begin{pgfscope}%
\pgfsys@transformshift{2.311879in}{0.952344in}%
\pgfsys@useobject{currentmarker}{}%
\end{pgfscope}%
\begin{pgfscope}%
\pgfsys@transformshift{2.311879in}{0.952344in}%
\pgfsys@useobject{currentmarker}{}%
\end{pgfscope}%
\begin{pgfscope}%
\pgfsys@transformshift{2.311879in}{0.952344in}%
\pgfsys@useobject{currentmarker}{}%
\end{pgfscope}%
\begin{pgfscope}%
\pgfsys@transformshift{2.311879in}{0.952344in}%
\pgfsys@useobject{currentmarker}{}%
\end{pgfscope}%
\begin{pgfscope}%
\pgfsys@transformshift{2.311879in}{0.952344in}%
\pgfsys@useobject{currentmarker}{}%
\end{pgfscope}%
\begin{pgfscope}%
\pgfsys@transformshift{2.311879in}{0.952344in}%
\pgfsys@useobject{currentmarker}{}%
\end{pgfscope}%
\begin{pgfscope}%
\pgfsys@transformshift{2.311879in}{0.952344in}%
\pgfsys@useobject{currentmarker}{}%
\end{pgfscope}%
\begin{pgfscope}%
\pgfsys@transformshift{2.311879in}{0.952344in}%
\pgfsys@useobject{currentmarker}{}%
\end{pgfscope}%
\begin{pgfscope}%
\pgfsys@transformshift{2.311879in}{0.952344in}%
\pgfsys@useobject{currentmarker}{}%
\end{pgfscope}%
\begin{pgfscope}%
\pgfsys@transformshift{2.311879in}{0.952344in}%
\pgfsys@useobject{currentmarker}{}%
\end{pgfscope}%
\begin{pgfscope}%
\pgfsys@transformshift{2.311879in}{0.952344in}%
\pgfsys@useobject{currentmarker}{}%
\end{pgfscope}%
\begin{pgfscope}%
\pgfsys@transformshift{2.311879in}{0.952344in}%
\pgfsys@useobject{currentmarker}{}%
\end{pgfscope}%
\begin{pgfscope}%
\pgfsys@transformshift{2.311879in}{0.952344in}%
\pgfsys@useobject{currentmarker}{}%
\end{pgfscope}%
\begin{pgfscope}%
\pgfsys@transformshift{2.311879in}{0.952344in}%
\pgfsys@useobject{currentmarker}{}%
\end{pgfscope}%
\begin{pgfscope}%
\pgfsys@transformshift{2.311879in}{0.952344in}%
\pgfsys@useobject{currentmarker}{}%
\end{pgfscope}%
\begin{pgfscope}%
\pgfsys@transformshift{2.311879in}{0.952344in}%
\pgfsys@useobject{currentmarker}{}%
\end{pgfscope}%
\begin{pgfscope}%
\pgfsys@transformshift{2.311879in}{0.952344in}%
\pgfsys@useobject{currentmarker}{}%
\end{pgfscope}%
\begin{pgfscope}%
\pgfsys@transformshift{2.311879in}{0.952344in}%
\pgfsys@useobject{currentmarker}{}%
\end{pgfscope}%
\begin{pgfscope}%
\pgfsys@transformshift{2.311879in}{0.952344in}%
\pgfsys@useobject{currentmarker}{}%
\end{pgfscope}%
\begin{pgfscope}%
\pgfsys@transformshift{2.311879in}{0.952344in}%
\pgfsys@useobject{currentmarker}{}%
\end{pgfscope}%
\begin{pgfscope}%
\pgfsys@transformshift{2.311879in}{0.952344in}%
\pgfsys@useobject{currentmarker}{}%
\end{pgfscope}%
\begin{pgfscope}%
\pgfsys@transformshift{2.311879in}{0.952344in}%
\pgfsys@useobject{currentmarker}{}%
\end{pgfscope}%
\begin{pgfscope}%
\pgfsys@transformshift{2.311879in}{0.952344in}%
\pgfsys@useobject{currentmarker}{}%
\end{pgfscope}%
\begin{pgfscope}%
\pgfsys@transformshift{2.311879in}{0.952344in}%
\pgfsys@useobject{currentmarker}{}%
\end{pgfscope}%
\begin{pgfscope}%
\pgfsys@transformshift{2.311879in}{0.952344in}%
\pgfsys@useobject{currentmarker}{}%
\end{pgfscope}%
\begin{pgfscope}%
\pgfsys@transformshift{2.311879in}{0.952344in}%
\pgfsys@useobject{currentmarker}{}%
\end{pgfscope}%
\begin{pgfscope}%
\pgfsys@transformshift{2.311879in}{0.952344in}%
\pgfsys@useobject{currentmarker}{}%
\end{pgfscope}%
\begin{pgfscope}%
\pgfsys@transformshift{2.311879in}{0.952344in}%
\pgfsys@useobject{currentmarker}{}%
\end{pgfscope}%
\begin{pgfscope}%
\pgfsys@transformshift{2.311879in}{0.952344in}%
\pgfsys@useobject{currentmarker}{}%
\end{pgfscope}%
\begin{pgfscope}%
\pgfsys@transformshift{2.311879in}{0.952344in}%
\pgfsys@useobject{currentmarker}{}%
\end{pgfscope}%
\begin{pgfscope}%
\pgfsys@transformshift{2.311879in}{0.952344in}%
\pgfsys@useobject{currentmarker}{}%
\end{pgfscope}%
\begin{pgfscope}%
\pgfsys@transformshift{2.311879in}{0.952344in}%
\pgfsys@useobject{currentmarker}{}%
\end{pgfscope}%
\begin{pgfscope}%
\pgfsys@transformshift{2.311879in}{0.952344in}%
\pgfsys@useobject{currentmarker}{}%
\end{pgfscope}%
\begin{pgfscope}%
\pgfsys@transformshift{2.311879in}{0.952344in}%
\pgfsys@useobject{currentmarker}{}%
\end{pgfscope}%
\begin{pgfscope}%
\pgfsys@transformshift{2.311879in}{0.952344in}%
\pgfsys@useobject{currentmarker}{}%
\end{pgfscope}%
\begin{pgfscope}%
\pgfsys@transformshift{2.311879in}{0.952344in}%
\pgfsys@useobject{currentmarker}{}%
\end{pgfscope}%
\begin{pgfscope}%
\pgfsys@transformshift{2.311879in}{0.952344in}%
\pgfsys@useobject{currentmarker}{}%
\end{pgfscope}%
\begin{pgfscope}%
\pgfsys@transformshift{2.311879in}{0.952344in}%
\pgfsys@useobject{currentmarker}{}%
\end{pgfscope}%
\begin{pgfscope}%
\pgfsys@transformshift{2.311879in}{0.952344in}%
\pgfsys@useobject{currentmarker}{}%
\end{pgfscope}%
\begin{pgfscope}%
\pgfsys@transformshift{2.311879in}{0.952344in}%
\pgfsys@useobject{currentmarker}{}%
\end{pgfscope}%
\begin{pgfscope}%
\pgfsys@transformshift{2.311879in}{0.952344in}%
\pgfsys@useobject{currentmarker}{}%
\end{pgfscope}%
\begin{pgfscope}%
\pgfsys@transformshift{2.311879in}{0.952344in}%
\pgfsys@useobject{currentmarker}{}%
\end{pgfscope}%
\begin{pgfscope}%
\pgfsys@transformshift{2.311879in}{0.952344in}%
\pgfsys@useobject{currentmarker}{}%
\end{pgfscope}%
\begin{pgfscope}%
\pgfsys@transformshift{2.311879in}{0.952344in}%
\pgfsys@useobject{currentmarker}{}%
\end{pgfscope}%
\begin{pgfscope}%
\pgfsys@transformshift{2.311879in}{0.952344in}%
\pgfsys@useobject{currentmarker}{}%
\end{pgfscope}%
\begin{pgfscope}%
\pgfsys@transformshift{2.311879in}{0.952344in}%
\pgfsys@useobject{currentmarker}{}%
\end{pgfscope}%
\begin{pgfscope}%
\pgfsys@transformshift{2.311879in}{0.952344in}%
\pgfsys@useobject{currentmarker}{}%
\end{pgfscope}%
\begin{pgfscope}%
\pgfsys@transformshift{2.311879in}{0.952344in}%
\pgfsys@useobject{currentmarker}{}%
\end{pgfscope}%
\begin{pgfscope}%
\pgfsys@transformshift{2.311879in}{0.952344in}%
\pgfsys@useobject{currentmarker}{}%
\end{pgfscope}%
\begin{pgfscope}%
\pgfsys@transformshift{2.311879in}{0.952344in}%
\pgfsys@useobject{currentmarker}{}%
\end{pgfscope}%
\begin{pgfscope}%
\pgfsys@transformshift{2.311879in}{0.952344in}%
\pgfsys@useobject{currentmarker}{}%
\end{pgfscope}%
\begin{pgfscope}%
\pgfsys@transformshift{2.311879in}{0.952344in}%
\pgfsys@useobject{currentmarker}{}%
\end{pgfscope}%
\begin{pgfscope}%
\pgfsys@transformshift{2.311879in}{0.952344in}%
\pgfsys@useobject{currentmarker}{}%
\end{pgfscope}%
\begin{pgfscope}%
\pgfsys@transformshift{2.311879in}{0.952344in}%
\pgfsys@useobject{currentmarker}{}%
\end{pgfscope}%
\begin{pgfscope}%
\pgfsys@transformshift{2.311879in}{0.952344in}%
\pgfsys@useobject{currentmarker}{}%
\end{pgfscope}%
\begin{pgfscope}%
\pgfsys@transformshift{2.311879in}{0.952344in}%
\pgfsys@useobject{currentmarker}{}%
\end{pgfscope}%
\begin{pgfscope}%
\pgfsys@transformshift{2.311879in}{0.952344in}%
\pgfsys@useobject{currentmarker}{}%
\end{pgfscope}%
\begin{pgfscope}%
\pgfsys@transformshift{2.311879in}{0.952344in}%
\pgfsys@useobject{currentmarker}{}%
\end{pgfscope}%
\begin{pgfscope}%
\pgfsys@transformshift{2.311879in}{0.952344in}%
\pgfsys@useobject{currentmarker}{}%
\end{pgfscope}%
\begin{pgfscope}%
\pgfsys@transformshift{2.311879in}{0.952344in}%
\pgfsys@useobject{currentmarker}{}%
\end{pgfscope}%
\begin{pgfscope}%
\pgfsys@transformshift{2.311879in}{0.952344in}%
\pgfsys@useobject{currentmarker}{}%
\end{pgfscope}%
\begin{pgfscope}%
\pgfsys@transformshift{2.311879in}{0.952344in}%
\pgfsys@useobject{currentmarker}{}%
\end{pgfscope}%
\begin{pgfscope}%
\pgfsys@transformshift{2.311879in}{0.952344in}%
\pgfsys@useobject{currentmarker}{}%
\end{pgfscope}%
\begin{pgfscope}%
\pgfsys@transformshift{2.311879in}{0.952344in}%
\pgfsys@useobject{currentmarker}{}%
\end{pgfscope}%
\begin{pgfscope}%
\pgfsys@transformshift{2.311879in}{0.952344in}%
\pgfsys@useobject{currentmarker}{}%
\end{pgfscope}%
\begin{pgfscope}%
\pgfsys@transformshift{2.311879in}{0.952344in}%
\pgfsys@useobject{currentmarker}{}%
\end{pgfscope}%
\begin{pgfscope}%
\pgfsys@transformshift{2.311879in}{0.952344in}%
\pgfsys@useobject{currentmarker}{}%
\end{pgfscope}%
\begin{pgfscope}%
\pgfsys@transformshift{2.311879in}{0.952344in}%
\pgfsys@useobject{currentmarker}{}%
\end{pgfscope}%
\begin{pgfscope}%
\pgfsys@transformshift{2.311879in}{0.952344in}%
\pgfsys@useobject{currentmarker}{}%
\end{pgfscope}%
\begin{pgfscope}%
\pgfsys@transformshift{2.311879in}{0.952344in}%
\pgfsys@useobject{currentmarker}{}%
\end{pgfscope}%
\begin{pgfscope}%
\pgfsys@transformshift{2.311879in}{0.952344in}%
\pgfsys@useobject{currentmarker}{}%
\end{pgfscope}%
\begin{pgfscope}%
\pgfsys@transformshift{2.311879in}{0.952344in}%
\pgfsys@useobject{currentmarker}{}%
\end{pgfscope}%
\begin{pgfscope}%
\pgfsys@transformshift{2.311879in}{0.952344in}%
\pgfsys@useobject{currentmarker}{}%
\end{pgfscope}%
\begin{pgfscope}%
\pgfsys@transformshift{2.311879in}{0.952344in}%
\pgfsys@useobject{currentmarker}{}%
\end{pgfscope}%
\begin{pgfscope}%
\pgfsys@transformshift{2.311879in}{0.952344in}%
\pgfsys@useobject{currentmarker}{}%
\end{pgfscope}%
\begin{pgfscope}%
\pgfsys@transformshift{2.311879in}{0.952344in}%
\pgfsys@useobject{currentmarker}{}%
\end{pgfscope}%
\begin{pgfscope}%
\pgfsys@transformshift{2.311879in}{0.952344in}%
\pgfsys@useobject{currentmarker}{}%
\end{pgfscope}%
\begin{pgfscope}%
\pgfsys@transformshift{2.311879in}{0.952344in}%
\pgfsys@useobject{currentmarker}{}%
\end{pgfscope}%
\begin{pgfscope}%
\pgfsys@transformshift{2.311879in}{0.952344in}%
\pgfsys@useobject{currentmarker}{}%
\end{pgfscope}%
\begin{pgfscope}%
\pgfsys@transformshift{2.311879in}{0.952344in}%
\pgfsys@useobject{currentmarker}{}%
\end{pgfscope}%
\begin{pgfscope}%
\pgfsys@transformshift{2.311879in}{0.952344in}%
\pgfsys@useobject{currentmarker}{}%
\end{pgfscope}%
\begin{pgfscope}%
\pgfsys@transformshift{2.311879in}{0.952344in}%
\pgfsys@useobject{currentmarker}{}%
\end{pgfscope}%
\begin{pgfscope}%
\pgfsys@transformshift{2.311879in}{0.952344in}%
\pgfsys@useobject{currentmarker}{}%
\end{pgfscope}%
\begin{pgfscope}%
\pgfsys@transformshift{2.311879in}{0.952344in}%
\pgfsys@useobject{currentmarker}{}%
\end{pgfscope}%
\begin{pgfscope}%
\pgfsys@transformshift{2.311879in}{0.952344in}%
\pgfsys@useobject{currentmarker}{}%
\end{pgfscope}%
\begin{pgfscope}%
\pgfsys@transformshift{2.311879in}{0.952344in}%
\pgfsys@useobject{currentmarker}{}%
\end{pgfscope}%
\begin{pgfscope}%
\pgfsys@transformshift{2.311879in}{0.952344in}%
\pgfsys@useobject{currentmarker}{}%
\end{pgfscope}%
\begin{pgfscope}%
\pgfsys@transformshift{2.311879in}{0.952344in}%
\pgfsys@useobject{currentmarker}{}%
\end{pgfscope}%
\begin{pgfscope}%
\pgfsys@transformshift{2.311879in}{0.952344in}%
\pgfsys@useobject{currentmarker}{}%
\end{pgfscope}%
\begin{pgfscope}%
\pgfsys@transformshift{2.311879in}{0.952344in}%
\pgfsys@useobject{currentmarker}{}%
\end{pgfscope}%
\begin{pgfscope}%
\pgfsys@transformshift{2.311879in}{0.952344in}%
\pgfsys@useobject{currentmarker}{}%
\end{pgfscope}%
\begin{pgfscope}%
\pgfsys@transformshift{2.311879in}{0.952344in}%
\pgfsys@useobject{currentmarker}{}%
\end{pgfscope}%
\begin{pgfscope}%
\pgfsys@transformshift{2.311879in}{0.952344in}%
\pgfsys@useobject{currentmarker}{}%
\end{pgfscope}%
\begin{pgfscope}%
\pgfsys@transformshift{2.311879in}{0.952344in}%
\pgfsys@useobject{currentmarker}{}%
\end{pgfscope}%
\begin{pgfscope}%
\pgfsys@transformshift{2.311879in}{0.952344in}%
\pgfsys@useobject{currentmarker}{}%
\end{pgfscope}%
\begin{pgfscope}%
\pgfsys@transformshift{2.311879in}{0.952344in}%
\pgfsys@useobject{currentmarker}{}%
\end{pgfscope}%
\begin{pgfscope}%
\pgfsys@transformshift{2.311879in}{0.952344in}%
\pgfsys@useobject{currentmarker}{}%
\end{pgfscope}%
\begin{pgfscope}%
\pgfsys@transformshift{2.311879in}{0.952344in}%
\pgfsys@useobject{currentmarker}{}%
\end{pgfscope}%
\begin{pgfscope}%
\pgfsys@transformshift{2.311879in}{0.952344in}%
\pgfsys@useobject{currentmarker}{}%
\end{pgfscope}%
\begin{pgfscope}%
\pgfsys@transformshift{2.311879in}{0.952344in}%
\pgfsys@useobject{currentmarker}{}%
\end{pgfscope}%
\begin{pgfscope}%
\pgfsys@transformshift{2.311879in}{0.952344in}%
\pgfsys@useobject{currentmarker}{}%
\end{pgfscope}%
\begin{pgfscope}%
\pgfsys@transformshift{2.311879in}{0.952344in}%
\pgfsys@useobject{currentmarker}{}%
\end{pgfscope}%
\begin{pgfscope}%
\pgfsys@transformshift{2.311879in}{0.952344in}%
\pgfsys@useobject{currentmarker}{}%
\end{pgfscope}%
\begin{pgfscope}%
\pgfsys@transformshift{2.311879in}{0.952344in}%
\pgfsys@useobject{currentmarker}{}%
\end{pgfscope}%
\begin{pgfscope}%
\pgfsys@transformshift{2.311879in}{0.952344in}%
\pgfsys@useobject{currentmarker}{}%
\end{pgfscope}%
\begin{pgfscope}%
\pgfsys@transformshift{2.311879in}{0.952344in}%
\pgfsys@useobject{currentmarker}{}%
\end{pgfscope}%
\begin{pgfscope}%
\pgfsys@transformshift{2.311879in}{0.952344in}%
\pgfsys@useobject{currentmarker}{}%
\end{pgfscope}%
\begin{pgfscope}%
\pgfsys@transformshift{2.311879in}{0.952344in}%
\pgfsys@useobject{currentmarker}{}%
\end{pgfscope}%
\begin{pgfscope}%
\pgfsys@transformshift{2.311879in}{0.952344in}%
\pgfsys@useobject{currentmarker}{}%
\end{pgfscope}%
\begin{pgfscope}%
\pgfsys@transformshift{2.311879in}{0.952344in}%
\pgfsys@useobject{currentmarker}{}%
\end{pgfscope}%
\begin{pgfscope}%
\pgfsys@transformshift{2.311879in}{0.952344in}%
\pgfsys@useobject{currentmarker}{}%
\end{pgfscope}%
\begin{pgfscope}%
\pgfsys@transformshift{2.311879in}{0.952344in}%
\pgfsys@useobject{currentmarker}{}%
\end{pgfscope}%
\begin{pgfscope}%
\pgfsys@transformshift{2.311879in}{0.952344in}%
\pgfsys@useobject{currentmarker}{}%
\end{pgfscope}%
\begin{pgfscope}%
\pgfsys@transformshift{2.311879in}{0.952344in}%
\pgfsys@useobject{currentmarker}{}%
\end{pgfscope}%
\begin{pgfscope}%
\pgfsys@transformshift{2.311879in}{0.952344in}%
\pgfsys@useobject{currentmarker}{}%
\end{pgfscope}%
\begin{pgfscope}%
\pgfsys@transformshift{2.311879in}{0.952344in}%
\pgfsys@useobject{currentmarker}{}%
\end{pgfscope}%
\begin{pgfscope}%
\pgfsys@transformshift{2.311879in}{0.952344in}%
\pgfsys@useobject{currentmarker}{}%
\end{pgfscope}%
\begin{pgfscope}%
\pgfsys@transformshift{2.311879in}{0.952344in}%
\pgfsys@useobject{currentmarker}{}%
\end{pgfscope}%
\begin{pgfscope}%
\pgfsys@transformshift{2.311879in}{0.952344in}%
\pgfsys@useobject{currentmarker}{}%
\end{pgfscope}%
\begin{pgfscope}%
\pgfsys@transformshift{2.311879in}{0.952344in}%
\pgfsys@useobject{currentmarker}{}%
\end{pgfscope}%
\begin{pgfscope}%
\pgfsys@transformshift{2.311879in}{0.952344in}%
\pgfsys@useobject{currentmarker}{}%
\end{pgfscope}%
\begin{pgfscope}%
\pgfsys@transformshift{2.311879in}{0.952344in}%
\pgfsys@useobject{currentmarker}{}%
\end{pgfscope}%
\begin{pgfscope}%
\pgfsys@transformshift{2.311879in}{0.952344in}%
\pgfsys@useobject{currentmarker}{}%
\end{pgfscope}%
\begin{pgfscope}%
\pgfsys@transformshift{2.311879in}{0.952344in}%
\pgfsys@useobject{currentmarker}{}%
\end{pgfscope}%
\begin{pgfscope}%
\pgfsys@transformshift{2.311879in}{0.952344in}%
\pgfsys@useobject{currentmarker}{}%
\end{pgfscope}%
\begin{pgfscope}%
\pgfsys@transformshift{2.311879in}{0.952344in}%
\pgfsys@useobject{currentmarker}{}%
\end{pgfscope}%
\begin{pgfscope}%
\pgfsys@transformshift{2.311879in}{0.952344in}%
\pgfsys@useobject{currentmarker}{}%
\end{pgfscope}%
\begin{pgfscope}%
\pgfsys@transformshift{2.311879in}{0.952344in}%
\pgfsys@useobject{currentmarker}{}%
\end{pgfscope}%
\begin{pgfscope}%
\pgfsys@transformshift{2.311879in}{0.952344in}%
\pgfsys@useobject{currentmarker}{}%
\end{pgfscope}%
\begin{pgfscope}%
\pgfsys@transformshift{2.311879in}{0.952344in}%
\pgfsys@useobject{currentmarker}{}%
\end{pgfscope}%
\begin{pgfscope}%
\pgfsys@transformshift{2.311879in}{0.952344in}%
\pgfsys@useobject{currentmarker}{}%
\end{pgfscope}%
\begin{pgfscope}%
\pgfsys@transformshift{2.311879in}{0.952344in}%
\pgfsys@useobject{currentmarker}{}%
\end{pgfscope}%
\begin{pgfscope}%
\pgfsys@transformshift{2.311879in}{0.952344in}%
\pgfsys@useobject{currentmarker}{}%
\end{pgfscope}%
\begin{pgfscope}%
\pgfsys@transformshift{2.311879in}{0.952344in}%
\pgfsys@useobject{currentmarker}{}%
\end{pgfscope}%
\begin{pgfscope}%
\pgfsys@transformshift{2.311879in}{0.952344in}%
\pgfsys@useobject{currentmarker}{}%
\end{pgfscope}%
\begin{pgfscope}%
\pgfsys@transformshift{2.311879in}{0.952344in}%
\pgfsys@useobject{currentmarker}{}%
\end{pgfscope}%
\begin{pgfscope}%
\pgfsys@transformshift{2.311879in}{0.952344in}%
\pgfsys@useobject{currentmarker}{}%
\end{pgfscope}%
\begin{pgfscope}%
\pgfsys@transformshift{2.311879in}{0.952344in}%
\pgfsys@useobject{currentmarker}{}%
\end{pgfscope}%
\begin{pgfscope}%
\pgfsys@transformshift{2.311879in}{0.952344in}%
\pgfsys@useobject{currentmarker}{}%
\end{pgfscope}%
\begin{pgfscope}%
\pgfsys@transformshift{2.311879in}{0.952344in}%
\pgfsys@useobject{currentmarker}{}%
\end{pgfscope}%
\begin{pgfscope}%
\pgfsys@transformshift{2.311879in}{0.952344in}%
\pgfsys@useobject{currentmarker}{}%
\end{pgfscope}%
\begin{pgfscope}%
\pgfsys@transformshift{2.311879in}{0.952344in}%
\pgfsys@useobject{currentmarker}{}%
\end{pgfscope}%
\begin{pgfscope}%
\pgfsys@transformshift{2.311879in}{0.952344in}%
\pgfsys@useobject{currentmarker}{}%
\end{pgfscope}%
\begin{pgfscope}%
\pgfsys@transformshift{2.311879in}{0.952344in}%
\pgfsys@useobject{currentmarker}{}%
\end{pgfscope}%
\begin{pgfscope}%
\pgfsys@transformshift{2.311879in}{0.952344in}%
\pgfsys@useobject{currentmarker}{}%
\end{pgfscope}%
\begin{pgfscope}%
\pgfsys@transformshift{2.311879in}{0.952344in}%
\pgfsys@useobject{currentmarker}{}%
\end{pgfscope}%
\begin{pgfscope}%
\pgfsys@transformshift{2.311879in}{0.952344in}%
\pgfsys@useobject{currentmarker}{}%
\end{pgfscope}%
\begin{pgfscope}%
\pgfsys@transformshift{2.311879in}{0.952344in}%
\pgfsys@useobject{currentmarker}{}%
\end{pgfscope}%
\begin{pgfscope}%
\pgfsys@transformshift{2.311879in}{0.952344in}%
\pgfsys@useobject{currentmarker}{}%
\end{pgfscope}%
\begin{pgfscope}%
\pgfsys@transformshift{2.311879in}{0.952344in}%
\pgfsys@useobject{currentmarker}{}%
\end{pgfscope}%
\begin{pgfscope}%
\pgfsys@transformshift{2.311879in}{0.952344in}%
\pgfsys@useobject{currentmarker}{}%
\end{pgfscope}%
\begin{pgfscope}%
\pgfsys@transformshift{2.311879in}{0.952344in}%
\pgfsys@useobject{currentmarker}{}%
\end{pgfscope}%
\begin{pgfscope}%
\pgfsys@transformshift{2.311879in}{0.952344in}%
\pgfsys@useobject{currentmarker}{}%
\end{pgfscope}%
\begin{pgfscope}%
\pgfsys@transformshift{2.311879in}{0.952344in}%
\pgfsys@useobject{currentmarker}{}%
\end{pgfscope}%
\begin{pgfscope}%
\pgfsys@transformshift{2.311879in}{0.952344in}%
\pgfsys@useobject{currentmarker}{}%
\end{pgfscope}%
\begin{pgfscope}%
\pgfsys@transformshift{2.311879in}{0.952344in}%
\pgfsys@useobject{currentmarker}{}%
\end{pgfscope}%
\begin{pgfscope}%
\pgfsys@transformshift{2.311879in}{0.952344in}%
\pgfsys@useobject{currentmarker}{}%
\end{pgfscope}%
\begin{pgfscope}%
\pgfsys@transformshift{2.311879in}{0.952344in}%
\pgfsys@useobject{currentmarker}{}%
\end{pgfscope}%
\begin{pgfscope}%
\pgfsys@transformshift{2.311879in}{0.952344in}%
\pgfsys@useobject{currentmarker}{}%
\end{pgfscope}%
\begin{pgfscope}%
\pgfsys@transformshift{2.311879in}{0.952344in}%
\pgfsys@useobject{currentmarker}{}%
\end{pgfscope}%
\begin{pgfscope}%
\pgfsys@transformshift{2.311879in}{0.952344in}%
\pgfsys@useobject{currentmarker}{}%
\end{pgfscope}%
\begin{pgfscope}%
\pgfsys@transformshift{2.311879in}{0.952344in}%
\pgfsys@useobject{currentmarker}{}%
\end{pgfscope}%
\begin{pgfscope}%
\pgfsys@transformshift{2.311879in}{0.952344in}%
\pgfsys@useobject{currentmarker}{}%
\end{pgfscope}%
\begin{pgfscope}%
\pgfsys@transformshift{2.311879in}{0.952344in}%
\pgfsys@useobject{currentmarker}{}%
\end{pgfscope}%
\begin{pgfscope}%
\pgfsys@transformshift{2.311879in}{0.952344in}%
\pgfsys@useobject{currentmarker}{}%
\end{pgfscope}%
\begin{pgfscope}%
\pgfsys@transformshift{2.311879in}{0.952344in}%
\pgfsys@useobject{currentmarker}{}%
\end{pgfscope}%
\begin{pgfscope}%
\pgfsys@transformshift{2.311879in}{0.952344in}%
\pgfsys@useobject{currentmarker}{}%
\end{pgfscope}%
\begin{pgfscope}%
\pgfsys@transformshift{2.311879in}{0.952344in}%
\pgfsys@useobject{currentmarker}{}%
\end{pgfscope}%
\begin{pgfscope}%
\pgfsys@transformshift{2.311879in}{0.952344in}%
\pgfsys@useobject{currentmarker}{}%
\end{pgfscope}%
\begin{pgfscope}%
\pgfsys@transformshift{2.311879in}{0.952344in}%
\pgfsys@useobject{currentmarker}{}%
\end{pgfscope}%
\begin{pgfscope}%
\pgfsys@transformshift{2.311879in}{0.952344in}%
\pgfsys@useobject{currentmarker}{}%
\end{pgfscope}%
\begin{pgfscope}%
\pgfsys@transformshift{2.311879in}{0.952344in}%
\pgfsys@useobject{currentmarker}{}%
\end{pgfscope}%
\begin{pgfscope}%
\pgfsys@transformshift{2.311879in}{0.952344in}%
\pgfsys@useobject{currentmarker}{}%
\end{pgfscope}%
\begin{pgfscope}%
\pgfsys@transformshift{2.311879in}{0.952344in}%
\pgfsys@useobject{currentmarker}{}%
\end{pgfscope}%
\begin{pgfscope}%
\pgfsys@transformshift{2.311879in}{0.952344in}%
\pgfsys@useobject{currentmarker}{}%
\end{pgfscope}%
\begin{pgfscope}%
\pgfsys@transformshift{2.311879in}{0.952344in}%
\pgfsys@useobject{currentmarker}{}%
\end{pgfscope}%
\begin{pgfscope}%
\pgfsys@transformshift{2.311879in}{0.952344in}%
\pgfsys@useobject{currentmarker}{}%
\end{pgfscope}%
\begin{pgfscope}%
\pgfsys@transformshift{2.311879in}{0.952344in}%
\pgfsys@useobject{currentmarker}{}%
\end{pgfscope}%
\begin{pgfscope}%
\pgfsys@transformshift{2.311879in}{0.952344in}%
\pgfsys@useobject{currentmarker}{}%
\end{pgfscope}%
\begin{pgfscope}%
\pgfsys@transformshift{2.311879in}{0.952344in}%
\pgfsys@useobject{currentmarker}{}%
\end{pgfscope}%
\begin{pgfscope}%
\pgfsys@transformshift{2.311879in}{0.952344in}%
\pgfsys@useobject{currentmarker}{}%
\end{pgfscope}%
\begin{pgfscope}%
\pgfsys@transformshift{2.311879in}{0.952344in}%
\pgfsys@useobject{currentmarker}{}%
\end{pgfscope}%
\begin{pgfscope}%
\pgfsys@transformshift{2.311879in}{0.952344in}%
\pgfsys@useobject{currentmarker}{}%
\end{pgfscope}%
\begin{pgfscope}%
\pgfsys@transformshift{2.311879in}{0.952344in}%
\pgfsys@useobject{currentmarker}{}%
\end{pgfscope}%
\begin{pgfscope}%
\pgfsys@transformshift{2.311879in}{0.952344in}%
\pgfsys@useobject{currentmarker}{}%
\end{pgfscope}%
\begin{pgfscope}%
\pgfsys@transformshift{2.311879in}{0.952344in}%
\pgfsys@useobject{currentmarker}{}%
\end{pgfscope}%
\begin{pgfscope}%
\pgfsys@transformshift{2.311879in}{0.952344in}%
\pgfsys@useobject{currentmarker}{}%
\end{pgfscope}%
\begin{pgfscope}%
\pgfsys@transformshift{2.311879in}{0.952344in}%
\pgfsys@useobject{currentmarker}{}%
\end{pgfscope}%
\begin{pgfscope}%
\pgfsys@transformshift{2.311879in}{0.952344in}%
\pgfsys@useobject{currentmarker}{}%
\end{pgfscope}%
\begin{pgfscope}%
\pgfsys@transformshift{2.311879in}{0.952344in}%
\pgfsys@useobject{currentmarker}{}%
\end{pgfscope}%
\begin{pgfscope}%
\pgfsys@transformshift{2.311879in}{0.952344in}%
\pgfsys@useobject{currentmarker}{}%
\end{pgfscope}%
\begin{pgfscope}%
\pgfsys@transformshift{2.311879in}{0.952344in}%
\pgfsys@useobject{currentmarker}{}%
\end{pgfscope}%
\begin{pgfscope}%
\pgfsys@transformshift{2.311879in}{0.952344in}%
\pgfsys@useobject{currentmarker}{}%
\end{pgfscope}%
\begin{pgfscope}%
\pgfsys@transformshift{2.311879in}{0.952344in}%
\pgfsys@useobject{currentmarker}{}%
\end{pgfscope}%
\begin{pgfscope}%
\pgfsys@transformshift{2.311879in}{0.952344in}%
\pgfsys@useobject{currentmarker}{}%
\end{pgfscope}%
\begin{pgfscope}%
\pgfsys@transformshift{2.311879in}{0.952344in}%
\pgfsys@useobject{currentmarker}{}%
\end{pgfscope}%
\begin{pgfscope}%
\pgfsys@transformshift{2.311879in}{0.952344in}%
\pgfsys@useobject{currentmarker}{}%
\end{pgfscope}%
\begin{pgfscope}%
\pgfsys@transformshift{2.311879in}{0.952344in}%
\pgfsys@useobject{currentmarker}{}%
\end{pgfscope}%
\begin{pgfscope}%
\pgfsys@transformshift{2.311879in}{0.952344in}%
\pgfsys@useobject{currentmarker}{}%
\end{pgfscope}%
\begin{pgfscope}%
\pgfsys@transformshift{2.311879in}{0.952344in}%
\pgfsys@useobject{currentmarker}{}%
\end{pgfscope}%
\begin{pgfscope}%
\pgfsys@transformshift{2.311879in}{0.952344in}%
\pgfsys@useobject{currentmarker}{}%
\end{pgfscope}%
\begin{pgfscope}%
\pgfsys@transformshift{2.311879in}{0.952344in}%
\pgfsys@useobject{currentmarker}{}%
\end{pgfscope}%
\begin{pgfscope}%
\pgfsys@transformshift{2.311879in}{0.952344in}%
\pgfsys@useobject{currentmarker}{}%
\end{pgfscope}%
\begin{pgfscope}%
\pgfsys@transformshift{2.311879in}{0.952344in}%
\pgfsys@useobject{currentmarker}{}%
\end{pgfscope}%
\begin{pgfscope}%
\pgfsys@transformshift{2.311879in}{0.952344in}%
\pgfsys@useobject{currentmarker}{}%
\end{pgfscope}%
\begin{pgfscope}%
\pgfsys@transformshift{2.311879in}{0.952344in}%
\pgfsys@useobject{currentmarker}{}%
\end{pgfscope}%
\begin{pgfscope}%
\pgfsys@transformshift{2.311879in}{0.952344in}%
\pgfsys@useobject{currentmarker}{}%
\end{pgfscope}%
\begin{pgfscope}%
\pgfsys@transformshift{2.311879in}{0.952344in}%
\pgfsys@useobject{currentmarker}{}%
\end{pgfscope}%
\begin{pgfscope}%
\pgfsys@transformshift{2.311879in}{0.952344in}%
\pgfsys@useobject{currentmarker}{}%
\end{pgfscope}%
\begin{pgfscope}%
\pgfsys@transformshift{2.311879in}{0.952344in}%
\pgfsys@useobject{currentmarker}{}%
\end{pgfscope}%
\begin{pgfscope}%
\pgfsys@transformshift{2.311879in}{0.952344in}%
\pgfsys@useobject{currentmarker}{}%
\end{pgfscope}%
\begin{pgfscope}%
\pgfsys@transformshift{2.311879in}{0.952344in}%
\pgfsys@useobject{currentmarker}{}%
\end{pgfscope}%
\begin{pgfscope}%
\pgfsys@transformshift{2.311879in}{0.952344in}%
\pgfsys@useobject{currentmarker}{}%
\end{pgfscope}%
\begin{pgfscope}%
\pgfsys@transformshift{2.311879in}{0.952344in}%
\pgfsys@useobject{currentmarker}{}%
\end{pgfscope}%
\begin{pgfscope}%
\pgfsys@transformshift{2.311879in}{0.952344in}%
\pgfsys@useobject{currentmarker}{}%
\end{pgfscope}%
\begin{pgfscope}%
\pgfsys@transformshift{2.311879in}{0.952344in}%
\pgfsys@useobject{currentmarker}{}%
\end{pgfscope}%
\begin{pgfscope}%
\pgfsys@transformshift{2.311879in}{0.952344in}%
\pgfsys@useobject{currentmarker}{}%
\end{pgfscope}%
\begin{pgfscope}%
\pgfsys@transformshift{2.311879in}{0.952344in}%
\pgfsys@useobject{currentmarker}{}%
\end{pgfscope}%
\begin{pgfscope}%
\pgfsys@transformshift{2.311879in}{0.952344in}%
\pgfsys@useobject{currentmarker}{}%
\end{pgfscope}%
\begin{pgfscope}%
\pgfsys@transformshift{2.311879in}{0.952344in}%
\pgfsys@useobject{currentmarker}{}%
\end{pgfscope}%
\begin{pgfscope}%
\pgfsys@transformshift{2.311879in}{0.952344in}%
\pgfsys@useobject{currentmarker}{}%
\end{pgfscope}%
\begin{pgfscope}%
\pgfsys@transformshift{2.311879in}{0.952344in}%
\pgfsys@useobject{currentmarker}{}%
\end{pgfscope}%
\begin{pgfscope}%
\pgfsys@transformshift{2.311879in}{0.952344in}%
\pgfsys@useobject{currentmarker}{}%
\end{pgfscope}%
\begin{pgfscope}%
\pgfsys@transformshift{2.311879in}{0.952344in}%
\pgfsys@useobject{currentmarker}{}%
\end{pgfscope}%
\begin{pgfscope}%
\pgfsys@transformshift{2.311879in}{0.952344in}%
\pgfsys@useobject{currentmarker}{}%
\end{pgfscope}%
\begin{pgfscope}%
\pgfsys@transformshift{2.311879in}{0.952344in}%
\pgfsys@useobject{currentmarker}{}%
\end{pgfscope}%
\begin{pgfscope}%
\pgfsys@transformshift{2.311879in}{0.952344in}%
\pgfsys@useobject{currentmarker}{}%
\end{pgfscope}%
\begin{pgfscope}%
\pgfsys@transformshift{2.311879in}{0.952344in}%
\pgfsys@useobject{currentmarker}{}%
\end{pgfscope}%
\begin{pgfscope}%
\pgfsys@transformshift{2.311879in}{0.952344in}%
\pgfsys@useobject{currentmarker}{}%
\end{pgfscope}%
\begin{pgfscope}%
\pgfsys@transformshift{2.311879in}{0.952344in}%
\pgfsys@useobject{currentmarker}{}%
\end{pgfscope}%
\begin{pgfscope}%
\pgfsys@transformshift{2.311879in}{0.952344in}%
\pgfsys@useobject{currentmarker}{}%
\end{pgfscope}%
\begin{pgfscope}%
\pgfsys@transformshift{2.311879in}{0.952344in}%
\pgfsys@useobject{currentmarker}{}%
\end{pgfscope}%
\begin{pgfscope}%
\pgfsys@transformshift{2.311879in}{0.952344in}%
\pgfsys@useobject{currentmarker}{}%
\end{pgfscope}%
\begin{pgfscope}%
\pgfsys@transformshift{2.311879in}{0.952344in}%
\pgfsys@useobject{currentmarker}{}%
\end{pgfscope}%
\begin{pgfscope}%
\pgfsys@transformshift{2.311879in}{0.952344in}%
\pgfsys@useobject{currentmarker}{}%
\end{pgfscope}%
\begin{pgfscope}%
\pgfsys@transformshift{2.311879in}{0.952344in}%
\pgfsys@useobject{currentmarker}{}%
\end{pgfscope}%
\begin{pgfscope}%
\pgfsys@transformshift{2.311879in}{0.952344in}%
\pgfsys@useobject{currentmarker}{}%
\end{pgfscope}%
\begin{pgfscope}%
\pgfsys@transformshift{2.311879in}{0.952344in}%
\pgfsys@useobject{currentmarker}{}%
\end{pgfscope}%
\begin{pgfscope}%
\pgfsys@transformshift{2.311879in}{0.952344in}%
\pgfsys@useobject{currentmarker}{}%
\end{pgfscope}%
\begin{pgfscope}%
\pgfsys@transformshift{2.311879in}{0.952344in}%
\pgfsys@useobject{currentmarker}{}%
\end{pgfscope}%
\begin{pgfscope}%
\pgfsys@transformshift{2.311879in}{0.952344in}%
\pgfsys@useobject{currentmarker}{}%
\end{pgfscope}%
\begin{pgfscope}%
\pgfsys@transformshift{2.311879in}{0.952344in}%
\pgfsys@useobject{currentmarker}{}%
\end{pgfscope}%
\begin{pgfscope}%
\pgfsys@transformshift{2.311879in}{0.952344in}%
\pgfsys@useobject{currentmarker}{}%
\end{pgfscope}%
\begin{pgfscope}%
\pgfsys@transformshift{2.311879in}{0.952344in}%
\pgfsys@useobject{currentmarker}{}%
\end{pgfscope}%
\begin{pgfscope}%
\pgfsys@transformshift{2.311879in}{0.952344in}%
\pgfsys@useobject{currentmarker}{}%
\end{pgfscope}%
\begin{pgfscope}%
\pgfsys@transformshift{2.311879in}{0.952344in}%
\pgfsys@useobject{currentmarker}{}%
\end{pgfscope}%
\begin{pgfscope}%
\pgfsys@transformshift{2.311879in}{0.952344in}%
\pgfsys@useobject{currentmarker}{}%
\end{pgfscope}%
\begin{pgfscope}%
\pgfsys@transformshift{2.311879in}{0.952344in}%
\pgfsys@useobject{currentmarker}{}%
\end{pgfscope}%
\begin{pgfscope}%
\pgfsys@transformshift{2.311879in}{0.952344in}%
\pgfsys@useobject{currentmarker}{}%
\end{pgfscope}%
\begin{pgfscope}%
\pgfsys@transformshift{2.311879in}{0.952344in}%
\pgfsys@useobject{currentmarker}{}%
\end{pgfscope}%
\begin{pgfscope}%
\pgfsys@transformshift{2.311879in}{0.952344in}%
\pgfsys@useobject{currentmarker}{}%
\end{pgfscope}%
\begin{pgfscope}%
\pgfsys@transformshift{2.311879in}{0.952344in}%
\pgfsys@useobject{currentmarker}{}%
\end{pgfscope}%
\begin{pgfscope}%
\pgfsys@transformshift{2.311879in}{0.952344in}%
\pgfsys@useobject{currentmarker}{}%
\end{pgfscope}%
\begin{pgfscope}%
\pgfsys@transformshift{2.311879in}{0.952344in}%
\pgfsys@useobject{currentmarker}{}%
\end{pgfscope}%
\begin{pgfscope}%
\pgfsys@transformshift{2.311879in}{0.952344in}%
\pgfsys@useobject{currentmarker}{}%
\end{pgfscope}%
\begin{pgfscope}%
\pgfsys@transformshift{2.311879in}{0.952344in}%
\pgfsys@useobject{currentmarker}{}%
\end{pgfscope}%
\begin{pgfscope}%
\pgfsys@transformshift{2.311879in}{0.952344in}%
\pgfsys@useobject{currentmarker}{}%
\end{pgfscope}%
\begin{pgfscope}%
\pgfsys@transformshift{2.311879in}{0.952344in}%
\pgfsys@useobject{currentmarker}{}%
\end{pgfscope}%
\begin{pgfscope}%
\pgfsys@transformshift{2.311879in}{0.952344in}%
\pgfsys@useobject{currentmarker}{}%
\end{pgfscope}%
\begin{pgfscope}%
\pgfsys@transformshift{2.311879in}{0.952344in}%
\pgfsys@useobject{currentmarker}{}%
\end{pgfscope}%
\begin{pgfscope}%
\pgfsys@transformshift{2.311879in}{0.952344in}%
\pgfsys@useobject{currentmarker}{}%
\end{pgfscope}%
\begin{pgfscope}%
\pgfsys@transformshift{2.311879in}{0.952344in}%
\pgfsys@useobject{currentmarker}{}%
\end{pgfscope}%
\begin{pgfscope}%
\pgfsys@transformshift{2.311879in}{0.952344in}%
\pgfsys@useobject{currentmarker}{}%
\end{pgfscope}%
\begin{pgfscope}%
\pgfsys@transformshift{2.311879in}{0.952344in}%
\pgfsys@useobject{currentmarker}{}%
\end{pgfscope}%
\begin{pgfscope}%
\pgfsys@transformshift{2.311879in}{0.952344in}%
\pgfsys@useobject{currentmarker}{}%
\end{pgfscope}%
\begin{pgfscope}%
\pgfsys@transformshift{2.311879in}{0.952344in}%
\pgfsys@useobject{currentmarker}{}%
\end{pgfscope}%
\begin{pgfscope}%
\pgfsys@transformshift{2.311879in}{0.952344in}%
\pgfsys@useobject{currentmarker}{}%
\end{pgfscope}%
\begin{pgfscope}%
\pgfsys@transformshift{2.311879in}{0.952344in}%
\pgfsys@useobject{currentmarker}{}%
\end{pgfscope}%
\begin{pgfscope}%
\pgfsys@transformshift{2.311879in}{0.952344in}%
\pgfsys@useobject{currentmarker}{}%
\end{pgfscope}%
\begin{pgfscope}%
\pgfsys@transformshift{2.311879in}{0.952344in}%
\pgfsys@useobject{currentmarker}{}%
\end{pgfscope}%
\begin{pgfscope}%
\pgfsys@transformshift{2.311879in}{0.952344in}%
\pgfsys@useobject{currentmarker}{}%
\end{pgfscope}%
\begin{pgfscope}%
\pgfsys@transformshift{2.311879in}{0.952344in}%
\pgfsys@useobject{currentmarker}{}%
\end{pgfscope}%
\begin{pgfscope}%
\pgfsys@transformshift{2.311879in}{0.952344in}%
\pgfsys@useobject{currentmarker}{}%
\end{pgfscope}%
\begin{pgfscope}%
\pgfsys@transformshift{2.311879in}{0.952344in}%
\pgfsys@useobject{currentmarker}{}%
\end{pgfscope}%
\begin{pgfscope}%
\pgfsys@transformshift{2.311879in}{0.952344in}%
\pgfsys@useobject{currentmarker}{}%
\end{pgfscope}%
\begin{pgfscope}%
\pgfsys@transformshift{2.311879in}{0.952344in}%
\pgfsys@useobject{currentmarker}{}%
\end{pgfscope}%
\begin{pgfscope}%
\pgfsys@transformshift{2.311879in}{0.952344in}%
\pgfsys@useobject{currentmarker}{}%
\end{pgfscope}%
\begin{pgfscope}%
\pgfsys@transformshift{2.311879in}{0.952344in}%
\pgfsys@useobject{currentmarker}{}%
\end{pgfscope}%
\begin{pgfscope}%
\pgfsys@transformshift{2.311879in}{0.952344in}%
\pgfsys@useobject{currentmarker}{}%
\end{pgfscope}%
\begin{pgfscope}%
\pgfsys@transformshift{2.311879in}{0.952344in}%
\pgfsys@useobject{currentmarker}{}%
\end{pgfscope}%
\begin{pgfscope}%
\pgfsys@transformshift{2.311879in}{0.952344in}%
\pgfsys@useobject{currentmarker}{}%
\end{pgfscope}%
\begin{pgfscope}%
\pgfsys@transformshift{2.311879in}{0.952344in}%
\pgfsys@useobject{currentmarker}{}%
\end{pgfscope}%
\begin{pgfscope}%
\pgfsys@transformshift{2.311879in}{0.952344in}%
\pgfsys@useobject{currentmarker}{}%
\end{pgfscope}%
\begin{pgfscope}%
\pgfsys@transformshift{2.311879in}{0.952344in}%
\pgfsys@useobject{currentmarker}{}%
\end{pgfscope}%
\begin{pgfscope}%
\pgfsys@transformshift{2.311879in}{0.952344in}%
\pgfsys@useobject{currentmarker}{}%
\end{pgfscope}%
\begin{pgfscope}%
\pgfsys@transformshift{2.311879in}{0.952344in}%
\pgfsys@useobject{currentmarker}{}%
\end{pgfscope}%
\begin{pgfscope}%
\pgfsys@transformshift{2.311879in}{0.952344in}%
\pgfsys@useobject{currentmarker}{}%
\end{pgfscope}%
\begin{pgfscope}%
\pgfsys@transformshift{2.311879in}{0.952344in}%
\pgfsys@useobject{currentmarker}{}%
\end{pgfscope}%
\begin{pgfscope}%
\pgfsys@transformshift{2.311879in}{0.952344in}%
\pgfsys@useobject{currentmarker}{}%
\end{pgfscope}%
\begin{pgfscope}%
\pgfsys@transformshift{2.311879in}{0.952344in}%
\pgfsys@useobject{currentmarker}{}%
\end{pgfscope}%
\begin{pgfscope}%
\pgfsys@transformshift{2.311879in}{0.952344in}%
\pgfsys@useobject{currentmarker}{}%
\end{pgfscope}%
\begin{pgfscope}%
\pgfsys@transformshift{2.311879in}{0.952344in}%
\pgfsys@useobject{currentmarker}{}%
\end{pgfscope}%
\begin{pgfscope}%
\pgfsys@transformshift{2.311879in}{0.952344in}%
\pgfsys@useobject{currentmarker}{}%
\end{pgfscope}%
\begin{pgfscope}%
\pgfsys@transformshift{2.311879in}{0.952344in}%
\pgfsys@useobject{currentmarker}{}%
\end{pgfscope}%
\begin{pgfscope}%
\pgfsys@transformshift{2.311879in}{0.952344in}%
\pgfsys@useobject{currentmarker}{}%
\end{pgfscope}%
\begin{pgfscope}%
\pgfsys@transformshift{2.311879in}{0.952344in}%
\pgfsys@useobject{currentmarker}{}%
\end{pgfscope}%
\begin{pgfscope}%
\pgfsys@transformshift{2.311879in}{0.952344in}%
\pgfsys@useobject{currentmarker}{}%
\end{pgfscope}%
\begin{pgfscope}%
\pgfsys@transformshift{2.311879in}{0.952344in}%
\pgfsys@useobject{currentmarker}{}%
\end{pgfscope}%
\begin{pgfscope}%
\pgfsys@transformshift{2.311879in}{0.952344in}%
\pgfsys@useobject{currentmarker}{}%
\end{pgfscope}%
\begin{pgfscope}%
\pgfsys@transformshift{2.311879in}{0.952344in}%
\pgfsys@useobject{currentmarker}{}%
\end{pgfscope}%
\begin{pgfscope}%
\pgfsys@transformshift{2.311879in}{0.952344in}%
\pgfsys@useobject{currentmarker}{}%
\end{pgfscope}%
\begin{pgfscope}%
\pgfsys@transformshift{2.311879in}{0.952344in}%
\pgfsys@useobject{currentmarker}{}%
\end{pgfscope}%
\begin{pgfscope}%
\pgfsys@transformshift{2.311879in}{0.952344in}%
\pgfsys@useobject{currentmarker}{}%
\end{pgfscope}%
\begin{pgfscope}%
\pgfsys@transformshift{2.311879in}{0.952344in}%
\pgfsys@useobject{currentmarker}{}%
\end{pgfscope}%
\begin{pgfscope}%
\pgfsys@transformshift{2.311879in}{0.952344in}%
\pgfsys@useobject{currentmarker}{}%
\end{pgfscope}%
\begin{pgfscope}%
\pgfsys@transformshift{2.311879in}{0.952344in}%
\pgfsys@useobject{currentmarker}{}%
\end{pgfscope}%
\begin{pgfscope}%
\pgfsys@transformshift{2.311879in}{0.952344in}%
\pgfsys@useobject{currentmarker}{}%
\end{pgfscope}%
\begin{pgfscope}%
\pgfsys@transformshift{2.311879in}{0.952344in}%
\pgfsys@useobject{currentmarker}{}%
\end{pgfscope}%
\begin{pgfscope}%
\pgfsys@transformshift{2.311879in}{0.952344in}%
\pgfsys@useobject{currentmarker}{}%
\end{pgfscope}%
\begin{pgfscope}%
\pgfsys@transformshift{2.311879in}{0.952344in}%
\pgfsys@useobject{currentmarker}{}%
\end{pgfscope}%
\begin{pgfscope}%
\pgfsys@transformshift{2.311879in}{0.952344in}%
\pgfsys@useobject{currentmarker}{}%
\end{pgfscope}%
\begin{pgfscope}%
\pgfsys@transformshift{2.311879in}{0.952344in}%
\pgfsys@useobject{currentmarker}{}%
\end{pgfscope}%
\begin{pgfscope}%
\pgfsys@transformshift{2.311879in}{0.952344in}%
\pgfsys@useobject{currentmarker}{}%
\end{pgfscope}%
\begin{pgfscope}%
\pgfsys@transformshift{2.311879in}{0.952344in}%
\pgfsys@useobject{currentmarker}{}%
\end{pgfscope}%
\begin{pgfscope}%
\pgfsys@transformshift{2.311879in}{0.952344in}%
\pgfsys@useobject{currentmarker}{}%
\end{pgfscope}%
\begin{pgfscope}%
\pgfsys@transformshift{2.311879in}{0.952344in}%
\pgfsys@useobject{currentmarker}{}%
\end{pgfscope}%
\begin{pgfscope}%
\pgfsys@transformshift{2.311879in}{0.952344in}%
\pgfsys@useobject{currentmarker}{}%
\end{pgfscope}%
\begin{pgfscope}%
\pgfsys@transformshift{2.311879in}{0.952344in}%
\pgfsys@useobject{currentmarker}{}%
\end{pgfscope}%
\begin{pgfscope}%
\pgfsys@transformshift{2.311879in}{0.952344in}%
\pgfsys@useobject{currentmarker}{}%
\end{pgfscope}%
\begin{pgfscope}%
\pgfsys@transformshift{2.311879in}{0.952344in}%
\pgfsys@useobject{currentmarker}{}%
\end{pgfscope}%
\begin{pgfscope}%
\pgfsys@transformshift{2.311879in}{0.952344in}%
\pgfsys@useobject{currentmarker}{}%
\end{pgfscope}%
\begin{pgfscope}%
\pgfsys@transformshift{2.311879in}{0.952344in}%
\pgfsys@useobject{currentmarker}{}%
\end{pgfscope}%
\begin{pgfscope}%
\pgfsys@transformshift{2.311879in}{0.952344in}%
\pgfsys@useobject{currentmarker}{}%
\end{pgfscope}%
\begin{pgfscope}%
\pgfsys@transformshift{2.311879in}{0.952344in}%
\pgfsys@useobject{currentmarker}{}%
\end{pgfscope}%
\begin{pgfscope}%
\pgfsys@transformshift{2.311879in}{0.952344in}%
\pgfsys@useobject{currentmarker}{}%
\end{pgfscope}%
\begin{pgfscope}%
\pgfsys@transformshift{2.311879in}{0.952344in}%
\pgfsys@useobject{currentmarker}{}%
\end{pgfscope}%
\begin{pgfscope}%
\pgfsys@transformshift{2.311879in}{0.952344in}%
\pgfsys@useobject{currentmarker}{}%
\end{pgfscope}%
\begin{pgfscope}%
\pgfsys@transformshift{2.311879in}{0.952344in}%
\pgfsys@useobject{currentmarker}{}%
\end{pgfscope}%
\begin{pgfscope}%
\pgfsys@transformshift{2.311879in}{0.952344in}%
\pgfsys@useobject{currentmarker}{}%
\end{pgfscope}%
\begin{pgfscope}%
\pgfsys@transformshift{2.311879in}{0.952344in}%
\pgfsys@useobject{currentmarker}{}%
\end{pgfscope}%
\begin{pgfscope}%
\pgfsys@transformshift{2.311879in}{0.952344in}%
\pgfsys@useobject{currentmarker}{}%
\end{pgfscope}%
\begin{pgfscope}%
\pgfsys@transformshift{2.311879in}{0.952344in}%
\pgfsys@useobject{currentmarker}{}%
\end{pgfscope}%
\begin{pgfscope}%
\pgfsys@transformshift{2.311879in}{0.952344in}%
\pgfsys@useobject{currentmarker}{}%
\end{pgfscope}%
\begin{pgfscope}%
\pgfsys@transformshift{2.311879in}{0.952344in}%
\pgfsys@useobject{currentmarker}{}%
\end{pgfscope}%
\begin{pgfscope}%
\pgfsys@transformshift{2.311879in}{0.952344in}%
\pgfsys@useobject{currentmarker}{}%
\end{pgfscope}%
\begin{pgfscope}%
\pgfsys@transformshift{2.311879in}{0.952344in}%
\pgfsys@useobject{currentmarker}{}%
\end{pgfscope}%
\begin{pgfscope}%
\pgfsys@transformshift{2.311879in}{0.952344in}%
\pgfsys@useobject{currentmarker}{}%
\end{pgfscope}%
\begin{pgfscope}%
\pgfsys@transformshift{2.311879in}{0.952344in}%
\pgfsys@useobject{currentmarker}{}%
\end{pgfscope}%
\begin{pgfscope}%
\pgfsys@transformshift{2.311879in}{0.952344in}%
\pgfsys@useobject{currentmarker}{}%
\end{pgfscope}%
\begin{pgfscope}%
\pgfsys@transformshift{2.311879in}{0.952344in}%
\pgfsys@useobject{currentmarker}{}%
\end{pgfscope}%
\begin{pgfscope}%
\pgfsys@transformshift{2.311879in}{0.952344in}%
\pgfsys@useobject{currentmarker}{}%
\end{pgfscope}%
\begin{pgfscope}%
\pgfsys@transformshift{2.311879in}{0.952344in}%
\pgfsys@useobject{currentmarker}{}%
\end{pgfscope}%
\begin{pgfscope}%
\pgfsys@transformshift{2.311879in}{0.952344in}%
\pgfsys@useobject{currentmarker}{}%
\end{pgfscope}%
\begin{pgfscope}%
\pgfsys@transformshift{2.311879in}{0.952344in}%
\pgfsys@useobject{currentmarker}{}%
\end{pgfscope}%
\begin{pgfscope}%
\pgfsys@transformshift{2.311879in}{0.952344in}%
\pgfsys@useobject{currentmarker}{}%
\end{pgfscope}%
\begin{pgfscope}%
\pgfsys@transformshift{2.311879in}{0.952344in}%
\pgfsys@useobject{currentmarker}{}%
\end{pgfscope}%
\begin{pgfscope}%
\pgfsys@transformshift{2.311879in}{0.952344in}%
\pgfsys@useobject{currentmarker}{}%
\end{pgfscope}%
\begin{pgfscope}%
\pgfsys@transformshift{2.311879in}{0.952344in}%
\pgfsys@useobject{currentmarker}{}%
\end{pgfscope}%
\begin{pgfscope}%
\pgfsys@transformshift{2.311879in}{0.952344in}%
\pgfsys@useobject{currentmarker}{}%
\end{pgfscope}%
\begin{pgfscope}%
\pgfsys@transformshift{2.311879in}{0.952344in}%
\pgfsys@useobject{currentmarker}{}%
\end{pgfscope}%
\begin{pgfscope}%
\pgfsys@transformshift{2.311879in}{0.952344in}%
\pgfsys@useobject{currentmarker}{}%
\end{pgfscope}%
\begin{pgfscope}%
\pgfsys@transformshift{2.311879in}{0.952344in}%
\pgfsys@useobject{currentmarker}{}%
\end{pgfscope}%
\begin{pgfscope}%
\pgfsys@transformshift{2.311879in}{0.952344in}%
\pgfsys@useobject{currentmarker}{}%
\end{pgfscope}%
\begin{pgfscope}%
\pgfsys@transformshift{2.311879in}{0.952344in}%
\pgfsys@useobject{currentmarker}{}%
\end{pgfscope}%
\begin{pgfscope}%
\pgfsys@transformshift{2.311879in}{0.952344in}%
\pgfsys@useobject{currentmarker}{}%
\end{pgfscope}%
\begin{pgfscope}%
\pgfsys@transformshift{2.311879in}{0.952344in}%
\pgfsys@useobject{currentmarker}{}%
\end{pgfscope}%
\begin{pgfscope}%
\pgfsys@transformshift{2.311879in}{0.952344in}%
\pgfsys@useobject{currentmarker}{}%
\end{pgfscope}%
\begin{pgfscope}%
\pgfsys@transformshift{2.311879in}{0.952344in}%
\pgfsys@useobject{currentmarker}{}%
\end{pgfscope}%
\begin{pgfscope}%
\pgfsys@transformshift{2.311879in}{0.952344in}%
\pgfsys@useobject{currentmarker}{}%
\end{pgfscope}%
\begin{pgfscope}%
\pgfsys@transformshift{2.311879in}{0.952344in}%
\pgfsys@useobject{currentmarker}{}%
\end{pgfscope}%
\begin{pgfscope}%
\pgfsys@transformshift{2.311879in}{0.952344in}%
\pgfsys@useobject{currentmarker}{}%
\end{pgfscope}%
\begin{pgfscope}%
\pgfsys@transformshift{2.311879in}{0.952344in}%
\pgfsys@useobject{currentmarker}{}%
\end{pgfscope}%
\begin{pgfscope}%
\pgfsys@transformshift{2.311879in}{0.952344in}%
\pgfsys@useobject{currentmarker}{}%
\end{pgfscope}%
\begin{pgfscope}%
\pgfsys@transformshift{2.311879in}{0.952344in}%
\pgfsys@useobject{currentmarker}{}%
\end{pgfscope}%
\begin{pgfscope}%
\pgfsys@transformshift{2.311879in}{0.952344in}%
\pgfsys@useobject{currentmarker}{}%
\end{pgfscope}%
\begin{pgfscope}%
\pgfsys@transformshift{2.311879in}{0.952344in}%
\pgfsys@useobject{currentmarker}{}%
\end{pgfscope}%
\begin{pgfscope}%
\pgfsys@transformshift{2.311879in}{0.952344in}%
\pgfsys@useobject{currentmarker}{}%
\end{pgfscope}%
\begin{pgfscope}%
\pgfsys@transformshift{2.311879in}{0.952344in}%
\pgfsys@useobject{currentmarker}{}%
\end{pgfscope}%
\begin{pgfscope}%
\pgfsys@transformshift{2.311879in}{0.952344in}%
\pgfsys@useobject{currentmarker}{}%
\end{pgfscope}%
\begin{pgfscope}%
\pgfsys@transformshift{2.311879in}{0.952344in}%
\pgfsys@useobject{currentmarker}{}%
\end{pgfscope}%
\begin{pgfscope}%
\pgfsys@transformshift{2.311879in}{0.952344in}%
\pgfsys@useobject{currentmarker}{}%
\end{pgfscope}%
\begin{pgfscope}%
\pgfsys@transformshift{2.311879in}{0.952344in}%
\pgfsys@useobject{currentmarker}{}%
\end{pgfscope}%
\begin{pgfscope}%
\pgfsys@transformshift{2.311879in}{0.952344in}%
\pgfsys@useobject{currentmarker}{}%
\end{pgfscope}%
\begin{pgfscope}%
\pgfsys@transformshift{2.311879in}{0.952344in}%
\pgfsys@useobject{currentmarker}{}%
\end{pgfscope}%
\begin{pgfscope}%
\pgfsys@transformshift{2.311879in}{0.952344in}%
\pgfsys@useobject{currentmarker}{}%
\end{pgfscope}%
\begin{pgfscope}%
\pgfsys@transformshift{2.311879in}{0.952344in}%
\pgfsys@useobject{currentmarker}{}%
\end{pgfscope}%
\begin{pgfscope}%
\pgfsys@transformshift{2.311879in}{0.952344in}%
\pgfsys@useobject{currentmarker}{}%
\end{pgfscope}%
\begin{pgfscope}%
\pgfsys@transformshift{2.311879in}{0.952344in}%
\pgfsys@useobject{currentmarker}{}%
\end{pgfscope}%
\begin{pgfscope}%
\pgfsys@transformshift{2.311879in}{0.952344in}%
\pgfsys@useobject{currentmarker}{}%
\end{pgfscope}%
\begin{pgfscope}%
\pgfsys@transformshift{2.311879in}{0.952344in}%
\pgfsys@useobject{currentmarker}{}%
\end{pgfscope}%
\begin{pgfscope}%
\pgfsys@transformshift{2.311879in}{0.952344in}%
\pgfsys@useobject{currentmarker}{}%
\end{pgfscope}%
\begin{pgfscope}%
\pgfsys@transformshift{2.311879in}{0.952344in}%
\pgfsys@useobject{currentmarker}{}%
\end{pgfscope}%
\begin{pgfscope}%
\pgfsys@transformshift{2.311879in}{0.952344in}%
\pgfsys@useobject{currentmarker}{}%
\end{pgfscope}%
\begin{pgfscope}%
\pgfsys@transformshift{2.311879in}{0.952344in}%
\pgfsys@useobject{currentmarker}{}%
\end{pgfscope}%
\begin{pgfscope}%
\pgfsys@transformshift{2.311879in}{0.952344in}%
\pgfsys@useobject{currentmarker}{}%
\end{pgfscope}%
\begin{pgfscope}%
\pgfsys@transformshift{2.311879in}{0.952344in}%
\pgfsys@useobject{currentmarker}{}%
\end{pgfscope}%
\begin{pgfscope}%
\pgfsys@transformshift{2.311879in}{0.952344in}%
\pgfsys@useobject{currentmarker}{}%
\end{pgfscope}%
\begin{pgfscope}%
\pgfsys@transformshift{2.311879in}{0.952344in}%
\pgfsys@useobject{currentmarker}{}%
\end{pgfscope}%
\begin{pgfscope}%
\pgfsys@transformshift{2.311879in}{0.952344in}%
\pgfsys@useobject{currentmarker}{}%
\end{pgfscope}%
\begin{pgfscope}%
\pgfsys@transformshift{2.311879in}{0.952344in}%
\pgfsys@useobject{currentmarker}{}%
\end{pgfscope}%
\begin{pgfscope}%
\pgfsys@transformshift{2.311879in}{0.952344in}%
\pgfsys@useobject{currentmarker}{}%
\end{pgfscope}%
\begin{pgfscope}%
\pgfsys@transformshift{2.311879in}{0.952344in}%
\pgfsys@useobject{currentmarker}{}%
\end{pgfscope}%
\begin{pgfscope}%
\pgfsys@transformshift{2.311879in}{0.952344in}%
\pgfsys@useobject{currentmarker}{}%
\end{pgfscope}%
\begin{pgfscope}%
\pgfsys@transformshift{2.311879in}{0.952344in}%
\pgfsys@useobject{currentmarker}{}%
\end{pgfscope}%
\begin{pgfscope}%
\pgfsys@transformshift{2.311879in}{0.952344in}%
\pgfsys@useobject{currentmarker}{}%
\end{pgfscope}%
\begin{pgfscope}%
\pgfsys@transformshift{2.311879in}{0.952344in}%
\pgfsys@useobject{currentmarker}{}%
\end{pgfscope}%
\begin{pgfscope}%
\pgfsys@transformshift{2.311879in}{0.952344in}%
\pgfsys@useobject{currentmarker}{}%
\end{pgfscope}%
\begin{pgfscope}%
\pgfsys@transformshift{2.311879in}{0.952344in}%
\pgfsys@useobject{currentmarker}{}%
\end{pgfscope}%
\begin{pgfscope}%
\pgfsys@transformshift{2.311879in}{0.952344in}%
\pgfsys@useobject{currentmarker}{}%
\end{pgfscope}%
\begin{pgfscope}%
\pgfsys@transformshift{2.311879in}{0.952344in}%
\pgfsys@useobject{currentmarker}{}%
\end{pgfscope}%
\begin{pgfscope}%
\pgfsys@transformshift{2.311879in}{0.952344in}%
\pgfsys@useobject{currentmarker}{}%
\end{pgfscope}%
\begin{pgfscope}%
\pgfsys@transformshift{2.311879in}{0.952344in}%
\pgfsys@useobject{currentmarker}{}%
\end{pgfscope}%
\begin{pgfscope}%
\pgfsys@transformshift{2.311879in}{0.952344in}%
\pgfsys@useobject{currentmarker}{}%
\end{pgfscope}%
\begin{pgfscope}%
\pgfsys@transformshift{2.311879in}{0.952344in}%
\pgfsys@useobject{currentmarker}{}%
\end{pgfscope}%
\begin{pgfscope}%
\pgfsys@transformshift{2.311879in}{0.952344in}%
\pgfsys@useobject{currentmarker}{}%
\end{pgfscope}%
\begin{pgfscope}%
\pgfsys@transformshift{2.311879in}{0.952344in}%
\pgfsys@useobject{currentmarker}{}%
\end{pgfscope}%
\begin{pgfscope}%
\pgfsys@transformshift{2.311879in}{0.952344in}%
\pgfsys@useobject{currentmarker}{}%
\end{pgfscope}%
\begin{pgfscope}%
\pgfsys@transformshift{2.311879in}{0.952344in}%
\pgfsys@useobject{currentmarker}{}%
\end{pgfscope}%
\begin{pgfscope}%
\pgfsys@transformshift{2.311879in}{0.952344in}%
\pgfsys@useobject{currentmarker}{}%
\end{pgfscope}%
\begin{pgfscope}%
\pgfsys@transformshift{2.311879in}{0.952344in}%
\pgfsys@useobject{currentmarker}{}%
\end{pgfscope}%
\begin{pgfscope}%
\pgfsys@transformshift{2.311879in}{0.952344in}%
\pgfsys@useobject{currentmarker}{}%
\end{pgfscope}%
\begin{pgfscope}%
\pgfsys@transformshift{2.311879in}{0.952344in}%
\pgfsys@useobject{currentmarker}{}%
\end{pgfscope}%
\begin{pgfscope}%
\pgfsys@transformshift{2.311879in}{0.952344in}%
\pgfsys@useobject{currentmarker}{}%
\end{pgfscope}%
\begin{pgfscope}%
\pgfsys@transformshift{2.311879in}{0.952344in}%
\pgfsys@useobject{currentmarker}{}%
\end{pgfscope}%
\begin{pgfscope}%
\pgfsys@transformshift{2.311879in}{0.952344in}%
\pgfsys@useobject{currentmarker}{}%
\end{pgfscope}%
\begin{pgfscope}%
\pgfsys@transformshift{2.311879in}{0.952344in}%
\pgfsys@useobject{currentmarker}{}%
\end{pgfscope}%
\begin{pgfscope}%
\pgfsys@transformshift{2.311879in}{0.952344in}%
\pgfsys@useobject{currentmarker}{}%
\end{pgfscope}%
\begin{pgfscope}%
\pgfsys@transformshift{2.311879in}{0.952344in}%
\pgfsys@useobject{currentmarker}{}%
\end{pgfscope}%
\begin{pgfscope}%
\pgfsys@transformshift{2.311879in}{0.952344in}%
\pgfsys@useobject{currentmarker}{}%
\end{pgfscope}%
\begin{pgfscope}%
\pgfsys@transformshift{2.311879in}{0.952344in}%
\pgfsys@useobject{currentmarker}{}%
\end{pgfscope}%
\begin{pgfscope}%
\pgfsys@transformshift{2.311879in}{0.952344in}%
\pgfsys@useobject{currentmarker}{}%
\end{pgfscope}%
\begin{pgfscope}%
\pgfsys@transformshift{2.311879in}{0.952344in}%
\pgfsys@useobject{currentmarker}{}%
\end{pgfscope}%
\begin{pgfscope}%
\pgfsys@transformshift{2.311879in}{0.952344in}%
\pgfsys@useobject{currentmarker}{}%
\end{pgfscope}%
\begin{pgfscope}%
\pgfsys@transformshift{2.311879in}{0.952344in}%
\pgfsys@useobject{currentmarker}{}%
\end{pgfscope}%
\begin{pgfscope}%
\pgfsys@transformshift{2.311879in}{0.952344in}%
\pgfsys@useobject{currentmarker}{}%
\end{pgfscope}%
\begin{pgfscope}%
\pgfsys@transformshift{2.311879in}{0.952344in}%
\pgfsys@useobject{currentmarker}{}%
\end{pgfscope}%
\begin{pgfscope}%
\pgfsys@transformshift{2.311879in}{0.952344in}%
\pgfsys@useobject{currentmarker}{}%
\end{pgfscope}%
\begin{pgfscope}%
\pgfsys@transformshift{2.311879in}{0.952344in}%
\pgfsys@useobject{currentmarker}{}%
\end{pgfscope}%
\begin{pgfscope}%
\pgfsys@transformshift{2.311879in}{0.952344in}%
\pgfsys@useobject{currentmarker}{}%
\end{pgfscope}%
\begin{pgfscope}%
\pgfsys@transformshift{2.311879in}{0.952344in}%
\pgfsys@useobject{currentmarker}{}%
\end{pgfscope}%
\begin{pgfscope}%
\pgfsys@transformshift{2.311879in}{0.952344in}%
\pgfsys@useobject{currentmarker}{}%
\end{pgfscope}%
\begin{pgfscope}%
\pgfsys@transformshift{2.311879in}{0.952344in}%
\pgfsys@useobject{currentmarker}{}%
\end{pgfscope}%
\begin{pgfscope}%
\pgfsys@transformshift{2.311879in}{0.952344in}%
\pgfsys@useobject{currentmarker}{}%
\end{pgfscope}%
\begin{pgfscope}%
\pgfsys@transformshift{2.311879in}{0.952344in}%
\pgfsys@useobject{currentmarker}{}%
\end{pgfscope}%
\begin{pgfscope}%
\pgfsys@transformshift{2.311879in}{0.952344in}%
\pgfsys@useobject{currentmarker}{}%
\end{pgfscope}%
\begin{pgfscope}%
\pgfsys@transformshift{2.311879in}{0.952344in}%
\pgfsys@useobject{currentmarker}{}%
\end{pgfscope}%
\begin{pgfscope}%
\pgfsys@transformshift{2.311879in}{0.952344in}%
\pgfsys@useobject{currentmarker}{}%
\end{pgfscope}%
\begin{pgfscope}%
\pgfsys@transformshift{2.311879in}{0.952344in}%
\pgfsys@useobject{currentmarker}{}%
\end{pgfscope}%
\begin{pgfscope}%
\pgfsys@transformshift{2.311879in}{0.952344in}%
\pgfsys@useobject{currentmarker}{}%
\end{pgfscope}%
\begin{pgfscope}%
\pgfsys@transformshift{2.311879in}{0.952344in}%
\pgfsys@useobject{currentmarker}{}%
\end{pgfscope}%
\begin{pgfscope}%
\pgfsys@transformshift{2.311879in}{0.952344in}%
\pgfsys@useobject{currentmarker}{}%
\end{pgfscope}%
\begin{pgfscope}%
\pgfsys@transformshift{2.311879in}{0.952344in}%
\pgfsys@useobject{currentmarker}{}%
\end{pgfscope}%
\begin{pgfscope}%
\pgfsys@transformshift{2.311879in}{0.952344in}%
\pgfsys@useobject{currentmarker}{}%
\end{pgfscope}%
\begin{pgfscope}%
\pgfsys@transformshift{2.311879in}{0.952344in}%
\pgfsys@useobject{currentmarker}{}%
\end{pgfscope}%
\begin{pgfscope}%
\pgfsys@transformshift{2.311879in}{0.952344in}%
\pgfsys@useobject{currentmarker}{}%
\end{pgfscope}%
\begin{pgfscope}%
\pgfsys@transformshift{2.311879in}{0.952344in}%
\pgfsys@useobject{currentmarker}{}%
\end{pgfscope}%
\begin{pgfscope}%
\pgfsys@transformshift{2.311879in}{0.952344in}%
\pgfsys@useobject{currentmarker}{}%
\end{pgfscope}%
\begin{pgfscope}%
\pgfsys@transformshift{2.311879in}{0.952344in}%
\pgfsys@useobject{currentmarker}{}%
\end{pgfscope}%
\begin{pgfscope}%
\pgfsys@transformshift{2.311879in}{0.952344in}%
\pgfsys@useobject{currentmarker}{}%
\end{pgfscope}%
\begin{pgfscope}%
\pgfsys@transformshift{2.311879in}{0.952344in}%
\pgfsys@useobject{currentmarker}{}%
\end{pgfscope}%
\begin{pgfscope}%
\pgfsys@transformshift{2.311879in}{0.952344in}%
\pgfsys@useobject{currentmarker}{}%
\end{pgfscope}%
\begin{pgfscope}%
\pgfsys@transformshift{2.311879in}{0.952344in}%
\pgfsys@useobject{currentmarker}{}%
\end{pgfscope}%
\begin{pgfscope}%
\pgfsys@transformshift{2.311879in}{0.952344in}%
\pgfsys@useobject{currentmarker}{}%
\end{pgfscope}%
\begin{pgfscope}%
\pgfsys@transformshift{2.311879in}{0.952344in}%
\pgfsys@useobject{currentmarker}{}%
\end{pgfscope}%
\begin{pgfscope}%
\pgfsys@transformshift{2.311879in}{0.952344in}%
\pgfsys@useobject{currentmarker}{}%
\end{pgfscope}%
\begin{pgfscope}%
\pgfsys@transformshift{2.311879in}{0.952344in}%
\pgfsys@useobject{currentmarker}{}%
\end{pgfscope}%
\begin{pgfscope}%
\pgfsys@transformshift{2.311879in}{0.952344in}%
\pgfsys@useobject{currentmarker}{}%
\end{pgfscope}%
\begin{pgfscope}%
\pgfsys@transformshift{2.311879in}{0.952344in}%
\pgfsys@useobject{currentmarker}{}%
\end{pgfscope}%
\begin{pgfscope}%
\pgfsys@transformshift{2.311879in}{0.952344in}%
\pgfsys@useobject{currentmarker}{}%
\end{pgfscope}%
\begin{pgfscope}%
\pgfsys@transformshift{2.311879in}{0.952344in}%
\pgfsys@useobject{currentmarker}{}%
\end{pgfscope}%
\begin{pgfscope}%
\pgfsys@transformshift{2.311879in}{0.952344in}%
\pgfsys@useobject{currentmarker}{}%
\end{pgfscope}%
\begin{pgfscope}%
\pgfsys@transformshift{2.311879in}{0.952344in}%
\pgfsys@useobject{currentmarker}{}%
\end{pgfscope}%
\begin{pgfscope}%
\pgfsys@transformshift{2.311879in}{0.952344in}%
\pgfsys@useobject{currentmarker}{}%
\end{pgfscope}%
\begin{pgfscope}%
\pgfsys@transformshift{2.311879in}{0.952344in}%
\pgfsys@useobject{currentmarker}{}%
\end{pgfscope}%
\begin{pgfscope}%
\pgfsys@transformshift{2.311879in}{0.952344in}%
\pgfsys@useobject{currentmarker}{}%
\end{pgfscope}%
\begin{pgfscope}%
\pgfsys@transformshift{2.311879in}{0.952344in}%
\pgfsys@useobject{currentmarker}{}%
\end{pgfscope}%
\begin{pgfscope}%
\pgfsys@transformshift{2.311879in}{0.952344in}%
\pgfsys@useobject{currentmarker}{}%
\end{pgfscope}%
\begin{pgfscope}%
\pgfsys@transformshift{2.311879in}{0.952344in}%
\pgfsys@useobject{currentmarker}{}%
\end{pgfscope}%
\begin{pgfscope}%
\pgfsys@transformshift{2.311879in}{0.952344in}%
\pgfsys@useobject{currentmarker}{}%
\end{pgfscope}%
\begin{pgfscope}%
\pgfsys@transformshift{2.311879in}{0.952344in}%
\pgfsys@useobject{currentmarker}{}%
\end{pgfscope}%
\begin{pgfscope}%
\pgfsys@transformshift{2.311879in}{0.952344in}%
\pgfsys@useobject{currentmarker}{}%
\end{pgfscope}%
\begin{pgfscope}%
\pgfsys@transformshift{2.311879in}{0.952344in}%
\pgfsys@useobject{currentmarker}{}%
\end{pgfscope}%
\begin{pgfscope}%
\pgfsys@transformshift{2.311879in}{0.952344in}%
\pgfsys@useobject{currentmarker}{}%
\end{pgfscope}%
\begin{pgfscope}%
\pgfsys@transformshift{2.311879in}{0.952344in}%
\pgfsys@useobject{currentmarker}{}%
\end{pgfscope}%
\begin{pgfscope}%
\pgfsys@transformshift{2.311879in}{0.952344in}%
\pgfsys@useobject{currentmarker}{}%
\end{pgfscope}%
\begin{pgfscope}%
\pgfsys@transformshift{2.311879in}{0.952344in}%
\pgfsys@useobject{currentmarker}{}%
\end{pgfscope}%
\begin{pgfscope}%
\pgfsys@transformshift{2.311879in}{0.952344in}%
\pgfsys@useobject{currentmarker}{}%
\end{pgfscope}%
\begin{pgfscope}%
\pgfsys@transformshift{2.311879in}{0.952344in}%
\pgfsys@useobject{currentmarker}{}%
\end{pgfscope}%
\begin{pgfscope}%
\pgfsys@transformshift{2.311879in}{0.952344in}%
\pgfsys@useobject{currentmarker}{}%
\end{pgfscope}%
\begin{pgfscope}%
\pgfsys@transformshift{2.311879in}{0.952344in}%
\pgfsys@useobject{currentmarker}{}%
\end{pgfscope}%
\begin{pgfscope}%
\pgfsys@transformshift{2.311879in}{0.952344in}%
\pgfsys@useobject{currentmarker}{}%
\end{pgfscope}%
\begin{pgfscope}%
\pgfsys@transformshift{2.311879in}{0.952344in}%
\pgfsys@useobject{currentmarker}{}%
\end{pgfscope}%
\begin{pgfscope}%
\pgfsys@transformshift{2.311879in}{0.952344in}%
\pgfsys@useobject{currentmarker}{}%
\end{pgfscope}%
\begin{pgfscope}%
\pgfsys@transformshift{2.311879in}{0.952344in}%
\pgfsys@useobject{currentmarker}{}%
\end{pgfscope}%
\begin{pgfscope}%
\pgfsys@transformshift{2.311879in}{0.952344in}%
\pgfsys@useobject{currentmarker}{}%
\end{pgfscope}%
\begin{pgfscope}%
\pgfsys@transformshift{2.311879in}{0.952344in}%
\pgfsys@useobject{currentmarker}{}%
\end{pgfscope}%
\begin{pgfscope}%
\pgfsys@transformshift{2.311879in}{0.952344in}%
\pgfsys@useobject{currentmarker}{}%
\end{pgfscope}%
\begin{pgfscope}%
\pgfsys@transformshift{2.311879in}{0.952344in}%
\pgfsys@useobject{currentmarker}{}%
\end{pgfscope}%
\begin{pgfscope}%
\pgfsys@transformshift{2.311879in}{0.952344in}%
\pgfsys@useobject{currentmarker}{}%
\end{pgfscope}%
\begin{pgfscope}%
\pgfsys@transformshift{2.311879in}{0.952344in}%
\pgfsys@useobject{currentmarker}{}%
\end{pgfscope}%
\begin{pgfscope}%
\pgfsys@transformshift{2.311879in}{0.952344in}%
\pgfsys@useobject{currentmarker}{}%
\end{pgfscope}%
\begin{pgfscope}%
\pgfsys@transformshift{2.311879in}{0.952344in}%
\pgfsys@useobject{currentmarker}{}%
\end{pgfscope}%
\begin{pgfscope}%
\pgfsys@transformshift{2.311879in}{0.952344in}%
\pgfsys@useobject{currentmarker}{}%
\end{pgfscope}%
\begin{pgfscope}%
\pgfsys@transformshift{2.311879in}{0.952344in}%
\pgfsys@useobject{currentmarker}{}%
\end{pgfscope}%
\begin{pgfscope}%
\pgfsys@transformshift{2.311879in}{0.952344in}%
\pgfsys@useobject{currentmarker}{}%
\end{pgfscope}%
\begin{pgfscope}%
\pgfsys@transformshift{2.311879in}{0.952344in}%
\pgfsys@useobject{currentmarker}{}%
\end{pgfscope}%
\begin{pgfscope}%
\pgfsys@transformshift{2.311879in}{0.952344in}%
\pgfsys@useobject{currentmarker}{}%
\end{pgfscope}%
\begin{pgfscope}%
\pgfsys@transformshift{2.311879in}{0.952344in}%
\pgfsys@useobject{currentmarker}{}%
\end{pgfscope}%
\begin{pgfscope}%
\pgfsys@transformshift{2.311879in}{0.952344in}%
\pgfsys@useobject{currentmarker}{}%
\end{pgfscope}%
\begin{pgfscope}%
\pgfsys@transformshift{2.311879in}{0.952344in}%
\pgfsys@useobject{currentmarker}{}%
\end{pgfscope}%
\begin{pgfscope}%
\pgfsys@transformshift{2.311879in}{0.952344in}%
\pgfsys@useobject{currentmarker}{}%
\end{pgfscope}%
\begin{pgfscope}%
\pgfsys@transformshift{2.311879in}{0.952344in}%
\pgfsys@useobject{currentmarker}{}%
\end{pgfscope}%
\begin{pgfscope}%
\pgfsys@transformshift{2.311879in}{0.952344in}%
\pgfsys@useobject{currentmarker}{}%
\end{pgfscope}%
\begin{pgfscope}%
\pgfsys@transformshift{2.311879in}{0.952344in}%
\pgfsys@useobject{currentmarker}{}%
\end{pgfscope}%
\begin{pgfscope}%
\pgfsys@transformshift{2.311879in}{0.952344in}%
\pgfsys@useobject{currentmarker}{}%
\end{pgfscope}%
\begin{pgfscope}%
\pgfsys@transformshift{2.311879in}{0.952344in}%
\pgfsys@useobject{currentmarker}{}%
\end{pgfscope}%
\begin{pgfscope}%
\pgfsys@transformshift{2.311879in}{0.952344in}%
\pgfsys@useobject{currentmarker}{}%
\end{pgfscope}%
\begin{pgfscope}%
\pgfsys@transformshift{2.311879in}{0.952344in}%
\pgfsys@useobject{currentmarker}{}%
\end{pgfscope}%
\begin{pgfscope}%
\pgfsys@transformshift{2.311879in}{0.952344in}%
\pgfsys@useobject{currentmarker}{}%
\end{pgfscope}%
\begin{pgfscope}%
\pgfsys@transformshift{2.311879in}{0.952344in}%
\pgfsys@useobject{currentmarker}{}%
\end{pgfscope}%
\begin{pgfscope}%
\pgfsys@transformshift{2.311879in}{0.952344in}%
\pgfsys@useobject{currentmarker}{}%
\end{pgfscope}%
\begin{pgfscope}%
\pgfsys@transformshift{2.311879in}{0.952344in}%
\pgfsys@useobject{currentmarker}{}%
\end{pgfscope}%
\begin{pgfscope}%
\pgfsys@transformshift{2.311879in}{0.952344in}%
\pgfsys@useobject{currentmarker}{}%
\end{pgfscope}%
\begin{pgfscope}%
\pgfsys@transformshift{2.311879in}{0.952344in}%
\pgfsys@useobject{currentmarker}{}%
\end{pgfscope}%
\begin{pgfscope}%
\pgfsys@transformshift{2.311879in}{0.952344in}%
\pgfsys@useobject{currentmarker}{}%
\end{pgfscope}%
\begin{pgfscope}%
\pgfsys@transformshift{2.311879in}{0.952344in}%
\pgfsys@useobject{currentmarker}{}%
\end{pgfscope}%
\begin{pgfscope}%
\pgfsys@transformshift{2.311879in}{0.952344in}%
\pgfsys@useobject{currentmarker}{}%
\end{pgfscope}%
\begin{pgfscope}%
\pgfsys@transformshift{2.311879in}{0.952344in}%
\pgfsys@useobject{currentmarker}{}%
\end{pgfscope}%
\begin{pgfscope}%
\pgfsys@transformshift{2.311879in}{0.952344in}%
\pgfsys@useobject{currentmarker}{}%
\end{pgfscope}%
\begin{pgfscope}%
\pgfsys@transformshift{2.311879in}{0.952344in}%
\pgfsys@useobject{currentmarker}{}%
\end{pgfscope}%
\begin{pgfscope}%
\pgfsys@transformshift{2.311879in}{0.952344in}%
\pgfsys@useobject{currentmarker}{}%
\end{pgfscope}%
\begin{pgfscope}%
\pgfsys@transformshift{2.311879in}{0.952344in}%
\pgfsys@useobject{currentmarker}{}%
\end{pgfscope}%
\begin{pgfscope}%
\pgfsys@transformshift{2.311879in}{0.952344in}%
\pgfsys@useobject{currentmarker}{}%
\end{pgfscope}%
\begin{pgfscope}%
\pgfsys@transformshift{2.311879in}{0.952344in}%
\pgfsys@useobject{currentmarker}{}%
\end{pgfscope}%
\begin{pgfscope}%
\pgfsys@transformshift{2.311879in}{0.952344in}%
\pgfsys@useobject{currentmarker}{}%
\end{pgfscope}%
\begin{pgfscope}%
\pgfsys@transformshift{2.311879in}{0.952344in}%
\pgfsys@useobject{currentmarker}{}%
\end{pgfscope}%
\begin{pgfscope}%
\pgfsys@transformshift{2.311879in}{0.952344in}%
\pgfsys@useobject{currentmarker}{}%
\end{pgfscope}%
\begin{pgfscope}%
\pgfsys@transformshift{2.311879in}{0.952344in}%
\pgfsys@useobject{currentmarker}{}%
\end{pgfscope}%
\begin{pgfscope}%
\pgfsys@transformshift{2.311879in}{0.952344in}%
\pgfsys@useobject{currentmarker}{}%
\end{pgfscope}%
\begin{pgfscope}%
\pgfsys@transformshift{2.311879in}{0.952344in}%
\pgfsys@useobject{currentmarker}{}%
\end{pgfscope}%
\begin{pgfscope}%
\pgfsys@transformshift{2.311879in}{0.952344in}%
\pgfsys@useobject{currentmarker}{}%
\end{pgfscope}%
\begin{pgfscope}%
\pgfsys@transformshift{2.311879in}{0.952344in}%
\pgfsys@useobject{currentmarker}{}%
\end{pgfscope}%
\begin{pgfscope}%
\pgfsys@transformshift{2.311879in}{0.952344in}%
\pgfsys@useobject{currentmarker}{}%
\end{pgfscope}%
\begin{pgfscope}%
\pgfsys@transformshift{2.311879in}{0.952344in}%
\pgfsys@useobject{currentmarker}{}%
\end{pgfscope}%
\begin{pgfscope}%
\pgfsys@transformshift{2.311879in}{0.952344in}%
\pgfsys@useobject{currentmarker}{}%
\end{pgfscope}%
\begin{pgfscope}%
\pgfsys@transformshift{2.311879in}{0.952344in}%
\pgfsys@useobject{currentmarker}{}%
\end{pgfscope}%
\begin{pgfscope}%
\pgfsys@transformshift{2.311879in}{0.952344in}%
\pgfsys@useobject{currentmarker}{}%
\end{pgfscope}%
\begin{pgfscope}%
\pgfsys@transformshift{2.311879in}{0.952344in}%
\pgfsys@useobject{currentmarker}{}%
\end{pgfscope}%
\begin{pgfscope}%
\pgfsys@transformshift{2.311879in}{0.952344in}%
\pgfsys@useobject{currentmarker}{}%
\end{pgfscope}%
\begin{pgfscope}%
\pgfsys@transformshift{2.311879in}{0.952344in}%
\pgfsys@useobject{currentmarker}{}%
\end{pgfscope}%
\begin{pgfscope}%
\pgfsys@transformshift{2.311879in}{0.952344in}%
\pgfsys@useobject{currentmarker}{}%
\end{pgfscope}%
\begin{pgfscope}%
\pgfsys@transformshift{2.311879in}{0.952344in}%
\pgfsys@useobject{currentmarker}{}%
\end{pgfscope}%
\begin{pgfscope}%
\pgfsys@transformshift{2.311879in}{0.952344in}%
\pgfsys@useobject{currentmarker}{}%
\end{pgfscope}%
\begin{pgfscope}%
\pgfsys@transformshift{2.311879in}{0.952344in}%
\pgfsys@useobject{currentmarker}{}%
\end{pgfscope}%
\begin{pgfscope}%
\pgfsys@transformshift{2.311879in}{0.952344in}%
\pgfsys@useobject{currentmarker}{}%
\end{pgfscope}%
\begin{pgfscope}%
\pgfsys@transformshift{2.311879in}{0.952344in}%
\pgfsys@useobject{currentmarker}{}%
\end{pgfscope}%
\begin{pgfscope}%
\pgfsys@transformshift{2.311879in}{0.952344in}%
\pgfsys@useobject{currentmarker}{}%
\end{pgfscope}%
\begin{pgfscope}%
\pgfsys@transformshift{2.311879in}{0.952344in}%
\pgfsys@useobject{currentmarker}{}%
\end{pgfscope}%
\begin{pgfscope}%
\pgfsys@transformshift{2.311879in}{0.952344in}%
\pgfsys@useobject{currentmarker}{}%
\end{pgfscope}%
\begin{pgfscope}%
\pgfsys@transformshift{2.311879in}{0.952344in}%
\pgfsys@useobject{currentmarker}{}%
\end{pgfscope}%
\begin{pgfscope}%
\pgfsys@transformshift{2.311879in}{0.952344in}%
\pgfsys@useobject{currentmarker}{}%
\end{pgfscope}%
\begin{pgfscope}%
\pgfsys@transformshift{2.311879in}{0.952344in}%
\pgfsys@useobject{currentmarker}{}%
\end{pgfscope}%
\begin{pgfscope}%
\pgfsys@transformshift{2.311879in}{0.952344in}%
\pgfsys@useobject{currentmarker}{}%
\end{pgfscope}%
\begin{pgfscope}%
\pgfsys@transformshift{2.311879in}{0.952344in}%
\pgfsys@useobject{currentmarker}{}%
\end{pgfscope}%
\begin{pgfscope}%
\pgfsys@transformshift{2.311879in}{0.952344in}%
\pgfsys@useobject{currentmarker}{}%
\end{pgfscope}%
\begin{pgfscope}%
\pgfsys@transformshift{2.311879in}{0.952344in}%
\pgfsys@useobject{currentmarker}{}%
\end{pgfscope}%
\begin{pgfscope}%
\pgfsys@transformshift{2.311879in}{0.952344in}%
\pgfsys@useobject{currentmarker}{}%
\end{pgfscope}%
\begin{pgfscope}%
\pgfsys@transformshift{2.311879in}{0.952344in}%
\pgfsys@useobject{currentmarker}{}%
\end{pgfscope}%
\begin{pgfscope}%
\pgfsys@transformshift{2.311879in}{0.952344in}%
\pgfsys@useobject{currentmarker}{}%
\end{pgfscope}%
\begin{pgfscope}%
\pgfsys@transformshift{2.311879in}{0.952344in}%
\pgfsys@useobject{currentmarker}{}%
\end{pgfscope}%
\begin{pgfscope}%
\pgfsys@transformshift{2.311879in}{0.952344in}%
\pgfsys@useobject{currentmarker}{}%
\end{pgfscope}%
\begin{pgfscope}%
\pgfsys@transformshift{2.311879in}{0.952344in}%
\pgfsys@useobject{currentmarker}{}%
\end{pgfscope}%
\begin{pgfscope}%
\pgfsys@transformshift{2.311879in}{0.952344in}%
\pgfsys@useobject{currentmarker}{}%
\end{pgfscope}%
\begin{pgfscope}%
\pgfsys@transformshift{2.311879in}{0.952344in}%
\pgfsys@useobject{currentmarker}{}%
\end{pgfscope}%
\begin{pgfscope}%
\pgfsys@transformshift{2.311879in}{0.952344in}%
\pgfsys@useobject{currentmarker}{}%
\end{pgfscope}%
\begin{pgfscope}%
\pgfsys@transformshift{2.311879in}{0.952344in}%
\pgfsys@useobject{currentmarker}{}%
\end{pgfscope}%
\begin{pgfscope}%
\pgfsys@transformshift{2.311879in}{0.952344in}%
\pgfsys@useobject{currentmarker}{}%
\end{pgfscope}%
\begin{pgfscope}%
\pgfsys@transformshift{2.311879in}{0.952344in}%
\pgfsys@useobject{currentmarker}{}%
\end{pgfscope}%
\begin{pgfscope}%
\pgfsys@transformshift{2.311879in}{0.952344in}%
\pgfsys@useobject{currentmarker}{}%
\end{pgfscope}%
\begin{pgfscope}%
\pgfsys@transformshift{2.311879in}{0.952344in}%
\pgfsys@useobject{currentmarker}{}%
\end{pgfscope}%
\begin{pgfscope}%
\pgfsys@transformshift{2.311879in}{0.952344in}%
\pgfsys@useobject{currentmarker}{}%
\end{pgfscope}%
\begin{pgfscope}%
\pgfsys@transformshift{2.311879in}{0.952344in}%
\pgfsys@useobject{currentmarker}{}%
\end{pgfscope}%
\begin{pgfscope}%
\pgfsys@transformshift{2.311879in}{0.952344in}%
\pgfsys@useobject{currentmarker}{}%
\end{pgfscope}%
\begin{pgfscope}%
\pgfsys@transformshift{2.311879in}{0.952344in}%
\pgfsys@useobject{currentmarker}{}%
\end{pgfscope}%
\begin{pgfscope}%
\pgfsys@transformshift{2.311879in}{0.952344in}%
\pgfsys@useobject{currentmarker}{}%
\end{pgfscope}%
\begin{pgfscope}%
\pgfsys@transformshift{2.311879in}{0.952344in}%
\pgfsys@useobject{currentmarker}{}%
\end{pgfscope}%
\begin{pgfscope}%
\pgfsys@transformshift{2.311879in}{0.952344in}%
\pgfsys@useobject{currentmarker}{}%
\end{pgfscope}%
\begin{pgfscope}%
\pgfsys@transformshift{2.311879in}{0.952344in}%
\pgfsys@useobject{currentmarker}{}%
\end{pgfscope}%
\begin{pgfscope}%
\pgfsys@transformshift{2.311879in}{0.952344in}%
\pgfsys@useobject{currentmarker}{}%
\end{pgfscope}%
\begin{pgfscope}%
\pgfsys@transformshift{2.311879in}{0.952344in}%
\pgfsys@useobject{currentmarker}{}%
\end{pgfscope}%
\begin{pgfscope}%
\pgfsys@transformshift{2.311879in}{0.952344in}%
\pgfsys@useobject{currentmarker}{}%
\end{pgfscope}%
\begin{pgfscope}%
\pgfsys@transformshift{2.311879in}{0.952344in}%
\pgfsys@useobject{currentmarker}{}%
\end{pgfscope}%
\begin{pgfscope}%
\pgfsys@transformshift{2.311879in}{0.952344in}%
\pgfsys@useobject{currentmarker}{}%
\end{pgfscope}%
\begin{pgfscope}%
\pgfsys@transformshift{2.311879in}{0.952344in}%
\pgfsys@useobject{currentmarker}{}%
\end{pgfscope}%
\begin{pgfscope}%
\pgfsys@transformshift{2.311879in}{0.952344in}%
\pgfsys@useobject{currentmarker}{}%
\end{pgfscope}%
\begin{pgfscope}%
\pgfsys@transformshift{2.311879in}{0.952344in}%
\pgfsys@useobject{currentmarker}{}%
\end{pgfscope}%
\begin{pgfscope}%
\pgfsys@transformshift{2.311879in}{0.952344in}%
\pgfsys@useobject{currentmarker}{}%
\end{pgfscope}%
\begin{pgfscope}%
\pgfsys@transformshift{2.311879in}{0.952344in}%
\pgfsys@useobject{currentmarker}{}%
\end{pgfscope}%
\begin{pgfscope}%
\pgfsys@transformshift{2.311879in}{0.952344in}%
\pgfsys@useobject{currentmarker}{}%
\end{pgfscope}%
\begin{pgfscope}%
\pgfsys@transformshift{2.311879in}{0.952344in}%
\pgfsys@useobject{currentmarker}{}%
\end{pgfscope}%
\begin{pgfscope}%
\pgfsys@transformshift{2.311879in}{0.952344in}%
\pgfsys@useobject{currentmarker}{}%
\end{pgfscope}%
\begin{pgfscope}%
\pgfsys@transformshift{2.311879in}{0.952344in}%
\pgfsys@useobject{currentmarker}{}%
\end{pgfscope}%
\begin{pgfscope}%
\pgfsys@transformshift{2.311879in}{0.952344in}%
\pgfsys@useobject{currentmarker}{}%
\end{pgfscope}%
\begin{pgfscope}%
\pgfsys@transformshift{2.311879in}{0.952344in}%
\pgfsys@useobject{currentmarker}{}%
\end{pgfscope}%
\begin{pgfscope}%
\pgfsys@transformshift{2.311879in}{0.952344in}%
\pgfsys@useobject{currentmarker}{}%
\end{pgfscope}%
\begin{pgfscope}%
\pgfsys@transformshift{2.311879in}{0.952344in}%
\pgfsys@useobject{currentmarker}{}%
\end{pgfscope}%
\begin{pgfscope}%
\pgfsys@transformshift{2.311879in}{0.952344in}%
\pgfsys@useobject{currentmarker}{}%
\end{pgfscope}%
\begin{pgfscope}%
\pgfsys@transformshift{2.311879in}{0.952344in}%
\pgfsys@useobject{currentmarker}{}%
\end{pgfscope}%
\begin{pgfscope}%
\pgfsys@transformshift{2.311879in}{0.952344in}%
\pgfsys@useobject{currentmarker}{}%
\end{pgfscope}%
\begin{pgfscope}%
\pgfsys@transformshift{2.311879in}{0.952344in}%
\pgfsys@useobject{currentmarker}{}%
\end{pgfscope}%
\begin{pgfscope}%
\pgfsys@transformshift{2.311879in}{0.952344in}%
\pgfsys@useobject{currentmarker}{}%
\end{pgfscope}%
\begin{pgfscope}%
\pgfsys@transformshift{2.311879in}{0.952344in}%
\pgfsys@useobject{currentmarker}{}%
\end{pgfscope}%
\begin{pgfscope}%
\pgfsys@transformshift{2.311879in}{0.952344in}%
\pgfsys@useobject{currentmarker}{}%
\end{pgfscope}%
\begin{pgfscope}%
\pgfsys@transformshift{2.311879in}{0.952344in}%
\pgfsys@useobject{currentmarker}{}%
\end{pgfscope}%
\begin{pgfscope}%
\pgfsys@transformshift{2.311879in}{0.952344in}%
\pgfsys@useobject{currentmarker}{}%
\end{pgfscope}%
\begin{pgfscope}%
\pgfsys@transformshift{2.311879in}{0.952344in}%
\pgfsys@useobject{currentmarker}{}%
\end{pgfscope}%
\begin{pgfscope}%
\pgfsys@transformshift{2.311879in}{0.952344in}%
\pgfsys@useobject{currentmarker}{}%
\end{pgfscope}%
\begin{pgfscope}%
\pgfsys@transformshift{2.311879in}{0.952344in}%
\pgfsys@useobject{currentmarker}{}%
\end{pgfscope}%
\begin{pgfscope}%
\pgfsys@transformshift{2.311879in}{0.952344in}%
\pgfsys@useobject{currentmarker}{}%
\end{pgfscope}%
\begin{pgfscope}%
\pgfsys@transformshift{2.311879in}{0.952344in}%
\pgfsys@useobject{currentmarker}{}%
\end{pgfscope}%
\begin{pgfscope}%
\pgfsys@transformshift{2.311879in}{0.952344in}%
\pgfsys@useobject{currentmarker}{}%
\end{pgfscope}%
\begin{pgfscope}%
\pgfsys@transformshift{2.311879in}{0.952344in}%
\pgfsys@useobject{currentmarker}{}%
\end{pgfscope}%
\begin{pgfscope}%
\pgfsys@transformshift{2.311879in}{0.952344in}%
\pgfsys@useobject{currentmarker}{}%
\end{pgfscope}%
\begin{pgfscope}%
\pgfsys@transformshift{2.311879in}{0.952344in}%
\pgfsys@useobject{currentmarker}{}%
\end{pgfscope}%
\begin{pgfscope}%
\pgfsys@transformshift{2.311879in}{0.952344in}%
\pgfsys@useobject{currentmarker}{}%
\end{pgfscope}%
\begin{pgfscope}%
\pgfsys@transformshift{2.311879in}{0.952344in}%
\pgfsys@useobject{currentmarker}{}%
\end{pgfscope}%
\begin{pgfscope}%
\pgfsys@transformshift{2.311879in}{0.952344in}%
\pgfsys@useobject{currentmarker}{}%
\end{pgfscope}%
\begin{pgfscope}%
\pgfsys@transformshift{2.311879in}{0.952344in}%
\pgfsys@useobject{currentmarker}{}%
\end{pgfscope}%
\begin{pgfscope}%
\pgfsys@transformshift{2.311879in}{0.952344in}%
\pgfsys@useobject{currentmarker}{}%
\end{pgfscope}%
\begin{pgfscope}%
\pgfsys@transformshift{2.311879in}{0.952344in}%
\pgfsys@useobject{currentmarker}{}%
\end{pgfscope}%
\begin{pgfscope}%
\pgfsys@transformshift{2.311879in}{0.952344in}%
\pgfsys@useobject{currentmarker}{}%
\end{pgfscope}%
\begin{pgfscope}%
\pgfsys@transformshift{2.311879in}{0.952344in}%
\pgfsys@useobject{currentmarker}{}%
\end{pgfscope}%
\begin{pgfscope}%
\pgfsys@transformshift{2.311879in}{0.952344in}%
\pgfsys@useobject{currentmarker}{}%
\end{pgfscope}%
\begin{pgfscope}%
\pgfsys@transformshift{2.311879in}{0.952344in}%
\pgfsys@useobject{currentmarker}{}%
\end{pgfscope}%
\begin{pgfscope}%
\pgfsys@transformshift{2.311879in}{0.952344in}%
\pgfsys@useobject{currentmarker}{}%
\end{pgfscope}%
\begin{pgfscope}%
\pgfsys@transformshift{2.311879in}{0.952344in}%
\pgfsys@useobject{currentmarker}{}%
\end{pgfscope}%
\begin{pgfscope}%
\pgfsys@transformshift{2.311879in}{0.952344in}%
\pgfsys@useobject{currentmarker}{}%
\end{pgfscope}%
\begin{pgfscope}%
\pgfsys@transformshift{2.311879in}{0.952344in}%
\pgfsys@useobject{currentmarker}{}%
\end{pgfscope}%
\begin{pgfscope}%
\pgfsys@transformshift{2.311879in}{0.952344in}%
\pgfsys@useobject{currentmarker}{}%
\end{pgfscope}%
\begin{pgfscope}%
\pgfsys@transformshift{2.311879in}{0.952344in}%
\pgfsys@useobject{currentmarker}{}%
\end{pgfscope}%
\begin{pgfscope}%
\pgfsys@transformshift{2.311879in}{0.952344in}%
\pgfsys@useobject{currentmarker}{}%
\end{pgfscope}%
\begin{pgfscope}%
\pgfsys@transformshift{2.311879in}{0.952344in}%
\pgfsys@useobject{currentmarker}{}%
\end{pgfscope}%
\begin{pgfscope}%
\pgfsys@transformshift{2.311879in}{0.952344in}%
\pgfsys@useobject{currentmarker}{}%
\end{pgfscope}%
\begin{pgfscope}%
\pgfsys@transformshift{2.311879in}{0.952344in}%
\pgfsys@useobject{currentmarker}{}%
\end{pgfscope}%
\begin{pgfscope}%
\pgfsys@transformshift{2.311879in}{0.952344in}%
\pgfsys@useobject{currentmarker}{}%
\end{pgfscope}%
\begin{pgfscope}%
\pgfsys@transformshift{2.311879in}{0.952344in}%
\pgfsys@useobject{currentmarker}{}%
\end{pgfscope}%
\begin{pgfscope}%
\pgfsys@transformshift{2.311879in}{0.952344in}%
\pgfsys@useobject{currentmarker}{}%
\end{pgfscope}%
\begin{pgfscope}%
\pgfsys@transformshift{2.311879in}{0.952344in}%
\pgfsys@useobject{currentmarker}{}%
\end{pgfscope}%
\begin{pgfscope}%
\pgfsys@transformshift{2.311879in}{0.952344in}%
\pgfsys@useobject{currentmarker}{}%
\end{pgfscope}%
\begin{pgfscope}%
\pgfsys@transformshift{2.311879in}{0.952344in}%
\pgfsys@useobject{currentmarker}{}%
\end{pgfscope}%
\begin{pgfscope}%
\pgfsys@transformshift{2.311879in}{0.952344in}%
\pgfsys@useobject{currentmarker}{}%
\end{pgfscope}%
\begin{pgfscope}%
\pgfsys@transformshift{2.311879in}{0.952344in}%
\pgfsys@useobject{currentmarker}{}%
\end{pgfscope}%
\begin{pgfscope}%
\pgfsys@transformshift{2.311879in}{0.952344in}%
\pgfsys@useobject{currentmarker}{}%
\end{pgfscope}%
\begin{pgfscope}%
\pgfsys@transformshift{2.311879in}{0.952344in}%
\pgfsys@useobject{currentmarker}{}%
\end{pgfscope}%
\begin{pgfscope}%
\pgfsys@transformshift{2.311879in}{0.952344in}%
\pgfsys@useobject{currentmarker}{}%
\end{pgfscope}%
\begin{pgfscope}%
\pgfsys@transformshift{2.311879in}{0.952344in}%
\pgfsys@useobject{currentmarker}{}%
\end{pgfscope}%
\begin{pgfscope}%
\pgfsys@transformshift{2.311879in}{0.952344in}%
\pgfsys@useobject{currentmarker}{}%
\end{pgfscope}%
\begin{pgfscope}%
\pgfsys@transformshift{2.311879in}{0.952344in}%
\pgfsys@useobject{currentmarker}{}%
\end{pgfscope}%
\begin{pgfscope}%
\pgfsys@transformshift{2.311879in}{0.952344in}%
\pgfsys@useobject{currentmarker}{}%
\end{pgfscope}%
\begin{pgfscope}%
\pgfsys@transformshift{2.311879in}{0.952344in}%
\pgfsys@useobject{currentmarker}{}%
\end{pgfscope}%
\begin{pgfscope}%
\pgfsys@transformshift{2.311879in}{0.952344in}%
\pgfsys@useobject{currentmarker}{}%
\end{pgfscope}%
\begin{pgfscope}%
\pgfsys@transformshift{2.311879in}{0.952344in}%
\pgfsys@useobject{currentmarker}{}%
\end{pgfscope}%
\begin{pgfscope}%
\pgfsys@transformshift{2.311879in}{0.952344in}%
\pgfsys@useobject{currentmarker}{}%
\end{pgfscope}%
\begin{pgfscope}%
\pgfsys@transformshift{2.311879in}{0.952344in}%
\pgfsys@useobject{currentmarker}{}%
\end{pgfscope}%
\begin{pgfscope}%
\pgfsys@transformshift{2.311879in}{0.952344in}%
\pgfsys@useobject{currentmarker}{}%
\end{pgfscope}%
\begin{pgfscope}%
\pgfsys@transformshift{2.311879in}{0.952344in}%
\pgfsys@useobject{currentmarker}{}%
\end{pgfscope}%
\begin{pgfscope}%
\pgfsys@transformshift{2.311879in}{0.952344in}%
\pgfsys@useobject{currentmarker}{}%
\end{pgfscope}%
\begin{pgfscope}%
\pgfsys@transformshift{2.311879in}{0.952344in}%
\pgfsys@useobject{currentmarker}{}%
\end{pgfscope}%
\begin{pgfscope}%
\pgfsys@transformshift{2.311879in}{0.952344in}%
\pgfsys@useobject{currentmarker}{}%
\end{pgfscope}%
\begin{pgfscope}%
\pgfsys@transformshift{2.311879in}{0.952344in}%
\pgfsys@useobject{currentmarker}{}%
\end{pgfscope}%
\begin{pgfscope}%
\pgfsys@transformshift{2.311879in}{0.952344in}%
\pgfsys@useobject{currentmarker}{}%
\end{pgfscope}%
\begin{pgfscope}%
\pgfsys@transformshift{2.311879in}{0.952344in}%
\pgfsys@useobject{currentmarker}{}%
\end{pgfscope}%
\begin{pgfscope}%
\pgfsys@transformshift{2.311879in}{0.952344in}%
\pgfsys@useobject{currentmarker}{}%
\end{pgfscope}%
\begin{pgfscope}%
\pgfsys@transformshift{2.311879in}{0.952344in}%
\pgfsys@useobject{currentmarker}{}%
\end{pgfscope}%
\begin{pgfscope}%
\pgfsys@transformshift{2.311879in}{0.952344in}%
\pgfsys@useobject{currentmarker}{}%
\end{pgfscope}%
\begin{pgfscope}%
\pgfsys@transformshift{2.311879in}{0.952344in}%
\pgfsys@useobject{currentmarker}{}%
\end{pgfscope}%
\begin{pgfscope}%
\pgfsys@transformshift{2.311879in}{0.952344in}%
\pgfsys@useobject{currentmarker}{}%
\end{pgfscope}%
\begin{pgfscope}%
\pgfsys@transformshift{2.311879in}{0.952344in}%
\pgfsys@useobject{currentmarker}{}%
\end{pgfscope}%
\begin{pgfscope}%
\pgfsys@transformshift{2.311879in}{0.952344in}%
\pgfsys@useobject{currentmarker}{}%
\end{pgfscope}%
\begin{pgfscope}%
\pgfsys@transformshift{2.311879in}{0.952344in}%
\pgfsys@useobject{currentmarker}{}%
\end{pgfscope}%
\begin{pgfscope}%
\pgfsys@transformshift{2.311879in}{0.952344in}%
\pgfsys@useobject{currentmarker}{}%
\end{pgfscope}%
\begin{pgfscope}%
\pgfsys@transformshift{2.311879in}{0.952344in}%
\pgfsys@useobject{currentmarker}{}%
\end{pgfscope}%
\begin{pgfscope}%
\pgfsys@transformshift{2.311879in}{0.952344in}%
\pgfsys@useobject{currentmarker}{}%
\end{pgfscope}%
\begin{pgfscope}%
\pgfsys@transformshift{2.311879in}{0.952344in}%
\pgfsys@useobject{currentmarker}{}%
\end{pgfscope}%
\end{pgfscope}%
\begin{pgfscope}%
\pgfpathrectangle{\pgfqpoint{0.562500in}{0.275000in}}{\pgfqpoint{3.487500in}{1.925000in}}%
\pgfusepath{clip}%
\pgfsetrectcap%
\pgfsetroundjoin%
\pgfsetlinewidth{0.501875pt}%
\definecolor{currentstroke}{rgb}{1.000000,0.498039,0.054902}%
\pgfsetstrokecolor{currentstroke}%
\pgfsetdash{}{0pt}%
\pgfpathmoveto{\pgfqpoint{2.311879in}{0.952344in}}%
\pgfpathlineto{\pgfqpoint{2.311879in}{0.952344in}}%
\pgfusepath{stroke}%
\end{pgfscope}%
\begin{pgfscope}%
\pgfpathrectangle{\pgfqpoint{0.562500in}{0.275000in}}{\pgfqpoint{3.487500in}{1.925000in}}%
\pgfusepath{clip}%
\pgfsetrectcap%
\pgfsetroundjoin%
\pgfsetlinewidth{0.501875pt}%
\definecolor{currentstroke}{rgb}{0.172549,0.627451,0.172549}%
\pgfsetstrokecolor{currentstroke}%
\pgfsetdash{}{0pt}%
\pgfpathmoveto{\pgfqpoint{3.822799in}{0.388145in}}%
\pgfpathlineto{\pgfqpoint{3.785534in}{0.485089in}}%
\pgfpathlineto{\pgfqpoint{3.698885in}{0.554518in}}%
\pgfpathlineto{\pgfqpoint{3.586773in}{0.602279in}}%
\pgfpathlineto{\pgfqpoint{3.464524in}{0.634300in}}%
\pgfpathlineto{\pgfqpoint{3.340449in}{0.656865in}}%
\pgfpathlineto{\pgfqpoint{3.218314in}{0.673670in}}%
\pgfpathlineto{\pgfqpoint{3.100471in}{0.686587in}}%
\pgfpathlineto{\pgfqpoint{2.786308in}{0.718078in}}%
\pgfpathlineto{\pgfqpoint{2.697116in}{0.728920in}}%
\pgfpathlineto{\pgfqpoint{2.616073in}{0.740435in}}%
\pgfpathlineto{\pgfqpoint{2.543219in}{0.752677in}}%
\pgfpathlineto{\pgfqpoint{2.478490in}{0.765605in}}%
\pgfpathlineto{\pgfqpoint{2.421727in}{0.779078in}}%
\pgfpathlineto{\pgfqpoint{2.372667in}{0.792868in}}%
\pgfpathlineto{\pgfqpoint{2.330909in}{0.806774in}}%
\pgfpathlineto{\pgfqpoint{2.295918in}{0.820615in}}%
\pgfpathlineto{\pgfqpoint{2.267157in}{0.834216in}}%
\pgfpathlineto{\pgfqpoint{2.244103in}{0.847417in}}%
\pgfpathlineto{\pgfqpoint{2.226243in}{0.860075in}}%
\pgfpathlineto{\pgfqpoint{2.213042in}{0.872069in}}%
\pgfpathlineto{\pgfqpoint{2.203894in}{0.883321in}}%
\pgfpathlineto{\pgfqpoint{2.198243in}{0.893777in}}%
\pgfpathlineto{\pgfqpoint{2.195580in}{0.903399in}}%
\pgfpathlineto{\pgfqpoint{2.195443in}{0.912162in}}%
\pgfpathlineto{\pgfqpoint{2.197418in}{0.920058in}}%
\pgfpathlineto{\pgfqpoint{2.201089in}{0.927100in}}%
\pgfpathlineto{\pgfqpoint{2.206077in}{0.933325in}}%
\pgfpathlineto{\pgfqpoint{2.218728in}{0.943494in}}%
\pgfpathlineto{\pgfqpoint{2.233274in}{0.950903in}}%
\pgfpathlineto{\pgfqpoint{2.248185in}{0.955951in}}%
\pgfpathlineto{\pgfqpoint{2.268929in}{0.960105in}}%
\pgfpathlineto{\pgfqpoint{2.286037in}{0.961344in}}%
\pgfpathlineto{\pgfqpoint{2.302226in}{0.960419in}}%
\pgfpathlineto{\pgfqpoint{2.313042in}{0.957689in}}%
\pgfpathlineto{\pgfqpoint{2.316844in}{0.954628in}}%
\pgfpathlineto{\pgfqpoint{2.316023in}{0.952802in}}%
\pgfpathlineto{\pgfqpoint{2.312674in}{0.951959in}}%
\pgfpathlineto{\pgfqpoint{2.311662in}{0.952272in}}%
\pgfpathlineto{\pgfqpoint{2.311880in}{0.952344in}}%
\pgfusepath{stroke}%
\end{pgfscope}%
\begin{pgfscope}%
\pgfpathrectangle{\pgfqpoint{0.562500in}{0.275000in}}{\pgfqpoint{3.487500in}{1.925000in}}%
\pgfusepath{clip}%
\pgfsetrectcap%
\pgfsetroundjoin%
\pgfsetlinewidth{0.501875pt}%
\definecolor{currentstroke}{rgb}{0.839216,0.152941,0.156863}%
\pgfsetstrokecolor{currentstroke}%
\pgfsetdash{}{0pt}%
\pgfpathmoveto{\pgfqpoint{3.822799in}{2.080740in}}%
\pgfpathlineto{\pgfqpoint{2.647638in}{2.086855in}}%
\pgfpathlineto{\pgfqpoint{1.870041in}{1.971575in}}%
\pgfpathlineto{\pgfqpoint{1.421076in}{1.825206in}}%
\pgfpathlineto{\pgfqpoint{1.201698in}{1.688815in}}%
\pgfpathlineto{\pgfqpoint{1.129231in}{1.576435in}}%
\pgfpathlineto{\pgfqpoint{1.143458in}{1.489122in}}%
\pgfpathlineto{\pgfqpoint{1.208446in}{1.423364in}}%
\pgfpathlineto{\pgfqpoint{1.301031in}{1.374642in}}%
\pgfpathlineto{\pgfqpoint{1.406276in}{1.337641in}}%
\pgfpathlineto{\pgfqpoint{1.516064in}{1.308486in}}%
\pgfpathlineto{\pgfqpoint{1.625991in}{1.284747in}}%
\pgfpathlineto{\pgfqpoint{2.013680in}{1.210376in}}%
\pgfpathlineto{\pgfqpoint{2.091349in}{1.192914in}}%
\pgfpathlineto{\pgfqpoint{2.160607in}{1.175378in}}%
\pgfpathlineto{\pgfqpoint{2.221440in}{1.157754in}}%
\pgfpathlineto{\pgfqpoint{2.274006in}{1.140129in}}%
\pgfpathlineto{\pgfqpoint{2.318638in}{1.122693in}}%
\pgfpathlineto{\pgfqpoint{2.355769in}{1.105615in}}%
\pgfpathlineto{\pgfqpoint{2.385994in}{1.089036in}}%
\pgfpathlineto{\pgfqpoint{2.409929in}{1.073096in}}%
\pgfpathlineto{\pgfqpoint{2.428167in}{1.057921in}}%
\pgfpathlineto{\pgfqpoint{2.441281in}{1.043625in}}%
\pgfpathlineto{\pgfqpoint{2.449840in}{1.030305in}}%
\pgfpathlineto{\pgfqpoint{2.454492in}{1.018010in}}%
\pgfpathlineto{\pgfqpoint{2.455853in}{1.006757in}}%
\pgfpathlineto{\pgfqpoint{2.454474in}{0.996550in}}%
\pgfpathlineto{\pgfqpoint{2.450846in}{0.987384in}}%
\pgfpathlineto{\pgfqpoint{2.445405in}{0.979243in}}%
\pgfpathlineto{\pgfqpoint{2.438546in}{0.972097in}}%
\pgfpathlineto{\pgfqpoint{2.430659in}{0.965890in}}%
\pgfpathlineto{\pgfqpoint{2.413059in}{0.956031in}}%
\pgfpathlineto{\pgfqpoint{2.394669in}{0.949162in}}%
\pgfpathlineto{\pgfqpoint{2.376925in}{0.944777in}}%
\pgfpathlineto{\pgfqpoint{2.353464in}{0.941671in}}%
\pgfpathlineto{\pgfqpoint{2.335051in}{0.941394in}}%
\pgfpathlineto{\pgfqpoint{2.318514in}{0.943325in}}%
\pgfpathlineto{\pgfqpoint{2.308382in}{0.946826in}}%
\pgfpathlineto{\pgfqpoint{2.305720in}{0.949789in}}%
\pgfpathlineto{\pgfqpoint{2.306613in}{0.951654in}}%
\pgfpathlineto{\pgfqpoint{2.309962in}{0.952797in}}%
\pgfpathlineto{\pgfqpoint{2.312147in}{0.952442in}}%
\pgfpathlineto{\pgfqpoint{2.311879in}{0.952344in}}%
\pgfusepath{stroke}%
\end{pgfscope}%
\begin{pgfscope}%
\pgfpathrectangle{\pgfqpoint{0.562500in}{0.275000in}}{\pgfqpoint{3.487500in}{1.925000in}}%
\pgfusepath{clip}%
\pgfsetrectcap%
\pgfsetroundjoin%
\pgfsetlinewidth{0.501875pt}%
\definecolor{currentstroke}{rgb}{0.580392,0.403922,0.741176}%
\pgfsetstrokecolor{currentstroke}%
\pgfsetdash{}{0pt}%
\pgfpathmoveto{\pgfqpoint{1.934149in}{2.080740in}}%
\pgfpathlineto{\pgfqpoint{1.257046in}{1.902834in}}%
\pgfpathlineto{\pgfqpoint{0.917074in}{1.723615in}}%
\pgfpathlineto{\pgfqpoint{0.790322in}{1.573736in}}%
\pgfpathlineto{\pgfqpoint{0.789701in}{1.459425in}}%
\pgfpathlineto{\pgfqpoint{0.857185in}{1.376461in}}%
\pgfpathlineto{\pgfqpoint{0.960729in}{1.318725in}}%
\pgfpathlineto{\pgfqpoint{1.080141in}{1.279539in}}%
\pgfpathlineto{\pgfqpoint{1.204692in}{1.252253in}}%
\pgfpathlineto{\pgfqpoint{1.329050in}{1.233167in}}%
\pgfpathlineto{\pgfqpoint{1.450418in}{1.219315in}}%
\pgfpathlineto{\pgfqpoint{1.871471in}{1.178936in}}%
\pgfpathlineto{\pgfqpoint{1.957281in}{1.168315in}}%
\pgfpathlineto{\pgfqpoint{2.034908in}{1.156899in}}%
\pgfpathlineto{\pgfqpoint{2.104346in}{1.144688in}}%
\pgfpathlineto{\pgfqpoint{2.165722in}{1.131834in}}%
\pgfpathlineto{\pgfqpoint{2.219250in}{1.118521in}}%
\pgfpathlineto{\pgfqpoint{2.265340in}{1.104933in}}%
\pgfpathlineto{\pgfqpoint{2.304454in}{1.091264in}}%
\pgfpathlineto{\pgfqpoint{2.337056in}{1.077700in}}%
\pgfpathlineto{\pgfqpoint{2.363611in}{1.064421in}}%
\pgfpathlineto{\pgfqpoint{2.384607in}{1.051592in}}%
\pgfpathlineto{\pgfqpoint{2.400650in}{1.039341in}}%
\pgfpathlineto{\pgfqpoint{2.412346in}{1.027758in}}%
\pgfpathlineto{\pgfqpoint{2.420248in}{1.016918in}}%
\pgfpathlineto{\pgfqpoint{2.424864in}{1.006877in}}%
\pgfpathlineto{\pgfqpoint{2.426653in}{0.997674in}}%
\pgfpathlineto{\pgfqpoint{2.426059in}{0.989329in}}%
\pgfpathlineto{\pgfqpoint{2.423535in}{0.981830in}}%
\pgfpathlineto{\pgfqpoint{2.419480in}{0.975151in}}%
\pgfpathlineto{\pgfqpoint{2.414244in}{0.969260in}}%
\pgfpathlineto{\pgfqpoint{2.401384in}{0.959705in}}%
\pgfpathlineto{\pgfqpoint{2.386892in}{0.952828in}}%
\pgfpathlineto{\pgfqpoint{2.372270in}{0.948180in}}%
\pgfpathlineto{\pgfqpoint{2.352171in}{0.944463in}}%
\pgfpathlineto{\pgfqpoint{2.331233in}{0.943552in}}%
\pgfpathlineto{\pgfqpoint{2.317677in}{0.945031in}}%
\pgfpathlineto{\pgfqpoint{2.309264in}{0.947824in}}%
\pgfpathlineto{\pgfqpoint{2.306942in}{0.950601in}}%
\pgfpathlineto{\pgfqpoint{2.308199in}{0.952126in}}%
\pgfpathlineto{\pgfqpoint{2.311786in}{0.952622in}}%
\pgfpathlineto{\pgfqpoint{2.312092in}{0.952410in}}%
\pgfpathlineto{\pgfqpoint{2.311879in}{0.952344in}}%
\pgfusepath{stroke}%
\end{pgfscope}%
\begin{pgfscope}%
\pgfpathrectangle{\pgfqpoint{0.562500in}{0.275000in}}{\pgfqpoint{3.487500in}{1.925000in}}%
\pgfusepath{clip}%
\pgfsetrectcap%
\pgfsetroundjoin%
\pgfsetlinewidth{0.501875pt}%
\definecolor{currentstroke}{rgb}{0.549020,0.337255,0.294118}%
\pgfsetstrokecolor{currentstroke}%
\pgfsetdash{}{0pt}%
\pgfpathmoveto{\pgfqpoint{1.178690in}{1.516542in}}%
\pgfpathlineto{\pgfqpoint{1.230492in}{1.446393in}}%
\pgfpathlineto{\pgfqpoint{1.313516in}{1.393131in}}%
\pgfpathlineto{\pgfqpoint{1.414221in}{1.352784in}}%
\pgfpathlineto{\pgfqpoint{1.523157in}{1.321862in}}%
\pgfpathlineto{\pgfqpoint{1.633084in}{1.297050in}}%
\pgfpathlineto{\pgfqpoint{2.019002in}{1.217540in}}%
\pgfpathlineto{\pgfqpoint{2.096884in}{1.199150in}}%
\pgfpathlineto{\pgfqpoint{2.166310in}{1.180770in}}%
\pgfpathlineto{\pgfqpoint{2.227233in}{1.162422in}}%
\pgfpathlineto{\pgfqpoint{2.279895in}{1.144188in}}%
\pgfpathlineto{\pgfqpoint{2.324667in}{1.126191in}}%
\pgfpathlineto{\pgfqpoint{2.361967in}{1.108570in}}%
\pgfpathlineto{\pgfqpoint{2.392259in}{1.091483in}}%
\pgfpathlineto{\pgfqpoint{2.416054in}{1.075104in}}%
\pgfpathlineto{\pgfqpoint{2.433990in}{1.059565in}}%
\pgfpathlineto{\pgfqpoint{2.446779in}{1.044948in}}%
\pgfpathlineto{\pgfqpoint{2.455079in}{1.031319in}}%
\pgfpathlineto{\pgfqpoint{2.459496in}{1.018732in}}%
\pgfpathlineto{\pgfqpoint{2.460582in}{1.007225in}}%
\pgfpathlineto{\pgfqpoint{2.458839in}{0.996820in}}%
\pgfpathlineto{\pgfqpoint{2.454797in}{0.987509in}}%
\pgfpathlineto{\pgfqpoint{2.448951in}{0.979246in}}%
\pgfpathlineto{\pgfqpoint{2.441733in}{0.971979in}}%
\pgfpathlineto{\pgfqpoint{2.433513in}{0.965660in}}%
\pgfpathlineto{\pgfqpoint{2.415281in}{0.955654in}}%
\pgfpathlineto{\pgfqpoint{2.396285in}{0.948744in}}%
\pgfpathlineto{\pgfqpoint{2.378030in}{0.944335in}}%
\pgfpathlineto{\pgfqpoint{2.353996in}{0.941272in}}%
\pgfpathlineto{\pgfqpoint{2.335183in}{0.941066in}}%
\pgfpathlineto{\pgfqpoint{2.318387in}{0.943114in}}%
\pgfpathlineto{\pgfqpoint{2.308142in}{0.946726in}}%
\pgfpathlineto{\pgfqpoint{2.305511in}{0.949762in}}%
\pgfpathlineto{\pgfqpoint{2.306480in}{0.951663in}}%
\pgfpathlineto{\pgfqpoint{2.309936in}{0.952814in}}%
\pgfpathlineto{\pgfqpoint{2.312156in}{0.952443in}}%
\pgfpathlineto{\pgfqpoint{2.311878in}{0.952344in}}%
\pgfusepath{stroke}%
\end{pgfscope}%
\begin{pgfscope}%
\pgfpathrectangle{\pgfqpoint{0.562500in}{0.275000in}}{\pgfqpoint{3.487500in}{1.925000in}}%
\pgfusepath{clip}%
\pgfsetrectcap%
\pgfsetroundjoin%
\pgfsetlinewidth{0.501875pt}%
\definecolor{currentstroke}{rgb}{0.890196,0.466667,0.760784}%
\pgfsetstrokecolor{currentstroke}%
\pgfsetdash{}{0pt}%
\pgfpathmoveto{\pgfqpoint{0.800960in}{0.388145in}}%
\pgfpathlineto{\pgfqpoint{1.060574in}{0.402061in}}%
\pgfpathlineto{\pgfqpoint{1.246504in}{0.434417in}}%
\pgfpathlineto{\pgfqpoint{1.374409in}{0.476281in}}%
\pgfpathlineto{\pgfqpoint{1.460159in}{0.521930in}}%
\pgfpathlineto{\pgfqpoint{1.520588in}{0.568262in}}%
\pgfpathlineto{\pgfqpoint{1.565593in}{0.613869in}}%
\pgfpathlineto{\pgfqpoint{1.600185in}{0.657873in}}%
\pgfpathlineto{\pgfqpoint{1.656845in}{0.738885in}}%
\pgfpathlineto{\pgfqpoint{1.684516in}{0.775186in}}%
\pgfpathlineto{\pgfqpoint{1.713551in}{0.808491in}}%
\pgfpathlineto{\pgfqpoint{1.744389in}{0.838753in}}%
\pgfpathlineto{\pgfqpoint{1.777174in}{0.865958in}}%
\pgfpathlineto{\pgfqpoint{1.811793in}{0.890129in}}%
\pgfpathlineto{\pgfqpoint{1.847875in}{0.911325in}}%
\pgfpathlineto{\pgfqpoint{1.884805in}{0.929645in}}%
\pgfpathlineto{\pgfqpoint{1.922044in}{0.945238in}}%
\pgfpathlineto{\pgfqpoint{1.959111in}{0.958304in}}%
\pgfpathlineto{\pgfqpoint{1.995534in}{0.969044in}}%
\pgfpathlineto{\pgfqpoint{2.030886in}{0.977652in}}%
\pgfpathlineto{\pgfqpoint{2.064784in}{0.984321in}}%
\pgfpathlineto{\pgfqpoint{2.096903in}{0.989251in}}%
\pgfpathlineto{\pgfqpoint{2.155038in}{0.994777in}}%
\pgfpathlineto{\pgfqpoint{2.204273in}{0.995823in}}%
\pgfpathlineto{\pgfqpoint{2.244276in}{0.993715in}}%
\pgfpathlineto{\pgfqpoint{2.275546in}{0.989621in}}%
\pgfpathlineto{\pgfqpoint{2.298967in}{0.984462in}}%
\pgfpathlineto{\pgfqpoint{2.315489in}{0.978899in}}%
\pgfpathlineto{\pgfqpoint{2.326326in}{0.973429in}}%
\pgfpathlineto{\pgfqpoint{2.332664in}{0.968382in}}%
\pgfpathlineto{\pgfqpoint{2.335536in}{0.963950in}}%
\pgfpathlineto{\pgfqpoint{2.335915in}{0.960217in}}%
\pgfpathlineto{\pgfqpoint{2.333504in}{0.955948in}}%
\pgfpathlineto{\pgfqpoint{2.327746in}{0.952458in}}%
\pgfpathlineto{\pgfqpoint{2.319155in}{0.950558in}}%
\pgfpathlineto{\pgfqpoint{2.311843in}{0.951149in}}%
\pgfpathlineto{\pgfqpoint{2.310850in}{0.952047in}}%
\pgfpathlineto{\pgfqpoint{2.311643in}{0.952425in}}%
\pgfpathlineto{\pgfqpoint{2.311925in}{0.952361in}}%
\pgfpathlineto{\pgfqpoint{2.311879in}{0.952344in}}%
\pgfusepath{stroke}%
\end{pgfscope}%
\begin{pgfscope}%
\pgfsetrectcap%
\pgfsetmiterjoin%
\pgfsetlinewidth{0.803000pt}%
\definecolor{currentstroke}{rgb}{0.000000,0.000000,0.000000}%
\pgfsetstrokecolor{currentstroke}%
\pgfsetdash{}{0pt}%
\pgfpathmoveto{\pgfqpoint{0.562500in}{0.275000in}}%
\pgfpathlineto{\pgfqpoint{0.562500in}{2.200000in}}%
\pgfusepath{stroke}%
\end{pgfscope}%
\begin{pgfscope}%
\pgfsetrectcap%
\pgfsetmiterjoin%
\pgfsetlinewidth{0.803000pt}%
\definecolor{currentstroke}{rgb}{0.000000,0.000000,0.000000}%
\pgfsetstrokecolor{currentstroke}%
\pgfsetdash{}{0pt}%
\pgfpathmoveto{\pgfqpoint{0.562500in}{0.275000in}}%
\pgfpathlineto{\pgfqpoint{4.050000in}{0.275000in}}%
\pgfusepath{stroke}%
\end{pgfscope}%
\begin{pgfscope}%
\pgfsetroundcap%
\pgfsetroundjoin%
\pgfsetlinewidth{0.501875pt}%
\definecolor{currentstroke}{rgb}{0.000000,0.000000,0.000000}%
\pgfsetstrokecolor{currentstroke}%
\pgfsetdash{}{0pt}%
\pgfpathmoveto{\pgfqpoint{4.111986in}{0.275000in}}%
\pgfpathquadraticcurveto{\pgfqpoint{4.084875in}{0.275000in}}{\pgfqpoint{4.050000in}{0.275000in}}%
\pgfusepath{stroke}%
\end{pgfscope}%
\begin{pgfscope}%
\pgfsetroundcap%
\pgfsetroundjoin%
\pgfsetlinewidth{0.501875pt}%
\definecolor{currentstroke}{rgb}{0.000000,0.000000,0.000000}%
\pgfsetstrokecolor{currentstroke}%
\pgfsetdash{}{0pt}%
\pgfpathmoveto{\pgfqpoint{4.056430in}{0.302778in}}%
\pgfpathlineto{\pgfqpoint{4.111986in}{0.275000in}}%
\pgfpathlineto{\pgfqpoint{4.056430in}{0.247222in}}%
\pgfusepath{stroke}%
\end{pgfscope}%
\begin{pgfscope}%
\pgfsetroundcap%
\pgfsetroundjoin%
\pgfsetlinewidth{0.501875pt}%
\definecolor{currentstroke}{rgb}{0.000000,0.000000,0.000000}%
\pgfsetstrokecolor{currentstroke}%
\pgfsetdash{}{0pt}%
\pgfpathmoveto{\pgfqpoint{0.562500in}{2.230736in}}%
\pgfpathquadraticcurveto{\pgfqpoint{0.562500in}{2.219250in}}{\pgfqpoint{0.562500in}{2.200000in}}%
\pgfusepath{stroke}%
\end{pgfscope}%
\begin{pgfscope}%
\pgfsetroundcap%
\pgfsetroundjoin%
\pgfsetlinewidth{0.501875pt}%
\definecolor{currentstroke}{rgb}{0.000000,0.000000,0.000000}%
\pgfsetstrokecolor{currentstroke}%
\pgfsetdash{}{0pt}%
\pgfpathmoveto{\pgfqpoint{0.534722in}{2.175180in}}%
\pgfpathlineto{\pgfqpoint{0.562500in}{2.230736in}}%
\pgfpathlineto{\pgfqpoint{0.590278in}{2.175180in}}%
\pgfusepath{stroke}%
\end{pgfscope}%
\end{pgfpicture}%
\makeatother%
\endgroup%

        \end{center}
    \end{frame}
    \begin{frame}
    \frametitle{Poincaré-Bendixson: Fall 2}
        \begin{enumerate}
            \setcounter{enumi}{1}
            \item $\omega(p)$ ist ein geschlossener Orbit
        \end{enumerate}
        \begin{center}
            %% Creator: Matplotlib, PGF backend
%%
%% To include the figure in your LaTeX document, write
%%   \input{<filename>.pgf}
%%
%% Make sure the required packages are loaded in your preamble
%%   \usepackage{pgf}
%%
%% Also ensure that all the required font packages are loaded; for instance,
%% the lmodern package is sometimes necessary when using math font.
%%   \usepackage{lmodern}
%%
%% Figures using additional raster images can only be included by \input if
%% they are in the same directory as the main LaTeX file. For loading figures
%% from other directories you can use the `import` package
%%   \usepackage{import}
%%
%% and then include the figures with
%%   \import{<path to file>}{<filename>.pgf}
%%
%% Matplotlib used the following preamble
%%   \usepackage{bm}
%%   \usepackage{amsmath}
%%   \usepackage{xcolor}
%%   \usepackage{tgtermes}
%%   \makeatletter\@ifpackageloaded{underscore}{}{\usepackage[strings]{underscore}}\makeatother
%%
\begingroup%
\makeatletter%
\begin{pgfpicture}%
\pgfpathrectangle{\pgfpointorigin}{\pgfqpoint{4.500000in}{2.500000in}}%
\pgfusepath{use as bounding box, clip}%
\begin{pgfscope}%
\pgfsetbuttcap%
\pgfsetmiterjoin%
\definecolor{currentfill}{rgb}{1.000000,1.000000,1.000000}%
\pgfsetfillcolor{currentfill}%
\pgfsetlinewidth{0.000000pt}%
\definecolor{currentstroke}{rgb}{1.000000,1.000000,1.000000}%
\pgfsetstrokecolor{currentstroke}%
\pgfsetdash{}{0pt}%
\pgfpathmoveto{\pgfqpoint{0.000000in}{0.000000in}}%
\pgfpathlineto{\pgfqpoint{4.500000in}{0.000000in}}%
\pgfpathlineto{\pgfqpoint{4.500000in}{2.500000in}}%
\pgfpathlineto{\pgfqpoint{0.000000in}{2.500000in}}%
\pgfpathlineto{\pgfqpoint{0.000000in}{0.000000in}}%
\pgfpathclose%
\pgfusepath{fill}%
\end{pgfscope}%
\begin{pgfscope}%
\pgfsetbuttcap%
\pgfsetmiterjoin%
\definecolor{currentfill}{rgb}{1.000000,1.000000,1.000000}%
\pgfsetfillcolor{currentfill}%
\pgfsetlinewidth{0.000000pt}%
\definecolor{currentstroke}{rgb}{0.000000,0.000000,0.000000}%
\pgfsetstrokecolor{currentstroke}%
\pgfsetstrokeopacity{0.000000}%
\pgfsetdash{}{0pt}%
\pgfpathmoveto{\pgfqpoint{0.562500in}{0.275000in}}%
\pgfpathlineto{\pgfqpoint{4.050000in}{0.275000in}}%
\pgfpathlineto{\pgfqpoint{4.050000in}{2.200000in}}%
\pgfpathlineto{\pgfqpoint{0.562500in}{2.200000in}}%
\pgfpathlineto{\pgfqpoint{0.562500in}{0.275000in}}%
\pgfpathclose%
\pgfusepath{fill}%
\end{pgfscope}%
\begin{pgfscope}%
\pgfpathrectangle{\pgfqpoint{0.562500in}{0.275000in}}{\pgfqpoint{3.487500in}{1.925000in}}%
\pgfusepath{clip}%
\pgfsetrectcap%
\pgfsetroundjoin%
\pgfsetlinewidth{0.803000pt}%
\definecolor{currentstroke}{rgb}{0.690196,0.690196,0.690196}%
\pgfsetstrokecolor{currentstroke}%
\pgfsetdash{}{0pt}%
\pgfpathmoveto{\pgfqpoint{0.721023in}{0.275000in}}%
\pgfpathlineto{\pgfqpoint{0.721023in}{2.200000in}}%
\pgfusepath{stroke}%
\end{pgfscope}%
\begin{pgfscope}%
\pgfsetbuttcap%
\pgfsetroundjoin%
\definecolor{currentfill}{rgb}{0.000000,0.000000,0.000000}%
\pgfsetfillcolor{currentfill}%
\pgfsetlinewidth{0.803000pt}%
\definecolor{currentstroke}{rgb}{0.000000,0.000000,0.000000}%
\pgfsetstrokecolor{currentstroke}%
\pgfsetdash{}{0pt}%
\pgfsys@defobject{currentmarker}{\pgfqpoint{0.000000in}{-0.048611in}}{\pgfqpoint{0.000000in}{0.000000in}}{%
\pgfpathmoveto{\pgfqpoint{0.000000in}{0.000000in}}%
\pgfpathlineto{\pgfqpoint{0.000000in}{-0.048611in}}%
\pgfusepath{stroke,fill}%
}%
\begin{pgfscope}%
\pgfsys@transformshift{0.721023in}{0.275000in}%
\pgfsys@useobject{currentmarker}{}%
\end{pgfscope}%
\end{pgfscope}%
\begin{pgfscope}%
\definecolor{textcolor}{rgb}{0.000000,0.000000,0.000000}%
\pgfsetstrokecolor{textcolor}%
\pgfsetfillcolor{textcolor}%
\pgftext[x=0.721023in,y=0.177778in,,top]{\color{textcolor}\rmfamily\fontsize{10.000000}{12.000000}\selectfont \(\displaystyle {-1.0}\)}%
\end{pgfscope}%
\begin{pgfscope}%
\pgfpathrectangle{\pgfqpoint{0.562500in}{0.275000in}}{\pgfqpoint{3.487500in}{1.925000in}}%
\pgfusepath{clip}%
\pgfsetrectcap%
\pgfsetroundjoin%
\pgfsetlinewidth{0.803000pt}%
\definecolor{currentstroke}{rgb}{0.690196,0.690196,0.690196}%
\pgfsetstrokecolor{currentstroke}%
\pgfsetdash{}{0pt}%
\pgfpathmoveto{\pgfqpoint{1.513636in}{0.275000in}}%
\pgfpathlineto{\pgfqpoint{1.513636in}{2.200000in}}%
\pgfusepath{stroke}%
\end{pgfscope}%
\begin{pgfscope}%
\pgfsetbuttcap%
\pgfsetroundjoin%
\definecolor{currentfill}{rgb}{0.000000,0.000000,0.000000}%
\pgfsetfillcolor{currentfill}%
\pgfsetlinewidth{0.803000pt}%
\definecolor{currentstroke}{rgb}{0.000000,0.000000,0.000000}%
\pgfsetstrokecolor{currentstroke}%
\pgfsetdash{}{0pt}%
\pgfsys@defobject{currentmarker}{\pgfqpoint{0.000000in}{-0.048611in}}{\pgfqpoint{0.000000in}{0.000000in}}{%
\pgfpathmoveto{\pgfqpoint{0.000000in}{0.000000in}}%
\pgfpathlineto{\pgfqpoint{0.000000in}{-0.048611in}}%
\pgfusepath{stroke,fill}%
}%
\begin{pgfscope}%
\pgfsys@transformshift{1.513636in}{0.275000in}%
\pgfsys@useobject{currentmarker}{}%
\end{pgfscope}%
\end{pgfscope}%
\begin{pgfscope}%
\definecolor{textcolor}{rgb}{0.000000,0.000000,0.000000}%
\pgfsetstrokecolor{textcolor}%
\pgfsetfillcolor{textcolor}%
\pgftext[x=1.513636in,y=0.177778in,,top]{\color{textcolor}\rmfamily\fontsize{10.000000}{12.000000}\selectfont \(\displaystyle {-0.5}\)}%
\end{pgfscope}%
\begin{pgfscope}%
\pgfpathrectangle{\pgfqpoint{0.562500in}{0.275000in}}{\pgfqpoint{3.487500in}{1.925000in}}%
\pgfusepath{clip}%
\pgfsetrectcap%
\pgfsetroundjoin%
\pgfsetlinewidth{0.803000pt}%
\definecolor{currentstroke}{rgb}{0.690196,0.690196,0.690196}%
\pgfsetstrokecolor{currentstroke}%
\pgfsetdash{}{0pt}%
\pgfpathmoveto{\pgfqpoint{2.306250in}{0.275000in}}%
\pgfpathlineto{\pgfqpoint{2.306250in}{2.200000in}}%
\pgfusepath{stroke}%
\end{pgfscope}%
\begin{pgfscope}%
\pgfsetbuttcap%
\pgfsetroundjoin%
\definecolor{currentfill}{rgb}{0.000000,0.000000,0.000000}%
\pgfsetfillcolor{currentfill}%
\pgfsetlinewidth{0.803000pt}%
\definecolor{currentstroke}{rgb}{0.000000,0.000000,0.000000}%
\pgfsetstrokecolor{currentstroke}%
\pgfsetdash{}{0pt}%
\pgfsys@defobject{currentmarker}{\pgfqpoint{0.000000in}{-0.048611in}}{\pgfqpoint{0.000000in}{0.000000in}}{%
\pgfpathmoveto{\pgfqpoint{0.000000in}{0.000000in}}%
\pgfpathlineto{\pgfqpoint{0.000000in}{-0.048611in}}%
\pgfusepath{stroke,fill}%
}%
\begin{pgfscope}%
\pgfsys@transformshift{2.306250in}{0.275000in}%
\pgfsys@useobject{currentmarker}{}%
\end{pgfscope}%
\end{pgfscope}%
\begin{pgfscope}%
\definecolor{textcolor}{rgb}{0.000000,0.000000,0.000000}%
\pgfsetstrokecolor{textcolor}%
\pgfsetfillcolor{textcolor}%
\pgftext[x=2.306250in,y=0.177778in,,top]{\color{textcolor}\rmfamily\fontsize{10.000000}{12.000000}\selectfont \(\displaystyle {0.0}\)}%
\end{pgfscope}%
\begin{pgfscope}%
\pgfpathrectangle{\pgfqpoint{0.562500in}{0.275000in}}{\pgfqpoint{3.487500in}{1.925000in}}%
\pgfusepath{clip}%
\pgfsetrectcap%
\pgfsetroundjoin%
\pgfsetlinewidth{0.803000pt}%
\definecolor{currentstroke}{rgb}{0.690196,0.690196,0.690196}%
\pgfsetstrokecolor{currentstroke}%
\pgfsetdash{}{0pt}%
\pgfpathmoveto{\pgfqpoint{3.098864in}{0.275000in}}%
\pgfpathlineto{\pgfqpoint{3.098864in}{2.200000in}}%
\pgfusepath{stroke}%
\end{pgfscope}%
\begin{pgfscope}%
\pgfsetbuttcap%
\pgfsetroundjoin%
\definecolor{currentfill}{rgb}{0.000000,0.000000,0.000000}%
\pgfsetfillcolor{currentfill}%
\pgfsetlinewidth{0.803000pt}%
\definecolor{currentstroke}{rgb}{0.000000,0.000000,0.000000}%
\pgfsetstrokecolor{currentstroke}%
\pgfsetdash{}{0pt}%
\pgfsys@defobject{currentmarker}{\pgfqpoint{0.000000in}{-0.048611in}}{\pgfqpoint{0.000000in}{0.000000in}}{%
\pgfpathmoveto{\pgfqpoint{0.000000in}{0.000000in}}%
\pgfpathlineto{\pgfqpoint{0.000000in}{-0.048611in}}%
\pgfusepath{stroke,fill}%
}%
\begin{pgfscope}%
\pgfsys@transformshift{3.098864in}{0.275000in}%
\pgfsys@useobject{currentmarker}{}%
\end{pgfscope}%
\end{pgfscope}%
\begin{pgfscope}%
\definecolor{textcolor}{rgb}{0.000000,0.000000,0.000000}%
\pgfsetstrokecolor{textcolor}%
\pgfsetfillcolor{textcolor}%
\pgftext[x=3.098864in,y=0.177778in,,top]{\color{textcolor}\rmfamily\fontsize{10.000000}{12.000000}\selectfont \(\displaystyle {0.5}\)}%
\end{pgfscope}%
\begin{pgfscope}%
\pgfpathrectangle{\pgfqpoint{0.562500in}{0.275000in}}{\pgfqpoint{3.487500in}{1.925000in}}%
\pgfusepath{clip}%
\pgfsetrectcap%
\pgfsetroundjoin%
\pgfsetlinewidth{0.803000pt}%
\definecolor{currentstroke}{rgb}{0.690196,0.690196,0.690196}%
\pgfsetstrokecolor{currentstroke}%
\pgfsetdash{}{0pt}%
\pgfpathmoveto{\pgfqpoint{3.891477in}{0.275000in}}%
\pgfpathlineto{\pgfqpoint{3.891477in}{2.200000in}}%
\pgfusepath{stroke}%
\end{pgfscope}%
\begin{pgfscope}%
\pgfsetbuttcap%
\pgfsetroundjoin%
\definecolor{currentfill}{rgb}{0.000000,0.000000,0.000000}%
\pgfsetfillcolor{currentfill}%
\pgfsetlinewidth{0.803000pt}%
\definecolor{currentstroke}{rgb}{0.000000,0.000000,0.000000}%
\pgfsetstrokecolor{currentstroke}%
\pgfsetdash{}{0pt}%
\pgfsys@defobject{currentmarker}{\pgfqpoint{0.000000in}{-0.048611in}}{\pgfqpoint{0.000000in}{0.000000in}}{%
\pgfpathmoveto{\pgfqpoint{0.000000in}{0.000000in}}%
\pgfpathlineto{\pgfqpoint{0.000000in}{-0.048611in}}%
\pgfusepath{stroke,fill}%
}%
\begin{pgfscope}%
\pgfsys@transformshift{3.891477in}{0.275000in}%
\pgfsys@useobject{currentmarker}{}%
\end{pgfscope}%
\end{pgfscope}%
\begin{pgfscope}%
\definecolor{textcolor}{rgb}{0.000000,0.000000,0.000000}%
\pgfsetstrokecolor{textcolor}%
\pgfsetfillcolor{textcolor}%
\pgftext[x=3.891477in,y=0.177778in,,top]{\color{textcolor}\rmfamily\fontsize{10.000000}{12.000000}\selectfont \(\displaystyle {1.0}\)}%
\end{pgfscope}%
\begin{pgfscope}%
\pgfpathrectangle{\pgfqpoint{0.562500in}{0.275000in}}{\pgfqpoint{3.487500in}{1.925000in}}%
\pgfusepath{clip}%
\pgfsetrectcap%
\pgfsetroundjoin%
\pgfsetlinewidth{0.803000pt}%
\definecolor{currentstroke}{rgb}{0.690196,0.690196,0.690196}%
\pgfsetstrokecolor{currentstroke}%
\pgfsetdash{}{0pt}%
\pgfpathmoveto{\pgfqpoint{0.562500in}{0.362500in}}%
\pgfpathlineto{\pgfqpoint{4.050000in}{0.362500in}}%
\pgfusepath{stroke}%
\end{pgfscope}%
\begin{pgfscope}%
\pgfsetbuttcap%
\pgfsetroundjoin%
\definecolor{currentfill}{rgb}{0.000000,0.000000,0.000000}%
\pgfsetfillcolor{currentfill}%
\pgfsetlinewidth{0.803000pt}%
\definecolor{currentstroke}{rgb}{0.000000,0.000000,0.000000}%
\pgfsetstrokecolor{currentstroke}%
\pgfsetdash{}{0pt}%
\pgfsys@defobject{currentmarker}{\pgfqpoint{-0.048611in}{0.000000in}}{\pgfqpoint{-0.000000in}{0.000000in}}{%
\pgfpathmoveto{\pgfqpoint{-0.000000in}{0.000000in}}%
\pgfpathlineto{\pgfqpoint{-0.048611in}{0.000000in}}%
\pgfusepath{stroke,fill}%
}%
\begin{pgfscope}%
\pgfsys@transformshift{0.562500in}{0.362500in}%
\pgfsys@useobject{currentmarker}{}%
\end{pgfscope}%
\end{pgfscope}%
\begin{pgfscope}%
\definecolor{textcolor}{rgb}{0.000000,0.000000,0.000000}%
\pgfsetstrokecolor{textcolor}%
\pgfsetfillcolor{textcolor}%
\pgftext[x=0.287808in, y=0.315799in, left, base]{\color{textcolor}\rmfamily\fontsize{10.000000}{12.000000}\selectfont \(\displaystyle {-1}\)}%
\end{pgfscope}%
\begin{pgfscope}%
\pgfpathrectangle{\pgfqpoint{0.562500in}{0.275000in}}{\pgfqpoint{3.487500in}{1.925000in}}%
\pgfusepath{clip}%
\pgfsetrectcap%
\pgfsetroundjoin%
\pgfsetlinewidth{0.803000pt}%
\definecolor{currentstroke}{rgb}{0.690196,0.690196,0.690196}%
\pgfsetstrokecolor{currentstroke}%
\pgfsetdash{}{0pt}%
\pgfpathmoveto{\pgfqpoint{0.562500in}{0.945833in}}%
\pgfpathlineto{\pgfqpoint{4.050000in}{0.945833in}}%
\pgfusepath{stroke}%
\end{pgfscope}%
\begin{pgfscope}%
\pgfsetbuttcap%
\pgfsetroundjoin%
\definecolor{currentfill}{rgb}{0.000000,0.000000,0.000000}%
\pgfsetfillcolor{currentfill}%
\pgfsetlinewidth{0.803000pt}%
\definecolor{currentstroke}{rgb}{0.000000,0.000000,0.000000}%
\pgfsetstrokecolor{currentstroke}%
\pgfsetdash{}{0pt}%
\pgfsys@defobject{currentmarker}{\pgfqpoint{-0.048611in}{0.000000in}}{\pgfqpoint{-0.000000in}{0.000000in}}{%
\pgfpathmoveto{\pgfqpoint{-0.000000in}{0.000000in}}%
\pgfpathlineto{\pgfqpoint{-0.048611in}{0.000000in}}%
\pgfusepath{stroke,fill}%
}%
\begin{pgfscope}%
\pgfsys@transformshift{0.562500in}{0.945833in}%
\pgfsys@useobject{currentmarker}{}%
\end{pgfscope}%
\end{pgfscope}%
\begin{pgfscope}%
\definecolor{textcolor}{rgb}{0.000000,0.000000,0.000000}%
\pgfsetstrokecolor{textcolor}%
\pgfsetfillcolor{textcolor}%
\pgftext[x=0.395833in, y=0.899132in, left, base]{\color{textcolor}\rmfamily\fontsize{10.000000}{12.000000}\selectfont \(\displaystyle {0}\)}%
\end{pgfscope}%
\begin{pgfscope}%
\pgfpathrectangle{\pgfqpoint{0.562500in}{0.275000in}}{\pgfqpoint{3.487500in}{1.925000in}}%
\pgfusepath{clip}%
\pgfsetrectcap%
\pgfsetroundjoin%
\pgfsetlinewidth{0.803000pt}%
\definecolor{currentstroke}{rgb}{0.690196,0.690196,0.690196}%
\pgfsetstrokecolor{currentstroke}%
\pgfsetdash{}{0pt}%
\pgfpathmoveto{\pgfqpoint{0.562500in}{1.529167in}}%
\pgfpathlineto{\pgfqpoint{4.050000in}{1.529167in}}%
\pgfusepath{stroke}%
\end{pgfscope}%
\begin{pgfscope}%
\pgfsetbuttcap%
\pgfsetroundjoin%
\definecolor{currentfill}{rgb}{0.000000,0.000000,0.000000}%
\pgfsetfillcolor{currentfill}%
\pgfsetlinewidth{0.803000pt}%
\definecolor{currentstroke}{rgb}{0.000000,0.000000,0.000000}%
\pgfsetstrokecolor{currentstroke}%
\pgfsetdash{}{0pt}%
\pgfsys@defobject{currentmarker}{\pgfqpoint{-0.048611in}{0.000000in}}{\pgfqpoint{-0.000000in}{0.000000in}}{%
\pgfpathmoveto{\pgfqpoint{-0.000000in}{0.000000in}}%
\pgfpathlineto{\pgfqpoint{-0.048611in}{0.000000in}}%
\pgfusepath{stroke,fill}%
}%
\begin{pgfscope}%
\pgfsys@transformshift{0.562500in}{1.529167in}%
\pgfsys@useobject{currentmarker}{}%
\end{pgfscope}%
\end{pgfscope}%
\begin{pgfscope}%
\definecolor{textcolor}{rgb}{0.000000,0.000000,0.000000}%
\pgfsetstrokecolor{textcolor}%
\pgfsetfillcolor{textcolor}%
\pgftext[x=0.395833in, y=1.482465in, left, base]{\color{textcolor}\rmfamily\fontsize{10.000000}{12.000000}\selectfont \(\displaystyle {1}\)}%
\end{pgfscope}%
\begin{pgfscope}%
\pgfpathrectangle{\pgfqpoint{0.562500in}{0.275000in}}{\pgfqpoint{3.487500in}{1.925000in}}%
\pgfusepath{clip}%
\pgfsetrectcap%
\pgfsetroundjoin%
\pgfsetlinewidth{0.803000pt}%
\definecolor{currentstroke}{rgb}{0.690196,0.690196,0.690196}%
\pgfsetstrokecolor{currentstroke}%
\pgfsetdash{}{0pt}%
\pgfpathmoveto{\pgfqpoint{0.562500in}{2.112500in}}%
\pgfpathlineto{\pgfqpoint{4.050000in}{2.112500in}}%
\pgfusepath{stroke}%
\end{pgfscope}%
\begin{pgfscope}%
\pgfsetbuttcap%
\pgfsetroundjoin%
\definecolor{currentfill}{rgb}{0.000000,0.000000,0.000000}%
\pgfsetfillcolor{currentfill}%
\pgfsetlinewidth{0.803000pt}%
\definecolor{currentstroke}{rgb}{0.000000,0.000000,0.000000}%
\pgfsetstrokecolor{currentstroke}%
\pgfsetdash{}{0pt}%
\pgfsys@defobject{currentmarker}{\pgfqpoint{-0.048611in}{0.000000in}}{\pgfqpoint{-0.000000in}{0.000000in}}{%
\pgfpathmoveto{\pgfqpoint{-0.000000in}{0.000000in}}%
\pgfpathlineto{\pgfqpoint{-0.048611in}{0.000000in}}%
\pgfusepath{stroke,fill}%
}%
\begin{pgfscope}%
\pgfsys@transformshift{0.562500in}{2.112500in}%
\pgfsys@useobject{currentmarker}{}%
\end{pgfscope}%
\end{pgfscope}%
\begin{pgfscope}%
\definecolor{textcolor}{rgb}{0.000000,0.000000,0.000000}%
\pgfsetstrokecolor{textcolor}%
\pgfsetfillcolor{textcolor}%
\pgftext[x=0.395833in, y=2.065799in, left, base]{\color{textcolor}\rmfamily\fontsize{10.000000}{12.000000}\selectfont \(\displaystyle {2}\)}%
\end{pgfscope}%
\begin{pgfscope}%
\pgfpathrectangle{\pgfqpoint{0.562500in}{0.275000in}}{\pgfqpoint{3.487500in}{1.925000in}}%
\pgfusepath{clip}%
\pgfsetrectcap%
\pgfsetroundjoin%
\pgfsetlinewidth{1.505625pt}%
\definecolor{currentstroke}{rgb}{0.121569,0.466667,0.705882}%
\pgfsetstrokecolor{currentstroke}%
\pgfsetdash{}{0pt}%
\pgfpathmoveto{\pgfqpoint{2.306250in}{0.945833in}}%
\pgfpathlineto{\pgfqpoint{2.306250in}{0.945833in}}%
\pgfusepath{stroke}%
\end{pgfscope}%
\begin{pgfscope}%
\pgfpathrectangle{\pgfqpoint{0.562500in}{0.275000in}}{\pgfqpoint{3.487500in}{1.925000in}}%
\pgfusepath{clip}%
\pgfsetrectcap%
\pgfsetroundjoin%
\pgfsetlinewidth{1.505625pt}%
\definecolor{currentstroke}{rgb}{1.000000,0.498039,0.054902}%
\pgfsetstrokecolor{currentstroke}%
\pgfsetdash{}{0pt}%
\pgfpathmoveto{\pgfqpoint{3.891477in}{0.362500in}}%
\pgfpathlineto{\pgfqpoint{3.781003in}{0.363501in}}%
\pgfpathlineto{\pgfqpoint{3.669965in}{0.366436in}}%
\pgfpathlineto{\pgfqpoint{3.556115in}{0.371251in}}%
\pgfpathlineto{\pgfqpoint{3.437544in}{0.377920in}}%
\pgfpathlineto{\pgfqpoint{3.247436in}{0.391416in}}%
\pgfpathlineto{\pgfqpoint{3.039148in}{0.409219in}}%
\pgfpathlineto{\pgfqpoint{2.806136in}{0.431656in}}%
\pgfpathlineto{\pgfqpoint{2.637728in}{0.449325in}}%
\pgfpathlineto{\pgfqpoint{2.465583in}{0.469110in}}%
\pgfpathlineto{\pgfqpoint{2.294772in}{0.490916in}}%
\pgfpathlineto{\pgfqpoint{2.129630in}{0.514610in}}%
\pgfpathlineto{\pgfqpoint{1.973751in}{0.540020in}}%
\pgfpathlineto{\pgfqpoint{1.829992in}{0.566932in}}%
\pgfpathlineto{\pgfqpoint{1.700471in}{0.595093in}}%
\pgfpathlineto{\pgfqpoint{1.641510in}{0.609553in}}%
\pgfpathlineto{\pgfqpoint{1.586568in}{0.624212in}}%
\pgfpathlineto{\pgfqpoint{1.535702in}{0.639027in}}%
\pgfpathlineto{\pgfqpoint{1.488923in}{0.653955in}}%
\pgfpathlineto{\pgfqpoint{1.446197in}{0.668946in}}%
\pgfpathlineto{\pgfqpoint{1.407440in}{0.683950in}}%
\pgfpathlineto{\pgfqpoint{1.372527in}{0.698915in}}%
\pgfpathlineto{\pgfqpoint{1.341283in}{0.713786in}}%
\pgfpathlineto{\pgfqpoint{1.313488in}{0.728504in}}%
\pgfpathlineto{\pgfqpoint{1.266531in}{0.757397in}}%
\pgfpathlineto{\pgfqpoint{1.228197in}{0.785632in}}%
\pgfpathlineto{\pgfqpoint{1.196762in}{0.813113in}}%
\pgfpathlineto{\pgfqpoint{1.170901in}{0.839765in}}%
\pgfpathlineto{\pgfqpoint{1.149599in}{0.865538in}}%
\pgfpathlineto{\pgfqpoint{1.132120in}{0.890388in}}%
\pgfpathlineto{\pgfqpoint{1.117843in}{0.914304in}}%
\pgfpathlineto{\pgfqpoint{1.106244in}{0.937294in}}%
\pgfpathlineto{\pgfqpoint{1.096959in}{0.959361in}}%
\pgfpathlineto{\pgfqpoint{1.089790in}{0.980508in}}%
\pgfpathlineto{\pgfqpoint{1.084698in}{1.000733in}}%
\pgfpathlineto{\pgfqpoint{1.081790in}{1.020031in}}%
\pgfpathlineto{\pgfqpoint{1.080813in}{1.038415in}}%
\pgfpathlineto{\pgfqpoint{1.081537in}{1.055903in}}%
\pgfpathlineto{\pgfqpoint{1.083881in}{1.072511in}}%
\pgfpathlineto{\pgfqpoint{1.087800in}{1.088254in}}%
\pgfpathlineto{\pgfqpoint{1.093279in}{1.103144in}}%
\pgfpathlineto{\pgfqpoint{1.100337in}{1.117194in}}%
\pgfpathlineto{\pgfqpoint{1.109031in}{1.130416in}}%
\pgfpathlineto{\pgfqpoint{1.119437in}{1.142824in}}%
\pgfpathlineto{\pgfqpoint{1.131486in}{1.154430in}}%
\pgfpathlineto{\pgfqpoint{1.145100in}{1.165241in}}%
\pgfpathlineto{\pgfqpoint{1.160239in}{1.175265in}}%
\pgfpathlineto{\pgfqpoint{1.185804in}{1.188845in}}%
\pgfpathlineto{\pgfqpoint{1.214939in}{1.200699in}}%
\pgfpathlineto{\pgfqpoint{1.247970in}{1.210852in}}%
\pgfpathlineto{\pgfqpoint{1.285409in}{1.219328in}}%
\pgfpathlineto{\pgfqpoint{1.327889in}{1.226149in}}%
\pgfpathlineto{\pgfqpoint{1.375411in}{1.231290in}}%
\pgfpathlineto{\pgfqpoint{1.428353in}{1.234725in}}%
\pgfpathlineto{\pgfqpoint{1.487312in}{1.236420in}}%
\pgfpathlineto{\pgfqpoint{1.552923in}{1.236331in}}%
\pgfpathlineto{\pgfqpoint{1.625852in}{1.234400in}}%
\pgfpathlineto{\pgfqpoint{1.735690in}{1.228839in}}%
\pgfpathlineto{\pgfqpoint{1.861545in}{1.219680in}}%
\pgfpathlineto{\pgfqpoint{2.005000in}{1.206681in}}%
\pgfpathlineto{\pgfqpoint{2.166482in}{1.189564in}}%
\pgfpathlineto{\pgfqpoint{2.343906in}{1.168128in}}%
\pgfpathlineto{\pgfqpoint{2.484391in}{1.149164in}}%
\pgfpathlineto{\pgfqpoint{2.626905in}{1.127795in}}%
\pgfpathlineto{\pgfqpoint{2.766529in}{1.104209in}}%
\pgfpathlineto{\pgfqpoint{2.855567in}{1.087397in}}%
\pgfpathlineto{\pgfqpoint{2.939661in}{1.069848in}}%
\pgfpathlineto{\pgfqpoint{3.017018in}{1.051693in}}%
\pgfpathlineto{\pgfqpoint{3.087013in}{1.033088in}}%
\pgfpathlineto{\pgfqpoint{3.149875in}{1.014196in}}%
\pgfpathlineto{\pgfqpoint{3.205851in}{0.995163in}}%
\pgfpathlineto{\pgfqpoint{3.255200in}{0.976124in}}%
\pgfpathlineto{\pgfqpoint{3.298201in}{0.957199in}}%
\pgfpathlineto{\pgfqpoint{3.335148in}{0.938496in}}%
\pgfpathlineto{\pgfqpoint{3.366348in}{0.920108in}}%
\pgfpathlineto{\pgfqpoint{3.392130in}{0.902115in}}%
\pgfpathlineto{\pgfqpoint{3.412833in}{0.884582in}}%
\pgfpathlineto{\pgfqpoint{3.428817in}{0.867564in}}%
\pgfpathlineto{\pgfqpoint{3.440458in}{0.851112in}}%
\pgfpathlineto{\pgfqpoint{3.448368in}{0.835290in}}%
\pgfpathlineto{\pgfqpoint{3.453279in}{0.820119in}}%
\pgfpathlineto{\pgfqpoint{3.455772in}{0.805612in}}%
\pgfpathlineto{\pgfqpoint{3.456269in}{0.791780in}}%
\pgfpathlineto{\pgfqpoint{3.455029in}{0.778632in}}%
\pgfpathlineto{\pgfqpoint{3.452154in}{0.766171in}}%
\pgfpathlineto{\pgfqpoint{3.447582in}{0.754402in}}%
\pgfpathlineto{\pgfqpoint{3.441095in}{0.743322in}}%
\pgfpathlineto{\pgfqpoint{3.432312in}{0.732927in}}%
\pgfpathlineto{\pgfqpoint{3.420694in}{0.723212in}}%
\pgfpathlineto{\pgfqpoint{3.406349in}{0.714186in}}%
\pgfpathlineto{\pgfqpoint{3.390104in}{0.705872in}}%
\pgfpathlineto{\pgfqpoint{3.362198in}{0.694739in}}%
\pgfpathlineto{\pgfqpoint{3.329982in}{0.685214in}}%
\pgfpathlineto{\pgfqpoint{3.293282in}{0.677304in}}%
\pgfpathlineto{\pgfqpoint{3.251812in}{0.671014in}}%
\pgfpathlineto{\pgfqpoint{3.205172in}{0.666354in}}%
\pgfpathlineto{\pgfqpoint{3.152945in}{0.663334in}}%
\pgfpathlineto{\pgfqpoint{3.094925in}{0.661989in}}%
\pgfpathlineto{\pgfqpoint{3.030182in}{0.662381in}}%
\pgfpathlineto{\pgfqpoint{2.931883in}{0.665727in}}%
\pgfpathlineto{\pgfqpoint{2.818410in}{0.672475in}}%
\pgfpathlineto{\pgfqpoint{2.688627in}{0.682832in}}%
\pgfpathlineto{\pgfqpoint{2.541921in}{0.697031in}}%
\pgfpathlineto{\pgfqpoint{2.378174in}{0.715326in}}%
\pgfpathlineto{\pgfqpoint{2.198222in}{0.737993in}}%
\pgfpathlineto{\pgfqpoint{2.058427in}{0.757810in}}%
\pgfpathlineto{\pgfqpoint{1.919952in}{0.779859in}}%
\pgfpathlineto{\pgfqpoint{1.787069in}{0.803901in}}%
\pgfpathlineto{\pgfqpoint{1.703430in}{0.820884in}}%
\pgfpathlineto{\pgfqpoint{1.624867in}{0.838506in}}%
\pgfpathlineto{\pgfqpoint{1.552208in}{0.856643in}}%
\pgfpathlineto{\pgfqpoint{1.486164in}{0.875158in}}%
\pgfpathlineto{\pgfqpoint{1.427326in}{0.893900in}}%
\pgfpathlineto{\pgfqpoint{1.376131in}{0.912704in}}%
\pgfpathlineto{\pgfqpoint{1.332106in}{0.931434in}}%
\pgfpathlineto{\pgfqpoint{1.294458in}{0.949984in}}%
\pgfpathlineto{\pgfqpoint{1.262565in}{0.968255in}}%
\pgfpathlineto{\pgfqpoint{1.235890in}{0.986161in}}%
\pgfpathlineto{\pgfqpoint{1.213991in}{1.003630in}}%
\pgfpathlineto{\pgfqpoint{1.196514in}{1.020596in}}%
\pgfpathlineto{\pgfqpoint{1.183100in}{1.037008in}}%
\pgfpathlineto{\pgfqpoint{1.173177in}{1.052827in}}%
\pgfpathlineto{\pgfqpoint{1.166351in}{1.068025in}}%
\pgfpathlineto{\pgfqpoint{1.162313in}{1.082579in}}%
\pgfpathlineto{\pgfqpoint{1.160838in}{1.096471in}}%
\pgfpathlineto{\pgfqpoint{1.161779in}{1.109688in}}%
\pgfpathlineto{\pgfqpoint{1.165070in}{1.122222in}}%
\pgfpathlineto{\pgfqpoint{1.170641in}{1.134066in}}%
\pgfpathlineto{\pgfqpoint{1.178329in}{1.145218in}}%
\pgfpathlineto{\pgfqpoint{1.187992in}{1.155671in}}%
\pgfpathlineto{\pgfqpoint{1.199535in}{1.165423in}}%
\pgfpathlineto{\pgfqpoint{1.212898in}{1.174472in}}%
\pgfpathlineto{\pgfqpoint{1.228067in}{1.182817in}}%
\pgfpathlineto{\pgfqpoint{1.254269in}{1.194017in}}%
\pgfpathlineto{\pgfqpoint{1.284853in}{1.203641in}}%
\pgfpathlineto{\pgfqpoint{1.320275in}{1.211700in}}%
\pgfpathlineto{\pgfqpoint{1.360663in}{1.218183in}}%
\pgfpathlineto{\pgfqpoint{1.406134in}{1.223059in}}%
\pgfpathlineto{\pgfqpoint{1.457132in}{1.226298in}}%
\pgfpathlineto{\pgfqpoint{1.514159in}{1.227861in}}%
\pgfpathlineto{\pgfqpoint{1.577773in}{1.227700in}}%
\pgfpathlineto{\pgfqpoint{1.648591in}{1.225756in}}%
\pgfpathlineto{\pgfqpoint{1.755392in}{1.220271in}}%
\pgfpathlineto{\pgfqpoint{1.877909in}{1.211302in}}%
\pgfpathlineto{\pgfqpoint{2.017583in}{1.198591in}}%
\pgfpathlineto{\pgfqpoint{2.174690in}{1.181882in}}%
\pgfpathlineto{\pgfqpoint{2.347380in}{1.160974in}}%
\pgfpathlineto{\pgfqpoint{2.484178in}{1.142485in}}%
\pgfpathlineto{\pgfqpoint{2.623199in}{1.121653in}}%
\pgfpathlineto{\pgfqpoint{2.760018in}{1.098647in}}%
\pgfpathlineto{\pgfqpoint{2.847510in}{1.082236in}}%
\pgfpathlineto{\pgfqpoint{2.930185in}{1.065066in}}%
\pgfpathlineto{\pgfqpoint{3.007086in}{1.047290in}}%
\pgfpathlineto{\pgfqpoint{3.077574in}{1.029069in}}%
\pgfpathlineto{\pgfqpoint{3.141215in}{1.010556in}}%
\pgfpathlineto{\pgfqpoint{3.197783in}{0.991889in}}%
\pgfpathlineto{\pgfqpoint{3.247260in}{0.973197in}}%
\pgfpathlineto{\pgfqpoint{3.289833in}{0.954597in}}%
\pgfpathlineto{\pgfqpoint{3.325899in}{0.936195in}}%
\pgfpathlineto{\pgfqpoint{3.356060in}{0.918086in}}%
\pgfpathlineto{\pgfqpoint{3.381115in}{0.900352in}}%
\pgfpathlineto{\pgfqpoint{3.401292in}{0.883085in}}%
\pgfpathlineto{\pgfqpoint{3.417120in}{0.866339in}}%
\pgfpathlineto{\pgfqpoint{3.429418in}{0.850150in}}%
\pgfpathlineto{\pgfqpoint{3.438802in}{0.834546in}}%
\pgfpathlineto{\pgfqpoint{3.445687in}{0.819553in}}%
\pgfpathlineto{\pgfqpoint{3.450282in}{0.805191in}}%
\pgfpathlineto{\pgfqpoint{3.452597in}{0.791478in}}%
\pgfpathlineto{\pgfqpoint{3.452436in}{0.778426in}}%
\pgfpathlineto{\pgfqpoint{3.449401in}{0.766043in}}%
\pgfpathlineto{\pgfqpoint{3.443236in}{0.754337in}}%
\pgfpathlineto{\pgfqpoint{3.434890in}{0.743330in}}%
\pgfpathlineto{\pgfqpoint{3.424574in}{0.733026in}}%
\pgfpathlineto{\pgfqpoint{3.412363in}{0.723428in}}%
\pgfpathlineto{\pgfqpoint{3.398301in}{0.714536in}}%
\pgfpathlineto{\pgfqpoint{3.382398in}{0.706350in}}%
\pgfpathlineto{\pgfqpoint{3.355031in}{0.695396in}}%
\pgfpathlineto{\pgfqpoint{3.323248in}{0.686028in}}%
\pgfpathlineto{\pgfqpoint{3.286697in}{0.678237in}}%
\pgfpathlineto{\pgfqpoint{3.245286in}{0.672038in}}%
\pgfpathlineto{\pgfqpoint{3.198719in}{0.667455in}}%
\pgfpathlineto{\pgfqpoint{3.146534in}{0.664519in}}%
\pgfpathlineto{\pgfqpoint{3.088215in}{0.663270in}}%
\pgfpathlineto{\pgfqpoint{3.023191in}{0.663759in}}%
\pgfpathlineto{\pgfqpoint{2.924969in}{0.667221in}}%
\pgfpathlineto{\pgfqpoint{2.812066in}{0.674051in}}%
\pgfpathlineto{\pgfqpoint{2.682805in}{0.684462in}}%
\pgfpathlineto{\pgfqpoint{2.536036in}{0.698726in}}%
\pgfpathlineto{\pgfqpoint{2.372254in}{0.717080in}}%
\pgfpathlineto{\pgfqpoint{2.194374in}{0.739691in}}%
\pgfpathlineto{\pgfqpoint{2.055716in}{0.759442in}}%
\pgfpathlineto{\pgfqpoint{1.917100in}{0.781464in}}%
\pgfpathlineto{\pgfqpoint{1.783280in}{0.805528in}}%
\pgfpathlineto{\pgfqpoint{1.699361in}{0.822558in}}%
\pgfpathlineto{\pgfqpoint{1.621117in}{0.840236in}}%
\pgfpathlineto{\pgfqpoint{1.549202in}{0.858394in}}%
\pgfpathlineto{\pgfqpoint{1.484071in}{0.876878in}}%
\pgfpathlineto{\pgfqpoint{1.425981in}{0.895543in}}%
\pgfpathlineto{\pgfqpoint{1.374987in}{0.914259in}}%
\pgfpathlineto{\pgfqpoint{1.330944in}{0.932905in}}%
\pgfpathlineto{\pgfqpoint{1.293511in}{0.951372in}}%
\pgfpathlineto{\pgfqpoint{1.262143in}{0.969564in}}%
\pgfpathlineto{\pgfqpoint{1.236099in}{0.987396in}}%
\pgfpathlineto{\pgfqpoint{1.214820in}{1.004785in}}%
\pgfpathlineto{\pgfqpoint{1.198134in}{1.021658in}}%
\pgfpathlineto{\pgfqpoint{1.185190in}{1.037979in}}%
\pgfpathlineto{\pgfqpoint{1.175309in}{1.053719in}}%
\pgfpathlineto{\pgfqpoint{1.168009in}{1.068853in}}%
\pgfpathlineto{\pgfqpoint{1.163009in}{1.083357in}}%
\pgfpathlineto{\pgfqpoint{1.160231in}{1.097216in}}%
\pgfpathlineto{\pgfqpoint{1.159795in}{1.110416in}}%
\pgfpathlineto{\pgfqpoint{1.162022in}{1.122948in}}%
\pgfpathlineto{\pgfqpoint{1.167396in}{1.134807in}}%
\pgfpathlineto{\pgfqpoint{1.175299in}{1.145974in}}%
\pgfpathlineto{\pgfqpoint{1.185200in}{1.156437in}}%
\pgfpathlineto{\pgfqpoint{1.197015in}{1.166196in}}%
\pgfpathlineto{\pgfqpoint{1.210692in}{1.175249in}}%
\pgfpathlineto{\pgfqpoint{1.226212in}{1.183594in}}%
\pgfpathlineto{\pgfqpoint{1.252981in}{1.194787in}}%
\pgfpathlineto{\pgfqpoint{1.284105in}{1.204391in}}%
\pgfpathlineto{\pgfqpoint{1.319919in}{1.212415in}}%
\pgfpathlineto{\pgfqpoint{1.360619in}{1.218851in}}%
\pgfpathlineto{\pgfqpoint{1.406398in}{1.223676in}}%
\pgfpathlineto{\pgfqpoint{1.457722in}{1.226859in}}%
\pgfpathlineto{\pgfqpoint{1.515106in}{1.228362in}}%
\pgfpathlineto{\pgfqpoint{1.579116in}{1.228135in}}%
\pgfpathlineto{\pgfqpoint{1.650370in}{1.226119in}}%
\pgfpathlineto{\pgfqpoint{1.757804in}{1.220526in}}%
\pgfpathlineto{\pgfqpoint{1.881010in}{1.211433in}}%
\pgfpathlineto{\pgfqpoint{2.021410in}{1.198589in}}%
\pgfpathlineto{\pgfqpoint{2.179287in}{1.181731in}}%
\pgfpathlineto{\pgfqpoint{2.352631in}{1.160665in}}%
\pgfpathlineto{\pgfqpoint{2.489834in}{1.142053in}}%
\pgfpathlineto{\pgfqpoint{2.629048in}{1.121098in}}%
\pgfpathlineto{\pgfqpoint{2.765799in}{1.097978in}}%
\pgfpathlineto{\pgfqpoint{2.853163in}{1.081498in}}%
\pgfpathlineto{\pgfqpoint{2.935626in}{1.064277in}}%
\pgfpathlineto{\pgfqpoint{3.012128in}{1.046449in}}%
\pgfpathlineto{\pgfqpoint{3.082124in}{1.028181in}}%
\pgfpathlineto{\pgfqpoint{3.145253in}{1.009627in}}%
\pgfpathlineto{\pgfqpoint{3.201338in}{0.990929in}}%
\pgfpathlineto{\pgfqpoint{3.250387in}{0.972217in}}%
\pgfpathlineto{\pgfqpoint{3.292593in}{0.953609in}}%
\pgfpathlineto{\pgfqpoint{3.328333in}{0.935209in}}%
\pgfpathlineto{\pgfqpoint{3.358167in}{0.917110in}}%
\pgfpathlineto{\pgfqpoint{3.382840in}{0.899394in}}%
\pgfpathlineto{\pgfqpoint{3.402867in}{0.882139in}}%
\pgfpathlineto{\pgfqpoint{3.418433in}{0.865414in}}%
\pgfpathlineto{\pgfqpoint{3.430374in}{0.849252in}}%
\pgfpathlineto{\pgfqpoint{3.439351in}{0.833681in}}%
\pgfpathlineto{\pgfqpoint{3.445830in}{0.818724in}}%
\pgfpathlineto{\pgfqpoint{3.450080in}{0.804402in}}%
\pgfpathlineto{\pgfqpoint{3.452176in}{0.790730in}}%
\pgfpathlineto{\pgfqpoint{3.451994in}{0.777720in}}%
\pgfpathlineto{\pgfqpoint{3.449216in}{0.765380in}}%
\pgfpathlineto{\pgfqpoint{3.443344in}{0.753712in}}%
\pgfpathlineto{\pgfqpoint{3.434887in}{0.742736in}}%
\pgfpathlineto{\pgfqpoint{3.424445in}{0.732463in}}%
\pgfpathlineto{\pgfqpoint{3.412093in}{0.722896in}}%
\pgfpathlineto{\pgfqpoint{3.397879in}{0.714034in}}%
\pgfpathlineto{\pgfqpoint{3.381819in}{0.705881in}}%
\pgfpathlineto{\pgfqpoint{3.354226in}{0.694976in}}%
\pgfpathlineto{\pgfqpoint{3.322262in}{0.685661in}}%
\pgfpathlineto{\pgfqpoint{3.285592in}{0.677930in}}%
\pgfpathlineto{\pgfqpoint{3.243975in}{0.671788in}}%
\pgfpathlineto{\pgfqpoint{3.197227in}{0.667262in}}%
\pgfpathlineto{\pgfqpoint{3.144847in}{0.664382in}}%
\pgfpathlineto{\pgfqpoint{3.086295in}{0.663190in}}%
\pgfpathlineto{\pgfqpoint{3.020989in}{0.663737in}}%
\pgfpathlineto{\pgfqpoint{2.922325in}{0.667282in}}%
\pgfpathlineto{\pgfqpoint{2.808954in}{0.674207in}}%
\pgfpathlineto{\pgfqpoint{2.679210in}{0.684723in}}%
\pgfpathlineto{\pgfqpoint{2.531963in}{0.699095in}}%
\pgfpathlineto{\pgfqpoint{2.367668in}{0.717569in}}%
\pgfpathlineto{\pgfqpoint{2.189500in}{0.740299in}}%
\pgfpathlineto{\pgfqpoint{2.050628in}{0.760140in}}%
\pgfpathlineto{\pgfqpoint{1.912108in}{0.782246in}}%
\pgfpathlineto{\pgfqpoint{1.778627in}{0.806381in}}%
\pgfpathlineto{\pgfqpoint{1.694774in}{0.823433in}}%
\pgfpathlineto{\pgfqpoint{1.616902in}{0.841117in}}%
\pgfpathlineto{\pgfqpoint{1.546154in}{0.859291in}}%
\pgfpathlineto{\pgfqpoint{1.482367in}{0.877795in}}%
\pgfpathlineto{\pgfqpoint{1.425321in}{0.896477in}}%
\pgfpathlineto{\pgfqpoint{1.374777in}{0.915203in}}%
\pgfpathlineto{\pgfqpoint{1.330481in}{0.933850in}}%
\pgfpathlineto{\pgfqpoint{1.292163in}{0.952309in}}%
\pgfpathlineto{\pgfqpoint{1.259535in}{0.970486in}}%
\pgfpathlineto{\pgfqpoint{1.232295in}{0.988298in}}%
\pgfpathlineto{\pgfqpoint{1.210123in}{1.005678in}}%
\pgfpathlineto{\pgfqpoint{1.192684in}{1.022571in}}%
\pgfpathlineto{\pgfqpoint{1.179624in}{1.038937in}}%
\pgfpathlineto{\pgfqpoint{1.170530in}{1.054715in}}%
\pgfpathlineto{\pgfqpoint{1.164752in}{1.069859in}}%
\pgfpathlineto{\pgfqpoint{1.161683in}{1.084350in}}%
\pgfpathlineto{\pgfqpoint{1.160860in}{1.098176in}}%
\pgfpathlineto{\pgfqpoint{1.161958in}{1.111327in}}%
\pgfpathlineto{\pgfqpoint{1.164792in}{1.123794in}}%
\pgfpathlineto{\pgfqpoint{1.169318in}{1.135574in}}%
\pgfpathlineto{\pgfqpoint{1.175632in}{1.146667in}}%
\pgfpathlineto{\pgfqpoint{1.183969in}{1.157073in}}%
\pgfpathlineto{\pgfqpoint{1.194705in}{1.166800in}}%
\pgfpathlineto{\pgfqpoint{1.208285in}{1.175854in}}%
\pgfpathlineto{\pgfqpoint{1.224087in}{1.184212in}}%
\pgfpathlineto{\pgfqpoint{1.251374in}{1.195422in}}%
\pgfpathlineto{\pgfqpoint{1.283007in}{1.205037in}}%
\pgfpathlineto{\pgfqpoint{1.319134in}{1.213049in}}%
\pgfpathlineto{\pgfqpoint{1.360011in}{1.219447in}}%
\pgfpathlineto{\pgfqpoint{1.406004in}{1.224221in}}%
\pgfpathlineto{\pgfqpoint{1.457587in}{1.227355in}}%
\pgfpathlineto{\pgfqpoint{1.514962in}{1.228832in}}%
\pgfpathlineto{\pgfqpoint{1.578780in}{1.228591in}}%
\pgfpathlineto{\pgfqpoint{1.650120in}{1.226555in}}%
\pgfpathlineto{\pgfqpoint{1.758364in}{1.220906in}}%
\pgfpathlineto{\pgfqpoint{1.882807in}{1.211714in}}%
\pgfpathlineto{\pgfqpoint{2.024064in}{1.198761in}}%
\pgfpathlineto{\pgfqpoint{2.182012in}{1.181819in}}%
\pgfpathlineto{\pgfqpoint{2.355890in}{1.160637in}}%
\pgfpathlineto{\pgfqpoint{2.494486in}{1.141843in}}%
\pgfpathlineto{\pgfqpoint{2.634352in}{1.120717in}}%
\pgfpathlineto{\pgfqpoint{2.770468in}{1.097499in}}%
\pgfpathlineto{\pgfqpoint{2.857013in}{1.081003in}}%
\pgfpathlineto{\pgfqpoint{2.938947in}{1.063808in}}%
\pgfpathlineto{\pgfqpoint{3.015372in}{1.046029in}}%
\pgfpathlineto{\pgfqpoint{3.085550in}{1.027788in}}%
\pgfpathlineto{\pgfqpoint{3.148899in}{1.009218in}}%
\pgfpathlineto{\pgfqpoint{3.204996in}{0.990467in}}%
\pgfpathlineto{\pgfqpoint{3.253581in}{0.971688in}}%
\pgfpathlineto{\pgfqpoint{3.295014in}{0.953032in}}%
\pgfpathlineto{\pgfqpoint{3.330221in}{0.934599in}}%
\pgfpathlineto{\pgfqpoint{3.359980in}{0.916478in}}%
\pgfpathlineto{\pgfqpoint{3.384907in}{0.898743in}}%
\pgfpathlineto{\pgfqpoint{3.405456in}{0.881465in}}%
\pgfpathlineto{\pgfqpoint{3.421918in}{0.864700in}}%
\pgfpathlineto{\pgfqpoint{3.434424in}{0.848498in}}%
\pgfpathlineto{\pgfqpoint{3.443112in}{0.832899in}}%
\pgfpathlineto{\pgfqpoint{3.448699in}{0.817936in}}%
\pgfpathlineto{\pgfqpoint{3.451560in}{0.803626in}}%
\pgfpathlineto{\pgfqpoint{3.451955in}{0.789985in}}%
\pgfpathlineto{\pgfqpoint{3.450071in}{0.777024in}}%
\pgfpathlineto{\pgfqpoint{3.446026in}{0.764751in}}%
\pgfpathlineto{\pgfqpoint{3.439866in}{0.753171in}}%
\pgfpathlineto{\pgfqpoint{3.431568in}{0.742285in}}%
\pgfpathlineto{\pgfqpoint{3.421167in}{0.732095in}}%
\pgfpathlineto{\pgfqpoint{3.408791in}{0.722603in}}%
\pgfpathlineto{\pgfqpoint{3.394507in}{0.713811in}}%
\pgfpathlineto{\pgfqpoint{3.378353in}{0.705723in}}%
\pgfpathlineto{\pgfqpoint{3.350624in}{0.694912in}}%
\pgfpathlineto{\pgfqpoint{3.318602in}{0.685691in}}%
\pgfpathlineto{\pgfqpoint{3.282044in}{0.678063in}}%
\pgfpathlineto{\pgfqpoint{3.240560in}{0.672028in}}%
\pgfpathlineto{\pgfqpoint{3.193693in}{0.667588in}}%
\pgfpathlineto{\pgfqpoint{3.141329in}{0.664780in}}%
\pgfpathlineto{\pgfqpoint{3.082814in}{0.663652in}}%
\pgfpathlineto{\pgfqpoint{3.017426in}{0.664259in}}%
\pgfpathlineto{\pgfqpoint{2.918391in}{0.667882in}}%
\pgfpathlineto{\pgfqpoint{2.804485in}{0.674891in}}%
\pgfpathlineto{\pgfqpoint{2.674406in}{0.685503in}}%
\pgfpathlineto{\pgfqpoint{2.526995in}{0.699966in}}%
\pgfpathlineto{\pgfqpoint{2.361669in}{0.718499in}}%
\pgfpathlineto{\pgfqpoint{2.228655in}{0.735209in}}%
\pgfpathlineto{\pgfqpoint{2.091308in}{0.754342in}}%
\pgfpathlineto{\pgfqpoint{1.953168in}{0.775796in}}%
\pgfpathlineto{\pgfqpoint{1.818225in}{0.799362in}}%
\pgfpathlineto{\pgfqpoint{1.732210in}{0.816094in}}%
\pgfpathlineto{\pgfqpoint{1.651003in}{0.833507in}}%
\pgfpathlineto{\pgfqpoint{1.576114in}{0.851459in}}%
\pgfpathlineto{\pgfqpoint{1.508650in}{0.869803in}}%
\pgfpathlineto{\pgfqpoint{1.448460in}{0.888397in}}%
\pgfpathlineto{\pgfqpoint{1.395198in}{0.907102in}}%
\pgfpathlineto{\pgfqpoint{1.348515in}{0.925793in}}%
\pgfpathlineto{\pgfqpoint{1.308069in}{0.944353in}}%
\pgfpathlineto{\pgfqpoint{1.273516in}{0.962679in}}%
\pgfpathlineto{\pgfqpoint{1.244514in}{0.980679in}}%
\pgfpathlineto{\pgfqpoint{1.220726in}{0.998272in}}%
\pgfpathlineto{\pgfqpoint{1.201814in}{1.015389in}}%
\pgfpathlineto{\pgfqpoint{1.187346in}{1.031967in}}%
\pgfpathlineto{\pgfqpoint{1.176621in}{1.047960in}}%
\pgfpathlineto{\pgfqpoint{1.169038in}{1.063342in}}%
\pgfpathlineto{\pgfqpoint{1.164135in}{1.078089in}}%
\pgfpathlineto{\pgfqpoint{1.161587in}{1.092185in}}%
\pgfpathlineto{\pgfqpoint{1.161211in}{1.105615in}}%
\pgfpathlineto{\pgfqpoint{1.162963in}{1.118370in}}%
\pgfpathlineto{\pgfqpoint{1.166937in}{1.130443in}}%
\pgfpathlineto{\pgfqpoint{1.173367in}{1.141833in}}%
\pgfpathlineto{\pgfqpoint{1.182380in}{1.152538in}}%
\pgfpathlineto{\pgfqpoint{1.193434in}{1.162543in}}%
\pgfpathlineto{\pgfqpoint{1.206413in}{1.171846in}}%
\pgfpathlineto{\pgfqpoint{1.221271in}{1.180445in}}%
\pgfpathlineto{\pgfqpoint{1.247052in}{1.192021in}}%
\pgfpathlineto{\pgfqpoint{1.277088in}{1.202006in}}%
\pgfpathlineto{\pgfqpoint{1.311585in}{1.210401in}}%
\pgfpathlineto{\pgfqpoint{1.350891in}{1.217207in}}%
\pgfpathlineto{\pgfqpoint{1.395345in}{1.222419in}}%
\pgfpathlineto{\pgfqpoint{1.445116in}{1.226006in}}%
\pgfpathlineto{\pgfqpoint{1.500805in}{1.227930in}}%
\pgfpathlineto{\pgfqpoint{1.563036in}{1.228142in}}%
\pgfpathlineto{\pgfqpoint{1.632439in}{1.226584in}}%
\pgfpathlineto{\pgfqpoint{1.737218in}{1.221629in}}%
\pgfpathlineto{\pgfqpoint{1.857360in}{1.213205in}}%
\pgfpathlineto{\pgfqpoint{1.994359in}{1.201080in}}%
\pgfpathlineto{\pgfqpoint{2.148923in}{1.184997in}}%
\pgfpathlineto{\pgfqpoint{2.319611in}{1.164727in}}%
\pgfpathlineto{\pgfqpoint{2.455516in}{1.146721in}}%
\pgfpathlineto{\pgfqpoint{2.501725in}{1.140189in}}%
\pgfpathlineto{\pgfqpoint{2.501725in}{1.140189in}}%
\pgfusepath{stroke}%
\end{pgfscope}%
\begin{pgfscope}%
\pgfpathrectangle{\pgfqpoint{0.562500in}{0.275000in}}{\pgfqpoint{3.487500in}{1.925000in}}%
\pgfusepath{clip}%
\pgfsetrectcap%
\pgfsetroundjoin%
\pgfsetlinewidth{1.505625pt}%
\definecolor{currentstroke}{rgb}{0.172549,0.627451,0.172549}%
\pgfsetstrokecolor{currentstroke}%
\pgfsetdash{}{0pt}%
\pgfpathmoveto{\pgfqpoint{3.891477in}{2.112500in}}%
\pgfpathlineto{\pgfqpoint{3.425176in}{2.071661in}}%
\pgfpathlineto{\pgfqpoint{3.203906in}{2.034784in}}%
\pgfpathlineto{\pgfqpoint{3.082401in}{2.000315in}}%
\pgfpathlineto{\pgfqpoint{3.014881in}{1.967535in}}%
\pgfpathlineto{\pgfqpoint{2.977831in}{1.936024in}}%
\pgfpathlineto{\pgfqpoint{2.960451in}{1.905538in}}%
\pgfpathlineto{\pgfqpoint{2.955053in}{1.875901in}}%
\pgfpathlineto{\pgfqpoint{2.957880in}{1.847005in}}%
\pgfpathlineto{\pgfqpoint{2.966493in}{1.818773in}}%
\pgfpathlineto{\pgfqpoint{2.978799in}{1.791142in}}%
\pgfpathlineto{\pgfqpoint{2.993505in}{1.764066in}}%
\pgfpathlineto{\pgfqpoint{3.010005in}{1.737518in}}%
\pgfpathlineto{\pgfqpoint{3.046163in}{1.685908in}}%
\pgfpathlineto{\pgfqpoint{3.102510in}{1.611927in}}%
\pgfpathlineto{\pgfqpoint{3.274511in}{1.392032in}}%
\pgfpathlineto{\pgfqpoint{3.331256in}{1.314558in}}%
\pgfpathlineto{\pgfqpoint{3.381306in}{1.242659in}}%
\pgfpathlineto{\pgfqpoint{3.413908in}{1.192152in}}%
\pgfpathlineto{\pgfqpoint{3.442136in}{1.144428in}}%
\pgfpathlineto{\pgfqpoint{3.466448in}{1.099435in}}%
\pgfpathlineto{\pgfqpoint{3.487533in}{1.057154in}}%
\pgfpathlineto{\pgfqpoint{3.505072in}{1.017438in}}%
\pgfpathlineto{\pgfqpoint{3.519075in}{0.980161in}}%
\pgfpathlineto{\pgfqpoint{3.529586in}{0.945235in}}%
\pgfpathlineto{\pgfqpoint{3.536612in}{0.912597in}}%
\pgfpathlineto{\pgfqpoint{3.540112in}{0.882202in}}%
\pgfpathlineto{\pgfqpoint{3.540208in}{0.853966in}}%
\pgfpathlineto{\pgfqpoint{3.537160in}{0.827806in}}%
\pgfpathlineto{\pgfqpoint{3.533427in}{0.811485in}}%
\pgfpathlineto{\pgfqpoint{3.528308in}{0.796039in}}%
\pgfpathlineto{\pgfqpoint{3.521738in}{0.781451in}}%
\pgfpathlineto{\pgfqpoint{3.513614in}{0.767709in}}%
\pgfpathlineto{\pgfqpoint{3.503797in}{0.754801in}}%
\pgfpathlineto{\pgfqpoint{3.492263in}{0.742712in}}%
\pgfpathlineto{\pgfqpoint{3.479122in}{0.731429in}}%
\pgfpathlineto{\pgfqpoint{3.464436in}{0.720945in}}%
\pgfpathlineto{\pgfqpoint{3.439560in}{0.706697in}}%
\pgfpathlineto{\pgfqpoint{3.411212in}{0.694199in}}%
\pgfpathlineto{\pgfqpoint{3.379171in}{0.683424in}}%
\pgfpathlineto{\pgfqpoint{3.343044in}{0.674351in}}%
\pgfpathlineto{\pgfqpoint{3.302267in}{0.666956in}}%
\pgfpathlineto{\pgfqpoint{3.256278in}{0.661228in}}%
\pgfpathlineto{\pgfqpoint{3.205073in}{0.657199in}}%
\pgfpathlineto{\pgfqpoint{3.148079in}{0.654901in}}%
\pgfpathlineto{\pgfqpoint{3.084629in}{0.654376in}}%
\pgfpathlineto{\pgfqpoint{3.014046in}{0.655675in}}%
\pgfpathlineto{\pgfqpoint{2.907646in}{0.660364in}}%
\pgfpathlineto{\pgfqpoint{2.785675in}{0.668608in}}%
\pgfpathlineto{\pgfqpoint{2.646457in}{0.680639in}}%
\pgfpathlineto{\pgfqpoint{2.489124in}{0.696722in}}%
\pgfpathlineto{\pgfqpoint{2.314887in}{0.717110in}}%
\pgfpathlineto{\pgfqpoint{2.175816in}{0.735285in}}%
\pgfpathlineto{\pgfqpoint{2.033208in}{0.755900in}}%
\pgfpathlineto{\pgfqpoint{1.891579in}{0.778836in}}%
\pgfpathlineto{\pgfqpoint{1.800463in}{0.795274in}}%
\pgfpathlineto{\pgfqpoint{1.713721in}{0.812498in}}%
\pgfpathlineto{\pgfqpoint{1.632549in}{0.830385in}}%
\pgfpathlineto{\pgfqpoint{1.557887in}{0.848801in}}%
\pgfpathlineto{\pgfqpoint{1.490415in}{0.867602in}}%
\pgfpathlineto{\pgfqpoint{1.430586in}{0.886628in}}%
\pgfpathlineto{\pgfqpoint{1.378490in}{0.905720in}}%
\pgfpathlineto{\pgfqpoint{1.333394in}{0.924751in}}%
\pgfpathlineto{\pgfqpoint{1.294582in}{0.943615in}}%
\pgfpathlineto{\pgfqpoint{1.261464in}{0.962211in}}%
\pgfpathlineto{\pgfqpoint{1.233580in}{0.980450in}}%
\pgfpathlineto{\pgfqpoint{1.210593in}{0.998255in}}%
\pgfpathlineto{\pgfqpoint{1.192298in}{1.015556in}}%
\pgfpathlineto{\pgfqpoint{1.178514in}{1.032294in}}%
\pgfpathlineto{\pgfqpoint{1.168414in}{1.048426in}}%
\pgfpathlineto{\pgfqpoint{1.161438in}{1.063925in}}%
\pgfpathlineto{\pgfqpoint{1.157206in}{1.078770in}}%
\pgfpathlineto{\pgfqpoint{1.155441in}{1.092945in}}%
\pgfpathlineto{\pgfqpoint{1.155970in}{1.106437in}}%
\pgfpathlineto{\pgfqpoint{1.158720in}{1.119237in}}%
\pgfpathlineto{\pgfqpoint{1.163720in}{1.131340in}}%
\pgfpathlineto{\pgfqpoint{1.171038in}{1.142746in}}%
\pgfpathlineto{\pgfqpoint{1.180437in}{1.153448in}}%
\pgfpathlineto{\pgfqpoint{1.191784in}{1.163444in}}%
\pgfpathlineto{\pgfqpoint{1.205005in}{1.172732in}}%
\pgfpathlineto{\pgfqpoint{1.220059in}{1.181311in}}%
\pgfpathlineto{\pgfqpoint{1.246071in}{1.192849in}}%
\pgfpathlineto{\pgfqpoint{1.276308in}{1.202792in}}%
\pgfpathlineto{\pgfqpoint{1.311046in}{1.211142in}}%
\pgfpathlineto{\pgfqpoint{1.350720in}{1.217908in}}%
\pgfpathlineto{\pgfqpoint{1.395482in}{1.223074in}}%
\pgfpathlineto{\pgfqpoint{1.445636in}{1.226605in}}%
\pgfpathlineto{\pgfqpoint{1.501745in}{1.228467in}}%
\pgfpathlineto{\pgfqpoint{1.564396in}{1.228609in}}%
\pgfpathlineto{\pgfqpoint{1.634204in}{1.226975in}}%
\pgfpathlineto{\pgfqpoint{1.739526in}{1.221909in}}%
\pgfpathlineto{\pgfqpoint{1.860301in}{1.213364in}}%
\pgfpathlineto{\pgfqpoint{1.998074in}{1.201110in}}%
\pgfpathlineto{\pgfqpoint{2.153445in}{1.184876in}}%
\pgfpathlineto{\pgfqpoint{2.324847in}{1.164446in}}%
\pgfpathlineto{\pgfqpoint{2.461144in}{1.146317in}}%
\pgfpathlineto{\pgfqpoint{2.600579in}{1.125812in}}%
\pgfpathlineto{\pgfqpoint{2.738374in}{1.103092in}}%
\pgfpathlineto{\pgfqpoint{2.869536in}{1.078437in}}%
\pgfpathlineto{\pgfqpoint{2.951163in}{1.061102in}}%
\pgfpathlineto{\pgfqpoint{3.026865in}{1.043195in}}%
\pgfpathlineto{\pgfqpoint{3.095689in}{1.024863in}}%
\pgfpathlineto{\pgfqpoint{3.156971in}{1.006264in}}%
\pgfpathlineto{\pgfqpoint{3.211078in}{0.987537in}}%
\pgfpathlineto{\pgfqpoint{3.258489in}{0.968811in}}%
\pgfpathlineto{\pgfqpoint{3.299644in}{0.950199in}}%
\pgfpathlineto{\pgfqpoint{3.334941in}{0.931808in}}%
\pgfpathlineto{\pgfqpoint{3.364737in}{0.913730in}}%
\pgfpathlineto{\pgfqpoint{3.389345in}{0.896048in}}%
\pgfpathlineto{\pgfqpoint{3.409041in}{0.878836in}}%
\pgfpathlineto{\pgfqpoint{3.424140in}{0.862154in}}%
\pgfpathlineto{\pgfqpoint{3.435390in}{0.846054in}}%
\pgfpathlineto{\pgfqpoint{3.443399in}{0.830564in}}%
\pgfpathlineto{\pgfqpoint{3.448618in}{0.815708in}}%
\pgfpathlineto{\pgfqpoint{3.451377in}{0.801504in}}%
\pgfpathlineto{\pgfqpoint{3.451879in}{0.787966in}}%
\pgfpathlineto{\pgfqpoint{3.450201in}{0.775106in}}%
\pgfpathlineto{\pgfqpoint{3.446296in}{0.762929in}}%
\pgfpathlineto{\pgfqpoint{3.439990in}{0.751439in}}%
\pgfpathlineto{\pgfqpoint{3.431240in}{0.740637in}}%
\pgfpathlineto{\pgfqpoint{3.420471in}{0.730535in}}%
\pgfpathlineto{\pgfqpoint{3.407774in}{0.721135in}}%
\pgfpathlineto{\pgfqpoint{3.393202in}{0.712439in}}%
\pgfpathlineto{\pgfqpoint{3.367863in}{0.700717in}}%
\pgfpathlineto{\pgfqpoint{3.338288in}{0.690584in}}%
\pgfpathlineto{\pgfqpoint{3.304271in}{0.682040in}}%
\pgfpathlineto{\pgfqpoint{3.265458in}{0.675083in}}%
\pgfpathlineto{\pgfqpoint{3.221511in}{0.669716in}}%
\pgfpathlineto{\pgfqpoint{3.172277in}{0.665972in}}%
\pgfpathlineto{\pgfqpoint{3.117176in}{0.663888in}}%
\pgfpathlineto{\pgfqpoint{3.055599in}{0.663509in}}%
\pgfpathlineto{\pgfqpoint{2.986925in}{0.664896in}}%
\pgfpathlineto{\pgfqpoint{2.883232in}{0.669609in}}%
\pgfpathlineto{\pgfqpoint{2.764285in}{0.677775in}}%
\pgfpathlineto{\pgfqpoint{2.628552in}{0.689623in}}%
\pgfpathlineto{\pgfqpoint{2.475257in}{0.705403in}}%
\pgfpathlineto{\pgfqpoint{2.305615in}{0.725370in}}%
\pgfpathlineto{\pgfqpoint{2.170223in}{0.743151in}}%
\pgfpathlineto{\pgfqpoint{2.031305in}{0.763294in}}%
\pgfpathlineto{\pgfqpoint{1.893143in}{0.785694in}}%
\pgfpathlineto{\pgfqpoint{1.760805in}{0.810094in}}%
\pgfpathlineto{\pgfqpoint{1.678182in}{0.827274in}}%
\pgfpathlineto{\pgfqpoint{1.601309in}{0.845038in}}%
\pgfpathlineto{\pgfqpoint{1.530968in}{0.863254in}}%
\pgfpathlineto{\pgfqpoint{1.467664in}{0.881786in}}%
\pgfpathlineto{\pgfqpoint{1.411633in}{0.900493in}}%
\pgfpathlineto{\pgfqpoint{1.362998in}{0.919219in}}%
\pgfpathlineto{\pgfqpoint{1.321590in}{0.937814in}}%
\pgfpathlineto{\pgfqpoint{1.286265in}{0.956192in}}%
\pgfpathlineto{\pgfqpoint{1.256058in}{0.974280in}}%
\pgfpathlineto{\pgfqpoint{1.230258in}{0.992009in}}%
\pgfpathlineto{\pgfqpoint{1.208405in}{1.009313in}}%
\pgfpathlineto{\pgfqpoint{1.190295in}{1.026134in}}%
\pgfpathlineto{\pgfqpoint{1.175975in}{1.042417in}}%
\pgfpathlineto{\pgfqpoint{1.165746in}{1.058114in}}%
\pgfpathlineto{\pgfqpoint{1.159971in}{1.073180in}}%
\pgfpathlineto{\pgfqpoint{1.157337in}{1.087582in}}%
\pgfpathlineto{\pgfqpoint{1.157214in}{1.101307in}}%
\pgfpathlineto{\pgfqpoint{1.159370in}{1.114345in}}%
\pgfpathlineto{\pgfqpoint{1.163636in}{1.126689in}}%
\pgfpathlineto{\pgfqpoint{1.169906in}{1.138333in}}%
\pgfpathlineto{\pgfqpoint{1.178137in}{1.149274in}}%
\pgfpathlineto{\pgfqpoint{1.188348in}{1.159514in}}%
\pgfpathlineto{\pgfqpoint{1.200622in}{1.169054in}}%
\pgfpathlineto{\pgfqpoint{1.214928in}{1.177895in}}%
\pgfpathlineto{\pgfqpoint{1.231162in}{1.186033in}}%
\pgfpathlineto{\pgfqpoint{1.259094in}{1.196921in}}%
\pgfpathlineto{\pgfqpoint{1.291372in}{1.206216in}}%
\pgfpathlineto{\pgfqpoint{1.328157in}{1.213909in}}%
\pgfpathlineto{\pgfqpoint{1.369729in}{1.219990in}}%
\pgfpathlineto{\pgfqpoint{1.416489in}{1.224448in}}%
\pgfpathlineto{\pgfqpoint{1.468957in}{1.227271in}}%
\pgfpathlineto{\pgfqpoint{1.527361in}{1.228438in}}%
\pgfpathlineto{\pgfqpoint{1.592305in}{1.227882in}}%
\pgfpathlineto{\pgfqpoint{1.690951in}{1.224319in}}%
\pgfpathlineto{\pgfqpoint{1.805081in}{1.217343in}}%
\pgfpathlineto{\pgfqpoint{1.935791in}{1.206744in}}%
\pgfpathlineto{\pgfqpoint{2.083359in}{1.192303in}}%
\pgfpathlineto{\pgfqpoint{2.247244in}{1.173793in}}%
\pgfpathlineto{\pgfqpoint{2.426070in}{1.150961in}}%
\pgfpathlineto{\pgfqpoint{2.565790in}{1.130977in}}%
\pgfpathlineto{\pgfqpoint{2.704276in}{1.108772in}}%
\pgfpathlineto{\pgfqpoint{2.836826in}{1.084622in}}%
\pgfpathlineto{\pgfqpoint{2.919950in}{1.067600in}}%
\pgfpathlineto{\pgfqpoint{2.997756in}{1.049967in}}%
\pgfpathlineto{\pgfqpoint{3.069466in}{1.031844in}}%
\pgfpathlineto{\pgfqpoint{3.134464in}{1.013361in}}%
\pgfpathlineto{\pgfqpoint{3.192298in}{0.994658in}}%
\pgfpathlineto{\pgfqpoint{3.242682in}{0.975887in}}%
\pgfpathlineto{\pgfqpoint{3.285698in}{0.957204in}}%
\pgfpathlineto{\pgfqpoint{3.322234in}{0.938726in}}%
\pgfpathlineto{\pgfqpoint{3.353156in}{0.920539in}}%
\pgfpathlineto{\pgfqpoint{3.379155in}{0.902721in}}%
\pgfpathlineto{\pgfqpoint{3.400743in}{0.885341in}}%
\pgfpathlineto{\pgfqpoint{3.418259in}{0.868460in}}%
\pgfpathlineto{\pgfqpoint{3.431866in}{0.852128in}}%
\pgfpathlineto{\pgfqpoint{3.441555in}{0.836388in}}%
\pgfpathlineto{\pgfqpoint{3.447804in}{0.821276in}}%
\pgfpathlineto{\pgfqpoint{3.451236in}{0.806815in}}%
\pgfpathlineto{\pgfqpoint{3.452137in}{0.793019in}}%
\pgfpathlineto{\pgfqpoint{3.450723in}{0.779902in}}%
\pgfpathlineto{\pgfqpoint{3.447143in}{0.767471in}}%
\pgfpathlineto{\pgfqpoint{3.441474in}{0.755733in}}%
\pgfpathlineto{\pgfqpoint{3.433727in}{0.744690in}}%
\pgfpathlineto{\pgfqpoint{3.423852in}{0.734342in}}%
\pgfpathlineto{\pgfqpoint{3.411936in}{0.724691in}}%
\pgfpathlineto{\pgfqpoint{3.398078in}{0.715739in}}%
\pgfpathlineto{\pgfqpoint{3.382325in}{0.707488in}}%
\pgfpathlineto{\pgfqpoint{3.355175in}{0.696432in}}%
\pgfpathlineto{\pgfqpoint{3.323742in}{0.686965in}}%
\pgfpathlineto{\pgfqpoint{3.287833in}{0.679093in}}%
\pgfpathlineto{\pgfqpoint{3.247118in}{0.672820in}}%
\pgfpathlineto{\pgfqpoint{3.201125in}{0.668147in}}%
\pgfpathlineto{\pgfqpoint{3.149579in}{0.665092in}}%
\pgfpathlineto{\pgfqpoint{3.092108in}{0.663706in}}%
\pgfpathlineto{\pgfqpoint{3.027880in}{0.664045in}}%
\pgfpathlineto{\pgfqpoint{2.930460in}{0.667295in}}%
\pgfpathlineto{\pgfqpoint{2.818202in}{0.673912in}}%
\pgfpathlineto{\pgfqpoint{2.689882in}{0.684107in}}%
\pgfpathlineto{\pgfqpoint{2.544561in}{0.698120in}}%
\pgfpathlineto{\pgfqpoint{2.380325in}{0.716289in}}%
\pgfpathlineto{\pgfqpoint{2.199059in}{0.738910in}}%
\pgfpathlineto{\pgfqpoint{2.059509in}{0.758638in}}%
\pgfpathlineto{\pgfqpoint{1.922195in}{0.780529in}}%
\pgfpathlineto{\pgfqpoint{1.791115in}{0.804341in}}%
\pgfpathlineto{\pgfqpoint{1.708834in}{0.821142in}}%
\pgfpathlineto{\pgfqpoint{1.631577in}{0.838569in}}%
\pgfpathlineto{\pgfqpoint{1.559986in}{0.856514in}}%
\pgfpathlineto{\pgfqpoint{1.494559in}{0.874856in}}%
\pgfpathlineto{\pgfqpoint{1.435649in}{0.893467in}}%
\pgfpathlineto{\pgfqpoint{1.383465in}{0.912207in}}%
\pgfpathlineto{\pgfqpoint{1.338070in}{0.930926in}}%
\pgfpathlineto{\pgfqpoint{1.299386in}{0.949465in}}%
\pgfpathlineto{\pgfqpoint{1.267048in}{0.967681in}}%
\pgfpathlineto{\pgfqpoint{1.240141in}{0.985532in}}%
\pgfpathlineto{\pgfqpoint{1.218037in}{1.002953in}}%
\pgfpathlineto{\pgfqpoint{1.200226in}{1.019882in}}%
\pgfpathlineto{\pgfqpoint{1.186291in}{1.036269in}}%
\pgfpathlineto{\pgfqpoint{1.175908in}{1.052072in}}%
\pgfpathlineto{\pgfqpoint{1.168740in}{1.067260in}}%
\pgfpathlineto{\pgfqpoint{1.164448in}{1.081810in}}%
\pgfpathlineto{\pgfqpoint{1.162786in}{1.095702in}}%
\pgfpathlineto{\pgfqpoint{1.163558in}{1.108924in}}%
\pgfpathlineto{\pgfqpoint{1.166625in}{1.121465in}}%
\pgfpathlineto{\pgfqpoint{1.171915in}{1.133320in}}%
\pgfpathlineto{\pgfqpoint{1.179387in}{1.144486in}}%
\pgfpathlineto{\pgfqpoint{1.188896in}{1.154959in}}%
\pgfpathlineto{\pgfqpoint{1.200324in}{1.164734in}}%
\pgfpathlineto{\pgfqpoint{1.213593in}{1.173809in}}%
\pgfpathlineto{\pgfqpoint{1.228663in}{1.182181in}}%
\pgfpathlineto{\pgfqpoint{1.254662in}{1.193422in}}%
\pgfpathlineto{\pgfqpoint{1.284903in}{1.203082in}}%
\pgfpathlineto{\pgfqpoint{1.319760in}{1.211171in}}%
\pgfpathlineto{\pgfqpoint{1.359770in}{1.217699in}}%
\pgfpathlineto{\pgfqpoint{1.404939in}{1.222634in}}%
\pgfpathlineto{\pgfqpoint{1.455583in}{1.225939in}}%
\pgfpathlineto{\pgfqpoint{1.512249in}{1.227575in}}%
\pgfpathlineto{\pgfqpoint{1.575516in}{1.227494in}}%
\pgfpathlineto{\pgfqpoint{1.645995in}{1.225635in}}%
\pgfpathlineto{\pgfqpoint{1.752304in}{1.220266in}}%
\pgfpathlineto{\pgfqpoint{1.874197in}{1.211411in}}%
\pgfpathlineto{\pgfqpoint{2.013151in}{1.198835in}}%
\pgfpathlineto{\pgfqpoint{2.169618in}{1.182267in}}%
\pgfpathlineto{\pgfqpoint{2.341721in}{1.161509in}}%
\pgfpathlineto{\pgfqpoint{2.478222in}{1.143138in}}%
\pgfpathlineto{\pgfqpoint{2.617302in}{1.122413in}}%
\pgfpathlineto{\pgfqpoint{2.754110in}{1.099512in}}%
\pgfpathlineto{\pgfqpoint{2.883964in}{1.074711in}}%
\pgfpathlineto{\pgfqpoint{2.964508in}{1.057303in}}%
\pgfpathlineto{\pgfqpoint{3.038697in}{1.039355in}}%
\pgfpathlineto{\pgfqpoint{3.105351in}{1.021022in}}%
\pgfpathlineto{\pgfqpoint{3.164829in}{1.002449in}}%
\pgfpathlineto{\pgfqpoint{3.217614in}{0.983771in}}%
\pgfpathlineto{\pgfqpoint{3.264137in}{0.965108in}}%
\pgfpathlineto{\pgfqpoint{3.304777in}{0.946574in}}%
\pgfpathlineto{\pgfqpoint{3.339859in}{0.928268in}}%
\pgfpathlineto{\pgfqpoint{3.369655in}{0.910280in}}%
\pgfpathlineto{\pgfqpoint{3.394387in}{0.892689in}}%
\pgfpathlineto{\pgfqpoint{3.414222in}{0.875562in}}%
\pgfpathlineto{\pgfqpoint{3.429274in}{0.858956in}}%
\pgfpathlineto{\pgfqpoint{3.439811in}{0.842925in}}%
\pgfpathlineto{\pgfqpoint{3.446823in}{0.827519in}}%
\pgfpathlineto{\pgfqpoint{3.450941in}{0.812760in}}%
\pgfpathlineto{\pgfqpoint{3.452626in}{0.798664in}}%
\pgfpathlineto{\pgfqpoint{3.452213in}{0.785242in}}%
\pgfpathlineto{\pgfqpoint{3.449911in}{0.772503in}}%
\pgfpathlineto{\pgfqpoint{3.445800in}{0.760452in}}%
\pgfpathlineto{\pgfqpoint{3.439830in}{0.749092in}}%
\pgfpathlineto{\pgfqpoint{3.431826in}{0.738422in}}%
\pgfpathlineto{\pgfqpoint{3.421484in}{0.728437in}}%
\pgfpathlineto{\pgfqpoint{3.408493in}{0.719134in}}%
\pgfpathlineto{\pgfqpoint{3.393419in}{0.710530in}}%
\pgfpathlineto{\pgfqpoint{3.367231in}{0.698948in}}%
\pgfpathlineto{\pgfqpoint{3.336731in}{0.688960in}}%
\pgfpathlineto{\pgfqpoint{3.301793in}{0.680572in}}%
\pgfpathlineto{\pgfqpoint{3.262175in}{0.673792in}}%
\pgfpathlineto{\pgfqpoint{3.217524in}{0.668630in}}%
\pgfpathlineto{\pgfqpoint{3.167372in}{0.665097in}}%
\pgfpathlineto{\pgfqpoint{3.111531in}{0.663214in}}%
\pgfpathlineto{\pgfqpoint{3.049349in}{0.663036in}}%
\pgfpathlineto{\pgfqpoint{2.979845in}{0.664635in}}%
\pgfpathlineto{\pgfqpoint{2.874419in}{0.669663in}}%
\pgfpathlineto{\pgfqpoint{2.753169in}{0.678186in}}%
\pgfpathlineto{\pgfqpoint{2.615265in}{0.690421in}}%
\pgfpathlineto{\pgfqpoint{2.460450in}{0.706607in}}%
\pgfpathlineto{\pgfqpoint{2.288814in}{0.727015in}}%
\pgfpathlineto{\pgfqpoint{2.151209in}{0.745208in}}%
\pgfpathlineto{\pgfqpoint{2.011584in}{0.765753in}}%
\pgfpathlineto{\pgfqpoint{1.874717in}{0.788450in}}%
\pgfpathlineto{\pgfqpoint{1.744746in}{0.813032in}}%
\pgfpathlineto{\pgfqpoint{1.663683in}{0.830306in}}%
\pgfpathlineto{\pgfqpoint{1.588130in}{0.848156in}}%
\pgfpathlineto{\pgfqpoint{1.518838in}{0.866453in}}%
\pgfpathlineto{\pgfqpoint{1.456432in}{0.885056in}}%
\pgfpathlineto{\pgfqpoint{1.401408in}{0.903810in}}%
\pgfpathlineto{\pgfqpoint{1.353928in}{0.922556in}}%
\pgfpathlineto{\pgfqpoint{1.313209in}{0.941173in}}%
\pgfpathlineto{\pgfqpoint{1.278536in}{0.959560in}}%
\pgfpathlineto{\pgfqpoint{1.249307in}{0.977626in}}%
\pgfpathlineto{\pgfqpoint{1.225026in}{0.995292in}}%
\pgfpathlineto{\pgfqpoint{1.205295in}{1.012490in}}%
\pgfpathlineto{\pgfqpoint{1.189819in}{1.029161in}}%
\pgfpathlineto{\pgfqpoint{1.178198in}{1.045258in}}%
\pgfpathlineto{\pgfqpoint{1.169875in}{1.060748in}}%
\pgfpathlineto{\pgfqpoint{1.164495in}{1.075605in}}%
\pgfpathlineto{\pgfqpoint{1.161778in}{1.089811in}}%
\pgfpathlineto{\pgfqpoint{1.161527in}{1.103348in}}%
\pgfpathlineto{\pgfqpoint{1.163619in}{1.116206in}}%
\pgfpathlineto{\pgfqpoint{1.168010in}{1.128378in}}%
\pgfpathlineto{\pgfqpoint{1.174631in}{1.139861in}}%
\pgfpathlineto{\pgfqpoint{1.183326in}{1.150649in}}%
\pgfpathlineto{\pgfqpoint{1.193973in}{1.160738in}}%
\pgfpathlineto{\pgfqpoint{1.206488in}{1.170127in}}%
\pgfpathlineto{\pgfqpoint{1.220826in}{1.178812in}}%
\pgfpathlineto{\pgfqpoint{1.245746in}{1.190521in}}%
\pgfpathlineto{\pgfqpoint{1.274896in}{1.200649in}}%
\pgfpathlineto{\pgfqpoint{1.308607in}{1.209202in}}%
\pgfpathlineto{\pgfqpoint{1.347338in}{1.216190in}}%
\pgfpathlineto{\pgfqpoint{1.391092in}{1.221584in}}%
\pgfpathlineto{\pgfqpoint{1.440208in}{1.225355in}}%
\pgfpathlineto{\pgfqpoint{1.495191in}{1.227467in}}%
\pgfpathlineto{\pgfqpoint{1.556595in}{1.227872in}}%
\pgfpathlineto{\pgfqpoint{1.625015in}{1.226516in}}%
\pgfpathlineto{\pgfqpoint{1.728275in}{1.221849in}}%
\pgfpathlineto{\pgfqpoint{1.846806in}{1.213743in}}%
\pgfpathlineto{\pgfqpoint{1.982172in}{1.201968in}}%
\pgfpathlineto{\pgfqpoint{2.135107in}{1.186252in}}%
\pgfpathlineto{\pgfqpoint{2.304277in}{1.166381in}}%
\pgfpathlineto{\pgfqpoint{2.439496in}{1.148672in}}%
\pgfpathlineto{\pgfqpoint{2.578274in}{1.128588in}}%
\pgfpathlineto{\pgfqpoint{2.716250in}{1.106265in}}%
\pgfpathlineto{\pgfqpoint{2.848711in}{1.081948in}}%
\pgfpathlineto{\pgfqpoint{2.931437in}{1.064794in}}%
\pgfpathlineto{\pgfqpoint{3.008129in}{1.047021in}}%
\pgfpathlineto{\pgfqpoint{3.078287in}{1.028793in}}%
\pgfpathlineto{\pgfqpoint{3.141618in}{1.010269in}}%
\pgfpathlineto{\pgfqpoint{3.197987in}{0.991593in}}%
\pgfpathlineto{\pgfqpoint{3.247413in}{0.972897in}}%
\pgfpathlineto{\pgfqpoint{3.290068in}{0.954299in}}%
\pgfpathlineto{\pgfqpoint{3.326283in}{0.935907in}}%
\pgfpathlineto{\pgfqpoint{3.356539in}{0.917812in}}%
\pgfpathlineto{\pgfqpoint{3.381478in}{0.900095in}}%
\pgfpathlineto{\pgfqpoint{3.401831in}{0.882826in}}%
\pgfpathlineto{\pgfqpoint{3.417691in}{0.866082in}}%
\pgfpathlineto{\pgfqpoint{3.429697in}{0.849903in}}%
\pgfpathlineto{\pgfqpoint{3.438579in}{0.834316in}}%
\pgfpathlineto{\pgfqpoint{3.444882in}{0.819346in}}%
\pgfpathlineto{\pgfqpoint{3.448962in}{0.805010in}}%
\pgfpathlineto{\pgfqpoint{3.450988in}{0.791325in}}%
\pgfpathlineto{\pgfqpoint{3.450942in}{0.778301in}}%
\pgfpathlineto{\pgfqpoint{3.448620in}{0.765946in}}%
\pgfpathlineto{\pgfqpoint{3.443627in}{0.754264in}}%
\pgfpathlineto{\pgfqpoint{3.435542in}{0.743257in}}%
\pgfpathlineto{\pgfqpoint{3.425204in}{0.732948in}}%
\pgfpathlineto{\pgfqpoint{3.412930in}{0.723342in}}%
\pgfpathlineto{\pgfqpoint{3.398772in}{0.714443in}}%
\pgfpathlineto{\pgfqpoint{3.382755in}{0.706251in}}%
\pgfpathlineto{\pgfqpoint{3.355227in}{0.695289in}}%
\pgfpathlineto{\pgfqpoint{3.323369in}{0.685920in}}%
\pgfpathlineto{\pgfqpoint{3.286905in}{0.678142in}}%
\pgfpathlineto{\pgfqpoint{3.245458in}{0.671954in}}%
\pgfpathlineto{\pgfqpoint{3.198898in}{0.667376in}}%
\pgfpathlineto{\pgfqpoint{3.146761in}{0.664441in}}%
\pgfpathlineto{\pgfqpoint{3.088455in}{0.663190in}}%
\pgfpathlineto{\pgfqpoint{3.023374in}{0.663675in}}%
\pgfpathlineto{\pgfqpoint{2.924980in}{0.667133in}}%
\pgfpathlineto{\pgfqpoint{2.811911in}{0.673971in}}%
\pgfpathlineto{\pgfqpoint{2.682595in}{0.684402in}}%
\pgfpathlineto{\pgfqpoint{2.535808in}{0.698671in}}%
\pgfpathlineto{\pgfqpoint{2.371870in}{0.717047in}}%
\pgfpathlineto{\pgfqpoint{2.193955in}{0.739681in}}%
\pgfpathlineto{\pgfqpoint{2.055242in}{0.759440in}}%
\pgfpathlineto{\pgfqpoint{2.055242in}{0.759440in}}%
\pgfusepath{stroke}%
\end{pgfscope}%
\begin{pgfscope}%
\pgfpathrectangle{\pgfqpoint{0.562500in}{0.275000in}}{\pgfqpoint{3.487500in}{1.925000in}}%
\pgfusepath{clip}%
\pgfsetrectcap%
\pgfsetroundjoin%
\pgfsetlinewidth{1.505625pt}%
\definecolor{currentstroke}{rgb}{0.839216,0.152941,0.156863}%
\pgfsetstrokecolor{currentstroke}%
\pgfsetdash{}{0pt}%
\pgfpathmoveto{\pgfqpoint{1.909943in}{2.112500in}}%
\pgfpathlineto{\pgfqpoint{2.066924in}{2.086542in}}%
\pgfpathlineto{\pgfqpoint{2.207506in}{2.059874in}}%
\pgfpathlineto{\pgfqpoint{2.331111in}{2.032664in}}%
\pgfpathlineto{\pgfqpoint{2.438155in}{2.005075in}}%
\pgfpathlineto{\pgfqpoint{2.529953in}{1.977263in}}%
\pgfpathlineto{\pgfqpoint{2.608531in}{1.949374in}}%
\pgfpathlineto{\pgfqpoint{2.675901in}{1.921512in}}%
\pgfpathlineto{\pgfqpoint{2.733926in}{1.893761in}}%
\pgfpathlineto{\pgfqpoint{2.784312in}{1.866202in}}%
\pgfpathlineto{\pgfqpoint{2.828633in}{1.838904in}}%
\pgfpathlineto{\pgfqpoint{2.868131in}{1.811901in}}%
\pgfpathlineto{\pgfqpoint{2.903693in}{1.785221in}}%
\pgfpathlineto{\pgfqpoint{2.936093in}{1.758888in}}%
\pgfpathlineto{\pgfqpoint{2.993920in}{1.707347in}}%
\pgfpathlineto{\pgfqpoint{3.044856in}{1.657350in}}%
\pgfpathlineto{\pgfqpoint{3.090714in}{1.608918in}}%
\pgfpathlineto{\pgfqpoint{3.152914in}{1.539235in}}%
\pgfpathlineto{\pgfqpoint{3.208460in}{1.473017in}}%
\pgfpathlineto{\pgfqpoint{3.274360in}{1.389982in}}%
\pgfpathlineto{\pgfqpoint{3.318304in}{1.331500in}}%
\pgfpathlineto{\pgfqpoint{3.357562in}{1.276119in}}%
\pgfpathlineto{\pgfqpoint{3.403605in}{1.206968in}}%
\pgfpathlineto{\pgfqpoint{3.433639in}{1.158486in}}%
\pgfpathlineto{\pgfqpoint{3.459648in}{1.112743in}}%
\pgfpathlineto{\pgfqpoint{3.481728in}{1.069641in}}%
\pgfpathlineto{\pgfqpoint{3.500190in}{1.029130in}}%
\pgfpathlineto{\pgfqpoint{3.515137in}{0.991139in}}%
\pgfpathlineto{\pgfqpoint{3.526647in}{0.955542in}}%
\pgfpathlineto{\pgfqpoint{3.534824in}{0.922240in}}%
\pgfpathlineto{\pgfqpoint{3.539661in}{0.891155in}}%
\pgfpathlineto{\pgfqpoint{3.540975in}{0.871638in}}%
\pgfpathlineto{\pgfqpoint{3.540692in}{0.853074in}}%
\pgfpathlineto{\pgfqpoint{3.538831in}{0.835443in}}%
\pgfpathlineto{\pgfqpoint{3.535490in}{0.818720in}}%
\pgfpathlineto{\pgfqpoint{3.530727in}{0.802885in}}%
\pgfpathlineto{\pgfqpoint{3.524569in}{0.787920in}}%
\pgfpathlineto{\pgfqpoint{3.517010in}{0.773808in}}%
\pgfpathlineto{\pgfqpoint{3.508014in}{0.760534in}}%
\pgfpathlineto{\pgfqpoint{3.497513in}{0.748084in}}%
\pgfpathlineto{\pgfqpoint{3.485405in}{0.736445in}}%
\pgfpathlineto{\pgfqpoint{3.471578in}{0.725606in}}%
\pgfpathlineto{\pgfqpoint{3.456047in}{0.715553in}}%
\pgfpathlineto{\pgfqpoint{3.438861in}{0.706280in}}%
\pgfpathlineto{\pgfqpoint{3.410009in}{0.693821in}}%
\pgfpathlineto{\pgfqpoint{3.377397in}{0.683089in}}%
\pgfpathlineto{\pgfqpoint{3.340815in}{0.674070in}}%
\pgfpathlineto{\pgfqpoint{3.299904in}{0.666752in}}%
\pgfpathlineto{\pgfqpoint{3.254160in}{0.661128in}}%
\pgfpathlineto{\pgfqpoint{3.202933in}{0.657192in}}%
\pgfpathlineto{\pgfqpoint{3.145818in}{0.654955in}}%
\pgfpathlineto{\pgfqpoint{3.082482in}{0.654479in}}%
\pgfpathlineto{\pgfqpoint{3.011906in}{0.655827in}}%
\pgfpathlineto{\pgfqpoint{2.905071in}{0.660591in}}%
\pgfpathlineto{\pgfqpoint{2.782260in}{0.668926in}}%
\pgfpathlineto{\pgfqpoint{2.642415in}{0.681056in}}%
\pgfpathlineto{\pgfqpoint{2.485009in}{0.697231in}}%
\pgfpathlineto{\pgfqpoint{2.309638in}{0.717743in}}%
\pgfpathlineto{\pgfqpoint{2.168984in}{0.736077in}}%
\pgfpathlineto{\pgfqpoint{2.026264in}{0.756829in}}%
\pgfpathlineto{\pgfqpoint{1.886244in}{0.779809in}}%
\pgfpathlineto{\pgfqpoint{1.753116in}{0.804754in}}%
\pgfpathlineto{\pgfqpoint{1.670000in}{0.822309in}}%
\pgfpathlineto{\pgfqpoint{1.592493in}{0.840467in}}%
\pgfpathlineto{\pgfqpoint{1.521410in}{0.859093in}}%
\pgfpathlineto{\pgfqpoint{1.457452in}{0.878036in}}%
\pgfpathlineto{\pgfqpoint{1.401205in}{0.897130in}}%
\pgfpathlineto{\pgfqpoint{1.352753in}{0.916214in}}%
\pgfpathlineto{\pgfqpoint{1.311190in}{0.935172in}}%
\pgfpathlineto{\pgfqpoint{1.275852in}{0.953896in}}%
\pgfpathlineto{\pgfqpoint{1.246156in}{0.972291in}}%
\pgfpathlineto{\pgfqpoint{1.221594in}{0.990276in}}%
\pgfpathlineto{\pgfqpoint{1.201730in}{1.007780in}}%
\pgfpathlineto{\pgfqpoint{1.186135in}{1.024744in}}%
\pgfpathlineto{\pgfqpoint{1.174244in}{1.041125in}}%
\pgfpathlineto{\pgfqpoint{1.165632in}{1.056892in}}%
\pgfpathlineto{\pgfqpoint{1.159965in}{1.072017in}}%
\pgfpathlineto{\pgfqpoint{1.156995in}{1.086481in}}%
\pgfpathlineto{\pgfqpoint{1.156559in}{1.100267in}}%
\pgfpathlineto{\pgfqpoint{1.158572in}{1.113367in}}%
\pgfpathlineto{\pgfqpoint{1.162898in}{1.125774in}}%
\pgfpathlineto{\pgfqpoint{1.169334in}{1.137480in}}%
\pgfpathlineto{\pgfqpoint{1.177728in}{1.148484in}}%
\pgfpathlineto{\pgfqpoint{1.187973in}{1.158781in}}%
\pgfpathlineto{\pgfqpoint{1.200013in}{1.168371in}}%
\pgfpathlineto{\pgfqpoint{1.213839in}{1.177254in}}%
\pgfpathlineto{\pgfqpoint{1.229493in}{1.185432in}}%
\pgfpathlineto{\pgfqpoint{1.256611in}{1.196383in}}%
\pgfpathlineto{\pgfqpoint{1.288526in}{1.205771in}}%
\pgfpathlineto{\pgfqpoint{1.325176in}{1.213580in}}%
\pgfpathlineto{\pgfqpoint{1.366654in}{1.219788in}}%
\pgfpathlineto{\pgfqpoint{1.413303in}{1.224373in}}%
\pgfpathlineto{\pgfqpoint{1.465543in}{1.227305in}}%
\pgfpathlineto{\pgfqpoint{1.523875in}{1.228544in}}%
\pgfpathlineto{\pgfqpoint{1.588877in}{1.228047in}}%
\pgfpathlineto{\pgfqpoint{1.687071in}{1.224586in}}%
\pgfpathlineto{\pgfqpoint{1.799934in}{1.217775in}}%
\pgfpathlineto{\pgfqpoint{1.929314in}{1.207361in}}%
\pgfpathlineto{\pgfqpoint{2.076254in}{1.193087in}}%
\pgfpathlineto{\pgfqpoint{2.240034in}{1.174730in}}%
\pgfpathlineto{\pgfqpoint{2.418135in}{1.152107in}}%
\pgfpathlineto{\pgfqpoint{2.557038in}{1.132338in}}%
\pgfpathlineto{\pgfqpoint{2.695475in}{1.110307in}}%
\pgfpathlineto{\pgfqpoint{2.828974in}{1.086244in}}%
\pgfpathlineto{\pgfqpoint{2.913082in}{1.069232in}}%
\pgfpathlineto{\pgfqpoint{2.991807in}{1.051585in}}%
\pgfpathlineto{\pgfqpoint{3.063931in}{1.033442in}}%
\pgfpathlineto{\pgfqpoint{3.128763in}{1.014956in}}%
\pgfpathlineto{\pgfqpoint{3.186431in}{0.996273in}}%
\pgfpathlineto{\pgfqpoint{3.237177in}{0.977530in}}%
\pgfpathlineto{\pgfqpoint{3.281289in}{0.958851in}}%
\pgfpathlineto{\pgfqpoint{3.319105in}{0.940349in}}%
\pgfpathlineto{\pgfqpoint{3.351009in}{0.922124in}}%
\pgfpathlineto{\pgfqpoint{3.377434in}{0.904266in}}%
\pgfpathlineto{\pgfqpoint{3.398859in}{0.886851in}}%
\pgfpathlineto{\pgfqpoint{3.415804in}{0.869948in}}%
\pgfpathlineto{\pgfqpoint{3.428905in}{0.853600in}}%
\pgfpathlineto{\pgfqpoint{3.438705in}{0.837840in}}%
\pgfpathlineto{\pgfqpoint{3.445612in}{0.822697in}}%
\pgfpathlineto{\pgfqpoint{3.449897in}{0.808193in}}%
\pgfpathlineto{\pgfqpoint{3.451694in}{0.794347in}}%
\pgfpathlineto{\pgfqpoint{3.451002in}{0.781172in}}%
\pgfpathlineto{\pgfqpoint{3.447691in}{0.768674in}}%
\pgfpathlineto{\pgfqpoint{3.441975in}{0.756864in}}%
\pgfpathlineto{\pgfqpoint{3.434167in}{0.745750in}}%
\pgfpathlineto{\pgfqpoint{3.424388in}{0.735335in}}%
\pgfpathlineto{\pgfqpoint{3.412724in}{0.725621in}}%
\pgfpathlineto{\pgfqpoint{3.399220in}{0.716610in}}%
\pgfpathlineto{\pgfqpoint{3.383884in}{0.708304in}}%
\pgfpathlineto{\pgfqpoint{3.357371in}{0.697162in}}%
\pgfpathlineto{\pgfqpoint{3.326404in}{0.687595in}}%
\pgfpathlineto{\pgfqpoint{3.290610in}{0.679595in}}%
\pgfpathlineto{\pgfqpoint{3.249993in}{0.673182in}}%
\pgfpathlineto{\pgfqpoint{3.204256in}{0.668379in}}%
\pgfpathlineto{\pgfqpoint{3.152972in}{0.665217in}}%
\pgfpathlineto{\pgfqpoint{3.095645in}{0.663734in}}%
\pgfpathlineto{\pgfqpoint{3.031716in}{0.663980in}}%
\pgfpathlineto{\pgfqpoint{2.935121in}{0.667095in}}%
\pgfpathlineto{\pgfqpoint{2.824016in}{0.673548in}}%
\pgfpathlineto{\pgfqpoint{2.696735in}{0.683553in}}%
\pgfpathlineto{\pgfqpoint{2.552009in}{0.697379in}}%
\pgfpathlineto{\pgfqpoint{2.390178in}{0.715265in}}%
\pgfpathlineto{\pgfqpoint{2.213749in}{0.737393in}}%
\pgfpathlineto{\pgfqpoint{2.075618in}{0.756786in}}%
\pgfpathlineto{\pgfqpoint{1.936899in}{0.778471in}}%
\pgfpathlineto{\pgfqpoint{1.802205in}{0.802234in}}%
\pgfpathlineto{\pgfqpoint{1.717211in}{0.819082in}}%
\pgfpathlineto{\pgfqpoint{1.637756in}{0.836614in}}%
\pgfpathlineto{\pgfqpoint{1.564536in}{0.854666in}}%
\pgfpathlineto{\pgfqpoint{1.498005in}{0.873079in}}%
\pgfpathlineto{\pgfqpoint{1.438435in}{0.891704in}}%
\pgfpathlineto{\pgfqpoint{1.385913in}{0.910406in}}%
\pgfpathlineto{\pgfqpoint{1.340344in}{0.929061in}}%
\pgfpathlineto{\pgfqpoint{1.301448in}{0.947558in}}%
\pgfpathlineto{\pgfqpoint{1.268762in}{0.965799in}}%
\pgfpathlineto{\pgfqpoint{1.241641in}{0.983698in}}%
\pgfpathlineto{\pgfqpoint{1.219287in}{1.001179in}}%
\pgfpathlineto{\pgfqpoint{1.201564in}{1.018160in}}%
\pgfpathlineto{\pgfqpoint{1.187904in}{1.034593in}}%
\pgfpathlineto{\pgfqpoint{1.177547in}{1.050449in}}%
\pgfpathlineto{\pgfqpoint{1.169922in}{1.065702in}}%
\pgfpathlineto{\pgfqpoint{1.164651in}{1.080329in}}%
\pgfpathlineto{\pgfqpoint{1.161547in}{1.094314in}}%
\pgfpathlineto{\pgfqpoint{1.160616in}{1.107641in}}%
\pgfpathlineto{\pgfqpoint{1.162057in}{1.120303in}}%
\pgfpathlineto{\pgfqpoint{1.166259in}{1.132292in}}%
\pgfpathlineto{\pgfqpoint{1.173537in}{1.143605in}}%
\pgfpathlineto{\pgfqpoint{1.182995in}{1.154219in}}%
\pgfpathlineto{\pgfqpoint{1.194402in}{1.164129in}}%
\pgfpathlineto{\pgfqpoint{1.207695in}{1.173333in}}%
\pgfpathlineto{\pgfqpoint{1.222841in}{1.181832in}}%
\pgfpathlineto{\pgfqpoint{1.249040in}{1.193253in}}%
\pgfpathlineto{\pgfqpoint{1.279532in}{1.203085in}}%
\pgfpathlineto{\pgfqpoint{1.314592in}{1.211330in}}%
\pgfpathlineto{\pgfqpoint{1.354573in}{1.217991in}}%
\pgfpathlineto{\pgfqpoint{1.399573in}{1.223047in}}%
\pgfpathlineto{\pgfqpoint{1.450031in}{1.226469in}}%
\pgfpathlineto{\pgfqpoint{1.506482in}{1.228218in}}%
\pgfpathlineto{\pgfqpoint{1.569497in}{1.228247in}}%
\pgfpathlineto{\pgfqpoint{1.639686in}{1.226496in}}%
\pgfpathlineto{\pgfqpoint{1.745555in}{1.221272in}}%
\pgfpathlineto{\pgfqpoint{1.866967in}{1.212565in}}%
\pgfpathlineto{\pgfqpoint{2.005433in}{1.200141in}}%
\pgfpathlineto{\pgfqpoint{2.161472in}{1.183728in}}%
\pgfpathlineto{\pgfqpoint{2.333317in}{1.163122in}}%
\pgfpathlineto{\pgfqpoint{2.469854in}{1.144857in}}%
\pgfpathlineto{\pgfqpoint{2.609149in}{1.124230in}}%
\pgfpathlineto{\pgfqpoint{2.746460in}{1.101408in}}%
\pgfpathlineto{\pgfqpoint{2.877100in}{1.076661in}}%
\pgfpathlineto{\pgfqpoint{2.958232in}{1.059277in}}%
\pgfpathlineto{\pgfqpoint{3.032792in}{1.041337in}}%
\pgfpathlineto{\pgfqpoint{3.099994in}{1.022993in}}%
\pgfpathlineto{\pgfqpoint{3.160193in}{1.004396in}}%
\pgfpathlineto{\pgfqpoint{3.213740in}{0.985684in}}%
\pgfpathlineto{\pgfqpoint{3.260966in}{0.966986in}}%
\pgfpathlineto{\pgfqpoint{3.302186in}{0.948415in}}%
\pgfpathlineto{\pgfqpoint{3.337696in}{0.930072in}}%
\pgfpathlineto{\pgfqpoint{3.367775in}{0.912048in}}%
\pgfpathlineto{\pgfqpoint{3.392683in}{0.894420in}}%
\pgfpathlineto{\pgfqpoint{3.412664in}{0.877253in}}%
\pgfpathlineto{\pgfqpoint{3.427944in}{0.860600in}}%
\pgfpathlineto{\pgfqpoint{3.438807in}{0.844510in}}%
\pgfpathlineto{\pgfqpoint{3.446003in}{0.829045in}}%
\pgfpathlineto{\pgfqpoint{3.450230in}{0.814225in}}%
\pgfpathlineto{\pgfqpoint{3.452028in}{0.800065in}}%
\pgfpathlineto{\pgfqpoint{3.451795in}{0.786578in}}%
\pgfpathlineto{\pgfqpoint{3.449785in}{0.773772in}}%
\pgfpathlineto{\pgfqpoint{3.446107in}{0.761651in}}%
\pgfpathlineto{\pgfqpoint{3.440728in}{0.750219in}}%
\pgfpathlineto{\pgfqpoint{3.433471in}{0.739474in}}%
\pgfpathlineto{\pgfqpoint{3.424014in}{0.729412in}}%
\pgfpathlineto{\pgfqpoint{3.411893in}{0.720025in}}%
\pgfpathlineto{\pgfqpoint{3.397019in}{0.711318in}}%
\pgfpathlineto{\pgfqpoint{3.380199in}{0.703317in}}%
\pgfpathlineto{\pgfqpoint{3.351383in}{0.692645in}}%
\pgfpathlineto{\pgfqpoint{3.318186in}{0.683572in}}%
\pgfpathlineto{\pgfqpoint{3.280421in}{0.676107in}}%
\pgfpathlineto{\pgfqpoint{3.237794in}{0.670261in}}%
\pgfpathlineto{\pgfqpoint{3.189904in}{0.666045in}}%
\pgfpathlineto{\pgfqpoint{3.136287in}{0.663475in}}%
\pgfpathlineto{\pgfqpoint{3.076800in}{0.662579in}}%
\pgfpathlineto{\pgfqpoint{3.010477in}{0.663427in}}%
\pgfpathlineto{\pgfqpoint{2.909713in}{0.667409in}}%
\pgfpathlineto{\pgfqpoint{2.793310in}{0.674834in}}%
\pgfpathlineto{\pgfqpoint{2.660287in}{0.685913in}}%
\pgfpathlineto{\pgfqpoint{2.510414in}{0.700869in}}%
\pgfpathlineto{\pgfqpoint{2.344206in}{0.719936in}}%
\pgfpathlineto{\pgfqpoint{2.163199in}{0.743379in}}%
\pgfpathlineto{\pgfqpoint{2.023092in}{0.763785in}}%
\pgfpathlineto{\pgfqpoint{1.885194in}{0.786368in}}%
\pgfpathlineto{\pgfqpoint{1.753999in}{0.810848in}}%
\pgfpathlineto{\pgfqpoint{1.672115in}{0.828060in}}%
\pgfpathlineto{\pgfqpoint{1.595762in}{0.845854in}}%
\pgfpathlineto{\pgfqpoint{1.525675in}{0.864108in}}%
\pgfpathlineto{\pgfqpoint{1.462433in}{0.882688in}}%
\pgfpathlineto{\pgfqpoint{1.406454in}{0.901447in}}%
\pgfpathlineto{\pgfqpoint{1.357992in}{0.920231in}}%
\pgfpathlineto{\pgfqpoint{1.316679in}{0.938891in}}%
\pgfpathlineto{\pgfqpoint{1.281584in}{0.957324in}}%
\pgfpathlineto{\pgfqpoint{1.251924in}{0.975444in}}%
\pgfpathlineto{\pgfqpoint{1.227081in}{0.993176in}}%
\pgfpathlineto{\pgfqpoint{1.206602in}{1.010451in}}%
\pgfpathlineto{\pgfqpoint{1.190196in}{1.027212in}}%
\pgfpathlineto{\pgfqpoint{1.177739in}{1.043408in}}%
\pgfpathlineto{\pgfqpoint{1.169107in}{1.059001in}}%
\pgfpathlineto{\pgfqpoint{1.163578in}{1.073957in}}%
\pgfpathlineto{\pgfqpoint{1.160771in}{1.088259in}}%
\pgfpathlineto{\pgfqpoint{1.160426in}{1.101893in}}%
\pgfpathlineto{\pgfqpoint{1.162357in}{1.114845in}}%
\pgfpathlineto{\pgfqpoint{1.166443in}{1.127109in}}%
\pgfpathlineto{\pgfqpoint{1.172639in}{1.138680in}}%
\pgfpathlineto{\pgfqpoint{1.180967in}{1.149556in}}%
\pgfpathlineto{\pgfqpoint{1.191401in}{1.159736in}}%
\pgfpathlineto{\pgfqpoint{1.203811in}{1.169219in}}%
\pgfpathlineto{\pgfqpoint{1.218130in}{1.178001in}}%
\pgfpathlineto{\pgfqpoint{1.234322in}{1.186079in}}%
\pgfpathlineto{\pgfqpoint{1.262109in}{1.196875in}}%
\pgfpathlineto{\pgfqpoint{1.294191in}{1.206081in}}%
\pgfpathlineto{\pgfqpoint{1.330808in}{1.213693in}}%
\pgfpathlineto{\pgfqpoint{1.372348in}{1.219712in}}%
\pgfpathlineto{\pgfqpoint{1.419275in}{1.224134in}}%
\pgfpathlineto{\pgfqpoint{1.471704in}{1.226925in}}%
\pgfpathlineto{\pgfqpoint{1.530282in}{1.228035in}}%
\pgfpathlineto{\pgfqpoint{1.595744in}{1.227410in}}%
\pgfpathlineto{\pgfqpoint{1.694898in}{1.223761in}}%
\pgfpathlineto{\pgfqpoint{1.808948in}{1.216723in}}%
\pgfpathlineto{\pgfqpoint{1.939178in}{1.206081in}}%
\pgfpathlineto{\pgfqpoint{2.086716in}{1.191585in}}%
\pgfpathlineto{\pgfqpoint{2.252357in}{1.173018in}}%
\pgfpathlineto{\pgfqpoint{2.385610in}{1.156285in}}%
\pgfpathlineto{\pgfqpoint{2.523006in}{1.137129in}}%
\pgfpathlineto{\pgfqpoint{2.660961in}{1.115652in}}%
\pgfpathlineto{\pgfqpoint{2.795575in}{1.092061in}}%
\pgfpathlineto{\pgfqpoint{2.881388in}{1.075311in}}%
\pgfpathlineto{\pgfqpoint{2.962513in}{1.057880in}}%
\pgfpathlineto{\pgfqpoint{3.037556in}{1.039908in}}%
\pgfpathlineto{\pgfqpoint{3.105202in}{1.021553in}}%
\pgfpathlineto{\pgfqpoint{3.165353in}{1.002953in}}%
\pgfpathlineto{\pgfqpoint{3.218446in}{0.984244in}}%
\pgfpathlineto{\pgfqpoint{3.264888in}{0.965553in}}%
\pgfpathlineto{\pgfqpoint{3.305072in}{0.946995in}}%
\pgfpathlineto{\pgfqpoint{3.339377in}{0.928673in}}%
\pgfpathlineto{\pgfqpoint{3.368162in}{0.910680in}}%
\pgfpathlineto{\pgfqpoint{3.391775in}{0.893097in}}%
\pgfpathlineto{\pgfqpoint{3.410545in}{0.875993in}}%
\pgfpathlineto{\pgfqpoint{3.424988in}{0.859432in}}%
\pgfpathlineto{\pgfqpoint{3.435790in}{0.843453in}}%
\pgfpathlineto{\pgfqpoint{3.443477in}{0.828084in}}%
\pgfpathlineto{\pgfqpoint{3.448448in}{0.813349in}}%
\pgfpathlineto{\pgfqpoint{3.450977in}{0.799266in}}%
\pgfpathlineto{\pgfqpoint{3.451208in}{0.785850in}}%
\pgfpathlineto{\pgfqpoint{3.449160in}{0.773111in}}%
\pgfpathlineto{\pgfqpoint{3.444726in}{0.761055in}}%
\pgfpathlineto{\pgfqpoint{3.437775in}{0.749684in}}%
\pgfpathlineto{\pgfqpoint{3.428694in}{0.739008in}}%
\pgfpathlineto{\pgfqpoint{3.417643in}{0.729032in}}%
\pgfpathlineto{\pgfqpoint{3.404695in}{0.719758in}}%
\pgfpathlineto{\pgfqpoint{3.389892in}{0.711188in}}%
\pgfpathlineto{\pgfqpoint{3.364221in}{0.699653in}}%
\pgfpathlineto{\pgfqpoint{3.334285in}{0.689705in}}%
\pgfpathlineto{\pgfqpoint{3.299818in}{0.681339in}}%
\pgfpathlineto{\pgfqpoint{3.260415in}{0.674551in}}%
\pgfpathlineto{\pgfqpoint{3.215969in}{0.669360in}}%
\pgfpathlineto{\pgfqpoint{3.166123in}{0.665796in}}%
\pgfpathlineto{\pgfqpoint{3.110335in}{0.663897in}}%
\pgfpathlineto{\pgfqpoint{3.048030in}{0.663711in}}%
\pgfpathlineto{\pgfqpoint{2.978603in}{0.665295in}}%
\pgfpathlineto{\pgfqpoint{2.873847in}{0.670284in}}%
\pgfpathlineto{\pgfqpoint{2.753690in}{0.678738in}}%
\pgfpathlineto{\pgfqpoint{2.616586in}{0.690886in}}%
\pgfpathlineto{\pgfqpoint{2.461916in}{0.707002in}}%
\pgfpathlineto{\pgfqpoint{2.291215in}{0.727297in}}%
\pgfpathlineto{\pgfqpoint{2.155333in}{0.745320in}}%
\pgfpathlineto{\pgfqpoint{2.016192in}{0.765716in}}%
\pgfpathlineto{\pgfqpoint{1.878595in}{0.788326in}}%
\pgfpathlineto{\pgfqpoint{1.747323in}{0.812883in}}%
\pgfpathlineto{\pgfqpoint{1.665491in}{0.830161in}}%
\pgfpathlineto{\pgfqpoint{1.589605in}{0.848016in}}%
\pgfpathlineto{\pgfqpoint{1.521082in}{0.866290in}}%
\pgfpathlineto{\pgfqpoint{1.460009in}{0.884831in}}%
\pgfpathlineto{\pgfqpoint{1.405816in}{0.903506in}}%
\pgfpathlineto{\pgfqpoint{1.358001in}{0.922193in}}%
\pgfpathlineto{\pgfqpoint{1.316132in}{0.940778in}}%
\pgfpathlineto{\pgfqpoint{1.279847in}{0.959158in}}%
\pgfpathlineto{\pgfqpoint{1.248852in}{0.977242in}}%
\pgfpathlineto{\pgfqpoint{1.222923in}{0.994948in}}%
\pgfpathlineto{\pgfqpoint{1.201906in}{1.012204in}}%
\pgfpathlineto{\pgfqpoint{1.185715in}{1.028948in}}%
\pgfpathlineto{\pgfqpoint{1.174235in}{1.045128in}}%
\pgfpathlineto{\pgfqpoint{1.166459in}{1.060689in}}%
\pgfpathlineto{\pgfqpoint{1.161677in}{1.075606in}}%
\pgfpathlineto{\pgfqpoint{1.159436in}{1.089864in}}%
\pgfpathlineto{\pgfqpoint{1.159401in}{1.103448in}}%
\pgfpathlineto{\pgfqpoint{1.161357in}{1.116350in}}%
\pgfpathlineto{\pgfqpoint{1.165207in}{1.128562in}}%
\pgfpathlineto{\pgfqpoint{1.170974in}{1.140082in}}%
\pgfpathlineto{\pgfqpoint{1.178797in}{1.150911in}}%
\pgfpathlineto{\pgfqpoint{1.188937in}{1.161050in}}%
\pgfpathlineto{\pgfqpoint{1.201582in}{1.170503in}}%
\pgfpathlineto{\pgfqpoint{1.216223in}{1.179254in}}%
\pgfpathlineto{\pgfqpoint{1.232769in}{1.187300in}}%
\pgfpathlineto{\pgfqpoint{1.261151in}{1.198042in}}%
\pgfpathlineto{\pgfqpoint{1.293886in}{1.207188in}}%
\pgfpathlineto{\pgfqpoint{1.331168in}{1.214732in}}%
\pgfpathlineto{\pgfqpoint{1.373311in}{1.220664in}}%
\pgfpathlineto{\pgfqpoint{1.420746in}{1.224977in}}%
\pgfpathlineto{\pgfqpoint{1.473789in}{1.227658in}}%
\pgfpathlineto{\pgfqpoint{1.532810in}{1.228658in}}%
\pgfpathlineto{\pgfqpoint{1.598752in}{1.227916in}}%
\pgfpathlineto{\pgfqpoint{1.698829in}{1.224091in}}%
\pgfpathlineto{\pgfqpoint{1.814174in}{1.216851in}}%
\pgfpathlineto{\pgfqpoint{1.945860in}{1.205985in}}%
\pgfpathlineto{\pgfqpoint{2.094501in}{1.191254in}}%
\pgfpathlineto{\pgfqpoint{2.260473in}{1.172389in}}%
\pgfpathlineto{\pgfqpoint{2.395493in}{1.155372in}}%
\pgfpathlineto{\pgfqpoint{2.534538in}{1.135952in}}%
\pgfpathlineto{\pgfqpoint{2.672861in}{1.114279in}}%
\pgfpathlineto{\pgfqpoint{2.806251in}{1.090578in}}%
\pgfpathlineto{\pgfqpoint{2.890597in}{1.073794in}}%
\pgfpathlineto{\pgfqpoint{2.970144in}{1.056344in}}%
\pgfpathlineto{\pgfqpoint{3.044038in}{1.038350in}}%
\pgfpathlineto{\pgfqpoint{3.111529in}{1.019946in}}%
\pgfpathlineto{\pgfqpoint{3.171973in}{1.001283in}}%
\pgfpathlineto{\pgfqpoint{3.224837in}{0.982523in}}%
\pgfpathlineto{\pgfqpoint{3.270362in}{0.963811in}}%
\pgfpathlineto{\pgfqpoint{3.309381in}{0.945255in}}%
\pgfpathlineto{\pgfqpoint{3.342510in}{0.926954in}}%
\pgfpathlineto{\pgfqpoint{3.370291in}{0.908998in}}%
\pgfpathlineto{\pgfqpoint{3.393186in}{0.891464in}}%
\pgfpathlineto{\pgfqpoint{3.411578in}{0.874418in}}%
\pgfpathlineto{\pgfqpoint{3.425846in}{0.857914in}}%
\pgfpathlineto{\pgfqpoint{3.436540in}{0.841993in}}%
\pgfpathlineto{\pgfqpoint{3.444063in}{0.826685in}}%
\pgfpathlineto{\pgfqpoint{3.448731in}{0.812014in}}%
\pgfpathlineto{\pgfqpoint{3.450778in}{0.798000in}}%
\pgfpathlineto{\pgfqpoint{3.450360in}{0.784657in}}%
\pgfpathlineto{\pgfqpoint{3.447552in}{0.771994in}}%
\pgfpathlineto{\pgfqpoint{3.442443in}{0.760018in}}%
\pgfpathlineto{\pgfqpoint{3.435206in}{0.748734in}}%
\pgfpathlineto{\pgfqpoint{3.425990in}{0.738146in}}%
\pgfpathlineto{\pgfqpoint{3.414895in}{0.728259in}}%
\pgfpathlineto{\pgfqpoint{3.401982in}{0.719073in}}%
\pgfpathlineto{\pgfqpoint{3.387266in}{0.710590in}}%
\pgfpathlineto{\pgfqpoint{3.361743in}{0.699182in}}%
\pgfpathlineto{\pgfqpoint{3.331820in}{0.689346in}}%
\pgfpathlineto{\pgfqpoint{3.297026in}{0.681068in}}%
\pgfpathlineto{\pgfqpoint{3.257338in}{0.674367in}}%
\pgfpathlineto{\pgfqpoint{3.212603in}{0.669271in}}%
\pgfpathlineto{\pgfqpoint{3.162405in}{0.665807in}}%
\pgfpathlineto{\pgfqpoint{3.106259in}{0.664014in}}%
\pgfpathlineto{\pgfqpoint{3.043619in}{0.663938in}}%
\pgfpathlineto{\pgfqpoint{2.973875in}{0.665634in}}%
\pgfpathlineto{\pgfqpoint{2.868653in}{0.670769in}}%
\pgfpathlineto{\pgfqpoint{2.747946in}{0.679355in}}%
\pgfpathlineto{\pgfqpoint{2.610152in}{0.691657in}}%
\pgfpathlineto{\pgfqpoint{2.454676in}{0.707952in}}%
\pgfpathlineto{\pgfqpoint{2.283504in}{0.728419in}}%
\pgfpathlineto{\pgfqpoint{2.147512in}{0.746563in}}%
\pgfpathlineto{\pgfqpoint{2.147512in}{0.746563in}}%
\pgfusepath{stroke}%
\end{pgfscope}%
\begin{pgfscope}%
\pgfpathrectangle{\pgfqpoint{0.562500in}{0.275000in}}{\pgfqpoint{3.487500in}{1.925000in}}%
\pgfusepath{clip}%
\pgfsetrectcap%
\pgfsetroundjoin%
\pgfsetlinewidth{1.505625pt}%
\definecolor{currentstroke}{rgb}{0.580392,0.403922,0.741176}%
\pgfsetstrokecolor{currentstroke}%
\pgfsetdash{}{0pt}%
\pgfpathmoveto{\pgfqpoint{1.117330in}{1.529167in}}%
\pgfpathlineto{\pgfqpoint{1.309244in}{1.516074in}}%
\pgfpathlineto{\pgfqpoint{1.519296in}{1.498581in}}%
\pgfpathlineto{\pgfqpoint{1.750176in}{1.476466in}}%
\pgfpathlineto{\pgfqpoint{1.999677in}{1.449565in}}%
\pgfpathlineto{\pgfqpoint{2.173013in}{1.428917in}}%
\pgfpathlineto{\pgfqpoint{2.347627in}{1.406101in}}%
\pgfpathlineto{\pgfqpoint{2.517389in}{1.381311in}}%
\pgfpathlineto{\pgfqpoint{2.676934in}{1.354806in}}%
\pgfpathlineto{\pgfqpoint{2.822138in}{1.326879in}}%
\pgfpathlineto{\pgfqpoint{2.888416in}{1.312484in}}%
\pgfpathlineto{\pgfqpoint{2.950127in}{1.297858in}}%
\pgfpathlineto{\pgfqpoint{3.007107in}{1.283048in}}%
\pgfpathlineto{\pgfqpoint{3.059267in}{1.268101in}}%
\pgfpathlineto{\pgfqpoint{3.106655in}{1.253071in}}%
\pgfpathlineto{\pgfqpoint{3.149682in}{1.238008in}}%
\pgfpathlineto{\pgfqpoint{3.188720in}{1.222945in}}%
\pgfpathlineto{\pgfqpoint{3.224101in}{1.207912in}}%
\pgfpathlineto{\pgfqpoint{3.256130in}{1.192938in}}%
\pgfpathlineto{\pgfqpoint{3.285086in}{1.178051in}}%
\pgfpathlineto{\pgfqpoint{3.311225in}{1.163279in}}%
\pgfpathlineto{\pgfqpoint{3.334776in}{1.148646in}}%
\pgfpathlineto{\pgfqpoint{3.375014in}{1.119889in}}%
\pgfpathlineto{\pgfqpoint{3.407956in}{1.091892in}}%
\pgfpathlineto{\pgfqpoint{3.435123in}{1.064716in}}%
\pgfpathlineto{\pgfqpoint{3.457587in}{1.038416in}}%
\pgfpathlineto{\pgfqpoint{3.476062in}{1.013037in}}%
\pgfpathlineto{\pgfqpoint{3.491175in}{0.988609in}}%
\pgfpathlineto{\pgfqpoint{3.503473in}{0.965128in}}%
\pgfpathlineto{\pgfqpoint{3.513338in}{0.942587in}}%
\pgfpathlineto{\pgfqpoint{3.521030in}{0.920981in}}%
\pgfpathlineto{\pgfqpoint{3.526688in}{0.900308in}}%
\pgfpathlineto{\pgfqpoint{3.530331in}{0.880566in}}%
\pgfpathlineto{\pgfqpoint{3.532042in}{0.861746in}}%
\pgfpathlineto{\pgfqpoint{3.532026in}{0.843830in}}%
\pgfpathlineto{\pgfqpoint{3.530377in}{0.826804in}}%
\pgfpathlineto{\pgfqpoint{3.527151in}{0.810650in}}%
\pgfpathlineto{\pgfqpoint{3.522373in}{0.795357in}}%
\pgfpathlineto{\pgfqpoint{3.516030in}{0.780909in}}%
\pgfpathlineto{\pgfqpoint{3.508078in}{0.767297in}}%
\pgfpathlineto{\pgfqpoint{3.498431in}{0.754505in}}%
\pgfpathlineto{\pgfqpoint{3.487110in}{0.742522in}}%
\pgfpathlineto{\pgfqpoint{3.474204in}{0.731338in}}%
\pgfpathlineto{\pgfqpoint{3.459767in}{0.720945in}}%
\pgfpathlineto{\pgfqpoint{3.435272in}{0.706820in}}%
\pgfpathlineto{\pgfqpoint{3.407271in}{0.694430in}}%
\pgfpathlineto{\pgfqpoint{3.375484in}{0.683748in}}%
\pgfpathlineto{\pgfqpoint{3.339442in}{0.674749in}}%
\pgfpathlineto{\pgfqpoint{3.298504in}{0.667410in}}%
\pgfpathlineto{\pgfqpoint{3.252538in}{0.661743in}}%
\pgfpathlineto{\pgfqpoint{3.201321in}{0.657775in}}%
\pgfpathlineto{\pgfqpoint{3.144274in}{0.655536in}}%
\pgfpathlineto{\pgfqpoint{3.080781in}{0.655067in}}%
\pgfpathlineto{\pgfqpoint{3.010187in}{0.656421in}}%
\pgfpathlineto{\pgfqpoint{2.903811in}{0.661177in}}%
\pgfpathlineto{\pgfqpoint{2.781827in}{0.669481in}}%
\pgfpathlineto{\pgfqpoint{2.642535in}{0.681563in}}%
\pgfpathlineto{\pgfqpoint{2.485145in}{0.697703in}}%
\pgfpathlineto{\pgfqpoint{2.311061in}{0.718125in}}%
\pgfpathlineto{\pgfqpoint{2.172080in}{0.736326in}}%
\pgfpathlineto{\pgfqpoint{2.029695in}{0.756967in}}%
\pgfpathlineto{\pgfqpoint{1.888619in}{0.779896in}}%
\pgfpathlineto{\pgfqpoint{1.753815in}{0.804848in}}%
\pgfpathlineto{\pgfqpoint{1.670032in}{0.822427in}}%
\pgfpathlineto{\pgfqpoint{1.592807in}{0.840618in}}%
\pgfpathlineto{\pgfqpoint{1.522519in}{0.859250in}}%
\pgfpathlineto{\pgfqpoint{1.459325in}{0.878162in}}%
\pgfpathlineto{\pgfqpoint{1.403252in}{0.897206in}}%
\pgfpathlineto{\pgfqpoint{1.354206in}{0.916245in}}%
\pgfpathlineto{\pgfqpoint{1.311965in}{0.935161in}}%
\pgfpathlineto{\pgfqpoint{1.276183in}{0.953846in}}%
\pgfpathlineto{\pgfqpoint{1.246387in}{0.972208in}}%
\pgfpathlineto{\pgfqpoint{1.221980in}{0.990167in}}%
\pgfpathlineto{\pgfqpoint{1.202247in}{1.007659in}}%
\pgfpathlineto{\pgfqpoint{1.186968in}{1.024608in}}%
\pgfpathlineto{\pgfqpoint{1.175613in}{1.040966in}}%
\pgfpathlineto{\pgfqpoint{1.167387in}{1.056711in}}%
\pgfpathlineto{\pgfqpoint{1.161687in}{1.071824in}}%
\pgfpathlineto{\pgfqpoint{1.158106in}{1.086287in}}%
\pgfpathlineto{\pgfqpoint{1.156431in}{1.100090in}}%
\pgfpathlineto{\pgfqpoint{1.156643in}{1.113221in}}%
\pgfpathlineto{\pgfqpoint{1.158919in}{1.125675in}}%
\pgfpathlineto{\pgfqpoint{1.163631in}{1.137448in}}%
\pgfpathlineto{\pgfqpoint{1.171345in}{1.148541in}}%
\pgfpathlineto{\pgfqpoint{1.181896in}{1.158942in}}%
\pgfpathlineto{\pgfqpoint{1.194378in}{1.168630in}}%
\pgfpathlineto{\pgfqpoint{1.208736in}{1.177606in}}%
\pgfpathlineto{\pgfqpoint{1.224944in}{1.185868in}}%
\pgfpathlineto{\pgfqpoint{1.252736in}{1.196924in}}%
\pgfpathlineto{\pgfqpoint{1.284819in}{1.206374in}}%
\pgfpathlineto{\pgfqpoint{1.321445in}{1.214220in}}%
\pgfpathlineto{\pgfqpoint{1.363002in}{1.220462in}}%
\pgfpathlineto{\pgfqpoint{1.409702in}{1.225090in}}%
\pgfpathlineto{\pgfqpoint{1.461915in}{1.228070in}}%
\pgfpathlineto{\pgfqpoint{1.520284in}{1.229361in}}%
\pgfpathlineto{\pgfqpoint{1.585448in}{1.228910in}}%
\pgfpathlineto{\pgfqpoint{1.684005in}{1.225491in}}%
\pgfpathlineto{\pgfqpoint{1.797279in}{1.218684in}}%
\pgfpathlineto{\pgfqpoint{1.926762in}{1.208273in}}%
\pgfpathlineto{\pgfqpoint{2.073779in}{1.194018in}}%
\pgfpathlineto{\pgfqpoint{2.238016in}{1.175654in}}%
\pgfpathlineto{\pgfqpoint{2.416277in}{1.152994in}}%
\pgfpathlineto{\pgfqpoint{2.555294in}{1.133207in}}%
\pgfpathlineto{\pgfqpoint{2.694411in}{1.111152in}}%
\pgfpathlineto{\pgfqpoint{2.828537in}{1.087021in}}%
\pgfpathlineto{\pgfqpoint{2.912747in}{1.069950in}}%
\pgfpathlineto{\pgfqpoint{2.991438in}{1.052260in}}%
\pgfpathlineto{\pgfqpoint{3.063761in}{1.034096in}}%
\pgfpathlineto{\pgfqpoint{3.129145in}{1.015595in}}%
\pgfpathlineto{\pgfqpoint{3.187295in}{0.996893in}}%
\pgfpathlineto{\pgfqpoint{3.238194in}{0.978119in}}%
\pgfpathlineto{\pgfqpoint{3.282103in}{0.959401in}}%
\pgfpathlineto{\pgfqpoint{3.319443in}{0.940865in}}%
\pgfpathlineto{\pgfqpoint{3.350439in}{0.922643in}}%
\pgfpathlineto{\pgfqpoint{3.376161in}{0.904800in}}%
\pgfpathlineto{\pgfqpoint{3.397577in}{0.887390in}}%
\pgfpathlineto{\pgfqpoint{3.415390in}{0.870461in}}%
\pgfpathlineto{\pgfqpoint{3.430040in}{0.854057in}}%
\pgfpathlineto{\pgfqpoint{3.441706in}{0.838219in}}%
\pgfpathlineto{\pgfqpoint{3.450300in}{0.822981in}}%
\pgfpathlineto{\pgfqpoint{3.455472in}{0.808373in}}%
\pgfpathlineto{\pgfqpoint{3.456734in}{0.794423in}}%
\pgfpathlineto{\pgfqpoint{3.455151in}{0.781158in}}%
\pgfpathlineto{\pgfqpoint{3.451324in}{0.768587in}}%
\pgfpathlineto{\pgfqpoint{3.445418in}{0.756716in}}%
\pgfpathlineto{\pgfqpoint{3.437551in}{0.745550in}}%
\pgfpathlineto{\pgfqpoint{3.427794in}{0.735089in}}%
\pgfpathlineto{\pgfqpoint{3.416168in}{0.725335in}}%
\pgfpathlineto{\pgfqpoint{3.402650in}{0.716286in}}%
\pgfpathlineto{\pgfqpoint{3.387165in}{0.707939in}}%
\pgfpathlineto{\pgfqpoint{3.360124in}{0.696729in}}%
\pgfpathlineto{\pgfqpoint{3.328653in}{0.687101in}}%
\pgfpathlineto{\pgfqpoint{3.292646in}{0.679064in}}%
\pgfpathlineto{\pgfqpoint{3.251883in}{0.672636in}}%
\pgfpathlineto{\pgfqpoint{3.206044in}{0.667835in}}%
\pgfpathlineto{\pgfqpoint{3.154710in}{0.664686in}}%
\pgfpathlineto{\pgfqpoint{3.097364in}{0.663217in}}%
\pgfpathlineto{\pgfqpoint{3.033390in}{0.663460in}}%
\pgfpathlineto{\pgfqpoint{2.937324in}{0.666458in}}%
\pgfpathlineto{\pgfqpoint{2.826488in}{0.672776in}}%
\pgfpathlineto{\pgfqpoint{2.698243in}{0.682784in}}%
\pgfpathlineto{\pgfqpoint{2.551993in}{0.696748in}}%
\pgfpathlineto{\pgfqpoint{2.389187in}{0.714829in}}%
\pgfpathlineto{\pgfqpoint{2.213319in}{0.737083in}}%
\pgfpathlineto{\pgfqpoint{2.076102in}{0.756488in}}%
\pgfpathlineto{\pgfqpoint{1.937709in}{0.778159in}}%
\pgfpathlineto{\pgfqpoint{1.803011in}{0.801971in}}%
\pgfpathlineto{\pgfqpoint{1.717993in}{0.818847in}}%
\pgfpathlineto{\pgfqpoint{1.638176in}{0.836360in}}%
\pgfpathlineto{\pgfqpoint{1.564453in}{0.854374in}}%
\pgfpathlineto{\pgfqpoint{1.497464in}{0.872754in}}%
\pgfpathlineto{\pgfqpoint{1.437590in}{0.891370in}}%
\pgfpathlineto{\pgfqpoint{1.384960in}{0.910092in}}%
\pgfpathlineto{\pgfqpoint{1.339444in}{0.928793in}}%
\pgfpathlineto{\pgfqpoint{1.300670in}{0.947350in}}%
\pgfpathlineto{\pgfqpoint{1.268406in}{0.965620in}}%
\pgfpathlineto{\pgfqpoint{1.241738in}{0.983520in}}%
\pgfpathlineto{\pgfqpoint{1.219599in}{1.000996in}}%
\pgfpathlineto{\pgfqpoint{1.201187in}{1.018001in}}%
\pgfpathlineto{\pgfqpoint{1.185969in}{1.034492in}}%
\pgfpathlineto{\pgfqpoint{1.173676in}{1.050427in}}%
\pgfpathlineto{\pgfqpoint{1.164307in}{1.065773in}}%
\pgfpathlineto{\pgfqpoint{1.158127in}{1.080497in}}%
\pgfpathlineto{\pgfqpoint{1.155668in}{1.094574in}}%
\pgfpathlineto{\pgfqpoint{1.156761in}{1.107976in}}%
\pgfpathlineto{\pgfqpoint{1.160168in}{1.120684in}}%
\pgfpathlineto{\pgfqpoint{1.165696in}{1.132693in}}%
\pgfpathlineto{\pgfqpoint{1.173213in}{1.143998in}}%
\pgfpathlineto{\pgfqpoint{1.182635in}{1.154598in}}%
\pgfpathlineto{\pgfqpoint{1.193920in}{1.164490in}}%
\pgfpathlineto{\pgfqpoint{1.207075in}{1.173677in}}%
\pgfpathlineto{\pgfqpoint{1.222152in}{1.182160in}}%
\pgfpathlineto{\pgfqpoint{1.239248in}{1.189943in}}%
\pgfpathlineto{\pgfqpoint{1.268658in}{1.200304in}}%
\pgfpathlineto{\pgfqpoint{1.302542in}{1.209080in}}%
\pgfpathlineto{\pgfqpoint{1.341076in}{1.216258in}}%
\pgfpathlineto{\pgfqpoint{1.384533in}{1.221818in}}%
\pgfpathlineto{\pgfqpoint{1.433281in}{1.225739in}}%
\pgfpathlineto{\pgfqpoint{1.487786in}{1.227992in}}%
\pgfpathlineto{\pgfqpoint{1.548609in}{1.228544in}}%
\pgfpathlineto{\pgfqpoint{1.616374in}{1.227360in}}%
\pgfpathlineto{\pgfqpoint{1.718201in}{1.222994in}}%
\pgfpathlineto{\pgfqpoint{1.835767in}{1.215171in}}%
\pgfpathlineto{\pgfqpoint{1.970896in}{1.203610in}}%
\pgfpathlineto{\pgfqpoint{2.123556in}{1.188091in}}%
\pgfpathlineto{\pgfqpoint{2.291848in}{1.168455in}}%
\pgfpathlineto{\pgfqpoint{2.426138in}{1.150963in}}%
\pgfpathlineto{\pgfqpoint{2.564819in}{1.131074in}}%
\pgfpathlineto{\pgfqpoint{2.703520in}{1.108848in}}%
\pgfpathlineto{\pgfqpoint{2.836801in}{1.084620in}}%
\pgfpathlineto{\pgfqpoint{2.920425in}{1.067548in}}%
\pgfpathlineto{\pgfqpoint{2.998629in}{1.049882in}}%
\pgfpathlineto{\pgfqpoint{3.070597in}{1.031744in}}%
\pgfpathlineto{\pgfqpoint{3.135730in}{1.013263in}}%
\pgfpathlineto{\pgfqpoint{3.193647in}{0.994569in}}%
\pgfpathlineto{\pgfqpoint{3.244183in}{0.975798in}}%
\pgfpathlineto{\pgfqpoint{3.287389in}{0.957087in}}%
\pgfpathlineto{\pgfqpoint{3.323688in}{0.938590in}}%
\pgfpathlineto{\pgfqpoint{3.354122in}{0.920401in}}%
\pgfpathlineto{\pgfqpoint{3.379638in}{0.902587in}}%
\pgfpathlineto{\pgfqpoint{3.400954in}{0.885210in}}%
\pgfpathlineto{\pgfqpoint{3.418552in}{0.868324in}}%
\pgfpathlineto{\pgfqpoint{3.432688in}{0.851979in}}%
\pgfpathlineto{\pgfqpoint{3.443383in}{0.836216in}}%
\pgfpathlineto{\pgfqpoint{3.450426in}{0.821070in}}%
\pgfpathlineto{\pgfqpoint{3.453749in}{0.806572in}}%
\pgfpathlineto{\pgfqpoint{3.454381in}{0.792747in}}%
\pgfpathlineto{\pgfqpoint{3.452650in}{0.779605in}}%
\pgfpathlineto{\pgfqpoint{3.448753in}{0.767155in}}%
\pgfpathlineto{\pgfqpoint{3.442827in}{0.755402in}}%
\pgfpathlineto{\pgfqpoint{3.434953in}{0.744349in}}%
\pgfpathlineto{\pgfqpoint{3.425152in}{0.733999in}}%
\pgfpathlineto{\pgfqpoint{3.413388in}{0.724349in}}%
\pgfpathlineto{\pgfqpoint{3.399569in}{0.715398in}}%
\pgfpathlineto{\pgfqpoint{3.383736in}{0.707144in}}%
\pgfpathlineto{\pgfqpoint{3.356363in}{0.696078in}}%
\pgfpathlineto{\pgfqpoint{3.324613in}{0.686598in}}%
\pgfpathlineto{\pgfqpoint{3.288354in}{0.678715in}}%
\pgfpathlineto{\pgfqpoint{3.247341in}{0.672442in}}%
\pgfpathlineto{\pgfqpoint{3.201219in}{0.667793in}}%
\pgfpathlineto{\pgfqpoint{3.149520in}{0.664787in}}%
\pgfpathlineto{\pgfqpoint{3.091700in}{0.663442in}}%
\pgfpathlineto{\pgfqpoint{3.027666in}{0.663784in}}%
\pgfpathlineto{\pgfqpoint{2.930730in}{0.667027in}}%
\pgfpathlineto{\pgfqpoint{2.818264in}{0.673677in}}%
\pgfpathlineto{\pgfqpoint{2.688875in}{0.683959in}}%
\pgfpathlineto{\pgfqpoint{2.542383in}{0.698082in}}%
\pgfpathlineto{\pgfqpoint{2.379823in}{0.716237in}}%
\pgfpathlineto{\pgfqpoint{2.203442in}{0.738600in}}%
\pgfpathlineto{\pgfqpoint{2.064665in}{0.758220in}}%
\pgfpathlineto{\pgfqpoint{1.925651in}{0.780147in}}%
\pgfpathlineto{\pgfqpoint{1.791699in}{0.804088in}}%
\pgfpathlineto{\pgfqpoint{1.707393in}{0.821001in}}%
\pgfpathlineto{\pgfqpoint{1.628332in}{0.838547in}}%
\pgfpathlineto{\pgfqpoint{1.555373in}{0.856601in}}%
\pgfpathlineto{\pgfqpoint{1.489184in}{0.875035in}}%
\pgfpathlineto{\pgfqpoint{1.430240in}{0.893706in}}%
\pgfpathlineto{\pgfqpoint{1.378828in}{0.912463in}}%
\pgfpathlineto{\pgfqpoint{1.334755in}{0.931148in}}%
\pgfpathlineto{\pgfqpoint{1.297109in}{0.949653in}}%
\pgfpathlineto{\pgfqpoint{1.265048in}{0.967889in}}%
\pgfpathlineto{\pgfqpoint{1.237916in}{0.985776in}}%
\pgfpathlineto{\pgfqpoint{1.215239in}{1.003241in}}%
\pgfpathlineto{\pgfqpoint{1.196729in}{1.020223in}}%
\pgfpathlineto{\pgfqpoint{1.182279in}{1.036665in}}%
\pgfpathlineto{\pgfqpoint{1.171963in}{1.052522in}}%
\pgfpathlineto{\pgfqpoint{1.165240in}{1.067754in}}%
\pgfpathlineto{\pgfqpoint{1.161385in}{1.082338in}}%
\pgfpathlineto{\pgfqpoint{1.160085in}{1.096259in}}%
\pgfpathlineto{\pgfqpoint{1.161104in}{1.109502in}}%
\pgfpathlineto{\pgfqpoint{1.164282in}{1.122060in}}%
\pgfpathlineto{\pgfqpoint{1.169534in}{1.133926in}}%
\pgfpathlineto{\pgfqpoint{1.176849in}{1.145096in}}%
\pgfpathlineto{\pgfqpoint{1.186296in}{1.155571in}}%
\pgfpathlineto{\pgfqpoint{1.197866in}{1.165352in}}%
\pgfpathlineto{\pgfqpoint{1.211394in}{1.174433in}}%
\pgfpathlineto{\pgfqpoint{1.226829in}{1.182813in}}%
\pgfpathlineto{\pgfqpoint{1.253515in}{1.194064in}}%
\pgfpathlineto{\pgfqpoint{1.284483in}{1.203725in}}%
\pgfpathlineto{\pgfqpoint{1.319901in}{1.211792in}}%
\pgfpathlineto{\pgfqpoint{1.360073in}{1.218258in}}%
\pgfpathlineto{\pgfqpoint{1.405436in}{1.223121in}}%
\pgfpathlineto{\pgfqpoint{1.456356in}{1.226369in}}%
\pgfpathlineto{\pgfqpoint{1.513088in}{1.227956in}}%
\pgfpathlineto{\pgfqpoint{1.576474in}{1.227824in}}%
\pgfpathlineto{\pgfqpoint{1.647290in}{1.225908in}}%
\pgfpathlineto{\pgfqpoint{1.754418in}{1.220451in}}%
\pgfpathlineto{\pgfqpoint{1.877254in}{1.211489in}}%
\pgfpathlineto{\pgfqpoint{2.016791in}{1.198793in}}%
\pgfpathlineto{\pgfqpoint{2.173826in}{1.182101in}}%
\pgfpathlineto{\pgfqpoint{2.349153in}{1.161112in}}%
\pgfpathlineto{\pgfqpoint{2.487198in}{1.142546in}}%
\pgfpathlineto{\pgfqpoint{2.625655in}{1.121682in}}%
\pgfpathlineto{\pgfqpoint{2.760385in}{1.098710in}}%
\pgfpathlineto{\pgfqpoint{2.887757in}{1.073894in}}%
\pgfpathlineto{\pgfqpoint{2.967018in}{1.056488in}}%
\pgfpathlineto{\pgfqpoint{3.040807in}{1.038530in}}%
\pgfpathlineto{\pgfqpoint{3.108421in}{1.020151in}}%
\pgfpathlineto{\pgfqpoint{3.169258in}{1.001499in}}%
\pgfpathlineto{\pgfqpoint{3.222815in}{0.982735in}}%
\pgfpathlineto{\pgfqpoint{3.268783in}{0.964031in}}%
\pgfpathlineto{\pgfqpoint{3.308007in}{0.945488in}}%
\pgfpathlineto{\pgfqpoint{3.341243in}{0.927201in}}%
\pgfpathlineto{\pgfqpoint{3.369047in}{0.909257in}}%
\pgfpathlineto{\pgfqpoint{3.391926in}{0.891735in}}%
\pgfpathlineto{\pgfqpoint{3.410335in}{0.874699in}}%
\pgfpathlineto{\pgfqpoint{3.424745in}{0.858202in}}%
\pgfpathlineto{\pgfqpoint{3.435604in}{0.842284in}}%
\pgfpathlineto{\pgfqpoint{3.443272in}{0.826977in}}%
\pgfpathlineto{\pgfqpoint{3.448039in}{0.812306in}}%
\pgfpathlineto{\pgfqpoint{3.450121in}{0.798290in}}%
\pgfpathlineto{\pgfqpoint{3.449627in}{0.784944in}}%
\pgfpathlineto{\pgfqpoint{3.446720in}{0.772277in}}%
\pgfpathlineto{\pgfqpoint{3.441636in}{0.760297in}}%
\pgfpathlineto{\pgfqpoint{3.434556in}{0.749009in}}%
\pgfpathlineto{\pgfqpoint{3.425603in}{0.738419in}}%
\pgfpathlineto{\pgfqpoint{3.414844in}{0.728529in}}%
\pgfpathlineto{\pgfqpoint{3.402288in}{0.719338in}}%
\pgfpathlineto{\pgfqpoint{3.387887in}{0.710846in}}%
\pgfpathlineto{\pgfqpoint{3.371537in}{0.703048in}}%
\pgfpathlineto{\pgfqpoint{3.342990in}{0.692643in}}%
\pgfpathlineto{\pgfqpoint{3.309647in}{0.683797in}}%
\pgfpathlineto{\pgfqpoint{3.271636in}{0.676536in}}%
\pgfpathlineto{\pgfqpoint{3.228686in}{0.670877in}}%
\pgfpathlineto{\pgfqpoint{3.180441in}{0.666848in}}%
\pgfpathlineto{\pgfqpoint{3.126460in}{0.664478in}}%
\pgfpathlineto{\pgfqpoint{3.066215in}{0.663808in}}%
\pgfpathlineto{\pgfqpoint{2.999094in}{0.664882in}}%
\pgfpathlineto{\pgfqpoint{2.897800in}{0.669120in}}%
\pgfpathlineto{\pgfqpoint{2.781404in}{0.676764in}}%
\pgfpathlineto{\pgfqpoint{2.647943in}{0.688079in}}%
\pgfpathlineto{\pgfqpoint{2.497017in}{0.703305in}}%
\pgfpathlineto{\pgfqpoint{2.329920in}{0.722644in}}%
\pgfpathlineto{\pgfqpoint{2.149615in}{0.746270in}}%
\pgfpathlineto{\pgfqpoint{2.010232in}{0.766785in}}%
\pgfpathlineto{\pgfqpoint{1.873046in}{0.789476in}}%
\pgfpathlineto{\pgfqpoint{1.742632in}{0.814065in}}%
\pgfpathlineto{\pgfqpoint{1.661381in}{0.831343in}}%
\pgfpathlineto{\pgfqpoint{1.585823in}{0.849192in}}%
\pgfpathlineto{\pgfqpoint{1.516776in}{0.867479in}}%
\pgfpathlineto{\pgfqpoint{1.454910in}{0.886058in}}%
\pgfpathlineto{\pgfqpoint{1.400677in}{0.904769in}}%
\pgfpathlineto{\pgfqpoint{1.353668in}{0.923467in}}%
\pgfpathlineto{\pgfqpoint{1.313123in}{0.942040in}}%
\pgfpathlineto{\pgfqpoint{1.278393in}{0.960390in}}%
\pgfpathlineto{\pgfqpoint{1.248951in}{0.978425in}}%
\pgfpathlineto{\pgfqpoint{1.224392in}{0.996065in}}%
\pgfpathlineto{\pgfqpoint{1.204428in}{1.013241in}}%
\pgfpathlineto{\pgfqpoint{1.188895in}{1.029892in}}%
\pgfpathlineto{\pgfqpoint{1.177465in}{1.045968in}}%
\pgfpathlineto{\pgfqpoint{1.169368in}{1.061434in}}%
\pgfpathlineto{\pgfqpoint{1.164190in}{1.076265in}}%
\pgfpathlineto{\pgfqpoint{1.161617in}{1.090444in}}%
\pgfpathlineto{\pgfqpoint{1.161426in}{1.103955in}}%
\pgfpathlineto{\pgfqpoint{1.163484in}{1.116786in}}%
\pgfpathlineto{\pgfqpoint{1.167748in}{1.128931in}}%
\pgfpathlineto{\pgfqpoint{1.174268in}{1.140386in}}%
\pgfpathlineto{\pgfqpoint{1.183007in}{1.151149in}}%
\pgfpathlineto{\pgfqpoint{1.193759in}{1.161216in}}%
\pgfpathlineto{\pgfqpoint{1.206439in}{1.170582in}}%
\pgfpathlineto{\pgfqpoint{1.220992in}{1.179247in}}%
\pgfpathlineto{\pgfqpoint{1.246290in}{1.190924in}}%
\pgfpathlineto{\pgfqpoint{1.275811in}{1.201015in}}%
\pgfpathlineto{\pgfqpoint{1.309765in}{1.209519in}}%
\pgfpathlineto{\pgfqpoint{1.348521in}{1.216440in}}%
\pgfpathlineto{\pgfqpoint{1.392474in}{1.221776in}}%
\pgfpathlineto{\pgfqpoint{1.441723in}{1.225493in}}%
\pgfpathlineto{\pgfqpoint{1.496838in}{1.227552in}}%
\pgfpathlineto{\pgfqpoint{1.558438in}{1.227906in}}%
\pgfpathlineto{\pgfqpoint{1.627148in}{1.226496in}}%
\pgfpathlineto{\pgfqpoint{1.730909in}{1.221754in}}%
\pgfpathlineto{\pgfqpoint{1.849930in}{1.213558in}}%
\pgfpathlineto{\pgfqpoint{1.985720in}{1.201681in}}%
\pgfpathlineto{\pgfqpoint{2.139071in}{1.185875in}}%
\pgfpathlineto{\pgfqpoint{2.308717in}{1.165881in}}%
\pgfpathlineto{\pgfqpoint{2.444068in}{1.148079in}}%
\pgfpathlineto{\pgfqpoint{2.582906in}{1.127918in}}%
\pgfpathlineto{\pgfqpoint{2.720908in}{1.105516in}}%
\pgfpathlineto{\pgfqpoint{2.853061in}{1.081101in}}%
\pgfpathlineto{\pgfqpoint{2.935559in}{1.063911in}}%
\pgfpathlineto{\pgfqpoint{3.012299in}{1.046145in}}%
\pgfpathlineto{\pgfqpoint{3.082506in}{1.027936in}}%
\pgfpathlineto{\pgfqpoint{3.145684in}{1.009418in}}%
\pgfpathlineto{\pgfqpoint{3.201618in}{0.990727in}}%
\pgfpathlineto{\pgfqpoint{3.250377in}{0.972000in}}%
\pgfpathlineto{\pgfqpoint{3.291981in}{0.953393in}}%
\pgfpathlineto{\pgfqpoint{3.327039in}{0.935022in}}%
\pgfpathlineto{\pgfqpoint{3.356699in}{0.916958in}}%
\pgfpathlineto{\pgfqpoint{3.381849in}{0.899264in}}%
\pgfpathlineto{\pgfqpoint{3.403113in}{0.882000in}}%
\pgfpathlineto{\pgfqpoint{3.420851in}{0.865221in}}%
\pgfpathlineto{\pgfqpoint{3.435160in}{0.848975in}}%
\pgfpathlineto{\pgfqpoint{3.445872in}{0.833308in}}%
\pgfpathlineto{\pgfqpoint{3.452559in}{0.818260in}}%
\pgfpathlineto{\pgfqpoint{3.455251in}{0.803867in}}%
\pgfpathlineto{\pgfqpoint{3.455334in}{0.790152in}}%
\pgfpathlineto{\pgfqpoint{3.453100in}{0.777124in}}%
\pgfpathlineto{\pgfqpoint{3.448733in}{0.764792in}}%
\pgfpathlineto{\pgfqpoint{3.442358in}{0.753159in}}%
\pgfpathlineto{\pgfqpoint{3.434048in}{0.742228in}}%
\pgfpathlineto{\pgfqpoint{3.423814in}{0.732002in}}%
\pgfpathlineto{\pgfqpoint{3.411615in}{0.722477in}}%
\pgfpathlineto{\pgfqpoint{3.397362in}{0.713652in}}%
\pgfpathlineto{\pgfqpoint{3.381122in}{0.705526in}}%
\pgfpathlineto{\pgfqpoint{3.353142in}{0.694654in}}%
\pgfpathlineto{\pgfqpoint{3.320778in}{0.685371in}}%
\pgfpathlineto{\pgfqpoint{3.283884in}{0.677688in}}%
\pgfpathlineto{\pgfqpoint{3.242199in}{0.671618in}}%
\pgfpathlineto{\pgfqpoint{3.195354in}{0.667177in}}%
\pgfpathlineto{\pgfqpoint{3.142868in}{0.664382in}}%
\pgfpathlineto{\pgfqpoint{3.084211in}{0.663251in}}%
\pgfpathlineto{\pgfqpoint{3.019277in}{0.663816in}}%
\pgfpathlineto{\pgfqpoint{2.920913in}{0.667377in}}%
\pgfpathlineto{\pgfqpoint{2.806820in}{0.674370in}}%
\pgfpathlineto{\pgfqpoint{2.675697in}{0.685019in}}%
\pgfpathlineto{\pgfqpoint{2.527488in}{0.699532in}}%
\pgfpathlineto{\pgfqpoint{2.363378in}{0.718097in}}%
\pgfpathlineto{\pgfqpoint{2.185796in}{0.740883in}}%
\pgfpathlineto{\pgfqpoint{2.046962in}{0.760778in}}%
\pgfpathlineto{\pgfqpoint{1.908470in}{0.782936in}}%
\pgfpathlineto{\pgfqpoint{1.775170in}{0.807117in}}%
\pgfpathlineto{\pgfqpoint{1.775170in}{0.807117in}}%
\pgfusepath{stroke}%
\end{pgfscope}%
\begin{pgfscope}%
\pgfpathrectangle{\pgfqpoint{0.562500in}{0.275000in}}{\pgfqpoint{3.487500in}{1.925000in}}%
\pgfusepath{clip}%
\pgfsetrectcap%
\pgfsetroundjoin%
\pgfsetlinewidth{1.505625pt}%
\definecolor{currentstroke}{rgb}{0.549020,0.337255,0.294118}%
\pgfsetstrokecolor{currentstroke}%
\pgfsetdash{}{0pt}%
\pgfpathmoveto{\pgfqpoint{0.721023in}{0.362500in}}%
\pgfpathlineto{\pgfqpoint{0.900108in}{0.390449in}}%
\pgfpathlineto{\pgfqpoint{1.011075in}{0.416426in}}%
\pgfpathlineto{\pgfqpoint{1.080172in}{0.440941in}}%
\pgfpathlineto{\pgfqpoint{1.127471in}{0.464339in}}%
\pgfpathlineto{\pgfqpoint{1.159575in}{0.486806in}}%
\pgfpathlineto{\pgfqpoint{1.180224in}{0.508476in}}%
\pgfpathlineto{\pgfqpoint{1.193151in}{0.529458in}}%
\pgfpathlineto{\pgfqpoint{1.200871in}{0.549836in}}%
\pgfpathlineto{\pgfqpoint{1.204452in}{0.569661in}}%
\pgfpathlineto{\pgfqpoint{1.204842in}{0.588974in}}%
\pgfpathlineto{\pgfqpoint{1.202898in}{0.607814in}}%
\pgfpathlineto{\pgfqpoint{1.194903in}{0.644199in}}%
\pgfpathlineto{\pgfqpoint{1.183629in}{0.678993in}}%
\pgfpathlineto{\pgfqpoint{1.163659in}{0.728431in}}%
\pgfpathlineto{\pgfqpoint{1.113406in}{0.845958in}}%
\pgfpathlineto{\pgfqpoint{1.098728in}{0.885132in}}%
\pgfpathlineto{\pgfqpoint{1.087026in}{0.921822in}}%
\pgfpathlineto{\pgfqpoint{1.078643in}{0.956096in}}%
\pgfpathlineto{\pgfqpoint{1.073312in}{0.988071in}}%
\pgfpathlineto{\pgfqpoint{1.070786in}{1.017859in}}%
\pgfpathlineto{\pgfqpoint{1.071058in}{1.045550in}}%
\pgfpathlineto{\pgfqpoint{1.072896in}{1.062879in}}%
\pgfpathlineto{\pgfqpoint{1.076200in}{1.079318in}}%
\pgfpathlineto{\pgfqpoint{1.081142in}{1.094877in}}%
\pgfpathlineto{\pgfqpoint{1.087942in}{1.109558in}}%
\pgfpathlineto{\pgfqpoint{1.096542in}{1.123377in}}%
\pgfpathlineto{\pgfqpoint{1.106676in}{1.136356in}}%
\pgfpathlineto{\pgfqpoint{1.118279in}{1.148508in}}%
\pgfpathlineto{\pgfqpoint{1.131316in}{1.159845in}}%
\pgfpathlineto{\pgfqpoint{1.145788in}{1.170380in}}%
\pgfpathlineto{\pgfqpoint{1.170274in}{1.184701in}}%
\pgfpathlineto{\pgfqpoint{1.198338in}{1.197276in}}%
\pgfpathlineto{\pgfqpoint{1.230414in}{1.208138in}}%
\pgfpathlineto{\pgfqpoint{1.267040in}{1.217312in}}%
\pgfpathlineto{\pgfqpoint{1.308215in}{1.224799in}}%
\pgfpathlineto{\pgfqpoint{1.354192in}{1.230589in}}%
\pgfpathlineto{\pgfqpoint{1.405413in}{1.234664in}}%
\pgfpathlineto{\pgfqpoint{1.462395in}{1.236997in}}%
\pgfpathlineto{\pgfqpoint{1.525729in}{1.237552in}}%
\pgfpathlineto{\pgfqpoint{1.596082in}{1.236281in}}%
\pgfpathlineto{\pgfqpoint{1.702092in}{1.231643in}}%
\pgfpathlineto{\pgfqpoint{1.823743in}{1.223478in}}%
\pgfpathlineto{\pgfqpoint{1.962808in}{1.211516in}}%
\pgfpathlineto{\pgfqpoint{2.120000in}{1.195494in}}%
\pgfpathlineto{\pgfqpoint{2.293937in}{1.175191in}}%
\pgfpathlineto{\pgfqpoint{2.433165in}{1.157065in}}%
\pgfpathlineto{\pgfqpoint{2.575818in}{1.136490in}}%
\pgfpathlineto{\pgfqpoint{2.717265in}{1.113613in}}%
\pgfpathlineto{\pgfqpoint{2.852715in}{1.088697in}}%
\pgfpathlineto{\pgfqpoint{2.937242in}{1.071132in}}%
\pgfpathlineto{\pgfqpoint{3.015389in}{1.052959in}}%
\pgfpathlineto{\pgfqpoint{3.086300in}{1.034327in}}%
\pgfpathlineto{\pgfqpoint{3.149947in}{1.015398in}}%
\pgfpathlineto{\pgfqpoint{3.206414in}{0.996320in}}%
\pgfpathlineto{\pgfqpoint{3.255871in}{0.977229in}}%
\pgfpathlineto{\pgfqpoint{3.298577in}{0.958248in}}%
\pgfpathlineto{\pgfqpoint{3.334877in}{0.939488in}}%
\pgfpathlineto{\pgfqpoint{3.365204in}{0.921045in}}%
\pgfpathlineto{\pgfqpoint{3.390079in}{0.903004in}}%
\pgfpathlineto{\pgfqpoint{3.410068in}{0.885439in}}%
\pgfpathlineto{\pgfqpoint{3.425582in}{0.868422in}}%
\pgfpathlineto{\pgfqpoint{3.437361in}{0.851988in}}%
\pgfpathlineto{\pgfqpoint{3.446063in}{0.836164in}}%
\pgfpathlineto{\pgfqpoint{3.452166in}{0.820973in}}%
\pgfpathlineto{\pgfqpoint{3.455970in}{0.806434in}}%
\pgfpathlineto{\pgfqpoint{3.457596in}{0.792561in}}%
\pgfpathlineto{\pgfqpoint{3.456988in}{0.779366in}}%
\pgfpathlineto{\pgfqpoint{3.453911in}{0.766855in}}%
\pgfpathlineto{\pgfqpoint{3.447967in}{0.755031in}}%
\pgfpathlineto{\pgfqpoint{3.439566in}{0.743906in}}%
\pgfpathlineto{\pgfqpoint{3.429194in}{0.733492in}}%
\pgfpathlineto{\pgfqpoint{3.416928in}{0.723787in}}%
\pgfpathlineto{\pgfqpoint{3.402817in}{0.714795in}}%
\pgfpathlineto{\pgfqpoint{3.386878in}{0.706514in}}%
\pgfpathlineto{\pgfqpoint{3.359500in}{0.695426in}}%
\pgfpathlineto{\pgfqpoint{3.327787in}{0.685935in}}%
\pgfpathlineto{\pgfqpoint{3.291397in}{0.678034in}}%
\pgfpathlineto{\pgfqpoint{3.250078in}{0.671727in}}%
\pgfpathlineto{\pgfqpoint{3.203665in}{0.667038in}}%
\pgfpathlineto{\pgfqpoint{3.151669in}{0.663998in}}%
\pgfpathlineto{\pgfqpoint{3.093555in}{0.662646in}}%
\pgfpathlineto{\pgfqpoint{3.028743in}{0.663033in}}%
\pgfpathlineto{\pgfqpoint{2.930826in}{0.666360in}}%
\pgfpathlineto{\pgfqpoint{2.818290in}{0.673061in}}%
\pgfpathlineto{\pgfqpoint{2.689449in}{0.683343in}}%
\pgfpathlineto{\pgfqpoint{2.543106in}{0.697471in}}%
\pgfpathlineto{\pgfqpoint{2.379590in}{0.715694in}}%
\pgfpathlineto{\pgfqpoint{2.201845in}{0.738176in}}%
\pgfpathlineto{\pgfqpoint{2.062961in}{0.757841in}}%
\pgfpathlineto{\pgfqpoint{1.924009in}{0.779790in}}%
\pgfpathlineto{\pgfqpoint{1.789661in}{0.803797in}}%
\pgfpathlineto{\pgfqpoint{1.705072in}{0.820784in}}%
\pgfpathlineto{\pgfqpoint{1.626288in}{0.838430in}}%
\pgfpathlineto{\pgfqpoint{1.554074in}{0.856584in}}%
\pgfpathlineto{\pgfqpoint{1.488674in}{0.875085in}}%
\pgfpathlineto{\pgfqpoint{1.430203in}{0.893780in}}%
\pgfpathlineto{\pgfqpoint{1.378647in}{0.912533in}}%
\pgfpathlineto{\pgfqpoint{1.333865in}{0.931219in}}%
\pgfpathlineto{\pgfqpoint{1.295588in}{0.949728in}}%
\pgfpathlineto{\pgfqpoint{1.263418in}{0.967963in}}%
\pgfpathlineto{\pgfqpoint{1.236830in}{0.985840in}}%
\pgfpathlineto{\pgfqpoint{1.215171in}{1.003288in}}%
\pgfpathlineto{\pgfqpoint{1.197886in}{1.020242in}}%
\pgfpathlineto{\pgfqpoint{1.184840in}{1.036633in}}%
\pgfpathlineto{\pgfqpoint{1.175261in}{1.052434in}}%
\pgfpathlineto{\pgfqpoint{1.168489in}{1.067620in}}%
\pgfpathlineto{\pgfqpoint{1.164050in}{1.082173in}}%
\pgfpathlineto{\pgfqpoint{1.161651in}{1.096076in}}%
\pgfpathlineto{\pgfqpoint{1.161182in}{1.109319in}}%
\pgfpathlineto{\pgfqpoint{1.162716in}{1.121892in}}%
\pgfpathlineto{\pgfqpoint{1.166509in}{1.133792in}}%
\pgfpathlineto{\pgfqpoint{1.173001in}{1.145017in}}%
\pgfpathlineto{\pgfqpoint{1.182561in}{1.155566in}}%
\pgfpathlineto{\pgfqpoint{1.194264in}{1.165413in}}%
\pgfpathlineto{\pgfqpoint{1.207878in}{1.174555in}}%
\pgfpathlineto{\pgfqpoint{1.223368in}{1.182989in}}%
\pgfpathlineto{\pgfqpoint{1.250104in}{1.194311in}}%
\pgfpathlineto{\pgfqpoint{1.281122in}{1.204037in}}%
\pgfpathlineto{\pgfqpoint{1.316633in}{1.212166in}}%
\pgfpathlineto{\pgfqpoint{1.356983in}{1.218698in}}%
\pgfpathlineto{\pgfqpoint{1.402498in}{1.223626in}}%
\pgfpathlineto{\pgfqpoint{1.453393in}{1.226921in}}%
\pgfpathlineto{\pgfqpoint{1.510308in}{1.228541in}}%
\pgfpathlineto{\pgfqpoint{1.573897in}{1.228438in}}%
\pgfpathlineto{\pgfqpoint{1.644798in}{1.226548in}}%
\pgfpathlineto{\pgfqpoint{1.751798in}{1.221125in}}%
\pgfpathlineto{\pgfqpoint{1.874382in}{1.212196in}}%
\pgfpathlineto{\pgfqpoint{2.013973in}{1.199526in}}%
\pgfpathlineto{\pgfqpoint{2.171257in}{1.182876in}}%
\pgfpathlineto{\pgfqpoint{2.344149in}{1.161986in}}%
\pgfpathlineto{\pgfqpoint{2.480996in}{1.143492in}}%
\pgfpathlineto{\pgfqpoint{2.620283in}{1.122657in}}%
\pgfpathlineto{\pgfqpoint{2.757555in}{1.099651in}}%
\pgfpathlineto{\pgfqpoint{2.845302in}{1.083230in}}%
\pgfpathlineto{\pgfqpoint{2.928306in}{1.066051in}}%
\pgfpathlineto{\pgfqpoint{3.005588in}{1.048270in}}%
\pgfpathlineto{\pgfqpoint{3.076425in}{1.030041in}}%
\pgfpathlineto{\pgfqpoint{3.140331in}{1.011512in}}%
\pgfpathlineto{\pgfqpoint{3.197067in}{0.992819in}}%
\pgfpathlineto{\pgfqpoint{3.246631in}{0.974090in}}%
\pgfpathlineto{\pgfqpoint{3.289263in}{0.955444in}}%
\pgfpathlineto{\pgfqpoint{3.325446in}{0.936991in}}%
\pgfpathlineto{\pgfqpoint{3.355871in}{0.918833in}}%
\pgfpathlineto{\pgfqpoint{3.380736in}{0.901079in}}%
\pgfpathlineto{\pgfqpoint{3.400803in}{0.883791in}}%
\pgfpathlineto{\pgfqpoint{3.416989in}{0.867010in}}%
\pgfpathlineto{\pgfqpoint{3.429974in}{0.850774in}}%
\pgfpathlineto{\pgfqpoint{3.440206in}{0.835114in}}%
\pgfpathlineto{\pgfqpoint{3.447898in}{0.820060in}}%
\pgfpathlineto{\pgfqpoint{3.453027in}{0.805634in}}%
\pgfpathlineto{\pgfqpoint{3.455336in}{0.791856in}}%
\pgfpathlineto{\pgfqpoint{3.454333in}{0.778741in}}%
\pgfpathlineto{\pgfqpoint{3.450169in}{0.766307in}}%
\pgfpathlineto{\pgfqpoint{3.443852in}{0.754573in}}%
\pgfpathlineto{\pgfqpoint{3.435525in}{0.743542in}}%
\pgfpathlineto{\pgfqpoint{3.425285in}{0.733217in}}%
\pgfpathlineto{\pgfqpoint{3.413189in}{0.723598in}}%
\pgfpathlineto{\pgfqpoint{3.399256in}{0.714686in}}%
\pgfpathlineto{\pgfqpoint{3.383468in}{0.706481in}}%
\pgfpathlineto{\pgfqpoint{3.356169in}{0.695492in}}%
\pgfpathlineto{\pgfqpoint{3.324246in}{0.686077in}}%
\pgfpathlineto{\pgfqpoint{3.287686in}{0.678245in}}%
\pgfpathlineto{\pgfqpoint{3.246303in}{0.672013in}}%
\pgfpathlineto{\pgfqpoint{3.199756in}{0.667404in}}%
\pgfpathlineto{\pgfqpoint{3.147627in}{0.664448in}}%
\pgfpathlineto{\pgfqpoint{3.089417in}{0.663184in}}%
\pgfpathlineto{\pgfqpoint{3.024547in}{0.663656in}}%
\pgfpathlineto{\pgfqpoint{2.926547in}{0.667080in}}%
\pgfpathlineto{\pgfqpoint{2.813890in}{0.673848in}}%
\pgfpathlineto{\pgfqpoint{2.684752in}{0.684215in}}%
\pgfpathlineto{\pgfqpoint{2.538050in}{0.698439in}}%
\pgfpathlineto{\pgfqpoint{2.374488in}{0.716743in}}%
\pgfpathlineto{\pgfqpoint{2.196584in}{0.739309in}}%
\pgfpathlineto{\pgfqpoint{2.057724in}{0.759036in}}%
\pgfpathlineto{\pgfqpoint{1.919255in}{0.781030in}}%
\pgfpathlineto{\pgfqpoint{1.785661in}{0.805058in}}%
\pgfpathlineto{\pgfqpoint{1.701455in}{0.822050in}}%
\pgfpathlineto{\pgfqpoint{1.622593in}{0.839680in}}%
\pgfpathlineto{\pgfqpoint{1.550293in}{0.857810in}}%
\pgfpathlineto{\pgfqpoint{1.485296in}{0.876286in}}%
\pgfpathlineto{\pgfqpoint{1.427477in}{0.894963in}}%
\pgfpathlineto{\pgfqpoint{1.376579in}{0.913703in}}%
\pgfpathlineto{\pgfqpoint{1.332303in}{0.932382in}}%
\pgfpathlineto{\pgfqpoint{1.294311in}{0.950888in}}%
\pgfpathlineto{\pgfqpoint{1.262224in}{0.969119in}}%
\pgfpathlineto{\pgfqpoint{1.235625in}{0.986986in}}%
\pgfpathlineto{\pgfqpoint{1.214053in}{1.004413in}}%
\pgfpathlineto{\pgfqpoint{1.197001in}{1.021330in}}%
\pgfpathlineto{\pgfqpoint{1.183833in}{1.037692in}}%
\pgfpathlineto{\pgfqpoint{1.173992in}{1.053466in}}%
\pgfpathlineto{\pgfqpoint{1.167053in}{1.068623in}}%
\pgfpathlineto{\pgfqpoint{1.162729in}{1.083141in}}%
\pgfpathlineto{\pgfqpoint{1.160868in}{1.097003in}}%
\pgfpathlineto{\pgfqpoint{1.161454in}{1.110193in}}%
\pgfpathlineto{\pgfqpoint{1.164608in}{1.122707in}}%
\pgfpathlineto{\pgfqpoint{1.170261in}{1.134534in}}%
\pgfpathlineto{\pgfqpoint{1.178021in}{1.145666in}}%
\pgfpathlineto{\pgfqpoint{1.187764in}{1.156100in}}%
\pgfpathlineto{\pgfqpoint{1.199403in}{1.165831in}}%
\pgfpathlineto{\pgfqpoint{1.212888in}{1.174860in}}%
\pgfpathlineto{\pgfqpoint{1.228204in}{1.183185in}}%
\pgfpathlineto{\pgfqpoint{1.254675in}{1.194353in}}%
\pgfpathlineto{\pgfqpoint{1.285562in}{1.203944in}}%
\pgfpathlineto{\pgfqpoint{1.321253in}{1.211965in}}%
\pgfpathlineto{\pgfqpoint{1.361795in}{1.218403in}}%
\pgfpathlineto{\pgfqpoint{1.407442in}{1.223231in}}%
\pgfpathlineto{\pgfqpoint{1.458632in}{1.226421in}}%
\pgfpathlineto{\pgfqpoint{1.515863in}{1.227932in}}%
\pgfpathlineto{\pgfqpoint{1.579692in}{1.227716in}}%
\pgfpathlineto{\pgfqpoint{1.676145in}{1.224639in}}%
\pgfpathlineto{\pgfqpoint{1.787075in}{1.218225in}}%
\pgfpathlineto{\pgfqpoint{1.914167in}{1.208262in}}%
\pgfpathlineto{\pgfqpoint{2.058682in}{1.194482in}}%
\pgfpathlineto{\pgfqpoint{2.220336in}{1.176643in}}%
\pgfpathlineto{\pgfqpoint{2.396613in}{1.154566in}}%
\pgfpathlineto{\pgfqpoint{2.534706in}{1.135209in}}%
\pgfpathlineto{\pgfqpoint{2.673478in}{1.113557in}}%
\pgfpathlineto{\pgfqpoint{2.808282in}{1.089823in}}%
\pgfpathlineto{\pgfqpoint{2.893342in}{1.072986in}}%
\pgfpathlineto{\pgfqpoint{2.972950in}{1.055464in}}%
\pgfpathlineto{\pgfqpoint{3.046387in}{1.037422in}}%
\pgfpathlineto{\pgfqpoint{3.113147in}{1.019019in}}%
\pgfpathlineto{\pgfqpoint{3.172924in}{1.000401in}}%
\pgfpathlineto{\pgfqpoint{3.225611in}{0.981702in}}%
\pgfpathlineto{\pgfqpoint{3.271302in}{0.963045in}}%
\pgfpathlineto{\pgfqpoint{3.310287in}{0.944541in}}%
\pgfpathlineto{\pgfqpoint{3.343056in}{0.926289in}}%
\pgfpathlineto{\pgfqpoint{3.370299in}{0.908377in}}%
\pgfpathlineto{\pgfqpoint{3.392750in}{0.890885in}}%
\pgfpathlineto{\pgfqpoint{3.410483in}{0.873895in}}%
\pgfpathlineto{\pgfqpoint{3.424261in}{0.857448in}}%
\pgfpathlineto{\pgfqpoint{3.434816in}{0.841576in}}%
\pgfpathlineto{\pgfqpoint{3.442680in}{0.826305in}}%
\pgfpathlineto{\pgfqpoint{3.448188in}{0.811657in}}%
\pgfpathlineto{\pgfqpoint{3.451476in}{0.797651in}}%
\pgfpathlineto{\pgfqpoint{3.452481in}{0.784301in}}%
\pgfpathlineto{\pgfqpoint{3.450943in}{0.771617in}}%
\pgfpathlineto{\pgfqpoint{3.446402in}{0.759605in}}%
\pgfpathlineto{\pgfqpoint{3.439007in}{0.748277in}}%
\pgfpathlineto{\pgfqpoint{3.429581in}{0.737651in}}%
\pgfpathlineto{\pgfqpoint{3.418222in}{0.727729in}}%
\pgfpathlineto{\pgfqpoint{3.404988in}{0.718512in}}%
\pgfpathlineto{\pgfqpoint{3.389909in}{0.710003in}}%
\pgfpathlineto{\pgfqpoint{3.363812in}{0.698563in}}%
\pgfpathlineto{\pgfqpoint{3.333402in}{0.688712in}}%
\pgfpathlineto{\pgfqpoint{3.298371in}{0.680444in}}%
\pgfpathlineto{\pgfqpoint{3.258444in}{0.673758in}}%
\pgfpathlineto{\pgfqpoint{3.213504in}{0.668680in}}%
\pgfpathlineto{\pgfqpoint{3.163097in}{0.665238in}}%
\pgfpathlineto{\pgfqpoint{3.106712in}{0.663468in}}%
\pgfpathlineto{\pgfqpoint{3.043791in}{0.663419in}}%
\pgfpathlineto{\pgfqpoint{2.973724in}{0.665150in}}%
\pgfpathlineto{\pgfqpoint{2.868043in}{0.670344in}}%
\pgfpathlineto{\pgfqpoint{2.746794in}{0.679012in}}%
\pgfpathlineto{\pgfqpoint{2.608492in}{0.691398in}}%
\pgfpathlineto{\pgfqpoint{2.452635in}{0.707770in}}%
\pgfpathlineto{\pgfqpoint{2.280952in}{0.728331in}}%
\pgfpathlineto{\pgfqpoint{2.144452in}{0.746564in}}%
\pgfpathlineto{\pgfqpoint{2.005281in}{0.767157in}}%
\pgfpathlineto{\pgfqpoint{1.867851in}{0.789951in}}%
\pgfpathlineto{\pgfqpoint{1.737023in}{0.814676in}}%
\pgfpathlineto{\pgfqpoint{1.656064in}{0.832069in}}%
\pgfpathlineto{\pgfqpoint{1.581266in}{0.850026in}}%
\pgfpathlineto{\pgfqpoint{1.513098in}{0.868384in}}%
\pgfpathlineto{\pgfqpoint{1.451850in}{0.886989in}}%
\pgfpathlineto{\pgfqpoint{1.397638in}{0.905703in}}%
\pgfpathlineto{\pgfqpoint{1.350405in}{0.924398in}}%
\pgfpathlineto{\pgfqpoint{1.309913in}{0.942960in}}%
\pgfpathlineto{\pgfqpoint{1.275753in}{0.961286in}}%
\pgfpathlineto{\pgfqpoint{1.247339in}{0.979289in}}%
\pgfpathlineto{\pgfqpoint{1.223908in}{0.996890in}}%
\pgfpathlineto{\pgfqpoint{1.205075in}{1.014010in}}%
\pgfpathlineto{\pgfqpoint{1.190545in}{1.030590in}}%
\pgfpathlineto{\pgfqpoint{1.179497in}{1.046597in}}%
\pgfpathlineto{\pgfqpoint{1.171301in}{1.062006in}}%
\pgfpathlineto{\pgfqpoint{1.165520in}{1.076794in}}%
\pgfpathlineto{\pgfqpoint{1.161911in}{1.090942in}}%
\pgfpathlineto{\pgfqpoint{1.160425in}{1.104436in}}%
\pgfpathlineto{\pgfqpoint{1.161206in}{1.117266in}}%
\pgfpathlineto{\pgfqpoint{1.164590in}{1.129425in}}%
\pgfpathlineto{\pgfqpoint{1.171075in}{1.140911in}}%
\pgfpathlineto{\pgfqpoint{1.180059in}{1.151704in}}%
\pgfpathlineto{\pgfqpoint{1.191014in}{1.161793in}}%
\pgfpathlineto{\pgfqpoint{1.203869in}{1.171177in}}%
\pgfpathlineto{\pgfqpoint{1.218584in}{1.179854in}}%
\pgfpathlineto{\pgfqpoint{1.244130in}{1.191542in}}%
\pgfpathlineto{\pgfqpoint{1.273934in}{1.201640in}}%
\pgfpathlineto{\pgfqpoint{1.308238in}{1.210148in}}%
\pgfpathlineto{\pgfqpoint{1.347413in}{1.217071in}}%
\pgfpathlineto{\pgfqpoint{1.391584in}{1.222394in}}%
\pgfpathlineto{\pgfqpoint{1.441120in}{1.226088in}}%
\pgfpathlineto{\pgfqpoint{1.496560in}{1.228116in}}%
\pgfpathlineto{\pgfqpoint{1.558472in}{1.228432in}}%
\pgfpathlineto{\pgfqpoint{1.627463in}{1.226979in}}%
\pgfpathlineto{\pgfqpoint{1.731569in}{1.222168in}}%
\pgfpathlineto{\pgfqpoint{1.851007in}{1.213899in}}%
\pgfpathlineto{\pgfqpoint{1.987348in}{1.201945in}}%
\pgfpathlineto{\pgfqpoint{2.141291in}{1.186033in}}%
\pgfpathlineto{\pgfqpoint{2.311439in}{1.165944in}}%
\pgfpathlineto{\pgfqpoint{2.447134in}{1.148072in}}%
\pgfpathlineto{\pgfqpoint{2.586357in}{1.127818in}}%
\pgfpathlineto{\pgfqpoint{2.724367in}{1.105334in}}%
\pgfpathlineto{\pgfqpoint{2.856390in}{1.080876in}}%
\pgfpathlineto{\pgfqpoint{2.938869in}{1.063647in}}%
\pgfpathlineto{\pgfqpoint{3.015448in}{1.045828in}}%
\pgfpathlineto{\pgfqpoint{3.084653in}{1.027572in}}%
\pgfpathlineto{\pgfqpoint{3.146502in}{1.009032in}}%
\pgfpathlineto{\pgfqpoint{3.201494in}{0.990343in}}%
\pgfpathlineto{\pgfqpoint{3.250075in}{0.971632in}}%
\pgfpathlineto{\pgfqpoint{3.292639in}{0.953014in}}%
\pgfpathlineto{\pgfqpoint{3.329528in}{0.934594in}}%
\pgfpathlineto{\pgfqpoint{3.361032in}{0.916465in}}%
\pgfpathlineto{\pgfqpoint{3.387390in}{0.898709in}}%
\pgfpathlineto{\pgfqpoint{3.408788in}{0.881399in}}%
\pgfpathlineto{\pgfqpoint{3.425360in}{0.864593in}}%
\pgfpathlineto{\pgfqpoint{3.437247in}{0.848345in}}%
\pgfpathlineto{\pgfqpoint{3.445312in}{0.832712in}}%
\pgfpathlineto{\pgfqpoint{3.450312in}{0.817721in}}%
\pgfpathlineto{\pgfqpoint{3.452751in}{0.803389in}}%
\pgfpathlineto{\pgfqpoint{3.453008in}{0.789728in}}%
\pgfpathlineto{\pgfqpoint{3.451329in}{0.776748in}}%
\pgfpathlineto{\pgfqpoint{3.447834in}{0.764456in}}%
\pgfpathlineto{\pgfqpoint{3.442512in}{0.752854in}}%
\pgfpathlineto{\pgfqpoint{3.435222in}{0.741944in}}%
\pgfpathlineto{\pgfqpoint{3.425696in}{0.731720in}}%
\pgfpathlineto{\pgfqpoint{3.413547in}{0.722178in}}%
\pgfpathlineto{\pgfqpoint{3.399107in}{0.713331in}}%
\pgfpathlineto{\pgfqpoint{3.382756in}{0.705189in}}%
\pgfpathlineto{\pgfqpoint{3.354658in}{0.694302in}}%
\pgfpathlineto{\pgfqpoint{3.322210in}{0.685013in}}%
\pgfpathlineto{\pgfqpoint{3.285234in}{0.677329in}}%
\pgfpathlineto{\pgfqpoint{3.243437in}{0.671257in}}%
\pgfpathlineto{\pgfqpoint{3.196412in}{0.666808in}}%
\pgfpathlineto{\pgfqpoint{3.143748in}{0.663994in}}%
\pgfpathlineto{\pgfqpoint{3.085233in}{0.662852in}}%
\pgfpathlineto{\pgfqpoint{3.019912in}{0.663447in}}%
\pgfpathlineto{\pgfqpoint{2.920715in}{0.667069in}}%
\pgfpathlineto{\pgfqpoint{2.806226in}{0.674099in}}%
\pgfpathlineto{\pgfqpoint{2.675371in}{0.684748in}}%
\pgfpathlineto{\pgfqpoint{2.527633in}{0.699246in}}%
\pgfpathlineto{\pgfqpoint{2.363023in}{0.717847in}}%
\pgfpathlineto{\pgfqpoint{2.182690in}{0.740819in}}%
\pgfpathlineto{\pgfqpoint{2.043092in}{0.760847in}}%
\pgfpathlineto{\pgfqpoint{1.905244in}{0.783079in}}%
\pgfpathlineto{\pgfqpoint{1.773401in}{0.807265in}}%
\pgfpathlineto{\pgfqpoint{1.690660in}{0.824320in}}%
\pgfpathlineto{\pgfqpoint{1.613135in}{0.841992in}}%
\pgfpathlineto{\pgfqpoint{1.541624in}{0.860157in}}%
\pgfpathlineto{\pgfqpoint{1.476802in}{0.878677in}}%
\pgfpathlineto{\pgfqpoint{1.419220in}{0.897401in}}%
\pgfpathlineto{\pgfqpoint{1.369241in}{0.916167in}}%
\pgfpathlineto{\pgfqpoint{1.326309in}{0.934841in}}%
\pgfpathlineto{\pgfqpoint{1.289644in}{0.953319in}}%
\pgfpathlineto{\pgfqpoint{1.258620in}{0.971506in}}%
\pgfpathlineto{\pgfqpoint{1.232710in}{0.989319in}}%
\pgfpathlineto{\pgfqpoint{1.211485in}{1.006684in}}%
\pgfpathlineto{\pgfqpoint{1.194613in}{1.023540in}}%
\pgfpathlineto{\pgfqpoint{1.181765in}{1.039838in}}%
\pgfpathlineto{\pgfqpoint{1.172362in}{1.055538in}}%
\pgfpathlineto{\pgfqpoint{1.166010in}{1.070613in}}%
\pgfpathlineto{\pgfqpoint{1.162407in}{1.085043in}}%
\pgfpathlineto{\pgfqpoint{1.161330in}{1.098810in}}%
\pgfpathlineto{\pgfqpoint{1.162635in}{1.111901in}}%
\pgfpathlineto{\pgfqpoint{1.166257in}{1.124308in}}%
\pgfpathlineto{\pgfqpoint{1.172149in}{1.136026in}}%
\pgfpathlineto{\pgfqpoint{1.180158in}{1.147051in}}%
\pgfpathlineto{\pgfqpoint{1.190148in}{1.157379in}}%
\pgfpathlineto{\pgfqpoint{1.202024in}{1.167006in}}%
\pgfpathlineto{\pgfqpoint{1.215731in}{1.175931in}}%
\pgfpathlineto{\pgfqpoint{1.231248in}{1.184152in}}%
\pgfpathlineto{\pgfqpoint{1.257979in}{1.195165in}}%
\pgfpathlineto{\pgfqpoint{1.289077in}{1.204601in}}%
\pgfpathlineto{\pgfqpoint{1.324973in}{1.212468in}}%
\pgfpathlineto{\pgfqpoint{1.365919in}{1.218762in}}%
\pgfpathlineto{\pgfqpoint{1.411975in}{1.223447in}}%
\pgfpathlineto{\pgfqpoint{1.463611in}{1.226494in}}%
\pgfpathlineto{\pgfqpoint{1.521349in}{1.227861in}}%
\pgfpathlineto{\pgfqpoint{1.585757in}{1.227500in}}%
\pgfpathlineto{\pgfqpoint{1.683088in}{1.224220in}}%
\pgfpathlineto{\pgfqpoint{1.794982in}{1.217585in}}%
\pgfpathlineto{\pgfqpoint{1.923117in}{1.207387in}}%
\pgfpathlineto{\pgfqpoint{2.068685in}{1.193358in}}%
\pgfpathlineto{\pgfqpoint{2.231349in}{1.175256in}}%
\pgfpathlineto{\pgfqpoint{2.408302in}{1.152911in}}%
\pgfpathlineto{\pgfqpoint{2.546614in}{1.133360in}}%
\pgfpathlineto{\pgfqpoint{2.685185in}{1.111527in}}%
\pgfpathlineto{\pgfqpoint{2.819357in}{1.087636in}}%
\pgfpathlineto{\pgfqpoint{2.903861in}{1.070721in}}%
\pgfpathlineto{\pgfqpoint{2.982888in}{1.053152in}}%
\pgfpathlineto{\pgfqpoint{3.055614in}{1.035082in}}%
\pgfpathlineto{\pgfqpoint{3.121484in}{1.016659in}}%
\pgfpathlineto{\pgfqpoint{3.180201in}{0.998023in}}%
\pgfpathlineto{\pgfqpoint{3.231730in}{0.979307in}}%
\pgfpathlineto{\pgfqpoint{3.276295in}{0.960639in}}%
\pgfpathlineto{\pgfqpoint{3.314382in}{0.942136in}}%
\pgfpathlineto{\pgfqpoint{3.314382in}{0.942136in}}%
\pgfusepath{stroke}%
\end{pgfscope}%
\begin{pgfscope}%
\pgfsetrectcap%
\pgfsetmiterjoin%
\pgfsetlinewidth{0.803000pt}%
\definecolor{currentstroke}{rgb}{0.000000,0.000000,0.000000}%
\pgfsetstrokecolor{currentstroke}%
\pgfsetdash{}{0pt}%
\pgfpathmoveto{\pgfqpoint{0.562500in}{0.275000in}}%
\pgfpathlineto{\pgfqpoint{0.562500in}{2.200000in}}%
\pgfusepath{stroke}%
\end{pgfscope}%
\begin{pgfscope}%
\pgfsetrectcap%
\pgfsetmiterjoin%
\pgfsetlinewidth{0.803000pt}%
\definecolor{currentstroke}{rgb}{0.000000,0.000000,0.000000}%
\pgfsetstrokecolor{currentstroke}%
\pgfsetdash{}{0pt}%
\pgfpathmoveto{\pgfqpoint{4.050000in}{0.275000in}}%
\pgfpathlineto{\pgfqpoint{4.050000in}{2.200000in}}%
\pgfusepath{stroke}%
\end{pgfscope}%
\begin{pgfscope}%
\pgfsetrectcap%
\pgfsetmiterjoin%
\pgfsetlinewidth{0.803000pt}%
\definecolor{currentstroke}{rgb}{0.000000,0.000000,0.000000}%
\pgfsetstrokecolor{currentstroke}%
\pgfsetdash{}{0pt}%
\pgfpathmoveto{\pgfqpoint{0.562500in}{0.275000in}}%
\pgfpathlineto{\pgfqpoint{4.050000in}{0.275000in}}%
\pgfusepath{stroke}%
\end{pgfscope}%
\begin{pgfscope}%
\pgfsetrectcap%
\pgfsetmiterjoin%
\pgfsetlinewidth{0.803000pt}%
\definecolor{currentstroke}{rgb}{0.000000,0.000000,0.000000}%
\pgfsetstrokecolor{currentstroke}%
\pgfsetdash{}{0pt}%
\pgfpathmoveto{\pgfqpoint{0.562500in}{2.200000in}}%
\pgfpathlineto{\pgfqpoint{4.050000in}{2.200000in}}%
\pgfusepath{stroke}%
\end{pgfscope}%
\end{pgfpicture}%
\makeatother%
\endgroup%

        \end{center}
    \end{frame}
    \begin{frame}
    \frametitle{Poincaré-Bendixson: Spezialfall 3}
        \begin{enumerate}
            \setcounter{enumi}{2}
            \item $\omega(p)$ ist ein geschlossener Orbit welcher Singularitäten verbindet
        \end{enumerate}
        %TODO Plot
    \end{frame}

    \begin{frame}
    \frametitle{Beispiel aus Nullklinen passt nicht ins Schema?}
        \begin{center}
            \includegraphics[width=0.7\textwidth]{../images/nullklinen.pdf}
        \end{center}
    \end{frame}
    \begin{frame}
    \frametitle{Kugeloberfläche}
        \begin{center}
            \includegraphics[width=0.7\textwidth]{../images/kugel_animation_thumbnail.jpg}
        \end{center}
    \end{frame}
    \begin{frame}
    \frametitle{Voraussetzungen für Poincaré-Bendixson}
        \begin{itemize}
            \item Dynamisches System $\Phi_t(p) \in \Xi^r(\mathbb{S}^2)$
            \item Finite Anzahl Singularitäten
            \item Anfangspunkt $p \in \mathbb{S}^2$
        \end{itemize}
    \end{frame}
    %Idee: 2D schränkt mögliche Lösungen stark ein. Das kann formuliert werden mit Poinbendix.
    % Wir brauchen dass, da unser ENSO model sehr stark vereinfacht ist.

    \section{Anwendung und Relevanz}

    \begin{frame}
    \frametitle{Lösungen Recharge Oszillator bei unterschiedlichen Anfangswerten}
        \begin{center}
            %% Creator: Matplotlib, PGF backend
%%
%% To include the figure in your LaTeX document, write
%%   \input{<filename>.pgf}
%%
%% Make sure the required packages are loaded in your preamble
%%   \usepackage{pgf}
%%
%% Also ensure that all the required font packages are loaded; for instance,
%% the lmodern package is sometimes necessary when using math font.
%%   \usepackage{lmodern}
%%
%% Figures using additional raster images can only be included by \input if
%% they are in the same directory as the main LaTeX file. For loading figures
%% from other directories you can use the `import` package
%%   \usepackage{import}
%%
%% and then include the figures with
%%   \import{<path to file>}{<filename>.pgf}
%%
%% Matplotlib used the following preamble
%%   \usepackage{bm}
%%   \usepackage{amsmath}
%%   \usepackage{xcolor}
%%   \usepackage{tgtermes}
%%   \makeatletter\@ifpackageloaded{underscore}{}{\usepackage[strings]{underscore}}\makeatother
%%
\begingroup%
\makeatletter%
\begin{pgfpicture}%
\pgfpathrectangle{\pgfpointorigin}{\pgfqpoint{4.500000in}{2.500000in}}%
\pgfusepath{use as bounding box, clip}%
\begin{pgfscope}%
\pgfsetbuttcap%
\pgfsetmiterjoin%
\definecolor{currentfill}{rgb}{1.000000,1.000000,1.000000}%
\pgfsetfillcolor{currentfill}%
\pgfsetlinewidth{0.000000pt}%
\definecolor{currentstroke}{rgb}{1.000000,1.000000,1.000000}%
\pgfsetstrokecolor{currentstroke}%
\pgfsetdash{}{0pt}%
\pgfpathmoveto{\pgfqpoint{0.000000in}{0.000000in}}%
\pgfpathlineto{\pgfqpoint{4.500000in}{0.000000in}}%
\pgfpathlineto{\pgfqpoint{4.500000in}{2.500000in}}%
\pgfpathlineto{\pgfqpoint{0.000000in}{2.500000in}}%
\pgfpathlineto{\pgfqpoint{0.000000in}{0.000000in}}%
\pgfpathclose%
\pgfusepath{fill}%
\end{pgfscope}%
\begin{pgfscope}%
\pgfsetbuttcap%
\pgfsetmiterjoin%
\definecolor{currentfill}{rgb}{1.000000,1.000000,1.000000}%
\pgfsetfillcolor{currentfill}%
\pgfsetlinewidth{0.000000pt}%
\definecolor{currentstroke}{rgb}{0.000000,0.000000,0.000000}%
\pgfsetstrokecolor{currentstroke}%
\pgfsetstrokeopacity{0.000000}%
\pgfsetdash{}{0pt}%
\pgfpathmoveto{\pgfqpoint{0.562500in}{0.275000in}}%
\pgfpathlineto{\pgfqpoint{4.050000in}{0.275000in}}%
\pgfpathlineto{\pgfqpoint{4.050000in}{2.200000in}}%
\pgfpathlineto{\pgfqpoint{0.562500in}{2.200000in}}%
\pgfpathlineto{\pgfqpoint{0.562500in}{0.275000in}}%
\pgfpathclose%
\pgfusepath{fill}%
\end{pgfscope}%
\begin{pgfscope}%
\pgfpathrectangle{\pgfqpoint{0.562500in}{0.275000in}}{\pgfqpoint{3.487500in}{1.925000in}}%
\pgfusepath{clip}%
\pgfsetrectcap%
\pgfsetroundjoin%
\pgfsetlinewidth{0.803000pt}%
\definecolor{currentstroke}{rgb}{0.690196,0.690196,0.690196}%
\pgfsetstrokecolor{currentstroke}%
\pgfsetdash{}{0pt}%
\pgfpathmoveto{\pgfqpoint{0.721023in}{0.275000in}}%
\pgfpathlineto{\pgfqpoint{0.721023in}{2.200000in}}%
\pgfusepath{stroke}%
\end{pgfscope}%
\begin{pgfscope}%
\pgfsetbuttcap%
\pgfsetroundjoin%
\definecolor{currentfill}{rgb}{0.000000,0.000000,0.000000}%
\pgfsetfillcolor{currentfill}%
\pgfsetlinewidth{0.803000pt}%
\definecolor{currentstroke}{rgb}{0.000000,0.000000,0.000000}%
\pgfsetstrokecolor{currentstroke}%
\pgfsetdash{}{0pt}%
\pgfsys@defobject{currentmarker}{\pgfqpoint{0.000000in}{-0.048611in}}{\pgfqpoint{0.000000in}{0.000000in}}{%
\pgfpathmoveto{\pgfqpoint{0.000000in}{0.000000in}}%
\pgfpathlineto{\pgfqpoint{0.000000in}{-0.048611in}}%
\pgfusepath{stroke,fill}%
}%
\begin{pgfscope}%
\pgfsys@transformshift{0.721023in}{0.275000in}%
\pgfsys@useobject{currentmarker}{}%
\end{pgfscope}%
\end{pgfscope}%
\begin{pgfscope}%
\definecolor{textcolor}{rgb}{0.000000,0.000000,0.000000}%
\pgfsetstrokecolor{textcolor}%
\pgfsetfillcolor{textcolor}%
\pgftext[x=0.721023in,y=0.177778in,,top]{\color{textcolor}\rmfamily\fontsize{10.000000}{12.000000}\selectfont \(\displaystyle {-1.0}\)}%
\end{pgfscope}%
\begin{pgfscope}%
\pgfpathrectangle{\pgfqpoint{0.562500in}{0.275000in}}{\pgfqpoint{3.487500in}{1.925000in}}%
\pgfusepath{clip}%
\pgfsetrectcap%
\pgfsetroundjoin%
\pgfsetlinewidth{0.803000pt}%
\definecolor{currentstroke}{rgb}{0.690196,0.690196,0.690196}%
\pgfsetstrokecolor{currentstroke}%
\pgfsetdash{}{0pt}%
\pgfpathmoveto{\pgfqpoint{1.513636in}{0.275000in}}%
\pgfpathlineto{\pgfqpoint{1.513636in}{2.200000in}}%
\pgfusepath{stroke}%
\end{pgfscope}%
\begin{pgfscope}%
\pgfsetbuttcap%
\pgfsetroundjoin%
\definecolor{currentfill}{rgb}{0.000000,0.000000,0.000000}%
\pgfsetfillcolor{currentfill}%
\pgfsetlinewidth{0.803000pt}%
\definecolor{currentstroke}{rgb}{0.000000,0.000000,0.000000}%
\pgfsetstrokecolor{currentstroke}%
\pgfsetdash{}{0pt}%
\pgfsys@defobject{currentmarker}{\pgfqpoint{0.000000in}{-0.048611in}}{\pgfqpoint{0.000000in}{0.000000in}}{%
\pgfpathmoveto{\pgfqpoint{0.000000in}{0.000000in}}%
\pgfpathlineto{\pgfqpoint{0.000000in}{-0.048611in}}%
\pgfusepath{stroke,fill}%
}%
\begin{pgfscope}%
\pgfsys@transformshift{1.513636in}{0.275000in}%
\pgfsys@useobject{currentmarker}{}%
\end{pgfscope}%
\end{pgfscope}%
\begin{pgfscope}%
\definecolor{textcolor}{rgb}{0.000000,0.000000,0.000000}%
\pgfsetstrokecolor{textcolor}%
\pgfsetfillcolor{textcolor}%
\pgftext[x=1.513636in,y=0.177778in,,top]{\color{textcolor}\rmfamily\fontsize{10.000000}{12.000000}\selectfont \(\displaystyle {-0.5}\)}%
\end{pgfscope}%
\begin{pgfscope}%
\pgfpathrectangle{\pgfqpoint{0.562500in}{0.275000in}}{\pgfqpoint{3.487500in}{1.925000in}}%
\pgfusepath{clip}%
\pgfsetrectcap%
\pgfsetroundjoin%
\pgfsetlinewidth{0.803000pt}%
\definecolor{currentstroke}{rgb}{0.690196,0.690196,0.690196}%
\pgfsetstrokecolor{currentstroke}%
\pgfsetdash{}{0pt}%
\pgfpathmoveto{\pgfqpoint{2.306250in}{0.275000in}}%
\pgfpathlineto{\pgfqpoint{2.306250in}{2.200000in}}%
\pgfusepath{stroke}%
\end{pgfscope}%
\begin{pgfscope}%
\pgfsetbuttcap%
\pgfsetroundjoin%
\definecolor{currentfill}{rgb}{0.000000,0.000000,0.000000}%
\pgfsetfillcolor{currentfill}%
\pgfsetlinewidth{0.803000pt}%
\definecolor{currentstroke}{rgb}{0.000000,0.000000,0.000000}%
\pgfsetstrokecolor{currentstroke}%
\pgfsetdash{}{0pt}%
\pgfsys@defobject{currentmarker}{\pgfqpoint{0.000000in}{-0.048611in}}{\pgfqpoint{0.000000in}{0.000000in}}{%
\pgfpathmoveto{\pgfqpoint{0.000000in}{0.000000in}}%
\pgfpathlineto{\pgfqpoint{0.000000in}{-0.048611in}}%
\pgfusepath{stroke,fill}%
}%
\begin{pgfscope}%
\pgfsys@transformshift{2.306250in}{0.275000in}%
\pgfsys@useobject{currentmarker}{}%
\end{pgfscope}%
\end{pgfscope}%
\begin{pgfscope}%
\definecolor{textcolor}{rgb}{0.000000,0.000000,0.000000}%
\pgfsetstrokecolor{textcolor}%
\pgfsetfillcolor{textcolor}%
\pgftext[x=2.306250in,y=0.177778in,,top]{\color{textcolor}\rmfamily\fontsize{10.000000}{12.000000}\selectfont \(\displaystyle {0.0}\)}%
\end{pgfscope}%
\begin{pgfscope}%
\pgfpathrectangle{\pgfqpoint{0.562500in}{0.275000in}}{\pgfqpoint{3.487500in}{1.925000in}}%
\pgfusepath{clip}%
\pgfsetrectcap%
\pgfsetroundjoin%
\pgfsetlinewidth{0.803000pt}%
\definecolor{currentstroke}{rgb}{0.690196,0.690196,0.690196}%
\pgfsetstrokecolor{currentstroke}%
\pgfsetdash{}{0pt}%
\pgfpathmoveto{\pgfqpoint{3.098864in}{0.275000in}}%
\pgfpathlineto{\pgfqpoint{3.098864in}{2.200000in}}%
\pgfusepath{stroke}%
\end{pgfscope}%
\begin{pgfscope}%
\pgfsetbuttcap%
\pgfsetroundjoin%
\definecolor{currentfill}{rgb}{0.000000,0.000000,0.000000}%
\pgfsetfillcolor{currentfill}%
\pgfsetlinewidth{0.803000pt}%
\definecolor{currentstroke}{rgb}{0.000000,0.000000,0.000000}%
\pgfsetstrokecolor{currentstroke}%
\pgfsetdash{}{0pt}%
\pgfsys@defobject{currentmarker}{\pgfqpoint{0.000000in}{-0.048611in}}{\pgfqpoint{0.000000in}{0.000000in}}{%
\pgfpathmoveto{\pgfqpoint{0.000000in}{0.000000in}}%
\pgfpathlineto{\pgfqpoint{0.000000in}{-0.048611in}}%
\pgfusepath{stroke,fill}%
}%
\begin{pgfscope}%
\pgfsys@transformshift{3.098864in}{0.275000in}%
\pgfsys@useobject{currentmarker}{}%
\end{pgfscope}%
\end{pgfscope}%
\begin{pgfscope}%
\definecolor{textcolor}{rgb}{0.000000,0.000000,0.000000}%
\pgfsetstrokecolor{textcolor}%
\pgfsetfillcolor{textcolor}%
\pgftext[x=3.098864in,y=0.177778in,,top]{\color{textcolor}\rmfamily\fontsize{10.000000}{12.000000}\selectfont \(\displaystyle {0.5}\)}%
\end{pgfscope}%
\begin{pgfscope}%
\pgfpathrectangle{\pgfqpoint{0.562500in}{0.275000in}}{\pgfqpoint{3.487500in}{1.925000in}}%
\pgfusepath{clip}%
\pgfsetrectcap%
\pgfsetroundjoin%
\pgfsetlinewidth{0.803000pt}%
\definecolor{currentstroke}{rgb}{0.690196,0.690196,0.690196}%
\pgfsetstrokecolor{currentstroke}%
\pgfsetdash{}{0pt}%
\pgfpathmoveto{\pgfqpoint{3.891477in}{0.275000in}}%
\pgfpathlineto{\pgfqpoint{3.891477in}{2.200000in}}%
\pgfusepath{stroke}%
\end{pgfscope}%
\begin{pgfscope}%
\pgfsetbuttcap%
\pgfsetroundjoin%
\definecolor{currentfill}{rgb}{0.000000,0.000000,0.000000}%
\pgfsetfillcolor{currentfill}%
\pgfsetlinewidth{0.803000pt}%
\definecolor{currentstroke}{rgb}{0.000000,0.000000,0.000000}%
\pgfsetstrokecolor{currentstroke}%
\pgfsetdash{}{0pt}%
\pgfsys@defobject{currentmarker}{\pgfqpoint{0.000000in}{-0.048611in}}{\pgfqpoint{0.000000in}{0.000000in}}{%
\pgfpathmoveto{\pgfqpoint{0.000000in}{0.000000in}}%
\pgfpathlineto{\pgfqpoint{0.000000in}{-0.048611in}}%
\pgfusepath{stroke,fill}%
}%
\begin{pgfscope}%
\pgfsys@transformshift{3.891477in}{0.275000in}%
\pgfsys@useobject{currentmarker}{}%
\end{pgfscope}%
\end{pgfscope}%
\begin{pgfscope}%
\definecolor{textcolor}{rgb}{0.000000,0.000000,0.000000}%
\pgfsetstrokecolor{textcolor}%
\pgfsetfillcolor{textcolor}%
\pgftext[x=3.891477in,y=0.177778in,,top]{\color{textcolor}\rmfamily\fontsize{10.000000}{12.000000}\selectfont \(\displaystyle {1.0}\)}%
\end{pgfscope}%
\begin{pgfscope}%
\pgfpathrectangle{\pgfqpoint{0.562500in}{0.275000in}}{\pgfqpoint{3.487500in}{1.925000in}}%
\pgfusepath{clip}%
\pgfsetrectcap%
\pgfsetroundjoin%
\pgfsetlinewidth{0.803000pt}%
\definecolor{currentstroke}{rgb}{0.690196,0.690196,0.690196}%
\pgfsetstrokecolor{currentstroke}%
\pgfsetdash{}{0pt}%
\pgfpathmoveto{\pgfqpoint{0.562500in}{0.362500in}}%
\pgfpathlineto{\pgfqpoint{4.050000in}{0.362500in}}%
\pgfusepath{stroke}%
\end{pgfscope}%
\begin{pgfscope}%
\pgfsetbuttcap%
\pgfsetroundjoin%
\definecolor{currentfill}{rgb}{0.000000,0.000000,0.000000}%
\pgfsetfillcolor{currentfill}%
\pgfsetlinewidth{0.803000pt}%
\definecolor{currentstroke}{rgb}{0.000000,0.000000,0.000000}%
\pgfsetstrokecolor{currentstroke}%
\pgfsetdash{}{0pt}%
\pgfsys@defobject{currentmarker}{\pgfqpoint{-0.048611in}{0.000000in}}{\pgfqpoint{-0.000000in}{0.000000in}}{%
\pgfpathmoveto{\pgfqpoint{-0.000000in}{0.000000in}}%
\pgfpathlineto{\pgfqpoint{-0.048611in}{0.000000in}}%
\pgfusepath{stroke,fill}%
}%
\begin{pgfscope}%
\pgfsys@transformshift{0.562500in}{0.362500in}%
\pgfsys@useobject{currentmarker}{}%
\end{pgfscope}%
\end{pgfscope}%
\begin{pgfscope}%
\definecolor{textcolor}{rgb}{0.000000,0.000000,0.000000}%
\pgfsetstrokecolor{textcolor}%
\pgfsetfillcolor{textcolor}%
\pgftext[x=0.287808in, y=0.315799in, left, base]{\color{textcolor}\rmfamily\fontsize{10.000000}{12.000000}\selectfont \(\displaystyle {-1}\)}%
\end{pgfscope}%
\begin{pgfscope}%
\pgfpathrectangle{\pgfqpoint{0.562500in}{0.275000in}}{\pgfqpoint{3.487500in}{1.925000in}}%
\pgfusepath{clip}%
\pgfsetrectcap%
\pgfsetroundjoin%
\pgfsetlinewidth{0.803000pt}%
\definecolor{currentstroke}{rgb}{0.690196,0.690196,0.690196}%
\pgfsetstrokecolor{currentstroke}%
\pgfsetdash{}{0pt}%
\pgfpathmoveto{\pgfqpoint{0.562500in}{0.945833in}}%
\pgfpathlineto{\pgfqpoint{4.050000in}{0.945833in}}%
\pgfusepath{stroke}%
\end{pgfscope}%
\begin{pgfscope}%
\pgfsetbuttcap%
\pgfsetroundjoin%
\definecolor{currentfill}{rgb}{0.000000,0.000000,0.000000}%
\pgfsetfillcolor{currentfill}%
\pgfsetlinewidth{0.803000pt}%
\definecolor{currentstroke}{rgb}{0.000000,0.000000,0.000000}%
\pgfsetstrokecolor{currentstroke}%
\pgfsetdash{}{0pt}%
\pgfsys@defobject{currentmarker}{\pgfqpoint{-0.048611in}{0.000000in}}{\pgfqpoint{-0.000000in}{0.000000in}}{%
\pgfpathmoveto{\pgfqpoint{-0.000000in}{0.000000in}}%
\pgfpathlineto{\pgfqpoint{-0.048611in}{0.000000in}}%
\pgfusepath{stroke,fill}%
}%
\begin{pgfscope}%
\pgfsys@transformshift{0.562500in}{0.945833in}%
\pgfsys@useobject{currentmarker}{}%
\end{pgfscope}%
\end{pgfscope}%
\begin{pgfscope}%
\definecolor{textcolor}{rgb}{0.000000,0.000000,0.000000}%
\pgfsetstrokecolor{textcolor}%
\pgfsetfillcolor{textcolor}%
\pgftext[x=0.395833in, y=0.899132in, left, base]{\color{textcolor}\rmfamily\fontsize{10.000000}{12.000000}\selectfont \(\displaystyle {0}\)}%
\end{pgfscope}%
\begin{pgfscope}%
\pgfpathrectangle{\pgfqpoint{0.562500in}{0.275000in}}{\pgfqpoint{3.487500in}{1.925000in}}%
\pgfusepath{clip}%
\pgfsetrectcap%
\pgfsetroundjoin%
\pgfsetlinewidth{0.803000pt}%
\definecolor{currentstroke}{rgb}{0.690196,0.690196,0.690196}%
\pgfsetstrokecolor{currentstroke}%
\pgfsetdash{}{0pt}%
\pgfpathmoveto{\pgfqpoint{0.562500in}{1.529167in}}%
\pgfpathlineto{\pgfqpoint{4.050000in}{1.529167in}}%
\pgfusepath{stroke}%
\end{pgfscope}%
\begin{pgfscope}%
\pgfsetbuttcap%
\pgfsetroundjoin%
\definecolor{currentfill}{rgb}{0.000000,0.000000,0.000000}%
\pgfsetfillcolor{currentfill}%
\pgfsetlinewidth{0.803000pt}%
\definecolor{currentstroke}{rgb}{0.000000,0.000000,0.000000}%
\pgfsetstrokecolor{currentstroke}%
\pgfsetdash{}{0pt}%
\pgfsys@defobject{currentmarker}{\pgfqpoint{-0.048611in}{0.000000in}}{\pgfqpoint{-0.000000in}{0.000000in}}{%
\pgfpathmoveto{\pgfqpoint{-0.000000in}{0.000000in}}%
\pgfpathlineto{\pgfqpoint{-0.048611in}{0.000000in}}%
\pgfusepath{stroke,fill}%
}%
\begin{pgfscope}%
\pgfsys@transformshift{0.562500in}{1.529167in}%
\pgfsys@useobject{currentmarker}{}%
\end{pgfscope}%
\end{pgfscope}%
\begin{pgfscope}%
\definecolor{textcolor}{rgb}{0.000000,0.000000,0.000000}%
\pgfsetstrokecolor{textcolor}%
\pgfsetfillcolor{textcolor}%
\pgftext[x=0.395833in, y=1.482465in, left, base]{\color{textcolor}\rmfamily\fontsize{10.000000}{12.000000}\selectfont \(\displaystyle {1}\)}%
\end{pgfscope}%
\begin{pgfscope}%
\pgfpathrectangle{\pgfqpoint{0.562500in}{0.275000in}}{\pgfqpoint{3.487500in}{1.925000in}}%
\pgfusepath{clip}%
\pgfsetrectcap%
\pgfsetroundjoin%
\pgfsetlinewidth{0.803000pt}%
\definecolor{currentstroke}{rgb}{0.690196,0.690196,0.690196}%
\pgfsetstrokecolor{currentstroke}%
\pgfsetdash{}{0pt}%
\pgfpathmoveto{\pgfqpoint{0.562500in}{2.112500in}}%
\pgfpathlineto{\pgfqpoint{4.050000in}{2.112500in}}%
\pgfusepath{stroke}%
\end{pgfscope}%
\begin{pgfscope}%
\pgfsetbuttcap%
\pgfsetroundjoin%
\definecolor{currentfill}{rgb}{0.000000,0.000000,0.000000}%
\pgfsetfillcolor{currentfill}%
\pgfsetlinewidth{0.803000pt}%
\definecolor{currentstroke}{rgb}{0.000000,0.000000,0.000000}%
\pgfsetstrokecolor{currentstroke}%
\pgfsetdash{}{0pt}%
\pgfsys@defobject{currentmarker}{\pgfqpoint{-0.048611in}{0.000000in}}{\pgfqpoint{-0.000000in}{0.000000in}}{%
\pgfpathmoveto{\pgfqpoint{-0.000000in}{0.000000in}}%
\pgfpathlineto{\pgfqpoint{-0.048611in}{0.000000in}}%
\pgfusepath{stroke,fill}%
}%
\begin{pgfscope}%
\pgfsys@transformshift{0.562500in}{2.112500in}%
\pgfsys@useobject{currentmarker}{}%
\end{pgfscope}%
\end{pgfscope}%
\begin{pgfscope}%
\definecolor{textcolor}{rgb}{0.000000,0.000000,0.000000}%
\pgfsetstrokecolor{textcolor}%
\pgfsetfillcolor{textcolor}%
\pgftext[x=0.395833in, y=2.065799in, left, base]{\color{textcolor}\rmfamily\fontsize{10.000000}{12.000000}\selectfont \(\displaystyle {2}\)}%
\end{pgfscope}%
\begin{pgfscope}%
\pgfpathrectangle{\pgfqpoint{0.562500in}{0.275000in}}{\pgfqpoint{3.487500in}{1.925000in}}%
\pgfusepath{clip}%
\pgfsetrectcap%
\pgfsetroundjoin%
\pgfsetlinewidth{1.505625pt}%
\definecolor{currentstroke}{rgb}{0.121569,0.466667,0.705882}%
\pgfsetstrokecolor{currentstroke}%
\pgfsetdash{}{0pt}%
\pgfpathmoveto{\pgfqpoint{3.891477in}{0.362500in}}%
\pgfpathlineto{\pgfqpoint{3.781102in}{0.363499in}}%
\pgfpathlineto{\pgfqpoint{3.670167in}{0.366429in}}%
\pgfpathlineto{\pgfqpoint{3.556428in}{0.371235in}}%
\pgfpathlineto{\pgfqpoint{3.437981in}{0.377893in}}%
\pgfpathlineto{\pgfqpoint{3.248091in}{0.391364in}}%
\pgfpathlineto{\pgfqpoint{3.040065in}{0.409135in}}%
\pgfpathlineto{\pgfqpoint{2.807396in}{0.431529in}}%
\pgfpathlineto{\pgfqpoint{2.639192in}{0.449165in}}%
\pgfpathlineto{\pgfqpoint{2.467211in}{0.468913in}}%
\pgfpathlineto{\pgfqpoint{2.296518in}{0.490680in}}%
\pgfpathlineto{\pgfqpoint{2.131443in}{0.514334in}}%
\pgfpathlineto{\pgfqpoint{1.975578in}{0.539702in}}%
\pgfpathlineto{\pgfqpoint{1.831781in}{0.566572in}}%
\pgfpathlineto{\pgfqpoint{1.702173in}{0.594693in}}%
\pgfpathlineto{\pgfqpoint{1.643153in}{0.609133in}}%
\pgfpathlineto{\pgfqpoint{1.588141in}{0.623774in}}%
\pgfpathlineto{\pgfqpoint{1.537198in}{0.638572in}}%
\pgfpathlineto{\pgfqpoint{1.490335in}{0.653483in}}%
\pgfpathlineto{\pgfqpoint{1.447518in}{0.668459in}}%
\pgfpathlineto{\pgfqpoint{1.408668in}{0.683451in}}%
\pgfpathlineto{\pgfqpoint{1.373659in}{0.698405in}}%
\pgfpathlineto{\pgfqpoint{1.342320in}{0.713266in}}%
\pgfpathlineto{\pgfqpoint{1.314432in}{0.727978in}}%
\pgfpathlineto{\pgfqpoint{1.267332in}{0.756856in}}%
\pgfpathlineto{\pgfqpoint{1.228884in}{0.785080in}}%
\pgfpathlineto{\pgfqpoint{1.197353in}{0.812553in}}%
\pgfpathlineto{\pgfqpoint{1.171409in}{0.839199in}}%
\pgfpathlineto{\pgfqpoint{1.150034in}{0.864968in}}%
\pgfpathlineto{\pgfqpoint{1.132490in}{0.889818in}}%
\pgfpathlineto{\pgfqpoint{1.118157in}{0.913734in}}%
\pgfpathlineto{\pgfqpoint{1.106506in}{0.936726in}}%
\pgfpathlineto{\pgfqpoint{1.097174in}{0.958797in}}%
\pgfpathlineto{\pgfqpoint{1.089956in}{0.979949in}}%
\pgfpathlineto{\pgfqpoint{1.084811in}{1.000181in}}%
\pgfpathlineto{\pgfqpoint{1.081844in}{1.019488in}}%
\pgfpathlineto{\pgfqpoint{1.080817in}{1.037882in}}%
\pgfpathlineto{\pgfqpoint{1.081490in}{1.055381in}}%
\pgfpathlineto{\pgfqpoint{1.083783in}{1.072001in}}%
\pgfpathlineto{\pgfqpoint{1.087648in}{1.087757in}}%
\pgfpathlineto{\pgfqpoint{1.093071in}{1.102661in}}%
\pgfpathlineto{\pgfqpoint{1.100069in}{1.116727in}}%
\pgfpathlineto{\pgfqpoint{1.108697in}{1.129966in}}%
\pgfpathlineto{\pgfqpoint{1.119033in}{1.142392in}}%
\pgfpathlineto{\pgfqpoint{1.131013in}{1.154016in}}%
\pgfpathlineto{\pgfqpoint{1.144558in}{1.164847in}}%
\pgfpathlineto{\pgfqpoint{1.159624in}{1.174892in}}%
\pgfpathlineto{\pgfqpoint{1.185076in}{1.188505in}}%
\pgfpathlineto{\pgfqpoint{1.214085in}{1.200394in}}%
\pgfpathlineto{\pgfqpoint{1.246970in}{1.210586in}}%
\pgfpathlineto{\pgfqpoint{1.284239in}{1.219103in}}%
\pgfpathlineto{\pgfqpoint{1.326528in}{1.225966in}}%
\pgfpathlineto{\pgfqpoint{1.373849in}{1.231155in}}%
\pgfpathlineto{\pgfqpoint{1.426563in}{1.234641in}}%
\pgfpathlineto{\pgfqpoint{1.485264in}{1.236392in}}%
\pgfpathlineto{\pgfqpoint{1.550583in}{1.236364in}}%
\pgfpathlineto{\pgfqpoint{1.623183in}{1.234499in}}%
\pgfpathlineto{\pgfqpoint{1.732522in}{1.229035in}}%
\pgfpathlineto{\pgfqpoint{1.857800in}{1.219987in}}%
\pgfpathlineto{\pgfqpoint{2.000613in}{1.207112in}}%
\pgfpathlineto{\pgfqpoint{2.161438in}{1.190134in}}%
\pgfpathlineto{\pgfqpoint{2.338283in}{1.168849in}}%
\pgfpathlineto{\pgfqpoint{2.478457in}{1.150004in}}%
\pgfpathlineto{\pgfqpoint{2.620857in}{1.128752in}}%
\pgfpathlineto{\pgfqpoint{2.760582in}{1.105277in}}%
\pgfpathlineto{\pgfqpoint{2.849813in}{1.088533in}}%
\pgfpathlineto{\pgfqpoint{2.934207in}{1.071045in}}%
\pgfpathlineto{\pgfqpoint{3.011996in}{1.052942in}}%
\pgfpathlineto{\pgfqpoint{3.082424in}{1.034380in}}%
\pgfpathlineto{\pgfqpoint{3.145716in}{1.015520in}}%
\pgfpathlineto{\pgfqpoint{3.202116in}{0.996509in}}%
\pgfpathlineto{\pgfqpoint{3.251883in}{0.977484in}}%
\pgfpathlineto{\pgfqpoint{3.295293in}{0.958564in}}%
\pgfpathlineto{\pgfqpoint{3.332636in}{0.939858in}}%
\pgfpathlineto{\pgfqpoint{3.364222in}{0.921461in}}%
\pgfpathlineto{\pgfqpoint{3.390373in}{0.903452in}}%
\pgfpathlineto{\pgfqpoint{3.411429in}{0.885898in}}%
\pgfpathlineto{\pgfqpoint{3.427746in}{0.868854in}}%
\pgfpathlineto{\pgfqpoint{3.439696in}{0.852370in}}%
\pgfpathlineto{\pgfqpoint{3.447862in}{0.836513in}}%
\pgfpathlineto{\pgfqpoint{3.452983in}{0.821304in}}%
\pgfpathlineto{\pgfqpoint{3.455650in}{0.806756in}}%
\pgfpathlineto{\pgfqpoint{3.456296in}{0.792881in}}%
\pgfpathlineto{\pgfqpoint{3.455194in}{0.779688in}}%
\pgfpathlineto{\pgfqpoint{3.452456in}{0.767182in}}%
\pgfpathlineto{\pgfqpoint{3.448037in}{0.755366in}}%
\pgfpathlineto{\pgfqpoint{3.441730in}{0.744237in}}%
\pgfpathlineto{\pgfqpoint{3.433170in}{0.733794in}}%
\pgfpathlineto{\pgfqpoint{3.421832in}{0.724029in}}%
\pgfpathlineto{\pgfqpoint{3.407685in}{0.714949in}}%
\pgfpathlineto{\pgfqpoint{3.391620in}{0.706580in}}%
\pgfpathlineto{\pgfqpoint{3.363993in}{0.695361in}}%
\pgfpathlineto{\pgfqpoint{3.332070in}{0.685748in}}%
\pgfpathlineto{\pgfqpoint{3.295685in}{0.677746in}}%
\pgfpathlineto{\pgfqpoint{3.254560in}{0.671360in}}%
\pgfpathlineto{\pgfqpoint{3.208302in}{0.666601in}}%
\pgfpathlineto{\pgfqpoint{3.156481in}{0.663479in}}%
\pgfpathlineto{\pgfqpoint{3.098915in}{0.662025in}}%
\pgfpathlineto{\pgfqpoint{3.034702in}{0.662299in}}%
\pgfpathlineto{\pgfqpoint{2.962942in}{0.664374in}}%
\pgfpathlineto{\pgfqpoint{2.854275in}{0.670072in}}%
\pgfpathlineto{\pgfqpoint{2.729615in}{0.679308in}}%
\pgfpathlineto{\pgfqpoint{2.588192in}{0.692307in}}%
\pgfpathlineto{\pgfqpoint{2.429756in}{0.709314in}}%
\pgfpathlineto{\pgfqpoint{2.254336in}{0.730616in}}%
\pgfpathlineto{\pgfqpoint{2.115488in}{0.749455in}}%
\pgfpathlineto{\pgfqpoint{1.976050in}{0.770600in}}%
\pgfpathlineto{\pgfqpoint{1.840525in}{0.793840in}}%
\pgfpathlineto{\pgfqpoint{1.754305in}{0.810360in}}%
\pgfpathlineto{\pgfqpoint{1.672586in}{0.827588in}}%
\pgfpathlineto{\pgfqpoint{1.596268in}{0.845407in}}%
\pgfpathlineto{\pgfqpoint{1.526128in}{0.863689in}}%
\pgfpathlineto{\pgfqpoint{1.462830in}{0.882293in}}%
\pgfpathlineto{\pgfqpoint{1.406922in}{0.901061in}}%
\pgfpathlineto{\pgfqpoint{1.358614in}{0.919833in}}%
\pgfpathlineto{\pgfqpoint{1.317125in}{0.938490in}}%
\pgfpathlineto{\pgfqpoint{1.281756in}{0.956928in}}%
\pgfpathlineto{\pgfqpoint{1.251917in}{0.975055in}}%
\pgfpathlineto{\pgfqpoint{1.227111in}{0.992789in}}%
\pgfpathlineto{\pgfqpoint{1.206929in}{1.010060in}}%
\pgfpathlineto{\pgfqpoint{1.191050in}{1.026809in}}%
\pgfpathlineto{\pgfqpoint{1.179008in}{1.042988in}}%
\pgfpathlineto{\pgfqpoint{1.170290in}{1.058564in}}%
\pgfpathlineto{\pgfqpoint{1.164538in}{1.073511in}}%
\pgfpathlineto{\pgfqpoint{1.161477in}{1.087809in}}%
\pgfpathlineto{\pgfqpoint{1.160910in}{1.101440in}}%
\pgfpathlineto{\pgfqpoint{1.162721in}{1.114393in}}%
\pgfpathlineto{\pgfqpoint{1.166864in}{1.126662in}}%
\pgfpathlineto{\pgfqpoint{1.173227in}{1.138242in}}%
\pgfpathlineto{\pgfqpoint{1.181645in}{1.149128in}}%
\pgfpathlineto{\pgfqpoint{1.191996in}{1.159316in}}%
\pgfpathlineto{\pgfqpoint{1.204197in}{1.168803in}}%
\pgfpathlineto{\pgfqpoint{1.218205in}{1.177588in}}%
\pgfpathlineto{\pgfqpoint{1.234019in}{1.185671in}}%
\pgfpathlineto{\pgfqpoint{1.261223in}{1.196479in}}%
\pgfpathlineto{\pgfqpoint{1.292892in}{1.205717in}}%
\pgfpathlineto{\pgfqpoint{1.329492in}{1.213395in}}%
\pgfpathlineto{\pgfqpoint{1.371018in}{1.219491in}}%
\pgfpathlineto{\pgfqpoint{1.417719in}{1.223977in}}%
\pgfpathlineto{\pgfqpoint{1.470050in}{1.226820in}}%
\pgfpathlineto{\pgfqpoint{1.528527in}{1.227980in}}%
\pgfpathlineto{\pgfqpoint{1.593718in}{1.227406in}}%
\pgfpathlineto{\pgfqpoint{1.692179in}{1.223837in}}%
\pgfpathlineto{\pgfqpoint{1.805347in}{1.216905in}}%
\pgfpathlineto{\pgfqpoint{1.934873in}{1.206394in}}%
\pgfpathlineto{\pgfqpoint{2.081870in}{1.192034in}}%
\pgfpathlineto{\pgfqpoint{2.245760in}{1.173591in}}%
\pgfpathlineto{\pgfqpoint{2.423580in}{1.150901in}}%
\pgfpathlineto{\pgfqpoint{2.562004in}{1.131102in}}%
\pgfpathlineto{\pgfqpoint{2.700257in}{1.109046in}}%
\pgfpathlineto{\pgfqpoint{2.833548in}{1.084964in}}%
\pgfpathlineto{\pgfqpoint{2.917017in}{1.067925in}}%
\pgfpathlineto{\pgfqpoint{2.994846in}{1.050253in}}%
\pgfpathlineto{\pgfqpoint{3.066364in}{1.032111in}}%
\pgfpathlineto{\pgfqpoint{3.131105in}{1.013653in}}%
\pgfpathlineto{\pgfqpoint{3.188809in}{0.995019in}}%
\pgfpathlineto{\pgfqpoint{3.239423in}{0.976340in}}%
\pgfpathlineto{\pgfqpoint{3.283100in}{0.957733in}}%
\pgfpathlineto{\pgfqpoint{3.320200in}{0.939308in}}%
\pgfpathlineto{\pgfqpoint{3.351289in}{0.921158in}}%
\pgfpathlineto{\pgfqpoint{3.377138in}{0.903371in}}%
\pgfpathlineto{\pgfqpoint{3.398142in}{0.886032in}}%
\pgfpathlineto{\pgfqpoint{3.414644in}{0.869209in}}%
\pgfpathlineto{\pgfqpoint{3.427495in}{0.852935in}}%
\pgfpathlineto{\pgfqpoint{3.437345in}{0.837241in}}%
\pgfpathlineto{\pgfqpoint{3.444642in}{0.822152in}}%
\pgfpathlineto{\pgfqpoint{3.449631in}{0.807691in}}%
\pgfpathlineto{\pgfqpoint{3.452358in}{0.793874in}}%
\pgfpathlineto{\pgfqpoint{3.452662in}{0.780715in}}%
\pgfpathlineto{\pgfqpoint{3.450183in}{0.768222in}}%
\pgfpathlineto{\pgfqpoint{3.444528in}{0.756403in}}%
\pgfpathlineto{\pgfqpoint{3.436556in}{0.745279in}}%
\pgfpathlineto{\pgfqpoint{3.426602in}{0.734857in}}%
\pgfpathlineto{\pgfqpoint{3.414745in}{0.725137in}}%
\pgfpathlineto{\pgfqpoint{3.401034in}{0.716123in}}%
\pgfpathlineto{\pgfqpoint{3.385486in}{0.707815in}}%
\pgfpathlineto{\pgfqpoint{3.358673in}{0.696674in}}%
\pgfpathlineto{\pgfqpoint{3.327488in}{0.687116in}}%
\pgfpathlineto{\pgfqpoint{3.291582in}{0.679134in}}%
\pgfpathlineto{\pgfqpoint{3.250831in}{0.672738in}}%
\pgfpathlineto{\pgfqpoint{3.204988in}{0.667951in}}%
\pgfpathlineto{\pgfqpoint{3.153600in}{0.664804in}}%
\pgfpathlineto{\pgfqpoint{3.096158in}{0.663335in}}%
\pgfpathlineto{\pgfqpoint{3.032100in}{0.663594in}}%
\pgfpathlineto{\pgfqpoint{2.935321in}{0.666731in}}%
\pgfpathlineto{\pgfqpoint{2.824057in}{0.673212in}}%
\pgfpathlineto{\pgfqpoint{2.696618in}{0.683243in}}%
\pgfpathlineto{\pgfqpoint{2.551767in}{0.697092in}}%
\pgfpathlineto{\pgfqpoint{2.389780in}{0.715003in}}%
\pgfpathlineto{\pgfqpoint{2.213275in}{0.737151in}}%
\pgfpathlineto{\pgfqpoint{2.075076in}{0.756559in}}%
\pgfpathlineto{\pgfqpoint{1.936307in}{0.778256in}}%
\pgfpathlineto{\pgfqpoint{1.801626in}{0.802030in}}%
\pgfpathlineto{\pgfqpoint{1.716702in}{0.818888in}}%
\pgfpathlineto{\pgfqpoint{1.637261in}{0.836427in}}%
\pgfpathlineto{\pgfqpoint{1.564017in}{0.854480in}}%
\pgfpathlineto{\pgfqpoint{1.497465in}{0.872890in}}%
\pgfpathlineto{\pgfqpoint{1.437903in}{0.891511in}}%
\pgfpathlineto{\pgfqpoint{1.385428in}{0.910210in}}%
\pgfpathlineto{\pgfqpoint{1.339941in}{0.928863in}}%
\pgfpathlineto{\pgfqpoint{1.301142in}{0.947361in}}%
\pgfpathlineto{\pgfqpoint{1.268533in}{0.965605in}}%
\pgfpathlineto{\pgfqpoint{1.241418in}{0.983507in}}%
\pgfpathlineto{\pgfqpoint{1.219096in}{1.000986in}}%
\pgfpathlineto{\pgfqpoint{1.201485in}{1.017961in}}%
\pgfpathlineto{\pgfqpoint{1.187788in}{1.034393in}}%
\pgfpathlineto{\pgfqpoint{1.177281in}{1.050251in}}%
\pgfpathlineto{\pgfqpoint{1.169441in}{1.065508in}}%
\pgfpathlineto{\pgfqpoint{1.163943in}{1.080143in}}%
\pgfpathlineto{\pgfqpoint{1.160662in}{1.094136in}}%
\pgfpathlineto{\pgfqpoint{1.159675in}{1.107474in}}%
\pgfpathlineto{\pgfqpoint{1.161255in}{1.120148in}}%
\pgfpathlineto{\pgfqpoint{1.165878in}{1.132150in}}%
\pgfpathlineto{\pgfqpoint{1.173301in}{1.143468in}}%
\pgfpathlineto{\pgfqpoint{1.182742in}{1.154084in}}%
\pgfpathlineto{\pgfqpoint{1.194109in}{1.163997in}}%
\pgfpathlineto{\pgfqpoint{1.207345in}{1.173205in}}%
\pgfpathlineto{\pgfqpoint{1.222422in}{1.181708in}}%
\pgfpathlineto{\pgfqpoint{1.248507in}{1.193139in}}%
\pgfpathlineto{\pgfqpoint{1.278901in}{1.202984in}}%
\pgfpathlineto{\pgfqpoint{1.313915in}{1.211249in}}%
\pgfpathlineto{\pgfqpoint{1.353806in}{1.217933in}}%
\pgfpathlineto{\pgfqpoint{1.398704in}{1.223012in}}%
\pgfpathlineto{\pgfqpoint{1.449057in}{1.226459in}}%
\pgfpathlineto{\pgfqpoint{1.505374in}{1.228235in}}%
\pgfpathlineto{\pgfqpoint{1.568211in}{1.228293in}}%
\pgfpathlineto{\pgfqpoint{1.638174in}{1.226575in}}%
\pgfpathlineto{\pgfqpoint{1.743688in}{1.221403in}}%
\pgfpathlineto{\pgfqpoint{1.864737in}{1.212765in}}%
\pgfpathlineto{\pgfqpoint{2.002807in}{1.200417in}}%
\pgfpathlineto{\pgfqpoint{2.158401in}{1.184092in}}%
\pgfpathlineto{\pgfqpoint{2.329825in}{1.163586in}}%
\pgfpathlineto{\pgfqpoint{2.466153in}{1.145399in}}%
\pgfpathlineto{\pgfqpoint{2.605201in}{1.124854in}}%
\pgfpathlineto{\pgfqpoint{2.742593in}{1.102110in}}%
\pgfpathlineto{\pgfqpoint{2.873451in}{1.077432in}}%
\pgfpathlineto{\pgfqpoint{2.954474in}{1.060065in}}%
\pgfpathlineto{\pgfqpoint{3.029399in}{1.042134in}}%
\pgfpathlineto{\pgfqpoint{3.097727in}{1.023801in}}%
\pgfpathlineto{\pgfqpoint{3.159142in}{1.005217in}}%
\pgfpathlineto{\pgfqpoint{3.213512in}{0.986521in}}%
\pgfpathlineto{\pgfqpoint{3.260887in}{0.967841in}}%
\pgfpathlineto{\pgfqpoint{3.301502in}{0.949289in}}%
\pgfpathlineto{\pgfqpoint{3.335777in}{0.930968in}}%
\pgfpathlineto{\pgfqpoint{3.364314in}{0.912967in}}%
\pgfpathlineto{\pgfqpoint{3.387897in}{0.895365in}}%
\pgfpathlineto{\pgfqpoint{3.406809in}{0.878244in}}%
\pgfpathlineto{\pgfqpoint{3.421462in}{0.861661in}}%
\pgfpathlineto{\pgfqpoint{3.432663in}{0.845646in}}%
\pgfpathlineto{\pgfqpoint{3.441030in}{0.830227in}}%
\pgfpathlineto{\pgfqpoint{3.446982in}{0.815426in}}%
\pgfpathlineto{\pgfqpoint{3.450742in}{0.801262in}}%
\pgfpathlineto{\pgfqpoint{3.452341in}{0.787751in}}%
\pgfpathlineto{\pgfqpoint{3.451613in}{0.774902in}}%
\pgfpathlineto{\pgfqpoint{3.448196in}{0.762722in}}%
\pgfpathlineto{\pgfqpoint{3.441645in}{0.751216in}}%
\pgfpathlineto{\pgfqpoint{3.432748in}{0.740406in}}%
\pgfpathlineto{\pgfqpoint{3.421888in}{0.730297in}}%
\pgfpathlineto{\pgfqpoint{3.409136in}{0.720893in}}%
\pgfpathlineto{\pgfqpoint{3.394531in}{0.712194in}}%
\pgfpathlineto{\pgfqpoint{3.369162in}{0.700470in}}%
\pgfpathlineto{\pgfqpoint{3.339540in}{0.690334in}}%
\pgfpathlineto{\pgfqpoint{3.305414in}{0.681783in}}%
\pgfpathlineto{\pgfqpoint{3.266424in}{0.674812in}}%
\pgfpathlineto{\pgfqpoint{3.222470in}{0.669438in}}%
\pgfpathlineto{\pgfqpoint{3.173162in}{0.665691in}}%
\pgfpathlineto{\pgfqpoint{3.117983in}{0.663606in}}%
\pgfpathlineto{\pgfqpoint{3.056372in}{0.663228in}}%
\pgfpathlineto{\pgfqpoint{2.987732in}{0.664614in}}%
\pgfpathlineto{\pgfqpoint{2.884163in}{0.669325in}}%
\pgfpathlineto{\pgfqpoint{2.765325in}{0.677480in}}%
\pgfpathlineto{\pgfqpoint{2.629642in}{0.689305in}}%
\pgfpathlineto{\pgfqpoint{2.476384in}{0.705075in}}%
\pgfpathlineto{\pgfqpoint{2.306895in}{0.725004in}}%
\pgfpathlineto{\pgfqpoint{2.171520in}{0.742754in}}%
\pgfpathlineto{\pgfqpoint{2.032555in}{0.762881in}}%
\pgfpathlineto{\pgfqpoint{1.894545in}{0.785244in}}%
\pgfpathlineto{\pgfqpoint{1.762171in}{0.809595in}}%
\pgfpathlineto{\pgfqpoint{1.679384in}{0.826763in}}%
\pgfpathlineto{\pgfqpoint{1.602920in}{0.844536in}}%
\pgfpathlineto{\pgfqpoint{1.533570in}{0.862768in}}%
\pgfpathlineto{\pgfqpoint{1.471128in}{0.881300in}}%
\pgfpathlineto{\pgfqpoint{1.415371in}{0.899984in}}%
\pgfpathlineto{\pgfqpoint{1.366058in}{0.918687in}}%
\pgfpathlineto{\pgfqpoint{1.322933in}{0.937290in}}%
\pgfpathlineto{\pgfqpoint{1.285723in}{0.955687in}}%
\pgfpathlineto{\pgfqpoint{1.254139in}{0.973786in}}%
\pgfpathlineto{\pgfqpoint{1.227875in}{0.991508in}}%
\pgfpathlineto{\pgfqpoint{1.206610in}{1.008787in}}%
\pgfpathlineto{\pgfqpoint{1.190007in}{1.025572in}}%
\pgfpathlineto{\pgfqpoint{1.177711in}{1.041824in}}%
\pgfpathlineto{\pgfqpoint{1.169271in}{1.057475in}}%
\pgfpathlineto{\pgfqpoint{1.164021in}{1.072489in}}%
\pgfpathlineto{\pgfqpoint{1.161384in}{1.086851in}}%
\pgfpathlineto{\pgfqpoint{1.160921in}{1.100546in}}%
\pgfpathlineto{\pgfqpoint{1.162332in}{1.113564in}}%
\pgfpathlineto{\pgfqpoint{1.165459in}{1.125901in}}%
\pgfpathlineto{\pgfqpoint{1.170281in}{1.137550in}}%
\pgfpathlineto{\pgfqpoint{1.176919in}{1.148514in}}%
\pgfpathlineto{\pgfqpoint{1.185631in}{1.158794in}}%
\pgfpathlineto{\pgfqpoint{1.196817in}{1.168396in}}%
\pgfpathlineto{\pgfqpoint{1.210838in}{1.177326in}}%
\pgfpathlineto{\pgfqpoint{1.226950in}{1.185556in}}%
\pgfpathlineto{\pgfqpoint{1.254696in}{1.196578in}}%
\pgfpathlineto{\pgfqpoint{1.286792in}{1.206006in}}%
\pgfpathlineto{\pgfqpoint{1.323396in}{1.213833in}}%
\pgfpathlineto{\pgfqpoint{1.364778in}{1.220048in}}%
\pgfpathlineto{\pgfqpoint{1.411312in}{1.224640in}}%
\pgfpathlineto{\pgfqpoint{1.463478in}{1.227594in}}%
\pgfpathlineto{\pgfqpoint{1.521431in}{1.228889in}}%
\pgfpathlineto{\pgfqpoint{1.585942in}{1.228462in}}%
\pgfpathlineto{\pgfqpoint{1.683939in}{1.225079in}}%
\pgfpathlineto{\pgfqpoint{1.797279in}{1.218297in}}%
\pgfpathlineto{\pgfqpoint{1.927071in}{1.207909in}}%
\pgfpathlineto{\pgfqpoint{2.073687in}{1.193694in}}%
\pgfpathlineto{\pgfqpoint{2.236766in}{1.175420in}}%
\pgfpathlineto{\pgfqpoint{2.415218in}{1.152822in}}%
\pgfpathlineto{\pgfqpoint{2.554902in}{1.133014in}}%
\pgfpathlineto{\pgfqpoint{2.693598in}{1.110975in}}%
\pgfpathlineto{\pgfqpoint{2.826643in}{1.086968in}}%
\pgfpathlineto{\pgfqpoint{2.910258in}{1.070026in}}%
\pgfpathlineto{\pgfqpoint{2.988676in}{1.052458in}}%
\pgfpathlineto{\pgfqpoint{3.061105in}{1.034384in}}%
\pgfpathlineto{\pgfqpoint{3.126908in}{1.015933in}}%
\pgfpathlineto{\pgfqpoint{3.185606in}{0.997246in}}%
\pgfpathlineto{\pgfqpoint{3.236876in}{0.978476in}}%
\pgfpathlineto{\pgfqpoint{3.280750in}{0.959778in}}%
\pgfpathlineto{\pgfqpoint{3.318086in}{0.941270in}}%
\pgfpathlineto{\pgfqpoint{3.349720in}{0.923043in}}%
\pgfpathlineto{\pgfqpoint{3.376326in}{0.905175in}}%
\pgfpathlineto{\pgfqpoint{3.398414in}{0.887738in}}%
\pgfpathlineto{\pgfqpoint{3.416335in}{0.870793in}}%
\pgfpathlineto{\pgfqpoint{3.430279in}{0.854394in}}%
\pgfpathlineto{\pgfqpoint{3.440298in}{0.838583in}}%
\pgfpathlineto{\pgfqpoint{3.446966in}{0.823396in}}%
\pgfpathlineto{\pgfqpoint{3.450799in}{0.808856in}}%
\pgfpathlineto{\pgfqpoint{3.452082in}{0.794977in}}%
\pgfpathlineto{\pgfqpoint{3.451030in}{0.781773in}}%
\pgfpathlineto{\pgfqpoint{3.447785in}{0.769254in}}%
\pgfpathlineto{\pgfqpoint{3.442420in}{0.757425in}}%
\pgfpathlineto{\pgfqpoint{3.434938in}{0.746290in}}%
\pgfpathlineto{\pgfqpoint{3.425316in}{0.735847in}}%
\pgfpathlineto{\pgfqpoint{3.413686in}{0.726101in}}%
\pgfpathlineto{\pgfqpoint{3.400128in}{0.717052in}}%
\pgfpathlineto{\pgfqpoint{3.384693in}{0.708705in}}%
\pgfpathlineto{\pgfqpoint{3.358053in}{0.697502in}}%
\pgfpathlineto{\pgfqpoint{3.327167in}{0.687886in}}%
\pgfpathlineto{\pgfqpoint{3.291828in}{0.679859in}}%
\pgfpathlineto{\pgfqpoint{3.251686in}{0.673422in}}%
\pgfpathlineto{\pgfqpoint{3.206256in}{0.668576in}}%
\pgfpathlineto{\pgfqpoint{3.155399in}{0.665345in}}%
\pgfpathlineto{\pgfqpoint{3.098596in}{0.663780in}}%
\pgfpathlineto{\pgfqpoint{3.035115in}{0.663931in}}%
\pgfpathlineto{\pgfqpoint{2.964269in}{0.665861in}}%
\pgfpathlineto{\pgfqpoint{2.857262in}{0.671330in}}%
\pgfpathlineto{\pgfqpoint{2.734672in}{0.680295in}}%
\pgfpathlineto{\pgfqpoint{2.595287in}{0.692990in}}%
\pgfpathlineto{\pgfqpoint{2.437949in}{0.709656in}}%
\pgfpathlineto{\pgfqpoint{2.264693in}{0.730513in}}%
\pgfpathlineto{\pgfqpoint{2.128298in}{0.748988in}}%
\pgfpathlineto{\pgfqpoint{1.990122in}{0.769817in}}%
\pgfpathlineto{\pgfqpoint{1.854024in}{0.792821in}}%
\pgfpathlineto{\pgfqpoint{1.724307in}{0.817713in}}%
\pgfpathlineto{\pgfqpoint{1.643700in}{0.835167in}}%
\pgfpathlineto{\pgfqpoint{1.569544in}{0.853145in}}%
\pgfpathlineto{\pgfqpoint{1.502826in}{0.871502in}}%
\pgfpathlineto{\pgfqpoint{1.443337in}{0.890095in}}%
\pgfpathlineto{\pgfqpoint{1.390732in}{0.908787in}}%
\pgfpathlineto{\pgfqpoint{1.344663in}{0.927454in}}%
\pgfpathlineto{\pgfqpoint{1.304789in}{0.945980in}}%
\pgfpathlineto{\pgfqpoint{1.270767in}{0.964265in}}%
\pgfpathlineto{\pgfqpoint{1.242258in}{0.982216in}}%
\pgfpathlineto{\pgfqpoint{1.218923in}{0.999755in}}%
\pgfpathlineto{\pgfqpoint{1.200427in}{1.016813in}}%
\pgfpathlineto{\pgfqpoint{1.186314in}{1.033329in}}%
\pgfpathlineto{\pgfqpoint{1.175881in}{1.049257in}}%
\pgfpathlineto{\pgfqpoint{1.168541in}{1.064574in}}%
\pgfpathlineto{\pgfqpoint{1.163844in}{1.079256in}}%
\pgfpathlineto{\pgfqpoint{1.161479in}{1.093286in}}%
\pgfpathlineto{\pgfqpoint{1.161272in}{1.106650in}}%
\pgfpathlineto{\pgfqpoint{1.163192in}{1.119340in}}%
\pgfpathlineto{\pgfqpoint{1.167343in}{1.131349in}}%
\pgfpathlineto{\pgfqpoint{1.173969in}{1.142677in}}%
\pgfpathlineto{\pgfqpoint{1.183147in}{1.153319in}}%
\pgfpathlineto{\pgfqpoint{1.194338in}{1.163262in}}%
\pgfpathlineto{\pgfqpoint{1.207446in}{1.172505in}}%
\pgfpathlineto{\pgfqpoint{1.222427in}{1.181044in}}%
\pgfpathlineto{\pgfqpoint{1.248387in}{1.192533in}}%
\pgfpathlineto{\pgfqpoint{1.278601in}{1.202435in}}%
\pgfpathlineto{\pgfqpoint{1.313281in}{1.210748in}}%
\pgfpathlineto{\pgfqpoint{1.352781in}{1.217476in}}%
\pgfpathlineto{\pgfqpoint{1.397424in}{1.222612in}}%
\pgfpathlineto{\pgfqpoint{1.447396in}{1.226124in}}%
\pgfpathlineto{\pgfqpoint{1.503300in}{1.227974in}}%
\pgfpathlineto{\pgfqpoint{1.565763in}{1.228113in}}%
\pgfpathlineto{\pgfqpoint{1.635410in}{1.226483in}}%
\pgfpathlineto{\pgfqpoint{1.740532in}{1.221432in}}%
\pgfpathlineto{\pgfqpoint{1.861034in}{1.212911in}}%
\pgfpathlineto{\pgfqpoint{1.998403in}{1.200690in}}%
\pgfpathlineto{\pgfqpoint{2.153321in}{1.184521in}}%
\pgfpathlineto{\pgfqpoint{2.324249in}{1.164149in}}%
\pgfpathlineto{\pgfqpoint{2.460186in}{1.146064in}}%
\pgfpathlineto{\pgfqpoint{2.599157in}{1.125634in}}%
\pgfpathlineto{\pgfqpoint{2.736719in}{1.102991in}}%
\pgfpathlineto{\pgfqpoint{2.867820in}{1.078368in}}%
\pgfpathlineto{\pgfqpoint{2.949358in}{1.061066in}}%
\pgfpathlineto{\pgfqpoint{3.024991in}{1.043214in}}%
\pgfpathlineto{\pgfqpoint{3.093994in}{1.024947in}}%
\pgfpathlineto{\pgfqpoint{3.155922in}{1.006399in}}%
\pgfpathlineto{\pgfqpoint{3.210610in}{0.987703in}}%
\pgfpathlineto{\pgfqpoint{3.258176in}{0.968993in}}%
\pgfpathlineto{\pgfqpoint{3.298851in}{0.950408in}}%
\pgfpathlineto{\pgfqpoint{3.332914in}{0.932082in}}%
\pgfpathlineto{\pgfqpoint{3.361500in}{0.914084in}}%
\pgfpathlineto{\pgfqpoint{3.385569in}{0.896475in}}%
\pgfpathlineto{\pgfqpoint{3.405815in}{0.879309in}}%
\pgfpathlineto{\pgfqpoint{3.422667in}{0.862638in}}%
\pgfpathlineto{\pgfqpoint{3.436293in}{0.846507in}}%
\pgfpathlineto{\pgfqpoint{3.446591in}{0.830957in}}%
\pgfpathlineto{\pgfqpoint{3.453200in}{0.816024in}}%
\pgfpathlineto{\pgfqpoint{3.455728in}{0.801742in}}%
\pgfpathlineto{\pgfqpoint{3.455439in}{0.788138in}}%
\pgfpathlineto{\pgfqpoint{3.452841in}{0.775221in}}%
\pgfpathlineto{\pgfqpoint{3.448115in}{0.763000in}}%
\pgfpathlineto{\pgfqpoint{3.441392in}{0.751478in}}%
\pgfpathlineto{\pgfqpoint{3.432751in}{0.740659in}}%
\pgfpathlineto{\pgfqpoint{3.422216in}{0.730543in}}%
\pgfpathlineto{\pgfqpoint{3.409761in}{0.721130in}}%
\pgfpathlineto{\pgfqpoint{3.395307in}{0.712415in}}%
\pgfpathlineto{\pgfqpoint{3.378818in}{0.704398in}}%
\pgfpathlineto{\pgfqpoint{3.350447in}{0.693685in}}%
\pgfpathlineto{\pgfqpoint{3.317668in}{0.684557in}}%
\pgfpathlineto{\pgfqpoint{3.280329in}{0.677026in}}%
\pgfpathlineto{\pgfqpoint{3.238171in}{0.671106in}}%
\pgfpathlineto{\pgfqpoint{3.190832in}{0.666815in}}%
\pgfpathlineto{\pgfqpoint{3.137843in}{0.664176in}}%
\pgfpathlineto{\pgfqpoint{3.078630in}{0.663211in}}%
\pgfpathlineto{\pgfqpoint{3.012991in}{0.663936in}}%
\pgfpathlineto{\pgfqpoint{2.914010in}{0.667681in}}%
\pgfpathlineto{\pgfqpoint{2.799174in}{0.674873in}}%
\pgfpathlineto{\pgfqpoint{2.666957in}{0.685763in}}%
\pgfpathlineto{\pgfqpoint{2.517409in}{0.700558in}}%
\pgfpathlineto{\pgfqpoint{2.352163in}{0.719417in}}%
\pgfpathlineto{\pgfqpoint{2.174431in}{0.742451in}}%
\pgfpathlineto{\pgfqpoint{2.035847in}{0.762509in}}%
\pgfpathlineto{\pgfqpoint{1.897550in}{0.784870in}}%
\pgfpathlineto{\pgfqpoint{1.765093in}{0.809202in}}%
\pgfpathlineto{\pgfqpoint{1.682248in}{0.826329in}}%
\pgfpathlineto{\pgfqpoint{1.604990in}{0.844040in}}%
\pgfpathlineto{\pgfqpoint{1.534106in}{0.862210in}}%
\pgfpathlineto{\pgfqpoint{1.470155in}{0.880707in}}%
\pgfpathlineto{\pgfqpoint{1.413467in}{0.899397in}}%
\pgfpathlineto{\pgfqpoint{1.364144in}{0.918140in}}%
\pgfpathlineto{\pgfqpoint{1.322066in}{0.936780in}}%
\pgfpathlineto{\pgfqpoint{1.286446in}{0.955199in}}%
\pgfpathlineto{\pgfqpoint{1.256309in}{0.973317in}}%
\pgfpathlineto{\pgfqpoint{1.230892in}{0.991061in}}%
\pgfpathlineto{\pgfqpoint{1.209647in}{1.008367in}}%
\pgfpathlineto{\pgfqpoint{1.192241in}{1.025176in}}%
\pgfpathlineto{\pgfqpoint{1.178556in}{1.041440in}}%
\pgfpathlineto{\pgfqpoint{1.168687in}{1.057114in}}%
\pgfpathlineto{\pgfqpoint{1.162735in}{1.072162in}}%
\pgfpathlineto{\pgfqpoint{1.159745in}{1.086555in}}%
\pgfpathlineto{\pgfqpoint{1.159286in}{1.100279in}}%
\pgfpathlineto{\pgfqpoint{1.161120in}{1.113322in}}%
\pgfpathlineto{\pgfqpoint{1.165076in}{1.125676in}}%
\pgfpathlineto{\pgfqpoint{1.171047in}{1.137335in}}%
\pgfpathlineto{\pgfqpoint{1.178993in}{1.148298in}}%
\pgfpathlineto{\pgfqpoint{1.188939in}{1.158563in}}%
\pgfpathlineto{\pgfqpoint{1.200974in}{1.168132in}}%
\pgfpathlineto{\pgfqpoint{1.215043in}{1.177006in}}%
\pgfpathlineto{\pgfqpoint{1.231039in}{1.185182in}}%
\pgfpathlineto{\pgfqpoint{1.258612in}{1.196129in}}%
\pgfpathlineto{\pgfqpoint{1.290521in}{1.205490in}}%
\pgfpathlineto{\pgfqpoint{1.326921in}{1.213256in}}%
\pgfpathlineto{\pgfqpoint{1.368089in}{1.219417in}}%
\pgfpathlineto{\pgfqpoint{1.414422in}{1.223962in}}%
\pgfpathlineto{\pgfqpoint{1.466437in}{1.226880in}}%
\pgfpathlineto{\pgfqpoint{1.524315in}{1.228148in}}%
\pgfpathlineto{\pgfqpoint{1.588730in}{1.227700in}}%
\pgfpathlineto{\pgfqpoint{1.686569in}{1.224295in}}%
\pgfpathlineto{\pgfqpoint{1.799722in}{1.217500in}}%
\pgfpathlineto{\pgfqpoint{1.929299in}{1.207107in}}%
\pgfpathlineto{\pgfqpoint{2.075678in}{1.192894in}}%
\pgfpathlineto{\pgfqpoint{2.238494in}{1.174630in}}%
\pgfpathlineto{\pgfqpoint{2.416664in}{1.152049in}}%
\pgfpathlineto{\pgfqpoint{2.556052in}{1.132264in}}%
\pgfpathlineto{\pgfqpoint{2.694418in}{1.110256in}}%
\pgfpathlineto{\pgfqpoint{2.827142in}{1.086285in}}%
\pgfpathlineto{\pgfqpoint{2.910562in}{1.069369in}}%
\pgfpathlineto{\pgfqpoint{2.988808in}{1.051828in}}%
\pgfpathlineto{\pgfqpoint{3.061090in}{1.033782in}}%
\pgfpathlineto{\pgfqpoint{3.126772in}{1.015361in}}%
\pgfpathlineto{\pgfqpoint{3.185373in}{0.996705in}}%
\pgfpathlineto{\pgfqpoint{3.236565in}{0.977965in}}%
\pgfpathlineto{\pgfqpoint{3.280387in}{0.959298in}}%
\pgfpathlineto{\pgfqpoint{3.317694in}{0.940821in}}%
\pgfpathlineto{\pgfqpoint{3.349310in}{0.922622in}}%
\pgfpathlineto{\pgfqpoint{3.375902in}{0.904782in}}%
\pgfpathlineto{\pgfqpoint{3.397973in}{0.887372in}}%
\pgfpathlineto{\pgfqpoint{3.415873in}{0.870454in}}%
\pgfpathlineto{\pgfqpoint{3.429790in}{0.854080in}}%
\pgfpathlineto{\pgfqpoint{3.439789in}{0.838293in}}%
\pgfpathlineto{\pgfqpoint{3.446454in}{0.823130in}}%
\pgfpathlineto{\pgfqpoint{3.450287in}{0.808611in}}%
\pgfpathlineto{\pgfqpoint{3.451569in}{0.794754in}}%
\pgfpathlineto{\pgfqpoint{3.450514in}{0.781570in}}%
\pgfpathlineto{\pgfqpoint{3.447264in}{0.769070in}}%
\pgfpathlineto{\pgfqpoint{3.441889in}{0.757260in}}%
\pgfpathlineto{\pgfqpoint{3.434391in}{0.746142in}}%
\pgfpathlineto{\pgfqpoint{3.424754in}{0.735716in}}%
\pgfpathlineto{\pgfqpoint{3.413110in}{0.725986in}}%
\pgfpathlineto{\pgfqpoint{3.399540in}{0.716954in}}%
\pgfpathlineto{\pgfqpoint{3.384091in}{0.708622in}}%
\pgfpathlineto{\pgfqpoint{3.357430in}{0.697442in}}%
\pgfpathlineto{\pgfqpoint{3.326520in}{0.687847in}}%
\pgfpathlineto{\pgfqpoint{3.291151in}{0.679841in}}%
\pgfpathlineto{\pgfqpoint{3.250970in}{0.673424in}}%
\pgfpathlineto{\pgfqpoint{3.205495in}{0.668596in}}%
\pgfpathlineto{\pgfqpoint{3.154593in}{0.665385in}}%
\pgfpathlineto{\pgfqpoint{3.097734in}{0.663839in}}%
\pgfpathlineto{\pgfqpoint{3.034188in}{0.664010in}}%
\pgfpathlineto{\pgfqpoint{2.937888in}{0.667018in}}%
\pgfpathlineto{\pgfqpoint{2.827011in}{0.673368in}}%
\pgfpathlineto{\pgfqpoint{2.700227in}{0.683270in}}%
\pgfpathlineto{\pgfqpoint{2.556357in}{0.696968in}}%
\pgfpathlineto{\pgfqpoint{2.394474in}{0.714687in}}%
\pgfpathlineto{\pgfqpoint{2.218223in}{0.736654in}}%
\pgfpathlineto{\pgfqpoint{2.080833in}{0.755950in}}%
\pgfpathlineto{\pgfqpoint{1.942895in}{0.777550in}}%
\pgfpathlineto{\pgfqpoint{1.808445in}{0.801236in}}%
\pgfpathlineto{\pgfqpoint{1.722957in}{0.818031in}}%
\pgfpathlineto{\pgfqpoint{1.642469in}{0.835490in}}%
\pgfpathlineto{\pgfqpoint{1.568518in}{0.853470in}}%
\pgfpathlineto{\pgfqpoint{1.501987in}{0.871828in}}%
\pgfpathlineto{\pgfqpoint{1.442636in}{0.890420in}}%
\pgfpathlineto{\pgfqpoint{1.390127in}{0.909109in}}%
\pgfpathlineto{\pgfqpoint{1.344125in}{0.927771in}}%
\pgfpathlineto{\pgfqpoint{1.304293in}{0.946291in}}%
\pgfpathlineto{\pgfqpoint{1.270298in}{0.964568in}}%
\pgfpathlineto{\pgfqpoint{1.241807in}{0.982512in}}%
\pgfpathlineto{\pgfqpoint{1.218489in}{1.000042in}}%
\pgfpathlineto{\pgfqpoint{1.200013in}{1.017094in}}%
\pgfpathlineto{\pgfqpoint{1.185959in}{1.033603in}}%
\pgfpathlineto{\pgfqpoint{1.175628in}{1.049523in}}%
\pgfpathlineto{\pgfqpoint{1.168404in}{1.064830in}}%
\pgfpathlineto{\pgfqpoint{1.163814in}{1.079501in}}%
\pgfpathlineto{\pgfqpoint{1.161526in}{1.093520in}}%
\pgfpathlineto{\pgfqpoint{1.161351in}{1.106874in}}%
\pgfpathlineto{\pgfqpoint{1.163240in}{1.119553in}}%
\pgfpathlineto{\pgfqpoint{1.167287in}{1.131551in}}%
\pgfpathlineto{\pgfqpoint{1.173726in}{1.142868in}}%
\pgfpathlineto{\pgfqpoint{1.182825in}{1.153503in}}%
\pgfpathlineto{\pgfqpoint{1.194055in}{1.163443in}}%
\pgfpathlineto{\pgfqpoint{1.207212in}{1.172680in}}%
\pgfpathlineto{\pgfqpoint{1.222250in}{1.181215in}}%
\pgfpathlineto{\pgfqpoint{1.248307in}{1.192696in}}%
\pgfpathlineto{\pgfqpoint{1.278622in}{1.202589in}}%
\pgfpathlineto{\pgfqpoint{1.313389in}{1.210891in}}%
\pgfpathlineto{\pgfqpoint{1.352941in}{1.217603in}}%
\pgfpathlineto{\pgfqpoint{1.397658in}{1.222723in}}%
\pgfpathlineto{\pgfqpoint{1.447690in}{1.226220in}}%
\pgfpathlineto{\pgfqpoint{1.503639in}{1.228054in}}%
\pgfpathlineto{\pgfqpoint{1.566163in}{1.228178in}}%
\pgfpathlineto{\pgfqpoint{1.635905in}{1.226530in}}%
\pgfpathlineto{\pgfqpoint{1.741206in}{1.221454in}}%
\pgfpathlineto{\pgfqpoint{1.861904in}{1.212906in}}%
\pgfpathlineto{\pgfqpoint{1.999403in}{1.200652in}}%
\pgfpathlineto{\pgfqpoint{2.154614in}{1.184457in}}%
\pgfpathlineto{\pgfqpoint{2.325742in}{1.164059in}}%
\pgfpathlineto{\pgfqpoint{2.461645in}{1.145935in}}%
\pgfpathlineto{\pgfqpoint{2.600509in}{1.125458in}}%
\pgfpathlineto{\pgfqpoint{2.738106in}{1.102787in}}%
\pgfpathlineto{\pgfqpoint{2.826606in}{1.086576in}}%
\pgfpathlineto{\pgfqpoint{2.910605in}{1.069587in}}%
\pgfpathlineto{\pgfqpoint{2.989024in}{1.051953in}}%
\pgfpathlineto{\pgfqpoint{3.061144in}{1.033836in}}%
\pgfpathlineto{\pgfqpoint{3.126468in}{1.015387in}}%
\pgfpathlineto{\pgfqpoint{3.184718in}{0.996748in}}%
\pgfpathlineto{\pgfqpoint{3.235835in}{0.978049in}}%
\pgfpathlineto{\pgfqpoint{3.279983in}{0.959411in}}%
\pgfpathlineto{\pgfqpoint{3.317546in}{0.940944in}}%
\pgfpathlineto{\pgfqpoint{3.349128in}{0.922747in}}%
\pgfpathlineto{\pgfqpoint{3.375421in}{0.904913in}}%
\pgfpathlineto{\pgfqpoint{3.396559in}{0.887538in}}%
\pgfpathlineto{\pgfqpoint{3.413391in}{0.870666in}}%
\pgfpathlineto{\pgfqpoint{3.426732in}{0.854336in}}%
\pgfpathlineto{\pgfqpoint{3.437175in}{0.838577in}}%
\pgfpathlineto{\pgfqpoint{3.445092in}{0.823419in}}%
\pgfpathlineto{\pgfqpoint{3.450634in}{0.808883in}}%
\pgfpathlineto{\pgfqpoint{3.453733in}{0.794989in}}%
\pgfpathlineto{\pgfqpoint{3.454098in}{0.781752in}}%
\pgfpathlineto{\pgfqpoint{3.451220in}{0.769181in}}%
\pgfpathlineto{\pgfqpoint{3.445441in}{0.757295in}}%
\pgfpathlineto{\pgfqpoint{3.437592in}{0.746109in}}%
\pgfpathlineto{\pgfqpoint{3.427783in}{0.735625in}}%
\pgfpathlineto{\pgfqpoint{3.416093in}{0.725847in}}%
\pgfpathlineto{\pgfqpoint{3.402563in}{0.716774in}}%
\pgfpathlineto{\pgfqpoint{3.387198in}{0.708407in}}%
\pgfpathlineto{\pgfqpoint{3.360637in}{0.697178in}}%
\pgfpathlineto{\pgfqpoint{3.329628in}{0.687527in}}%
\pgfpathlineto{\pgfqpoint{3.293882in}{0.679450in}}%
\pgfpathlineto{\pgfqpoint{3.253369in}{0.672963in}}%
\pgfpathlineto{\pgfqpoint{3.207763in}{0.668088in}}%
\pgfpathlineto{\pgfqpoint{3.156643in}{0.664854in}}%
\pgfpathlineto{\pgfqpoint{3.099521in}{0.663300in}}%
\pgfpathlineto{\pgfqpoint{3.035840in}{0.663472in}}%
\pgfpathlineto{\pgfqpoint{2.939635in}{0.666484in}}%
\pgfpathlineto{\pgfqpoint{2.828976in}{0.672822in}}%
\pgfpathlineto{\pgfqpoint{2.702199in}{0.682705in}}%
\pgfpathlineto{\pgfqpoint{2.557982in}{0.696400in}}%
\pgfpathlineto{\pgfqpoint{2.396642in}{0.714146in}}%
\pgfpathlineto{\pgfqpoint{2.220541in}{0.736128in}}%
\pgfpathlineto{\pgfqpoint{2.082456in}{0.755414in}}%
\pgfpathlineto{\pgfqpoint{1.943658in}{0.776998in}}%
\pgfpathlineto{\pgfqpoint{1.808659in}{0.800671in}}%
\pgfpathlineto{\pgfqpoint{1.723202in}{0.817462in}}%
\pgfpathlineto{\pgfqpoint{1.643243in}{0.834936in}}%
\pgfpathlineto{\pgfqpoint{1.569712in}{0.852955in}}%
\pgfpathlineto{\pgfqpoint{1.502904in}{0.871353in}}%
\pgfpathlineto{\pgfqpoint{1.442975in}{0.889976in}}%
\pgfpathlineto{\pgfqpoint{1.389954in}{0.908686in}}%
\pgfpathlineto{\pgfqpoint{1.343736in}{0.927355in}}%
\pgfpathlineto{\pgfqpoint{1.304086in}{0.945871in}}%
\pgfpathlineto{\pgfqpoint{1.270636in}{0.964134in}}%
\pgfpathlineto{\pgfqpoint{1.242888in}{0.982058in}}%
\pgfpathlineto{\pgfqpoint{1.220213in}{0.999571in}}%
\pgfpathlineto{\pgfqpoint{1.201922in}{1.016609in}}%
\pgfpathlineto{\pgfqpoint{1.187927in}{1.033101in}}%
\pgfpathlineto{\pgfqpoint{1.177586in}{1.049010in}}%
\pgfpathlineto{\pgfqpoint{1.170208in}{1.064311in}}%
\pgfpathlineto{\pgfqpoint{1.165283in}{1.078984in}}%
\pgfpathlineto{\pgfqpoint{1.162473in}{1.093012in}}%
\pgfpathlineto{\pgfqpoint{1.161623in}{1.106383in}}%
\pgfpathlineto{\pgfqpoint{1.162751in}{1.119088in}}%
\pgfpathlineto{\pgfqpoint{1.166056in}{1.131120in}}%
\pgfpathlineto{\pgfqpoint{1.171911in}{1.142479in}}%
\pgfpathlineto{\pgfqpoint{1.180795in}{1.153165in}}%
\pgfpathlineto{\pgfqpoint{1.192010in}{1.163156in}}%
\pgfpathlineto{\pgfqpoint{1.205148in}{1.172444in}}%
\pgfpathlineto{\pgfqpoint{1.220167in}{1.181027in}}%
\pgfpathlineto{\pgfqpoint{1.246192in}{1.192577in}}%
\pgfpathlineto{\pgfqpoint{1.276473in}{1.202535in}}%
\pgfpathlineto{\pgfqpoint{1.311200in}{1.210901in}}%
\pgfpathlineto{\pgfqpoint{1.350695in}{1.217673in}}%
\pgfpathlineto{\pgfqpoint{1.395324in}{1.222848in}}%
\pgfpathlineto{\pgfqpoint{1.445251in}{1.226399in}}%
\pgfpathlineto{\pgfqpoint{1.501082in}{1.228288in}}%
\pgfpathlineto{\pgfqpoint{1.563469in}{1.228464in}}%
\pgfpathlineto{\pgfqpoint{1.633056in}{1.226869in}}%
\pgfpathlineto{\pgfqpoint{1.738122in}{1.221865in}}%
\pgfpathlineto{\pgfqpoint{1.858562in}{1.213390in}}%
\pgfpathlineto{\pgfqpoint{1.995796in}{1.201212in}}%
\pgfpathlineto{\pgfqpoint{2.150737in}{1.185093in}}%
\pgfpathlineto{\pgfqpoint{2.321684in}{1.164771in}}%
\pgfpathlineto{\pgfqpoint{2.457577in}{1.146704in}}%
\pgfpathlineto{\pgfqpoint{2.596550in}{1.126281in}}%
\pgfpathlineto{\pgfqpoint{2.734360in}{1.103656in}}%
\pgfpathlineto{\pgfqpoint{2.823034in}{1.087468in}}%
\pgfpathlineto{\pgfqpoint{2.907277in}{1.070493in}}%
\pgfpathlineto{\pgfqpoint{2.986020in}{1.052872in}}%
\pgfpathlineto{\pgfqpoint{3.058493in}{1.034765in}}%
\pgfpathlineto{\pgfqpoint{3.124164in}{1.016320in}}%
\pgfpathlineto{\pgfqpoint{3.182731in}{0.997677in}}%
\pgfpathlineto{\pgfqpoint{3.234130in}{0.978968in}}%
\pgfpathlineto{\pgfqpoint{3.278529in}{0.960311in}}%
\pgfpathlineto{\pgfqpoint{3.316332in}{0.941820in}}%
\pgfpathlineto{\pgfqpoint{3.348175in}{0.923596in}}%
\pgfpathlineto{\pgfqpoint{3.374571in}{0.905741in}}%
\pgfpathlineto{\pgfqpoint{3.395822in}{0.888342in}}%
\pgfpathlineto{\pgfqpoint{3.412898in}{0.871440in}}%
\pgfpathlineto{\pgfqpoint{3.426570in}{0.855075in}}%
\pgfpathlineto{\pgfqpoint{3.437379in}{0.839277in}}%
\pgfpathlineto{\pgfqpoint{3.445636in}{0.824077in}}%
\pgfpathlineto{\pgfqpoint{3.451423in}{0.809499in}}%
\pgfpathlineto{\pgfqpoint{3.454592in}{0.795562in}}%
\pgfpathlineto{\pgfqpoint{3.454764in}{0.782283in}}%
\pgfpathlineto{\pgfqpoint{3.451503in}{0.769673in}}%
\pgfpathlineto{\pgfqpoint{3.445780in}{0.757757in}}%
\pgfpathlineto{\pgfqpoint{3.438006in}{0.746541in}}%
\pgfpathlineto{\pgfqpoint{3.428290in}{0.736029in}}%
\pgfpathlineto{\pgfqpoint{3.416704in}{0.726221in}}%
\pgfpathlineto{\pgfqpoint{3.403281in}{0.717120in}}%
\pgfpathlineto{\pgfqpoint{3.388021in}{0.708724in}}%
\pgfpathlineto{\pgfqpoint{3.361584in}{0.697449in}}%
\pgfpathlineto{\pgfqpoint{3.330621in}{0.687749in}}%
\pgfpathlineto{\pgfqpoint{3.294981in}{0.679625in}}%
\pgfpathlineto{\pgfqpoint{3.254586in}{0.673092in}}%
\pgfpathlineto{\pgfqpoint{3.209105in}{0.668174in}}%
\pgfpathlineto{\pgfqpoint{3.158134in}{0.664898in}}%
\pgfpathlineto{\pgfqpoint{3.101189in}{0.663302in}}%
\pgfpathlineto{\pgfqpoint{3.037711in}{0.663431in}}%
\pgfpathlineto{\pgfqpoint{2.941804in}{0.666378in}}%
\pgfpathlineto{\pgfqpoint{2.831485in}{0.672638in}}%
\pgfpathlineto{\pgfqpoint{2.705037in}{0.682445in}}%
\pgfpathlineto{\pgfqpoint{2.561121in}{0.696063in}}%
\pgfpathlineto{\pgfqpoint{2.400119in}{0.713725in}}%
\pgfpathlineto{\pgfqpoint{2.224252in}{0.735620in}}%
\pgfpathlineto{\pgfqpoint{2.086105in}{0.754847in}}%
\pgfpathlineto{\pgfqpoint{1.947331in}{0.776372in}}%
\pgfpathlineto{\pgfqpoint{1.812381in}{0.799990in}}%
\pgfpathlineto{\pgfqpoint{1.726764in}{0.816748in}}%
\pgfpathlineto{\pgfqpoint{1.646203in}{0.834183in}}%
\pgfpathlineto{\pgfqpoint{1.572043in}{0.852158in}}%
\pgfpathlineto{\pgfqpoint{1.504965in}{0.870528in}}%
\pgfpathlineto{\pgfqpoint{1.444940in}{0.889141in}}%
\pgfpathlineto{\pgfqpoint{1.391840in}{0.907858in}}%
\pgfpathlineto{\pgfqpoint{1.345461in}{0.926549in}}%
\pgfpathlineto{\pgfqpoint{1.305525in}{0.945100in}}%
\pgfpathlineto{\pgfqpoint{1.271676in}{0.963407in}}%
\pgfpathlineto{\pgfqpoint{1.243487in}{0.981377in}}%
\pgfpathlineto{\pgfqpoint{1.220450in}{0.998933in}}%
\pgfpathlineto{\pgfqpoint{1.202081in}{1.016000in}}%
\pgfpathlineto{\pgfqpoint{1.187848in}{1.032524in}}%
\pgfpathlineto{\pgfqpoint{1.177085in}{1.048470in}}%
\pgfpathlineto{\pgfqpoint{1.169275in}{1.063811in}}%
\pgfpathlineto{\pgfqpoint{1.164054in}{1.078523in}}%
\pgfpathlineto{\pgfqpoint{1.161213in}{1.092586in}}%
\pgfpathlineto{\pgfqpoint{1.160697in}{1.105987in}}%
\pgfpathlineto{\pgfqpoint{1.162606in}{1.118716in}}%
\pgfpathlineto{\pgfqpoint{1.167191in}{1.130767in}}%
\pgfpathlineto{\pgfqpoint{1.174283in}{1.142132in}}%
\pgfpathlineto{\pgfqpoint{1.183414in}{1.152799in}}%
\pgfpathlineto{\pgfqpoint{1.194485in}{1.162766in}}%
\pgfpathlineto{\pgfqpoint{1.207431in}{1.172031in}}%
\pgfpathlineto{\pgfqpoint{1.222217in}{1.180593in}}%
\pgfpathlineto{\pgfqpoint{1.247860in}{1.192116in}}%
\pgfpathlineto{\pgfqpoint{1.277795in}{1.202057in}}%
\pgfpathlineto{\pgfqpoint{1.312333in}{1.210422in}}%
\pgfpathlineto{\pgfqpoint{1.351784in}{1.217213in}}%
\pgfpathlineto{\pgfqpoint{1.396221in}{1.222403in}}%
\pgfpathlineto{\pgfqpoint{1.446080in}{1.225966in}}%
\pgfpathlineto{\pgfqpoint{1.501863in}{1.227864in}}%
\pgfpathlineto{\pgfqpoint{1.564122in}{1.228051in}}%
\pgfpathlineto{\pgfqpoint{1.633460in}{1.226468in}}%
\pgfpathlineto{\pgfqpoint{1.738053in}{1.221489in}}%
\pgfpathlineto{\pgfqpoint{1.858072in}{1.213057in}}%
\pgfpathlineto{\pgfqpoint{1.995026in}{1.200934in}}%
\pgfpathlineto{\pgfqpoint{2.149510in}{1.184853in}}%
\pgfpathlineto{\pgfqpoint{2.319967in}{1.164603in}}%
\pgfpathlineto{\pgfqpoint{2.455803in}{1.146612in}}%
\pgfpathlineto{\pgfqpoint{2.594675in}{1.126258in}}%
\pgfpathlineto{\pgfqpoint{2.732244in}{1.103690in}}%
\pgfpathlineto{\pgfqpoint{2.863695in}{1.079167in}}%
\pgfpathlineto{\pgfqpoint{2.945335in}{1.061891in}}%
\pgfpathlineto{\pgfqpoint{3.020941in}{1.044029in}}%
\pgfpathlineto{\pgfqpoint{3.090005in}{1.025746in}}%
\pgfpathlineto{\pgfqpoint{3.152193in}{1.007196in}}%
\pgfpathlineto{\pgfqpoint{3.207357in}{0.988518in}}%
\pgfpathlineto{\pgfqpoint{3.255527in}{0.969841in}}%
\pgfpathlineto{\pgfqpoint{3.296914in}{0.951280in}}%
\pgfpathlineto{\pgfqpoint{3.331908in}{0.932938in}}%
\pgfpathlineto{\pgfqpoint{3.361082in}{0.914908in}}%
\pgfpathlineto{\pgfqpoint{3.385189in}{0.897266in}}%
\pgfpathlineto{\pgfqpoint{3.404678in}{0.880094in}}%
\pgfpathlineto{\pgfqpoint{3.419784in}{0.863455in}}%
\pgfpathlineto{\pgfqpoint{3.431337in}{0.847381in}}%
\pgfpathlineto{\pgfqpoint{3.439981in}{0.831899in}}%
\pgfpathlineto{\pgfqpoint{3.446165in}{0.817034in}}%
\pgfpathlineto{\pgfqpoint{3.450145in}{0.802804in}}%
\pgfpathlineto{\pgfqpoint{3.451984in}{0.789224in}}%
\pgfpathlineto{\pgfqpoint{3.451550in}{0.776305in}}%
\pgfpathlineto{\pgfqpoint{3.448517in}{0.764055in}}%
\pgfpathlineto{\pgfqpoint{3.442390in}{0.752477in}}%
\pgfpathlineto{\pgfqpoint{3.433729in}{0.741590in}}%
\pgfpathlineto{\pgfqpoint{3.423091in}{0.731404in}}%
\pgfpathlineto{\pgfqpoint{3.410552in}{0.721923in}}%
\pgfpathlineto{\pgfqpoint{3.396154in}{0.713146in}}%
\pgfpathlineto{\pgfqpoint{3.379911in}{0.705075in}}%
\pgfpathlineto{\pgfqpoint{3.352045in}{0.694293in}}%
\pgfpathlineto{\pgfqpoint{3.319801in}{0.685097in}}%
\pgfpathlineto{\pgfqpoint{3.282844in}{0.677483in}}%
\pgfpathlineto{\pgfqpoint{3.240936in}{0.671456in}}%
\pgfpathlineto{\pgfqpoint{3.193881in}{0.667043in}}%
\pgfpathlineto{\pgfqpoint{3.141172in}{0.664275in}}%
\pgfpathlineto{\pgfqpoint{3.082263in}{0.663194in}}%
\pgfpathlineto{\pgfqpoint{3.016568in}{0.663853in}}%
\pgfpathlineto{\pgfqpoint{2.917338in}{0.667548in}}%
\pgfpathlineto{\pgfqpoint{2.803354in}{0.674627in}}%
\pgfpathlineto{\pgfqpoint{2.672964in}{0.685302in}}%
\pgfpathlineto{\pgfqpoint{2.525084in}{0.699836in}}%
\pgfpathlineto{\pgfqpoint{2.360251in}{0.718475in}}%
\pgfpathlineto{\pgfqpoint{2.181796in}{0.741365in}}%
\pgfpathlineto{\pgfqpoint{2.042928in}{0.761318in}}%
\pgfpathlineto{\pgfqpoint{1.904686in}{0.783520in}}%
\pgfpathlineto{\pgfqpoint{1.771769in}{0.807732in}}%
\pgfpathlineto{\pgfqpoint{1.688399in}{0.824823in}}%
\pgfpathlineto{\pgfqpoint{1.610922in}{0.842528in}}%
\pgfpathlineto{\pgfqpoint{1.540700in}{0.860705in}}%
\pgfpathlineto{\pgfqpoint{1.477561in}{0.879199in}}%
\pgfpathlineto{\pgfqpoint{1.421186in}{0.897864in}}%
\pgfpathlineto{\pgfqpoint{1.371263in}{0.916568in}}%
\pgfpathlineto{\pgfqpoint{1.327489in}{0.935192in}}%
\pgfpathlineto{\pgfqpoint{1.289572in}{0.953627in}}%
\pgfpathlineto{\pgfqpoint{1.257227in}{0.971780in}}%
\pgfpathlineto{\pgfqpoint{1.230177in}{0.989568in}}%
\pgfpathlineto{\pgfqpoint{1.208155in}{1.006922in}}%
\pgfpathlineto{\pgfqpoint{1.190904in}{1.023786in}}%
\pgfpathlineto{\pgfqpoint{1.178170in}{1.040114in}}%
\pgfpathlineto{\pgfqpoint{1.169446in}{1.055843in}}%
\pgfpathlineto{\pgfqpoint{1.163980in}{1.070935in}}%
\pgfpathlineto{\pgfqpoint{1.161164in}{1.085374in}}%
\pgfpathlineto{\pgfqpoint{1.160531in}{1.099147in}}%
\pgfpathlineto{\pgfqpoint{1.161763in}{1.112245in}}%
\pgfpathlineto{\pgfqpoint{1.164682in}{1.124660in}}%
\pgfpathlineto{\pgfqpoint{1.169255in}{1.136389in}}%
\pgfpathlineto{\pgfqpoint{1.175595in}{1.147431in}}%
\pgfpathlineto{\pgfqpoint{1.183957in}{1.157790in}}%
\pgfpathlineto{\pgfqpoint{1.194739in}{1.167471in}}%
\pgfpathlineto{\pgfqpoint{1.208420in}{1.176482in}}%
\pgfpathlineto{\pgfqpoint{1.224357in}{1.184798in}}%
\pgfpathlineto{\pgfqpoint{1.251838in}{1.195946in}}%
\pgfpathlineto{\pgfqpoint{1.283660in}{1.205500in}}%
\pgfpathlineto{\pgfqpoint{1.319973in}{1.213451in}}%
\pgfpathlineto{\pgfqpoint{1.361038in}{1.219790in}}%
\pgfpathlineto{\pgfqpoint{1.407220in}{1.224505in}}%
\pgfpathlineto{\pgfqpoint{1.458994in}{1.227582in}}%
\pgfpathlineto{\pgfqpoint{1.516566in}{1.229001in}}%
\pgfpathlineto{\pgfqpoint{1.580584in}{1.228703in}}%
\pgfpathlineto{\pgfqpoint{1.652137in}{1.226611in}}%
\pgfpathlineto{\pgfqpoint{1.760697in}{1.220884in}}%
\pgfpathlineto{\pgfqpoint{1.885483in}{1.211613in}}%
\pgfpathlineto{\pgfqpoint{2.027080in}{1.198580in}}%
\pgfpathlineto{\pgfqpoint{2.185310in}{1.181560in}}%
\pgfpathlineto{\pgfqpoint{2.359318in}{1.160304in}}%
\pgfpathlineto{\pgfqpoint{2.497915in}{1.141459in}}%
\pgfpathlineto{\pgfqpoint{2.637716in}{1.120285in}}%
\pgfpathlineto{\pgfqpoint{2.773684in}{1.097029in}}%
\pgfpathlineto{\pgfqpoint{2.860079in}{1.080512in}}%
\pgfpathlineto{\pgfqpoint{2.941823in}{1.063304in}}%
\pgfpathlineto{\pgfqpoint{3.018026in}{1.045516in}}%
\pgfpathlineto{\pgfqpoint{3.087954in}{1.027272in}}%
\pgfpathlineto{\pgfqpoint{3.151037in}{1.008705in}}%
\pgfpathlineto{\pgfqpoint{3.206863in}{0.989961in}}%
\pgfpathlineto{\pgfqpoint{3.255187in}{0.971193in}}%
\pgfpathlineto{\pgfqpoint{3.296368in}{0.952551in}}%
\pgfpathlineto{\pgfqpoint{3.331342in}{0.934136in}}%
\pgfpathlineto{\pgfqpoint{3.360901in}{0.916035in}}%
\pgfpathlineto{\pgfqpoint{3.385670in}{0.898323in}}%
\pgfpathlineto{\pgfqpoint{3.406104in}{0.881067in}}%
\pgfpathlineto{\pgfqpoint{3.422492in}{0.864324in}}%
\pgfpathlineto{\pgfqpoint{3.434956in}{0.848145in}}%
\pgfpathlineto{\pgfqpoint{3.443576in}{0.832568in}}%
\pgfpathlineto{\pgfqpoint{3.449072in}{0.817627in}}%
\pgfpathlineto{\pgfqpoint{3.451849in}{0.803339in}}%
\pgfpathlineto{\pgfqpoint{3.452167in}{0.789719in}}%
\pgfpathlineto{\pgfqpoint{3.450217in}{0.776778in}}%
\pgfpathlineto{\pgfqpoint{3.446121in}{0.764526in}}%
\pgfpathlineto{\pgfqpoint{3.439929in}{0.752966in}}%
\pgfpathlineto{\pgfqpoint{3.431624in}{0.742099in}}%
\pgfpathlineto{\pgfqpoint{3.421206in}{0.731927in}}%
\pgfpathlineto{\pgfqpoint{3.408809in}{0.722452in}}%
\pgfpathlineto{\pgfqpoint{3.394500in}{0.713675in}}%
\pgfpathlineto{\pgfqpoint{3.378318in}{0.705601in}}%
\pgfpathlineto{\pgfqpoint{3.350546in}{0.694808in}}%
\pgfpathlineto{\pgfqpoint{3.318485in}{0.685604in}}%
\pgfpathlineto{\pgfqpoint{3.281905in}{0.677990in}}%
\pgfpathlineto{\pgfqpoint{3.240430in}{0.671970in}}%
\pgfpathlineto{\pgfqpoint{3.193581in}{0.667544in}}%
\pgfpathlineto{\pgfqpoint{3.141236in}{0.664744in}}%
\pgfpathlineto{\pgfqpoint{3.082778in}{0.663621in}}%
\pgfpathlineto{\pgfqpoint{3.017454in}{0.664230in}}%
\pgfpathlineto{\pgfqpoint{2.918488in}{0.667852in}}%
\pgfpathlineto{\pgfqpoint{2.804635in}{0.674854in}}%
\pgfpathlineto{\pgfqpoint{2.674641in}{0.685454in}}%
\pgfpathlineto{\pgfqpoint{2.527454in}{0.699897in}}%
\pgfpathlineto{\pgfqpoint{2.359928in}{0.718527in}}%
\pgfpathlineto{\pgfqpoint{2.224184in}{0.735346in}}%
\pgfpathlineto{\pgfqpoint{2.085427in}{0.754530in}}%
\pgfpathlineto{\pgfqpoint{1.948091in}{0.775941in}}%
\pgfpathlineto{\pgfqpoint{1.816018in}{0.799373in}}%
\pgfpathlineto{\pgfqpoint{1.732523in}{0.815987in}}%
\pgfpathlineto{\pgfqpoint{1.653639in}{0.833283in}}%
\pgfpathlineto{\pgfqpoint{1.580065in}{0.851153in}}%
\pgfpathlineto{\pgfqpoint{1.512382in}{0.869473in}}%
\pgfpathlineto{\pgfqpoint{1.451056in}{0.888109in}}%
\pgfpathlineto{\pgfqpoint{1.396432in}{0.906911in}}%
\pgfpathlineto{\pgfqpoint{1.348740in}{0.925718in}}%
\pgfpathlineto{\pgfqpoint{1.308094in}{0.944355in}}%
\pgfpathlineto{\pgfqpoint{1.274253in}{0.962667in}}%
\pgfpathlineto{\pgfqpoint{1.246014in}{0.980632in}}%
\pgfpathlineto{\pgfqpoint{1.222740in}{0.998182in}}%
\pgfpathlineto{\pgfqpoint{1.203928in}{1.015253in}}%
\pgfpathlineto{\pgfqpoint{1.189127in}{1.031791in}}%
\pgfpathlineto{\pgfqpoint{1.177918in}{1.047755in}}%
\pgfpathlineto{\pgfqpoint{1.169944in}{1.063113in}}%
\pgfpathlineto{\pgfqpoint{1.164907in}{1.077837in}}%
\pgfpathlineto{\pgfqpoint{1.162599in}{1.091909in}}%
\pgfpathlineto{\pgfqpoint{1.162931in}{1.105314in}}%
\pgfpathlineto{\pgfqpoint{1.165628in}{1.118042in}}%
\pgfpathlineto{\pgfqpoint{1.170445in}{1.130085in}}%
\pgfpathlineto{\pgfqpoint{1.177203in}{1.141436in}}%
\pgfpathlineto{\pgfqpoint{1.185789in}{1.152092in}}%
\pgfpathlineto{\pgfqpoint{1.196155in}{1.162050in}}%
\pgfpathlineto{\pgfqpoint{1.208321in}{1.171311in}}%
\pgfpathlineto{\pgfqpoint{1.222373in}{1.179878in}}%
\pgfpathlineto{\pgfqpoint{1.238461in}{1.187756in}}%
\pgfpathlineto{\pgfqpoint{1.256798in}{1.194952in}}%
\pgfpathlineto{\pgfqpoint{1.288260in}{1.204447in}}%
\pgfpathlineto{\pgfqpoint{1.324296in}{1.212364in}}%
\pgfpathlineto{\pgfqpoint{1.365132in}{1.218686in}}%
\pgfpathlineto{\pgfqpoint{1.411085in}{1.223391in}}%
\pgfpathlineto{\pgfqpoint{1.462561in}{1.226452in}}%
\pgfpathlineto{\pgfqpoint{1.520059in}{1.227835in}}%
\pgfpathlineto{\pgfqpoint{1.584165in}{1.227501in}}%
\pgfpathlineto{\pgfqpoint{1.680859in}{1.224312in}}%
\pgfpathlineto{\pgfqpoint{1.792035in}{1.217797in}}%
\pgfpathlineto{\pgfqpoint{1.920064in}{1.207674in}}%
\pgfpathlineto{\pgfqpoint{2.065678in}{1.193703in}}%
\pgfpathlineto{\pgfqpoint{2.227852in}{1.175689in}}%
\pgfpathlineto{\pgfqpoint{2.403806in}{1.153484in}}%
\pgfpathlineto{\pgfqpoint{2.542030in}{1.134009in}}%
\pgfpathlineto{\pgfqpoint{2.680899in}{1.112205in}}%
\pgfpathlineto{\pgfqpoint{2.815010in}{1.088375in}}%
\pgfpathlineto{\pgfqpoint{2.899545in}{1.071533in}}%
\pgfpathlineto{\pgfqpoint{2.978913in}{1.054058in}}%
\pgfpathlineto{\pgfqpoint{3.052250in}{1.036071in}}%
\pgfpathlineto{\pgfqpoint{3.118896in}{1.017699in}}%
\pgfpathlineto{\pgfqpoint{3.178386in}{0.999077in}}%
\pgfpathlineto{\pgfqpoint{3.230460in}{0.980348in}}%
\pgfpathlineto{\pgfqpoint{3.275105in}{0.961669in}}%
\pgfpathlineto{\pgfqpoint{3.313042in}{0.943172in}}%
\pgfpathlineto{\pgfqpoint{3.345238in}{0.924943in}}%
\pgfpathlineto{\pgfqpoint{3.372468in}{0.907059in}}%
\pgfpathlineto{\pgfqpoint{3.395304in}{0.889589in}}%
\pgfpathlineto{\pgfqpoint{3.414115in}{0.872596in}}%
\pgfpathlineto{\pgfqpoint{3.429069in}{0.856134in}}%
\pgfpathlineto{\pgfqpoint{3.440135in}{0.840248in}}%
\pgfpathlineto{\pgfqpoint{3.447238in}{0.824977in}}%
\pgfpathlineto{\pgfqpoint{3.451272in}{0.810353in}}%
\pgfpathlineto{\pgfqpoint{3.452694in}{0.796392in}}%
\pgfpathlineto{\pgfqpoint{3.451758in}{0.783107in}}%
\pgfpathlineto{\pgfqpoint{3.448649in}{0.770506in}}%
\pgfpathlineto{\pgfqpoint{3.443485in}{0.758597in}}%
\pgfpathlineto{\pgfqpoint{3.436313in}{0.747382in}}%
\pgfpathlineto{\pgfqpoint{3.427112in}{0.736864in}}%
\pgfpathlineto{\pgfqpoint{3.415798in}{0.727040in}}%
\pgfpathlineto{\pgfqpoint{3.402450in}{0.717912in}}%
\pgfpathlineto{\pgfqpoint{3.387173in}{0.709481in}}%
\pgfpathlineto{\pgfqpoint{3.360696in}{0.698151in}}%
\pgfpathlineto{\pgfqpoint{3.329923in}{0.688405in}}%
\pgfpathlineto{\pgfqpoint{3.294717in}{0.680252in}}%
\pgfpathlineto{\pgfqpoint{3.254815in}{0.673699in}}%
\pgfpathlineto{\pgfqpoint{3.209826in}{0.668755in}}%
\pgfpathlineto{\pgfqpoint{3.159246in}{0.665427in}}%
\pgfpathlineto{\pgfqpoint{3.102925in}{0.663743in}}%
\pgfpathlineto{\pgfqpoint{3.040159in}{0.663765in}}%
\pgfpathlineto{\pgfqpoint{2.970003in}{0.665564in}}%
\pgfpathlineto{\pgfqpoint{2.863659in}{0.670860in}}%
\pgfpathlineto{\pgfqpoint{2.741479in}{0.679652in}}%
\pgfpathlineto{\pgfqpoint{2.602664in}{0.692161in}}%
\pgfpathlineto{\pgfqpoint{2.446963in}{0.708626in}}%
\pgfpathlineto{\pgfqpoint{2.274396in}{0.729325in}}%
\pgfpathlineto{\pgfqpoint{2.136744in}{0.747714in}}%
\pgfpathlineto{\pgfqpoint{1.997655in}{0.768426in}}%
\pgfpathlineto{\pgfqpoint{1.861741in}{0.791260in}}%
\pgfpathlineto{\pgfqpoint{1.733007in}{0.815947in}}%
\pgfpathlineto{\pgfqpoint{1.652878in}{0.833271in}}%
\pgfpathlineto{\pgfqpoint{1.578324in}{0.851152in}}%
\pgfpathlineto{\pgfqpoint{1.510084in}{0.869460in}}%
\pgfpathlineto{\pgfqpoint{1.448777in}{0.888052in}}%
\pgfpathlineto{\pgfqpoint{1.394905in}{0.906770in}}%
\pgfpathlineto{\pgfqpoint{1.348476in}{0.925460in}}%
\pgfpathlineto{\pgfqpoint{1.308653in}{0.944009in}}%
\pgfpathlineto{\pgfqpoint{1.274769in}{0.962317in}}%
\pgfpathlineto{\pgfqpoint{1.246250in}{0.980295in}}%
\pgfpathlineto{\pgfqpoint{1.222620in}{0.997864in}}%
\pgfpathlineto{\pgfqpoint{1.203500in}{1.014958in}}%
\pgfpathlineto{\pgfqpoint{1.188597in}{1.031520in}}%
\pgfpathlineto{\pgfqpoint{1.177429in}{1.047506in}}%
\pgfpathlineto{\pgfqpoint{1.169499in}{1.062883in}}%
\pgfpathlineto{\pgfqpoint{1.164463in}{1.077627in}}%
\pgfpathlineto{\pgfqpoint{1.162060in}{1.091719in}}%
\pgfpathlineto{\pgfqpoint{1.162104in}{1.105143in}}%
\pgfpathlineto{\pgfqpoint{1.164489in}{1.117890in}}%
\pgfpathlineto{\pgfqpoint{1.169178in}{1.129951in}}%
\pgfpathlineto{\pgfqpoint{1.176067in}{1.141324in}}%
\pgfpathlineto{\pgfqpoint{1.185003in}{1.152004in}}%
\pgfpathlineto{\pgfqpoint{1.195868in}{1.161987in}}%
\pgfpathlineto{\pgfqpoint{1.208585in}{1.171270in}}%
\pgfpathlineto{\pgfqpoint{1.223117in}{1.179852in}}%
\pgfpathlineto{\pgfqpoint{1.248329in}{1.191408in}}%
\pgfpathlineto{\pgfqpoint{1.277797in}{1.201388in}}%
\pgfpathlineto{\pgfqpoint{1.311883in}{1.209799in}}%
\pgfpathlineto{\pgfqpoint{1.351006in}{1.216648in}}%
\pgfpathlineto{\pgfqpoint{1.395134in}{1.221904in}}%
\pgfpathlineto{\pgfqpoint{1.444660in}{1.225535in}}%
\pgfpathlineto{\pgfqpoint{1.500085in}{1.227507in}}%
\pgfpathlineto{\pgfqpoint{1.561959in}{1.227772in}}%
\pgfpathlineto{\pgfqpoint{1.630880in}{1.226274in}}%
\pgfpathlineto{\pgfqpoint{1.734865in}{1.221415in}}%
\pgfpathlineto{\pgfqpoint{1.854202in}{1.213114in}}%
\pgfpathlineto{\pgfqpoint{1.990413in}{1.201135in}}%
\pgfpathlineto{\pgfqpoint{2.144146in}{1.185209in}}%
\pgfpathlineto{\pgfqpoint{2.313931in}{1.165126in}}%
\pgfpathlineto{\pgfqpoint{2.449397in}{1.147263in}}%
\pgfpathlineto{\pgfqpoint{2.588097in}{1.127035in}}%
\pgfpathlineto{\pgfqpoint{2.725721in}{1.104586in}}%
\pgfpathlineto{\pgfqpoint{2.857484in}{1.080168in}}%
\pgfpathlineto{\pgfqpoint{2.939480in}{1.062953in}}%
\pgfpathlineto{\pgfqpoint{3.015505in}{1.045141in}}%
\pgfpathlineto{\pgfqpoint{3.085028in}{1.026897in}}%
\pgfpathlineto{\pgfqpoint{3.147704in}{1.008374in}}%
\pgfpathlineto{\pgfqpoint{3.203365in}{0.989714in}}%
\pgfpathlineto{\pgfqpoint{3.252029in}{0.971045in}}%
\pgfpathlineto{\pgfqpoint{3.293890in}{0.952483in}}%
\pgfpathlineto{\pgfqpoint{3.329329in}{0.934133in}}%
\pgfpathlineto{\pgfqpoint{3.358905in}{0.916088in}}%
\pgfpathlineto{\pgfqpoint{3.383359in}{0.898425in}}%
\pgfpathlineto{\pgfqpoint{3.403191in}{0.881225in}}%
\pgfpathlineto{\pgfqpoint{3.418593in}{0.864555in}}%
\pgfpathlineto{\pgfqpoint{3.430393in}{0.848447in}}%
\pgfpathlineto{\pgfqpoint{3.439247in}{0.832929in}}%
\pgfpathlineto{\pgfqpoint{3.445614in}{0.818025in}}%
\pgfpathlineto{\pgfqpoint{3.449761in}{0.803755in}}%
\pgfpathlineto{\pgfqpoint{3.451762in}{0.790133in}}%
\pgfpathlineto{\pgfqpoint{3.451496in}{0.777172in}}%
\pgfpathlineto{\pgfqpoint{3.448648in}{0.764879in}}%
\pgfpathlineto{\pgfqpoint{3.442722in}{0.753258in}}%
\pgfpathlineto{\pgfqpoint{3.434209in}{0.742325in}}%
\pgfpathlineto{\pgfqpoint{3.423713in}{0.732094in}}%
\pgfpathlineto{\pgfqpoint{3.411310in}{0.722566in}}%
\pgfpathlineto{\pgfqpoint{3.397047in}{0.713743in}}%
\pgfpathlineto{\pgfqpoint{3.380940in}{0.705626in}}%
\pgfpathlineto{\pgfqpoint{3.353279in}{0.694773in}}%
\pgfpathlineto{\pgfqpoint{3.321250in}{0.685507in}}%
\pgfpathlineto{\pgfqpoint{3.284522in}{0.677822in}}%
\pgfpathlineto{\pgfqpoint{3.242847in}{0.671723in}}%
\pgfpathlineto{\pgfqpoint{3.196043in}{0.667236in}}%
\pgfpathlineto{\pgfqpoint{3.143609in}{0.664393in}}%
\pgfpathlineto{\pgfqpoint{3.085000in}{0.663235in}}%
\pgfpathlineto{\pgfqpoint{3.019635in}{0.663813in}}%
\pgfpathlineto{\pgfqpoint{2.920893in}{0.667396in}}%
\pgfpathlineto{\pgfqpoint{2.807453in}{0.674356in}}%
\pgfpathlineto{\pgfqpoint{2.677658in}{0.684903in}}%
\pgfpathlineto{\pgfqpoint{2.530394in}{0.699299in}}%
\pgfpathlineto{\pgfqpoint{2.366121in}{0.717793in}}%
\pgfpathlineto{\pgfqpoint{2.188059in}{0.740535in}}%
\pgfpathlineto{\pgfqpoint{2.049294in}{0.760379in}}%
\pgfpathlineto{\pgfqpoint{1.910935in}{0.782481in}}%
\pgfpathlineto{\pgfqpoint{1.777674in}{0.806606in}}%
\pgfpathlineto{\pgfqpoint{1.693960in}{0.823649in}}%
\pgfpathlineto{\pgfqpoint{1.616074in}{0.841316in}}%
\pgfpathlineto{\pgfqpoint{1.545391in}{0.859467in}}%
\pgfpathlineto{\pgfqpoint{1.481767in}{0.877945in}}%
\pgfpathlineto{\pgfqpoint{1.424910in}{0.896604in}}%
\pgfpathlineto{\pgfqpoint{1.374526in}{0.915309in}}%
\pgfpathlineto{\pgfqpoint{1.330328in}{0.933940in}}%
\pgfpathlineto{\pgfqpoint{1.292027in}{0.952388in}}%
\pgfpathlineto{\pgfqpoint{1.259339in}{0.970557in}}%
\pgfpathlineto{\pgfqpoint{1.231982in}{0.988367in}}%
\pgfpathlineto{\pgfqpoint{1.209675in}{1.005746in}}%
\pgfpathlineto{\pgfqpoint{1.192141in}{1.022639in}}%
\pgfpathlineto{\pgfqpoint{1.179102in}{1.039000in}}%
\pgfpathlineto{\pgfqpoint{1.170104in}{1.054770in}}%
\pgfpathlineto{\pgfqpoint{1.164416in}{1.069904in}}%
\pgfpathlineto{\pgfqpoint{1.161419in}{1.084386in}}%
\pgfpathlineto{\pgfqpoint{1.160639in}{1.098204in}}%
\pgfpathlineto{\pgfqpoint{1.161745in}{1.111347in}}%
\pgfpathlineto{\pgfqpoint{1.164549in}{1.123808in}}%
\pgfpathlineto{\pgfqpoint{1.169007in}{1.135583in}}%
\pgfpathlineto{\pgfqpoint{1.175219in}{1.146671in}}%
\pgfpathlineto{\pgfqpoint{1.183428in}{1.157076in}}%
\pgfpathlineto{\pgfqpoint{1.194021in}{1.166803in}}%
\pgfpathlineto{\pgfqpoint{1.207493in}{1.175859in}}%
\pgfpathlineto{\pgfqpoint{1.223290in}{1.184223in}}%
\pgfpathlineto{\pgfqpoint{1.250564in}{1.195444in}}%
\pgfpathlineto{\pgfqpoint{1.282174in}{1.205071in}}%
\pgfpathlineto{\pgfqpoint{1.318267in}{1.213096in}}%
\pgfpathlineto{\pgfqpoint{1.359097in}{1.219510in}}%
\pgfpathlineto{\pgfqpoint{1.405026in}{1.224300in}}%
\pgfpathlineto{\pgfqpoint{1.456522in}{1.227453in}}%
\pgfpathlineto{\pgfqpoint{1.513819in}{1.228949in}}%
\pgfpathlineto{\pgfqpoint{1.577502in}{1.228734in}}%
\pgfpathlineto{\pgfqpoint{1.648676in}{1.226727in}}%
\pgfpathlineto{\pgfqpoint{1.756679in}{1.221121in}}%
\pgfpathlineto{\pgfqpoint{1.880876in}{1.211977in}}%
\pgfpathlineto{\pgfqpoint{2.021877in}{1.199081in}}%
\pgfpathlineto{\pgfqpoint{2.179526in}{1.182205in}}%
\pgfpathlineto{\pgfqpoint{2.352976in}{1.161105in}}%
\pgfpathlineto{\pgfqpoint{2.491364in}{1.142375in}}%
\pgfpathlineto{\pgfqpoint{2.631222in}{1.121306in}}%
\pgfpathlineto{\pgfqpoint{2.767470in}{1.098141in}}%
\pgfpathlineto{\pgfqpoint{2.854155in}{1.081678in}}%
\pgfpathlineto{\pgfqpoint{2.936257in}{1.064516in}}%
\pgfpathlineto{\pgfqpoint{3.012874in}{1.046766in}}%
\pgfpathlineto{\pgfqpoint{3.083263in}{1.028550in}}%
\pgfpathlineto{\pgfqpoint{3.146841in}{1.010003in}}%
\pgfpathlineto{\pgfqpoint{3.203187in}{0.991266in}}%
\pgfpathlineto{\pgfqpoint{3.252039in}{0.972495in}}%
\pgfpathlineto{\pgfqpoint{3.293687in}{0.953841in}}%
\pgfpathlineto{\pgfqpoint{3.329065in}{0.935409in}}%
\pgfpathlineto{\pgfqpoint{3.358979in}{0.917284in}}%
\pgfpathlineto{\pgfqpoint{3.384066in}{0.899543in}}%
\pgfpathlineto{\pgfqpoint{3.404796in}{0.882253in}}%
\pgfpathlineto{\pgfqpoint{3.421468in}{0.865473in}}%
\pgfpathlineto{\pgfqpoint{3.434215in}{0.849253in}}%
\pgfpathlineto{\pgfqpoint{3.443092in}{0.833633in}}%
\pgfpathlineto{\pgfqpoint{3.448788in}{0.818647in}}%
\pgfpathlineto{\pgfqpoint{3.451744in}{0.804313in}}%
\pgfpathlineto{\pgfqpoint{3.452224in}{0.790646in}}%
\pgfpathlineto{\pgfqpoint{3.450426in}{0.777658in}}%
\pgfpathlineto{\pgfqpoint{3.446476in}{0.765357in}}%
\pgfpathlineto{\pgfqpoint{3.440431in}{0.753748in}}%
\pgfpathlineto{\pgfqpoint{3.432279in}{0.742834in}}%
\pgfpathlineto{\pgfqpoint{3.422006in}{0.732613in}}%
\pgfpathlineto{\pgfqpoint{3.409744in}{0.723089in}}%
\pgfpathlineto{\pgfqpoint{3.395567in}{0.714264in}}%
\pgfpathlineto{\pgfqpoint{3.379514in}{0.706140in}}%
\pgfpathlineto{\pgfqpoint{3.351934in}{0.695274in}}%
\pgfpathlineto{\pgfqpoint{3.320073in}{0.685995in}}%
\pgfpathlineto{\pgfqpoint{3.283707in}{0.678308in}}%
\pgfpathlineto{\pgfqpoint{3.242470in}{0.672213in}}%
\pgfpathlineto{\pgfqpoint{3.195876in}{0.667713in}}%
\pgfpathlineto{\pgfqpoint{3.143795in}{0.664837in}}%
\pgfpathlineto{\pgfqpoint{3.085642in}{0.663635in}}%
\pgfpathlineto{\pgfqpoint{3.020656in}{0.664162in}}%
\pgfpathlineto{\pgfqpoint{2.922184in}{0.667669in}}%
\pgfpathlineto{\pgfqpoint{2.808866in}{0.674551in}}%
\pgfpathlineto{\pgfqpoint{2.679446in}{0.685021in}}%
\pgfpathlineto{\pgfqpoint{2.532878in}{0.699326in}}%
\pgfpathlineto{\pgfqpoint{2.366183in}{0.717809in}}%
\pgfpathlineto{\pgfqpoint{2.230695in}{0.734526in}}%
\pgfpathlineto{\pgfqpoint{2.091941in}{0.753613in}}%
\pgfpathlineto{\pgfqpoint{1.954412in}{0.774930in}}%
\pgfpathlineto{\pgfqpoint{1.821999in}{0.798273in}}%
\pgfpathlineto{\pgfqpoint{1.738212in}{0.814831in}}%
\pgfpathlineto{\pgfqpoint{1.658997in}{0.832076in}}%
\pgfpathlineto{\pgfqpoint{1.585058in}{0.849900in}}%
\pgfpathlineto{\pgfqpoint{1.516982in}{0.868182in}}%
\pgfpathlineto{\pgfqpoint{1.455236in}{0.886788in}}%
\pgfpathlineto{\pgfqpoint{1.400168in}{0.905572in}}%
\pgfpathlineto{\pgfqpoint{1.352007in}{0.924375in}}%
\pgfpathlineto{\pgfqpoint{1.310862in}{0.943026in}}%
\pgfpathlineto{\pgfqpoint{1.276559in}{0.961362in}}%
\pgfpathlineto{\pgfqpoint{1.247934in}{0.979352in}}%
\pgfpathlineto{\pgfqpoint{1.224317in}{0.996933in}}%
\pgfpathlineto{\pgfqpoint{1.205200in}{1.014039in}}%
\pgfpathlineto{\pgfqpoint{1.190121in}{1.030617in}}%
\pgfpathlineto{\pgfqpoint{1.178660in}{1.046623in}}%
\pgfpathlineto{\pgfqpoint{1.170458in}{1.062024in}}%
\pgfpathlineto{\pgfqpoint{1.165217in}{1.076794in}}%
\pgfpathlineto{\pgfqpoint{1.162800in}{1.090913in}}%
\pgfpathlineto{\pgfqpoint{1.162998in}{1.104366in}}%
\pgfpathlineto{\pgfqpoint{1.165504in}{1.117142in}}%
\pgfpathlineto{\pgfqpoint{1.170083in}{1.129232in}}%
\pgfpathlineto{\pgfqpoint{1.176571in}{1.140632in}}%
\pgfpathlineto{\pgfqpoint{1.184877in}{1.151337in}}%
\pgfpathlineto{\pgfqpoint{1.194985in}{1.161346in}}%
\pgfpathlineto{\pgfqpoint{1.206948in}{1.170661in}}%
\pgfpathlineto{\pgfqpoint{1.220894in}{1.179285in}}%
\pgfpathlineto{\pgfqpoint{1.237022in}{1.187224in}}%
\pgfpathlineto{\pgfqpoint{1.265318in}{1.197845in}}%
\pgfpathlineto{\pgfqpoint{1.298084in}{1.206892in}}%
\pgfpathlineto{\pgfqpoint{1.335466in}{1.214353in}}%
\pgfpathlineto{\pgfqpoint{1.377716in}{1.220211in}}%
\pgfpathlineto{\pgfqpoint{1.425181in}{1.224443in}}%
\pgfpathlineto{\pgfqpoint{1.478303in}{1.227024in}}%
\pgfpathlineto{\pgfqpoint{1.537620in}{1.227922in}}%
\pgfpathlineto{\pgfqpoint{1.603638in}{1.227112in}}%
\pgfpathlineto{\pgfqpoint{1.702682in}{1.223336in}}%
\pgfpathlineto{\pgfqpoint{1.817405in}{1.216136in}}%
\pgfpathlineto{\pgfqpoint{1.949802in}{1.205187in}}%
\pgfpathlineto{\pgfqpoint{2.099903in}{1.190259in}}%
\pgfpathlineto{\pgfqpoint{2.265763in}{1.171217in}}%
\pgfpathlineto{\pgfqpoint{2.443474in}{1.148025in}}%
\pgfpathlineto{\pgfqpoint{2.581069in}{1.127939in}}%
\pgfpathlineto{\pgfqpoint{2.718642in}{1.105608in}}%
\pgfpathlineto{\pgfqpoint{2.850600in}{1.081243in}}%
\pgfpathlineto{\pgfqpoint{2.933081in}{1.064089in}}%
\pgfpathlineto{\pgfqpoint{3.009940in}{1.046366in}}%
\pgfpathlineto{\pgfqpoint{3.080409in}{1.028205in}}%
\pgfpathlineto{\pgfqpoint{3.143968in}{1.009740in}}%
\pgfpathlineto{\pgfqpoint{3.200351in}{0.991100in}}%
\pgfpathlineto{\pgfqpoint{3.249541in}{0.972412in}}%
\pgfpathlineto{\pgfqpoint{3.291772in}{0.953801in}}%
\pgfpathlineto{\pgfqpoint{3.327363in}{0.935402in}}%
\pgfpathlineto{\pgfqpoint{3.356797in}{0.917338in}}%
\pgfpathlineto{\pgfqpoint{3.381223in}{0.899666in}}%
\pgfpathlineto{\pgfqpoint{3.401567in}{0.882438in}}%
\pgfpathlineto{\pgfqpoint{3.418489in}{0.865699in}}%
\pgfpathlineto{\pgfqpoint{3.432382in}{0.849491in}}%
\pgfpathlineto{\pgfqpoint{3.443372in}{0.833853in}}%
\pgfpathlineto{\pgfqpoint{3.451320in}{0.818817in}}%
\pgfpathlineto{\pgfqpoint{3.455819in}{0.804414in}}%
\pgfpathlineto{\pgfqpoint{3.456373in}{0.790670in}}%
\pgfpathlineto{\pgfqpoint{3.454171in}{0.777611in}}%
\pgfpathlineto{\pgfqpoint{3.449763in}{0.765247in}}%
\pgfpathlineto{\pgfqpoint{3.443307in}{0.753583in}}%
\pgfpathlineto{\pgfqpoint{3.434910in}{0.742622in}}%
\pgfpathlineto{\pgfqpoint{3.424635in}{0.732365in}}%
\pgfpathlineto{\pgfqpoint{3.412501in}{0.722814in}}%
\pgfpathlineto{\pgfqpoint{3.398478in}{0.713967in}}%
\pgfpathlineto{\pgfqpoint{3.382492in}{0.705819in}}%
\pgfpathlineto{\pgfqpoint{3.354703in}{0.694907in}}%
\pgfpathlineto{\pgfqpoint{3.322472in}{0.685573in}}%
\pgfpathlineto{\pgfqpoint{3.285673in}{0.677829in}}%
\pgfpathlineto{\pgfqpoint{3.244070in}{0.671692in}}%
\pgfpathlineto{\pgfqpoint{3.197330in}{0.667183in}}%
\pgfpathlineto{\pgfqpoint{3.145021in}{0.664328in}}%
\pgfpathlineto{\pgfqpoint{3.086618in}{0.663157in}}%
\pgfpathlineto{\pgfqpoint{3.021498in}{0.663706in}}%
\pgfpathlineto{\pgfqpoint{2.923547in}{0.667176in}}%
\pgfpathlineto{\pgfqpoint{2.810667in}{0.674012in}}%
\pgfpathlineto{\pgfqpoint{2.680543in}{0.684512in}}%
\pgfpathlineto{\pgfqpoint{2.532718in}{0.698911in}}%
\pgfpathlineto{\pgfqpoint{2.368598in}{0.717383in}}%
\pgfpathlineto{\pgfqpoint{2.191447in}{0.740039in}}%
\pgfpathlineto{\pgfqpoint{2.053073in}{0.759811in}}%
\pgfpathlineto{\pgfqpoint{1.914433in}{0.781928in}}%
\pgfpathlineto{\pgfqpoint{1.780982in}{0.806063in}}%
\pgfpathlineto{\pgfqpoint{1.697133in}{0.823080in}}%
\pgfpathlineto{\pgfqpoint{1.618627in}{0.840698in}}%
\pgfpathlineto{\pgfqpoint{1.546291in}{0.858792in}}%
\pgfpathlineto{\pgfqpoint{1.480736in}{0.877235in}}%
\pgfpathlineto{\pgfqpoint{1.422355in}{0.895899in}}%
\pgfpathlineto{\pgfqpoint{1.371325in}{0.914649in}}%
\pgfpathlineto{\pgfqpoint{1.327607in}{0.933351in}}%
\pgfpathlineto{\pgfqpoint{1.290870in}{0.951855in}}%
\pgfpathlineto{\pgfqpoint{1.260139in}{0.970052in}}%
\pgfpathlineto{\pgfqpoint{1.234417in}{0.987877in}}%
\pgfpathlineto{\pgfqpoint{1.212938in}{1.005269in}}%
\pgfpathlineto{\pgfqpoint{1.195173in}{1.022173in}}%
\pgfpathlineto{\pgfqpoint{1.180823in}{1.038542in}}%
\pgfpathlineto{\pgfqpoint{1.169827in}{1.054334in}}%
\pgfpathlineto{\pgfqpoint{1.162353in}{1.069513in}}%
\pgfpathlineto{\pgfqpoint{1.158670in}{1.084049in}}%
\pgfpathlineto{\pgfqpoint{1.157863in}{1.097916in}}%
\pgfpathlineto{\pgfqpoint{1.159442in}{1.111101in}}%
\pgfpathlineto{\pgfqpoint{1.163205in}{1.123596in}}%
\pgfpathlineto{\pgfqpoint{1.169005in}{1.135394in}}%
\pgfpathlineto{\pgfqpoint{1.176752in}{1.146494in}}%
\pgfpathlineto{\pgfqpoint{1.186412in}{1.156892in}}%
\pgfpathlineto{\pgfqpoint{1.198005in}{1.166589in}}%
\pgfpathlineto{\pgfqpoint{1.211609in}{1.175588in}}%
\pgfpathlineto{\pgfqpoint{1.227264in}{1.183892in}}%
\pgfpathlineto{\pgfqpoint{1.254379in}{1.195036in}}%
\pgfpathlineto{\pgfqpoint{1.285867in}{1.204598in}}%
\pgfpathlineto{\pgfqpoint{1.321853in}{1.212568in}}%
\pgfpathlineto{\pgfqpoint{1.362571in}{1.218931in}}%
\pgfpathlineto{\pgfqpoint{1.408362in}{1.223672in}}%
\pgfpathlineto{\pgfqpoint{1.459677in}{1.226773in}}%
\pgfpathlineto{\pgfqpoint{1.517076in}{1.228212in}}%
\pgfpathlineto{\pgfqpoint{1.580731in}{1.227973in}}%
\pgfpathlineto{\pgfqpoint{1.676814in}{1.224895in}}%
\pgfpathlineto{\pgfqpoint{1.788307in}{1.218427in}}%
\pgfpathlineto{\pgfqpoint{1.916785in}{1.208332in}}%
\pgfpathlineto{\pgfqpoint{2.062480in}{1.194400in}}%
\pgfpathlineto{\pgfqpoint{2.224284in}{1.176450in}}%
\pgfpathlineto{\pgfqpoint{2.399741in}{1.154325in}}%
\pgfpathlineto{\pgfqpoint{2.537885in}{1.134913in}}%
\pgfpathlineto{\pgfqpoint{2.676937in}{1.113163in}}%
\pgfpathlineto{\pgfqpoint{2.811381in}{1.089369in}}%
\pgfpathlineto{\pgfqpoint{2.896177in}{1.072541in}}%
\pgfpathlineto{\pgfqpoint{2.975811in}{1.055070in}}%
\pgfpathlineto{\pgfqpoint{3.049404in}{1.037079in}}%
\pgfpathlineto{\pgfqpoint{3.116280in}{1.018698in}}%
\pgfpathlineto{\pgfqpoint{3.175966in}{1.000066in}}%
\pgfpathlineto{\pgfqpoint{3.228194in}{0.981328in}}%
\pgfpathlineto{\pgfqpoint{3.273022in}{0.962646in}}%
\pgfpathlineto{\pgfqpoint{3.311297in}{0.944134in}}%
\pgfpathlineto{\pgfqpoint{3.343909in}{0.925882in}}%
\pgfpathlineto{\pgfqpoint{3.371554in}{0.907969in}}%
\pgfpathlineto{\pgfqpoint{3.394739in}{0.890467in}}%
\pgfpathlineto{\pgfqpoint{3.413779in}{0.873440in}}%
\pgfpathlineto{\pgfqpoint{3.428802in}{0.856943in}}%
\pgfpathlineto{\pgfqpoint{3.439741in}{0.841024in}}%
\pgfpathlineto{\pgfqpoint{3.446835in}{0.825724in}}%
\pgfpathlineto{\pgfqpoint{3.450997in}{0.811070in}}%
\pgfpathlineto{\pgfqpoint{3.452566in}{0.797077in}}%
\pgfpathlineto{\pgfqpoint{3.451791in}{0.783758in}}%
\pgfpathlineto{\pgfqpoint{3.448851in}{0.771124in}}%
\pgfpathlineto{\pgfqpoint{3.443846in}{0.759180in}}%
\pgfpathlineto{\pgfqpoint{3.436803in}{0.747931in}}%
\pgfpathlineto{\pgfqpoint{3.427676in}{0.737376in}}%
\pgfpathlineto{\pgfqpoint{3.416396in}{0.727515in}}%
\pgfpathlineto{\pgfqpoint{3.403129in}{0.718350in}}%
\pgfpathlineto{\pgfqpoint{3.387946in}{0.709885in}}%
\pgfpathlineto{\pgfqpoint{3.361631in}{0.698503in}}%
\pgfpathlineto{\pgfqpoint{3.331041in}{0.688707in}}%
\pgfpathlineto{\pgfqpoint{3.296029in}{0.680503in}}%
\pgfpathlineto{\pgfqpoint{3.256314in}{0.673898in}}%
\pgfpathlineto{\pgfqpoint{3.211488in}{0.668896in}}%
\pgfpathlineto{\pgfqpoint{3.161101in}{0.665505in}}%
\pgfpathlineto{\pgfqpoint{3.104998in}{0.663764in}}%
\pgfpathlineto{\pgfqpoint{3.042362in}{0.663731in}}%
\pgfpathlineto{\pgfqpoint{2.972367in}{0.665473in}}%
\pgfpathlineto{\pgfqpoint{2.866391in}{0.670684in}}%
\pgfpathlineto{\pgfqpoint{2.744760in}{0.679380in}}%
\pgfpathlineto{\pgfqpoint{2.606529in}{0.691786in}}%
\pgfpathlineto{\pgfqpoint{2.451142in}{0.708153in}}%
\pgfpathlineto{\pgfqpoint{2.277880in}{0.728774in}}%
\pgfpathlineto{\pgfqpoint{2.140376in}{0.747076in}}%
\pgfpathlineto{\pgfqpoint{2.001724in}{0.767696in}}%
\pgfpathlineto{\pgfqpoint{1.866200in}{0.790449in}}%
\pgfpathlineto{\pgfqpoint{1.737589in}{0.815077in}}%
\pgfpathlineto{\pgfqpoint{1.657341in}{0.832375in}}%
\pgfpathlineto{\pgfqpoint{1.582504in}{0.850243in}}%
\pgfpathlineto{\pgfqpoint{1.513839in}{0.868548in}}%
\pgfpathlineto{\pgfqpoint{1.452015in}{0.887143in}}%
\pgfpathlineto{\pgfqpoint{1.397603in}{0.905866in}}%
\pgfpathlineto{\pgfqpoint{1.350749in}{0.924558in}}%
\pgfpathlineto{\pgfqpoint{1.310546in}{0.943116in}}%
\pgfpathlineto{\pgfqpoint{1.276364in}{0.961437in}}%
\pgfpathlineto{\pgfqpoint{1.247654in}{0.979430in}}%
\pgfpathlineto{\pgfqpoint{1.223932in}{0.997016in}}%
\pgfpathlineto{\pgfqpoint{1.204771in}{1.014126in}}%
\pgfpathlineto{\pgfqpoint{1.189750in}{1.030706in}}%
\pgfpathlineto{\pgfqpoint{1.178357in}{1.046713in}}%
\pgfpathlineto{\pgfqpoint{1.170196in}{1.062115in}}%
\pgfpathlineto{\pgfqpoint{1.164955in}{1.076886in}}%
\pgfpathlineto{\pgfqpoint{1.162395in}{1.091006in}}%
\pgfpathlineto{\pgfqpoint{1.162360in}{1.104459in}}%
\pgfpathlineto{\pgfqpoint{1.164766in}{1.117236in}}%
\pgfpathlineto{\pgfqpoint{1.169468in}{1.129329in}}%
\pgfpathlineto{\pgfqpoint{1.176271in}{1.140733in}}%
\pgfpathlineto{\pgfqpoint{1.185026in}{1.151441in}}%
\pgfpathlineto{\pgfqpoint{1.195633in}{1.161451in}}%
\pgfpathlineto{\pgfqpoint{1.208041in}{1.170761in}}%
\pgfpathlineto{\pgfqpoint{1.222246in}{1.179371in}}%
\pgfpathlineto{\pgfqpoint{1.238294in}{1.187283in}}%
\pgfpathlineto{\pgfqpoint{1.266038in}{1.197847in}}%
\pgfpathlineto{\pgfqpoint{1.298632in}{1.206860in}}%
\pgfpathlineto{\pgfqpoint{1.336013in}{1.214305in}}%
\pgfpathlineto{\pgfqpoint{1.378286in}{1.220156in}}%
\pgfpathlineto{\pgfqpoint{1.425810in}{1.224389in}}%
\pgfpathlineto{\pgfqpoint{1.479019in}{1.226972in}}%
\pgfpathlineto{\pgfqpoint{1.538423in}{1.227862in}}%
\pgfpathlineto{\pgfqpoint{1.604607in}{1.227013in}}%
\pgfpathlineto{\pgfqpoint{1.704551in}{1.223072in}}%
\pgfpathlineto{\pgfqpoint{1.819382in}{1.215763in}}%
\pgfpathlineto{\pgfqpoint{1.950852in}{1.204832in}}%
\pgfpathlineto{\pgfqpoint{2.099814in}{1.190023in}}%
\pgfpathlineto{\pgfqpoint{2.265350in}{1.171115in}}%
\pgfpathlineto{\pgfqpoint{2.444474in}{1.147948in}}%
\pgfpathlineto{\pgfqpoint{2.583115in}{1.127802in}}%
\pgfpathlineto{\pgfqpoint{2.720640in}{1.105433in}}%
\pgfpathlineto{\pgfqpoint{2.852580in}{1.081089in}}%
\pgfpathlineto{\pgfqpoint{2.935168in}{1.063931in}}%
\pgfpathlineto{\pgfqpoint{3.011834in}{1.046178in}}%
\pgfpathlineto{\pgfqpoint{3.081607in}{1.027970in}}%
\pgfpathlineto{\pgfqpoint{3.144374in}{1.009462in}}%
\pgfpathlineto{\pgfqpoint{3.200182in}{0.990797in}}%
\pgfpathlineto{\pgfqpoint{3.249171in}{0.972108in}}%
\pgfpathlineto{\pgfqpoint{3.291572in}{0.953516in}}%
\pgfpathlineto{\pgfqpoint{3.327709in}{0.935128in}}%
\pgfpathlineto{\pgfqpoint{3.357998in}{0.917040in}}%
\pgfpathlineto{\pgfqpoint{3.382947in}{0.899337in}}%
\pgfpathlineto{\pgfqpoint{3.403093in}{0.882093in}}%
\pgfpathlineto{\pgfqpoint{3.418862in}{0.865374in}}%
\pgfpathlineto{\pgfqpoint{3.430942in}{0.849218in}}%
\pgfpathlineto{\pgfqpoint{3.439909in}{0.833654in}}%
\pgfpathlineto{\pgfqpoint{3.446183in}{0.818710in}}%
\pgfpathlineto{\pgfqpoint{3.450022in}{0.804404in}}%
\pgfpathlineto{\pgfqpoint{3.451530in}{0.790755in}}%
\pgfpathlineto{\pgfqpoint{3.450647in}{0.777773in}}%
\pgfpathlineto{\pgfqpoint{3.447160in}{0.765466in}}%
\pgfpathlineto{\pgfqpoint{3.440944in}{0.753839in}}%
\pgfpathlineto{\pgfqpoint{3.432596in}{0.742906in}}%
\pgfpathlineto{\pgfqpoint{3.422275in}{0.732673in}}%
\pgfpathlineto{\pgfqpoint{3.410059in}{0.723140in}}%
\pgfpathlineto{\pgfqpoint{3.395992in}{0.714311in}}%
\pgfpathlineto{\pgfqpoint{3.380087in}{0.706185in}}%
\pgfpathlineto{\pgfqpoint{3.352725in}{0.695314in}}%
\pgfpathlineto{\pgfqpoint{3.320954in}{0.686022in}}%
\pgfpathlineto{\pgfqpoint{3.284404in}{0.678302in}}%
\pgfpathlineto{\pgfqpoint{3.242959in}{0.672166in}}%
\pgfpathlineto{\pgfqpoint{3.196362in}{0.667641in}}%
\pgfpathlineto{\pgfqpoint{3.144142in}{0.664757in}}%
\pgfpathlineto{\pgfqpoint{3.085779in}{0.663556in}}%
\pgfpathlineto{\pgfqpoint{3.020703in}{0.664088in}}%
\pgfpathlineto{\pgfqpoint{2.922405in}{0.667603in}}%
\pgfpathlineto{\pgfqpoint{2.809442in}{0.674482in}}%
\pgfpathlineto{\pgfqpoint{2.680143in}{0.684936in}}%
\pgfpathlineto{\pgfqpoint{2.533383in}{0.699235in}}%
\pgfpathlineto{\pgfqpoint{2.369667in}{0.717618in}}%
\pgfpathlineto{\pgfqpoint{2.191967in}{0.740246in}}%
\pgfpathlineto{\pgfqpoint{2.053504in}{0.760001in}}%
\pgfpathlineto{\pgfqpoint{1.915170in}{0.782019in}}%
\pgfpathlineto{\pgfqpoint{1.781673in}{0.806069in}}%
\pgfpathlineto{\pgfqpoint{1.697946in}{0.823079in}}%
\pgfpathlineto{\pgfqpoint{1.619942in}{0.840735in}}%
\pgfpathlineto{\pgfqpoint{1.548284in}{0.858873in}}%
\pgfpathlineto{\pgfqpoint{1.483394in}{0.877334in}}%
\pgfpathlineto{\pgfqpoint{1.425506in}{0.895975in}}%
\pgfpathlineto{\pgfqpoint{1.374665in}{0.914664in}}%
\pgfpathlineto{\pgfqpoint{1.330728in}{0.933281in}}%
\pgfpathlineto{\pgfqpoint{1.293365in}{0.951717in}}%
\pgfpathlineto{\pgfqpoint{1.262056in}{0.969877in}}%
\pgfpathlineto{\pgfqpoint{1.236096in}{0.987677in}}%
\pgfpathlineto{\pgfqpoint{1.214827in}{1.005038in}}%
\pgfpathlineto{\pgfqpoint{1.198161in}{1.021884in}}%
\pgfpathlineto{\pgfqpoint{1.185297in}{1.038178in}}%
\pgfpathlineto{\pgfqpoint{1.175532in}{1.053890in}}%
\pgfpathlineto{\pgfqpoint{1.168361in}{1.068996in}}%
\pgfpathlineto{\pgfqpoint{1.163475in}{1.083474in}}%
\pgfpathlineto{\pgfqpoint{1.160761in}{1.097307in}}%
\pgfpathlineto{\pgfqpoint{1.160302in}{1.110483in}}%
\pgfpathlineto{\pgfqpoint{1.162377in}{1.122993in}}%
\pgfpathlineto{\pgfqpoint{1.167462in}{1.134832in}}%
\pgfpathlineto{\pgfqpoint{1.175344in}{1.145986in}}%
\pgfpathlineto{\pgfqpoint{1.185233in}{1.156439in}}%
\pgfpathlineto{\pgfqpoint{1.197043in}{1.166188in}}%
\pgfpathlineto{\pgfqpoint{1.210719in}{1.175233in}}%
\pgfpathlineto{\pgfqpoint{1.226238in}{1.183572in}}%
\pgfpathlineto{\pgfqpoint{1.252998in}{1.194758in}}%
\pgfpathlineto{\pgfqpoint{1.284086in}{1.204357in}}%
\pgfpathlineto{\pgfqpoint{1.319813in}{1.212375in}}%
\pgfpathlineto{\pgfqpoint{1.360457in}{1.218811in}}%
\pgfpathlineto{\pgfqpoint{1.406154in}{1.223638in}}%
\pgfpathlineto{\pgfqpoint{1.457377in}{1.226829in}}%
\pgfpathlineto{\pgfqpoint{1.514654in}{1.228344in}}%
\pgfpathlineto{\pgfqpoint{1.578554in}{1.228134in}}%
\pgfpathlineto{\pgfqpoint{1.649689in}{1.226137in}}%
\pgfpathlineto{\pgfqpoint{1.756934in}{1.220576in}}%
\pgfpathlineto{\pgfqpoint{1.879888in}{1.211519in}}%
\pgfpathlineto{\pgfqpoint{2.019979in}{1.198723in}}%
\pgfpathlineto{\pgfqpoint{2.177552in}{1.181922in}}%
\pgfpathlineto{\pgfqpoint{2.350568in}{1.160922in}}%
\pgfpathlineto{\pgfqpoint{2.487607in}{1.142364in}}%
\pgfpathlineto{\pgfqpoint{2.626725in}{1.121466in}}%
\pgfpathlineto{\pgfqpoint{2.763373in}{1.098403in}}%
\pgfpathlineto{\pgfqpoint{2.850758in}{1.081960in}}%
\pgfpathlineto{\pgfqpoint{2.933462in}{1.064790in}}%
\pgfpathlineto{\pgfqpoint{3.009992in}{1.047006in}}%
\pgfpathlineto{\pgfqpoint{3.079848in}{1.028765in}}%
\pgfpathlineto{\pgfqpoint{3.142866in}{1.010229in}}%
\pgfpathlineto{\pgfqpoint{3.198996in}{0.991543in}}%
\pgfpathlineto{\pgfqpoint{3.248306in}{0.972839in}}%
\pgfpathlineto{\pgfqpoint{3.290979in}{0.954237in}}%
\pgfpathlineto{\pgfqpoint{3.327311in}{0.935844in}}%
\pgfpathlineto{\pgfqpoint{3.357718in}{0.917752in}}%
\pgfpathlineto{\pgfqpoint{3.382729in}{0.900040in}}%
\pgfpathlineto{\pgfqpoint{3.402989in}{0.882774in}}%
\pgfpathlineto{\pgfqpoint{3.418940in}{0.866020in}}%
\pgfpathlineto{\pgfqpoint{3.430812in}{0.849838in}}%
\pgfpathlineto{\pgfqpoint{3.439384in}{0.834255in}}%
\pgfpathlineto{\pgfqpoint{3.445277in}{0.819293in}}%
\pgfpathlineto{\pgfqpoint{3.448933in}{0.804968in}}%
\pgfpathlineto{\pgfqpoint{3.450616in}{0.791295in}}%
\pgfpathlineto{\pgfqpoint{3.450412in}{0.778286in}}%
\pgfpathlineto{\pgfqpoint{3.448225in}{0.765946in}}%
\pgfpathlineto{\pgfqpoint{3.443786in}{0.754279in}}%
\pgfpathlineto{\pgfqpoint{3.436642in}{0.743286in}}%
\pgfpathlineto{\pgfqpoint{3.426459in}{0.732967in}}%
\pgfpathlineto{\pgfqpoint{3.414177in}{0.723349in}}%
\pgfpathlineto{\pgfqpoint{3.399992in}{0.714435in}}%
\pgfpathlineto{\pgfqpoint{3.383934in}{0.706227in}}%
\pgfpathlineto{\pgfqpoint{3.356336in}{0.695242in}}%
\pgfpathlineto{\pgfqpoint{3.324431in}{0.685850in}}%
\pgfpathlineto{\pgfqpoint{3.287994in}{0.678054in}}%
\pgfpathlineto{\pgfqpoint{3.246672in}{0.671855in}}%
\pgfpathlineto{\pgfqpoint{3.200135in}{0.667260in}}%
\pgfpathlineto{\pgfqpoint{3.148153in}{0.664301in}}%
\pgfpathlineto{\pgfqpoint{3.090046in}{0.663022in}}%
\pgfpathlineto{\pgfqpoint{3.025134in}{0.663475in}}%
\pgfpathlineto{\pgfqpoint{2.926882in}{0.666885in}}%
\pgfpathlineto{\pgfqpoint{2.813914in}{0.673669in}}%
\pgfpathlineto{\pgfqpoint{2.684846in}{0.684042in}}%
\pgfpathlineto{\pgfqpoint{2.538361in}{0.698247in}}%
\pgfpathlineto{\pgfqpoint{2.374256in}{0.716509in}}%
\pgfpathlineto{\pgfqpoint{2.241948in}{0.733022in}}%
\pgfpathlineto{\pgfqpoint{2.104698in}{0.751962in}}%
\pgfpathlineto{\pgfqpoint{1.965968in}{0.773230in}}%
\pgfpathlineto{\pgfqpoint{1.829949in}{0.796625in}}%
\pgfpathlineto{\pgfqpoint{1.743174in}{0.813255in}}%
\pgfpathlineto{\pgfqpoint{1.661413in}{0.830580in}}%
\pgfpathlineto{\pgfqpoint{1.586253in}{0.848477in}}%
\pgfpathlineto{\pgfqpoint{1.518084in}{0.866797in}}%
\pgfpathlineto{\pgfqpoint{1.456834in}{0.885384in}}%
\pgfpathlineto{\pgfqpoint{1.402377in}{0.904095in}}%
\pgfpathlineto{\pgfqpoint{1.354529in}{0.922799in}}%
\pgfpathlineto{\pgfqpoint{1.313050in}{0.941381in}}%
\pgfpathlineto{\pgfqpoint{1.277646in}{0.959735in}}%
\pgfpathlineto{\pgfqpoint{1.247964in}{0.977771in}}%
\pgfpathlineto{\pgfqpoint{1.223596in}{0.995411in}}%
\pgfpathlineto{\pgfqpoint{1.204081in}{1.012592in}}%
\pgfpathlineto{\pgfqpoint{1.188964in}{1.029256in}}%
\pgfpathlineto{\pgfqpoint{1.177996in}{1.045336in}}%
\pgfpathlineto{\pgfqpoint{1.170417in}{1.060804in}}%
\pgfpathlineto{\pgfqpoint{1.165540in}{1.075640in}}%
\pgfpathlineto{\pgfqpoint{1.162852in}{1.089829in}}%
\pgfpathlineto{\pgfqpoint{1.162017in}{1.103360in}}%
\pgfpathlineto{\pgfqpoint{1.162874in}{1.116223in}}%
\pgfpathlineto{\pgfqpoint{1.165440in}{1.128414in}}%
\pgfpathlineto{\pgfqpoint{1.169907in}{1.139929in}}%
\pgfpathlineto{\pgfqpoint{1.176644in}{1.150771in}}%
\pgfpathlineto{\pgfqpoint{1.186194in}{1.160942in}}%
\pgfpathlineto{\pgfqpoint{1.198837in}{1.170441in}}%
\pgfpathlineto{\pgfqpoint{1.213503in}{1.179235in}}%
\pgfpathlineto{\pgfqpoint{1.230058in}{1.187323in}}%
\pgfpathlineto{\pgfqpoint{1.258432in}{1.198125in}}%
\pgfpathlineto{\pgfqpoint{1.291142in}{1.207328in}}%
\pgfpathlineto{\pgfqpoint{1.328387in}{1.214927in}}%
\pgfpathlineto{\pgfqpoint{1.370484in}{1.220917in}}%
\pgfpathlineto{\pgfqpoint{1.417866in}{1.225289in}}%
\pgfpathlineto{\pgfqpoint{1.470735in}{1.228025in}}%
\pgfpathlineto{\pgfqpoint{1.529654in}{1.229078in}}%
\pgfpathlineto{\pgfqpoint{1.595491in}{1.228390in}}%
\pgfpathlineto{\pgfqpoint{1.695326in}{1.224643in}}%
\pgfpathlineto{\pgfqpoint{1.810276in}{1.217491in}}%
\pgfpathlineto{\pgfqpoint{1.941479in}{1.206721in}}%
\pgfpathlineto{\pgfqpoint{2.089724in}{1.192090in}}%
\pgfpathlineto{\pgfqpoint{2.256010in}{1.173315in}}%
\pgfpathlineto{\pgfqpoint{2.391180in}{1.156376in}}%
\pgfpathlineto{\pgfqpoint{2.530072in}{1.137043in}}%
\pgfpathlineto{\pgfqpoint{2.668160in}{1.115455in}}%
\pgfpathlineto{\pgfqpoint{2.801405in}{1.091828in}}%
\pgfpathlineto{\pgfqpoint{2.885769in}{1.075083in}}%
\pgfpathlineto{\pgfqpoint{2.965456in}{1.057664in}}%
\pgfpathlineto{\pgfqpoint{3.039629in}{1.039690in}}%
\pgfpathlineto{\pgfqpoint{3.107544in}{1.021295in}}%
\pgfpathlineto{\pgfqpoint{3.168556in}{1.002629in}}%
\pgfpathlineto{\pgfqpoint{3.222113in}{0.983857in}}%
\pgfpathlineto{\pgfqpoint{3.268108in}{0.965137in}}%
\pgfpathlineto{\pgfqpoint{3.307501in}{0.946569in}}%
\pgfpathlineto{\pgfqpoint{3.340920in}{0.928251in}}%
\pgfpathlineto{\pgfqpoint{3.368908in}{0.910276in}}%
\pgfpathlineto{\pgfqpoint{3.391956in}{0.892719in}}%
\pgfpathlineto{\pgfqpoint{3.410500in}{0.875649in}}%
\pgfpathlineto{\pgfqpoint{3.425008in}{0.859116in}}%
\pgfpathlineto{\pgfqpoint{3.435956in}{0.843163in}}%
\pgfpathlineto{\pgfqpoint{3.443718in}{0.827819in}}%
\pgfpathlineto{\pgfqpoint{3.448588in}{0.813110in}}%
\pgfpathlineto{\pgfqpoint{3.450786in}{0.799057in}}%
\pgfpathlineto{\pgfqpoint{3.450451in}{0.785672in}}%
\pgfpathlineto{\pgfqpoint{3.447677in}{0.772966in}}%
\pgfpathlineto{\pgfqpoint{3.442679in}{0.760947in}}%
\pgfpathlineto{\pgfqpoint{3.435643in}{0.749619in}}%
\pgfpathlineto{\pgfqpoint{3.426704in}{0.738988in}}%
\pgfpathlineto{\pgfqpoint{3.415946in}{0.729056in}}%
\pgfpathlineto{\pgfqpoint{3.403396in}{0.719824in}}%
\pgfpathlineto{\pgfqpoint{3.389035in}{0.711291in}}%
\pgfpathlineto{\pgfqpoint{3.372786in}{0.703456in}}%
\pgfpathlineto{\pgfqpoint{3.344583in}{0.692999in}}%
\pgfpathlineto{\pgfqpoint{3.311477in}{0.684094in}}%
\pgfpathlineto{\pgfqpoint{3.273699in}{0.676767in}}%
\pgfpathlineto{\pgfqpoint{3.231002in}{0.671038in}}%
\pgfpathlineto{\pgfqpoint{3.183031in}{0.666932in}}%
\pgfpathlineto{\pgfqpoint{3.129349in}{0.664482in}}%
\pgfpathlineto{\pgfqpoint{3.069438in}{0.663728in}}%
\pgfpathlineto{\pgfqpoint{3.002699in}{0.664716in}}%
\pgfpathlineto{\pgfqpoint{2.901921in}{0.668840in}}%
\pgfpathlineto{\pgfqpoint{2.786195in}{0.676349in}}%
\pgfpathlineto{\pgfqpoint{2.653606in}{0.687505in}}%
\pgfpathlineto{\pgfqpoint{2.503546in}{0.702554in}}%
\pgfpathlineto{\pgfqpoint{2.337121in}{0.721708in}}%
\pgfpathlineto{\pgfqpoint{2.157261in}{0.745138in}}%
\pgfpathlineto{\pgfqpoint{2.018321in}{0.765478in}}%
\pgfpathlineto{\pgfqpoint{1.881146in}{0.788017in}}%
\pgfpathlineto{\pgfqpoint{1.750140in}{0.812496in}}%
\pgfpathlineto{\pgfqpoint{1.668265in}{0.829725in}}%
\pgfpathlineto{\pgfqpoint{1.592094in}{0.847538in}}%
\pgfpathlineto{\pgfqpoint{1.522704in}{0.865790in}}%
\pgfpathlineto{\pgfqpoint{1.460752in}{0.884326in}}%
\pgfpathlineto{\pgfqpoint{1.405994in}{0.903004in}}%
\pgfpathlineto{\pgfqpoint{1.358045in}{0.921693in}}%
\pgfpathlineto{\pgfqpoint{1.316519in}{0.940275in}}%
\pgfpathlineto{\pgfqpoint{1.281030in}{0.958642in}}%
\pgfpathlineto{\pgfqpoint{1.251186in}{0.976699in}}%
\pgfpathlineto{\pgfqpoint{1.226594in}{0.994362in}}%
\pgfpathlineto{\pgfqpoint{1.206857in}{1.011559in}}%
\pgfpathlineto{\pgfqpoint{1.191414in}{1.028227in}}%
\pgfpathlineto{\pgfqpoint{1.179646in}{1.044327in}}%
\pgfpathlineto{\pgfqpoint{1.171065in}{1.059828in}}%
\pgfpathlineto{\pgfqpoint{1.165304in}{1.074703in}}%
\pgfpathlineto{\pgfqpoint{1.162115in}{1.088931in}}%
\pgfpathlineto{\pgfqpoint{1.161373in}{1.102497in}}%
\pgfpathlineto{\pgfqpoint{1.163073in}{1.115389in}}%
\pgfpathlineto{\pgfqpoint{1.167299in}{1.127601in}}%
\pgfpathlineto{\pgfqpoint{1.173794in}{1.139125in}}%
\pgfpathlineto{\pgfqpoint{1.182340in}{1.149954in}}%
\pgfpathlineto{\pgfqpoint{1.192828in}{1.160086in}}%
\pgfpathlineto{\pgfqpoint{1.205182in}{1.169517in}}%
\pgfpathlineto{\pgfqpoint{1.219367in}{1.178247in}}%
\pgfpathlineto{\pgfqpoint{1.235382in}{1.186274in}}%
\pgfpathlineto{\pgfqpoint{1.262931in}{1.197001in}}%
\pgfpathlineto{\pgfqpoint{1.294978in}{1.206157in}}%
\pgfpathlineto{\pgfqpoint{1.331900in}{1.213748in}}%
\pgfpathlineto{\pgfqpoint{1.373696in}{1.219752in}}%
\pgfpathlineto{\pgfqpoint{1.420697in}{1.224144in}}%
\pgfpathlineto{\pgfqpoint{1.473357in}{1.226892in}}%
\pgfpathlineto{\pgfqpoint{1.532190in}{1.227953in}}%
\pgfpathlineto{\pgfqpoint{1.597767in}{1.227277in}}%
\pgfpathlineto{\pgfqpoint{1.696799in}{1.223567in}}%
\pgfpathlineto{\pgfqpoint{1.810615in}{1.216489in}}%
\pgfpathlineto{\pgfqpoint{1.940850in}{1.205820in}}%
\pgfpathlineto{\pgfqpoint{2.088576in}{1.191292in}}%
\pgfpathlineto{\pgfqpoint{2.253114in}{1.172673in}}%
\pgfpathlineto{\pgfqpoint{2.431384in}{1.149804in}}%
\pgfpathlineto{\pgfqpoint{2.569894in}{1.129875in}}%
\pgfpathlineto{\pgfqpoint{2.707976in}{1.107700in}}%
\pgfpathlineto{\pgfqpoint{2.840821in}{1.083515in}}%
\pgfpathlineto{\pgfqpoint{2.923843in}{1.066420in}}%
\pgfpathlineto{\pgfqpoint{3.001138in}{1.048704in}}%
\pgfpathlineto{\pgfqpoint{3.072068in}{1.030530in}}%
\pgfpathlineto{\pgfqpoint{3.136196in}{1.012052in}}%
\pgfpathlineto{\pgfqpoint{3.193287in}{0.993410in}}%
\pgfpathlineto{\pgfqpoint{3.243307in}{0.974733in}}%
\pgfpathlineto{\pgfqpoint{3.286422in}{0.956141in}}%
\pgfpathlineto{\pgfqpoint{3.323002in}{0.937738in}}%
\pgfpathlineto{\pgfqpoint{3.353616in}{0.919620in}}%
\pgfpathlineto{\pgfqpoint{3.379035in}{0.901870in}}%
\pgfpathlineto{\pgfqpoint{3.399682in}{0.884573in}}%
\pgfpathlineto{\pgfqpoint{3.415853in}{0.867796in}}%
\pgfpathlineto{\pgfqpoint{3.428391in}{0.851573in}}%
\pgfpathlineto{\pgfqpoint{3.437946in}{0.835932in}}%
\pgfpathlineto{\pgfqpoint{3.444966in}{0.820899in}}%
\pgfpathlineto{\pgfqpoint{3.449700in}{0.806495in}}%
\pgfpathlineto{\pgfqpoint{3.452199in}{0.792737in}}%
\pgfpathlineto{\pgfqpoint{3.452315in}{0.779638in}}%
\pgfpathlineto{\pgfqpoint{3.449698in}{0.767206in}}%
\pgfpathlineto{\pgfqpoint{3.443922in}{0.755447in}}%
\pgfpathlineto{\pgfqpoint{3.435788in}{0.744382in}}%
\pgfpathlineto{\pgfqpoint{3.425672in}{0.734019in}}%
\pgfpathlineto{\pgfqpoint{3.413653in}{0.724358in}}%
\pgfpathlineto{\pgfqpoint{3.399780in}{0.715403in}}%
\pgfpathlineto{\pgfqpoint{3.384067in}{0.707153in}}%
\pgfpathlineto{\pgfqpoint{3.357004in}{0.696101in}}%
\pgfpathlineto{\pgfqpoint{3.325569in}{0.686632in}}%
\pgfpathlineto{\pgfqpoint{3.289413in}{0.678740in}}%
\pgfpathlineto{\pgfqpoint{3.248388in}{0.672434in}}%
\pgfpathlineto{\pgfqpoint{3.202257in}{0.667739in}}%
\pgfpathlineto{\pgfqpoint{3.150553in}{0.664684in}}%
\pgfpathlineto{\pgfqpoint{3.092760in}{0.663309in}}%
\pgfpathlineto{\pgfqpoint{3.028311in}{0.663665in}}%
\pgfpathlineto{\pgfqpoint{2.930948in}{0.666937in}}%
\pgfpathlineto{\pgfqpoint{2.819034in}{0.673563in}}%
\pgfpathlineto{\pgfqpoint{2.690886in}{0.683749in}}%
\pgfpathlineto{\pgfqpoint{2.545306in}{0.697764in}}%
\pgfpathlineto{\pgfqpoint{2.382632in}{0.715850in}}%
\pgfpathlineto{\pgfqpoint{2.205636in}{0.738178in}}%
\pgfpathlineto{\pgfqpoint{2.067277in}{0.757717in}}%
\pgfpathlineto{\pgfqpoint{1.928619in}{0.779537in}}%
\pgfpathlineto{\pgfqpoint{1.794326in}{0.803418in}}%
\pgfpathlineto{\pgfqpoint{1.709785in}{0.820333in}}%
\pgfpathlineto{\pgfqpoint{1.630853in}{0.837918in}}%
\pgfpathlineto{\pgfqpoint{1.558200in}{0.856007in}}%
\pgfpathlineto{\pgfqpoint{1.492274in}{0.874442in}}%
\pgfpathlineto{\pgfqpoint{1.433335in}{0.893078in}}%
\pgfpathlineto{\pgfqpoint{1.381455in}{0.911779in}}%
\pgfpathlineto{\pgfqpoint{1.336518in}{0.930425in}}%
\pgfpathlineto{\pgfqpoint{1.298219in}{0.948905in}}%
\pgfpathlineto{\pgfqpoint{1.266067in}{0.967121in}}%
\pgfpathlineto{\pgfqpoint{1.239380in}{0.984988in}}%
\pgfpathlineto{\pgfqpoint{1.217427in}{1.002428in}}%
\pgfpathlineto{\pgfqpoint{1.200141in}{1.019360in}}%
\pgfpathlineto{\pgfqpoint{1.186776in}{1.035745in}}%
\pgfpathlineto{\pgfqpoint{1.176601in}{1.051551in}}%
\pgfpathlineto{\pgfqpoint{1.169079in}{1.066754in}}%
\pgfpathlineto{\pgfqpoint{1.163874in}{1.081332in}}%
\pgfpathlineto{\pgfqpoint{1.160842in}{1.095267in}}%
\pgfpathlineto{\pgfqpoint{1.160038in}{1.108546in}}%
\pgfpathlineto{\pgfqpoint{1.161712in}{1.121159in}}%
\pgfpathlineto{\pgfqpoint{1.166312in}{1.133102in}}%
\pgfpathlineto{\pgfqpoint{1.173868in}{1.144363in}}%
\pgfpathlineto{\pgfqpoint{1.183470in}{1.154924in}}%
\pgfpathlineto{\pgfqpoint{1.195003in}{1.164782in}}%
\pgfpathlineto{\pgfqpoint{1.208407in}{1.173934in}}%
\pgfpathlineto{\pgfqpoint{1.223656in}{1.182381in}}%
\pgfpathlineto{\pgfqpoint{1.250003in}{1.193728in}}%
\pgfpathlineto{\pgfqpoint{1.280653in}{1.203488in}}%
\pgfpathlineto{\pgfqpoint{1.315903in}{1.211666in}}%
\pgfpathlineto{\pgfqpoint{1.356061in}{1.218262in}}%
\pgfpathlineto{\pgfqpoint{1.401238in}{1.223253in}}%
\pgfpathlineto{\pgfqpoint{1.451890in}{1.226610in}}%
\pgfpathlineto{\pgfqpoint{1.508541in}{1.228296in}}%
\pgfpathlineto{\pgfqpoint{1.571755in}{1.228261in}}%
\pgfpathlineto{\pgfqpoint{1.642143in}{1.226446in}}%
\pgfpathlineto{\pgfqpoint{1.748284in}{1.221139in}}%
\pgfpathlineto{\pgfqpoint{1.870009in}{1.212352in}}%
\pgfpathlineto{\pgfqpoint{2.008790in}{1.199846in}}%
\pgfpathlineto{\pgfqpoint{2.165098in}{1.183353in}}%
\pgfpathlineto{\pgfqpoint{2.337100in}{1.162672in}}%
\pgfpathlineto{\pgfqpoint{2.473700in}{1.144352in}}%
\pgfpathlineto{\pgfqpoint{2.612822in}{1.123681in}}%
\pgfpathlineto{\pgfqpoint{2.749937in}{1.100823in}}%
\pgfpathlineto{\pgfqpoint{2.880311in}{1.076050in}}%
\pgfpathlineto{\pgfqpoint{2.960990in}{1.058649in}}%
\pgfpathlineto{\pgfqpoint{3.035215in}{1.040685in}}%
\pgfpathlineto{\pgfqpoint{3.102696in}{1.022322in}}%
\pgfpathlineto{\pgfqpoint{3.163307in}{1.003716in}}%
\pgfpathlineto{\pgfqpoint{3.217040in}{0.985008in}}%
\pgfpathlineto{\pgfqpoint{3.264001in}{0.966327in}}%
\pgfpathlineto{\pgfqpoint{3.304414in}{0.947786in}}%
\pgfpathlineto{\pgfqpoint{3.338617in}{0.929487in}}%
\pgfpathlineto{\pgfqpoint{3.367067in}{0.911518in}}%
\pgfpathlineto{\pgfqpoint{3.390334in}{0.893955in}}%
\pgfpathlineto{\pgfqpoint{3.409092in}{0.876858in}}%
\pgfpathlineto{\pgfqpoint{3.423525in}{0.860300in}}%
\pgfpathlineto{\pgfqpoint{3.434161in}{0.844324in}}%
\pgfpathlineto{\pgfqpoint{3.441737in}{0.828956in}}%
\pgfpathlineto{\pgfqpoint{3.446809in}{0.814214in}}%
\pgfpathlineto{\pgfqpoint{3.449758in}{0.800115in}}%
\pgfpathlineto{\pgfqpoint{3.450784in}{0.786673in}}%
\pgfpathlineto{\pgfqpoint{3.449908in}{0.773897in}}%
\pgfpathlineto{\pgfqpoint{3.446972in}{0.761792in}}%
\pgfpathlineto{\pgfqpoint{3.441642in}{0.750360in}}%
\pgfpathlineto{\pgfqpoint{3.433402in}{0.739600in}}%
\pgfpathlineto{\pgfqpoint{3.422389in}{0.729524in}}%
\pgfpathlineto{\pgfqpoint{3.409436in}{0.720152in}}%
\pgfpathlineto{\pgfqpoint{3.394595in}{0.711485in}}%
\pgfpathlineto{\pgfqpoint{3.368827in}{0.699809in}}%
\pgfpathlineto{\pgfqpoint{3.338810in}{0.689726in}}%
\pgfpathlineto{\pgfqpoint{3.304375in}{0.681237in}}%
\pgfpathlineto{\pgfqpoint{3.265223in}{0.674346in}}%
\pgfpathlineto{\pgfqpoint{3.220952in}{0.669052in}}%
\pgfpathlineto{\pgfqpoint{3.171409in}{0.665379in}}%
\pgfpathlineto{\pgfqpoint{3.116041in}{0.663366in}}%
\pgfpathlineto{\pgfqpoint{3.054160in}{0.663061in}}%
\pgfpathlineto{\pgfqpoint{2.985105in}{0.664522in}}%
\pgfpathlineto{\pgfqpoint{2.880777in}{0.669342in}}%
\pgfpathlineto{\pgfqpoint{2.761135in}{0.677622in}}%
\pgfpathlineto{\pgfqpoint{2.624830in}{0.689593in}}%
\pgfpathlineto{\pgfqpoint{2.470656in}{0.705490in}}%
\pgfpathlineto{\pgfqpoint{2.300132in}{0.725562in}}%
\pgfpathlineto{\pgfqpoint{2.164643in}{0.743456in}}%
\pgfpathlineto{\pgfqpoint{2.026065in}{0.763733in}}%
\pgfpathlineto{\pgfqpoint{1.888286in}{0.786236in}}%
\pgfpathlineto{\pgfqpoint{1.755931in}{0.810704in}}%
\pgfpathlineto{\pgfqpoint{1.673330in}{0.827932in}}%
\pgfpathlineto{\pgfqpoint{1.597142in}{0.845752in}}%
\pgfpathlineto{\pgfqpoint{1.527889in}{0.864021in}}%
\pgfpathlineto{\pgfqpoint{1.465552in}{0.882578in}}%
\pgfpathlineto{\pgfqpoint{1.410042in}{0.901280in}}%
\pgfpathlineto{\pgfqpoint{1.361206in}{0.919993in}}%
\pgfpathlineto{\pgfqpoint{1.318820in}{0.938599in}}%
\pgfpathlineto{\pgfqpoint{1.282598in}{0.956992in}}%
\pgfpathlineto{\pgfqpoint{1.252182in}{0.975079in}}%
\pgfpathlineto{\pgfqpoint{1.227151in}{0.992781in}}%
\pgfpathlineto{\pgfqpoint{1.207015in}{1.010032in}}%
\pgfpathlineto{\pgfqpoint{1.191284in}{1.026774in}}%
\pgfpathlineto{\pgfqpoint{1.179748in}{1.042939in}}%
\pgfpathlineto{\pgfqpoint{1.171663in}{1.058495in}}%
\pgfpathlineto{\pgfqpoint{1.166330in}{1.073424in}}%
\pgfpathlineto{\pgfqpoint{1.163228in}{1.087711in}}%
\pgfpathlineto{\pgfqpoint{1.162015in}{1.101341in}}%
\pgfpathlineto{\pgfqpoint{1.162526in}{1.114306in}}%
\pgfpathlineto{\pgfqpoint{1.164776in}{1.126599in}}%
\pgfpathlineto{\pgfqpoint{1.168954in}{1.138219in}}%
\pgfpathlineto{\pgfqpoint{1.175432in}{1.149164in}}%
\pgfpathlineto{\pgfqpoint{1.184756in}{1.159440in}}%
\pgfpathlineto{\pgfqpoint{1.197135in}{1.169041in}}%
\pgfpathlineto{\pgfqpoint{1.211508in}{1.177937in}}%
\pgfpathlineto{\pgfqpoint{1.227769in}{1.186126in}}%
\pgfpathlineto{\pgfqpoint{1.255694in}{1.197081in}}%
\pgfpathlineto{\pgfqpoint{1.287937in}{1.206439in}}%
\pgfpathlineto{\pgfqpoint{1.324692in}{1.214194in}}%
\pgfpathlineto{\pgfqpoint{1.366274in}{1.220341in}}%
\pgfpathlineto{\pgfqpoint{1.413110in}{1.224875in}}%
\pgfpathlineto{\pgfqpoint{1.465389in}{1.227775in}}%
\pgfpathlineto{\pgfqpoint{1.523686in}{1.228996in}}%
\pgfpathlineto{\pgfqpoint{1.588828in}{1.228482in}}%
\pgfpathlineto{\pgfqpoint{1.687592in}{1.224980in}}%
\pgfpathlineto{\pgfqpoint{1.801314in}{1.218090in}}%
\pgfpathlineto{\pgfqpoint{1.931190in}{1.207600in}}%
\pgfpathlineto{\pgfqpoint{2.078127in}{1.193267in}}%
\pgfpathlineto{\pgfqpoint{2.243631in}{1.174794in}}%
\pgfpathlineto{\pgfqpoint{2.378590in}{1.158078in}}%
\pgfpathlineto{\pgfqpoint{2.517419in}{1.138966in}}%
\pgfpathlineto{\pgfqpoint{2.655571in}{1.117595in}}%
\pgfpathlineto{\pgfqpoint{2.789030in}{1.094171in}}%
\pgfpathlineto{\pgfqpoint{2.873653in}{1.077549in}}%
\pgfpathlineto{\pgfqpoint{2.953730in}{1.060237in}}%
\pgfpathlineto{\pgfqpoint{3.028466in}{1.042350in}}%
\pgfpathlineto{\pgfqpoint{3.097169in}{1.024016in}}%
\pgfpathlineto{\pgfqpoint{3.159255in}{1.005378in}}%
\pgfpathlineto{\pgfqpoint{3.214243in}{0.986594in}}%
\pgfpathlineto{\pgfqpoint{3.261758in}{0.967833in}}%
\pgfpathlineto{\pgfqpoint{3.302106in}{0.949236in}}%
\pgfpathlineto{\pgfqpoint{3.336337in}{0.930879in}}%
\pgfpathlineto{\pgfqpoint{3.365033in}{0.912852in}}%
\pgfpathlineto{\pgfqpoint{3.388719in}{0.895235in}}%
\pgfpathlineto{\pgfqpoint{3.407874in}{0.878094in}}%
\pgfpathlineto{\pgfqpoint{3.422959in}{0.861484in}}%
\pgfpathlineto{\pgfqpoint{3.434425in}{0.845446in}}%
\pgfpathlineto{\pgfqpoint{3.442643in}{0.830014in}}%
\pgfpathlineto{\pgfqpoint{3.447912in}{0.815215in}}%
\pgfpathlineto{\pgfqpoint{3.450463in}{0.801068in}}%
\pgfpathlineto{\pgfqpoint{3.450402in}{0.787587in}}%
\pgfpathlineto{\pgfqpoint{3.447912in}{0.774784in}}%
\pgfpathlineto{\pgfqpoint{3.443240in}{0.762667in}}%
\pgfpathlineto{\pgfqpoint{3.436573in}{0.751241in}}%
\pgfpathlineto{\pgfqpoint{3.428037in}{0.740512in}}%
\pgfpathlineto{\pgfqpoint{3.417700in}{0.730481in}}%
\pgfpathlineto{\pgfqpoint{3.405570in}{0.721149in}}%
\pgfpathlineto{\pgfqpoint{3.391593in}{0.712514in}}%
\pgfpathlineto{\pgfqpoint{3.375657in}{0.704572in}}%
\pgfpathlineto{\pgfqpoint{3.357590in}{0.697317in}}%
\pgfpathlineto{\pgfqpoint{3.326408in}{0.687726in}}%
\pgfpathlineto{\pgfqpoint{3.290649in}{0.679712in}}%
\pgfpathlineto{\pgfqpoint{3.250105in}{0.673291in}}%
\pgfpathlineto{\pgfqpoint{3.204461in}{0.668484in}}%
\pgfpathlineto{\pgfqpoint{3.153317in}{0.665320in}}%
\pgfpathlineto{\pgfqpoint{3.096185in}{0.663833in}}%
\pgfpathlineto{\pgfqpoint{3.032490in}{0.664063in}}%
\pgfpathlineto{\pgfqpoint{2.936276in}{0.667120in}}%
\pgfpathlineto{\pgfqpoint{2.825757in}{0.673489in}}%
\pgfpathlineto{\pgfqpoint{2.698620in}{0.683439in}}%
\pgfpathlineto{\pgfqpoint{2.553949in}{0.697219in}}%
\pgfpathlineto{\pgfqpoint{2.392545in}{0.715033in}}%
\pgfpathlineto{\pgfqpoint{2.216928in}{0.737051in}}%
\pgfpathlineto{\pgfqpoint{2.078551in}{0.756417in}}%
\pgfpathlineto{\pgfqpoint{1.939519in}{0.778102in}}%
\pgfpathlineto{\pgfqpoint{1.805102in}{0.801815in}}%
\pgfpathlineto{\pgfqpoint{1.720232in}{0.818587in}}%
\pgfpathlineto{\pgfqpoint{1.640409in}{0.836004in}}%
\pgfpathlineto{\pgfqpoint{1.566500in}{0.853947in}}%
\pgfpathlineto{\pgfqpoint{1.499187in}{0.872290in}}%
\pgfpathlineto{\pgfqpoint{1.438968in}{0.890898in}}%
\pgfpathlineto{\pgfqpoint{1.386154in}{0.909628in}}%
\pgfpathlineto{\pgfqpoint{1.340826in}{0.928325in}}%
\pgfpathlineto{\pgfqpoint{1.302328in}{0.946848in}}%
\pgfpathlineto{\pgfqpoint{1.269678in}{0.965110in}}%
\pgfpathlineto{\pgfqpoint{1.242081in}{0.983031in}}%
\pgfpathlineto{\pgfqpoint{1.218940in}{1.000542in}}%
\pgfpathlineto{\pgfqpoint{1.199858in}{1.017580in}}%
\pgfpathlineto{\pgfqpoint{1.184635in}{1.034092in}}%
\pgfpathlineto{\pgfqpoint{1.173269in}{1.050029in}}%
\pgfpathlineto{\pgfqpoint{1.165847in}{1.065355in}}%
\pgfpathlineto{\pgfqpoint{1.161555in}{1.080035in}}%
\pgfpathlineto{\pgfqpoint{1.159900in}{1.094055in}}%
\pgfpathlineto{\pgfqpoint{1.160625in}{1.107400in}}%
\pgfpathlineto{\pgfqpoint{1.163541in}{1.120061in}}%
\pgfpathlineto{\pgfqpoint{1.168526in}{1.132031in}}%
\pgfpathlineto{\pgfqpoint{1.175527in}{1.143307in}}%
\pgfpathlineto{\pgfqpoint{1.184561in}{1.153886in}}%
\pgfpathlineto{\pgfqpoint{1.195704in}{1.163771in}}%
\pgfpathlineto{\pgfqpoint{1.208879in}{1.172961in}}%
\pgfpathlineto{\pgfqpoint{1.223983in}{1.181453in}}%
\pgfpathlineto{\pgfqpoint{1.250199in}{1.192876in}}%
\pgfpathlineto{\pgfqpoint{1.280702in}{1.202714in}}%
\pgfpathlineto{\pgfqpoint{1.315623in}{1.210960in}}%
\pgfpathlineto{\pgfqpoint{1.355223in}{1.217607in}}%
\pgfpathlineto{\pgfqpoint{1.399889in}{1.222647in}}%
\pgfpathlineto{\pgfqpoint{1.450125in}{1.226072in}}%
\pgfpathlineto{\pgfqpoint{1.506083in}{1.227856in}}%
\pgfpathlineto{\pgfqpoint{1.568451in}{1.227937in}}%
\pgfpathlineto{\pgfqpoint{1.638159in}{1.226245in}}%
\pgfpathlineto{\pgfqpoint{1.743824in}{1.221100in}}%
\pgfpathlineto{\pgfqpoint{1.865245in}{1.212470in}}%
\pgfpathlineto{\pgfqpoint{2.003268in}{1.200135in}}%
\pgfpathlineto{\pgfqpoint{2.158218in}{1.183852in}}%
\pgfpathlineto{\pgfqpoint{2.330218in}{1.163344in}}%
\pgfpathlineto{\pgfqpoint{2.467619in}{1.145100in}}%
\pgfpathlineto{\pgfqpoint{2.606663in}{1.124527in}}%
\pgfpathlineto{\pgfqpoint{2.742781in}{1.101816in}}%
\pgfpathlineto{\pgfqpoint{2.871978in}{1.077233in}}%
\pgfpathlineto{\pgfqpoint{2.952550in}{1.059964in}}%
\pgfpathlineto{\pgfqpoint{3.027641in}{1.042126in}}%
\pgfpathlineto{\pgfqpoint{3.096493in}{1.023848in}}%
\pgfpathlineto{\pgfqpoint{3.158460in}{1.005274in}}%
\pgfpathlineto{\pgfqpoint{3.213012in}{0.986561in}}%
\pgfpathlineto{\pgfqpoint{3.260073in}{0.967866in}}%
\pgfpathlineto{\pgfqpoint{3.300476in}{0.949302in}}%
\pgfpathlineto{\pgfqpoint{3.334882in}{0.930969in}}%
\pgfpathlineto{\pgfqpoint{3.363863in}{0.912960in}}%
\pgfpathlineto{\pgfqpoint{3.387899in}{0.895352in}}%
\pgfpathlineto{\pgfqpoint{3.407381in}{0.878215in}}%
\pgfpathlineto{\pgfqpoint{3.422617in}{0.861605in}}%
\pgfpathlineto{\pgfqpoint{3.434101in}{0.845569in}}%
\pgfpathlineto{\pgfqpoint{3.442321in}{0.830138in}}%
\pgfpathlineto{\pgfqpoint{3.447620in}{0.815337in}}%
\pgfpathlineto{\pgfqpoint{3.450264in}{0.801187in}}%
\pgfpathlineto{\pgfqpoint{3.450439in}{0.787704in}}%
\pgfpathlineto{\pgfqpoint{3.448251in}{0.774897in}}%
\pgfpathlineto{\pgfqpoint{3.443742in}{0.762775in}}%
\pgfpathlineto{\pgfqpoint{3.437032in}{0.751340in}}%
\pgfpathlineto{\pgfqpoint{3.428278in}{0.740599in}}%
\pgfpathlineto{\pgfqpoint{3.417599in}{0.730555in}}%
\pgfpathlineto{\pgfqpoint{3.405071in}{0.721211in}}%
\pgfpathlineto{\pgfqpoint{3.390732in}{0.712568in}}%
\pgfpathlineto{\pgfqpoint{3.365815in}{0.700918in}}%
\pgfpathlineto{\pgfqpoint{3.336634in}{0.690844in}}%
\pgfpathlineto{\pgfqpoint{3.302812in}{0.682336in}}%
\pgfpathlineto{\pgfqpoint{3.263968in}{0.675389in}}%
\pgfpathlineto{\pgfqpoint{3.220147in}{0.670037in}}%
\pgfpathlineto{\pgfqpoint{3.170949in}{0.666307in}}%
\pgfpathlineto{\pgfqpoint{3.115888in}{0.664235in}}%
\pgfpathlineto{\pgfqpoint{3.054421in}{0.663867in}}%
\pgfpathlineto{\pgfqpoint{2.985952in}{0.665259in}}%
\pgfpathlineto{\pgfqpoint{2.882639in}{0.669968in}}%
\pgfpathlineto{\pgfqpoint{2.764036in}{0.678107in}}%
\pgfpathlineto{\pgfqpoint{2.628607in}{0.689914in}}%
\pgfpathlineto{\pgfqpoint{2.475660in}{0.705656in}}%
\pgfpathlineto{\pgfqpoint{2.306534in}{0.725547in}}%
\pgfpathlineto{\pgfqpoint{2.171399in}{0.743265in}}%
\pgfpathlineto{\pgfqpoint{2.032801in}{0.763353in}}%
\pgfpathlineto{\pgfqpoint{1.894955in}{0.785674in}}%
\pgfpathlineto{\pgfqpoint{1.762711in}{0.809983in}}%
\pgfpathlineto{\pgfqpoint{1.680269in}{0.827148in}}%
\pgfpathlineto{\pgfqpoint{1.603624in}{0.844915in}}%
\pgfpathlineto{\pgfqpoint{1.533397in}{0.863123in}}%
\pgfpathlineto{\pgfqpoint{1.470008in}{0.881621in}}%
\pgfpathlineto{\pgfqpoint{1.413671in}{0.900269in}}%
\pgfpathlineto{\pgfqpoint{1.364400in}{0.918938in}}%
\pgfpathlineto{\pgfqpoint{1.322003in}{0.937513in}}%
\pgfpathlineto{\pgfqpoint{1.286087in}{0.955888in}}%
\pgfpathlineto{\pgfqpoint{1.256055in}{0.973969in}}%
\pgfpathlineto{\pgfqpoint{1.231112in}{0.991674in}}%
\pgfpathlineto{\pgfqpoint{1.211010in}{1.008914in}}%
\pgfpathlineto{\pgfqpoint{1.195259in}{1.025631in}}%
\pgfpathlineto{\pgfqpoint{1.183038in}{1.041793in}}%
\pgfpathlineto{\pgfqpoint{1.173727in}{1.057369in}}%
\pgfpathlineto{\pgfqpoint{1.166909in}{1.072336in}}%
\pgfpathlineto{\pgfqpoint{1.162367in}{1.086672in}}%
\pgfpathlineto{\pgfqpoint{1.160086in}{1.100361in}}%
\pgfpathlineto{\pgfqpoint{1.160250in}{1.113390in}}%
\pgfpathlineto{\pgfqpoint{1.163246in}{1.125752in}}%
\pgfpathlineto{\pgfqpoint{1.169384in}{1.137438in}}%
\pgfpathlineto{\pgfqpoint{1.177736in}{1.148428in}}%
\pgfpathlineto{\pgfqpoint{1.188058in}{1.158715in}}%
\pgfpathlineto{\pgfqpoint{1.200272in}{1.168299in}}%
\pgfpathlineto{\pgfqpoint{1.214337in}{1.177179in}}%
\pgfpathlineto{\pgfqpoint{1.230240in}{1.185352in}}%
\pgfpathlineto{\pgfqpoint{1.257599in}{1.196291in}}%
\pgfpathlineto{\pgfqpoint{1.289359in}{1.205647in}}%
\pgfpathlineto{\pgfqpoint{1.325873in}{1.213427in}}%
\pgfpathlineto{\pgfqpoint{1.367247in}{1.219619in}}%
\pgfpathlineto{\pgfqpoint{1.413764in}{1.224198in}}%
\pgfpathlineto{\pgfqpoint{1.465888in}{1.227134in}}%
\pgfpathlineto{\pgfqpoint{1.524137in}{1.228386in}}%
\pgfpathlineto{\pgfqpoint{1.589081in}{1.227904in}}%
\pgfpathlineto{\pgfqpoint{1.687174in}{1.224456in}}%
\pgfpathlineto{\pgfqpoint{1.799911in}{1.217646in}}%
\pgfpathlineto{\pgfqpoint{1.928968in}{1.207262in}}%
\pgfpathlineto{\pgfqpoint{2.075495in}{1.193035in}}%
\pgfpathlineto{\pgfqpoint{2.239021in}{1.174726in}}%
\pgfpathlineto{\pgfqpoint{2.416645in}{1.152169in}}%
\pgfpathlineto{\pgfqpoint{2.555157in}{1.132462in}}%
\pgfpathlineto{\pgfqpoint{2.693670in}{1.110487in}}%
\pgfpathlineto{\pgfqpoint{2.827461in}{1.086470in}}%
\pgfpathlineto{\pgfqpoint{2.911425in}{1.069473in}}%
\pgfpathlineto{\pgfqpoint{2.989723in}{1.051825in}}%
\pgfpathlineto{\pgfqpoint{3.061703in}{1.033692in}}%
\pgfpathlineto{\pgfqpoint{3.126920in}{1.015231in}}%
\pgfpathlineto{\pgfqpoint{3.185120in}{0.996584in}}%
\pgfpathlineto{\pgfqpoint{3.236245in}{0.977884in}}%
\pgfpathlineto{\pgfqpoint{3.280435in}{0.959252in}}%
\pgfpathlineto{\pgfqpoint{3.318020in}{0.940796in}}%
\pgfpathlineto{\pgfqpoint{3.349530in}{0.922612in}}%
\pgfpathlineto{\pgfqpoint{3.375686in}{0.904786in}}%
\pgfpathlineto{\pgfqpoint{3.397109in}{0.887398in}}%
\pgfpathlineto{\pgfqpoint{3.413916in}{0.870524in}}%
\pgfpathlineto{\pgfqpoint{3.426937in}{0.854200in}}%
\pgfpathlineto{\pgfqpoint{3.436868in}{0.838455in}}%
\pgfpathlineto{\pgfqpoint{3.444204in}{0.823317in}}%
\pgfpathlineto{\pgfqpoint{3.449242in}{0.808805in}}%
\pgfpathlineto{\pgfqpoint{3.452080in}{0.794937in}}%
\pgfpathlineto{\pgfqpoint{3.452618in}{0.781727in}}%
\pgfpathlineto{\pgfqpoint{3.450557in}{0.769182in}}%
\pgfpathlineto{\pgfqpoint{3.445406in}{0.757309in}}%
\pgfpathlineto{\pgfqpoint{3.437594in}{0.746125in}}%
\pgfpathlineto{\pgfqpoint{3.427780in}{0.735641in}}%
\pgfpathlineto{\pgfqpoint{3.416049in}{0.725862in}}%
\pgfpathlineto{\pgfqpoint{3.402456in}{0.716787in}}%
\pgfpathlineto{\pgfqpoint{3.387022in}{0.708418in}}%
\pgfpathlineto{\pgfqpoint{3.360392in}{0.697188in}}%
\pgfpathlineto{\pgfqpoint{3.329429in}{0.687544in}}%
\pgfpathlineto{\pgfqpoint{3.293812in}{0.679479in}}%
\pgfpathlineto{\pgfqpoint{3.253301in}{0.672997in}}%
\pgfpathlineto{\pgfqpoint{3.207741in}{0.668124in}}%
\pgfpathlineto{\pgfqpoint{3.156664in}{0.664886in}}%
\pgfpathlineto{\pgfqpoint{3.099554in}{0.663325in}}%
\pgfpathlineto{\pgfqpoint{3.035846in}{0.663488in}}%
\pgfpathlineto{\pgfqpoint{2.964927in}{0.665435in}}%
\pgfpathlineto{\pgfqpoint{2.858002in}{0.670927in}}%
\pgfpathlineto{\pgfqpoint{2.735387in}{0.679907in}}%
\pgfpathlineto{\pgfqpoint{2.595648in}{0.692623in}}%
\pgfpathlineto{\pgfqpoint{2.438430in}{0.709338in}}%
\pgfpathlineto{\pgfqpoint{2.265682in}{0.730250in}}%
\pgfpathlineto{\pgfqpoint{2.128770in}{0.748742in}}%
\pgfpathlineto{\pgfqpoint{1.989671in}{0.769577in}}%
\pgfpathlineto{\pgfqpoint{1.852844in}{0.792584in}}%
\pgfpathlineto{\pgfqpoint{1.765278in}{0.808994in}}%
\pgfpathlineto{\pgfqpoint{1.682457in}{0.826146in}}%
\pgfpathlineto{\pgfqpoint{1.605609in}{0.843918in}}%
\pgfpathlineto{\pgfqpoint{1.535257in}{0.862144in}}%
\pgfpathlineto{\pgfqpoint{1.471735in}{0.880668in}}%
\pgfpathlineto{\pgfqpoint{1.415205in}{0.899346in}}%
\pgfpathlineto{\pgfqpoint{1.365661in}{0.918047in}}%
\pgfpathlineto{\pgfqpoint{1.322931in}{0.936652in}}%
\pgfpathlineto{\pgfqpoint{1.286672in}{0.955054in}}%
\pgfpathlineto{\pgfqpoint{1.256373in}{0.973161in}}%
\pgfpathlineto{\pgfqpoint{1.231355in}{0.990892in}}%
\pgfpathlineto{\pgfqpoint{1.210923in}{1.008174in}}%
\pgfpathlineto{\pgfqpoint{1.195015in}{1.024931in}}%
\pgfpathlineto{\pgfqpoint{1.182885in}{1.041126in}}%
\pgfpathlineto{\pgfqpoint{1.173825in}{1.056731in}}%
\pgfpathlineto{\pgfqpoint{1.167318in}{1.071723in}}%
\pgfpathlineto{\pgfqpoint{1.163037in}{1.086081in}}%
\pgfpathlineto{\pgfqpoint{1.160846in}{1.099792in}}%
\pgfpathlineto{\pgfqpoint{1.160801in}{1.112842in}}%
\pgfpathlineto{\pgfqpoint{1.163149in}{1.125224in}}%
\pgfpathlineto{\pgfqpoint{1.168326in}{1.136935in}}%
\pgfpathlineto{\pgfqpoint{1.176515in}{1.147967in}}%
\pgfpathlineto{\pgfqpoint{1.186774in}{1.158298in}}%
\pgfpathlineto{\pgfqpoint{1.198956in}{1.167926in}}%
\pgfpathlineto{\pgfqpoint{1.213009in}{1.176849in}}%
\pgfpathlineto{\pgfqpoint{1.228912in}{1.185066in}}%
\pgfpathlineto{\pgfqpoint{1.256255in}{1.196066in}}%
\pgfpathlineto{\pgfqpoint{1.287926in}{1.205478in}}%
\pgfpathlineto{\pgfqpoint{1.324218in}{1.213303in}}%
\pgfpathlineto{\pgfqpoint{1.365474in}{1.219544in}}%
\pgfpathlineto{\pgfqpoint{1.411815in}{1.224176in}}%
\pgfpathlineto{\pgfqpoint{1.463728in}{1.227167in}}%
\pgfpathlineto{\pgfqpoint{1.521773in}{1.228479in}}%
\pgfpathlineto{\pgfqpoint{1.586538in}{1.228058in}}%
\pgfpathlineto{\pgfqpoint{1.684418in}{1.224695in}}%
\pgfpathlineto{\pgfqpoint{1.796892in}{1.217965in}}%
\pgfpathlineto{\pgfqpoint{1.925589in}{1.207658in}}%
\pgfpathlineto{\pgfqpoint{2.071699in}{1.193524in}}%
\pgfpathlineto{\pgfqpoint{2.234979in}{1.175300in}}%
\pgfpathlineto{\pgfqpoint{2.412329in}{1.152835in}}%
\pgfpathlineto{\pgfqpoint{2.550909in}{1.133194in}}%
\pgfpathlineto{\pgfqpoint{2.689658in}{1.111271in}}%
\pgfpathlineto{\pgfqpoint{2.823514in}{1.087306in}}%
\pgfpathlineto{\pgfqpoint{2.907794in}{1.070356in}}%
\pgfpathlineto{\pgfqpoint{2.986724in}{1.052765in}}%
\pgfpathlineto{\pgfqpoint{3.059253in}{1.034669in}}%
\pgfpathlineto{\pgfqpoint{3.124409in}{1.016224in}}%
\pgfpathlineto{\pgfqpoint{3.182266in}{0.997577in}}%
\pgfpathlineto{\pgfqpoint{3.233264in}{0.978862in}}%
\pgfpathlineto{\pgfqpoint{3.277812in}{0.960199in}}%
\pgfpathlineto{\pgfqpoint{3.316294in}{0.941700in}}%
\pgfpathlineto{\pgfqpoint{3.349068in}{0.923465in}}%
\pgfpathlineto{\pgfqpoint{3.376463in}{0.905581in}}%
\pgfpathlineto{\pgfqpoint{3.398781in}{0.888126in}}%
\pgfpathlineto{\pgfqpoint{3.416300in}{0.871167in}}%
\pgfpathlineto{\pgfqpoint{3.429529in}{0.854763in}}%
\pgfpathlineto{\pgfqpoint{3.439204in}{0.838955in}}%
\pgfpathlineto{\pgfqpoint{3.445871in}{0.823766in}}%
\pgfpathlineto{\pgfqpoint{3.449944in}{0.809218in}}%
\pgfpathlineto{\pgfqpoint{3.451706in}{0.795328in}}%
\pgfpathlineto{\pgfqpoint{3.451309in}{0.782107in}}%
\pgfpathlineto{\pgfqpoint{3.448772in}{0.769564in}}%
\pgfpathlineto{\pgfqpoint{3.443984in}{0.757705in}}%
\pgfpathlineto{\pgfqpoint{3.436702in}{0.746529in}}%
\pgfpathlineto{\pgfqpoint{3.427041in}{0.736043in}}%
\pgfpathlineto{\pgfqpoint{3.415410in}{0.726256in}}%
\pgfpathlineto{\pgfqpoint{3.401873in}{0.717172in}}%
\pgfpathlineto{\pgfqpoint{3.386470in}{0.708790in}}%
\pgfpathlineto{\pgfqpoint{3.359879in}{0.697539in}}%
\pgfpathlineto{\pgfqpoint{3.329014in}{0.687876in}}%
\pgfpathlineto{\pgfqpoint{3.293642in}{0.679800in}}%
\pgfpathlineto{\pgfqpoint{3.253387in}{0.673311in}}%
\pgfpathlineto{\pgfqpoint{3.207967in}{0.668416in}}%
\pgfpathlineto{\pgfqpoint{3.157145in}{0.665150in}}%
\pgfpathlineto{\pgfqpoint{3.100298in}{0.663553in}}%
\pgfpathlineto{\pgfqpoint{3.036799in}{0.663675in}}%
\pgfpathlineto{\pgfqpoint{2.966018in}{0.665577in}}%
\pgfpathlineto{\pgfqpoint{2.859234in}{0.671007in}}%
\pgfpathlineto{\pgfqpoint{2.736903in}{0.679932in}}%
\pgfpathlineto{\pgfqpoint{2.597561in}{0.692585in}}%
\pgfpathlineto{\pgfqpoint{2.440703in}{0.709215in}}%
\pgfpathlineto{\pgfqpoint{2.268208in}{0.730068in}}%
\pgfpathlineto{\pgfqpoint{2.131578in}{0.748512in}}%
\pgfpathlineto{\pgfqpoint{1.992508in}{0.769285in}}%
\pgfpathlineto{\pgfqpoint{1.855587in}{0.792243in}}%
\pgfpathlineto{\pgfqpoint{1.768022in}{0.808646in}}%
\pgfpathlineto{\pgfqpoint{1.685030in}{0.825776in}}%
\pgfpathlineto{\pgfqpoint{1.607691in}{0.843493in}}%
\pgfpathlineto{\pgfqpoint{1.536806in}{0.861662in}}%
\pgfpathlineto{\pgfqpoint{1.472898in}{0.880145in}}%
\pgfpathlineto{\pgfqpoint{1.416209in}{0.898811in}}%
\pgfpathlineto{\pgfqpoint{1.366702in}{0.917528in}}%
\pgfpathlineto{\pgfqpoint{1.324069in}{0.936166in}}%
\pgfpathlineto{\pgfqpoint{1.288197in}{0.954579in}}%
\pgfpathlineto{\pgfqpoint{1.258250in}{0.972675in}}%
\pgfpathlineto{\pgfqpoint{1.233148in}{0.990394in}}%
\pgfpathlineto{\pgfqpoint{1.212075in}{1.007682in}}%
\pgfpathlineto{\pgfqpoint{1.194481in}{1.024487in}}%
\pgfpathlineto{\pgfqpoint{1.180078in}{1.040764in}}%
\pgfpathlineto{\pgfqpoint{1.168841in}{1.056471in}}%
\pgfpathlineto{\pgfqpoint{1.161011in}{1.071572in}}%
\pgfpathlineto{\pgfqpoint{1.157091in}{1.086035in}}%
\pgfpathlineto{\pgfqpoint{1.156689in}{1.099829in}}%
\pgfpathlineto{\pgfqpoint{1.158671in}{1.112938in}}%
\pgfpathlineto{\pgfqpoint{1.162838in}{1.125353in}}%
\pgfpathlineto{\pgfqpoint{1.169039in}{1.137069in}}%
\pgfpathlineto{\pgfqpoint{1.177177in}{1.148084in}}%
\pgfpathlineto{\pgfqpoint{1.187204in}{1.158395in}}%
\pgfpathlineto{\pgfqpoint{1.199122in}{1.168002in}}%
\pgfpathlineto{\pgfqpoint{1.212987in}{1.176909in}}%
\pgfpathlineto{\pgfqpoint{1.228890in}{1.185118in}}%
\pgfpathlineto{\pgfqpoint{1.256447in}{1.196121in}}%
\pgfpathlineto{\pgfqpoint{1.288406in}{1.205543in}}%
\pgfpathlineto{\pgfqpoint{1.324899in}{1.213373in}}%
\pgfpathlineto{\pgfqpoint{1.366160in}{1.219594in}}%
\pgfpathlineto{\pgfqpoint{1.412528in}{1.224189in}}%
\pgfpathlineto{\pgfqpoint{1.464445in}{1.227136in}}%
\pgfpathlineto{\pgfqpoint{1.522454in}{1.228408in}}%
\pgfpathlineto{\pgfqpoint{1.587020in}{1.227987in}}%
\pgfpathlineto{\pgfqpoint{1.683966in}{1.224710in}}%
\pgfpathlineto{\pgfqpoint{1.796332in}{1.218034in}}%
\pgfpathlineto{\pgfqpoint{1.926078in}{1.207675in}}%
\pgfpathlineto{\pgfqpoint{2.073428in}{1.193414in}}%
\pgfpathlineto{\pgfqpoint{2.236868in}{1.175093in}}%
\pgfpathlineto{\pgfqpoint{2.413146in}{1.152618in}}%
\pgfpathlineto{\pgfqpoint{2.550845in}{1.133014in}}%
\pgfpathlineto{\pgfqpoint{2.689478in}{1.111073in}}%
\pgfpathlineto{\pgfqpoint{2.823328in}{1.087057in}}%
\pgfpathlineto{\pgfqpoint{2.907576in}{1.070099in}}%
\pgfpathlineto{\pgfqpoint{2.986544in}{1.052531in}}%
\pgfpathlineto{\pgfqpoint{3.059384in}{1.034478in}}%
\pgfpathlineto{\pgfqpoint{3.125479in}{1.016070in}}%
\pgfpathlineto{\pgfqpoint{3.184439in}{0.997436in}}%
\pgfpathlineto{\pgfqpoint{3.236105in}{0.978704in}}%
\pgfpathlineto{\pgfqpoint{3.280547in}{0.960009in}}%
\pgfpathlineto{\pgfqpoint{3.317953in}{0.941495in}}%
\pgfpathlineto{\pgfqpoint{3.348956in}{0.923296in}}%
\pgfpathlineto{\pgfqpoint{3.374802in}{0.905471in}}%
\pgfpathlineto{\pgfqpoint{3.396483in}{0.888067in}}%
\pgfpathlineto{\pgfqpoint{3.414706in}{0.871132in}}%
\pgfpathlineto{\pgfqpoint{3.429890in}{0.854711in}}%
\pgfpathlineto{\pgfqpoint{3.442167in}{0.838842in}}%
\pgfpathlineto{\pgfqpoint{3.451384in}{0.823564in}}%
\pgfpathlineto{\pgfqpoint{3.457099in}{0.808911in}}%
\pgfpathlineto{\pgfqpoint{3.458636in}{0.794914in}}%
\pgfpathlineto{\pgfqpoint{3.457024in}{0.781603in}}%
\pgfpathlineto{\pgfqpoint{3.453174in}{0.768988in}}%
\pgfpathlineto{\pgfqpoint{3.447249in}{0.757076in}}%
\pgfpathlineto{\pgfqpoint{3.439367in}{0.745869in}}%
\pgfpathlineto{\pgfqpoint{3.429599in}{0.735370in}}%
\pgfpathlineto{\pgfqpoint{3.417969in}{0.725577in}}%
\pgfpathlineto{\pgfqpoint{3.404459in}{0.716491in}}%
\pgfpathlineto{\pgfqpoint{3.389001in}{0.708109in}}%
\pgfpathlineto{\pgfqpoint{3.362033in}{0.696847in}}%
\pgfpathlineto{\pgfqpoint{3.330651in}{0.687166in}}%
\pgfpathlineto{\pgfqpoint{3.294753in}{0.679077in}}%
\pgfpathlineto{\pgfqpoint{3.254121in}{0.672595in}}%
\pgfpathlineto{\pgfqpoint{3.208438in}{0.667740in}}%
\pgfpathlineto{\pgfqpoint{3.157288in}{0.664534in}}%
\pgfpathlineto{\pgfqpoint{3.100156in}{0.663006in}}%
\pgfpathlineto{\pgfqpoint{3.036430in}{0.663188in}}%
\pgfpathlineto{\pgfqpoint{2.940699in}{0.666102in}}%
\pgfpathlineto{\pgfqpoint{2.830344in}{0.672317in}}%
\pgfpathlineto{\pgfqpoint{2.702649in}{0.682209in}}%
\pgfpathlineto{\pgfqpoint{2.556943in}{0.696049in}}%
\pgfpathlineto{\pgfqpoint{2.394612in}{0.714001in}}%
\pgfpathlineto{\pgfqpoint{2.219093in}{0.736124in}}%
\pgfpathlineto{\pgfqpoint{2.082026in}{0.755430in}}%
\pgfpathlineto{\pgfqpoint{1.943662in}{0.776998in}}%
\pgfpathlineto{\pgfqpoint{1.808716in}{0.800715in}}%
\pgfpathlineto{\pgfqpoint{1.723424in}{0.817538in}}%
\pgfpathlineto{\pgfqpoint{1.643266in}{0.835007in}}%
\pgfpathlineto{\pgfqpoint{1.569149in}{0.852985in}}%
\pgfpathlineto{\pgfqpoint{1.501727in}{0.871338in}}%
\pgfpathlineto{\pgfqpoint{1.441397in}{0.889935in}}%
\pgfpathlineto{\pgfqpoint{1.388302in}{0.908645in}}%
\pgfpathlineto{\pgfqpoint{1.342330in}{0.927343in}}%
\pgfpathlineto{\pgfqpoint{1.303113in}{0.945907in}}%
\pgfpathlineto{\pgfqpoint{1.270407in}{0.964194in}}%
\pgfpathlineto{\pgfqpoint{1.243390in}{0.982114in}}%
\pgfpathlineto{\pgfqpoint{1.220974in}{0.999615in}}%
\pgfpathlineto{\pgfqpoint{1.202337in}{1.016648in}}%
\pgfpathlineto{\pgfqpoint{1.186921in}{1.033170in}}%
\pgfpathlineto{\pgfqpoint{1.174434in}{1.049141in}}%
\pgfpathlineto{\pgfqpoint{1.164850in}{1.064525in}}%
\pgfpathlineto{\pgfqpoint{1.158406in}{1.079291in}}%
\pgfpathlineto{\pgfqpoint{1.155607in}{1.093412in}}%
\pgfpathlineto{\pgfqpoint{1.156470in}{1.106862in}}%
\pgfpathlineto{\pgfqpoint{1.159691in}{1.119620in}}%
\pgfpathlineto{\pgfqpoint{1.165042in}{1.131681in}}%
\pgfpathlineto{\pgfqpoint{1.172389in}{1.143039in}}%
\pgfpathlineto{\pgfqpoint{1.181642in}{1.153692in}}%
\pgfpathlineto{\pgfqpoint{1.192757in}{1.163640in}}%
\pgfpathlineto{\pgfqpoint{1.205735in}{1.172883in}}%
\pgfpathlineto{\pgfqpoint{1.220623in}{1.181423in}}%
\pgfpathlineto{\pgfqpoint{1.237514in}{1.189263in}}%
\pgfpathlineto{\pgfqpoint{1.266628in}{1.199715in}}%
\pgfpathlineto{\pgfqpoint{1.300200in}{1.208584in}}%
\pgfpathlineto{\pgfqpoint{1.338399in}{1.215858in}}%
\pgfpathlineto{\pgfqpoint{1.381492in}{1.221519in}}%
\pgfpathlineto{\pgfqpoint{1.429840in}{1.225545in}}%
\pgfpathlineto{\pgfqpoint{1.483901in}{1.227908in}}%
\pgfpathlineto{\pgfqpoint{1.544227in}{1.228575in}}%
\pgfpathlineto{\pgfqpoint{1.611453in}{1.227511in}}%
\pgfpathlineto{\pgfqpoint{1.712467in}{1.223319in}}%
\pgfpathlineto{\pgfqpoint{1.829035in}{1.215693in}}%
\pgfpathlineto{\pgfqpoint{1.963081in}{1.204350in}}%
\pgfpathlineto{\pgfqpoint{2.114674in}{1.189066in}}%
\pgfpathlineto{\pgfqpoint{2.282033in}{1.169680in}}%
\pgfpathlineto{\pgfqpoint{2.461523in}{1.146086in}}%
\pgfpathlineto{\pgfqpoint{2.600641in}{1.125595in}}%
\pgfpathlineto{\pgfqpoint{2.738320in}{1.102840in}}%
\pgfpathlineto{\pgfqpoint{2.869355in}{1.078185in}}%
\pgfpathlineto{\pgfqpoint{2.950947in}{1.060888in}}%
\pgfpathlineto{\pgfqpoint{3.026789in}{1.043043in}}%
\pgfpathlineto{\pgfqpoint{3.096151in}{1.024775in}}%
\pgfpathlineto{\pgfqpoint{3.158517in}{1.006213in}}%
\pgfpathlineto{\pgfqpoint{3.213589in}{0.987490in}}%
\pgfpathlineto{\pgfqpoint{3.261284in}{0.968741in}}%
\pgfpathlineto{\pgfqpoint{3.301746in}{0.950111in}}%
\pgfpathlineto{\pgfqpoint{3.335690in}{0.931739in}}%
\pgfpathlineto{\pgfqpoint{3.364175in}{0.913698in}}%
\pgfpathlineto{\pgfqpoint{3.388051in}{0.896056in}}%
\pgfpathlineto{\pgfqpoint{3.407939in}{0.878871in}}%
\pgfpathlineto{\pgfqpoint{3.424229in}{0.862195in}}%
\pgfpathlineto{\pgfqpoint{3.437081in}{0.846075in}}%
\pgfpathlineto{\pgfqpoint{3.446426in}{0.830550in}}%
\pgfpathlineto{\pgfqpoint{3.451968in}{0.815653in}}%
\pgfpathlineto{\pgfqpoint{3.454218in}{0.801415in}}%
\pgfpathlineto{\pgfqpoint{3.453961in}{0.787853in}}%
\pgfpathlineto{\pgfqpoint{3.451426in}{0.774976in}}%
\pgfpathlineto{\pgfqpoint{3.446787in}{0.762791in}}%
\pgfpathlineto{\pgfqpoint{3.440159in}{0.751304in}}%
\pgfpathlineto{\pgfqpoint{3.431599in}{0.740516in}}%
\pgfpathlineto{\pgfqpoint{3.421105in}{0.730429in}}%
\pgfpathlineto{\pgfqpoint{3.408617in}{0.721041in}}%
\pgfpathlineto{\pgfqpoint{3.394060in}{0.712348in}}%
\pgfpathlineto{\pgfqpoint{3.377545in}{0.704354in}}%
\pgfpathlineto{\pgfqpoint{3.349159in}{0.693676in}}%
\pgfpathlineto{\pgfqpoint{3.316385in}{0.684584in}}%
\pgfpathlineto{\pgfqpoint{3.279064in}{0.677089in}}%
\pgfpathlineto{\pgfqpoint{3.236924in}{0.671204in}}%
\pgfpathlineto{\pgfqpoint{3.189582in}{0.666945in}}%
\pgfpathlineto{\pgfqpoint{3.136543in}{0.664328in}}%
\pgfpathlineto{\pgfqpoint{3.077351in}{0.663372in}}%
\pgfpathlineto{\pgfqpoint{3.011803in}{0.664119in}}%
\pgfpathlineto{\pgfqpoint{2.912382in}{0.667937in}}%
\pgfpathlineto{\pgfqpoint{2.797146in}{0.675193in}}%
\pgfpathlineto{\pgfqpoint{2.664936in}{0.686110in}}%
\pgfpathlineto{\pgfqpoint{2.515767in}{0.700891in}}%
\pgfpathlineto{\pgfqpoint{2.350834in}{0.719730in}}%
\pgfpathlineto{\pgfqpoint{2.172511in}{0.742804in}}%
\pgfpathlineto{\pgfqpoint{2.033253in}{0.762952in}}%
\pgfpathlineto{\pgfqpoint{1.895083in}{0.785337in}}%
\pgfpathlineto{\pgfqpoint{1.763032in}{0.809658in}}%
\pgfpathlineto{\pgfqpoint{1.680455in}{0.826778in}}%
\pgfpathlineto{\pgfqpoint{1.603409in}{0.844490in}}%
\pgfpathlineto{\pgfqpoint{1.532688in}{0.862668in}}%
\pgfpathlineto{\pgfqpoint{1.468895in}{0.881179in}}%
\pgfpathlineto{\pgfqpoint{1.412447in}{0.899879in}}%
\pgfpathlineto{\pgfqpoint{1.363553in}{0.918613in}}%
\pgfpathlineto{\pgfqpoint{1.321723in}{0.937234in}}%
\pgfpathlineto{\pgfqpoint{1.286021in}{0.955644in}}%
\pgfpathlineto{\pgfqpoint{1.255668in}{0.973759in}}%
\pgfpathlineto{\pgfqpoint{1.230067in}{0.991500in}}%
\pgfpathlineto{\pgfqpoint{1.208806in}{1.008799in}}%
\pgfpathlineto{\pgfqpoint{1.191650in}{1.025596in}}%
\pgfpathlineto{\pgfqpoint{1.178549in}{1.041838in}}%
\pgfpathlineto{\pgfqpoint{1.169526in}{1.057483in}}%
\pgfpathlineto{\pgfqpoint{1.163761in}{1.072495in}}%
\pgfpathlineto{\pgfqpoint{1.160749in}{1.086855in}}%
\pgfpathlineto{\pgfqpoint{1.160206in}{1.100547in}}%
\pgfpathlineto{\pgfqpoint{1.161919in}{1.113561in}}%
\pgfpathlineto{\pgfqpoint{1.165755in}{1.125888in}}%
\pgfpathlineto{\pgfqpoint{1.171653in}{1.137523in}}%
\pgfpathlineto{\pgfqpoint{1.179629in}{1.148464in}}%
\pgfpathlineto{\pgfqpoint{1.189761in}{1.158712in}}%
\pgfpathlineto{\pgfqpoint{1.201941in}{1.168265in}}%
\pgfpathlineto{\pgfqpoint{1.216055in}{1.177119in}}%
\pgfpathlineto{\pgfqpoint{1.232061in}{1.185272in}}%
\pgfpathlineto{\pgfqpoint{1.259596in}{1.196183in}}%
\pgfpathlineto{\pgfqpoint{1.291426in}{1.205507in}}%
\pgfpathlineto{\pgfqpoint{1.327751in}{1.213238in}}%
\pgfpathlineto{\pgfqpoint{1.368902in}{1.219372in}}%
\pgfpathlineto{\pgfqpoint{1.415347in}{1.223904in}}%
\pgfpathlineto{\pgfqpoint{1.467338in}{1.226818in}}%
\pgfpathlineto{\pgfqpoint{1.525294in}{1.228062in}}%
\pgfpathlineto{\pgfqpoint{1.590060in}{1.227578in}}%
\pgfpathlineto{\pgfqpoint{1.688281in}{1.224129in}}%
\pgfpathlineto{\pgfqpoint{1.801421in}{1.217306in}}%
\pgfpathlineto{\pgfqpoint{1.930668in}{1.206897in}}%
\pgfpathlineto{\pgfqpoint{2.076905in}{1.192662in}}%
\pgfpathlineto{\pgfqpoint{2.241490in}{1.174308in}}%
\pgfpathlineto{\pgfqpoint{2.375834in}{1.157693in}}%
\pgfpathlineto{\pgfqpoint{2.514218in}{1.138686in}}%
\pgfpathlineto{\pgfqpoint{2.652107in}{1.117418in}}%
\pgfpathlineto{\pgfqpoint{2.785481in}{1.094095in}}%
\pgfpathlineto{\pgfqpoint{2.870139in}{1.077537in}}%
\pgfpathlineto{\pgfqpoint{2.950315in}{1.060286in}}%
\pgfpathlineto{\pgfqpoint{3.025201in}{1.042457in}}%
\pgfpathlineto{\pgfqpoint{3.094093in}{1.024176in}}%
\pgfpathlineto{\pgfqpoint{3.156387in}{1.005587in}}%
\pgfpathlineto{\pgfqpoint{3.211584in}{0.986847in}}%
\pgfpathlineto{\pgfqpoint{3.259285in}{0.968126in}}%
\pgfpathlineto{\pgfqpoint{3.299879in}{0.949557in}}%
\pgfpathlineto{\pgfqpoint{3.334355in}{0.931222in}}%
\pgfpathlineto{\pgfqpoint{3.363283in}{0.913212in}}%
\pgfpathlineto{\pgfqpoint{3.387186in}{0.895606in}}%
\pgfpathlineto{\pgfqpoint{3.406536in}{0.878473in}}%
\pgfpathlineto{\pgfqpoint{3.421792in}{0.861867in}}%
\pgfpathlineto{\pgfqpoint{3.433412in}{0.845830in}}%
\pgfpathlineto{\pgfqpoint{3.441770in}{0.830396in}}%
\pgfpathlineto{\pgfqpoint{3.447168in}{0.815593in}}%
\pgfpathlineto{\pgfqpoint{3.449839in}{0.801439in}}%
\pgfpathlineto{\pgfqpoint{3.449907in}{0.787951in}}%
\pgfpathlineto{\pgfqpoint{3.447522in}{0.775139in}}%
\pgfpathlineto{\pgfqpoint{3.442933in}{0.763011in}}%
\pgfpathlineto{\pgfqpoint{3.436330in}{0.751573in}}%
\pgfpathlineto{\pgfqpoint{3.427845in}{0.740831in}}%
\pgfpathlineto{\pgfqpoint{3.417554in}{0.730787in}}%
\pgfpathlineto{\pgfqpoint{3.405469in}{0.721441in}}%
\pgfpathlineto{\pgfqpoint{3.391549in}{0.712792in}}%
\pgfpathlineto{\pgfqpoint{3.375690in}{0.704836in}}%
\pgfpathlineto{\pgfqpoint{3.357732in}{0.697569in}}%
\pgfpathlineto{\pgfqpoint{3.326678in}{0.687956in}}%
\pgfpathlineto{\pgfqpoint{3.291028in}{0.679917in}}%
\pgfpathlineto{\pgfqpoint{3.250593in}{0.673469in}}%
\pgfpathlineto{\pgfqpoint{3.205062in}{0.668634in}}%
\pgfpathlineto{\pgfqpoint{3.154037in}{0.665441in}}%
\pgfpathlineto{\pgfqpoint{3.097033in}{0.663923in}}%
\pgfpathlineto{\pgfqpoint{3.033479in}{0.664122in}}%
\pgfpathlineto{\pgfqpoint{2.937446in}{0.667140in}}%
\pgfpathlineto{\pgfqpoint{2.827140in}{0.673463in}}%
\pgfpathlineto{\pgfqpoint{2.700308in}{0.683360in}}%
\pgfpathlineto{\pgfqpoint{2.555969in}{0.697079in}}%
\pgfpathlineto{\pgfqpoint{2.394844in}{0.714829in}}%
\pgfpathlineto{\pgfqpoint{2.219358in}{0.736786in}}%
\pgfpathlineto{\pgfqpoint{2.080982in}{0.756105in}}%
\pgfpathlineto{\pgfqpoint{1.942024in}{0.777726in}}%
\pgfpathlineto{\pgfqpoint{1.807636in}{0.801383in}}%
\pgfpathlineto{\pgfqpoint{1.722733in}{0.818129in}}%
\pgfpathlineto{\pgfqpoint{1.642836in}{0.835529in}}%
\pgfpathlineto{\pgfqpoint{1.568829in}{0.853464in}}%
\pgfpathlineto{\pgfqpoint{1.501422in}{0.871804in}}%
\pgfpathlineto{\pgfqpoint{1.441155in}{0.890409in}}%
\pgfpathlineto{\pgfqpoint{1.388396in}{0.909128in}}%
\pgfpathlineto{\pgfqpoint{1.343035in}{0.927802in}}%
\pgfpathlineto{\pgfqpoint{1.304209in}{0.946318in}}%
\pgfpathlineto{\pgfqpoint{1.271097in}{0.964582in}}%
\pgfpathlineto{\pgfqpoint{1.243049in}{0.982513in}}%
\pgfpathlineto{\pgfqpoint{1.219585in}{1.000036in}}%
\pgfpathlineto{\pgfqpoint{1.200390in}{1.017086in}}%
\pgfpathlineto{\pgfqpoint{1.185320in}{1.033606in}}%
\pgfpathlineto{\pgfqpoint{1.174390in}{1.049549in}}%
\pgfpathlineto{\pgfqpoint{1.167037in}{1.064875in}}%
\pgfpathlineto{\pgfqpoint{1.162596in}{1.079560in}}%
\pgfpathlineto{\pgfqpoint{1.160749in}{1.093587in}}%
\pgfpathlineto{\pgfqpoint{1.161256in}{1.106942in}}%
\pgfpathlineto{\pgfqpoint{1.163954in}{1.119615in}}%
\pgfpathlineto{\pgfqpoint{1.168757in}{1.131599in}}%
\pgfpathlineto{\pgfqpoint{1.175656in}{1.142891in}}%
\pgfpathlineto{\pgfqpoint{1.184718in}{1.153491in}}%
\pgfpathlineto{\pgfqpoint{1.195878in}{1.163397in}}%
\pgfpathlineto{\pgfqpoint{1.208992in}{1.172606in}}%
\pgfpathlineto{\pgfqpoint{1.224006in}{1.181115in}}%
\pgfpathlineto{\pgfqpoint{1.250038in}{1.192562in}}%
\pgfpathlineto{\pgfqpoint{1.280319in}{1.202424in}}%
\pgfpathlineto{\pgfqpoint{1.315014in}{1.210696in}}%
\pgfpathlineto{\pgfqpoint{1.354430in}{1.217375in}}%
\pgfpathlineto{\pgfqpoint{1.399013in}{1.222460in}}%
\pgfpathlineto{\pgfqpoint{1.449060in}{1.225937in}}%
\pgfpathlineto{\pgfqpoint{1.504884in}{1.227758in}}%
\pgfpathlineto{\pgfqpoint{1.567271in}{1.227872in}}%
\pgfpathlineto{\pgfqpoint{1.636943in}{1.226216in}}%
\pgfpathlineto{\pgfqpoint{1.742293in}{1.221130in}}%
\pgfpathlineto{\pgfqpoint{1.863114in}{1.212573in}}%
\pgfpathlineto{\pgfqpoint{2.000553in}{1.200314in}}%
\pgfpathlineto{\pgfqpoint{2.156107in}{1.184077in}}%
\pgfpathlineto{\pgfqpoint{2.331414in}{1.163512in}}%
\pgfpathlineto{\pgfqpoint{2.469546in}{1.145271in}}%
\pgfpathlineto{\pgfqpoint{2.608043in}{1.124744in}}%
\pgfpathlineto{\pgfqpoint{2.742767in}{1.102115in}}%
\pgfpathlineto{\pgfqpoint{2.870196in}{1.077630in}}%
\pgfpathlineto{\pgfqpoint{2.949635in}{1.060425in}}%
\pgfpathlineto{\pgfqpoint{3.023817in}{1.042639in}}%
\pgfpathlineto{\pgfqpoint{3.092153in}{1.024390in}}%
\pgfpathlineto{\pgfqpoint{3.154177in}{1.005811in}}%
\pgfpathlineto{\pgfqpoint{3.209543in}{0.987046in}}%
\pgfpathlineto{\pgfqpoint{3.258028in}{0.968250in}}%
\pgfpathlineto{\pgfqpoint{3.299527in}{0.949594in}}%
\pgfpathlineto{\pgfqpoint{3.334177in}{0.931243in}}%
\pgfpathlineto{\pgfqpoint{3.363085in}{0.913236in}}%
\pgfpathlineto{\pgfqpoint{3.386962in}{0.895634in}}%
\pgfpathlineto{\pgfqpoint{3.406319in}{0.878504in}}%
\pgfpathlineto{\pgfqpoint{3.421610in}{0.861899in}}%
\pgfpathlineto{\pgfqpoint{3.433255in}{0.845864in}}%
\pgfpathlineto{\pgfqpoint{3.441623in}{0.830432in}}%
\pgfpathlineto{\pgfqpoint{3.447020in}{0.815629in}}%
\pgfpathlineto{\pgfqpoint{3.449690in}{0.801476in}}%
\pgfpathlineto{\pgfqpoint{3.449722in}{0.787989in}}%
\pgfpathlineto{\pgfqpoint{3.447315in}{0.775177in}}%
\pgfpathlineto{\pgfqpoint{3.442728in}{0.763049in}}%
\pgfpathlineto{\pgfqpoint{3.436157in}{0.751612in}}%
\pgfpathlineto{\pgfqpoint{3.427732in}{0.740870in}}%
\pgfpathlineto{\pgfqpoint{3.417525in}{0.730825in}}%
\pgfpathlineto{\pgfqpoint{3.405539in}{0.721478in}}%
\pgfpathlineto{\pgfqpoint{3.391718in}{0.712828in}}%
\pgfpathlineto{\pgfqpoint{3.375940in}{0.704869in}}%
\pgfpathlineto{\pgfqpoint{3.358022in}{0.697596in}}%
\pgfpathlineto{\pgfqpoint{3.326976in}{0.687973in}}%
\pgfpathlineto{\pgfqpoint{3.291339in}{0.679925in}}%
\pgfpathlineto{\pgfqpoint{3.250920in}{0.673469in}}%
\pgfpathlineto{\pgfqpoint{3.205408in}{0.668626in}}%
\pgfpathlineto{\pgfqpoint{3.154404in}{0.665424in}}%
\pgfpathlineto{\pgfqpoint{3.097425in}{0.663897in}}%
\pgfpathlineto{\pgfqpoint{3.033896in}{0.664087in}}%
\pgfpathlineto{\pgfqpoint{2.937925in}{0.667090in}}%
\pgfpathlineto{\pgfqpoint{2.827685in}{0.673398in}}%
\pgfpathlineto{\pgfqpoint{2.700873in}{0.683284in}}%
\pgfpathlineto{\pgfqpoint{2.556536in}{0.696992in}}%
\pgfpathlineto{\pgfqpoint{2.395437in}{0.714732in}}%
\pgfpathlineto{\pgfqpoint{2.220047in}{0.736674in}}%
\pgfpathlineto{\pgfqpoint{2.081745in}{0.755982in}}%
\pgfpathlineto{\pgfqpoint{1.942696in}{0.777610in}}%
\pgfpathlineto{\pgfqpoint{1.808163in}{0.801272in}}%
\pgfpathlineto{\pgfqpoint{1.723164in}{0.818017in}}%
\pgfpathlineto{\pgfqpoint{1.643178in}{0.835411in}}%
\pgfpathlineto{\pgfqpoint{1.569077in}{0.853337in}}%
\pgfpathlineto{\pgfqpoint{1.501553in}{0.871668in}}%
\pgfpathlineto{\pgfqpoint{1.441115in}{0.890268in}}%
\pgfpathlineto{\pgfqpoint{1.388089in}{0.908994in}}%
\pgfpathlineto{\pgfqpoint{1.342552in}{0.927688in}}%
\pgfpathlineto{\pgfqpoint{1.303812in}{0.946214in}}%
\pgfpathlineto{\pgfqpoint{1.270911in}{0.964483in}}%
\pgfpathlineto{\pgfqpoint{1.243078in}{0.982416in}}%
\pgfpathlineto{\pgfqpoint{1.219736in}{0.999941in}}%
\pgfpathlineto{\pgfqpoint{1.200499in}{1.016995in}}%
\pgfpathlineto{\pgfqpoint{1.185176in}{1.033524in}}%
\pgfpathlineto{\pgfqpoint{1.173767in}{1.049479in}}%
\pgfpathlineto{\pgfqpoint{1.166294in}{1.064823in}}%
\pgfpathlineto{\pgfqpoint{1.161891in}{1.079523in}}%
\pgfpathlineto{\pgfqpoint{1.160126in}{1.093562in}}%
\pgfpathlineto{\pgfqpoint{1.160741in}{1.106928in}}%
\pgfpathlineto{\pgfqpoint{1.163550in}{1.119611in}}%
\pgfpathlineto{\pgfqpoint{1.168435in}{1.131604in}}%
\pgfpathlineto{\pgfqpoint{1.175349in}{1.142902in}}%
\pgfpathlineto{\pgfqpoint{1.184315in}{1.153505in}}%
\pgfpathlineto{\pgfqpoint{1.195407in}{1.163415in}}%
\pgfpathlineto{\pgfqpoint{1.208514in}{1.172629in}}%
\pgfpathlineto{\pgfqpoint{1.223546in}{1.181144in}}%
\pgfpathlineto{\pgfqpoint{1.249644in}{1.192602in}}%
\pgfpathlineto{\pgfqpoint{1.280019in}{1.202476in}}%
\pgfpathlineto{\pgfqpoint{1.314806in}{1.210758in}}%
\pgfpathlineto{\pgfqpoint{1.354270in}{1.217442in}}%
\pgfpathlineto{\pgfqpoint{1.398806in}{1.222521in}}%
\pgfpathlineto{\pgfqpoint{1.448905in}{1.225989in}}%
\pgfpathlineto{\pgfqpoint{1.504699in}{1.227813in}}%
\pgfpathlineto{\pgfqpoint{1.566938in}{1.227933in}}%
\pgfpathlineto{\pgfqpoint{1.636499in}{1.226283in}}%
\pgfpathlineto{\pgfqpoint{1.741893in}{1.221200in}}%
\pgfpathlineto{\pgfqpoint{1.862954in}{1.212637in}}%
\pgfpathlineto{\pgfqpoint{2.000589in}{1.200375in}}%
\pgfpathlineto{\pgfqpoint{2.155254in}{1.184166in}}%
\pgfpathlineto{\pgfqpoint{2.327384in}{1.163724in}}%
\pgfpathlineto{\pgfqpoint{2.464673in}{1.145543in}}%
\pgfpathlineto{\pgfqpoint{2.603526in}{1.125033in}}%
\pgfpathlineto{\pgfqpoint{2.739522in}{1.102380in}}%
\pgfpathlineto{\pgfqpoint{2.868756in}{1.077840in}}%
\pgfpathlineto{\pgfqpoint{2.949461in}{1.060593in}}%
\pgfpathlineto{\pgfqpoint{3.024768in}{1.042769in}}%
\pgfpathlineto{\pgfqpoint{3.093901in}{1.024499in}}%
\pgfpathlineto{\pgfqpoint{3.156187in}{1.005929in}}%
\pgfpathlineto{\pgfqpoint{3.211057in}{0.987217in}}%
\pgfpathlineto{\pgfqpoint{3.258395in}{0.968522in}}%
\pgfpathlineto{\pgfqpoint{3.299059in}{0.949952in}}%
\pgfpathlineto{\pgfqpoint{3.333688in}{0.931611in}}%
\pgfpathlineto{\pgfqpoint{3.362839in}{0.913590in}}%
\pgfpathlineto{\pgfqpoint{3.386993in}{0.895970in}}%
\pgfpathlineto{\pgfqpoint{3.406557in}{0.878819in}}%
\pgfpathlineto{\pgfqpoint{3.421891in}{0.862194in}}%
\pgfpathlineto{\pgfqpoint{3.433522in}{0.846141in}}%
\pgfpathlineto{\pgfqpoint{3.441888in}{0.830691in}}%
\pgfpathlineto{\pgfqpoint{3.447322in}{0.815869in}}%
\pgfpathlineto{\pgfqpoint{3.450080in}{0.801698in}}%
\pgfpathlineto{\pgfqpoint{3.450336in}{0.788193in}}%
\pgfpathlineto{\pgfqpoint{3.448187in}{0.775363in}}%
\pgfpathlineto{\pgfqpoint{3.443701in}{0.763217in}}%
\pgfpathlineto{\pgfqpoint{3.437049in}{0.751760in}}%
\pgfpathlineto{\pgfqpoint{3.428388in}{0.740996in}}%
\pgfpathlineto{\pgfqpoint{3.417832in}{0.730930in}}%
\pgfpathlineto{\pgfqpoint{3.405451in}{0.721563in}}%
\pgfpathlineto{\pgfqpoint{3.391270in}{0.712898in}}%
\pgfpathlineto{\pgfqpoint{3.366577in}{0.701213in}}%
\pgfpathlineto{\pgfqpoint{3.337545in}{0.691098in}}%
\pgfpathlineto{\pgfqpoint{3.303716in}{0.682539in}}%
\pgfpathlineto{\pgfqpoint{3.264980in}{0.675548in}}%
\pgfpathlineto{\pgfqpoint{3.221275in}{0.670154in}}%
\pgfpathlineto{\pgfqpoint{3.172198in}{0.666384in}}%
\pgfpathlineto{\pgfqpoint{3.117283in}{0.664273in}}%
\pgfpathlineto{\pgfqpoint{3.055997in}{0.663866in}}%
\pgfpathlineto{\pgfqpoint{2.987742in}{0.665217in}}%
\pgfpathlineto{\pgfqpoint{2.884741in}{0.669863in}}%
\pgfpathlineto{\pgfqpoint{2.766467in}{0.677928in}}%
\pgfpathlineto{\pgfqpoint{2.631344in}{0.689663in}}%
\pgfpathlineto{\pgfqpoint{2.478734in}{0.705327in}}%
\pgfpathlineto{\pgfqpoint{2.309895in}{0.725135in}}%
\pgfpathlineto{\pgfqpoint{2.174882in}{0.742792in}}%
\pgfpathlineto{\pgfqpoint{2.036356in}{0.762820in}}%
\pgfpathlineto{\pgfqpoint{1.898474in}{0.785086in}}%
\pgfpathlineto{\pgfqpoint{1.765993in}{0.809345in}}%
\pgfpathlineto{\pgfqpoint{1.683297in}{0.826472in}}%
\pgfpathlineto{\pgfqpoint{1.606484in}{0.844218in}}%
\pgfpathlineto{\pgfqpoint{1.536106in}{0.862416in}}%
\pgfpathlineto{\pgfqpoint{1.472533in}{0.880910in}}%
\pgfpathlineto{\pgfqpoint{1.415955in}{0.899559in}}%
\pgfpathlineto{\pgfqpoint{1.366383in}{0.918232in}}%
\pgfpathlineto{\pgfqpoint{1.323647in}{0.936811in}}%
\pgfpathlineto{\pgfqpoint{1.287398in}{0.955190in}}%
\pgfpathlineto{\pgfqpoint{1.257107in}{0.973277in}}%
\pgfpathlineto{\pgfqpoint{1.232065in}{0.990990in}}%
\pgfpathlineto{\pgfqpoint{1.211597in}{1.008255in}}%
\pgfpathlineto{\pgfqpoint{1.195640in}{1.024996in}}%
\pgfpathlineto{\pgfqpoint{1.183428in}{1.041178in}}%
\pgfpathlineto{\pgfqpoint{1.174273in}{1.056773in}}%
\pgfpathlineto{\pgfqpoint{1.167676in}{1.071755in}}%
\pgfpathlineto{\pgfqpoint{1.163330in}{1.086106in}}%
\pgfpathlineto{\pgfqpoint{1.161117in}{1.099808in}}%
\pgfpathlineto{\pgfqpoint{1.161109in}{1.112851in}}%
\pgfpathlineto{\pgfqpoint{1.163567in}{1.125227in}}%
\pgfpathlineto{\pgfqpoint{1.168943in}{1.136931in}}%
\pgfpathlineto{\pgfqpoint{1.177169in}{1.147952in}}%
\pgfpathlineto{\pgfqpoint{1.187405in}{1.158273in}}%
\pgfpathlineto{\pgfqpoint{1.199561in}{1.167891in}}%
\pgfpathlineto{\pgfqpoint{1.213585in}{1.176805in}}%
\pgfpathlineto{\pgfqpoint{1.229457in}{1.185013in}}%
\pgfpathlineto{\pgfqpoint{1.256756in}{1.196003in}}%
\pgfpathlineto{\pgfqpoint{1.288394in}{1.205405in}}%
\pgfpathlineto{\pgfqpoint{1.324679in}{1.213224in}}%
\pgfpathlineto{\pgfqpoint{1.365922in}{1.219460in}}%
\pgfpathlineto{\pgfqpoint{1.412254in}{1.224087in}}%
\pgfpathlineto{\pgfqpoint{1.464166in}{1.227073in}}%
\pgfpathlineto{\pgfqpoint{1.522208in}{1.228380in}}%
\pgfpathlineto{\pgfqpoint{1.586961in}{1.227955in}}%
\pgfpathlineto{\pgfqpoint{1.684809in}{1.224586in}}%
\pgfpathlineto{\pgfqpoint{1.797249in}{1.217852in}}%
\pgfpathlineto{\pgfqpoint{1.925929in}{1.207545in}}%
\pgfpathlineto{\pgfqpoint{2.072027in}{1.193409in}}%
\pgfpathlineto{\pgfqpoint{2.235264in}{1.175186in}}%
\pgfpathlineto{\pgfqpoint{2.412561in}{1.152724in}}%
\pgfpathlineto{\pgfqpoint{2.551143in}{1.133083in}}%
\pgfpathlineto{\pgfqpoint{2.689758in}{1.111168in}}%
\pgfpathlineto{\pgfqpoint{2.823593in}{1.087208in}}%
\pgfpathlineto{\pgfqpoint{2.907912in}{1.070258in}}%
\pgfpathlineto{\pgfqpoint{2.986818in}{1.052668in}}%
\pgfpathlineto{\pgfqpoint{3.058812in}{1.034580in}}%
\pgfpathlineto{\pgfqpoint{3.123347in}{1.016151in}}%
\pgfpathlineto{\pgfqpoint{3.180898in}{0.997522in}}%
\pgfpathlineto{\pgfqpoint{3.231900in}{0.978824in}}%
\pgfpathlineto{\pgfqpoint{3.276746in}{0.960176in}}%
\pgfpathlineto{\pgfqpoint{3.315782in}{0.941689in}}%
\pgfpathlineto{\pgfqpoint{3.349312in}{0.923458in}}%
\pgfpathlineto{\pgfqpoint{3.377597in}{0.905570in}}%
\pgfpathlineto{\pgfqpoint{3.400850in}{0.888101in}}%
\pgfpathlineto{\pgfqpoint{3.419244in}{0.871113in}}%
\pgfpathlineto{\pgfqpoint{3.432904in}{0.854659in}}%
\pgfpathlineto{\pgfqpoint{3.442332in}{0.838797in}}%
\pgfpathlineto{\pgfqpoint{3.448442in}{0.823568in}}%
\pgfpathlineto{\pgfqpoint{3.451810in}{0.808990in}}%
\pgfpathlineto{\pgfqpoint{3.452880in}{0.795078in}}%
\pgfpathlineto{\pgfqpoint{3.451962in}{0.781841in}}%
\pgfpathlineto{\pgfqpoint{3.449235in}{0.769287in}}%
\pgfpathlineto{\pgfqpoint{3.444743in}{0.757422in}}%
\pgfpathlineto{\pgfqpoint{3.438398in}{0.746246in}}%
\pgfpathlineto{\pgfqpoint{3.429979in}{0.735756in}}%
\pgfpathlineto{\pgfqpoint{3.419133in}{0.725949in}}%
\pgfpathlineto{\pgfqpoint{3.405588in}{0.716821in}}%
\pgfpathlineto{\pgfqpoint{3.390019in}{0.708394in}}%
\pgfpathlineto{\pgfqpoint{3.363097in}{0.697077in}}%
\pgfpathlineto{\pgfqpoint{3.331857in}{0.687353in}}%
\pgfpathlineto{\pgfqpoint{3.296155in}{0.679229in}}%
\pgfpathlineto{\pgfqpoint{3.255734in}{0.672713in}}%
\pgfpathlineto{\pgfqpoint{3.210226in}{0.667816in}}%
\pgfpathlineto{\pgfqpoint{3.159161in}{0.664550in}}%
\pgfpathlineto{\pgfqpoint{3.102383in}{0.662936in}}%
\pgfpathlineto{\pgfqpoint{3.039138in}{0.663037in}}%
\pgfpathlineto{\pgfqpoint{2.968442in}{0.664923in}}%
\pgfpathlineto{\pgfqpoint{2.861256in}{0.670352in}}%
\pgfpathlineto{\pgfqpoint{2.738102in}{0.679296in}}%
\pgfpathlineto{\pgfqpoint{2.598246in}{0.691976in}}%
\pgfpathlineto{\pgfqpoint{2.441559in}{0.708628in}}%
\pgfpathlineto{\pgfqpoint{2.268310in}{0.729524in}}%
\pgfpathlineto{\pgfqpoint{2.130126in}{0.748079in}}%
\pgfpathlineto{\pgfqpoint{1.990595in}{0.768961in}}%
\pgfpathlineto{\pgfqpoint{1.854481in}{0.791954in}}%
\pgfpathlineto{\pgfqpoint{1.725872in}{0.816780in}}%
\pgfpathlineto{\pgfqpoint{1.646013in}{0.834182in}}%
\pgfpathlineto{\pgfqpoint{1.571861in}{0.852130in}}%
\pgfpathlineto{\pgfqpoint{1.504124in}{0.870494in}}%
\pgfpathlineto{\pgfqpoint{1.443377in}{0.889132in}}%
\pgfpathlineto{\pgfqpoint{1.390061in}{0.907887in}}%
\pgfpathlineto{\pgfqpoint{1.344184in}{0.926604in}}%
\pgfpathlineto{\pgfqpoint{1.304912in}{0.945169in}}%
\pgfpathlineto{\pgfqpoint{1.271537in}{0.963482in}}%
\pgfpathlineto{\pgfqpoint{1.243468in}{0.981457in}}%
\pgfpathlineto{\pgfqpoint{1.220220in}{0.999017in}}%
\pgfpathlineto{\pgfqpoint{1.201417in}{1.016095in}}%
\pgfpathlineto{\pgfqpoint{1.186794in}{1.032637in}}%
\pgfpathlineto{\pgfqpoint{1.175934in}{1.048598in}}%
\pgfpathlineto{\pgfqpoint{1.168293in}{1.063946in}}%
\pgfpathlineto{\pgfqpoint{1.163526in}{1.078657in}}%
\pgfpathlineto{\pgfqpoint{1.161369in}{1.092713in}}%
\pgfpathlineto{\pgfqpoint{1.161631in}{1.106100in}}%
\pgfpathlineto{\pgfqpoint{1.164201in}{1.118806in}}%
\pgfpathlineto{\pgfqpoint{1.169040in}{1.130826in}}%
\pgfpathlineto{\pgfqpoint{1.176086in}{1.142157in}}%
\pgfpathlineto{\pgfqpoint{1.185188in}{1.152793in}}%
\pgfpathlineto{\pgfqpoint{1.196232in}{1.162732in}}%
\pgfpathlineto{\pgfqpoint{1.209138in}{1.171971in}}%
\pgfpathlineto{\pgfqpoint{1.223866in}{1.180509in}}%
\pgfpathlineto{\pgfqpoint{1.249374in}{1.191996in}}%
\pgfpathlineto{\pgfqpoint{1.279118in}{1.201905in}}%
\pgfpathlineto{\pgfqpoint{1.313425in}{1.210239in}}%
\pgfpathlineto{\pgfqpoint{1.352760in}{1.217009in}}%
\pgfpathlineto{\pgfqpoint{1.397155in}{1.222186in}}%
\pgfpathlineto{\pgfqpoint{1.446952in}{1.225740in}}%
\pgfpathlineto{\pgfqpoint{1.502679in}{1.227632in}}%
\pgfpathlineto{\pgfqpoint{1.564899in}{1.227815in}}%
\pgfpathlineto{\pgfqpoint{1.634215in}{1.226232in}}%
\pgfpathlineto{\pgfqpoint{1.738790in}{1.221252in}}%
\pgfpathlineto{\pgfqpoint{1.858749in}{1.212817in}}%
\pgfpathlineto{\pgfqpoint{1.995623in}{1.200696in}}%
\pgfpathlineto{\pgfqpoint{2.150026in}{1.184617in}}%
\pgfpathlineto{\pgfqpoint{2.320401in}{1.164372in}}%
\pgfpathlineto{\pgfqpoint{2.456121in}{1.146388in}}%
\pgfpathlineto{\pgfqpoint{2.594998in}{1.126040in}}%
\pgfpathlineto{\pgfqpoint{2.732420in}{1.103483in}}%
\pgfpathlineto{\pgfqpoint{2.863723in}{1.078973in}}%
\pgfpathlineto{\pgfqpoint{2.945559in}{1.061725in}}%
\pgfpathlineto{\pgfqpoint{3.020921in}{1.043895in}}%
\pgfpathlineto{\pgfqpoint{3.089108in}{1.025634in}}%
\pgfpathlineto{\pgfqpoint{3.150383in}{1.007097in}}%
\pgfpathlineto{\pgfqpoint{3.205011in}{0.988426in}}%
\pgfpathlineto{\pgfqpoint{3.253264in}{0.969753in}}%
\pgfpathlineto{\pgfqpoint{3.295419in}{0.951194in}}%
\pgfpathlineto{\pgfqpoint{3.331756in}{0.932853in}}%
\pgfpathlineto{\pgfqpoint{3.362561in}{0.914820in}}%
\pgfpathlineto{\pgfqpoint{3.388124in}{0.897173in}}%
\pgfpathlineto{\pgfqpoint{3.408740in}{0.879976in}}%
\pgfpathlineto{\pgfqpoint{3.424708in}{0.863280in}}%
\pgfpathlineto{\pgfqpoint{3.436343in}{0.847127in}}%
\pgfpathlineto{\pgfqpoint{3.444187in}{0.831586in}}%
\pgfpathlineto{\pgfqpoint{3.448938in}{0.816685in}}%
\pgfpathlineto{\pgfqpoint{3.451160in}{0.802439in}}%
\pgfpathlineto{\pgfqpoint{3.451278in}{0.788861in}}%
\pgfpathlineto{\pgfqpoint{3.449574in}{0.775960in}}%
\pgfpathlineto{\pgfqpoint{3.446190in}{0.763742in}}%
\pgfpathlineto{\pgfqpoint{3.441124in}{0.752210in}}%
\pgfpathlineto{\pgfqpoint{3.434235in}{0.741364in}}%
\pgfpathlineto{\pgfqpoint{3.425239in}{0.731200in}}%
\pgfpathlineto{\pgfqpoint{3.413710in}{0.721712in}}%
\pgfpathlineto{\pgfqpoint{3.399326in}{0.712896in}}%
\pgfpathlineto{\pgfqpoint{3.382889in}{0.704781in}}%
\pgfpathlineto{\pgfqpoint{3.354651in}{0.693933in}}%
\pgfpathlineto{\pgfqpoint{3.322050in}{0.684679in}}%
\pgfpathlineto{\pgfqpoint{3.284914in}{0.677028in}}%
\pgfpathlineto{\pgfqpoint{3.242966in}{0.670990in}}%
\pgfpathlineto{\pgfqpoint{3.195821in}{0.666578in}}%
\pgfpathlineto{\pgfqpoint{3.142997in}{0.663807in}}%
\pgfpathlineto{\pgfqpoint{3.084355in}{0.662699in}}%
\pgfpathlineto{\pgfqpoint{3.019060in}{0.663320in}}%
\pgfpathlineto{\pgfqpoint{2.919863in}{0.666978in}}%
\pgfpathlineto{\pgfqpoint{2.805168in}{0.674053in}}%
\pgfpathlineto{\pgfqpoint{2.673926in}{0.684754in}}%
\pgfpathlineto{\pgfqpoint{2.525857in}{0.699302in}}%
\pgfpathlineto{\pgfqpoint{2.361454in}{0.717926in}}%
\pgfpathlineto{\pgfqpoint{2.182049in}{0.740887in}}%
\pgfpathlineto{\pgfqpoint{2.042195in}{0.760956in}}%
\pgfpathlineto{\pgfqpoint{1.903815in}{0.783235in}}%
\pgfpathlineto{\pgfqpoint{1.771541in}{0.807449in}}%
\pgfpathlineto{\pgfqpoint{1.688668in}{0.824506in}}%
\pgfpathlineto{\pgfqpoint{1.611150in}{0.842170in}}%
\pgfpathlineto{\pgfqpoint{1.539757in}{0.860317in}}%
\pgfpathlineto{\pgfqpoint{1.475094in}{0.878819in}}%
\pgfpathlineto{\pgfqpoint{1.417607in}{0.897532in}}%
\pgfpathlineto{\pgfqpoint{1.367581in}{0.916305in}}%
\pgfpathlineto{\pgfqpoint{1.324880in}{0.934983in}}%
\pgfpathlineto{\pgfqpoint{1.288594in}{0.953452in}}%
\pgfpathlineto{\pgfqpoint{1.257885in}{0.971627in}}%
\pgfpathlineto{\pgfqpoint{1.232086in}{0.989430in}}%
\pgfpathlineto{\pgfqpoint{1.210699in}{1.006791in}}%
\pgfpathlineto{\pgfqpoint{1.193397in}{1.023652in}}%
\pgfpathlineto{\pgfqpoint{1.180021in}{1.039960in}}%
\pgfpathlineto{\pgfqpoint{1.170555in}{1.055674in}}%
\pgfpathlineto{\pgfqpoint{1.164407in}{1.070760in}}%
\pgfpathlineto{\pgfqpoint{1.161051in}{1.085195in}}%
\pgfpathlineto{\pgfqpoint{1.160208in}{1.098965in}}%
\pgfpathlineto{\pgfqpoint{1.161669in}{1.112058in}}%
\pgfpathlineto{\pgfqpoint{1.165295in}{1.124464in}}%
\pgfpathlineto{\pgfqpoint{1.171016in}{1.136179in}}%
\pgfpathlineto{\pgfqpoint{1.178833in}{1.147200in}}%
\pgfpathlineto{\pgfqpoint{1.188780in}{1.157527in}}%
\pgfpathlineto{\pgfqpoint{1.200738in}{1.167159in}}%
\pgfpathlineto{\pgfqpoint{1.214624in}{1.176091in}}%
\pgfpathlineto{\pgfqpoint{1.230393in}{1.184323in}}%
\pgfpathlineto{\pgfqpoint{1.257547in}{1.195352in}}%
\pgfpathlineto{\pgfqpoint{1.288970in}{1.204794in}}%
\pgfpathlineto{\pgfqpoint{1.324866in}{1.212645in}}%
\pgfpathlineto{\pgfqpoint{1.365584in}{1.218902in}}%
\pgfpathlineto{\pgfqpoint{1.411608in}{1.223565in}}%
\pgfpathlineto{\pgfqpoint{1.463118in}{1.226609in}}%
\pgfpathlineto{\pgfqpoint{1.520604in}{1.227986in}}%
\pgfpathlineto{\pgfqpoint{1.584847in}{1.227639in}}%
\pgfpathlineto{\pgfqpoint{1.682234in}{1.224383in}}%
\pgfpathlineto{\pgfqpoint{1.794376in}{1.217767in}}%
\pgfpathlineto{\pgfqpoint{1.922534in}{1.207581in}}%
\pgfpathlineto{\pgfqpoint{2.067744in}{1.193580in}}%
\pgfpathlineto{\pgfqpoint{2.232594in}{1.175426in}}%
\pgfpathlineto{\pgfqpoint{2.367470in}{1.158937in}}%
\pgfpathlineto{\pgfqpoint{2.506195in}{1.140064in}}%
\pgfpathlineto{\pgfqpoint{2.644139in}{1.118952in}}%
\pgfpathlineto{\pgfqpoint{2.777296in}{1.095803in}}%
\pgfpathlineto{\pgfqpoint{2.861716in}{1.079366in}}%
\pgfpathlineto{\pgfqpoint{2.941650in}{1.062234in}}%
\pgfpathlineto{\pgfqpoint{3.016377in}{1.044511in}}%
\pgfpathlineto{\pgfqpoint{3.085300in}{1.026315in}}%
\pgfpathlineto{\pgfqpoint{3.147946in}{1.007776in}}%
\pgfpathlineto{\pgfqpoint{3.203966in}{0.989034in}}%
\pgfpathlineto{\pgfqpoint{3.253136in}{0.970244in}}%
\pgfpathlineto{\pgfqpoint{3.295354in}{0.951571in}}%
\pgfpathlineto{\pgfqpoint{3.330701in}{0.933185in}}%
\pgfpathlineto{\pgfqpoint{3.360184in}{0.915140in}}%
\pgfpathlineto{\pgfqpoint{3.384578in}{0.897493in}}%
\pgfpathlineto{\pgfqpoint{3.404403in}{0.880310in}}%
\pgfpathlineto{\pgfqpoint{3.420115in}{0.863648in}}%
\pgfpathlineto{\pgfqpoint{3.432135in}{0.847551in}}%
\pgfpathlineto{\pgfqpoint{3.440840in}{0.832053in}}%
\pgfpathlineto{\pgfqpoint{3.446544in}{0.817182in}}%
\pgfpathlineto{\pgfqpoint{3.449501in}{0.802960in}}%
\pgfpathlineto{\pgfqpoint{3.449851in}{0.789402in}}%
\pgfpathlineto{\pgfqpoint{3.447707in}{0.776517in}}%
\pgfpathlineto{\pgfqpoint{3.443333in}{0.764315in}}%
\pgfpathlineto{\pgfqpoint{3.436936in}{0.752804in}}%
\pgfpathlineto{\pgfqpoint{3.428661in}{0.741987in}}%
\pgfpathlineto{\pgfqpoint{3.418593in}{0.731867in}}%
\pgfpathlineto{\pgfqpoint{3.406756in}{0.722446in}}%
\pgfpathlineto{\pgfqpoint{3.393112in}{0.713723in}}%
\pgfpathlineto{\pgfqpoint{3.377562in}{0.705693in}}%
\pgfpathlineto{\pgfqpoint{3.359947in}{0.698352in}}%
\pgfpathlineto{\pgfqpoint{3.329315in}{0.688622in}}%
\pgfpathlineto{\pgfqpoint{3.294018in}{0.680461in}}%
\pgfpathlineto{\pgfqpoint{3.253955in}{0.673890in}}%
\pgfpathlineto{\pgfqpoint{3.208820in}{0.668929in}}%
\pgfpathlineto{\pgfqpoint{3.158221in}{0.665607in}}%
\pgfpathlineto{\pgfqpoint{3.101682in}{0.663958in}}%
\pgfpathlineto{\pgfqpoint{3.038640in}{0.664023in}}%
\pgfpathlineto{\pgfqpoint{2.943351in}{0.666860in}}%
\pgfpathlineto{\pgfqpoint{2.833884in}{0.672991in}}%
\pgfpathlineto{\pgfqpoint{2.708062in}{0.682676in}}%
\pgfpathlineto{\pgfqpoint{2.564768in}{0.696166in}}%
\pgfpathlineto{\pgfqpoint{2.404570in}{0.713678in}}%
\pgfpathlineto{\pgfqpoint{2.229711in}{0.735395in}}%
\pgfpathlineto{\pgfqpoint{2.091554in}{0.754526in}}%
\pgfpathlineto{\pgfqpoint{1.952570in}{0.775963in}}%
\pgfpathlineto{\pgfqpoint{1.817783in}{0.799463in}}%
\pgfpathlineto{\pgfqpoint{1.732396in}{0.816127in}}%
\pgfpathlineto{\pgfqpoint{1.651869in}{0.833465in}}%
\pgfpathlineto{\pgfqpoint{1.577131in}{0.851357in}}%
\pgfpathlineto{\pgfqpoint{1.508964in}{0.869670in}}%
\pgfpathlineto{\pgfqpoint{1.448004in}{0.888257in}}%
\pgfpathlineto{\pgfqpoint{1.394660in}{0.906960in}}%
\pgfpathlineto{\pgfqpoint{1.348506in}{0.925634in}}%
\pgfpathlineto{\pgfqpoint{1.308753in}{0.944171in}}%
\pgfpathlineto{\pgfqpoint{1.274738in}{0.962472in}}%
\pgfpathlineto{\pgfqpoint{1.245926in}{0.980449in}}%
\pgfpathlineto{\pgfqpoint{1.221911in}{0.998025in}}%
\pgfpathlineto{\pgfqpoint{1.202417in}{1.015131in}}%
\pgfpathlineto{\pgfqpoint{1.187295in}{1.031708in}}%
\pgfpathlineto{\pgfqpoint{1.176257in}{1.047706in}}%
\pgfpathlineto{\pgfqpoint{1.168516in}{1.063092in}}%
\pgfpathlineto{\pgfqpoint{1.163663in}{1.077842in}}%
\pgfpathlineto{\pgfqpoint{1.161390in}{1.091938in}}%
\pgfpathlineto{\pgfqpoint{1.161476in}{1.105365in}}%
\pgfpathlineto{\pgfqpoint{1.163789in}{1.118113in}}%
\pgfpathlineto{\pgfqpoint{1.168286in}{1.130175in}}%
\pgfpathlineto{\pgfqpoint{1.175010in}{1.141548in}}%
\pgfpathlineto{\pgfqpoint{1.183946in}{1.152230in}}%
\pgfpathlineto{\pgfqpoint{1.194891in}{1.162216in}}%
\pgfpathlineto{\pgfqpoint{1.207759in}{1.171504in}}%
\pgfpathlineto{\pgfqpoint{1.222500in}{1.180091in}}%
\pgfpathlineto{\pgfqpoint{1.248079in}{1.191653in}}%
\pgfpathlineto{\pgfqpoint{1.277881in}{1.201632in}}%
\pgfpathlineto{\pgfqpoint{1.312116in}{1.210026in}}%
\pgfpathlineto{\pgfqpoint{1.351146in}{1.216837in}}%
\pgfpathlineto{\pgfqpoint{1.395385in}{1.222065in}}%
\pgfpathlineto{\pgfqpoint{1.444934in}{1.225676in}}%
\pgfpathlineto{\pgfqpoint{1.500363in}{1.227630in}}%
\pgfpathlineto{\pgfqpoint{1.562308in}{1.227880in}}%
\pgfpathlineto{\pgfqpoint{1.631403in}{1.226365in}}%
\pgfpathlineto{\pgfqpoint{1.735736in}{1.221480in}}%
\pgfpathlineto{\pgfqpoint{1.855368in}{1.213138in}}%
\pgfpathlineto{\pgfqpoint{1.991759in}{1.201111in}}%
\pgfpathlineto{\pgfqpoint{2.145734in}{1.185155in}}%
\pgfpathlineto{\pgfqpoint{2.315820in}{1.165009in}}%
\pgfpathlineto{\pgfqpoint{2.451291in}{1.147089in}}%
\pgfpathlineto{\pgfqpoint{2.590052in}{1.126818in}}%
\pgfpathlineto{\pgfqpoint{2.727788in}{1.104336in}}%
\pgfpathlineto{\pgfqpoint{2.859345in}{1.079855in}}%
\pgfpathlineto{\pgfqpoint{2.941329in}{1.062616in}}%
\pgfpathlineto{\pgfqpoint{3.017518in}{1.044812in}}%
\pgfpathlineto{\pgfqpoint{3.087171in}{1.026587in}}%
\pgfpathlineto{\pgfqpoint{3.149814in}{1.008075in}}%
\pgfpathlineto{\pgfqpoint{3.205248in}{0.989410in}}%
\pgfpathlineto{\pgfqpoint{3.253541in}{0.970718in}}%
\pgfpathlineto{\pgfqpoint{3.295030in}{0.952123in}}%
\pgfpathlineto{\pgfqpoint{3.330140in}{0.933749in}}%
\pgfpathlineto{\pgfqpoint{3.359166in}{0.915716in}}%
\pgfpathlineto{\pgfqpoint{3.383188in}{0.898082in}}%
\pgfpathlineto{\pgfqpoint{3.403116in}{0.880899in}}%
\pgfpathlineto{\pgfqpoint{3.419603in}{0.864212in}}%
\pgfpathlineto{\pgfqpoint{3.433047in}{0.848064in}}%
\pgfpathlineto{\pgfqpoint{3.443588in}{0.832490in}}%
\pgfpathlineto{\pgfqpoint{3.451110in}{0.817524in}}%
\pgfpathlineto{\pgfqpoint{3.455241in}{0.803193in}}%
\pgfpathlineto{\pgfqpoint{3.455560in}{0.789521in}}%
\pgfpathlineto{\pgfqpoint{3.453206in}{0.776534in}}%
\pgfpathlineto{\pgfqpoint{3.448655in}{0.764241in}}%
\pgfpathlineto{\pgfqpoint{3.442060in}{0.752647in}}%
\pgfpathlineto{\pgfqpoint{3.433528in}{0.741755in}}%
\pgfpathlineto{\pgfqpoint{3.423121in}{0.731567in}}%
\pgfpathlineto{\pgfqpoint{3.410855in}{0.722083in}}%
\pgfpathlineto{\pgfqpoint{3.396699in}{0.713302in}}%
\pgfpathlineto{\pgfqpoint{3.380577in}{0.705221in}}%
\pgfpathlineto{\pgfqpoint{3.352572in}{0.694405in}}%
\pgfpathlineto{\pgfqpoint{3.320113in}{0.685167in}}%
\pgfpathlineto{\pgfqpoint{3.283070in}{0.677519in}}%
\pgfpathlineto{\pgfqpoint{3.241202in}{0.671477in}}%
\pgfpathlineto{\pgfqpoint{3.194171in}{0.667064in}}%
\pgfpathlineto{\pgfqpoint{3.141544in}{0.664305in}}%
\pgfpathlineto{\pgfqpoint{3.082791in}{0.663233in}}%
\pgfpathlineto{\pgfqpoint{3.017290in}{0.663883in}}%
\pgfpathlineto{\pgfqpoint{2.918733in}{0.667499in}}%
\pgfpathlineto{\pgfqpoint{2.805181in}{0.674490in}}%
\pgfpathlineto{\pgfqpoint{2.674384in}{0.685148in}}%
\pgfpathlineto{\pgfqpoint{2.525932in}{0.699707in}}%
\pgfpathlineto{\pgfqpoint{2.361258in}{0.718343in}}%
\pgfpathlineto{\pgfqpoint{2.183640in}{0.741171in}}%
\pgfpathlineto{\pgfqpoint{2.045026in}{0.761086in}}%
\pgfpathlineto{\pgfqpoint{1.906526in}{0.783326in}}%
\pgfpathlineto{\pgfqpoint{1.773539in}{0.807552in}}%
\pgfpathlineto{\pgfqpoint{1.690144in}{0.824616in}}%
\pgfpathlineto{\pgfqpoint{1.612184in}{0.842273in}}%
\pgfpathlineto{\pgfqpoint{1.540467in}{0.860399in}}%
\pgfpathlineto{\pgfqpoint{1.475587in}{0.878866in}}%
\pgfpathlineto{\pgfqpoint{1.417921in}{0.897543in}}%
\pgfpathlineto{\pgfqpoint{1.367631in}{0.916294in}}%
\pgfpathlineto{\pgfqpoint{1.324665in}{0.934977in}}%
\pgfpathlineto{\pgfqpoint{1.288526in}{0.953443in}}%
\pgfpathlineto{\pgfqpoint{1.258160in}{0.971605in}}%
\pgfpathlineto{\pgfqpoint{1.232659in}{0.989392in}}%
\pgfpathlineto{\pgfqpoint{1.211343in}{1.006744in}}%
\pgfpathlineto{\pgfqpoint{1.193756in}{1.023604in}}%
\pgfpathlineto{\pgfqpoint{1.179675in}{1.039925in}}%
\pgfpathlineto{\pgfqpoint{1.169100in}{1.055663in}}%
\pgfpathlineto{\pgfqpoint{1.162261in}{1.070783in}}%
\pgfpathlineto{\pgfqpoint{1.159024in}{1.085255in}}%
\pgfpathlineto{\pgfqpoint{1.158404in}{1.099055in}}%
\pgfpathlineto{\pgfqpoint{1.160130in}{1.112174in}}%
\pgfpathlineto{\pgfqpoint{1.164011in}{1.124603in}}%
\pgfpathlineto{\pgfqpoint{1.169915in}{1.136336in}}%
\pgfpathlineto{\pgfqpoint{1.177768in}{1.147371in}}%
\pgfpathlineto{\pgfqpoint{1.187556in}{1.157705in}}%
\pgfpathlineto{\pgfqpoint{1.199325in}{1.167341in}}%
\pgfpathlineto{\pgfqpoint{1.213163in}{1.176281in}}%
\pgfpathlineto{\pgfqpoint{1.228988in}{1.184524in}}%
\pgfpathlineto{\pgfqpoint{1.256335in}{1.195575in}}%
\pgfpathlineto{\pgfqpoint{1.288042in}{1.205043in}}%
\pgfpathlineto{\pgfqpoint{1.324245in}{1.212916in}}%
\pgfpathlineto{\pgfqpoint{1.365191in}{1.219183in}}%
\pgfpathlineto{\pgfqpoint{1.411242in}{1.223829in}}%
\pgfpathlineto{\pgfqpoint{1.462871in}{1.226838in}}%
\pgfpathlineto{\pgfqpoint{1.520579in}{1.228194in}}%
\pgfpathlineto{\pgfqpoint{1.584497in}{1.227860in}}%
\pgfpathlineto{\pgfqpoint{1.681369in}{1.224626in}}%
\pgfpathlineto{\pgfqpoint{1.793702in}{1.217996in}}%
\pgfpathlineto{\pgfqpoint{1.922809in}{1.207751in}}%
\pgfpathlineto{\pgfqpoint{2.068902in}{1.193682in}}%
\pgfpathlineto{\pgfqpoint{2.231092in}{1.175589in}}%
\pgfpathlineto{\pgfqpoint{2.407389in}{1.153280in}}%
\pgfpathlineto{\pgfqpoint{2.546197in}{1.133688in}}%
\pgfpathlineto{\pgfqpoint{2.685096in}{1.111807in}}%
\pgfpathlineto{\pgfqpoint{2.818932in}{1.087920in}}%
\pgfpathlineto{\pgfqpoint{2.903210in}{1.071044in}}%
\pgfpathlineto{\pgfqpoint{2.982302in}{1.053533in}}%
\pgfpathlineto{\pgfqpoint{3.055356in}{1.035510in}}%
\pgfpathlineto{\pgfqpoint{3.121702in}{1.017104in}}%
\pgfpathlineto{\pgfqpoint{3.180849in}{0.998455in}}%
\pgfpathlineto{\pgfqpoint{3.232485in}{0.979713in}}%
\pgfpathlineto{\pgfqpoint{3.276745in}{0.961036in}}%
\pgfpathlineto{\pgfqpoint{3.314520in}{0.942539in}}%
\pgfpathlineto{\pgfqpoint{3.346663in}{0.924309in}}%
\pgfpathlineto{\pgfqpoint{3.373849in}{0.906427in}}%
\pgfpathlineto{\pgfqpoint{3.396569in}{0.888964in}}%
\pgfpathlineto{\pgfqpoint{3.415132in}{0.871984in}}%
\pgfpathlineto{\pgfqpoint{3.429668in}{0.855541in}}%
\pgfpathlineto{\pgfqpoint{3.440125in}{0.839680in}}%
\pgfpathlineto{\pgfqpoint{3.446964in}{0.824441in}}%
\pgfpathlineto{\pgfqpoint{3.450928in}{0.809848in}}%
\pgfpathlineto{\pgfqpoint{3.452326in}{0.795918in}}%
\pgfpathlineto{\pgfqpoint{3.451396in}{0.782662in}}%
\pgfpathlineto{\pgfqpoint{3.448300in}{0.770090in}}%
\pgfpathlineto{\pgfqpoint{3.443125in}{0.758209in}}%
\pgfpathlineto{\pgfqpoint{3.435886in}{0.747022in}}%
\pgfpathlineto{\pgfqpoint{3.426523in}{0.736528in}}%
\pgfpathlineto{\pgfqpoint{3.415049in}{0.726729in}}%
\pgfpathlineto{\pgfqpoint{3.401619in}{0.717627in}}%
\pgfpathlineto{\pgfqpoint{3.386286in}{0.709224in}}%
\pgfpathlineto{\pgfqpoint{3.359761in}{0.697938in}}%
\pgfpathlineto{\pgfqpoint{3.328969in}{0.688238in}}%
\pgfpathlineto{\pgfqpoint{3.293743in}{0.680129in}}%
\pgfpathlineto{\pgfqpoint{3.253781in}{0.673617in}}%
\pgfpathlineto{\pgfqpoint{3.208646in}{0.668705in}}%
\pgfpathlineto{\pgfqpoint{3.157973in}{0.665403in}}%
\pgfpathlineto{\pgfqpoint{3.101514in}{0.663758in}}%
\pgfpathlineto{\pgfqpoint{3.038438in}{0.663826in}}%
\pgfpathlineto{\pgfqpoint{2.967976in}{0.665672in}}%
\pgfpathlineto{\pgfqpoint{2.861391in}{0.671026in}}%
\pgfpathlineto{\pgfqpoint{2.739171in}{0.679872in}}%
\pgfpathlineto{\pgfqpoint{2.600300in}{0.692438in}}%
\pgfpathlineto{\pgfqpoint{2.443937in}{0.708984in}}%
\pgfpathlineto{\pgfqpoint{2.269066in}{0.729821in}}%
\pgfpathlineto{\pgfqpoint{2.131205in}{0.748270in}}%
\pgfpathlineto{\pgfqpoint{1.992821in}{0.769016in}}%
\pgfpathlineto{\pgfqpoint{1.858056in}{0.791872in}}%
\pgfpathlineto{\pgfqpoint{1.730528in}{0.816580in}}%
\pgfpathlineto{\pgfqpoint{1.651083in}{0.833920in}}%
\pgfpathlineto{\pgfqpoint{1.577033in}{0.851822in}}%
\pgfpathlineto{\pgfqpoint{1.509066in}{0.870155in}}%
\pgfpathlineto{\pgfqpoint{1.447767in}{0.888775in}}%
\pgfpathlineto{\pgfqpoint{1.393616in}{0.907524in}}%
\pgfpathlineto{\pgfqpoint{1.346986in}{0.926229in}}%
\pgfpathlineto{\pgfqpoint{1.307308in}{0.944770in}}%
\pgfpathlineto{\pgfqpoint{1.273665in}{0.963063in}}%
\pgfpathlineto{\pgfqpoint{1.245492in}{0.981019in}}%
\pgfpathlineto{\pgfqpoint{1.222273in}{0.998560in}}%
\pgfpathlineto{\pgfqpoint{1.203540in}{1.015621in}}%
\pgfpathlineto{\pgfqpoint{1.188833in}{1.032147in}}%
\pgfpathlineto{\pgfqpoint{1.177703in}{1.048098in}}%
\pgfpathlineto{\pgfqpoint{1.169788in}{1.063442in}}%
\pgfpathlineto{\pgfqpoint{1.164795in}{1.078153in}}%
\pgfpathlineto{\pgfqpoint{1.162501in}{1.092211in}}%
\pgfpathlineto{\pgfqpoint{1.162803in}{1.105602in}}%
\pgfpathlineto{\pgfqpoint{1.165519in}{1.118316in}}%
\pgfpathlineto{\pgfqpoint{1.170409in}{1.130345in}}%
\pgfpathlineto{\pgfqpoint{1.177289in}{1.141683in}}%
\pgfpathlineto{\pgfqpoint{1.186035in}{1.152324in}}%
\pgfpathlineto{\pgfqpoint{1.196583in}{1.162268in}}%
\pgfpathlineto{\pgfqpoint{1.208925in}{1.171514in}}%
\pgfpathlineto{\pgfqpoint{1.223115in}{1.180063in}}%
\pgfpathlineto{\pgfqpoint{1.239263in}{1.187919in}}%
\pgfpathlineto{\pgfqpoint{1.257540in}{1.195087in}}%
\pgfpathlineto{\pgfqpoint{1.289044in}{1.204551in}}%
\pgfpathlineto{\pgfqpoint{1.325142in}{1.212437in}}%
\pgfpathlineto{\pgfqpoint{1.366051in}{1.218731in}}%
\pgfpathlineto{\pgfqpoint{1.412091in}{1.223410in}}%
\pgfpathlineto{\pgfqpoint{1.463667in}{1.226444in}}%
\pgfpathlineto{\pgfqpoint{1.521273in}{1.227799in}}%
\pgfpathlineto{\pgfqpoint{1.585489in}{1.227433in}}%
\pgfpathlineto{\pgfqpoint{1.682489in}{1.224185in}}%
\pgfpathlineto{\pgfqpoint{1.793915in}{1.217613in}}%
\pgfpathlineto{\pgfqpoint{1.921995in}{1.207449in}}%
\pgfpathlineto{\pgfqpoint{2.067587in}{1.193447in}}%
\pgfpathlineto{\pgfqpoint{2.229847in}{1.175399in}}%
\pgfpathlineto{\pgfqpoint{2.406235in}{1.153132in}}%
\pgfpathlineto{\pgfqpoint{2.544914in}{1.133583in}}%
\pgfpathlineto{\pgfqpoint{2.683644in}{1.111753in}}%
\pgfpathlineto{\pgfqpoint{2.817357in}{1.087917in}}%
\pgfpathlineto{\pgfqpoint{2.901601in}{1.071070in}}%
\pgfpathlineto{\pgfqpoint{2.980700in}{1.053586in}}%
\pgfpathlineto{\pgfqpoint{3.053798in}{1.035585in}}%
\pgfpathlineto{\pgfqpoint{3.120208in}{1.017197in}}%
\pgfpathlineto{\pgfqpoint{3.179415in}{0.998566in}}%
\pgfpathlineto{\pgfqpoint{3.231083in}{0.979844in}}%
\pgfpathlineto{\pgfqpoint{3.275446in}{0.961187in}}%
\pgfpathlineto{\pgfqpoint{3.313400in}{0.942702in}}%
\pgfpathlineto{\pgfqpoint{3.345742in}{0.924480in}}%
\pgfpathlineto{\pgfqpoint{3.373101in}{0.906604in}}%
\pgfpathlineto{\pgfqpoint{3.395934in}{0.889145in}}%
\pgfpathlineto{\pgfqpoint{3.414533in}{0.872168in}}%
\pgfpathlineto{\pgfqpoint{3.429016in}{0.855728in}}%
\pgfpathlineto{\pgfqpoint{3.439375in}{0.839872in}}%
\pgfpathlineto{\pgfqpoint{3.446286in}{0.824636in}}%
\pgfpathlineto{\pgfqpoint{3.450335in}{0.810045in}}%
\pgfpathlineto{\pgfqpoint{3.451828in}{0.796115in}}%
\pgfpathlineto{\pgfqpoint{3.450994in}{0.782857in}}%
\pgfpathlineto{\pgfqpoint{3.447988in}{0.770283in}}%
\pgfpathlineto{\pgfqpoint{3.442883in}{0.758398in}}%
\pgfpathlineto{\pgfqpoint{3.435679in}{0.747206in}}%
\pgfpathlineto{\pgfqpoint{3.426305in}{0.736706in}}%
\pgfpathlineto{\pgfqpoint{3.414860in}{0.726900in}}%
\pgfpathlineto{\pgfqpoint{3.401469in}{0.717792in}}%
\pgfpathlineto{\pgfqpoint{3.386184in}{0.709383in}}%
\pgfpathlineto{\pgfqpoint{3.359745in}{0.698088in}}%
\pgfpathlineto{\pgfqpoint{3.329048in}{0.688379in}}%
\pgfpathlineto{\pgfqpoint{3.293914in}{0.680259in}}%
\pgfpathlineto{\pgfqpoint{3.254022in}{0.673733in}}%
\pgfpathlineto{\pgfqpoint{3.208917in}{0.668802in}}%
\pgfpathlineto{\pgfqpoint{3.158344in}{0.665482in}}%
\pgfpathlineto{\pgfqpoint{3.101925in}{0.663822in}}%
\pgfpathlineto{\pgfqpoint{3.038871in}{0.663875in}}%
\pgfpathlineto{\pgfqpoint{2.968465in}{0.665704in}}%
\pgfpathlineto{\pgfqpoint{2.862041in}{0.671033in}}%
\pgfpathlineto{\pgfqpoint{2.740055in}{0.679849in}}%
\pgfpathlineto{\pgfqpoint{2.601379in}{0.692384in}}%
\pgfpathlineto{\pgfqpoint{2.444381in}{0.708923in}}%
\pgfpathlineto{\pgfqpoint{2.268020in}{0.729808in}}%
\pgfpathlineto{\pgfqpoint{2.129637in}{0.748280in}}%
\pgfpathlineto{\pgfqpoint{1.991307in}{0.769022in}}%
\pgfpathlineto{\pgfqpoint{1.857123in}{0.791842in}}%
\pgfpathlineto{\pgfqpoint{1.730557in}{0.816488in}}%
\pgfpathlineto{\pgfqpoint{1.651844in}{0.833779in}}%
\pgfpathlineto{\pgfqpoint{1.578487in}{0.851635in}}%
\pgfpathlineto{\pgfqpoint{1.511056in}{0.869934in}}%
\pgfpathlineto{\pgfqpoint{1.449997in}{0.888543in}}%
\pgfpathlineto{\pgfqpoint{1.395632in}{0.907317in}}%
\pgfpathlineto{\pgfqpoint{1.348163in}{0.926098in}}%
\pgfpathlineto{\pgfqpoint{1.307668in}{0.944717in}}%
\pgfpathlineto{\pgfqpoint{1.273912in}{0.963019in}}%
\pgfpathlineto{\pgfqpoint{1.245754in}{0.980971in}}%
\pgfpathlineto{\pgfqpoint{1.222548in}{0.998509in}}%
\pgfpathlineto{\pgfqpoint{1.203790in}{1.015567in}}%
\pgfpathlineto{\pgfqpoint{1.189038in}{1.032093in}}%
\pgfpathlineto{\pgfqpoint{1.177880in}{1.048043in}}%
\pgfpathlineto{\pgfqpoint{1.169953in}{1.063387in}}%
\pgfpathlineto{\pgfqpoint{1.164957in}{1.078097in}}%
\pgfpathlineto{\pgfqpoint{1.162653in}{1.092155in}}%
\pgfpathlineto{\pgfqpoint{1.162930in}{1.105547in}}%
\pgfpathlineto{\pgfqpoint{1.165647in}{1.118262in}}%
\pgfpathlineto{\pgfqpoint{1.170550in}{1.130292in}}%
\pgfpathlineto{\pgfqpoint{1.177447in}{1.141631in}}%
\pgfpathlineto{\pgfqpoint{1.186201in}{1.152274in}}%
\pgfpathlineto{\pgfqpoint{1.196739in}{1.162219in}}%
\pgfpathlineto{\pgfqpoint{1.209047in}{1.171466in}}%
\pgfpathlineto{\pgfqpoint{1.223169in}{1.180016in}}%
\pgfpathlineto{\pgfqpoint{1.239213in}{1.187872in}}%
\pgfpathlineto{\pgfqpoint{1.257343in}{1.195040in}}%
\pgfpathlineto{\pgfqpoint{1.288749in}{1.204508in}}%
\pgfpathlineto{\pgfqpoint{1.324819in}{1.212405in}}%
\pgfpathlineto{\pgfqpoint{1.365700in}{1.218710in}}%
\pgfpathlineto{\pgfqpoint{1.411715in}{1.223401in}}%
\pgfpathlineto{\pgfqpoint{1.463269in}{1.226449in}}%
\pgfpathlineto{\pgfqpoint{1.520853in}{1.227816in}}%
\pgfpathlineto{\pgfqpoint{1.585041in}{1.227462in}}%
\pgfpathlineto{\pgfqpoint{1.682029in}{1.224221in}}%
\pgfpathlineto{\pgfqpoint{1.793443in}{1.217659in}}%
\pgfpathlineto{\pgfqpoint{1.921388in}{1.207513in}}%
\pgfpathlineto{\pgfqpoint{2.066812in}{1.193535in}}%
\pgfpathlineto{\pgfqpoint{2.228988in}{1.175510in}}%
\pgfpathlineto{\pgfqpoint{2.405564in}{1.153251in}}%
\pgfpathlineto{\pgfqpoint{2.544179in}{1.133732in}}%
\pgfpathlineto{\pgfqpoint{2.682661in}{1.111948in}}%
\pgfpathlineto{\pgfqpoint{2.816272in}{1.088137in}}%
\pgfpathlineto{\pgfqpoint{2.900580in}{1.071290in}}%
\pgfpathlineto{\pgfqpoint{2.979809in}{1.053793in}}%
\pgfpathlineto{\pgfqpoint{3.053012in}{1.035773in}}%
\pgfpathlineto{\pgfqpoint{3.119363in}{1.017372in}}%
\pgfpathlineto{\pgfqpoint{3.178192in}{0.998746in}}%
\pgfpathlineto{\pgfqpoint{3.229598in}{0.980048in}}%
\pgfpathlineto{\pgfqpoint{3.274242in}{0.961400in}}%
\pgfpathlineto{\pgfqpoint{3.312721in}{0.942910in}}%
\pgfpathlineto{\pgfqpoint{3.345548in}{0.924679in}}%
\pgfpathlineto{\pgfqpoint{3.373148in}{0.906793in}}%
\pgfpathlineto{\pgfqpoint{3.395860in}{0.889330in}}%
\pgfpathlineto{\pgfqpoint{3.413937in}{0.872357in}}%
\pgfpathlineto{\pgfqpoint{3.427694in}{0.855930in}}%
\pgfpathlineto{\pgfqpoint{3.437866in}{0.840093in}}%
\pgfpathlineto{\pgfqpoint{3.444927in}{0.824871in}}%
\pgfpathlineto{\pgfqpoint{3.449233in}{0.810288in}}%
\pgfpathlineto{\pgfqpoint{3.451044in}{0.796361in}}%
\pgfpathlineto{\pgfqpoint{3.450517in}{0.783105in}}%
\pgfpathlineto{\pgfqpoint{3.447715in}{0.770529in}}%
\pgfpathlineto{\pgfqpoint{3.442598in}{0.758638in}}%
\pgfpathlineto{\pgfqpoint{3.435188in}{0.747435in}}%
\pgfpathlineto{\pgfqpoint{3.425724in}{0.736927in}}%
\pgfpathlineto{\pgfqpoint{3.414313in}{0.727116in}}%
\pgfpathlineto{\pgfqpoint{3.401023in}{0.718006in}}%
\pgfpathlineto{\pgfqpoint{3.385893in}{0.709597in}}%
\pgfpathlineto{\pgfqpoint{3.359745in}{0.698302in}}%
\pgfpathlineto{\pgfqpoint{3.329326in}{0.688588in}}%
\pgfpathlineto{\pgfqpoint{3.294340in}{0.680452in}}%
\pgfpathlineto{\pgfqpoint{3.254356in}{0.673887in}}%
\pgfpathlineto{\pgfqpoint{3.209294in}{0.668917in}}%
\pgfpathlineto{\pgfqpoint{3.158778in}{0.665575in}}%
\pgfpathlineto{\pgfqpoint{3.102259in}{0.663899in}}%
\pgfpathlineto{\pgfqpoint{3.039156in}{0.663939in}}%
\pgfpathlineto{\pgfqpoint{2.968861in}{0.665753in}}%
\pgfpathlineto{\pgfqpoint{2.862837in}{0.671057in}}%
\pgfpathlineto{\pgfqpoint{2.741293in}{0.679839in}}%
\pgfpathlineto{\pgfqpoint{2.602732in}{0.692332in}}%
\pgfpathlineto{\pgfqpoint{2.446669in}{0.708807in}}%
\pgfpathlineto{\pgfqpoint{2.274894in}{0.729467in}}%
\pgfpathlineto{\pgfqpoint{2.138604in}{0.747758in}}%
\pgfpathlineto{\pgfqpoint{1.999540in}{0.768407in}}%
\pgfpathlineto{\pgfqpoint{1.862599in}{0.791238in}}%
\pgfpathlineto{\pgfqpoint{1.732552in}{0.815972in}}%
\pgfpathlineto{\pgfqpoint{1.651792in}{0.833340in}}%
\pgfpathlineto{\pgfqpoint{1.577099in}{0.851259in}}%
\pgfpathlineto{\pgfqpoint{1.509575in}{0.869579in}}%
\pgfpathlineto{\pgfqpoint{1.449393in}{0.888149in}}%
\pgfpathlineto{\pgfqpoint{1.396144in}{0.906831in}}%
\pgfpathlineto{\pgfqpoint{1.349439in}{0.925501in}}%
\pgfpathlineto{\pgfqpoint{1.308906in}{0.944045in}}%
\pgfpathlineto{\pgfqpoint{1.274198in}{0.962358in}}%
\pgfpathlineto{\pgfqpoint{1.244983in}{0.980349in}}%
\pgfpathlineto{\pgfqpoint{1.220952in}{0.997938in}}%
\pgfpathlineto{\pgfqpoint{1.201814in}{1.015054in}}%
\pgfpathlineto{\pgfqpoint{1.187221in}{1.031636in}}%
\pgfpathlineto{\pgfqpoint{1.176455in}{1.047634in}}%
\pgfpathlineto{\pgfqpoint{1.168884in}{1.063020in}}%
\pgfpathlineto{\pgfqpoint{1.164026in}{1.077772in}}%
\pgfpathlineto{\pgfqpoint{1.161533in}{1.091873in}}%
\pgfpathlineto{\pgfqpoint{1.161194in}{1.105308in}}%
\pgfpathlineto{\pgfqpoint{1.162932in}{1.118069in}}%
\pgfpathlineto{\pgfqpoint{1.166807in}{1.130149in}}%
\pgfpathlineto{\pgfqpoint{1.173013in}{1.141545in}}%
\pgfpathlineto{\pgfqpoint{1.181825in}{1.152260in}}%
\pgfpathlineto{\pgfqpoint{1.192813in}{1.162279in}}%
\pgfpathlineto{\pgfqpoint{1.205735in}{1.171598in}}%
\pgfpathlineto{\pgfqpoint{1.220544in}{1.180214in}}%
\pgfpathlineto{\pgfqpoint{1.246255in}{1.191818in}}%
\pgfpathlineto{\pgfqpoint{1.276213in}{1.201835in}}%
\pgfpathlineto{\pgfqpoint{1.310603in}{1.210261in}}%
\pgfpathlineto{\pgfqpoint{1.349746in}{1.217098in}}%
\pgfpathlineto{\pgfqpoint{1.394041in}{1.222343in}}%
\pgfpathlineto{\pgfqpoint{1.443624in}{1.225969in}}%
\pgfpathlineto{\pgfqpoint{1.499070in}{1.227937in}}%
\pgfpathlineto{\pgfqpoint{1.561038in}{1.228198in}}%
\pgfpathlineto{\pgfqpoint{1.630175in}{1.226693in}}%
\pgfpathlineto{\pgfqpoint{1.734594in}{1.221818in}}%
\pgfpathlineto{\pgfqpoint{1.854325in}{1.213483in}}%
\pgfpathlineto{\pgfqpoint{1.990769in}{1.201459in}}%
\pgfpathlineto{\pgfqpoint{2.144932in}{1.185507in}}%
\pgfpathlineto{\pgfqpoint{2.315212in}{1.165368in}}%
\pgfpathlineto{\pgfqpoint{2.450708in}{1.147437in}}%
\pgfpathlineto{\pgfqpoint{2.589449in}{1.127143in}}%
\pgfpathlineto{\pgfqpoint{2.727305in}{1.104639in}}%
\pgfpathlineto{\pgfqpoint{2.859181in}{1.080172in}}%
\pgfpathlineto{\pgfqpoint{2.941149in}{1.062921in}}%
\pgfpathlineto{\pgfqpoint{3.017220in}{1.045085in}}%
\pgfpathlineto{\pgfqpoint{3.086776in}{1.026824in}}%
\pgfpathlineto{\pgfqpoint{3.149414in}{1.008287in}}%
\pgfpathlineto{\pgfqpoint{3.204944in}{0.989611in}}%
\pgfpathlineto{\pgfqpoint{3.253390in}{0.970923in}}%
\pgfpathlineto{\pgfqpoint{3.294993in}{0.952340in}}%
\pgfpathlineto{\pgfqpoint{3.330205in}{0.933968in}}%
\pgfpathlineto{\pgfqpoint{3.359695in}{0.915901in}}%
\pgfpathlineto{\pgfqpoint{3.384048in}{0.898231in}}%
\pgfpathlineto{\pgfqpoint{3.403457in}{0.881043in}}%
\pgfpathlineto{\pgfqpoint{3.418833in}{0.864375in}}%
\pgfpathlineto{\pgfqpoint{3.430927in}{0.848261in}}%
\pgfpathlineto{\pgfqpoint{3.440274in}{0.832732in}}%
\pgfpathlineto{\pgfqpoint{3.447189in}{0.817813in}}%
\pgfpathlineto{\pgfqpoint{3.451772in}{0.803524in}}%
\pgfpathlineto{\pgfqpoint{3.453904in}{0.789883in}}%
\pgfpathlineto{\pgfqpoint{3.453251in}{0.776903in}}%
\pgfpathlineto{\pgfqpoint{3.449300in}{0.764591in}}%
\pgfpathlineto{\pgfqpoint{3.442728in}{0.752970in}}%
\pgfpathlineto{\pgfqpoint{3.434124in}{0.742050in}}%
\pgfpathlineto{\pgfqpoint{3.423586in}{0.731834in}}%
\pgfpathlineto{\pgfqpoint{3.411181in}{0.722322in}}%
\pgfpathlineto{\pgfqpoint{3.396940in}{0.713516in}}%
\pgfpathlineto{\pgfqpoint{3.380860in}{0.705416in}}%
\pgfpathlineto{\pgfqpoint{3.353199in}{0.694587in}}%
\pgfpathlineto{\pgfqpoint{3.321035in}{0.685336in}}%
\pgfpathlineto{\pgfqpoint{3.284101in}{0.677661in}}%
\pgfpathlineto{\pgfqpoint{3.242342in}{0.671580in}}%
\pgfpathlineto{\pgfqpoint{3.195390in}{0.667116in}}%
\pgfpathlineto{\pgfqpoint{3.142804in}{0.664302in}}%
\pgfpathlineto{\pgfqpoint{3.084075in}{0.663178in}}%
\pgfpathlineto{\pgfqpoint{3.018629in}{0.663793in}}%
\pgfpathlineto{\pgfqpoint{2.919805in}{0.667423in}}%
\pgfpathlineto{\pgfqpoint{2.806206in}{0.674416in}}%
\pgfpathlineto{\pgfqpoint{2.676192in}{0.685002in}}%
\pgfpathlineto{\pgfqpoint{2.528653in}{0.699448in}}%
\pgfpathlineto{\pgfqpoint{2.364251in}{0.717986in}}%
\pgfpathlineto{\pgfqpoint{2.185933in}{0.740781in}}%
\pgfpathlineto{\pgfqpoint{2.047270in}{0.760662in}}%
\pgfpathlineto{\pgfqpoint{1.908942in}{0.782798in}}%
\pgfpathlineto{\pgfqpoint{1.775654in}{0.806955in}}%
\pgfpathlineto{\pgfqpoint{1.692194in}{0.824025in}}%
\pgfpathlineto{\pgfqpoint{1.614593in}{0.841734in}}%
\pgfpathlineto{\pgfqpoint{1.543418in}{0.859916in}}%
\pgfpathlineto{\pgfqpoint{1.479034in}{0.878413in}}%
\pgfpathlineto{\pgfqpoint{1.421636in}{0.897079in}}%
\pgfpathlineto{\pgfqpoint{1.371246in}{0.915783in}}%
\pgfpathlineto{\pgfqpoint{1.327714in}{0.934404in}}%
\pgfpathlineto{\pgfqpoint{1.290714in}{0.952834in}}%
\pgfpathlineto{\pgfqpoint{1.259752in}{0.970980in}}%
\pgfpathlineto{\pgfqpoint{1.234159in}{0.988760in}}%
\pgfpathlineto{\pgfqpoint{1.213176in}{1.006100in}}%
\pgfpathlineto{\pgfqpoint{1.196735in}{1.022924in}}%
\pgfpathlineto{\pgfqpoint{1.184173in}{1.039189in}}%
\pgfpathlineto{\pgfqpoint{1.174764in}{1.054867in}}%
\pgfpathlineto{\pgfqpoint{1.167972in}{1.069934in}}%
\pgfpathlineto{\pgfqpoint{1.163447in}{1.084371in}}%
\pgfpathlineto{\pgfqpoint{1.161032in}{1.098160in}}%
\pgfpathlineto{\pgfqpoint{1.160756in}{1.111290in}}%
\pgfpathlineto{\pgfqpoint{1.162839in}{1.123752in}}%
\pgfpathlineto{\pgfqpoint{1.167688in}{1.135543in}}%
\pgfpathlineto{\pgfqpoint{1.175577in}{1.146656in}}%
\pgfpathlineto{\pgfqpoint{1.185590in}{1.157070in}}%
\pgfpathlineto{\pgfqpoint{1.197534in}{1.166781in}}%
\pgfpathlineto{\pgfqpoint{1.211355in}{1.175787in}}%
\pgfpathlineto{\pgfqpoint{1.227025in}{1.184088in}}%
\pgfpathlineto{\pgfqpoint{1.254015in}{1.195215in}}%
\pgfpathlineto{\pgfqpoint{1.285316in}{1.204753in}}%
\pgfpathlineto{\pgfqpoint{1.321212in}{1.212705in}}%
\pgfpathlineto{\pgfqpoint{1.362059in}{1.219073in}}%
\pgfpathlineto{\pgfqpoint{1.407967in}{1.223836in}}%
\pgfpathlineto{\pgfqpoint{1.459405in}{1.226961in}}%
\pgfpathlineto{\pgfqpoint{1.516927in}{1.228409in}}%
\pgfpathlineto{\pgfqpoint{1.581121in}{1.228130in}}%
\pgfpathlineto{\pgfqpoint{1.678156in}{1.224965in}}%
\pgfpathlineto{\pgfqpoint{1.789687in}{1.218445in}}%
\pgfpathlineto{\pgfqpoint{1.917346in}{1.208362in}}%
\pgfpathlineto{\pgfqpoint{2.062375in}{1.194470in}}%
\pgfpathlineto{\pgfqpoint{2.224655in}{1.176500in}}%
\pgfpathlineto{\pgfqpoint{2.401282in}{1.154297in}}%
\pgfpathlineto{\pgfqpoint{2.539632in}{1.134848in}}%
\pgfpathlineto{\pgfqpoint{2.678538in}{1.113104in}}%
\pgfpathlineto{\pgfqpoint{2.812953in}{1.089296in}}%
\pgfpathlineto{\pgfqpoint{2.897809in}{1.072433in}}%
\pgfpathlineto{\pgfqpoint{2.977449in}{1.054913in}}%
\pgfpathlineto{\pgfqpoint{3.050790in}{1.036872in}}%
\pgfpathlineto{\pgfqpoint{3.116804in}{1.018463in}}%
\pgfpathlineto{\pgfqpoint{3.175517in}{0.999835in}}%
\pgfpathlineto{\pgfqpoint{3.227348in}{0.981122in}}%
\pgfpathlineto{\pgfqpoint{3.272692in}{0.962448in}}%
\pgfpathlineto{\pgfqpoint{3.311924in}{0.943925in}}%
\pgfpathlineto{\pgfqpoint{3.345397in}{0.925653in}}%
\pgfpathlineto{\pgfqpoint{3.373441in}{0.907724in}}%
\pgfpathlineto{\pgfqpoint{3.396363in}{0.890214in}}%
\pgfpathlineto{\pgfqpoint{3.414451in}{0.873191in}}%
\pgfpathlineto{\pgfqpoint{3.428146in}{0.856717in}}%
\pgfpathlineto{\pgfqpoint{3.438198in}{0.840836in}}%
\pgfpathlineto{\pgfqpoint{3.445182in}{0.825571in}}%
\pgfpathlineto{\pgfqpoint{3.449538in}{0.810945in}}%
\pgfpathlineto{\pgfqpoint{3.451573in}{0.796974in}}%
\pgfpathlineto{\pgfqpoint{3.451460in}{0.783671in}}%
\pgfpathlineto{\pgfqpoint{3.449236in}{0.771046in}}%
\pgfpathlineto{\pgfqpoint{3.444805in}{0.759103in}}%
\pgfpathlineto{\pgfqpoint{3.437937in}{0.747843in}}%
\pgfpathlineto{\pgfqpoint{3.428546in}{0.737269in}}%
\pgfpathlineto{\pgfqpoint{3.417145in}{0.727395in}}%
\pgfpathlineto{\pgfqpoint{3.403828in}{0.718222in}}%
\pgfpathlineto{\pgfqpoint{3.388639in}{0.709752in}}%
\pgfpathlineto{\pgfqpoint{3.362364in}{0.698368in}}%
\pgfpathlineto{\pgfqpoint{3.331828in}{0.688573in}}%
\pgfpathlineto{\pgfqpoint{3.296816in}{0.680366in}}%
\pgfpathlineto{\pgfqpoint{3.256974in}{0.673747in}}%
\pgfpathlineto{\pgfqpoint{3.211962in}{0.668722in}}%
\pgfpathlineto{\pgfqpoint{3.161600in}{0.665322in}}%
\pgfpathlineto{\pgfqpoint{3.105270in}{0.663587in}}%
\pgfpathlineto{\pgfqpoint{3.042334in}{0.663566in}}%
\pgfpathlineto{\pgfqpoint{2.972158in}{0.665320in}}%
\pgfpathlineto{\pgfqpoint{2.866245in}{0.670545in}}%
\pgfpathlineto{\pgfqpoint{2.744876in}{0.679251in}}%
\pgfpathlineto{\pgfqpoint{2.606598in}{0.691672in}}%
\pgfpathlineto{\pgfqpoint{2.450731in}{0.708049in}}%
\pgfpathlineto{\pgfqpoint{2.279056in}{0.728642in}}%
\pgfpathlineto{\pgfqpoint{2.142811in}{0.746899in}}%
\pgfpathlineto{\pgfqpoint{2.003793in}{0.767496in}}%
\pgfpathlineto{\pgfqpoint{1.866456in}{0.790279in}}%
\pgfpathlineto{\pgfqpoint{1.778429in}{0.806578in}}%
\pgfpathlineto{\pgfqpoint{1.694858in}{0.823636in}}%
\pgfpathlineto{\pgfqpoint{1.616823in}{0.841304in}}%
\pgfpathlineto{\pgfqpoint{1.545138in}{0.859435in}}%
\pgfpathlineto{\pgfqpoint{1.480353in}{0.877890in}}%
\pgfpathlineto{\pgfqpoint{1.422748in}{0.896535in}}%
\pgfpathlineto{\pgfqpoint{1.372339in}{0.915242in}}%
\pgfpathlineto{\pgfqpoint{1.328874in}{0.933888in}}%
\pgfpathlineto{\pgfqpoint{1.291841in}{0.952354in}}%
\pgfpathlineto{\pgfqpoint{1.261021in}{0.970513in}}%
\pgfpathlineto{\pgfqpoint{1.235653in}{0.988283in}}%
\pgfpathlineto{\pgfqpoint{1.214712in}{1.005612in}}%
\pgfpathlineto{\pgfqpoint{1.197423in}{1.022455in}}%
\pgfpathlineto{\pgfqpoint{1.183268in}{1.038770in}}%
\pgfpathlineto{\pgfqpoint{1.171981in}{1.054519in}}%
\pgfpathlineto{\pgfqpoint{1.163549in}{1.069669in}}%
\pgfpathlineto{\pgfqpoint{1.158216in}{1.084194in}}%
\pgfpathlineto{\pgfqpoint{1.156476in}{1.098069in}}%
\pgfpathlineto{\pgfqpoint{1.158118in}{1.111269in}}%
\pgfpathlineto{\pgfqpoint{1.162032in}{1.123777in}}%
\pgfpathlineto{\pgfqpoint{1.168043in}{1.135586in}}%
\pgfpathlineto{\pgfqpoint{1.176025in}{1.146695in}}%
\pgfpathlineto{\pgfqpoint{1.185898in}{1.157099in}}%
\pgfpathlineto{\pgfqpoint{1.197624in}{1.166798in}}%
\pgfpathlineto{\pgfqpoint{1.211214in}{1.175794in}}%
\pgfpathlineto{\pgfqpoint{1.226719in}{1.184087in}}%
\pgfpathlineto{\pgfqpoint{1.244237in}{1.191682in}}%
\pgfpathlineto{\pgfqpoint{1.274310in}{1.201767in}}%
\pgfpathlineto{\pgfqpoint{1.308878in}{1.210270in}}%
\pgfpathlineto{\pgfqpoint{1.348136in}{1.217177in}}%
\pgfpathlineto{\pgfqpoint{1.392370in}{1.222469in}}%
\pgfpathlineto{\pgfqpoint{1.441960in}{1.226121in}}%
\pgfpathlineto{\pgfqpoint{1.497379in}{1.228102in}}%
\pgfpathlineto{\pgfqpoint{1.559193in}{1.228378in}}%
\pgfpathlineto{\pgfqpoint{1.628050in}{1.226908in}}%
\pgfpathlineto{\pgfqpoint{1.731609in}{1.222134in}}%
\pgfpathlineto{\pgfqpoint{1.850964in}{1.213890in}}%
\pgfpathlineto{\pgfqpoint{1.987804in}{1.201906in}}%
\pgfpathlineto{\pgfqpoint{2.142012in}{1.185963in}}%
\pgfpathlineto{\pgfqpoint{2.311666in}{1.165892in}}%
\pgfpathlineto{\pgfqpoint{2.446870in}{1.148055in}}%
\pgfpathlineto{\pgfqpoint{2.586091in}{1.127792in}}%
\pgfpathlineto{\pgfqpoint{2.724207in}{1.105275in}}%
\pgfpathlineto{\pgfqpoint{2.856040in}{1.080835in}}%
\pgfpathlineto{\pgfqpoint{2.938349in}{1.063652in}}%
\pgfpathlineto{\pgfqpoint{3.015031in}{1.045893in}}%
\pgfpathlineto{\pgfqpoint{3.085318in}{1.027683in}}%
\pgfpathlineto{\pgfqpoint{3.148648in}{1.009151in}}%
\pgfpathlineto{\pgfqpoint{3.204663in}{0.990438in}}%
\pgfpathlineto{\pgfqpoint{3.253210in}{0.971689in}}%
\pgfpathlineto{\pgfqpoint{3.294534in}{0.953062in}}%
\pgfpathlineto{\pgfqpoint{3.329589in}{0.934661in}}%
\pgfpathlineto{\pgfqpoint{3.359280in}{0.916569in}}%
\pgfpathlineto{\pgfqpoint{3.384303in}{0.898857in}}%
\pgfpathlineto{\pgfqpoint{3.405151in}{0.881591in}}%
\pgfpathlineto{\pgfqpoint{3.422110in}{0.864829in}}%
\pgfpathlineto{\pgfqpoint{3.435258in}{0.848621in}}%
\pgfpathlineto{\pgfqpoint{3.444470in}{0.833009in}}%
\pgfpathlineto{\pgfqpoint{3.449991in}{0.818029in}}%
\pgfpathlineto{\pgfqpoint{3.452725in}{0.803706in}}%
\pgfpathlineto{\pgfqpoint{3.452973in}{0.790053in}}%
\pgfpathlineto{\pgfqpoint{3.450958in}{0.777081in}}%
\pgfpathlineto{\pgfqpoint{3.446836in}{0.764799in}}%
\pgfpathlineto{\pgfqpoint{3.440695in}{0.753210in}}%
\pgfpathlineto{\pgfqpoint{3.432555in}{0.742319in}}%
\pgfpathlineto{\pgfqpoint{3.422366in}{0.732124in}}%
\pgfpathlineto{\pgfqpoint{3.410065in}{0.722623in}}%
\pgfpathlineto{\pgfqpoint{3.395793in}{0.713819in}}%
\pgfpathlineto{\pgfqpoint{3.379607in}{0.705716in}}%
\pgfpathlineto{\pgfqpoint{3.351766in}{0.694877in}}%
\pgfpathlineto{\pgfqpoint{3.319599in}{0.685626in}}%
\pgfpathlineto{\pgfqpoint{3.282930in}{0.677971in}}%
\pgfpathlineto{\pgfqpoint{3.241460in}{0.671920in}}%
\pgfpathlineto{\pgfqpoint{3.194766in}{0.667481in}}%
\pgfpathlineto{\pgfqpoint{3.142367in}{0.664665in}}%
\pgfpathlineto{\pgfqpoint{3.084124in}{0.663508in}}%
\pgfpathlineto{\pgfqpoint{3.019145in}{0.664079in}}%
\pgfpathlineto{\pgfqpoint{2.920468in}{0.667654in}}%
\pgfpathlineto{\pgfqpoint{2.806541in}{0.674622in}}%
\pgfpathlineto{\pgfqpoint{2.676289in}{0.685192in}}%
\pgfpathlineto{\pgfqpoint{2.529221in}{0.699593in}}%
\pgfpathlineto{\pgfqpoint{2.365430in}{0.718072in}}%
\pgfpathlineto{\pgfqpoint{2.185931in}{0.740902in}}%
\pgfpathlineto{\pgfqpoint{2.046683in}{0.760823in}}%
\pgfpathlineto{\pgfqpoint{1.909016in}{0.782945in}}%
\pgfpathlineto{\pgfqpoint{1.777219in}{0.807020in}}%
\pgfpathlineto{\pgfqpoint{1.694443in}{0.824002in}}%
\pgfpathlineto{\pgfqpoint{1.616833in}{0.841603in}}%
\pgfpathlineto{\pgfqpoint{1.545183in}{0.859701in}}%
\pgfpathlineto{\pgfqpoint{1.480160in}{0.878162in}}%
\pgfpathlineto{\pgfqpoint{1.422304in}{0.896835in}}%
\pgfpathlineto{\pgfqpoint{1.371998in}{0.915561in}}%
\pgfpathlineto{\pgfqpoint{1.328786in}{0.934201in}}%
\pgfpathlineto{\pgfqpoint{1.291867in}{0.952651in}}%
\pgfpathlineto{\pgfqpoint{1.260597in}{0.970816in}}%
\pgfpathlineto{\pgfqpoint{1.234438in}{0.988614in}}%
\pgfpathlineto{\pgfqpoint{1.212955in}{1.005970in}}%
\pgfpathlineto{\pgfqpoint{1.195814in}{1.022823in}}%
\pgfpathlineto{\pgfqpoint{1.182736in}{1.039122in}}%
\pgfpathlineto{\pgfqpoint{1.173155in}{1.054827in}}%
\pgfpathlineto{\pgfqpoint{1.166647in}{1.069911in}}%
\pgfpathlineto{\pgfqpoint{1.162902in}{1.084352in}}%
\pgfpathlineto{\pgfqpoint{1.161687in}{1.098133in}}%
\pgfpathlineto{\pgfqpoint{1.162846in}{1.111241in}}%
\pgfpathlineto{\pgfqpoint{1.166302in}{1.123666in}}%
\pgfpathlineto{\pgfqpoint{1.172027in}{1.135405in}}%
\pgfpathlineto{\pgfqpoint{1.179893in}{1.146453in}}%
\pgfpathlineto{\pgfqpoint{1.189759in}{1.156806in}}%
\pgfpathlineto{\pgfqpoint{1.201527in}{1.166461in}}%
\pgfpathlineto{\pgfqpoint{1.215134in}{1.175416in}}%
\pgfpathlineto{\pgfqpoint{1.230555in}{1.183668in}}%
\pgfpathlineto{\pgfqpoint{1.257126in}{1.194731in}}%
\pgfpathlineto{\pgfqpoint{1.288016in}{1.204216in}}%
\pgfpathlineto{\pgfqpoint{1.323615in}{1.212133in}}%
\pgfpathlineto{\pgfqpoint{1.364298in}{1.218483in}}%
\pgfpathlineto{\pgfqpoint{1.410067in}{1.223230in}}%
\pgfpathlineto{\pgfqpoint{1.461380in}{1.226345in}}%
\pgfpathlineto{\pgfqpoint{1.518766in}{1.227786in}}%
\pgfpathlineto{\pgfqpoint{1.582796in}{1.227505in}}%
\pgfpathlineto{\pgfqpoint{1.679570in}{1.224344in}}%
\pgfpathlineto{\pgfqpoint{1.790816in}{1.217838in}}%
\pgfpathlineto{\pgfqpoint{1.918200in}{1.207781in}}%
\pgfpathlineto{\pgfqpoint{2.062942in}{1.193915in}}%
\pgfpathlineto{\pgfqpoint{2.224840in}{1.175989in}}%
\pgfpathlineto{\pgfqpoint{2.401108in}{1.153835in}}%
\pgfpathlineto{\pgfqpoint{2.539201in}{1.134427in}}%
\pgfpathlineto{\pgfqpoint{2.677725in}{1.112735in}}%
\pgfpathlineto{\pgfqpoint{2.812065in}{1.088975in}}%
\pgfpathlineto{\pgfqpoint{2.896942in}{1.072140in}}%
\pgfpathlineto{\pgfqpoint{2.976290in}{1.054640in}}%
\pgfpathlineto{\pgfqpoint{3.049071in}{1.036609in}}%
\pgfpathlineto{\pgfqpoint{3.115078in}{1.018211in}}%
\pgfpathlineto{\pgfqpoint{3.174226in}{0.999598in}}%
\pgfpathlineto{\pgfqpoint{3.226542in}{0.980910in}}%
\pgfpathlineto{\pgfqpoint{3.272161in}{0.962271in}}%
\pgfpathlineto{\pgfqpoint{3.311326in}{0.943793in}}%
\pgfpathlineto{\pgfqpoint{3.344389in}{0.925576in}}%
\pgfpathlineto{\pgfqpoint{3.371809in}{0.907703in}}%
\pgfpathlineto{\pgfqpoint{3.394154in}{0.890248in}}%
\pgfpathlineto{\pgfqpoint{3.412071in}{0.873271in}}%
\pgfpathlineto{\pgfqpoint{3.425707in}{0.856842in}}%
\pgfpathlineto{\pgfqpoint{3.435660in}{0.841001in}}%
\pgfpathlineto{\pgfqpoint{3.442642in}{0.825771in}}%
\pgfpathlineto{\pgfqpoint{3.447189in}{0.811172in}}%
\pgfpathlineto{\pgfqpoint{3.449660in}{0.797219in}}%
\pgfpathlineto{\pgfqpoint{3.450239in}{0.783925in}}%
\pgfpathlineto{\pgfqpoint{3.448932in}{0.771297in}}%
\pgfpathlineto{\pgfqpoint{3.445571in}{0.759342in}}%
\pgfpathlineto{\pgfqpoint{3.439810in}{0.748060in}}%
\pgfpathlineto{\pgfqpoint{3.431127in}{0.737449in}}%
\pgfpathlineto{\pgfqpoint{3.419717in}{0.727522in}}%
\pgfpathlineto{\pgfqpoint{3.406375in}{0.718299in}}%
\pgfpathlineto{\pgfqpoint{3.391145in}{0.709781in}}%
\pgfpathlineto{\pgfqpoint{3.364788in}{0.698328in}}%
\pgfpathlineto{\pgfqpoint{3.334160in}{0.688468in}}%
\pgfpathlineto{\pgfqpoint{3.299085in}{0.680202in}}%
\pgfpathlineto{\pgfqpoint{3.259254in}{0.673535in}}%
\pgfpathlineto{\pgfqpoint{3.214263in}{0.668469in}}%
\pgfpathlineto{\pgfqpoint{3.163955in}{0.665026in}}%
\pgfpathlineto{\pgfqpoint{3.107749in}{0.663247in}}%
\pgfpathlineto{\pgfqpoint{3.044935in}{0.663184in}}%
\pgfpathlineto{\pgfqpoint{2.974839in}{0.664897in}}%
\pgfpathlineto{\pgfqpoint{2.868960in}{0.670068in}}%
\pgfpathlineto{\pgfqpoint{2.747610in}{0.678722in}}%
\pgfpathlineto{\pgfqpoint{2.609507in}{0.691093in}}%
\pgfpathlineto{\pgfqpoint{2.453278in}{0.707411in}}%
\pgfpathlineto{\pgfqpoint{2.280693in}{0.727909in}}%
\pgfpathlineto{\pgfqpoint{2.144505in}{0.746127in}}%
\pgfpathlineto{\pgfqpoint{2.006224in}{0.766727in}}%
\pgfpathlineto{\pgfqpoint{1.869609in}{0.789540in}}%
\pgfpathlineto{\pgfqpoint{1.738815in}{0.814290in}}%
\pgfpathlineto{\pgfqpoint{1.657087in}{0.831678in}}%
\pgfpathlineto{\pgfqpoint{1.581414in}{0.849616in}}%
\pgfpathlineto{\pgfqpoint{1.513145in}{0.867949in}}%
\pgfpathlineto{\pgfqpoint{1.452355in}{0.886539in}}%
\pgfpathlineto{\pgfqpoint{1.398619in}{0.905250in}}%
\pgfpathlineto{\pgfqpoint{1.351533in}{0.923955in}}%
\pgfpathlineto{\pgfqpoint{1.310711in}{0.942538in}}%
\pgfpathlineto{\pgfqpoint{1.275785in}{0.960894in}}%
\pgfpathlineto{\pgfqpoint{1.246402in}{0.978932in}}%
\pgfpathlineto{\pgfqpoint{1.222230in}{0.996567in}}%
\pgfpathlineto{\pgfqpoint{1.202950in}{1.013730in}}%
\pgfpathlineto{\pgfqpoint{1.188133in}{1.030358in}}%
\pgfpathlineto{\pgfqpoint{1.177059in}{1.046405in}}%
\pgfpathlineto{\pgfqpoint{1.169155in}{1.061844in}}%
\pgfpathlineto{\pgfqpoint{1.163985in}{1.076650in}}%
\pgfpathlineto{\pgfqpoint{1.161240in}{1.090807in}}%
\pgfpathlineto{\pgfqpoint{1.160744in}{1.104298in}}%
\pgfpathlineto{\pgfqpoint{1.162452in}{1.117114in}}%
\pgfpathlineto{\pgfqpoint{1.166448in}{1.129249in}}%
\pgfpathlineto{\pgfqpoint{1.172947in}{1.140700in}}%
\pgfpathlineto{\pgfqpoint{1.181797in}{1.151461in}}%
\pgfpathlineto{\pgfqpoint{1.192637in}{1.161524in}}%
\pgfpathlineto{\pgfqpoint{1.205390in}{1.170885in}}%
\pgfpathlineto{\pgfqpoint{1.220007in}{1.179544in}}%
\pgfpathlineto{\pgfqpoint{1.245400in}{1.191212in}}%
\pgfpathlineto{\pgfqpoint{1.275029in}{1.201294in}}%
\pgfpathlineto{\pgfqpoint{1.309117in}{1.209791in}}%
\pgfpathlineto{\pgfqpoint{1.348039in}{1.216708in}}%
\pgfpathlineto{\pgfqpoint{1.392050in}{1.222036in}}%
\pgfpathlineto{\pgfqpoint{1.441367in}{1.225742in}}%
\pgfpathlineto{\pgfqpoint{1.496562in}{1.227789in}}%
\pgfpathlineto{\pgfqpoint{1.558226in}{1.228132in}}%
\pgfpathlineto{\pgfqpoint{1.626967in}{1.226713in}}%
\pgfpathlineto{\pgfqpoint{1.730724in}{1.221958in}}%
\pgfpathlineto{\pgfqpoint{1.849729in}{1.213755in}}%
\pgfpathlineto{\pgfqpoint{1.985539in}{1.201878in}}%
\pgfpathlineto{\pgfqpoint{2.138876in}{1.186064in}}%
\pgfpathlineto{\pgfqpoint{2.308497in}{1.166077in}}%
\pgfpathlineto{\pgfqpoint{2.443809in}{1.148293in}}%
\pgfpathlineto{\pgfqpoint{2.582642in}{1.128140in}}%
\pgfpathlineto{\pgfqpoint{2.720705in}{1.105736in}}%
\pgfpathlineto{\pgfqpoint{2.852748in}{1.081364in}}%
\pgfpathlineto{\pgfqpoint{2.935204in}{1.064199in}}%
\pgfpathlineto{\pgfqpoint{3.011957in}{1.046442in}}%
\pgfpathlineto{\pgfqpoint{3.082183in}{1.028226in}}%
\pgfpathlineto{\pgfqpoint{3.145272in}{1.009699in}}%
\pgfpathlineto{\pgfqpoint{3.200782in}{0.991020in}}%
\pgfpathlineto{\pgfqpoint{3.249183in}{0.972325in}}%
\pgfpathlineto{\pgfqpoint{3.291164in}{0.953725in}}%
\pgfpathlineto{\pgfqpoint{3.327303in}{0.935325in}}%
\pgfpathlineto{\pgfqpoint{3.358073in}{0.917216in}}%
\pgfpathlineto{\pgfqpoint{3.383839in}{0.899484in}}%
\pgfpathlineto{\pgfqpoint{3.404856in}{0.882200in}}%
\pgfpathlineto{\pgfqpoint{3.421276in}{0.865428in}}%
\pgfpathlineto{\pgfqpoint{3.433405in}{0.849223in}}%
\pgfpathlineto{\pgfqpoint{3.442095in}{0.833624in}}%
\pgfpathlineto{\pgfqpoint{3.447814in}{0.818653in}}%
\pgfpathlineto{\pgfqpoint{3.450916in}{0.804332in}}%
\pgfpathlineto{\pgfqpoint{3.451652in}{0.790674in}}%
\pgfpathlineto{\pgfqpoint{3.450171in}{0.777693in}}%
\pgfpathlineto{\pgfqpoint{3.446519in}{0.765396in}}%
\pgfpathlineto{\pgfqpoint{3.440644in}{0.753786in}}%
\pgfpathlineto{\pgfqpoint{3.432431in}{0.742865in}}%
\pgfpathlineto{\pgfqpoint{3.422121in}{0.732638in}}%
\pgfpathlineto{\pgfqpoint{3.409860in}{0.723110in}}%
\pgfpathlineto{\pgfqpoint{3.395711in}{0.714283in}}%
\pgfpathlineto{\pgfqpoint{3.379703in}{0.706158in}}%
\pgfpathlineto{\pgfqpoint{3.352202in}{0.695291in}}%
\pgfpathlineto{\pgfqpoint{3.320397in}{0.686010in}}%
\pgfpathlineto{\pgfqpoint{3.284020in}{0.678314in}}%
\pgfpathlineto{\pgfqpoint{3.242652in}{0.672203in}}%
\pgfpathlineto{\pgfqpoint{3.196061in}{0.667688in}}%
\pgfpathlineto{\pgfqpoint{3.143948in}{0.664809in}}%
\pgfpathlineto{\pgfqpoint{3.085670in}{0.663607in}}%
\pgfpathlineto{\pgfqpoint{3.020589in}{0.664136in}}%
\pgfpathlineto{\pgfqpoint{2.922142in}{0.667645in}}%
\pgfpathlineto{\pgfqpoint{2.809000in}{0.674526in}}%
\pgfpathlineto{\pgfqpoint{2.679702in}{0.684995in}}%
\pgfpathlineto{\pgfqpoint{2.532957in}{0.699292in}}%
\pgfpathlineto{\pgfqpoint{2.369132in}{0.717677in}}%
\pgfpathlineto{\pgfqpoint{2.191508in}{0.740335in}}%
\pgfpathlineto{\pgfqpoint{2.053108in}{0.760105in}}%
\pgfpathlineto{\pgfqpoint{1.914679in}{0.782124in}}%
\pgfpathlineto{\pgfqpoint{1.781283in}{0.806192in}}%
\pgfpathlineto{\pgfqpoint{1.697588in}{0.823219in}}%
\pgfpathlineto{\pgfqpoint{1.619391in}{0.840860in}}%
\pgfpathlineto{\pgfqpoint{1.547515in}{0.858969in}}%
\pgfpathlineto{\pgfqpoint{1.482516in}{0.877407in}}%
\pgfpathlineto{\pgfqpoint{1.424685in}{0.896039in}}%
\pgfpathlineto{\pgfqpoint{1.374048in}{0.914737in}}%
\pgfpathlineto{\pgfqpoint{1.330363in}{0.933378in}}%
\pgfpathlineto{\pgfqpoint{1.293125in}{0.951845in}}%
\pgfpathlineto{\pgfqpoint{1.262094in}{0.970008in}}%
\pgfpathlineto{\pgfqpoint{1.236561in}{0.987785in}}%
\pgfpathlineto{\pgfqpoint{1.215492in}{1.005123in}}%
\pgfpathlineto{\pgfqpoint{1.198101in}{1.021976in}}%
\pgfpathlineto{\pgfqpoint{1.183858in}{1.038302in}}%
\pgfpathlineto{\pgfqpoint{1.172486in}{1.054064in}}%
\pgfpathlineto{\pgfqpoint{1.163960in}{1.069229in}}%
\pgfpathlineto{\pgfqpoint{1.158509in}{1.083769in}}%
\pgfpathlineto{\pgfqpoint{1.156614in}{1.097660in}}%
\pgfpathlineto{\pgfqpoint{1.158162in}{1.110879in}}%
\pgfpathlineto{\pgfqpoint{1.162003in}{1.123405in}}%
\pgfpathlineto{\pgfqpoint{1.167947in}{1.135234in}}%
\pgfpathlineto{\pgfqpoint{1.175866in}{1.146362in}}%
\pgfpathlineto{\pgfqpoint{1.185679in}{1.156786in}}%
\pgfpathlineto{\pgfqpoint{1.197346in}{1.166505in}}%
\pgfpathlineto{\pgfqpoint{1.210874in}{1.175521in}}%
\pgfpathlineto{\pgfqpoint{1.226314in}{1.183834in}}%
\pgfpathlineto{\pgfqpoint{1.243761in}{1.191449in}}%
\pgfpathlineto{\pgfqpoint{1.273737in}{1.201564in}}%
\pgfpathlineto{\pgfqpoint{1.308208in}{1.210098in}}%
\pgfpathlineto{\pgfqpoint{1.347365in}{1.217037in}}%
\pgfpathlineto{\pgfqpoint{1.391492in}{1.222361in}}%
\pgfpathlineto{\pgfqpoint{1.440968in}{1.226046in}}%
\pgfpathlineto{\pgfqpoint{1.496262in}{1.228061in}}%
\pgfpathlineto{\pgfqpoint{1.557937in}{1.228372in}}%
\pgfpathlineto{\pgfqpoint{1.626645in}{1.226936in}}%
\pgfpathlineto{\pgfqpoint{1.729996in}{1.222211in}}%
\pgfpathlineto{\pgfqpoint{1.849085in}{1.214022in}}%
\pgfpathlineto{\pgfqpoint{1.985634in}{1.202097in}}%
\pgfpathlineto{\pgfqpoint{2.139565in}{1.186218in}}%
\pgfpathlineto{\pgfqpoint{2.309000in}{1.166212in}}%
\pgfpathlineto{\pgfqpoint{2.444106in}{1.148424in}}%
\pgfpathlineto{\pgfqpoint{2.583310in}{1.128207in}}%
\pgfpathlineto{\pgfqpoint{2.721494in}{1.105735in}}%
\pgfpathlineto{\pgfqpoint{2.853478in}{1.081334in}}%
\pgfpathlineto{\pgfqpoint{2.935930in}{1.064171in}}%
\pgfpathlineto{\pgfqpoint{3.012782in}{1.046429in}}%
\pgfpathlineto{\pgfqpoint{3.083262in}{1.028230in}}%
\pgfpathlineto{\pgfqpoint{3.146797in}{1.009707in}}%
\pgfpathlineto{\pgfqpoint{3.203021in}{0.990997in}}%
\pgfpathlineto{\pgfqpoint{3.251768in}{0.972248in}}%
\pgfpathlineto{\pgfqpoint{3.293289in}{0.953618in}}%
\pgfpathlineto{\pgfqpoint{3.328539in}{0.935210in}}%
\pgfpathlineto{\pgfqpoint{3.358413in}{0.917108in}}%
\pgfpathlineto{\pgfqpoint{3.383600in}{0.899383in}}%
\pgfpathlineto{\pgfqpoint{3.404586in}{0.882103in}}%
\pgfpathlineto{\pgfqpoint{3.421655in}{0.865326in}}%
\pgfpathlineto{\pgfqpoint{3.434885in}{0.849101in}}%
\pgfpathlineto{\pgfqpoint{3.444152in}{0.833472in}}%
\pgfpathlineto{\pgfqpoint{3.449758in}{0.818474in}}%
\pgfpathlineto{\pgfqpoint{3.452579in}{0.804133in}}%
\pgfpathlineto{\pgfqpoint{3.452911in}{0.790461in}}%
\pgfpathlineto{\pgfqpoint{3.450977in}{0.777470in}}%
\pgfpathlineto{\pgfqpoint{3.446932in}{0.765167in}}%
\pgfpathlineto{\pgfqpoint{3.440863in}{0.753559in}}%
\pgfpathlineto{\pgfqpoint{3.432788in}{0.742647in}}%
\pgfpathlineto{\pgfqpoint{3.422654in}{0.732431in}}%
\pgfpathlineto{\pgfqpoint{3.410407in}{0.722909in}}%
\pgfpathlineto{\pgfqpoint{3.396193in}{0.714084in}}%
\pgfpathlineto{\pgfqpoint{3.380068in}{0.705960in}}%
\pgfpathlineto{\pgfqpoint{3.352322in}{0.695090in}}%
\pgfpathlineto{\pgfqpoint{3.320255in}{0.685808in}}%
\pgfpathlineto{\pgfqpoint{3.283693in}{0.678122in}}%
\pgfpathlineto{\pgfqpoint{3.242333in}{0.672038in}}%
\pgfpathlineto{\pgfqpoint{3.195749in}{0.667566in}}%
\pgfpathlineto{\pgfqpoint{3.143476in}{0.664715in}}%
\pgfpathlineto{\pgfqpoint{3.085362in}{0.663524in}}%
\pgfpathlineto{\pgfqpoint{3.020508in}{0.664060in}}%
\pgfpathlineto{\pgfqpoint{2.922031in}{0.667585in}}%
\pgfpathlineto{\pgfqpoint{2.808361in}{0.674499in}}%
\pgfpathlineto{\pgfqpoint{2.678407in}{0.685011in}}%
\pgfpathlineto{\pgfqpoint{2.531630in}{0.699350in}}%
\pgfpathlineto{\pgfqpoint{2.368016in}{0.717767in}}%
\pgfpathlineto{\pgfqpoint{2.188571in}{0.740533in}}%
\pgfpathlineto{\pgfqpoint{2.049425in}{0.760401in}}%
\pgfpathlineto{\pgfqpoint{1.911805in}{0.782474in}}%
\pgfpathlineto{\pgfqpoint{1.779944in}{0.806508in}}%
\pgfpathlineto{\pgfqpoint{1.697054in}{0.823470in}}%
\pgfpathlineto{\pgfqpoint{1.619275in}{0.841056in}}%
\pgfpathlineto{\pgfqpoint{1.547413in}{0.859144in}}%
\pgfpathlineto{\pgfqpoint{1.482154in}{0.877598in}}%
\pgfpathlineto{\pgfqpoint{1.424066in}{0.896267in}}%
\pgfpathlineto{\pgfqpoint{1.373551in}{0.914991in}}%
\pgfpathlineto{\pgfqpoint{1.330121in}{0.933633in}}%
\pgfpathlineto{\pgfqpoint{1.292992in}{0.952090in}}%
\pgfpathlineto{\pgfqpoint{1.261537in}{0.970264in}}%
\pgfpathlineto{\pgfqpoint{1.235226in}{0.988072in}}%
\pgfpathlineto{\pgfqpoint{1.213625in}{1.005442in}}%
\pgfpathlineto{\pgfqpoint{1.196394in}{1.022309in}}%
\pgfpathlineto{\pgfqpoint{1.183218in}{1.038623in}}%
\pgfpathlineto{\pgfqpoint{1.173525in}{1.054345in}}%
\pgfpathlineto{\pgfqpoint{1.166909in}{1.069447in}}%
\pgfpathlineto{\pgfqpoint{1.163062in}{1.083908in}}%
\pgfpathlineto{\pgfqpoint{1.161756in}{1.097708in}}%
\pgfpathlineto{\pgfqpoint{1.162840in}{1.110836in}}%
\pgfpathlineto{\pgfqpoint{1.166242in}{1.123281in}}%
\pgfpathlineto{\pgfqpoint{1.171922in}{1.135041in}}%
\pgfpathlineto{\pgfqpoint{1.179729in}{1.146110in}}%
\pgfpathlineto{\pgfqpoint{1.189527in}{1.156484in}}%
\pgfpathlineto{\pgfqpoint{1.201215in}{1.166160in}}%
\pgfpathlineto{\pgfqpoint{1.214735in}{1.175134in}}%
\pgfpathlineto{\pgfqpoint{1.230065in}{1.183408in}}%
\pgfpathlineto{\pgfqpoint{1.256501in}{1.194501in}}%
\pgfpathlineto{\pgfqpoint{1.287279in}{1.204020in}}%
\pgfpathlineto{\pgfqpoint{1.322816in}{1.211974in}}%
\pgfpathlineto{\pgfqpoint{1.363402in}{1.218360in}}%
\pgfpathlineto{\pgfqpoint{1.409065in}{1.223142in}}%
\pgfpathlineto{\pgfqpoint{1.460268in}{1.226291in}}%
\pgfpathlineto{\pgfqpoint{1.517526in}{1.227768in}}%
\pgfpathlineto{\pgfqpoint{1.581404in}{1.227522in}}%
\pgfpathlineto{\pgfqpoint{1.677938in}{1.224410in}}%
\pgfpathlineto{\pgfqpoint{1.788923in}{1.217959in}}%
\pgfpathlineto{\pgfqpoint{1.916035in}{1.207963in}}%
\pgfpathlineto{\pgfqpoint{2.060508in}{1.194160in}}%
\pgfpathlineto{\pgfqpoint{2.222121in}{1.176302in}}%
\pgfpathlineto{\pgfqpoint{2.398225in}{1.154216in}}%
\pgfpathlineto{\pgfqpoint{2.536197in}{1.134860in}}%
\pgfpathlineto{\pgfqpoint{2.674758in}{1.113216in}}%
\pgfpathlineto{\pgfqpoint{2.809301in}{1.089497in}}%
\pgfpathlineto{\pgfqpoint{2.894258in}{1.072681in}}%
\pgfpathlineto{\pgfqpoint{2.973704in}{1.055182in}}%
\pgfpathlineto{\pgfqpoint{3.046921in}{1.037160in}}%
\pgfpathlineto{\pgfqpoint{3.113462in}{1.018775in}}%
\pgfpathlineto{\pgfqpoint{3.173060in}{1.000175in}}%
\pgfpathlineto{\pgfqpoint{3.225630in}{0.981496in}}%
\pgfpathlineto{\pgfqpoint{3.271267in}{0.962860in}}%
\pgfpathlineto{\pgfqpoint{3.310248in}{0.944379in}}%
\pgfpathlineto{\pgfqpoint{3.343030in}{0.926153in}}%
\pgfpathlineto{\pgfqpoint{3.370251in}{0.908266in}}%
\pgfpathlineto{\pgfqpoint{3.392690in}{0.890796in}}%
\pgfpathlineto{\pgfqpoint{3.410469in}{0.873824in}}%
\pgfpathlineto{\pgfqpoint{3.424192in}{0.857398in}}%
\pgfpathlineto{\pgfqpoint{3.434623in}{0.841546in}}%
\pgfpathlineto{\pgfqpoint{3.442331in}{0.826296in}}%
\pgfpathlineto{\pgfqpoint{3.447694in}{0.811668in}}%
\pgfpathlineto{\pgfqpoint{3.450896in}{0.797682in}}%
\pgfpathlineto{\pgfqpoint{3.451925in}{0.784350in}}%
\pgfpathlineto{\pgfqpoint{3.450577in}{0.771682in}}%
\pgfpathlineto{\pgfqpoint{3.446455in}{0.759684in}}%
\pgfpathlineto{\pgfqpoint{3.439238in}{0.748362in}}%
\pgfpathlineto{\pgfqpoint{3.429841in}{0.737738in}}%
\pgfpathlineto{\pgfqpoint{3.418499in}{0.727816in}}%
\pgfpathlineto{\pgfqpoint{3.405273in}{0.718598in}}%
\pgfpathlineto{\pgfqpoint{3.390197in}{0.710085in}}%
\pgfpathlineto{\pgfqpoint{3.364112in}{0.698639in}}%
\pgfpathlineto{\pgfqpoint{3.333744in}{0.688781in}}%
\pgfpathlineto{\pgfqpoint{3.298820in}{0.680506in}}%
\pgfpathlineto{\pgfqpoint{3.258990in}{0.673812in}}%
\pgfpathlineto{\pgfqpoint{3.214158in}{0.668720in}}%
\pgfpathlineto{\pgfqpoint{3.163888in}{0.665259in}}%
\pgfpathlineto{\pgfqpoint{3.107652in}{0.663466in}}%
\pgfpathlineto{\pgfqpoint{3.044880in}{0.663389in}}%
\pgfpathlineto{\pgfqpoint{2.974967in}{0.665087in}}%
\pgfpathlineto{\pgfqpoint{2.869520in}{0.670232in}}%
\pgfpathlineto{\pgfqpoint{2.748592in}{0.678849in}}%
\pgfpathlineto{\pgfqpoint{2.610666in}{0.691170in}}%
\pgfpathlineto{\pgfqpoint{2.455190in}{0.707467in}}%
\pgfpathlineto{\pgfqpoint{2.283860in}{0.727945in}}%
\pgfpathlineto{\pgfqpoint{2.147608in}{0.746109in}}%
\pgfpathlineto{\pgfqpoint{2.008467in}{0.766634in}}%
\pgfpathlineto{\pgfqpoint{1.871119in}{0.789358in}}%
\pgfpathlineto{\pgfqpoint{1.740241in}{0.814016in}}%
\pgfpathlineto{\pgfqpoint{1.658860in}{0.831349in}}%
\pgfpathlineto{\pgfqpoint{1.584050in}{0.849247in}}%
\pgfpathlineto{\pgfqpoint{1.516502in}{0.867559in}}%
\pgfpathlineto{\pgfqpoint{1.455905in}{0.886133in}}%
\pgfpathlineto{\pgfqpoint{1.401950in}{0.904827in}}%
\pgfpathlineto{\pgfqpoint{1.354337in}{0.923513in}}%
\pgfpathlineto{\pgfqpoint{1.312770in}{0.942077in}}%
\pgfpathlineto{\pgfqpoint{1.276961in}{0.960416in}}%
\pgfpathlineto{\pgfqpoint{1.246626in}{0.978439in}}%
\pgfpathlineto{\pgfqpoint{1.221491in}{0.996071in}}%
\pgfpathlineto{\pgfqpoint{1.201286in}{1.013246in}}%
\pgfpathlineto{\pgfqpoint{1.185748in}{1.029913in}}%
\pgfpathlineto{\pgfqpoint{1.174583in}{1.046026in}}%
\pgfpathlineto{\pgfqpoint{1.167141in}{1.061519in}}%
\pgfpathlineto{\pgfqpoint{1.162716in}{1.076369in}}%
\pgfpathlineto{\pgfqpoint{1.160753in}{1.090562in}}%
\pgfpathlineto{\pgfqpoint{1.160841in}{1.104084in}}%
\pgfpathlineto{\pgfqpoint{1.162710in}{1.116928in}}%
\pgfpathlineto{\pgfqpoint{1.166239in}{1.129088in}}%
\pgfpathlineto{\pgfqpoint{1.171446in}{1.140561in}}%
\pgfpathlineto{\pgfqpoint{1.178496in}{1.151349in}}%
\pgfpathlineto{\pgfqpoint{1.187696in}{1.161455in}}%
\pgfpathlineto{\pgfqpoint{1.199499in}{1.170886in}}%
\pgfpathlineto{\pgfqpoint{1.214144in}{1.179643in}}%
\pgfpathlineto{\pgfqpoint{1.230787in}{1.187696in}}%
\pgfpathlineto{\pgfqpoint{1.259331in}{1.198451in}}%
\pgfpathlineto{\pgfqpoint{1.292242in}{1.207609in}}%
\pgfpathlineto{\pgfqpoint{1.329700in}{1.215162in}}%
\pgfpathlineto{\pgfqpoint{1.371990in}{1.221101in}}%
\pgfpathlineto{\pgfqpoint{1.419505in}{1.225412in}}%
\pgfpathlineto{\pgfqpoint{1.472721in}{1.228080in}}%
\pgfpathlineto{\pgfqpoint{1.531770in}{1.229082in}}%
\pgfpathlineto{\pgfqpoint{1.597555in}{1.228349in}}%
\pgfpathlineto{\pgfqpoint{1.697495in}{1.224532in}}%
\pgfpathlineto{\pgfqpoint{1.812997in}{1.217288in}}%
\pgfpathlineto{\pgfqpoint{1.945075in}{1.206406in}}%
\pgfpathlineto{\pgfqpoint{2.093980in}{1.191666in}}%
\pgfpathlineto{\pgfqpoint{2.259198in}{1.172836in}}%
\pgfpathlineto{\pgfqpoint{2.439304in}{1.149654in}}%
\pgfpathlineto{\pgfqpoint{2.579258in}{1.129431in}}%
\pgfpathlineto{\pgfqpoint{2.717398in}{1.107015in}}%
\pgfpathlineto{\pgfqpoint{2.849154in}{1.082683in}}%
\pgfpathlineto{\pgfqpoint{2.931556in}{1.065558in}}%
\pgfpathlineto{\pgfqpoint{3.008521in}{1.047838in}}%
\pgfpathlineto{\pgfqpoint{3.079297in}{1.029646in}}%
\pgfpathlineto{\pgfqpoint{3.143292in}{1.011114in}}%
\pgfpathlineto{\pgfqpoint{3.200074in}{0.992384in}}%
\pgfpathlineto{\pgfqpoint{3.249372in}{0.973610in}}%
\pgfpathlineto{\pgfqpoint{3.291420in}{0.954945in}}%
\pgfpathlineto{\pgfqpoint{3.327145in}{0.936497in}}%
\pgfpathlineto{\pgfqpoint{3.357362in}{0.918351in}}%
\pgfpathlineto{\pgfqpoint{3.382720in}{0.900584in}}%
\pgfpathlineto{\pgfqpoint{3.403696in}{0.883266in}}%
\pgfpathlineto{\pgfqpoint{3.420601in}{0.866454in}}%
\pgfpathlineto{\pgfqpoint{3.433578in}{0.850199in}}%
\pgfpathlineto{\pgfqpoint{3.442668in}{0.834543in}}%
\pgfpathlineto{\pgfqpoint{3.448536in}{0.819518in}}%
\pgfpathlineto{\pgfqpoint{3.451646in}{0.805144in}}%
\pgfpathlineto{\pgfqpoint{3.452267in}{0.791437in}}%
\pgfpathlineto{\pgfqpoint{3.450599in}{0.778409in}}%
\pgfpathlineto{\pgfqpoint{3.446774in}{0.766067in}}%
\pgfpathlineto{\pgfqpoint{3.440853in}{0.754417in}}%
\pgfpathlineto{\pgfqpoint{3.432827in}{0.743461in}}%
\pgfpathlineto{\pgfqpoint{3.422675in}{0.733199in}}%
\pgfpathlineto{\pgfqpoint{3.410529in}{0.723634in}}%
\pgfpathlineto{\pgfqpoint{3.396463in}{0.714767in}}%
\pgfpathlineto{\pgfqpoint{3.380521in}{0.706602in}}%
\pgfpathlineto{\pgfqpoint{3.353108in}{0.695673in}}%
\pgfpathlineto{\pgfqpoint{3.321420in}{0.686331in}}%
\pgfpathlineto{\pgfqpoint{3.285238in}{0.678581in}}%
\pgfpathlineto{\pgfqpoint{3.244202in}{0.672423in}}%
\pgfpathlineto{\pgfqpoint{3.197825in}{0.667860in}}%
\pgfpathlineto{\pgfqpoint{3.145967in}{0.664919in}}%
\pgfpathlineto{\pgfqpoint{3.088069in}{0.663649in}}%
\pgfpathlineto{\pgfqpoint{3.023365in}{0.664107in}}%
\pgfpathlineto{\pgfqpoint{2.925311in}{0.667517in}}%
\pgfpathlineto{\pgfqpoint{2.812454in}{0.674295in}}%
\pgfpathlineto{\pgfqpoint{2.683529in}{0.684656in}}%
\pgfpathlineto{\pgfqpoint{2.537481in}{0.698843in}}%
\pgfpathlineto{\pgfqpoint{2.371383in}{0.717196in}}%
\pgfpathlineto{\pgfqpoint{2.236125in}{0.733816in}}%
\pgfpathlineto{\pgfqpoint{2.097417in}{0.752809in}}%
\pgfpathlineto{\pgfqpoint{1.959770in}{0.774040in}}%
\pgfpathlineto{\pgfqpoint{1.827092in}{0.797308in}}%
\pgfpathlineto{\pgfqpoint{1.743059in}{0.813823in}}%
\pgfpathlineto{\pgfqpoint{1.663549in}{0.831032in}}%
\pgfpathlineto{\pgfqpoint{1.589274in}{0.848826in}}%
\pgfpathlineto{\pgfqpoint{1.520829in}{0.867087in}}%
\pgfpathlineto{\pgfqpoint{1.458689in}{0.885680in}}%
\pgfpathlineto{\pgfqpoint{1.403210in}{0.904461in}}%
\pgfpathlineto{\pgfqpoint{1.354631in}{0.923270in}}%
\pgfpathlineto{\pgfqpoint{1.313071in}{0.941935in}}%
\pgfpathlineto{\pgfqpoint{1.278395in}{0.960291in}}%
\pgfpathlineto{\pgfqpoint{1.249459in}{0.978304in}}%
\pgfpathlineto{\pgfqpoint{1.225564in}{0.995911in}}%
\pgfpathlineto{\pgfqpoint{1.206197in}{1.013047in}}%
\pgfpathlineto{\pgfqpoint{1.190895in}{1.029657in}}%
\pgfpathlineto{\pgfqpoint{1.179232in}{1.045698in}}%
\pgfpathlineto{\pgfqpoint{1.170848in}{1.061136in}}%
\pgfpathlineto{\pgfqpoint{1.165442in}{1.075944in}}%
\pgfpathlineto{\pgfqpoint{1.162887in}{1.090101in}}%
\pgfpathlineto{\pgfqpoint{1.162947in}{1.103593in}}%
\pgfpathlineto{\pgfqpoint{1.165312in}{1.116409in}}%
\pgfpathlineto{\pgfqpoint{1.169748in}{1.128540in}}%
\pgfpathlineto{\pgfqpoint{1.176094in}{1.139981in}}%
\pgfpathlineto{\pgfqpoint{1.184260in}{1.150727in}}%
\pgfpathlineto{\pgfqpoint{1.194236in}{1.160777in}}%
\pgfpathlineto{\pgfqpoint{1.206080in}{1.170134in}}%
\pgfpathlineto{\pgfqpoint{1.219928in}{1.178800in}}%
\pgfpathlineto{\pgfqpoint{1.235988in}{1.186783in}}%
\pgfpathlineto{\pgfqpoint{1.264127in}{1.197466in}}%
\pgfpathlineto{\pgfqpoint{1.296719in}{1.206575in}}%
\pgfpathlineto{\pgfqpoint{1.333916in}{1.214098in}}%
\pgfpathlineto{\pgfqpoint{1.375966in}{1.220017in}}%
\pgfpathlineto{\pgfqpoint{1.423215in}{1.224312in}}%
\pgfpathlineto{\pgfqpoint{1.476103in}{1.226956in}}%
\pgfpathlineto{\pgfqpoint{1.535168in}{1.227921in}}%
\pgfpathlineto{\pgfqpoint{1.600862in}{1.227186in}}%
\pgfpathlineto{\pgfqpoint{1.699483in}{1.223495in}}%
\pgfpathlineto{\pgfqpoint{1.813792in}{1.216377in}}%
\pgfpathlineto{\pgfqpoint{1.945701in}{1.205529in}}%
\pgfpathlineto{\pgfqpoint{2.095240in}{1.190729in}}%
\pgfpathlineto{\pgfqpoint{2.260562in}{1.171837in}}%
\pgfpathlineto{\pgfqpoint{2.437941in}{1.148797in}}%
\pgfpathlineto{\pgfqpoint{2.575588in}{1.128808in}}%
\pgfpathlineto{\pgfqpoint{2.713420in}{1.106543in}}%
\pgfpathlineto{\pgfqpoint{2.845729in}{1.082245in}}%
\pgfpathlineto{\pgfqpoint{2.928536in}{1.065132in}}%
\pgfpathlineto{\pgfqpoint{3.005783in}{1.047442in}}%
\pgfpathlineto{\pgfqpoint{3.076683in}{1.029305in}}%
\pgfpathlineto{\pgfqpoint{3.140696in}{1.010851in}}%
\pgfpathlineto{\pgfqpoint{3.197530in}{0.992210in}}%
\pgfpathlineto{\pgfqpoint{3.247139in}{0.973511in}}%
\pgfpathlineto{\pgfqpoint{3.289725in}{0.954884in}}%
\pgfpathlineto{\pgfqpoint{3.325503in}{0.936474in}}%
\pgfpathlineto{\pgfqpoint{3.355192in}{0.918392in}}%
\pgfpathlineto{\pgfqpoint{3.379960in}{0.900695in}}%
\pgfpathlineto{\pgfqpoint{3.400702in}{0.883433in}}%
\pgfpathlineto{\pgfqpoint{3.418043in}{0.866654in}}%
\pgfpathlineto{\pgfqpoint{3.432333in}{0.850401in}}%
\pgfpathlineto{\pgfqpoint{3.443649in}{0.834714in}}%
\pgfpathlineto{\pgfqpoint{3.451793in}{0.819629in}}%
\pgfpathlineto{\pgfqpoint{3.456297in}{0.805178in}}%
\pgfpathlineto{\pgfqpoint{3.456793in}{0.791388in}}%
\pgfpathlineto{\pgfqpoint{3.454682in}{0.778285in}}%
\pgfpathlineto{\pgfqpoint{3.450373in}{0.765878in}}%
\pgfpathlineto{\pgfqpoint{3.444019in}{0.754172in}}%
\pgfpathlineto{\pgfqpoint{3.435726in}{0.743168in}}%
\pgfpathlineto{\pgfqpoint{3.425554in}{0.732871in}}%
\pgfpathlineto{\pgfqpoint{3.413514in}{0.723278in}}%
\pgfpathlineto{\pgfqpoint{3.399571in}{0.714389in}}%
\pgfpathlineto{\pgfqpoint{3.383643in}{0.706200in}}%
\pgfpathlineto{\pgfqpoint{3.355970in}{0.695225in}}%
\pgfpathlineto{\pgfqpoint{3.323879in}{0.685832in}}%
\pgfpathlineto{\pgfqpoint{3.287237in}{0.678030in}}%
\pgfpathlineto{\pgfqpoint{3.245809in}{0.671836in}}%
\pgfpathlineto{\pgfqpoint{3.199262in}{0.667270in}}%
\pgfpathlineto{\pgfqpoint{3.147165in}{0.664356in}}%
\pgfpathlineto{\pgfqpoint{3.088985in}{0.663124in}}%
\pgfpathlineto{\pgfqpoint{3.024107in}{0.663605in}}%
\pgfpathlineto{\pgfqpoint{2.926795in}{0.666921in}}%
\pgfpathlineto{\pgfqpoint{2.814357in}{0.673598in}}%
\pgfpathlineto{\pgfqpoint{2.684388in}{0.683989in}}%
\pgfpathlineto{\pgfqpoint{2.536510in}{0.698347in}}%
\pgfpathlineto{\pgfqpoint{2.372371in}{0.716821in}}%
\pgfpathlineto{\pgfqpoint{2.195642in}{0.739460in}}%
\pgfpathlineto{\pgfqpoint{2.058187in}{0.759145in}}%
\pgfpathlineto{\pgfqpoint{1.920012in}{0.781086in}}%
\pgfpathlineto{\pgfqpoint{1.786335in}{0.805131in}}%
\pgfpathlineto{\pgfqpoint{1.702298in}{0.822125in}}%
\pgfpathlineto{\pgfqpoint{1.623642in}{0.839728in}}%
\pgfpathlineto{\pgfqpoint{1.551209in}{0.857806in}}%
\pgfpathlineto{\pgfqpoint{1.485588in}{0.876226in}}%
\pgfpathlineto{\pgfqpoint{1.427115in}{0.894856in}}%
\pgfpathlineto{\pgfqpoint{1.375871in}{0.913569in}}%
\pgfpathlineto{\pgfqpoint{1.331683in}{0.932239in}}%
\pgfpathlineto{\pgfqpoint{1.294180in}{0.950738in}}%
\pgfpathlineto{\pgfqpoint{1.263088in}{0.968928in}}%
\pgfpathlineto{\pgfqpoint{1.237346in}{0.986739in}}%
\pgfpathlineto{\pgfqpoint{1.215941in}{1.004119in}}%
\pgfpathlineto{\pgfqpoint{1.198131in}{1.021020in}}%
\pgfpathlineto{\pgfqpoint{1.183437in}{1.037401in}}%
\pgfpathlineto{\pgfqpoint{1.171649in}{1.053221in}}%
\pgfpathlineto{\pgfqpoint{1.162821in}{1.068447in}}%
\pgfpathlineto{\pgfqpoint{1.157277in}{1.083047in}}%
\pgfpathlineto{\pgfqpoint{1.155596in}{1.096996in}}%
\pgfpathlineto{\pgfqpoint{1.157130in}{1.110265in}}%
\pgfpathlineto{\pgfqpoint{1.160921in}{1.122841in}}%
\pgfpathlineto{\pgfqpoint{1.166802in}{1.134717in}}%
\pgfpathlineto{\pgfqpoint{1.174649in}{1.145891in}}%
\pgfpathlineto{\pgfqpoint{1.184387in}{1.156359in}}%
\pgfpathlineto{\pgfqpoint{1.195983in}{1.166123in}}%
\pgfpathlineto{\pgfqpoint{1.209453in}{1.175181in}}%
\pgfpathlineto{\pgfqpoint{1.224857in}{1.183538in}}%
\pgfpathlineto{\pgfqpoint{1.242295in}{1.191196in}}%
\pgfpathlineto{\pgfqpoint{1.272178in}{1.201372in}}%
\pgfpathlineto{\pgfqpoint{1.306541in}{1.209964in}}%
\pgfpathlineto{\pgfqpoint{1.345570in}{1.216958in}}%
\pgfpathlineto{\pgfqpoint{1.389547in}{1.222335in}}%
\pgfpathlineto{\pgfqpoint{1.438851in}{1.226073in}}%
\pgfpathlineto{\pgfqpoint{1.493959in}{1.228142in}}%
\pgfpathlineto{\pgfqpoint{1.555440in}{1.228511in}}%
\pgfpathlineto{\pgfqpoint{1.648261in}{1.226298in}}%
\pgfpathlineto{\pgfqpoint{1.754836in}{1.220840in}}%
\pgfpathlineto{\pgfqpoint{1.877950in}{1.211829in}}%
\pgfpathlineto{\pgfqpoint{2.018806in}{1.199006in}}%
\pgfpathlineto{\pgfqpoint{2.176742in}{1.182171in}}%
\pgfpathlineto{\pgfqpoint{2.349241in}{1.161188in}}%
\pgfpathlineto{\pgfqpoint{2.485608in}{1.142681in}}%
\pgfpathlineto{\pgfqpoint{2.624839in}{1.121782in}}%
\pgfpathlineto{\pgfqpoint{2.761684in}{1.098667in}}%
\pgfpathlineto{\pgfqpoint{2.891114in}{1.073717in}}%
\pgfpathlineto{\pgfqpoint{2.971300in}{1.056263in}}%
\pgfpathlineto{\pgfqpoint{3.045535in}{1.038296in}}%
\pgfpathlineto{\pgfqpoint{3.113144in}{1.019940in}}%
\pgfpathlineto{\pgfqpoint{3.173668in}{1.001326in}}%
\pgfpathlineto{\pgfqpoint{3.226868in}{0.982585in}}%
\pgfpathlineto{\pgfqpoint{3.272719in}{0.963852in}}%
\pgfpathlineto{\pgfqpoint{3.311433in}{0.945274in}}%
\pgfpathlineto{\pgfqpoint{3.343829in}{0.926978in}}%
\pgfpathlineto{\pgfqpoint{3.370956in}{0.909035in}}%
\pgfpathlineto{\pgfqpoint{3.393641in}{0.891508in}}%
\pgfpathlineto{\pgfqpoint{3.412479in}{0.874453in}}%
\pgfpathlineto{\pgfqpoint{3.427829in}{0.857919in}}%
\pgfpathlineto{\pgfqpoint{3.439818in}{0.841952in}}%
\pgfpathlineto{\pgfqpoint{3.448340in}{0.826587in}}%
\pgfpathlineto{\pgfqpoint{3.453073in}{0.811857in}}%
\pgfpathlineto{\pgfqpoint{3.454634in}{0.797790in}}%
\pgfpathlineto{\pgfqpoint{3.453740in}{0.784403in}}%
\pgfpathlineto{\pgfqpoint{3.450611in}{0.771703in}}%
\pgfpathlineto{\pgfqpoint{3.445407in}{0.759698in}}%
\pgfpathlineto{\pgfqpoint{3.438236in}{0.748391in}}%
\pgfpathlineto{\pgfqpoint{3.429148in}{0.737786in}}%
\pgfpathlineto{\pgfqpoint{3.418134in}{0.727881in}}%
\pgfpathlineto{\pgfqpoint{3.405132in}{0.718675in}}%
\pgfpathlineto{\pgfqpoint{3.390066in}{0.710165in}}%
\pgfpathlineto{\pgfqpoint{3.363810in}{0.698709in}}%
\pgfpathlineto{\pgfqpoint{3.333194in}{0.688834in}}%
\pgfpathlineto{\pgfqpoint{3.298114in}{0.680549in}}%
\pgfpathlineto{\pgfqpoint{3.258359in}{0.673868in}}%
\pgfpathlineto{\pgfqpoint{3.213607in}{0.668805in}}%
\pgfpathlineto{\pgfqpoint{3.163427in}{0.665378in}}%
\pgfpathlineto{\pgfqpoint{3.107277in}{0.663608in}}%
\pgfpathlineto{\pgfqpoint{3.044875in}{0.663512in}}%
\pgfpathlineto{\pgfqpoint{2.975668in}{0.665156in}}%
\pgfpathlineto{\pgfqpoint{2.870549in}{0.670247in}}%
\pgfpathlineto{\pgfqpoint{2.748931in}{0.678872in}}%
\pgfpathlineto{\pgfqpoint{2.610056in}{0.691250in}}%
\pgfpathlineto{\pgfqpoint{2.454486in}{0.707575in}}%
\pgfpathlineto{\pgfqpoint{2.284094in}{0.728018in}}%
\pgfpathlineto{\pgfqpoint{2.148443in}{0.746141in}}%
\pgfpathlineto{\pgfqpoint{2.009101in}{0.766684in}}%
\pgfpathlineto{\pgfqpoint{1.871593in}{0.789432in}}%
\pgfpathlineto{\pgfqpoint{1.740949in}{0.814066in}}%
\pgfpathlineto{\pgfqpoint{1.659691in}{0.831358in}}%
\pgfpathlineto{\pgfqpoint{1.584229in}{0.849207in}}%
\pgfpathlineto{\pgfqpoint{1.515300in}{0.867487in}}%
\pgfpathlineto{\pgfqpoint{1.453442in}{0.886061in}}%
\pgfpathlineto{\pgfqpoint{1.398991in}{0.904784in}}%
\pgfpathlineto{\pgfqpoint{1.352040in}{0.923498in}}%
\pgfpathlineto{\pgfqpoint{1.311919in}{0.942070in}}%
\pgfpathlineto{\pgfqpoint{1.277697in}{0.960408in}}%
\pgfpathlineto{\pgfqpoint{1.248630in}{0.978430in}}%
\pgfpathlineto{\pgfqpoint{1.224160in}{0.996062in}}%
\pgfpathlineto{\pgfqpoint{1.203920in}{1.013236in}}%
\pgfpathlineto{\pgfqpoint{1.187729in}{1.029895in}}%
\pgfpathlineto{\pgfqpoint{1.175597in}{1.045989in}}%
\pgfpathlineto{\pgfqpoint{1.167496in}{1.061475in}}%
\pgfpathlineto{\pgfqpoint{1.162503in}{1.076322in}}%
\pgfpathlineto{\pgfqpoint{1.160190in}{1.090511in}}%
\pgfpathlineto{\pgfqpoint{1.160283in}{1.104030in}}%
\pgfpathlineto{\pgfqpoint{1.162587in}{1.116868in}}%
\pgfpathlineto{\pgfqpoint{1.166980in}{1.129016in}}%
\pgfpathlineto{\pgfqpoint{1.173414in}{1.140471in}}%
\pgfpathlineto{\pgfqpoint{1.181919in}{1.151231in}}%
\pgfpathlineto{\pgfqpoint{1.192579in}{1.161298in}}%
\pgfpathlineto{\pgfqpoint{1.205271in}{1.170670in}}%
\pgfpathlineto{\pgfqpoint{1.219889in}{1.179342in}}%
\pgfpathlineto{\pgfqpoint{1.245362in}{1.191036in}}%
\pgfpathlineto{\pgfqpoint{1.275092in}{1.201145in}}%
\pgfpathlineto{\pgfqpoint{1.309205in}{1.209663in}}%
\pgfpathlineto{\pgfqpoint{1.347956in}{1.216585in}}%
\pgfpathlineto{\pgfqpoint{1.391733in}{1.221904in}}%
\pgfpathlineto{\pgfqpoint{1.441039in}{1.225616in}}%
\pgfpathlineto{\pgfqpoint{1.458763in}{1.226488in}}%
\pgfpathlineto{\pgfqpoint{1.458763in}{1.226488in}}%
\pgfusepath{stroke}%
\end{pgfscope}%
\begin{pgfscope}%
\pgfpathrectangle{\pgfqpoint{0.562500in}{0.275000in}}{\pgfqpoint{3.487500in}{1.925000in}}%
\pgfusepath{clip}%
\pgfsetrectcap%
\pgfsetroundjoin%
\pgfsetlinewidth{1.505625pt}%
\definecolor{currentstroke}{rgb}{1.000000,0.498039,0.054902}%
\pgfsetstrokecolor{currentstroke}%
\pgfsetdash{}{0pt}%
\pgfpathmoveto{\pgfqpoint{3.891477in}{2.112500in}}%
\pgfpathlineto{\pgfqpoint{3.425452in}{2.071696in}}%
\pgfpathlineto{\pgfqpoint{3.204197in}{2.034848in}}%
\pgfpathlineto{\pgfqpoint{3.082641in}{2.000406in}}%
\pgfpathlineto{\pgfqpoint{3.015062in}{1.967650in}}%
\pgfpathlineto{\pgfqpoint{2.977946in}{1.936163in}}%
\pgfpathlineto{\pgfqpoint{2.960507in}{1.905701in}}%
\pgfpathlineto{\pgfqpoint{2.955058in}{1.876085in}}%
\pgfpathlineto{\pgfqpoint{2.957836in}{1.847210in}}%
\pgfpathlineto{\pgfqpoint{2.966406in}{1.818999in}}%
\pgfpathlineto{\pgfqpoint{2.978677in}{1.791388in}}%
\pgfpathlineto{\pgfqpoint{2.993350in}{1.764331in}}%
\pgfpathlineto{\pgfqpoint{3.009819in}{1.737802in}}%
\pgfpathlineto{\pgfqpoint{3.045928in}{1.686228in}}%
\pgfpathlineto{\pgfqpoint{3.102222in}{1.612295in}}%
\pgfpathlineto{\pgfqpoint{3.274145in}{1.392520in}}%
\pgfpathlineto{\pgfqpoint{3.330883in}{1.315079in}}%
\pgfpathlineto{\pgfqpoint{3.380940in}{1.243204in}}%
\pgfpathlineto{\pgfqpoint{3.413561in}{1.192712in}}%
\pgfpathlineto{\pgfqpoint{3.441814in}{1.144999in}}%
\pgfpathlineto{\pgfqpoint{3.466149in}{1.100011in}}%
\pgfpathlineto{\pgfqpoint{3.487260in}{1.057732in}}%
\pgfpathlineto{\pgfqpoint{3.504835in}{1.018015in}}%
\pgfpathlineto{\pgfqpoint{3.518880in}{0.980736in}}%
\pgfpathlineto{\pgfqpoint{3.529438in}{0.945803in}}%
\pgfpathlineto{\pgfqpoint{3.536518in}{0.913155in}}%
\pgfpathlineto{\pgfqpoint{3.540079in}{0.882748in}}%
\pgfpathlineto{\pgfqpoint{3.540236in}{0.854496in}}%
\pgfpathlineto{\pgfqpoint{3.537253in}{0.828319in}}%
\pgfpathlineto{\pgfqpoint{3.533565in}{0.811986in}}%
\pgfpathlineto{\pgfqpoint{3.528496in}{0.796526in}}%
\pgfpathlineto{\pgfqpoint{3.521981in}{0.781923in}}%
\pgfpathlineto{\pgfqpoint{3.513919in}{0.768165in}}%
\pgfpathlineto{\pgfqpoint{3.504171in}{0.755240in}}%
\pgfpathlineto{\pgfqpoint{3.492707in}{0.743133in}}%
\pgfpathlineto{\pgfqpoint{3.479635in}{0.731832in}}%
\pgfpathlineto{\pgfqpoint{3.465019in}{0.721327in}}%
\pgfpathlineto{\pgfqpoint{3.440253in}{0.707047in}}%
\pgfpathlineto{\pgfqpoint{3.412024in}{0.694514in}}%
\pgfpathlineto{\pgfqpoint{3.380118in}{0.683702in}}%
\pgfpathlineto{\pgfqpoint{3.344146in}{0.674589in}}%
\pgfpathlineto{\pgfqpoint{3.303552in}{0.667152in}}%
\pgfpathlineto{\pgfqpoint{3.257761in}{0.661378in}}%
\pgfpathlineto{\pgfqpoint{3.206771in}{0.657300in}}%
\pgfpathlineto{\pgfqpoint{3.150023in}{0.654948in}}%
\pgfpathlineto{\pgfqpoint{3.086853in}{0.654364in}}%
\pgfpathlineto{\pgfqpoint{3.016586in}{0.655600in}}%
\pgfpathlineto{\pgfqpoint{2.910665in}{0.660194in}}%
\pgfpathlineto{\pgfqpoint{2.789246in}{0.668333in}}%
\pgfpathlineto{\pgfqpoint{2.650653in}{0.680244in}}%
\pgfpathlineto{\pgfqpoint{2.493981in}{0.696191in}}%
\pgfpathlineto{\pgfqpoint{2.320352in}{0.716431in}}%
\pgfpathlineto{\pgfqpoint{2.181629in}{0.734489in}}%
\pgfpathlineto{\pgfqpoint{2.039215in}{0.754985in}}%
\pgfpathlineto{\pgfqpoint{1.897560in}{0.777808in}}%
\pgfpathlineto{\pgfqpoint{1.762178in}{0.802664in}}%
\pgfpathlineto{\pgfqpoint{1.677797in}{0.820176in}}%
\pgfpathlineto{\pgfqpoint{1.599414in}{0.838293in}}%
\pgfpathlineto{\pgfqpoint{1.527854in}{0.856877in}}%
\pgfpathlineto{\pgfqpoint{1.463685in}{0.875779in}}%
\pgfpathlineto{\pgfqpoint{1.407286in}{0.894834in}}%
\pgfpathlineto{\pgfqpoint{1.358336in}{0.913896in}}%
\pgfpathlineto{\pgfqpoint{1.316051in}{0.932851in}}%
\pgfpathlineto{\pgfqpoint{1.279771in}{0.951595in}}%
\pgfpathlineto{\pgfqpoint{1.248963in}{0.970033in}}%
\pgfpathlineto{\pgfqpoint{1.223220in}{0.988080in}}%
\pgfpathlineto{\pgfqpoint{1.202264in}{1.005663in}}%
\pgfpathlineto{\pgfqpoint{1.185941in}{1.022716in}}%
\pgfpathlineto{\pgfqpoint{1.173836in}{1.039187in}}%
\pgfpathlineto{\pgfqpoint{1.165116in}{1.055040in}}%
\pgfpathlineto{\pgfqpoint{1.159341in}{1.070252in}}%
\pgfpathlineto{\pgfqpoint{1.156177in}{1.084804in}}%
\pgfpathlineto{\pgfqpoint{1.155390in}{1.098682in}}%
\pgfpathlineto{\pgfqpoint{1.156850in}{1.111873in}}%
\pgfpathlineto{\pgfqpoint{1.160527in}{1.124371in}}%
\pgfpathlineto{\pgfqpoint{1.166492in}{1.136173in}}%
\pgfpathlineto{\pgfqpoint{1.174699in}{1.147277in}}%
\pgfpathlineto{\pgfqpoint{1.184909in}{1.157677in}}%
\pgfpathlineto{\pgfqpoint{1.197027in}{1.167372in}}%
\pgfpathlineto{\pgfqpoint{1.210997in}{1.176359in}}%
\pgfpathlineto{\pgfqpoint{1.226789in}{1.184638in}}%
\pgfpathlineto{\pgfqpoint{1.253916in}{1.195728in}}%
\pgfpathlineto{\pgfqpoint{1.285320in}{1.205227in}}%
\pgfpathlineto{\pgfqpoint{1.321321in}{1.213138in}}%
\pgfpathlineto{\pgfqpoint{1.362346in}{1.219467in}}%
\pgfpathlineto{\pgfqpoint{1.408470in}{1.224190in}}%
\pgfpathlineto{\pgfqpoint{1.460127in}{1.227272in}}%
\pgfpathlineto{\pgfqpoint{1.517882in}{1.228676in}}%
\pgfpathlineto{\pgfqpoint{1.582331in}{1.228349in}}%
\pgfpathlineto{\pgfqpoint{1.679748in}{1.225115in}}%
\pgfpathlineto{\pgfqpoint{1.791708in}{1.218518in}}%
\pgfpathlineto{\pgfqpoint{1.919832in}{1.208350in}}%
\pgfpathlineto{\pgfqpoint{2.065354in}{1.194363in}}%
\pgfpathlineto{\pgfqpoint{2.228119in}{1.176291in}}%
\pgfpathlineto{\pgfqpoint{2.405150in}{1.153977in}}%
\pgfpathlineto{\pgfqpoint{2.543659in}{1.134447in}}%
\pgfpathlineto{\pgfqpoint{2.682642in}{1.112621in}}%
\pgfpathlineto{\pgfqpoint{2.816951in}{1.088739in}}%
\pgfpathlineto{\pgfqpoint{2.901619in}{1.071835in}}%
\pgfpathlineto{\pgfqpoint{2.981013in}{1.054279in}}%
\pgfpathlineto{\pgfqpoint{3.054121in}{1.036207in}}%
\pgfpathlineto{\pgfqpoint{3.120012in}{1.017772in}}%
\pgfpathlineto{\pgfqpoint{3.178440in}{0.999125in}}%
\pgfpathlineto{\pgfqpoint{3.229893in}{0.980401in}}%
\pgfpathlineto{\pgfqpoint{3.274836in}{0.961722in}}%
\pgfpathlineto{\pgfqpoint{3.313689in}{0.943199in}}%
\pgfpathlineto{\pgfqpoint{3.346834in}{0.924932in}}%
\pgfpathlineto{\pgfqpoint{3.374611in}{0.907011in}}%
\pgfpathlineto{\pgfqpoint{3.397319in}{0.889514in}}%
\pgfpathlineto{\pgfqpoint{3.415214in}{0.872509in}}%
\pgfpathlineto{\pgfqpoint{3.428766in}{0.856056in}}%
\pgfpathlineto{\pgfqpoint{3.438734in}{0.840195in}}%
\pgfpathlineto{\pgfqpoint{3.445649in}{0.824953in}}%
\pgfpathlineto{\pgfqpoint{3.449917in}{0.810351in}}%
\pgfpathlineto{\pgfqpoint{3.451820in}{0.796405in}}%
\pgfpathlineto{\pgfqpoint{3.451512in}{0.783129in}}%
\pgfpathlineto{\pgfqpoint{3.449026in}{0.770532in}}%
\pgfpathlineto{\pgfqpoint{3.444267in}{0.758619in}}%
\pgfpathlineto{\pgfqpoint{3.437025in}{0.747391in}}%
\pgfpathlineto{\pgfqpoint{3.427523in}{0.736855in}}%
\pgfpathlineto{\pgfqpoint{3.416051in}{0.727018in}}%
\pgfpathlineto{\pgfqpoint{3.402678in}{0.717883in}}%
\pgfpathlineto{\pgfqpoint{3.387443in}{0.709451in}}%
\pgfpathlineto{\pgfqpoint{3.361113in}{0.698124in}}%
\pgfpathlineto{\pgfqpoint{3.330519in}{0.688383in}}%
\pgfpathlineto{\pgfqpoint{3.295421in}{0.680228in}}%
\pgfpathlineto{\pgfqpoint{3.255426in}{0.673656in}}%
\pgfpathlineto{\pgfqpoint{3.210297in}{0.668677in}}%
\pgfpathlineto{\pgfqpoint{3.159770in}{0.665326in}}%
\pgfpathlineto{\pgfqpoint{3.103243in}{0.663642in}}%
\pgfpathlineto{\pgfqpoint{3.040107in}{0.663674in}}%
\pgfpathlineto{\pgfqpoint{2.969740in}{0.665483in}}%
\pgfpathlineto{\pgfqpoint{2.863578in}{0.670781in}}%
\pgfpathlineto{\pgfqpoint{2.741924in}{0.679565in}}%
\pgfpathlineto{\pgfqpoint{2.603281in}{0.692064in}}%
\pgfpathlineto{\pgfqpoint{2.447140in}{0.708537in}}%
\pgfpathlineto{\pgfqpoint{2.275198in}{0.729219in}}%
\pgfpathlineto{\pgfqpoint{2.138801in}{0.747527in}}%
\pgfpathlineto{\pgfqpoint{1.999773in}{0.768175in}}%
\pgfpathlineto{\pgfqpoint{1.862588in}{0.791038in}}%
\pgfpathlineto{\pgfqpoint{1.774704in}{0.807367in}}%
\pgfpathlineto{\pgfqpoint{1.691312in}{0.824424in}}%
\pgfpathlineto{\pgfqpoint{1.613506in}{0.842086in}}%
\pgfpathlineto{\pgfqpoint{1.542111in}{0.860220in}}%
\pgfpathlineto{\pgfqpoint{1.477676in}{0.878693in}}%
\pgfpathlineto{\pgfqpoint{1.420481in}{0.897364in}}%
\pgfpathlineto{\pgfqpoint{1.370628in}{0.916082in}}%
\pgfpathlineto{\pgfqpoint{1.328122in}{0.934689in}}%
\pgfpathlineto{\pgfqpoint{1.291875in}{0.953092in}}%
\pgfpathlineto{\pgfqpoint{1.260882in}{0.971219in}}%
\pgfpathlineto{\pgfqpoint{1.234392in}{0.988997in}}%
\pgfpathlineto{\pgfqpoint{1.211903in}{1.006363in}}%
\pgfpathlineto{\pgfqpoint{1.193166in}{1.023257in}}%
\pgfpathlineto{\pgfqpoint{1.178187in}{1.039622in}}%
\pgfpathlineto{\pgfqpoint{1.167220in}{1.055410in}}%
\pgfpathlineto{\pgfqpoint{1.160716in}{1.070576in}}%
\pgfpathlineto{\pgfqpoint{1.157613in}{1.085084in}}%
\pgfpathlineto{\pgfqpoint{1.157063in}{1.098917in}}%
\pgfpathlineto{\pgfqpoint{1.158826in}{1.112067in}}%
\pgfpathlineto{\pgfqpoint{1.162720in}{1.124524in}}%
\pgfpathlineto{\pgfqpoint{1.168629in}{1.136283in}}%
\pgfpathlineto{\pgfqpoint{1.176498in}{1.147342in}}%
\pgfpathlineto{\pgfqpoint{1.186337in}{1.157700in}}%
\pgfpathlineto{\pgfqpoint{1.198214in}{1.167359in}}%
\pgfpathlineto{\pgfqpoint{1.212149in}{1.176320in}}%
\pgfpathlineto{\pgfqpoint{1.228015in}{1.184581in}}%
\pgfpathlineto{\pgfqpoint{1.255390in}{1.195655in}}%
\pgfpathlineto{\pgfqpoint{1.287092in}{1.205140in}}%
\pgfpathlineto{\pgfqpoint{1.323265in}{1.213028in}}%
\pgfpathlineto{\pgfqpoint{1.364175in}{1.219308in}}%
\pgfpathlineto{\pgfqpoint{1.410206in}{1.223969in}}%
\pgfpathlineto{\pgfqpoint{1.461862in}{1.227000in}}%
\pgfpathlineto{\pgfqpoint{1.519449in}{1.228382in}}%
\pgfpathlineto{\pgfqpoint{1.583417in}{1.228055in}}%
\pgfpathlineto{\pgfqpoint{1.680542in}{1.224819in}}%
\pgfpathlineto{\pgfqpoint{1.792972in}{1.218198in}}%
\pgfpathlineto{\pgfqpoint{1.921879in}{1.207980in}}%
\pgfpathlineto{\pgfqpoint{2.067623in}{1.193950in}}%
\pgfpathlineto{\pgfqpoint{2.229748in}{1.175882in}}%
\pgfpathlineto{\pgfqpoint{2.407030in}{1.153526in}}%
\pgfpathlineto{\pgfqpoint{2.546380in}{1.133885in}}%
\pgfpathlineto{\pgfqpoint{2.685213in}{1.111994in}}%
\pgfpathlineto{\pgfqpoint{2.818726in}{1.088119in}}%
\pgfpathlineto{\pgfqpoint{2.902782in}{1.071255in}}%
\pgfpathlineto{\pgfqpoint{2.981715in}{1.053758in}}%
\pgfpathlineto{\pgfqpoint{3.054713in}{1.035745in}}%
\pgfpathlineto{\pgfqpoint{3.121130in}{1.017344in}}%
\pgfpathlineto{\pgfqpoint{3.180479in}{0.998693in}}%
\pgfpathlineto{\pgfqpoint{3.232438in}{0.979940in}}%
\pgfpathlineto{\pgfqpoint{3.276951in}{0.961243in}}%
\pgfpathlineto{\pgfqpoint{3.314781in}{0.942729in}}%
\pgfpathlineto{\pgfqpoint{3.346835in}{0.924488in}}%
\pgfpathlineto{\pgfqpoint{3.373838in}{0.906599in}}%
\pgfpathlineto{\pgfqpoint{3.396341in}{0.889133in}}%
\pgfpathlineto{\pgfqpoint{3.414721in}{0.872152in}}%
\pgfpathlineto{\pgfqpoint{3.429177in}{0.855709in}}%
\pgfpathlineto{\pgfqpoint{3.439736in}{0.839848in}}%
\pgfpathlineto{\pgfqpoint{3.446660in}{0.824607in}}%
\pgfpathlineto{\pgfqpoint{3.450694in}{0.810011in}}%
\pgfpathlineto{\pgfqpoint{3.452143in}{0.796077in}}%
\pgfpathlineto{\pgfqpoint{3.451238in}{0.782817in}}%
\pgfpathlineto{\pgfqpoint{3.448144in}{0.770241in}}%
\pgfpathlineto{\pgfqpoint{3.442953in}{0.758355in}}%
\pgfpathlineto{\pgfqpoint{3.435691in}{0.747163in}}%
\pgfpathlineto{\pgfqpoint{3.426315in}{0.736665in}}%
\pgfpathlineto{\pgfqpoint{3.414867in}{0.726862in}}%
\pgfpathlineto{\pgfqpoint{3.401466in}{0.717755in}}%
\pgfpathlineto{\pgfqpoint{3.386166in}{0.709349in}}%
\pgfpathlineto{\pgfqpoint{3.359698in}{0.698055in}}%
\pgfpathlineto{\pgfqpoint{3.328974in}{0.688348in}}%
\pgfpathlineto{\pgfqpoint{3.293821in}{0.680232in}}%
\pgfpathlineto{\pgfqpoint{3.253931in}{0.673710in}}%
\pgfpathlineto{\pgfqpoint{3.208858in}{0.668787in}}%
\pgfpathlineto{\pgfqpoint{3.158246in}{0.665472in}}%
\pgfpathlineto{\pgfqpoint{3.101840in}{0.663814in}}%
\pgfpathlineto{\pgfqpoint{3.038817in}{0.663869in}}%
\pgfpathlineto{\pgfqpoint{2.968422in}{0.665701in}}%
\pgfpathlineto{\pgfqpoint{2.861954in}{0.671034in}}%
\pgfpathlineto{\pgfqpoint{2.739877in}{0.679856in}}%
\pgfpathlineto{\pgfqpoint{2.601146in}{0.692396in}}%
\pgfpathlineto{\pgfqpoint{2.444714in}{0.708927in}}%
\pgfpathlineto{\pgfqpoint{2.268615in}{0.729868in}}%
\pgfpathlineto{\pgfqpoint{2.129807in}{0.748417in}}%
\pgfpathlineto{\pgfqpoint{1.990945in}{0.769225in}}%
\pgfpathlineto{\pgfqpoint{1.856383in}{0.792079in}}%
\pgfpathlineto{\pgfqpoint{1.729747in}{0.816715in}}%
\pgfpathlineto{\pgfqpoint{1.651186in}{0.833973in}}%
\pgfpathlineto{\pgfqpoint{1.578128in}{0.851778in}}%
\pgfpathlineto{\pgfqpoint{1.511106in}{0.870012in}}%
\pgfpathlineto{\pgfqpoint{1.450512in}{0.888549in}}%
\pgfpathlineto{\pgfqpoint{1.396590in}{0.907252in}}%
\pgfpathlineto{\pgfqpoint{1.349444in}{0.925975in}}%
\pgfpathlineto{\pgfqpoint{1.309030in}{0.944561in}}%
\pgfpathlineto{\pgfqpoint{1.275115in}{0.962853in}}%
\pgfpathlineto{\pgfqpoint{1.246852in}{0.980791in}}%
\pgfpathlineto{\pgfqpoint{1.223535in}{0.998318in}}%
\pgfpathlineto{\pgfqpoint{1.204630in}{1.015369in}}%
\pgfpathlineto{\pgfqpoint{1.189697in}{1.031892in}}%
\pgfpathlineto{\pgfqpoint{1.178390in}{1.047843in}}%
\pgfpathlineto{\pgfqpoint{1.170395in}{1.063187in}}%
\pgfpathlineto{\pgfqpoint{1.165350in}{1.077900in}}%
\pgfpathlineto{\pgfqpoint{1.162992in}{1.091961in}}%
\pgfpathlineto{\pgfqpoint{1.163112in}{1.105356in}}%
\pgfpathlineto{\pgfqpoint{1.165556in}{1.118074in}}%
\pgfpathlineto{\pgfqpoint{1.170228in}{1.130108in}}%
\pgfpathlineto{\pgfqpoint{1.177102in}{1.141455in}}%
\pgfpathlineto{\pgfqpoint{1.186050in}{1.152111in}}%
\pgfpathlineto{\pgfqpoint{1.196942in}{1.162072in}}%
\pgfpathlineto{\pgfqpoint{1.209690in}{1.171335in}}%
\pgfpathlineto{\pgfqpoint{1.224243in}{1.179897in}}%
\pgfpathlineto{\pgfqpoint{1.249447in}{1.191425in}}%
\pgfpathlineto{\pgfqpoint{1.278838in}{1.201374in}}%
\pgfpathlineto{\pgfqpoint{1.312750in}{1.209753in}}%
\pgfpathlineto{\pgfqpoint{1.351717in}{1.216572in}}%
\pgfpathlineto{\pgfqpoint{1.395857in}{1.221811in}}%
\pgfpathlineto{\pgfqpoint{1.445361in}{1.225430in}}%
\pgfpathlineto{\pgfqpoint{1.500768in}{1.227391in}}%
\pgfpathlineto{\pgfqpoint{1.562651in}{1.227647in}}%
\pgfpathlineto{\pgfqpoint{1.631611in}{1.226142in}}%
\pgfpathlineto{\pgfqpoint{1.735671in}{1.221272in}}%
\pgfpathlineto{\pgfqpoint{1.855043in}{1.212955in}}%
\pgfpathlineto{\pgfqpoint{1.991270in}{1.200961in}}%
\pgfpathlineto{\pgfqpoint{2.145020in}{1.185022in}}%
\pgfpathlineto{\pgfqpoint{2.314853in}{1.164921in}}%
\pgfpathlineto{\pgfqpoint{2.450221in}{1.147052in}}%
\pgfpathlineto{\pgfqpoint{2.589037in}{1.126813in}}%
\pgfpathlineto{\pgfqpoint{2.726575in}{1.104356in}}%
\pgfpathlineto{\pgfqpoint{2.858099in}{1.079938in}}%
\pgfpathlineto{\pgfqpoint{2.940252in}{1.062742in}}%
\pgfpathlineto{\pgfqpoint{3.016521in}{1.044961in}}%
\pgfpathlineto{\pgfqpoint{3.085445in}{1.026747in}}%
\pgfpathlineto{\pgfqpoint{3.147044in}{1.008251in}}%
\pgfpathlineto{\pgfqpoint{3.201819in}{0.989608in}}%
\pgfpathlineto{\pgfqpoint{3.250214in}{0.970945in}}%
\pgfpathlineto{\pgfqpoint{3.292624in}{0.952376in}}%
\pgfpathlineto{\pgfqpoint{3.329389in}{0.934005in}}%
\pgfpathlineto{\pgfqpoint{3.360798in}{0.915924in}}%
\pgfpathlineto{\pgfqpoint{3.387088in}{0.898215in}}%
\pgfpathlineto{\pgfqpoint{3.408442in}{0.880950in}}%
\pgfpathlineto{\pgfqpoint{3.424992in}{0.864188in}}%
\pgfpathlineto{\pgfqpoint{3.436867in}{0.847980in}}%
\pgfpathlineto{\pgfqpoint{3.444917in}{0.832386in}}%
\pgfpathlineto{\pgfqpoint{3.449903in}{0.817431in}}%
\pgfpathlineto{\pgfqpoint{3.452325in}{0.803133in}}%
\pgfpathlineto{\pgfqpoint{3.452561in}{0.789505in}}%
\pgfpathlineto{\pgfqpoint{3.450857in}{0.776555in}}%
\pgfpathlineto{\pgfqpoint{3.447334in}{0.764292in}}%
\pgfpathlineto{\pgfqpoint{3.441987in}{0.752718in}}%
\pgfpathlineto{\pgfqpoint{3.434680in}{0.741833in}}%
\pgfpathlineto{\pgfqpoint{3.425154in}{0.731634in}}%
\pgfpathlineto{\pgfqpoint{3.413026in}{0.722114in}}%
\pgfpathlineto{\pgfqpoint{3.398596in}{0.713287in}}%
\pgfpathlineto{\pgfqpoint{3.382257in}{0.705163in}}%
\pgfpathlineto{\pgfqpoint{3.354179in}{0.694301in}}%
\pgfpathlineto{\pgfqpoint{3.321756in}{0.685032in}}%
\pgfpathlineto{\pgfqpoint{3.284809in}{0.677364in}}%
\pgfpathlineto{\pgfqpoint{3.243048in}{0.671306in}}%
\pgfpathlineto{\pgfqpoint{3.196069in}{0.666867in}}%
\pgfpathlineto{\pgfqpoint{3.143455in}{0.664060in}}%
\pgfpathlineto{\pgfqpoint{3.085006in}{0.662921in}}%
\pgfpathlineto{\pgfqpoint{3.019773in}{0.663515in}}%
\pgfpathlineto{\pgfqpoint{2.920714in}{0.667130in}}%
\pgfpathlineto{\pgfqpoint{2.806378in}{0.674146in}}%
\pgfpathlineto{\pgfqpoint{2.675697in}{0.684774in}}%
\pgfpathlineto{\pgfqpoint{2.528175in}{0.699243in}}%
\pgfpathlineto{\pgfqpoint{2.363872in}{0.717801in}}%
\pgfpathlineto{\pgfqpoint{2.183866in}{0.740722in}}%
\pgfpathlineto{\pgfqpoint{2.044387in}{0.760712in}}%
\pgfpathlineto{\pgfqpoint{1.906596in}{0.782902in}}%
\pgfpathlineto{\pgfqpoint{1.774770in}{0.807044in}}%
\pgfpathlineto{\pgfqpoint{1.692023in}{0.824069in}}%
\pgfpathlineto{\pgfqpoint{1.614479in}{0.841711in}}%
\pgfpathlineto{\pgfqpoint{1.542930in}{0.859847in}}%
\pgfpathlineto{\pgfqpoint{1.478045in}{0.878341in}}%
\pgfpathlineto{\pgfqpoint{1.420366in}{0.897043in}}%
\pgfpathlineto{\pgfqpoint{1.370263in}{0.915791in}}%
\pgfpathlineto{\pgfqpoint{1.327228in}{0.934449in}}%
\pgfpathlineto{\pgfqpoint{1.290470in}{0.952914in}}%
\pgfpathlineto{\pgfqpoint{1.259352in}{0.971091in}}%
\pgfpathlineto{\pgfqpoint{1.233342in}{0.988895in}}%
\pgfpathlineto{\pgfqpoint{1.212007in}{1.006256in}}%
\pgfpathlineto{\pgfqpoint{1.195015in}{1.023110in}}%
\pgfpathlineto{\pgfqpoint{1.182066in}{1.039408in}}%
\pgfpathlineto{\pgfqpoint{1.172584in}{1.055111in}}%
\pgfpathlineto{\pgfqpoint{1.166159in}{1.070191in}}%
\pgfpathlineto{\pgfqpoint{1.162488in}{1.084626in}}%
\pgfpathlineto{\pgfqpoint{1.161342in}{1.098401in}}%
\pgfpathlineto{\pgfqpoint{1.162571in}{1.111501in}}%
\pgfpathlineto{\pgfqpoint{1.166103in}{1.123918in}}%
\pgfpathlineto{\pgfqpoint{1.171904in}{1.135647in}}%
\pgfpathlineto{\pgfqpoint{1.179832in}{1.146686in}}%
\pgfpathlineto{\pgfqpoint{1.189749in}{1.157029in}}%
\pgfpathlineto{\pgfqpoint{1.201558in}{1.166673in}}%
\pgfpathlineto{\pgfqpoint{1.215200in}{1.175615in}}%
\pgfpathlineto{\pgfqpoint{1.230652in}{1.183856in}}%
\pgfpathlineto{\pgfqpoint{1.257271in}{1.194900in}}%
\pgfpathlineto{\pgfqpoint{1.288225in}{1.204368in}}%
\pgfpathlineto{\pgfqpoint{1.323925in}{1.212268in}}%
\pgfpathlineto{\pgfqpoint{1.364691in}{1.218600in}}%
\pgfpathlineto{\pgfqpoint{1.410546in}{1.223328in}}%
\pgfpathlineto{\pgfqpoint{1.461953in}{1.226422in}}%
\pgfpathlineto{\pgfqpoint{1.519438in}{1.227841in}}%
\pgfpathlineto{\pgfqpoint{1.583569in}{1.227536in}}%
\pgfpathlineto{\pgfqpoint{1.680483in}{1.224341in}}%
\pgfpathlineto{\pgfqpoint{1.791890in}{1.217800in}}%
\pgfpathlineto{\pgfqpoint{1.919460in}{1.207707in}}%
\pgfpathlineto{\pgfqpoint{2.064407in}{1.193800in}}%
\pgfpathlineto{\pgfqpoint{2.226479in}{1.175831in}}%
\pgfpathlineto{\pgfqpoint{2.402911in}{1.153631in}}%
\pgfpathlineto{\pgfqpoint{2.541025in}{1.134190in}}%
\pgfpathlineto{\pgfqpoint{2.679539in}{1.112467in}}%
\pgfpathlineto{\pgfqpoint{2.813847in}{1.088677in}}%
\pgfpathlineto{\pgfqpoint{2.898611in}{1.071825in}}%
\pgfpathlineto{\pgfqpoint{2.977763in}{1.054300in}}%
\pgfpathlineto{\pgfqpoint{3.050554in}{1.036254in}}%
\pgfpathlineto{\pgfqpoint{3.116632in}{1.017849in}}%
\pgfpathlineto{\pgfqpoint{3.175796in}{0.999235in}}%
\pgfpathlineto{\pgfqpoint{3.228003in}{0.980548in}}%
\pgfpathlineto{\pgfqpoint{3.273362in}{0.961913in}}%
\pgfpathlineto{\pgfqpoint{3.312139in}{0.943441in}}%
\pgfpathlineto{\pgfqpoint{3.344751in}{0.925230in}}%
\pgfpathlineto{\pgfqpoint{3.371772in}{0.907364in}}%
\pgfpathlineto{\pgfqpoint{3.393931in}{0.889917in}}%
\pgfpathlineto{\pgfqpoint{3.411626in}{0.872963in}}%
\pgfpathlineto{\pgfqpoint{3.425121in}{0.856560in}}%
\pgfpathlineto{\pgfqpoint{3.435218in}{0.840738in}}%
\pgfpathlineto{\pgfqpoint{3.442536in}{0.825522in}}%
\pgfpathlineto{\pgfqpoint{3.447509in}{0.810932in}}%
\pgfpathlineto{\pgfqpoint{3.450385in}{0.796985in}}%
\pgfpathlineto{\pgfqpoint{3.451223in}{0.783695in}}%
\pgfpathlineto{\pgfqpoint{3.449897in}{0.771069in}}%
\pgfpathlineto{\pgfqpoint{3.446094in}{0.759114in}}%
\pgfpathlineto{\pgfqpoint{3.439314in}{0.747830in}}%
\pgfpathlineto{\pgfqpoint{3.429828in}{0.737232in}}%
\pgfpathlineto{\pgfqpoint{3.418377in}{0.727337in}}%
\pgfpathlineto{\pgfqpoint{3.405028in}{0.718146in}}%
\pgfpathlineto{\pgfqpoint{3.389816in}{0.709659in}}%
\pgfpathlineto{\pgfqpoint{3.363515in}{0.698254in}}%
\pgfpathlineto{\pgfqpoint{3.332943in}{0.688439in}}%
\pgfpathlineto{\pgfqpoint{3.297868in}{0.680213in}}%
\pgfpathlineto{\pgfqpoint{3.257915in}{0.673573in}}%
\pgfpathlineto{\pgfqpoint{3.212898in}{0.668531in}}%
\pgfpathlineto{\pgfqpoint{3.162480in}{0.665120in}}%
\pgfpathlineto{\pgfqpoint{3.106078in}{0.663377in}}%
\pgfpathlineto{\pgfqpoint{3.043089in}{0.663351in}}%
\pgfpathlineto{\pgfqpoint{2.972899in}{0.665101in}}%
\pgfpathlineto{\pgfqpoint{2.867014in}{0.670321in}}%
\pgfpathlineto{\pgfqpoint{2.745655in}{0.679024in}}%
\pgfpathlineto{\pgfqpoint{2.607305in}{0.691438in}}%
\pgfpathlineto{\pgfqpoint{2.451431in}{0.707829in}}%
\pgfpathlineto{\pgfqpoint{2.279665in}{0.728422in}}%
\pgfpathlineto{\pgfqpoint{2.143300in}{0.746665in}}%
\pgfpathlineto{\pgfqpoint{2.004152in}{0.767261in}}%
\pgfpathlineto{\pgfqpoint{1.866687in}{0.790076in}}%
\pgfpathlineto{\pgfqpoint{1.736145in}{0.814800in}}%
\pgfpathlineto{\pgfqpoint{1.655136in}{0.832155in}}%
\pgfpathlineto{\pgfqpoint{1.580107in}{0.850064in}}%
\pgfpathlineto{\pgfqpoint{1.511784in}{0.868391in}}%
\pgfpathlineto{\pgfqpoint{1.450652in}{0.886991in}}%
\pgfpathlineto{\pgfqpoint{1.397045in}{0.905707in}}%
\pgfpathlineto{\pgfqpoint{1.350591in}{0.924399in}}%
\pgfpathlineto{\pgfqpoint{1.310490in}{0.942960in}}%
\pgfpathlineto{\pgfqpoint{1.276081in}{0.961294in}}%
\pgfpathlineto{\pgfqpoint{1.246840in}{0.979311in}}%
\pgfpathlineto{\pgfqpoint{1.222383in}{0.996932in}}%
\pgfpathlineto{\pgfqpoint{1.202469in}{1.014088in}}%
\pgfpathlineto{\pgfqpoint{1.186994in}{1.030719in}}%
\pgfpathlineto{\pgfqpoint{1.175791in}{1.046774in}}%
\pgfpathlineto{\pgfqpoint{1.167976in}{1.062214in}}%
\pgfpathlineto{\pgfqpoint{1.163073in}{1.077017in}}%
\pgfpathlineto{\pgfqpoint{1.160747in}{1.091165in}}%
\pgfpathlineto{\pgfqpoint{1.160757in}{1.104643in}}%
\pgfpathlineto{\pgfqpoint{1.162956in}{1.117441in}}%
\pgfpathlineto{\pgfqpoint{1.167285in}{1.129552in}}%
\pgfpathlineto{\pgfqpoint{1.173780in}{1.140973in}}%
\pgfpathlineto{\pgfqpoint{1.182544in}{1.151704in}}%
\pgfpathlineto{\pgfqpoint{1.193395in}{1.161740in}}%
\pgfpathlineto{\pgfqpoint{1.206191in}{1.171077in}}%
\pgfpathlineto{\pgfqpoint{1.220876in}{1.179713in}}%
\pgfpathlineto{\pgfqpoint{1.246397in}{1.191349in}}%
\pgfpathlineto{\pgfqpoint{1.276146in}{1.201400in}}%
\pgfpathlineto{\pgfqpoint{1.310300in}{1.209862in}}%
\pgfpathlineto{\pgfqpoint{1.349177in}{1.216735in}}%
\pgfpathlineto{\pgfqpoint{1.393228in}{1.222020in}}%
\pgfpathlineto{\pgfqpoint{1.442604in}{1.225693in}}%
\pgfpathlineto{\pgfqpoint{1.497784in}{1.227711in}}%
\pgfpathlineto{\pgfqpoint{1.559462in}{1.228026in}}%
\pgfpathlineto{\pgfqpoint{1.628305in}{1.226580in}}%
\pgfpathlineto{\pgfqpoint{1.732333in}{1.221788in}}%
\pgfpathlineto{\pgfqpoint{1.851649in}{1.213543in}}%
\pgfpathlineto{\pgfqpoint{1.987586in}{1.201616in}}%
\pgfpathlineto{\pgfqpoint{2.141366in}{1.185773in}}%
\pgfpathlineto{\pgfqpoint{2.311541in}{1.165768in}}%
\pgfpathlineto{\pgfqpoint{2.446745in}{1.147928in}}%
\pgfpathlineto{\pgfqpoint{2.585022in}{1.127707in}}%
\pgfpathlineto{\pgfqpoint{2.722542in}{1.105258in}}%
\pgfpathlineto{\pgfqpoint{2.854779in}{1.080843in}}%
\pgfpathlineto{\pgfqpoint{2.937450in}{1.063649in}}%
\pgfpathlineto{\pgfqpoint{3.013778in}{1.045868in}}%
\pgfpathlineto{\pgfqpoint{3.083003in}{1.027636in}}%
\pgfpathlineto{\pgfqpoint{3.145257in}{1.009112in}}%
\pgfpathlineto{\pgfqpoint{3.200725in}{0.990438in}}%
\pgfpathlineto{\pgfqpoint{3.249627in}{0.971748in}}%
\pgfpathlineto{\pgfqpoint{3.292215in}{0.953160in}}%
\pgfpathlineto{\pgfqpoint{3.328775in}{0.934779in}}%
\pgfpathlineto{\pgfqpoint{3.359625in}{0.916698in}}%
\pgfpathlineto{\pgfqpoint{3.385117in}{0.898999in}}%
\pgfpathlineto{\pgfqpoint{3.405637in}{0.881748in}}%
\pgfpathlineto{\pgfqpoint{3.421602in}{0.864999in}}%
\pgfpathlineto{\pgfqpoint{3.433306in}{0.848818in}}%
\pgfpathlineto{\pgfqpoint{3.441363in}{0.833249in}}%
\pgfpathlineto{\pgfqpoint{3.446516in}{0.818310in}}%
\pgfpathlineto{\pgfqpoint{3.449335in}{0.804018in}}%
\pgfpathlineto{\pgfqpoint{3.450221in}{0.790385in}}%
\pgfpathlineto{\pgfqpoint{3.449406in}{0.777420in}}%
\pgfpathlineto{\pgfqpoint{3.446952in}{0.765130in}}%
\pgfpathlineto{\pgfqpoint{3.442750in}{0.753516in}}%
\pgfpathlineto{\pgfqpoint{3.436523in}{0.742578in}}%
\pgfpathlineto{\pgfqpoint{3.427823in}{0.732313in}}%
\pgfpathlineto{\pgfqpoint{3.416043in}{0.722714in}}%
\pgfpathlineto{\pgfqpoint{3.401692in}{0.713803in}}%
\pgfpathlineto{\pgfqpoint{3.385439in}{0.705598in}}%
\pgfpathlineto{\pgfqpoint{3.357508in}{0.694619in}}%
\pgfpathlineto{\pgfqpoint{3.325246in}{0.685238in}}%
\pgfpathlineto{\pgfqpoint{3.288477in}{0.677460in}}%
\pgfpathlineto{\pgfqpoint{3.246908in}{0.671294in}}%
\pgfpathlineto{\pgfqpoint{3.200139in}{0.666749in}}%
\pgfpathlineto{\pgfqpoint{3.147807in}{0.663836in}}%
\pgfpathlineto{\pgfqpoint{3.089630in}{0.662595in}}%
\pgfpathlineto{\pgfqpoint{3.024668in}{0.663088in}}%
\pgfpathlineto{\pgfqpoint{2.926057in}{0.666564in}}%
\pgfpathlineto{\pgfqpoint{2.812308in}{0.673433in}}%
\pgfpathlineto{\pgfqpoint{2.682307in}{0.683906in}}%
\pgfpathlineto{\pgfqpoint{2.535417in}{0.698214in}}%
\pgfpathlineto{\pgfqpoint{2.371382in}{0.716619in}}%
\pgfpathlineto{\pgfqpoint{2.191376in}{0.739382in}}%
\pgfpathlineto{\pgfqpoint{2.052014in}{0.759247in}}%
\pgfpathlineto{\pgfqpoint{1.914191in}{0.781327in}}%
\pgfpathlineto{\pgfqpoint{1.782056in}{0.805383in}}%
\pgfpathlineto{\pgfqpoint{1.698927in}{0.822368in}}%
\pgfpathlineto{\pgfqpoint{1.620868in}{0.839984in}}%
\pgfpathlineto{\pgfqpoint{1.548701in}{0.858107in}}%
\pgfpathlineto{\pgfqpoint{1.483140in}{0.876599in}}%
\pgfpathlineto{\pgfqpoint{1.424790in}{0.895307in}}%
\pgfpathlineto{\pgfqpoint{1.374080in}{0.914067in}}%
\pgfpathlineto{\pgfqpoint{1.330451in}{0.932748in}}%
\pgfpathlineto{\pgfqpoint{1.293144in}{0.951244in}}%
\pgfpathlineto{\pgfqpoint{1.261555in}{0.969458in}}%
\pgfpathlineto{\pgfqpoint{1.235165in}{0.987306in}}%
\pgfpathlineto{\pgfqpoint{1.213536in}{1.004713in}}%
\pgfpathlineto{\pgfqpoint{1.196312in}{1.021617in}}%
\pgfpathlineto{\pgfqpoint{1.183080in}{1.037966in}}%
\pgfpathlineto{\pgfqpoint{1.173298in}{1.053724in}}%
\pgfpathlineto{\pgfqpoint{1.166589in}{1.068862in}}%
\pgfpathlineto{\pgfqpoint{1.162656in}{1.083357in}}%
\pgfpathlineto{\pgfqpoint{1.161282in}{1.097192in}}%
\pgfpathlineto{\pgfqpoint{1.162332in}{1.110353in}}%
\pgfpathlineto{\pgfqpoint{1.165744in}{1.122833in}}%
\pgfpathlineto{\pgfqpoint{1.171418in}{1.134625in}}%
\pgfpathlineto{\pgfqpoint{1.179182in}{1.145726in}}%
\pgfpathlineto{\pgfqpoint{1.188900in}{1.156129in}}%
\pgfpathlineto{\pgfqpoint{1.200478in}{1.165833in}}%
\pgfpathlineto{\pgfqpoint{1.213867in}{1.174836in}}%
\pgfpathlineto{\pgfqpoint{1.229057in}{1.183137in}}%
\pgfpathlineto{\pgfqpoint{1.255305in}{1.194274in}}%
\pgfpathlineto{\pgfqpoint{1.285982in}{1.203842in}}%
\pgfpathlineto{\pgfqpoint{1.321556in}{1.211852in}}%
\pgfpathlineto{\pgfqpoint{1.362024in}{1.218285in}}%
\pgfpathlineto{\pgfqpoint{1.407589in}{1.223112in}}%
\pgfpathlineto{\pgfqpoint{1.458684in}{1.226304in}}%
\pgfpathlineto{\pgfqpoint{1.515803in}{1.227822in}}%
\pgfpathlineto{\pgfqpoint{1.579503in}{1.227617in}}%
\pgfpathlineto{\pgfqpoint{1.675751in}{1.224564in}}%
\pgfpathlineto{\pgfqpoint{1.786442in}{1.218181in}}%
\pgfpathlineto{\pgfqpoint{1.913247in}{1.208259in}}%
\pgfpathlineto{\pgfqpoint{2.057437in}{1.194528in}}%
\pgfpathlineto{\pgfqpoint{2.218732in}{1.176751in}}%
\pgfpathlineto{\pgfqpoint{2.394671in}{1.154744in}}%
\pgfpathlineto{\pgfqpoint{2.532550in}{1.135445in}}%
\pgfpathlineto{\pgfqpoint{2.671181in}{1.113855in}}%
\pgfpathlineto{\pgfqpoint{2.805941in}{1.090182in}}%
\pgfpathlineto{\pgfqpoint{2.891033in}{1.073385in}}%
\pgfpathlineto{\pgfqpoint{2.970716in}{1.055899in}}%
\pgfpathlineto{\pgfqpoint{3.044261in}{1.037890in}}%
\pgfpathlineto{\pgfqpoint{3.111159in}{1.019517in}}%
\pgfpathlineto{\pgfqpoint{3.171096in}{1.000925in}}%
\pgfpathlineto{\pgfqpoint{3.223958in}{0.982248in}}%
\pgfpathlineto{\pgfqpoint{3.269831in}{0.963608in}}%
\pgfpathlineto{\pgfqpoint{3.308998in}{0.945118in}}%
\pgfpathlineto{\pgfqpoint{3.341941in}{0.926877in}}%
\pgfpathlineto{\pgfqpoint{3.369339in}{0.908971in}}%
\pgfpathlineto{\pgfqpoint{3.391944in}{0.891481in}}%
\pgfpathlineto{\pgfqpoint{3.409828in}{0.874489in}}%
\pgfpathlineto{\pgfqpoint{3.423732in}{0.858039in}}%
\pgfpathlineto{\pgfqpoint{3.434392in}{0.842161in}}%
\pgfpathlineto{\pgfqpoint{3.442346in}{0.826882in}}%
\pgfpathlineto{\pgfqpoint{3.447937in}{0.812224in}}%
\pgfpathlineto{\pgfqpoint{3.451309in}{0.798206in}}%
\pgfpathlineto{\pgfqpoint{3.452409in}{0.784843in}}%
\pgfpathlineto{\pgfqpoint{3.450988in}{0.772143in}}%
\pgfpathlineto{\pgfqpoint{3.446597in}{0.760114in}}%
\pgfpathlineto{\pgfqpoint{3.439310in}{0.748767in}}%
\pgfpathlineto{\pgfqpoint{3.429985in}{0.738120in}}%
\pgfpathlineto{\pgfqpoint{3.418725in}{0.728176in}}%
\pgfpathlineto{\pgfqpoint{3.405592in}{0.718936in}}%
\pgfpathlineto{\pgfqpoint{3.390616in}{0.710401in}}%
\pgfpathlineto{\pgfqpoint{3.364682in}{0.698921in}}%
\pgfpathlineto{\pgfqpoint{3.334449in}{0.689027in}}%
\pgfpathlineto{\pgfqpoint{3.299618in}{0.680713in}}%
\pgfpathlineto{\pgfqpoint{3.259901in}{0.673978in}}%
\pgfpathlineto{\pgfqpoint{3.215192in}{0.668847in}}%
\pgfpathlineto{\pgfqpoint{3.165042in}{0.665347in}}%
\pgfpathlineto{\pgfqpoint{3.108946in}{0.663515in}}%
\pgfpathlineto{\pgfqpoint{3.046347in}{0.663399in}}%
\pgfpathlineto{\pgfqpoint{2.976640in}{0.665057in}}%
\pgfpathlineto{\pgfqpoint{2.871504in}{0.670143in}}%
\pgfpathlineto{\pgfqpoint{2.750881in}{0.678692in}}%
\pgfpathlineto{\pgfqpoint{2.613274in}{0.690942in}}%
\pgfpathlineto{\pgfqpoint{2.458129in}{0.707163in}}%
\pgfpathlineto{\pgfqpoint{2.287090in}{0.727560in}}%
\pgfpathlineto{\pgfqpoint{2.150940in}{0.745665in}}%
\pgfpathlineto{\pgfqpoint{2.011929in}{0.766130in}}%
\pgfpathlineto{\pgfqpoint{1.874435in}{0.788802in}}%
\pgfpathlineto{\pgfqpoint{1.743282in}{0.813417in}}%
\pgfpathlineto{\pgfqpoint{1.661958in}{0.830744in}}%
\pgfpathlineto{\pgfqpoint{1.586747in}{0.848648in}}%
\pgfpathlineto{\pgfqpoint{1.518124in}{0.866963in}}%
\pgfpathlineto{\pgfqpoint{1.456392in}{0.885538in}}%
\pgfpathlineto{\pgfqpoint{1.401679in}{0.904230in}}%
\pgfpathlineto{\pgfqpoint{1.353939in}{0.922914in}}%
\pgfpathlineto{\pgfqpoint{1.312952in}{0.941473in}}%
\pgfpathlineto{\pgfqpoint{1.278324in}{0.959806in}}%
\pgfpathlineto{\pgfqpoint{1.249487in}{0.977821in}}%
\pgfpathlineto{\pgfqpoint{1.225699in}{0.995442in}}%
\pgfpathlineto{\pgfqpoint{1.206484in}{1.012590in}}%
\pgfpathlineto{\pgfqpoint{1.191637in}{1.029202in}}%
\pgfpathlineto{\pgfqpoint{1.180332in}{1.045244in}}%
\pgfpathlineto{\pgfqpoint{1.171925in}{1.060692in}}%
\pgfpathlineto{\pgfqpoint{1.165959in}{1.075521in}}%
\pgfpathlineto{\pgfqpoint{1.162175in}{1.089713in}}%
\pgfpathlineto{\pgfqpoint{1.160502in}{1.103253in}}%
\pgfpathlineto{\pgfqpoint{1.161064in}{1.116132in}}%
\pgfpathlineto{\pgfqpoint{1.164176in}{1.128341in}}%
\pgfpathlineto{\pgfqpoint{1.170340in}{1.139878in}}%
\pgfpathlineto{\pgfqpoint{1.179127in}{1.150726in}}%
\pgfpathlineto{\pgfqpoint{1.189891in}{1.160872in}}%
\pgfpathlineto{\pgfqpoint{1.202558in}{1.170315in}}%
\pgfpathlineto{\pgfqpoint{1.217085in}{1.179051in}}%
\pgfpathlineto{\pgfqpoint{1.242340in}{1.190832in}}%
\pgfpathlineto{\pgfqpoint{1.271835in}{1.201025in}}%
\pgfpathlineto{\pgfqpoint{1.305799in}{1.209630in}}%
\pgfpathlineto{\pgfqpoint{1.344603in}{1.216654in}}%
\pgfpathlineto{\pgfqpoint{1.388388in}{1.222082in}}%
\pgfpathlineto{\pgfqpoint{1.437492in}{1.225885in}}%
\pgfpathlineto{\pgfqpoint{1.492450in}{1.228030in}}%
\pgfpathlineto{\pgfqpoint{1.553829in}{1.228468in}}%
\pgfpathlineto{\pgfqpoint{1.622231in}{1.227145in}}%
\pgfpathlineto{\pgfqpoint{1.725457in}{1.222522in}}%
\pgfpathlineto{\pgfqpoint{1.843896in}{1.214458in}}%
\pgfpathlineto{\pgfqpoint{1.979132in}{1.202729in}}%
\pgfpathlineto{\pgfqpoint{2.131938in}{1.187064in}}%
\pgfpathlineto{\pgfqpoint{2.301087in}{1.167240in}}%
\pgfpathlineto{\pgfqpoint{2.436225in}{1.149573in}}%
\pgfpathlineto{\pgfqpoint{2.575207in}{1.129522in}}%
\pgfpathlineto{\pgfqpoint{2.713380in}{1.107229in}}%
\pgfpathlineto{\pgfqpoint{2.845933in}{1.082942in}}%
\pgfpathlineto{\pgfqpoint{2.928972in}{1.065810in}}%
\pgfpathlineto{\pgfqpoint{3.006306in}{1.048072in}}%
\pgfpathlineto{\pgfqpoint{3.076483in}{1.029879in}}%
\pgfpathlineto{\pgfqpoint{3.139180in}{1.011384in}}%
\pgfpathlineto{\pgfqpoint{3.194934in}{0.992726in}}%
\pgfpathlineto{\pgfqpoint{3.244224in}{0.974029in}}%
\pgfpathlineto{\pgfqpoint{3.287468in}{0.955410in}}%
\pgfpathlineto{\pgfqpoint{3.325026in}{0.936975in}}%
\pgfpathlineto{\pgfqpoint{3.357200in}{0.918817in}}%
\pgfpathlineto{\pgfqpoint{3.384232in}{0.901020in}}%
\pgfpathlineto{\pgfqpoint{3.406303in}{0.883659in}}%
\pgfpathlineto{\pgfqpoint{3.423538in}{0.866795in}}%
\pgfpathlineto{\pgfqpoint{3.436023in}{0.850481in}}%
\pgfpathlineto{\pgfqpoint{3.444556in}{0.834775in}}%
\pgfpathlineto{\pgfqpoint{3.449963in}{0.819707in}}%
\pgfpathlineto{\pgfqpoint{3.452749in}{0.805295in}}%
\pgfpathlineto{\pgfqpoint{3.453293in}{0.791552in}}%
\pgfpathlineto{\pgfqpoint{3.451849in}{0.778489in}}%
\pgfpathlineto{\pgfqpoint{3.448548in}{0.766111in}}%
\pgfpathlineto{\pgfqpoint{3.443395in}{0.754424in}}%
\pgfpathlineto{\pgfqpoint{3.436271in}{0.743427in}}%
\pgfpathlineto{\pgfqpoint{3.426932in}{0.733117in}}%
\pgfpathlineto{\pgfqpoint{3.415025in}{0.723490in}}%
\pgfpathlineto{\pgfqpoint{3.400875in}{0.714557in}}%
\pgfpathlineto{\pgfqpoint{3.384816in}{0.706328in}}%
\pgfpathlineto{\pgfqpoint{3.357169in}{0.695307in}}%
\pgfpathlineto{\pgfqpoint{3.325195in}{0.685880in}}%
\pgfpathlineto{\pgfqpoint{3.288724in}{0.678054in}}%
\pgfpathlineto{\pgfqpoint{3.247470in}{0.671835in}}%
\pgfpathlineto{\pgfqpoint{3.201032in}{0.667234in}}%
\pgfpathlineto{\pgfqpoint{3.148998in}{0.664261in}}%
\pgfpathlineto{\pgfqpoint{3.091164in}{0.662952in}}%
\pgfpathlineto{\pgfqpoint{3.026606in}{0.663372in}}%
\pgfpathlineto{\pgfqpoint{2.928587in}{0.666742in}}%
\pgfpathlineto{\pgfqpoint{2.815456in}{0.673496in}}%
\pgfpathlineto{\pgfqpoint{2.686094in}{0.683843in}}%
\pgfpathlineto{\pgfqpoint{2.539884in}{0.698013in}}%
\pgfpathlineto{\pgfqpoint{2.376681in}{0.716262in}}%
\pgfpathlineto{\pgfqpoint{2.197350in}{0.738861in}}%
\pgfpathlineto{\pgfqpoint{2.058100in}{0.758611in}}%
\pgfpathlineto{\pgfqpoint{1.920152in}{0.780583in}}%
\pgfpathlineto{\pgfqpoint{1.787725in}{0.804540in}}%
\pgfpathlineto{\pgfqpoint{1.704329in}{0.821464in}}%
\pgfpathlineto{\pgfqpoint{1.625957in}{0.839025in}}%
\pgfpathlineto{\pgfqpoint{1.553431in}{0.857102in}}%
\pgfpathlineto{\pgfqpoint{1.487464in}{0.875557in}}%
\pgfpathlineto{\pgfqpoint{1.428654in}{0.894240in}}%
\pgfpathlineto{\pgfqpoint{1.377448in}{0.912988in}}%
\pgfpathlineto{\pgfqpoint{1.333391in}{0.931665in}}%
\pgfpathlineto{\pgfqpoint{1.295694in}{0.950164in}}%
\pgfpathlineto{\pgfqpoint{1.263741in}{0.968388in}}%
\pgfpathlineto{\pgfqpoint{1.237001in}{0.986252in}}%
\pgfpathlineto{\pgfqpoint{1.215032in}{1.003681in}}%
\pgfpathlineto{\pgfqpoint{1.197479in}{1.020612in}}%
\pgfpathlineto{\pgfqpoint{1.183988in}{1.036992in}}%
\pgfpathlineto{\pgfqpoint{1.173995in}{1.052783in}}%
\pgfpathlineto{\pgfqpoint{1.167101in}{1.067955in}}%
\pgfpathlineto{\pgfqpoint{1.162999in}{1.082487in}}%
\pgfpathlineto{\pgfqpoint{1.161462in}{1.096360in}}%
\pgfpathlineto{\pgfqpoint{1.162342in}{1.109560in}}%
\pgfpathlineto{\pgfqpoint{1.165569in}{1.122079in}}%
\pgfpathlineto{\pgfqpoint{1.171078in}{1.133912in}}%
\pgfpathlineto{\pgfqpoint{1.178705in}{1.145054in}}%
\pgfpathlineto{\pgfqpoint{1.188312in}{1.155500in}}%
\pgfpathlineto{\pgfqpoint{1.199799in}{1.165246in}}%
\pgfpathlineto{\pgfqpoint{1.213107in}{1.174292in}}%
\pgfpathlineto{\pgfqpoint{1.228219in}{1.182636in}}%
\pgfpathlineto{\pgfqpoint{1.254329in}{1.193837in}}%
\pgfpathlineto{\pgfqpoint{1.284806in}{1.203465in}}%
\pgfpathlineto{\pgfqpoint{1.320103in}{1.211532in}}%
\pgfpathlineto{\pgfqpoint{1.360369in}{1.218028in}}%
\pgfpathlineto{\pgfqpoint{1.405702in}{1.222920in}}%
\pgfpathlineto{\pgfqpoint{1.456546in}{1.226179in}}%
\pgfpathlineto{\pgfqpoint{1.513400in}{1.227768in}}%
\pgfpathlineto{\pgfqpoint{1.576821in}{1.227637in}}%
\pgfpathlineto{\pgfqpoint{1.647421in}{1.225727in}}%
\pgfpathlineto{\pgfqpoint{1.753887in}{1.220297in}}%
\pgfpathlineto{\pgfqpoint{1.876011in}{1.211392in}}%
\pgfpathlineto{\pgfqpoint{2.015237in}{1.198758in}}%
\pgfpathlineto{\pgfqpoint{2.171875in}{1.182138in}}%
\pgfpathlineto{\pgfqpoint{2.344115in}{1.161330in}}%
\pgfpathlineto{\pgfqpoint{2.480652in}{1.142922in}}%
\pgfpathlineto{\pgfqpoint{2.619511in}{1.122172in}}%
\pgfpathlineto{\pgfqpoint{2.756301in}{1.099248in}}%
\pgfpathlineto{\pgfqpoint{2.843867in}{1.082889in}}%
\pgfpathlineto{\pgfqpoint{2.926684in}{1.065768in}}%
\pgfpathlineto{\pgfqpoint{3.003772in}{1.048034in}}%
\pgfpathlineto{\pgfqpoint{3.074483in}{1.029850in}}%
\pgfpathlineto{\pgfqpoint{3.138377in}{1.011367in}}%
\pgfpathlineto{\pgfqpoint{3.195220in}{0.992724in}}%
\pgfpathlineto{\pgfqpoint{3.244983in}{0.974051in}}%
\pgfpathlineto{\pgfqpoint{3.287843in}{0.955464in}}%
\pgfpathlineto{\pgfqpoint{3.324185in}{0.937069in}}%
\pgfpathlineto{\pgfqpoint{3.354599in}{0.918962in}}%
\pgfpathlineto{\pgfqpoint{3.379877in}{0.901226in}}%
\pgfpathlineto{\pgfqpoint{3.400290in}{0.883950in}}%
\pgfpathlineto{\pgfqpoint{3.416313in}{0.867194in}}%
\pgfpathlineto{\pgfqpoint{3.428773in}{0.850991in}}%
\pgfpathlineto{\pgfqpoint{3.438297in}{0.835370in}}%
\pgfpathlineto{\pgfqpoint{3.445308in}{0.820358in}}%
\pgfpathlineto{\pgfqpoint{3.450028in}{0.805975in}}%
\pgfpathlineto{\pgfqpoint{3.452476in}{0.792239in}}%
\pgfpathlineto{\pgfqpoint{3.452468in}{0.779161in}}%
\pgfpathlineto{\pgfqpoint{3.449622in}{0.766751in}}%
\pgfpathlineto{\pgfqpoint{3.443626in}{0.755016in}}%
\pgfpathlineto{\pgfqpoint{3.435411in}{0.743977in}}%
\pgfpathlineto{\pgfqpoint{3.425223in}{0.733641in}}%
\pgfpathlineto{\pgfqpoint{3.413141in}{0.724008in}}%
\pgfpathlineto{\pgfqpoint{3.399208in}{0.715080in}}%
\pgfpathlineto{\pgfqpoint{3.383438in}{0.706857in}}%
\pgfpathlineto{\pgfqpoint{3.356282in}{0.695845in}}%
\pgfpathlineto{\pgfqpoint{3.324730in}{0.686416in}}%
\pgfpathlineto{\pgfqpoint{3.288432in}{0.678562in}}%
\pgfpathlineto{\pgfqpoint{3.247286in}{0.672296in}}%
\pgfpathlineto{\pgfqpoint{3.201013in}{0.667641in}}%
\pgfpathlineto{\pgfqpoint{3.149155in}{0.664628in}}%
\pgfpathlineto{\pgfqpoint{3.091200in}{0.663298in}}%
\pgfpathlineto{\pgfqpoint{3.026581in}{0.663699in}}%
\pgfpathlineto{\pgfqpoint{2.928971in}{0.667034in}}%
\pgfpathlineto{\pgfqpoint{2.816773in}{0.673724in}}%
\pgfpathlineto{\pgfqpoint{2.688306in}{0.683979in}}%
\pgfpathlineto{\pgfqpoint{2.542386in}{0.698069in}}%
\pgfpathlineto{\pgfqpoint{2.379420in}{0.716233in}}%
\pgfpathlineto{\pgfqpoint{2.202194in}{0.738641in}}%
\pgfpathlineto{\pgfqpoint{2.063793in}{0.758239in}}%
\pgfpathlineto{\pgfqpoint{1.925176in}{0.780114in}}%
\pgfpathlineto{\pgfqpoint{1.791051in}{0.804043in}}%
\pgfpathlineto{\pgfqpoint{1.706741in}{0.820991in}}%
\pgfpathlineto{\pgfqpoint{1.628022in}{0.838601in}}%
\pgfpathlineto{\pgfqpoint{1.555569in}{0.856705in}}%
\pgfpathlineto{\pgfqpoint{1.489856in}{0.875148in}}%
\pgfpathlineto{\pgfqpoint{1.431154in}{0.893785in}}%
\pgfpathlineto{\pgfqpoint{1.379538in}{0.912485in}}%
\pgfpathlineto{\pgfqpoint{1.334883in}{0.931125in}}%
\pgfpathlineto{\pgfqpoint{1.296867in}{0.949597in}}%
\pgfpathlineto{\pgfqpoint{1.264967in}{0.967803in}}%
\pgfpathlineto{\pgfqpoint{1.238462in}{0.985657in}}%
\pgfpathlineto{\pgfqpoint{1.216723in}{1.003078in}}%
\pgfpathlineto{\pgfqpoint{1.199632in}{1.019989in}}%
\pgfpathlineto{\pgfqpoint{1.186358in}{1.036353in}}%
\pgfpathlineto{\pgfqpoint{1.176200in}{1.052139in}}%
\pgfpathlineto{\pgfqpoint{1.168658in}{1.067322in}}%
\pgfpathlineto{\pgfqpoint{1.163433in}{1.081879in}}%
\pgfpathlineto{\pgfqpoint{1.160423in}{1.095793in}}%
\pgfpathlineto{\pgfqpoint{1.159729in}{1.109051in}}%
\pgfpathlineto{\pgfqpoint{1.161648in}{1.121643in}}%
\pgfpathlineto{\pgfqpoint{1.166672in}{1.133564in}}%
\pgfpathlineto{\pgfqpoint{1.174346in}{1.144796in}}%
\pgfpathlineto{\pgfqpoint{1.184025in}{1.155327in}}%
\pgfpathlineto{\pgfqpoint{1.195622in}{1.165154in}}%
\pgfpathlineto{\pgfqpoint{1.209083in}{1.174277in}}%
\pgfpathlineto{\pgfqpoint{1.224383in}{1.182693in}}%
\pgfpathlineto{\pgfqpoint{1.250810in}{1.193995in}}%
\pgfpathlineto{\pgfqpoint{1.281564in}{1.203712in}}%
\pgfpathlineto{\pgfqpoint{1.316970in}{1.211849in}}%
\pgfpathlineto{\pgfqpoint{1.357255in}{1.218404in}}%
\pgfpathlineto{\pgfqpoint{1.402577in}{1.223352in}}%
\pgfpathlineto{\pgfqpoint{1.453394in}{1.226665in}}%
\pgfpathlineto{\pgfqpoint{1.510218in}{1.228304in}}%
\pgfpathlineto{\pgfqpoint{1.573609in}{1.228220in}}%
\pgfpathlineto{\pgfqpoint{1.644178in}{1.226355in}}%
\pgfpathlineto{\pgfqpoint{1.750586in}{1.220979in}}%
\pgfpathlineto{\pgfqpoint{1.872633in}{1.212123in}}%
\pgfpathlineto{\pgfqpoint{2.011771in}{1.199540in}}%
\pgfpathlineto{\pgfqpoint{2.168401in}{1.182966in}}%
\pgfpathlineto{\pgfqpoint{2.340678in}{1.162201in}}%
\pgfpathlineto{\pgfqpoint{2.477378in}{1.143820in}}%
\pgfpathlineto{\pgfqpoint{2.616462in}{1.123089in}}%
\pgfpathlineto{\pgfqpoint{2.753512in}{1.100177in}}%
\pgfpathlineto{\pgfqpoint{2.841343in}{1.083821in}}%
\pgfpathlineto{\pgfqpoint{2.924498in}{1.066713in}}%
\pgfpathlineto{\pgfqpoint{3.001799in}{1.048975in}}%
\pgfpathlineto{\pgfqpoint{3.072669in}{1.030776in}}%
\pgfpathlineto{\pgfqpoint{3.136725in}{1.012270in}}%
\pgfpathlineto{\pgfqpoint{3.193764in}{0.993601in}}%
\pgfpathlineto{\pgfqpoint{3.243768in}{0.974900in}}%
\pgfpathlineto{\pgfqpoint{3.286903in}{0.956287in}}%
\pgfpathlineto{\pgfqpoint{3.323518in}{0.937867in}}%
\pgfpathlineto{\pgfqpoint{3.354145in}{0.919736in}}%
\pgfpathlineto{\pgfqpoint{3.379498in}{0.901975in}}%
\pgfpathlineto{\pgfqpoint{3.400205in}{0.884662in}}%
\pgfpathlineto{\pgfqpoint{3.416364in}{0.867871in}}%
\pgfpathlineto{\pgfqpoint{3.428788in}{0.851638in}}%
\pgfpathlineto{\pgfqpoint{3.438165in}{0.835990in}}%
\pgfpathlineto{\pgfqpoint{3.444990in}{0.820953in}}%
\pgfpathlineto{\pgfqpoint{3.449563in}{0.806546in}}%
\pgfpathlineto{\pgfqpoint{3.451987in}{0.792786in}}%
\pgfpathlineto{\pgfqpoint{3.452172in}{0.779686in}}%
\pgfpathlineto{\pgfqpoint{3.449832in}{0.767253in}}%
\pgfpathlineto{\pgfqpoint{3.444483in}{0.755491in}}%
\pgfpathlineto{\pgfqpoint{3.436354in}{0.744415in}}%
\pgfpathlineto{\pgfqpoint{3.426228in}{0.734040in}}%
\pgfpathlineto{\pgfqpoint{3.414185in}{0.724369in}}%
\pgfpathlineto{\pgfqpoint{3.400278in}{0.715403in}}%
\pgfpathlineto{\pgfqpoint{3.384528in}{0.707143in}}%
\pgfpathlineto{\pgfqpoint{3.357416in}{0.696077in}}%
\pgfpathlineto{\pgfqpoint{3.325967in}{0.686598in}}%
\pgfpathlineto{\pgfqpoint{3.289866in}{0.678702in}}%
\pgfpathlineto{\pgfqpoint{3.248831in}{0.672389in}}%
\pgfpathlineto{\pgfqpoint{3.202719in}{0.667686in}}%
\pgfpathlineto{\pgfqpoint{3.151044in}{0.664623in}}%
\pgfpathlineto{\pgfqpoint{3.093270in}{0.663238in}}%
\pgfpathlineto{\pgfqpoint{3.028822in}{0.663584in}}%
\pgfpathlineto{\pgfqpoint{2.931437in}{0.666843in}}%
\pgfpathlineto{\pgfqpoint{2.819514in}{0.673460in}}%
\pgfpathlineto{\pgfqpoint{2.691384in}{0.683641in}}%
\pgfpathlineto{\pgfqpoint{2.545844in}{0.697646in}}%
\pgfpathlineto{\pgfqpoint{2.383138in}{0.715728in}}%
\pgfpathlineto{\pgfqpoint{2.206185in}{0.738048in}}%
\pgfpathlineto{\pgfqpoint{2.067709in}{0.757584in}}%
\pgfpathlineto{\pgfqpoint{1.929018in}{0.779403in}}%
\pgfpathlineto{\pgfqpoint{1.794780in}{0.803281in}}%
\pgfpathlineto{\pgfqpoint{1.710086in}{0.820187in}}%
\pgfpathlineto{\pgfqpoint{1.630997in}{0.837745in}}%
\pgfpathlineto{\pgfqpoint{1.558967in}{0.855823in}}%
\pgfpathlineto{\pgfqpoint{1.493941in}{0.874258in}}%
\pgfpathlineto{\pgfqpoint{1.435691in}{0.892900in}}%
\pgfpathlineto{\pgfqpoint{1.383978in}{0.911610in}}%
\pgfpathlineto{\pgfqpoint{1.338545in}{0.930265in}}%
\pgfpathlineto{\pgfqpoint{1.299125in}{0.948752in}}%
\pgfpathlineto{\pgfqpoint{1.265434in}{0.966975in}}%
\pgfpathlineto{\pgfqpoint{1.237175in}{0.984850in}}%
\pgfpathlineto{\pgfqpoint{1.214037in}{1.002306in}}%
\pgfpathlineto{\pgfqpoint{1.195695in}{1.019285in}}%
\pgfpathlineto{\pgfqpoint{1.181808in}{1.035745in}}%
\pgfpathlineto{\pgfqpoint{1.171994in}{1.051633in}}%
\pgfpathlineto{\pgfqpoint{1.165625in}{1.066891in}}%
\pgfpathlineto{\pgfqpoint{1.162070in}{1.081502in}}%
\pgfpathlineto{\pgfqpoint{1.160834in}{1.095450in}}%
\pgfpathlineto{\pgfqpoint{1.161565in}{1.108726in}}%
\pgfpathlineto{\pgfqpoint{1.164050in}{1.121322in}}%
\pgfpathlineto{\pgfqpoint{1.168216in}{1.133232in}}%
\pgfpathlineto{\pgfqpoint{1.174130in}{1.144455in}}%
\pgfpathlineto{\pgfqpoint{1.182001in}{1.154994in}}%
\pgfpathlineto{\pgfqpoint{1.192177in}{1.164852in}}%
\pgfpathlineto{\pgfqpoint{1.205140in}{1.174039in}}%
\pgfpathlineto{\pgfqpoint{1.220531in}{1.182537in}}%
\pgfpathlineto{\pgfqpoint{1.247199in}{1.193961in}}%
\pgfpathlineto{\pgfqpoint{1.278193in}{1.203793in}}%
\pgfpathlineto{\pgfqpoint{1.313644in}{1.212026in}}%
\pgfpathlineto{\pgfqpoint{1.353794in}{1.218650in}}%
\pgfpathlineto{\pgfqpoint{1.398992in}{1.223653in}}%
\pgfpathlineto{\pgfqpoint{1.449693in}{1.227022in}}%
\pgfpathlineto{\pgfqpoint{1.506180in}{1.228740in}}%
\pgfpathlineto{\pgfqpoint{1.568930in}{1.228756in}}%
\pgfpathlineto{\pgfqpoint{1.639044in}{1.226991in}}%
\pgfpathlineto{\pgfqpoint{1.745469in}{1.221726in}}%
\pgfpathlineto{\pgfqpoint{1.867955in}{1.212948in}}%
\pgfpathlineto{\pgfqpoint{2.007196in}{1.200441in}}%
\pgfpathlineto{\pgfqpoint{2.163148in}{1.183979in}}%
\pgfpathlineto{\pgfqpoint{2.335083in}{1.163317in}}%
\pgfpathlineto{\pgfqpoint{2.472862in}{1.144916in}}%
\pgfpathlineto{\pgfqpoint{2.612766in}{1.124154in}}%
\pgfpathlineto{\pgfqpoint{2.749660in}{1.101262in}}%
\pgfpathlineto{\pgfqpoint{2.879202in}{1.076529in}}%
\pgfpathlineto{\pgfqpoint{2.959697in}{1.059183in}}%
\pgfpathlineto{\pgfqpoint{3.034482in}{1.041285in}}%
\pgfpathlineto{\pgfqpoint{3.102860in}{1.022960in}}%
\pgfpathlineto{\pgfqpoint{3.164295in}{1.004345in}}%
\pgfpathlineto{\pgfqpoint{3.218405in}{0.985586in}}%
\pgfpathlineto{\pgfqpoint{3.265012in}{0.966843in}}%
\pgfpathlineto{\pgfqpoint{3.304709in}{0.948251in}}%
\pgfpathlineto{\pgfqpoint{3.338406in}{0.929907in}}%
\pgfpathlineto{\pgfqpoint{3.366840in}{0.911893in}}%
\pgfpathlineto{\pgfqpoint{3.390586in}{0.894285in}}%
\pgfpathlineto{\pgfqpoint{3.410057in}{0.877146in}}%
\pgfpathlineto{\pgfqpoint{3.425503in}{0.860534in}}%
\pgfpathlineto{\pgfqpoint{3.437011in}{0.844495in}}%
\pgfpathlineto{\pgfqpoint{3.444844in}{0.829068in}}%
\pgfpathlineto{\pgfqpoint{3.449709in}{0.814280in}}%
\pgfpathlineto{\pgfqpoint{3.451923in}{0.800148in}}%
\pgfpathlineto{\pgfqpoint{3.451728in}{0.786687in}}%
\pgfpathlineto{\pgfqpoint{3.449296in}{0.773908in}}%
\pgfpathlineto{\pgfqpoint{3.444727in}{0.761817in}}%
\pgfpathlineto{\pgfqpoint{3.438051in}{0.750418in}}%
\pgfpathlineto{\pgfqpoint{3.429235in}{0.739713in}}%
\pgfpathlineto{\pgfqpoint{3.418352in}{0.729703in}}%
\pgfpathlineto{\pgfqpoint{3.405515in}{0.720390in}}%
\pgfpathlineto{\pgfqpoint{3.390784in}{0.711777in}}%
\pgfpathlineto{\pgfqpoint{3.365195in}{0.700174in}}%
\pgfpathlineto{\pgfqpoint{3.335386in}{0.690156in}}%
\pgfpathlineto{\pgfqpoint{3.301191in}{0.681728in}}%
\pgfpathlineto{\pgfqpoint{3.262299in}{0.674890in}}%
\pgfpathlineto{\pgfqpoint{3.218250in}{0.669643in}}%
\pgfpathlineto{\pgfqpoint{3.168765in}{0.666000in}}%
\pgfpathlineto{\pgfqpoint{3.113536in}{0.664010in}}%
\pgfpathlineto{\pgfqpoint{3.051810in}{0.663723in}}%
\pgfpathlineto{\pgfqpoint{2.982885in}{0.665198in}}%
\pgfpathlineto{\pgfqpoint{2.878670in}{0.670030in}}%
\pgfpathlineto{\pgfqpoint{2.759113in}{0.678317in}}%
\pgfpathlineto{\pgfqpoint{2.622985in}{0.690287in}}%
\pgfpathlineto{\pgfqpoint{2.469096in}{0.706187in}}%
\pgfpathlineto{\pgfqpoint{2.298256in}{0.726233in}}%
\pgfpathlineto{\pgfqpoint{2.162807in}{0.744096in}}%
\pgfpathlineto{\pgfqpoint{2.024778in}{0.764343in}}%
\pgfpathlineto{\pgfqpoint{1.887900in}{0.786822in}}%
\pgfpathlineto{\pgfqpoint{1.756280in}{0.811274in}}%
\pgfpathlineto{\pgfqpoint{1.673651in}{0.828496in}}%
\pgfpathlineto{\pgfqpoint{1.596764in}{0.846300in}}%
\pgfpathlineto{\pgfqpoint{1.527027in}{0.864533in}}%
\pgfpathlineto{\pgfqpoint{1.464773in}{0.883055in}}%
\pgfpathlineto{\pgfqpoint{1.409619in}{0.901727in}}%
\pgfpathlineto{\pgfqpoint{1.361186in}{0.920418in}}%
\pgfpathlineto{\pgfqpoint{1.319102in}{0.939011in}}%
\pgfpathlineto{\pgfqpoint{1.283005in}{0.957399in}}%
\pgfpathlineto{\pgfqpoint{1.252542in}{0.975486in}}%
\pgfpathlineto{\pgfqpoint{1.227365in}{0.993189in}}%
\pgfpathlineto{\pgfqpoint{1.207137in}{1.010433in}}%
\pgfpathlineto{\pgfqpoint{1.191457in}{1.027155in}}%
\pgfpathlineto{\pgfqpoint{1.179641in}{1.043305in}}%
\pgfpathlineto{\pgfqpoint{1.171091in}{1.058854in}}%
\pgfpathlineto{\pgfqpoint{1.165346in}{1.073776in}}%
\pgfpathlineto{\pgfqpoint{1.162078in}{1.088053in}}%
\pgfpathlineto{\pgfqpoint{1.161091in}{1.101669in}}%
\pgfpathlineto{\pgfqpoint{1.162320in}{1.114612in}}%
\pgfpathlineto{\pgfqpoint{1.165832in}{1.126876in}}%
\pgfpathlineto{\pgfqpoint{1.171827in}{1.138458in}}%
\pgfpathlineto{\pgfqpoint{1.180252in}{1.149352in}}%
\pgfpathlineto{\pgfqpoint{1.190687in}{1.159548in}}%
\pgfpathlineto{\pgfqpoint{1.203045in}{1.169045in}}%
\pgfpathlineto{\pgfqpoint{1.217274in}{1.177839in}}%
\pgfpathlineto{\pgfqpoint{1.233352in}{1.185930in}}%
\pgfpathlineto{\pgfqpoint{1.260962in}{1.196745in}}%
\pgfpathlineto{\pgfqpoint{1.292913in}{1.205977in}}%
\pgfpathlineto{\pgfqpoint{1.329516in}{1.213627in}}%
\pgfpathlineto{\pgfqpoint{1.371151in}{1.219697in}}%
\pgfpathlineto{\pgfqpoint{1.417914in}{1.224161in}}%
\pgfpathlineto{\pgfqpoint{1.470292in}{1.226985in}}%
\pgfpathlineto{\pgfqpoint{1.528859in}{1.228127in}}%
\pgfpathlineto{\pgfqpoint{1.594213in}{1.227536in}}%
\pgfpathlineto{\pgfqpoint{1.692982in}{1.223939in}}%
\pgfpathlineto{\pgfqpoint{1.806450in}{1.216965in}}%
\pgfpathlineto{\pgfqpoint{1.936215in}{1.206400in}}%
\pgfpathlineto{\pgfqpoint{2.083408in}{1.191996in}}%
\pgfpathlineto{\pgfqpoint{2.247647in}{1.173488in}}%
\pgfpathlineto{\pgfqpoint{2.425607in}{1.150738in}}%
\pgfpathlineto{\pgfqpoint{2.564190in}{1.130897in}}%
\pgfpathlineto{\pgfqpoint{2.702624in}{1.108783in}}%
\pgfpathlineto{\pgfqpoint{2.835686in}{1.084661in}}%
\pgfpathlineto{\pgfqpoint{2.919156in}{1.067632in}}%
\pgfpathlineto{\pgfqpoint{2.997132in}{1.049982in}}%
\pgfpathlineto{\pgfqpoint{3.068713in}{1.031846in}}%
\pgfpathlineto{\pgfqpoint{3.133160in}{1.013372in}}%
\pgfpathlineto{\pgfqpoint{3.190041in}{0.994720in}}%
\pgfpathlineto{\pgfqpoint{3.239871in}{0.976021in}}%
\pgfpathlineto{\pgfqpoint{3.283239in}{0.957393in}}%
\pgfpathlineto{\pgfqpoint{3.320653in}{0.938944in}}%
\pgfpathlineto{\pgfqpoint{3.352544in}{0.920769in}}%
\pgfpathlineto{\pgfqpoint{3.379263in}{0.902955in}}%
\pgfpathlineto{\pgfqpoint{3.401083in}{0.885578in}}%
\pgfpathlineto{\pgfqpoint{3.418196in}{0.868704in}}%
\pgfpathlineto{\pgfqpoint{3.431008in}{0.852389in}}%
\pgfpathlineto{\pgfqpoint{3.440324in}{0.836673in}}%
\pgfpathlineto{\pgfqpoint{3.446629in}{0.821580in}}%
\pgfpathlineto{\pgfqpoint{3.450295in}{0.807131in}}%
\pgfpathlineto{\pgfqpoint{3.451583in}{0.793342in}}%
\pgfpathlineto{\pgfqpoint{3.450641in}{0.780227in}}%
\pgfpathlineto{\pgfqpoint{3.447510in}{0.767793in}}%
\pgfpathlineto{\pgfqpoint{3.442117in}{0.756045in}}%
\pgfpathlineto{\pgfqpoint{3.434323in}{0.744984in}}%
\pgfpathlineto{\pgfqpoint{3.424400in}{0.734616in}}%
\pgfpathlineto{\pgfqpoint{3.412521in}{0.724948in}}%
\pgfpathlineto{\pgfqpoint{3.398749in}{0.715980in}}%
\pgfpathlineto{\pgfqpoint{3.383119in}{0.707715in}}%
\pgfpathlineto{\pgfqpoint{3.356194in}{0.696637in}}%
\pgfpathlineto{\pgfqpoint{3.324982in}{0.687144in}}%
\pgfpathlineto{\pgfqpoint{3.289222in}{0.679236in}}%
\pgfpathlineto{\pgfqpoint{3.248501in}{0.672910in}}%
\pgfpathlineto{\pgfqpoint{3.202614in}{0.668179in}}%
\pgfpathlineto{\pgfqpoint{3.151253in}{0.665079in}}%
\pgfpathlineto{\pgfqpoint{3.093803in}{0.663652in}}%
\pgfpathlineto{\pgfqpoint{3.029645in}{0.663949in}}%
\pgfpathlineto{\pgfqpoint{2.932590in}{0.667134in}}%
\pgfpathlineto{\pgfqpoint{2.821008in}{0.673671in}}%
\pgfpathlineto{\pgfqpoint{2.693389in}{0.683773in}}%
\pgfpathlineto{\pgfqpoint{2.548408in}{0.697678in}}%
\pgfpathlineto{\pgfqpoint{2.386224in}{0.715654in}}%
\pgfpathlineto{\pgfqpoint{2.209687in}{0.737881in}}%
\pgfpathlineto{\pgfqpoint{2.071563in}{0.757330in}}%
\pgfpathlineto{\pgfqpoint{1.932876in}{0.779061in}}%
\pgfpathlineto{\pgfqpoint{1.798486in}{0.802893in}}%
\pgfpathlineto{\pgfqpoint{1.713755in}{0.819765in}}%
\pgfpathlineto{\pgfqpoint{1.634332in}{0.837274in}}%
\pgfpathlineto{\pgfqpoint{1.561118in}{0.855286in}}%
\pgfpathlineto{\pgfqpoint{1.494735in}{0.873670in}}%
\pgfpathlineto{\pgfqpoint{1.435525in}{0.892289in}}%
\pgfpathlineto{\pgfqpoint{1.383551in}{0.911006in}}%
\pgfpathlineto{\pgfqpoint{1.338750in}{0.929673in}}%
\pgfpathlineto{\pgfqpoint{1.300902in}{0.948146in}}%
\pgfpathlineto{\pgfqpoint{1.268858in}{0.966348in}}%
\pgfpathlineto{\pgfqpoint{1.241647in}{0.984211in}}%
\pgfpathlineto{\pgfqpoint{1.218556in}{1.001674in}}%
\pgfpathlineto{\pgfqpoint{1.199135in}{1.018678in}}%
\pgfpathlineto{\pgfqpoint{1.183196in}{1.035171in}}%
\pgfpathlineto{\pgfqpoint{1.170812in}{1.051106in}}%
\pgfpathlineto{\pgfqpoint{1.162318in}{1.066439in}}%
\pgfpathlineto{\pgfqpoint{1.158098in}{1.081132in}}%
\pgfpathlineto{\pgfqpoint{1.156860in}{1.095155in}}%
\pgfpathlineto{\pgfqpoint{1.158030in}{1.108496in}}%
\pgfpathlineto{\pgfqpoint{1.161401in}{1.121147in}}%
\pgfpathlineto{\pgfqpoint{1.166825in}{1.133100in}}%
\pgfpathlineto{\pgfqpoint{1.174207in}{1.144354in}}%
\pgfpathlineto{\pgfqpoint{1.183515in}{1.154905in}}%
\pgfpathlineto{\pgfqpoint{1.194770in}{1.164755in}}%
\pgfpathlineto{\pgfqpoint{1.208051in}{1.173906in}}%
\pgfpathlineto{\pgfqpoint{1.223362in}{1.182359in}}%
\pgfpathlineto{\pgfqpoint{1.249942in}{1.193726in}}%
\pgfpathlineto{\pgfqpoint{1.280865in}{1.203508in}}%
\pgfpathlineto{\pgfqpoint{1.316246in}{1.211697in}}%
\pgfpathlineto{\pgfqpoint{1.356310in}{1.218279in}}%
\pgfpathlineto{\pgfqpoint{1.401396in}{1.223242in}}%
\pgfpathlineto{\pgfqpoint{1.451955in}{1.226568in}}%
\pgfpathlineto{\pgfqpoint{1.508540in}{1.228242in}}%
\pgfpathlineto{\pgfqpoint{1.571264in}{1.228238in}}%
\pgfpathlineto{\pgfqpoint{1.641059in}{1.226472in}}%
\pgfpathlineto{\pgfqpoint{1.747113in}{1.221206in}}%
\pgfpathlineto{\pgfqpoint{1.869605in}{1.212405in}}%
\pgfpathlineto{\pgfqpoint{2.009186in}{1.199855in}}%
\pgfpathlineto{\pgfqpoint{2.165363in}{1.183355in}}%
\pgfpathlineto{\pgfqpoint{2.336492in}{1.162715in}}%
\pgfpathlineto{\pgfqpoint{2.472944in}{1.144408in}}%
\pgfpathlineto{\pgfqpoint{2.612535in}{1.123690in}}%
\pgfpathlineto{\pgfqpoint{2.749704in}{1.100805in}}%
\pgfpathlineto{\pgfqpoint{2.879676in}{1.076063in}}%
\pgfpathlineto{\pgfqpoint{2.960391in}{1.058713in}}%
\pgfpathlineto{\pgfqpoint{3.035273in}{1.040813in}}%
\pgfpathlineto{\pgfqpoint{3.103597in}{1.022494in}}%
\pgfpathlineto{\pgfqpoint{3.164823in}{1.003891in}}%
\pgfpathlineto{\pgfqpoint{3.218595in}{0.985153in}}%
\pgfpathlineto{\pgfqpoint{3.264865in}{0.966439in}}%
\pgfpathlineto{\pgfqpoint{3.304401in}{0.947877in}}%
\pgfpathlineto{\pgfqpoint{3.338099in}{0.929556in}}%
\pgfpathlineto{\pgfqpoint{3.366673in}{0.911561in}}%
\pgfpathlineto{\pgfqpoint{3.390657in}{0.893966in}}%
\pgfpathlineto{\pgfqpoint{3.410403in}{0.876836in}}%
\pgfpathlineto{\pgfqpoint{3.426084in}{0.860229in}}%
\pgfpathlineto{\pgfqpoint{3.437690in}{0.844193in}}%
\pgfpathlineto{\pgfqpoint{3.445428in}{0.828769in}}%
\pgfpathlineto{\pgfqpoint{3.450172in}{0.813986in}}%
\pgfpathlineto{\pgfqpoint{3.452275in}{0.799861in}}%
\pgfpathlineto{\pgfqpoint{3.451993in}{0.786408in}}%
\pgfpathlineto{\pgfqpoint{3.449509in}{0.773636in}}%
\pgfpathlineto{\pgfqpoint{3.444931in}{0.761554in}}%
\pgfpathlineto{\pgfqpoint{3.438290in}{0.750166in}}%
\pgfpathlineto{\pgfqpoint{3.429543in}{0.739471in}}%
\pgfpathlineto{\pgfqpoint{3.418647in}{0.729470in}}%
\pgfpathlineto{\pgfqpoint{3.405768in}{0.720166in}}%
\pgfpathlineto{\pgfqpoint{3.390976in}{0.711561in}}%
\pgfpathlineto{\pgfqpoint{3.365260in}{0.699969in}}%
\pgfpathlineto{\pgfqpoint{3.335294in}{0.689963in}}%
\pgfpathlineto{\pgfqpoint{3.300936in}{0.681548in}}%
\pgfpathlineto{\pgfqpoint{3.261909in}{0.674728in}}%
\pgfpathlineto{\pgfqpoint{3.217800in}{0.669507in}}%
\pgfpathlineto{\pgfqpoint{3.168185in}{0.665893in}}%
\pgfpathlineto{\pgfqpoint{3.112895in}{0.663925in}}%
\pgfpathlineto{\pgfqpoint{3.051140in}{0.663660in}}%
\pgfpathlineto{\pgfqpoint{2.982141in}{0.665159in}}%
\pgfpathlineto{\pgfqpoint{2.877700in}{0.670029in}}%
\pgfpathlineto{\pgfqpoint{2.757805in}{0.678358in}}%
\pgfpathlineto{\pgfqpoint{2.621384in}{0.690374in}}%
\pgfpathlineto{\pgfqpoint{2.467616in}{0.706330in}}%
\pgfpathlineto{\pgfqpoint{2.293902in}{0.726660in}}%
\pgfpathlineto{\pgfqpoint{2.155540in}{0.744786in}}%
\pgfpathlineto{\pgfqpoint{2.016204in}{0.765208in}}%
\pgfpathlineto{\pgfqpoint{1.880440in}{0.787716in}}%
\pgfpathlineto{\pgfqpoint{1.752042in}{0.812054in}}%
\pgfpathlineto{\pgfqpoint{1.672072in}{0.829145in}}%
\pgfpathlineto{\pgfqpoint{1.597462in}{0.846809in}}%
\pgfpathlineto{\pgfqpoint{1.528783in}{0.864934in}}%
\pgfpathlineto{\pgfqpoint{1.466458in}{0.883396in}}%
\pgfpathlineto{\pgfqpoint{1.410759in}{0.902062in}}%
\pgfpathlineto{\pgfqpoint{1.361813in}{0.920791in}}%
\pgfpathlineto{\pgfqpoint{1.319599in}{0.939428in}}%
\pgfpathlineto{\pgfqpoint{1.283944in}{0.957814in}}%
\pgfpathlineto{\pgfqpoint{1.254209in}{0.975853in}}%
\pgfpathlineto{\pgfqpoint{1.229589in}{0.993499in}}%
\pgfpathlineto{\pgfqpoint{1.209504in}{1.010688in}}%
\pgfpathlineto{\pgfqpoint{1.193480in}{1.027362in}}%
\pgfpathlineto{\pgfqpoint{1.181146in}{1.043476in}}%
\pgfpathlineto{\pgfqpoint{1.172231in}{1.058993in}}%
\pgfpathlineto{\pgfqpoint{1.166404in}{1.073884in}}%
\pgfpathlineto{\pgfqpoint{1.163337in}{1.088128in}}%
\pgfpathlineto{\pgfqpoint{1.162799in}{1.101710in}}%
\pgfpathlineto{\pgfqpoint{1.164614in}{1.114618in}}%
\pgfpathlineto{\pgfqpoint{1.168655in}{1.126842in}}%
\pgfpathlineto{\pgfqpoint{1.174853in}{1.138378in}}%
\pgfpathlineto{\pgfqpoint{1.183187in}{1.149226in}}%
\pgfpathlineto{\pgfqpoint{1.193549in}{1.159381in}}%
\pgfpathlineto{\pgfqpoint{1.205833in}{1.168840in}}%
\pgfpathlineto{\pgfqpoint{1.219972in}{1.177601in}}%
\pgfpathlineto{\pgfqpoint{1.244591in}{1.189426in}}%
\pgfpathlineto{\pgfqpoint{1.273352in}{1.199671in}}%
\pgfpathlineto{\pgfqpoint{1.306470in}{1.208335in}}%
\pgfpathlineto{\pgfqpoint{1.344344in}{1.215425in}}%
\pgfpathlineto{\pgfqpoint{1.387491in}{1.220946in}}%
\pgfpathlineto{\pgfqpoint{1.435944in}{1.224861in}}%
\pgfpathlineto{\pgfqpoint{1.490159in}{1.227130in}}%
\pgfpathlineto{\pgfqpoint{1.550762in}{1.227705in}}%
\pgfpathlineto{\pgfqpoint{1.618380in}{1.226531in}}%
\pgfpathlineto{\pgfqpoint{1.720534in}{1.222124in}}%
\pgfpathlineto{\pgfqpoint{1.837768in}{1.214295in}}%
\pgfpathlineto{\pgfqpoint{1.971575in}{1.202819in}}%
\pgfpathlineto{\pgfqpoint{2.122903in}{1.187450in}}%
\pgfpathlineto{\pgfqpoint{2.290737in}{1.167926in}}%
\pgfpathlineto{\pgfqpoint{2.425099in}{1.150482in}}%
\pgfpathlineto{\pgfqpoint{2.563456in}{1.130674in}}%
\pgfpathlineto{\pgfqpoint{2.701650in}{1.108615in}}%
\pgfpathlineto{\pgfqpoint{2.834728in}{1.084498in}}%
\pgfpathlineto{\pgfqpoint{2.918206in}{1.067472in}}%
\pgfpathlineto{\pgfqpoint{2.996151in}{1.049843in}}%
\pgfpathlineto{\pgfqpoint{3.067728in}{1.031745in}}%
\pgfpathlineto{\pgfqpoint{3.132384in}{1.013311in}}%
\pgfpathlineto{\pgfqpoint{3.189844in}{0.994676in}}%
\pgfpathlineto{\pgfqpoint{3.240114in}{0.975975in}}%
\pgfpathlineto{\pgfqpoint{3.283371in}{0.957347in}}%
\pgfpathlineto{\pgfqpoint{3.319776in}{0.938938in}}%
\pgfpathlineto{\pgfqpoint{3.350435in}{0.920826in}}%
\pgfpathlineto{\pgfqpoint{3.376345in}{0.903073in}}%
\pgfpathlineto{\pgfqpoint{3.398236in}{0.885738in}}%
\pgfpathlineto{\pgfqpoint{3.416579in}{0.868876in}}%
\pgfpathlineto{\pgfqpoint{3.431580in}{0.852535in}}%
\pgfpathlineto{\pgfqpoint{3.443182in}{0.836761in}}%
\pgfpathlineto{\pgfqpoint{3.451066in}{0.821593in}}%
\pgfpathlineto{\pgfqpoint{3.454769in}{0.807066in}}%
\pgfpathlineto{\pgfqpoint{3.455427in}{0.793211in}}%
\pgfpathlineto{\pgfqpoint{3.453706in}{0.780040in}}%
\pgfpathlineto{\pgfqpoint{3.449806in}{0.767560in}}%
\pgfpathlineto{\pgfqpoint{3.443873in}{0.755779in}}%
\pgfpathlineto{\pgfqpoint{3.435996in}{0.744698in}}%
\pgfpathlineto{\pgfqpoint{3.426213in}{0.734319in}}%
\pgfpathlineto{\pgfqpoint{3.414504in}{0.724643in}}%
\pgfpathlineto{\pgfqpoint{3.400797in}{0.715665in}}%
\pgfpathlineto{\pgfqpoint{3.385044in}{0.707385in}}%
\pgfpathlineto{\pgfqpoint{3.357782in}{0.696276in}}%
\pgfpathlineto{\pgfqpoint{3.326145in}{0.686750in}}%
\pgfpathlineto{\pgfqpoint{3.290004in}{0.678819in}}%
\pgfpathlineto{\pgfqpoint{3.249126in}{0.672494in}}%
\pgfpathlineto{\pgfqpoint{3.203167in}{0.667794in}}%
\pgfpathlineto{\pgfqpoint{3.151675in}{0.664736in}}%
\pgfpathlineto{\pgfqpoint{3.094090in}{0.663342in}}%
\pgfpathlineto{\pgfqpoint{3.030202in}{0.663629in}}%
\pgfpathlineto{\pgfqpoint{2.933840in}{0.666768in}}%
\pgfpathlineto{\pgfqpoint{2.822028in}{0.673304in}}%
\pgfpathlineto{\pgfqpoint{2.693157in}{0.683477in}}%
\pgfpathlineto{\pgfqpoint{2.547018in}{0.697497in}}%
\pgfpathlineto{\pgfqpoint{2.384798in}{0.715543in}}%
\pgfpathlineto{\pgfqpoint{2.209084in}{0.737761in}}%
\pgfpathlineto{\pgfqpoint{2.070917in}{0.757236in}}%
\pgfpathlineto{\pgfqpoint{1.931896in}{0.779049in}}%
\pgfpathlineto{\pgfqpoint{1.797582in}{0.802902in}}%
\pgfpathlineto{\pgfqpoint{1.712940in}{0.819765in}}%
\pgfpathlineto{\pgfqpoint{1.633512in}{0.837264in}}%
\pgfpathlineto{\pgfqpoint{1.560171in}{0.855277in}}%
\pgfpathlineto{\pgfqpoint{1.493579in}{0.873674in}}%
\pgfpathlineto{\pgfqpoint{1.434187in}{0.892316in}}%
\pgfpathlineto{\pgfqpoint{1.382238in}{0.911058in}}%
\pgfpathlineto{\pgfqpoint{1.337678in}{0.929740in}}%
\pgfpathlineto{\pgfqpoint{1.299671in}{0.948246in}}%
\pgfpathlineto{\pgfqpoint{1.267312in}{0.966488in}}%
\pgfpathlineto{\pgfqpoint{1.239893in}{0.984388in}}%
\pgfpathlineto{\pgfqpoint{1.216899in}{1.001873in}}%
\pgfpathlineto{\pgfqpoint{1.198011in}{1.018882in}}%
\pgfpathlineto{\pgfqpoint{1.183104in}{1.035359in}}%
\pgfpathlineto{\pgfqpoint{1.172247in}{1.051256in}}%
\pgfpathlineto{\pgfqpoint{1.165269in}{1.066535in}}%
\pgfpathlineto{\pgfqpoint{1.161227in}{1.081167in}}%
\pgfpathlineto{\pgfqpoint{1.159774in}{1.095136in}}%
\pgfpathlineto{\pgfqpoint{1.160659in}{1.108429in}}%
\pgfpathlineto{\pgfqpoint{1.163706in}{1.121038in}}%
\pgfpathlineto{\pgfqpoint{1.168810in}{1.132956in}}%
\pgfpathlineto{\pgfqpoint{1.175941in}{1.144179in}}%
\pgfpathlineto{\pgfqpoint{1.185143in}{1.154707in}}%
\pgfpathlineto{\pgfqpoint{1.196492in}{1.164541in}}%
\pgfpathlineto{\pgfqpoint{1.209837in}{1.173679in}}%
\pgfpathlineto{\pgfqpoint{1.225100in}{1.182117in}}%
\pgfpathlineto{\pgfqpoint{1.251540in}{1.193459in}}%
\pgfpathlineto{\pgfqpoint{1.282262in}{1.203214in}}%
\pgfpathlineto{\pgfqpoint{1.317413in}{1.211377in}}%
\pgfpathlineto{\pgfqpoint{1.357269in}{1.217941in}}%
\pgfpathlineto{\pgfqpoint{1.402236in}{1.222900in}}%
\pgfpathlineto{\pgfqpoint{1.452783in}{1.226245in}}%
\pgfpathlineto{\pgfqpoint{1.509049in}{1.227942in}}%
\pgfpathlineto{\pgfqpoint{1.571847in}{1.227929in}}%
\pgfpathlineto{\pgfqpoint{1.642030in}{1.226140in}}%
\pgfpathlineto{\pgfqpoint{1.748316in}{1.220860in}}%
\pgfpathlineto{\pgfqpoint{1.870317in}{1.212090in}}%
\pgfpathlineto{\pgfqpoint{2.008932in}{1.199606in}}%
\pgfpathlineto{\pgfqpoint{2.164630in}{1.183158in}}%
\pgfpathlineto{\pgfqpoint{2.337887in}{1.162457in}}%
\pgfpathlineto{\pgfqpoint{2.475484in}{1.144090in}}%
\pgfpathlineto{\pgfqpoint{2.614232in}{1.123407in}}%
\pgfpathlineto{\pgfqpoint{2.749788in}{1.100596in}}%
\pgfpathlineto{\pgfqpoint{2.878316in}{1.075920in}}%
\pgfpathlineto{\pgfqpoint{2.958429in}{1.058597in}}%
\pgfpathlineto{\pgfqpoint{3.033063in}{1.040710in}}%
\pgfpathlineto{\pgfqpoint{3.101454in}{1.022394in}}%
\pgfpathlineto{\pgfqpoint{3.162935in}{1.003794in}}%
\pgfpathlineto{\pgfqpoint{3.216944in}{0.985074in}}%
\pgfpathlineto{\pgfqpoint{3.263478in}{0.966385in}}%
\pgfpathlineto{\pgfqpoint{3.303419in}{0.947835in}}%
\pgfpathlineto{\pgfqpoint{3.337384in}{0.929524in}}%
\pgfpathlineto{\pgfqpoint{3.365919in}{0.911544in}}%
\pgfpathlineto{\pgfqpoint{3.389499in}{0.893974in}}%
\pgfpathlineto{\pgfqpoint{3.408529in}{0.876881in}}%
\pgfpathlineto{\pgfqpoint{3.423398in}{0.860320in}}%
\pgfpathlineto{\pgfqpoint{3.434638in}{0.844334in}}%
\pgfpathlineto{\pgfqpoint{3.442657in}{0.828955in}}%
\pgfpathlineto{\pgfqpoint{3.447777in}{0.814208in}}%
\pgfpathlineto{\pgfqpoint{3.450238in}{0.800112in}}%
\pgfpathlineto{\pgfqpoint{3.450203in}{0.786684in}}%
\pgfpathlineto{\pgfqpoint{3.447755in}{0.773933in}}%
\pgfpathlineto{\pgfqpoint{3.442992in}{0.761866in}}%
\pgfpathlineto{\pgfqpoint{3.436100in}{0.750488in}}%
\pgfpathlineto{\pgfqpoint{3.427228in}{0.739805in}}%
\pgfpathlineto{\pgfqpoint{3.416483in}{0.729820in}}%
\pgfpathlineto{\pgfqpoint{3.403925in}{0.720535in}}%
\pgfpathlineto{\pgfqpoint{3.389569in}{0.711951in}}%
\pgfpathlineto{\pgfqpoint{3.364591in}{0.700387in}}%
\pgfpathlineto{\pgfqpoint{3.335206in}{0.690389in}}%
\pgfpathlineto{\pgfqpoint{3.300940in}{0.681944in}}%
\pgfpathlineto{\pgfqpoint{3.261836in}{0.675073in}}%
\pgfpathlineto{\pgfqpoint{3.217728in}{0.669802in}}%
\pgfpathlineto{\pgfqpoint{3.168217in}{0.666158in}}%
\pgfpathlineto{\pgfqpoint{3.112833in}{0.664176in}}%
\pgfpathlineto{\pgfqpoint{3.051042in}{0.663903in}}%
\pgfpathlineto{\pgfqpoint{2.982237in}{0.665392in}}%
\pgfpathlineto{\pgfqpoint{2.878413in}{0.670228in}}%
\pgfpathlineto{\pgfqpoint{2.759224in}{0.678494in}}%
\pgfpathlineto{\pgfqpoint{2.623083in}{0.690448in}}%
\pgfpathlineto{\pgfqpoint{2.469488in}{0.706344in}}%
\pgfpathlineto{\pgfqpoint{2.299806in}{0.726395in}}%
\pgfpathlineto{\pgfqpoint{2.164358in}{0.744235in}}%
\pgfpathlineto{\pgfqpoint{2.025733in}{0.764441in}}%
\pgfpathlineto{\pgfqpoint{1.888139in}{0.786868in}}%
\pgfpathlineto{\pgfqpoint{1.756273in}{0.811267in}}%
\pgfpathlineto{\pgfqpoint{1.674115in}{0.828460in}}%
\pgfpathlineto{\pgfqpoint{1.598093in}{0.846261in}}%
\pgfpathlineto{\pgfqpoint{1.528643in}{0.864502in}}%
\pgfpathlineto{\pgfqpoint{1.466010in}{0.883027in}}%
\pgfpathlineto{\pgfqpoint{1.410296in}{0.901691in}}%
\pgfpathlineto{\pgfqpoint{1.361463in}{0.920364in}}%
\pgfpathlineto{\pgfqpoint{1.319330in}{0.938927in}}%
\pgfpathlineto{\pgfqpoint{1.283572in}{0.957277in}}%
\pgfpathlineto{\pgfqpoint{1.253725in}{0.975321in}}%
\pgfpathlineto{\pgfqpoint{1.229181in}{0.992982in}}%
\pgfpathlineto{\pgfqpoint{1.209198in}{1.010195in}}%
\pgfpathlineto{\pgfqpoint{1.193578in}{1.026886in}}%
\pgfpathlineto{\pgfqpoint{1.181863in}{1.043006in}}%
\pgfpathlineto{\pgfqpoint{1.173313in}{1.058531in}}%
\pgfpathlineto{\pgfqpoint{1.167368in}{1.073437in}}%
\pgfpathlineto{\pgfqpoint{1.163647in}{1.087707in}}%
\pgfpathlineto{\pgfqpoint{1.161949in}{1.101325in}}%
\pgfpathlineto{\pgfqpoint{1.162250in}{1.114281in}}%
\pgfpathlineto{\pgfqpoint{1.164707in}{1.126568in}}%
\pgfpathlineto{\pgfqpoint{1.169655in}{1.138182in}}%
\pgfpathlineto{\pgfqpoint{1.177594in}{1.149124in}}%
\pgfpathlineto{\pgfqpoint{1.188039in}{1.159377in}}%
\pgfpathlineto{\pgfqpoint{1.200426in}{1.168927in}}%
\pgfpathlineto{\pgfqpoint{1.214703in}{1.177772in}}%
\pgfpathlineto{\pgfqpoint{1.230843in}{1.185913in}}%
\pgfpathlineto{\pgfqpoint{1.258558in}{1.196799in}}%
\pgfpathlineto{\pgfqpoint{1.290596in}{1.206095in}}%
\pgfpathlineto{\pgfqpoint{1.327213in}{1.213800in}}%
\pgfpathlineto{\pgfqpoint{1.368797in}{1.219915in}}%
\pgfpathlineto{\pgfqpoint{1.415526in}{1.224424in}}%
\pgfpathlineto{\pgfqpoint{1.467804in}{1.227293in}}%
\pgfpathlineto{\pgfqpoint{1.526261in}{1.228481in}}%
\pgfpathlineto{\pgfqpoint{1.591527in}{1.227935in}}%
\pgfpathlineto{\pgfqpoint{1.690227in}{1.224396in}}%
\pgfpathlineto{\pgfqpoint{1.803639in}{1.217478in}}%
\pgfpathlineto{\pgfqpoint{1.933248in}{1.206966in}}%
\pgfpathlineto{\pgfqpoint{2.080332in}{1.192619in}}%
\pgfpathlineto{\pgfqpoint{2.244491in}{1.174172in}}%
\pgfpathlineto{\pgfqpoint{2.422466in}{1.151451in}}%
\pgfpathlineto{\pgfqpoint{2.561093in}{1.131636in}}%
\pgfpathlineto{\pgfqpoint{2.699639in}{1.109567in}}%
\pgfpathlineto{\pgfqpoint{2.833060in}{1.085436in}}%
\pgfpathlineto{\pgfqpoint{2.916768in}{1.068383in}}%
\pgfpathlineto{\pgfqpoint{2.994946in}{1.050724in}}%
\pgfpathlineto{\pgfqpoint{3.066755in}{1.032596in}}%
\pgfpathlineto{\pgfqpoint{3.131638in}{1.014137in}}%
\pgfpathlineto{\pgfqpoint{3.189315in}{0.995480in}}%
\pgfpathlineto{\pgfqpoint{3.239783in}{0.976755in}}%
\pgfpathlineto{\pgfqpoint{3.283317in}{0.958091in}}%
\pgfpathlineto{\pgfqpoint{3.320184in}{0.939626in}}%
\pgfpathlineto{\pgfqpoint{3.350854in}{0.921473in}}%
\pgfpathlineto{\pgfqpoint{3.376459in}{0.903693in}}%
\pgfpathlineto{\pgfqpoint{3.397891in}{0.886341in}}%
\pgfpathlineto{\pgfqpoint{3.415779in}{0.869468in}}%
\pgfpathlineto{\pgfqpoint{3.430486in}{0.853118in}}%
\pgfpathlineto{\pgfqpoint{3.442114in}{0.837332in}}%
\pgfpathlineto{\pgfqpoint{3.450501in}{0.822147in}}%
\pgfpathlineto{\pgfqpoint{3.455223in}{0.807594in}}%
\pgfpathlineto{\pgfqpoint{3.456088in}{0.793703in}}%
\pgfpathlineto{\pgfqpoint{3.454414in}{0.780496in}}%
\pgfpathlineto{\pgfqpoint{3.450522in}{0.767982in}}%
\pgfpathlineto{\pgfqpoint{3.444571in}{0.756167in}}%
\pgfpathlineto{\pgfqpoint{3.436670in}{0.745053in}}%
\pgfpathlineto{\pgfqpoint{3.426881in}{0.734643in}}%
\pgfpathlineto{\pgfqpoint{3.415213in}{0.724937in}}%
\pgfpathlineto{\pgfqpoint{3.401629in}{0.715933in}}%
\pgfpathlineto{\pgfqpoint{3.386040in}{0.707628in}}%
\pgfpathlineto{\pgfqpoint{3.358888in}{0.696479in}}%
\pgfpathlineto{\pgfqpoint{3.327334in}{0.686909in}}%
\pgfpathlineto{\pgfqpoint{3.291255in}{0.678931in}}%
\pgfpathlineto{\pgfqpoint{3.250429in}{0.672558in}}%
\pgfpathlineto{\pgfqpoint{3.204529in}{0.667809in}}%
\pgfpathlineto{\pgfqpoint{3.153129in}{0.664709in}}%
\pgfpathlineto{\pgfqpoint{3.095702in}{0.663282in}}%
\pgfpathlineto{\pgfqpoint{3.031668in}{0.663557in}}%
\pgfpathlineto{\pgfqpoint{2.935637in}{0.666604in}}%
\pgfpathlineto{\pgfqpoint{2.824530in}{0.673013in}}%
\pgfpathlineto{\pgfqpoint{2.696059in}{0.683098in}}%
\pgfpathlineto{\pgfqpoint{2.549805in}{0.697095in}}%
\pgfpathlineto{\pgfqpoint{2.387208in}{0.715163in}}%
\pgfpathlineto{\pgfqpoint{2.211578in}{0.737383in}}%
\pgfpathlineto{\pgfqpoint{2.074343in}{0.756775in}}%
\pgfpathlineto{\pgfqpoint{1.935735in}{0.778478in}}%
\pgfpathlineto{\pgfqpoint{1.801216in}{0.802310in}}%
\pgfpathlineto{\pgfqpoint{1.716331in}{0.819174in}}%
\pgfpathlineto{\pgfqpoint{1.636642in}{0.836670in}}%
\pgfpathlineto{\pgfqpoint{1.563035in}{0.854667in}}%
\pgfpathlineto{\pgfqpoint{1.496151in}{0.873035in}}%
\pgfpathlineto{\pgfqpoint{1.436385in}{0.891643in}}%
\pgfpathlineto{\pgfqpoint{1.383882in}{0.910362in}}%
\pgfpathlineto{\pgfqpoint{1.338545in}{0.929059in}}%
\pgfpathlineto{\pgfqpoint{1.300164in}{0.947593in}}%
\pgfpathlineto{\pgfqpoint{1.268247in}{0.965828in}}%
\pgfpathlineto{\pgfqpoint{1.241590in}{0.983704in}}%
\pgfpathlineto{\pgfqpoint{1.219213in}{1.001167in}}%
\pgfpathlineto{\pgfqpoint{1.200419in}{1.018167in}}%
\pgfpathlineto{\pgfqpoint{1.184784in}{1.034659in}}%
\pgfpathlineto{\pgfqpoint{1.172165in}{1.050601in}}%
\pgfpathlineto{\pgfqpoint{1.162695in}{1.065953in}}%
\pgfpathlineto{\pgfqpoint{1.156786in}{1.080682in}}%
\pgfpathlineto{\pgfqpoint{1.155017in}{1.094757in}}%
\pgfpathlineto{\pgfqpoint{1.156223in}{1.108149in}}%
\pgfpathlineto{\pgfqpoint{1.159687in}{1.120847in}}%
\pgfpathlineto{\pgfqpoint{1.165243in}{1.132846in}}%
\pgfpathlineto{\pgfqpoint{1.172767in}{1.144141in}}%
\pgfpathlineto{\pgfqpoint{1.182188in}{1.154731in}}%
\pgfpathlineto{\pgfqpoint{1.193479in}{1.164615in}}%
\pgfpathlineto{\pgfqpoint{1.206664in}{1.173795in}}%
\pgfpathlineto{\pgfqpoint{1.221814in}{1.182273in}}%
\pgfpathlineto{\pgfqpoint{1.248325in}{1.193679in}}%
\pgfpathlineto{\pgfqpoint{1.279235in}{1.203506in}}%
\pgfpathlineto{\pgfqpoint{1.314642in}{1.211743in}}%
\pgfpathlineto{\pgfqpoint{1.354752in}{1.218375in}}%
\pgfpathlineto{\pgfqpoint{1.399877in}{1.223383in}}%
\pgfpathlineto{\pgfqpoint{1.450425in}{1.226747in}}%
\pgfpathlineto{\pgfqpoint{1.506909in}{1.228438in}}%
\pgfpathlineto{\pgfqpoint{1.569938in}{1.228428in}}%
\pgfpathlineto{\pgfqpoint{1.664533in}{1.225749in}}%
\pgfpathlineto{\pgfqpoint{1.773742in}{1.219758in}}%
\pgfpathlineto{\pgfqpoint{1.900157in}{1.210120in}}%
\pgfpathlineto{\pgfqpoint{2.044460in}{1.196583in}}%
\pgfpathlineto{\pgfqpoint{2.205419in}{1.178979in}}%
\pgfpathlineto{\pgfqpoint{2.379893in}{1.157223in}}%
\pgfpathlineto{\pgfqpoint{2.516648in}{1.138180in}}%
\pgfpathlineto{\pgfqpoint{2.655315in}{1.116829in}}%
\pgfpathlineto{\pgfqpoint{2.790877in}{1.093296in}}%
\pgfpathlineto{\pgfqpoint{2.876840in}{1.076588in}}%
\pgfpathlineto{\pgfqpoint{2.957845in}{1.059218in}}%
\pgfpathlineto{\pgfqpoint{3.032948in}{1.041318in}}%
\pgfpathlineto{\pgfqpoint{3.101451in}{1.023018in}}%
\pgfpathlineto{\pgfqpoint{3.162906in}{1.004450in}}%
\pgfpathlineto{\pgfqpoint{3.217114in}{0.985743in}}%
\pgfpathlineto{\pgfqpoint{3.264123in}{0.967026in}}%
\pgfpathlineto{\pgfqpoint{3.304201in}{0.948431in}}%
\pgfpathlineto{\pgfqpoint{3.337580in}{0.930109in}}%
\pgfpathlineto{\pgfqpoint{3.365334in}{0.912138in}}%
\pgfpathlineto{\pgfqpoint{3.388533in}{0.894571in}}%
\pgfpathlineto{\pgfqpoint{3.407974in}{0.877458in}}%
\pgfpathlineto{\pgfqpoint{3.424177in}{0.860845in}}%
\pgfpathlineto{\pgfqpoint{3.437388in}{0.844775in}}%
\pgfpathlineto{\pgfqpoint{3.447576in}{0.829286in}}%
\pgfpathlineto{\pgfqpoint{3.454435in}{0.814411in}}%
\pgfpathlineto{\pgfqpoint{3.457385in}{0.800181in}}%
\pgfpathlineto{\pgfqpoint{3.456759in}{0.786626in}}%
\pgfpathlineto{\pgfqpoint{3.453813in}{0.773764in}}%
\pgfpathlineto{\pgfqpoint{3.448735in}{0.761599in}}%
\pgfpathlineto{\pgfqpoint{3.441659in}{0.750138in}}%
\pgfpathlineto{\pgfqpoint{3.432673in}{0.739381in}}%
\pgfpathlineto{\pgfqpoint{3.421817in}{0.729330in}}%
\pgfpathlineto{\pgfqpoint{3.409086in}{0.719984in}}%
\pgfpathlineto{\pgfqpoint{3.394426in}{0.711342in}}%
\pgfpathlineto{\pgfqpoint{3.377741in}{0.703398in}}%
\pgfpathlineto{\pgfqpoint{3.349000in}{0.692795in}}%
\pgfpathlineto{\pgfqpoint{3.315823in}{0.683775in}}%
\pgfpathlineto{\pgfqpoint{3.278052in}{0.676352in}}%
\pgfpathlineto{\pgfqpoint{3.235427in}{0.670542in}}%
\pgfpathlineto{\pgfqpoint{3.187591in}{0.666366in}}%
\pgfpathlineto{\pgfqpoint{3.134087in}{0.663850in}}%
\pgfpathlineto{\pgfqpoint{3.074356in}{0.663021in}}%
\pgfpathlineto{\pgfqpoint{3.007858in}{0.663906in}}%
\pgfpathlineto{\pgfqpoint{2.908101in}{0.667795in}}%
\pgfpathlineto{\pgfqpoint{2.792607in}{0.675121in}}%
\pgfpathlineto{\pgfqpoint{2.659373in}{0.686211in}}%
\pgfpathlineto{\pgfqpoint{2.508402in}{0.701295in}}%
\pgfpathlineto{\pgfqpoint{2.341691in}{0.720506in}}%
\pgfpathlineto{\pgfqpoint{2.163241in}{0.743879in}}%
\pgfpathlineto{\pgfqpoint{2.025232in}{0.764108in}}%
\pgfpathlineto{\pgfqpoint{1.887494in}{0.786584in}}%
\pgfpathlineto{\pgfqpoint{1.755688in}{0.811079in}}%
\pgfpathlineto{\pgfqpoint{1.673442in}{0.828308in}}%
\pgfpathlineto{\pgfqpoint{1.596905in}{0.846098in}}%
\pgfpathlineto{\pgfqpoint{1.526831in}{0.864315in}}%
\pgfpathlineto{\pgfqpoint{1.463718in}{0.882829in}}%
\pgfpathlineto{\pgfqpoint{1.407816in}{0.901508in}}%
\pgfpathlineto{\pgfqpoint{1.359119in}{0.920225in}}%
\pgfpathlineto{\pgfqpoint{1.317371in}{0.938854in}}%
\pgfpathlineto{\pgfqpoint{1.282277in}{0.957258in}}%
\pgfpathlineto{\pgfqpoint{1.253255in}{0.975320in}}%
\pgfpathlineto{\pgfqpoint{1.229152in}{0.992985in}}%
\pgfpathlineto{\pgfqpoint{1.209063in}{1.010203in}}%
\pgfpathlineto{\pgfqpoint{1.192352in}{1.026928in}}%
\pgfpathlineto{\pgfqpoint{1.178650in}{1.043117in}}%
\pgfpathlineto{\pgfqpoint{1.167861in}{1.058734in}}%
\pgfpathlineto{\pgfqpoint{1.160153in}{1.073744in}}%
\pgfpathlineto{\pgfqpoint{1.155967in}{1.088119in}}%
\pgfpathlineto{\pgfqpoint{1.155714in}{1.101831in}}%
\pgfpathlineto{\pgfqpoint{1.158091in}{1.114855in}}%
\pgfpathlineto{\pgfqpoint{1.162656in}{1.127183in}}%
\pgfpathlineto{\pgfqpoint{1.169258in}{1.138810in}}%
\pgfpathlineto{\pgfqpoint{1.177791in}{1.149734in}}%
\pgfpathlineto{\pgfqpoint{1.188198in}{1.159951in}}%
\pgfpathlineto{\pgfqpoint{1.200465in}{1.169463in}}%
\pgfpathlineto{\pgfqpoint{1.214627in}{1.178271in}}%
\pgfpathlineto{\pgfqpoint{1.230764in}{1.186379in}}%
\pgfpathlineto{\pgfqpoint{1.258766in}{1.197231in}}%
\pgfpathlineto{\pgfqpoint{1.291203in}{1.206503in}}%
\pgfpathlineto{\pgfqpoint{1.328212in}{1.214183in}}%
\pgfpathlineto{\pgfqpoint{1.370034in}{1.220255in}}%
\pgfpathlineto{\pgfqpoint{1.417008in}{1.224697in}}%
\pgfpathlineto{\pgfqpoint{1.469570in}{1.227483in}}%
\pgfpathlineto{\pgfqpoint{1.528251in}{1.228584in}}%
\pgfpathlineto{\pgfqpoint{1.593682in}{1.227964in}}%
\pgfpathlineto{\pgfqpoint{1.692016in}{1.224401in}}%
\pgfpathlineto{\pgfqpoint{1.805434in}{1.217457in}}%
\pgfpathlineto{\pgfqpoint{1.936181in}{1.206836in}}%
\pgfpathlineto{\pgfqpoint{2.084628in}{1.192304in}}%
\pgfpathlineto{\pgfqpoint{2.249277in}{1.173692in}}%
\pgfpathlineto{\pgfqpoint{2.426756in}{1.150897in}}%
\pgfpathlineto{\pgfqpoint{2.565181in}{1.131028in}}%
\pgfpathlineto{\pgfqpoint{2.703687in}{1.108820in}}%
\pgfpathlineto{\pgfqpoint{2.836821in}{1.084600in}}%
\pgfpathlineto{\pgfqpoint{2.920381in}{1.067536in}}%
\pgfpathlineto{\pgfqpoint{2.998551in}{1.049880in}}%
\pgfpathlineto{\pgfqpoint{3.070521in}{1.031758in}}%
\pgfpathlineto{\pgfqpoint{3.135692in}{1.013297in}}%
\pgfpathlineto{\pgfqpoint{3.193687in}{0.994625in}}%
\pgfpathlineto{\pgfqpoint{3.244341in}{0.975874in}}%
\pgfpathlineto{\pgfqpoint{3.287709in}{0.957176in}}%
\pgfpathlineto{\pgfqpoint{3.324101in}{0.938679in}}%
\pgfpathlineto{\pgfqpoint{3.354439in}{0.920496in}}%
\pgfpathlineto{\pgfqpoint{3.379756in}{0.902694in}}%
\pgfpathlineto{\pgfqpoint{3.400850in}{0.885331in}}%
\pgfpathlineto{\pgfqpoint{3.418282in}{0.868460in}}%
\pgfpathlineto{\pgfqpoint{3.432377in}{0.852126in}}%
\pgfpathlineto{\pgfqpoint{3.443220in}{0.836370in}}%
\pgfpathlineto{\pgfqpoint{3.450663in}{0.821228in}}%
\pgfpathlineto{\pgfqpoint{3.454351in}{0.806727in}}%
\pgfpathlineto{\pgfqpoint{3.454989in}{0.792896in}}%
\pgfpathlineto{\pgfqpoint{3.453238in}{0.779749in}}%
\pgfpathlineto{\pgfqpoint{3.449301in}{0.767292in}}%
\pgfpathlineto{\pgfqpoint{3.443329in}{0.755532in}}%
\pgfpathlineto{\pgfqpoint{3.435415in}{0.744473in}}%
\pgfpathlineto{\pgfqpoint{3.425603in}{0.734116in}}%
\pgfpathlineto{\pgfqpoint{3.413879in}{0.724461in}}%
\pgfpathlineto{\pgfqpoint{3.400177in}{0.715504in}}%
\pgfpathlineto{\pgfqpoint{3.384417in}{0.707244in}}%
\pgfpathlineto{\pgfqpoint{3.357113in}{0.696164in}}%
\pgfpathlineto{\pgfqpoint{3.325421in}{0.686665in}}%
\pgfpathlineto{\pgfqpoint{3.289212in}{0.678759in}}%
\pgfpathlineto{\pgfqpoint{3.248254in}{0.672459in}}%
\pgfpathlineto{\pgfqpoint{3.202209in}{0.667783in}}%
\pgfpathlineto{\pgfqpoint{3.150632in}{0.664752in}}%
\pgfpathlineto{\pgfqpoint{3.092972in}{0.663388in}}%
\pgfpathlineto{\pgfqpoint{3.028904in}{0.663708in}}%
\pgfpathlineto{\pgfqpoint{2.932478in}{0.666873in}}%
\pgfpathlineto{\pgfqpoint{2.820617in}{0.673435in}}%
\pgfpathlineto{\pgfqpoint{2.691568in}{0.683649in}}%
\pgfpathlineto{\pgfqpoint{2.545116in}{0.697728in}}%
\pgfpathlineto{\pgfqpoint{2.382580in}{0.715841in}}%
\pgfpathlineto{\pgfqpoint{2.206820in}{0.738115in}}%
\pgfpathlineto{\pgfqpoint{2.068928in}{0.757603in}}%
\pgfpathlineto{\pgfqpoint{1.930025in}{0.779439in}}%
\pgfpathlineto{\pgfqpoint{1.795780in}{0.803327in}}%
\pgfpathlineto{\pgfqpoint{1.711208in}{0.820210in}}%
\pgfpathlineto{\pgfqpoint{1.631884in}{0.837721in}}%
\pgfpathlineto{\pgfqpoint{1.558677in}{0.855739in}}%
\pgfpathlineto{\pgfqpoint{1.492230in}{0.874133in}}%
\pgfpathlineto{\pgfqpoint{1.432960in}{0.892769in}}%
\pgfpathlineto{\pgfqpoint{1.381062in}{0.911507in}}%
\pgfpathlineto{\pgfqpoint{1.336517in}{0.930193in}}%
\pgfpathlineto{\pgfqpoint{1.298708in}{0.948694in}}%
\pgfpathlineto{\pgfqpoint{1.266642in}{0.966925in}}%
\pgfpathlineto{\pgfqpoint{1.239521in}{0.984808in}}%
\pgfpathlineto{\pgfqpoint{1.216756in}{1.002277in}}%
\pgfpathlineto{\pgfqpoint{1.197969in}{1.019268in}}%
\pgfpathlineto{\pgfqpoint{1.182990in}{1.035730in}}%
\pgfpathlineto{\pgfqpoint{1.171858in}{1.051616in}}%
\pgfpathlineto{\pgfqpoint{1.164729in}{1.066887in}}%
\pgfpathlineto{\pgfqpoint{1.160762in}{1.081510in}}%
\pgfpathlineto{\pgfqpoint{1.159413in}{1.095469in}}%
\pgfpathlineto{\pgfqpoint{1.160423in}{1.108752in}}%
\pgfpathlineto{\pgfqpoint{1.163602in}{1.121348in}}%
\pgfpathlineto{\pgfqpoint{1.168828in}{1.133253in}}%
\pgfpathlineto{\pgfqpoint{1.176045in}{1.144462in}}%
\pgfpathlineto{\pgfqpoint{1.185268in}{1.154973in}}%
\pgfpathlineto{\pgfqpoint{1.196578in}{1.164790in}}%
\pgfpathlineto{\pgfqpoint{1.209956in}{1.173911in}}%
\pgfpathlineto{\pgfqpoint{1.225268in}{1.182334in}}%
\pgfpathlineto{\pgfqpoint{1.251809in}{1.193655in}}%
\pgfpathlineto{\pgfqpoint{1.282656in}{1.203390in}}%
\pgfpathlineto{\pgfqpoint{1.317942in}{1.211533in}}%
\pgfpathlineto{\pgfqpoint{1.357924in}{1.218074in}}%
\pgfpathlineto{\pgfqpoint{1.402979in}{1.223003in}}%
\pgfpathlineto{\pgfqpoint{1.453608in}{1.226312in}}%
\pgfpathlineto{\pgfqpoint{1.510036in}{1.227980in}}%
\pgfpathlineto{\pgfqpoint{1.572832in}{1.227943in}}%
\pgfpathlineto{\pgfqpoint{1.643014in}{1.226130in}}%
\pgfpathlineto{\pgfqpoint{1.749463in}{1.220810in}}%
\pgfpathlineto{\pgfqpoint{1.871849in}{1.211987in}}%
\pgfpathlineto{\pgfqpoint{2.010910in}{1.199445in}}%
\pgfpathlineto{\pgfqpoint{2.166747in}{1.182947in}}%
\pgfpathlineto{\pgfqpoint{2.339001in}{1.162232in}}%
\pgfpathlineto{\pgfqpoint{2.476707in}{1.143812in}}%
\pgfpathlineto{\pgfqpoint{2.616057in}{1.123060in}}%
\pgfpathlineto{\pgfqpoint{2.752220in}{1.100191in}}%
\pgfpathlineto{\pgfqpoint{2.881059in}{1.075481in}}%
\pgfpathlineto{\pgfqpoint{2.961155in}{1.058150in}}%
\pgfpathlineto{\pgfqpoint{3.035605in}{1.040267in}}%
\pgfpathlineto{\pgfqpoint{3.103701in}{1.021959in}}%
\pgfpathlineto{\pgfqpoint{3.164870in}{1.003366in}}%
\pgfpathlineto{\pgfqpoint{3.218677in}{0.984643in}}%
\pgfpathlineto{\pgfqpoint{3.265015in}{0.965946in}}%
\pgfpathlineto{\pgfqpoint{3.304667in}{0.947395in}}%
\pgfpathlineto{\pgfqpoint{3.338385in}{0.929088in}}%
\pgfpathlineto{\pgfqpoint{3.366794in}{0.911113in}}%
\pgfpathlineto{\pgfqpoint{3.390394in}{0.893546in}}%
\pgfpathlineto{\pgfqpoint{3.409567in}{0.876453in}}%
\pgfpathlineto{\pgfqpoint{3.424574in}{0.859891in}}%
\pgfpathlineto{\pgfqpoint{3.435687in}{0.843906in}}%
\pgfpathlineto{\pgfqpoint{3.443517in}{0.828531in}}%
\pgfpathlineto{\pgfqpoint{3.448440in}{0.813792in}}%
\pgfpathlineto{\pgfqpoint{3.450737in}{0.799706in}}%
\pgfpathlineto{\pgfqpoint{3.450616in}{0.786288in}}%
\pgfpathlineto{\pgfqpoint{3.448206in}{0.773549in}}%
\pgfpathlineto{\pgfqpoint{3.443565in}{0.761496in}}%
\pgfpathlineto{\pgfqpoint{3.436701in}{0.750132in}}%
\pgfpathlineto{\pgfqpoint{3.427740in}{0.739461in}}%
\pgfpathlineto{\pgfqpoint{3.416804in}{0.729485in}}%
\pgfpathlineto{\pgfqpoint{3.403977in}{0.720208in}}%
\pgfpathlineto{\pgfqpoint{3.389308in}{0.711633in}}%
\pgfpathlineto{\pgfqpoint{3.363878in}{0.700086in}}%
\pgfpathlineto{\pgfqpoint{3.334237in}{0.690122in}}%
\pgfpathlineto{\pgfqpoint{3.300116in}{0.681736in}}%
\pgfpathlineto{\pgfqpoint{3.261072in}{0.674921in}}%
\pgfpathlineto{\pgfqpoint{3.216860in}{0.669689in}}%
\pgfpathlineto{\pgfqpoint{3.167307in}{0.666078in}}%
\pgfpathlineto{\pgfqpoint{3.111848in}{0.664124in}}%
\pgfpathlineto{\pgfqpoint{3.049896in}{0.663876in}}%
\pgfpathlineto{\pgfqpoint{2.980841in}{0.665392in}}%
\pgfpathlineto{\pgfqpoint{2.876625in}{0.670277in}}%
\pgfpathlineto{\pgfqpoint{2.757104in}{0.678615in}}%
\pgfpathlineto{\pgfqpoint{2.620735in}{0.690633in}}%
\pgfpathlineto{\pgfqpoint{2.466855in}{0.706596in}}%
\pgfpathlineto{\pgfqpoint{2.296832in}{0.726732in}}%
\pgfpathlineto{\pgfqpoint{2.161382in}{0.744625in}}%
\pgfpathlineto{\pgfqpoint{2.022559in}{0.764885in}}%
\pgfpathlineto{\pgfqpoint{1.884827in}{0.787377in}}%
\pgfpathlineto{\pgfqpoint{1.753351in}{0.811819in}}%
\pgfpathlineto{\pgfqpoint{1.671341in}{0.829021in}}%
\pgfpathlineto{\pgfqpoint{1.595083in}{0.846809in}}%
\pgfpathlineto{\pgfqpoint{1.525426in}{0.865043in}}%
\pgfpathlineto{\pgfqpoint{1.463086in}{0.883571in}}%
\pgfpathlineto{\pgfqpoint{1.408227in}{0.902238in}}%
\pgfpathlineto{\pgfqpoint{1.360243in}{0.920918in}}%
\pgfpathlineto{\pgfqpoint{1.318568in}{0.939497in}}%
\pgfpathlineto{\pgfqpoint{1.282716in}{0.957871in}}%
\pgfpathlineto{\pgfqpoint{1.252278in}{0.975946in}}%
\pgfpathlineto{\pgfqpoint{1.226925in}{0.993638in}}%
\pgfpathlineto{\pgfqpoint{1.206406in}{1.010875in}}%
\pgfpathlineto{\pgfqpoint{1.190546in}{1.027594in}}%
\pgfpathlineto{\pgfqpoint{1.178749in}{1.043740in}}%
\pgfpathlineto{\pgfqpoint{1.170305in}{1.059281in}}%
\pgfpathlineto{\pgfqpoint{1.164760in}{1.074193in}}%
\pgfpathlineto{\pgfqpoint{1.161772in}{1.088455in}}%
\pgfpathlineto{\pgfqpoint{1.161111in}{1.102054in}}%
\pgfpathlineto{\pgfqpoint{1.162655in}{1.114977in}}%
\pgfpathlineto{\pgfqpoint{1.166395in}{1.127217in}}%
\pgfpathlineto{\pgfqpoint{1.172430in}{1.138772in}}%
\pgfpathlineto{\pgfqpoint{1.180848in}{1.149639in}}%
\pgfpathlineto{\pgfqpoint{1.191319in}{1.159810in}}%
\pgfpathlineto{\pgfqpoint{1.203727in}{1.169282in}}%
\pgfpathlineto{\pgfqpoint{1.218017in}{1.178052in}}%
\pgfpathlineto{\pgfqpoint{1.234160in}{1.186120in}}%
\pgfpathlineto{\pgfqpoint{1.261868in}{1.196902in}}%
\pgfpathlineto{\pgfqpoint{1.293900in}{1.206099in}}%
\pgfpathlineto{\pgfqpoint{1.330543in}{1.213711in}}%
\pgfpathlineto{\pgfqpoint{1.372221in}{1.219742in}}%
\pgfpathlineto{\pgfqpoint{1.419078in}{1.224171in}}%
\pgfpathlineto{\pgfqpoint{1.471519in}{1.226962in}}%
\pgfpathlineto{\pgfqpoint{1.530163in}{1.228073in}}%
\pgfpathlineto{\pgfqpoint{1.595632in}{1.227449in}}%
\pgfpathlineto{\pgfqpoint{1.694622in}{1.223808in}}%
\pgfpathlineto{\pgfqpoint{1.808350in}{1.216786in}}%
\pgfpathlineto{\pgfqpoint{1.938320in}{1.206169in}}%
\pgfpathlineto{\pgfqpoint{2.085753in}{1.191713in}}%
\pgfpathlineto{\pgfqpoint{2.250181in}{1.173152in}}%
\pgfpathlineto{\pgfqpoint{2.428282in}{1.150329in}}%
\pgfpathlineto{\pgfqpoint{2.566838in}{1.130446in}}%
\pgfpathlineto{\pgfqpoint{2.705103in}{1.108310in}}%
\pgfpathlineto{\pgfqpoint{2.838111in}{1.084124in}}%
\pgfpathlineto{\pgfqpoint{2.921464in}{1.067059in}}%
\pgfpathlineto{\pgfqpoint{2.999232in}{1.049396in}}%
\pgfpathlineto{\pgfqpoint{3.070591in}{1.031267in}}%
\pgfpathlineto{\pgfqpoint{3.134998in}{1.012806in}}%
\pgfpathlineto{\pgfqpoint{3.192191in}{0.994149in}}%
\pgfpathlineto{\pgfqpoint{3.242188in}{0.975432in}}%
\pgfpathlineto{\pgfqpoint{3.285099in}{0.956801in}}%
\pgfpathlineto{\pgfqpoint{3.321230in}{0.938393in}}%
\pgfpathlineto{\pgfqpoint{3.351738in}{0.920281in}}%
\pgfpathlineto{\pgfqpoint{3.377577in}{0.902528in}}%
\pgfpathlineto{\pgfqpoint{3.399436in}{0.885195in}}%
\pgfpathlineto{\pgfqpoint{3.417741in}{0.868337in}}%
\pgfpathlineto{\pgfqpoint{3.432657in}{0.852002in}}%
\pgfpathlineto{\pgfqpoint{3.444083in}{0.836238in}}%
\pgfpathlineto{\pgfqpoint{3.451655in}{0.821082in}}%
\pgfpathlineto{\pgfqpoint{3.455046in}{0.806573in}}%
\pgfpathlineto{\pgfqpoint{3.455596in}{0.792738in}}%
\pgfpathlineto{\pgfqpoint{3.453786in}{0.779587in}}%
\pgfpathlineto{\pgfqpoint{3.449813in}{0.767129in}}%
\pgfpathlineto{\pgfqpoint{3.443816in}{0.755369in}}%
\pgfpathlineto{\pgfqpoint{3.435877in}{0.744311in}}%
\pgfpathlineto{\pgfqpoint{3.426024in}{0.733954in}}%
\pgfpathlineto{\pgfqpoint{3.414231in}{0.724300in}}%
\pgfpathlineto{\pgfqpoint{3.400414in}{0.715344in}}%
\pgfpathlineto{\pgfqpoint{3.384574in}{0.707085in}}%
\pgfpathlineto{\pgfqpoint{3.357198in}{0.696012in}}%
\pgfpathlineto{\pgfqpoint{3.325453in}{0.686522in}}%
\pgfpathlineto{\pgfqpoint{3.289208in}{0.678628in}}%
\pgfpathlineto{\pgfqpoint{3.248223in}{0.672342in}}%
\pgfpathlineto{\pgfqpoint{3.202146in}{0.667680in}}%
\pgfpathlineto{\pgfqpoint{3.150515in}{0.664659in}}%
\pgfpathlineto{\pgfqpoint{3.092767in}{0.663300in}}%
\pgfpathlineto{\pgfqpoint{3.028794in}{0.663624in}}%
\pgfpathlineto{\pgfqpoint{2.932091in}{0.666829in}}%
\pgfpathlineto{\pgfqpoint{2.819893in}{0.673435in}}%
\pgfpathlineto{\pgfqpoint{2.690724in}{0.683673in}}%
\pgfpathlineto{\pgfqpoint{2.544398in}{0.697752in}}%
\pgfpathlineto{\pgfqpoint{2.382020in}{0.715856in}}%
\pgfpathlineto{\pgfqpoint{2.205984in}{0.738149in}}%
\pgfpathlineto{\pgfqpoint{2.067488in}{0.757700in}}%
\pgfpathlineto{\pgfqpoint{1.928491in}{0.779571in}}%
\pgfpathlineto{\pgfqpoint{1.794404in}{0.803465in}}%
\pgfpathlineto{\pgfqpoint{1.709965in}{0.820350in}}%
\pgfpathlineto{\pgfqpoint{1.630753in}{0.837869in}}%
\pgfpathlineto{\pgfqpoint{1.557634in}{0.855900in}}%
\pgfpathlineto{\pgfqpoint{1.491274in}{0.874312in}}%
\pgfpathlineto{\pgfqpoint{1.432140in}{0.892965in}}%
\pgfpathlineto{\pgfqpoint{1.380504in}{0.911710in}}%
\pgfpathlineto{\pgfqpoint{1.336221in}{0.930387in}}%
\pgfpathlineto{\pgfqpoint{1.298392in}{0.948887in}}%
\pgfpathlineto{\pgfqpoint{1.266161in}{0.967122in}}%
\pgfpathlineto{\pgfqpoint{1.238860in}{0.985012in}}%
\pgfpathlineto{\pgfqpoint{1.216006in}{1.002484in}}%
\pgfpathlineto{\pgfqpoint{1.197305in}{1.019476in}}%
\pgfpathlineto{\pgfqpoint{1.182647in}{1.035933in}}%
\pgfpathlineto{\pgfqpoint{1.172113in}{1.051807in}}%
\pgfpathlineto{\pgfqpoint{1.165270in}{1.067060in}}%
\pgfpathlineto{\pgfqpoint{1.161311in}{1.081665in}}%
\pgfpathlineto{\pgfqpoint{1.159917in}{1.095608in}}%
\pgfpathlineto{\pgfqpoint{1.160847in}{1.108876in}}%
\pgfpathlineto{\pgfqpoint{1.163933in}{1.121459in}}%
\pgfpathlineto{\pgfqpoint{1.169085in}{1.133351in}}%
\pgfpathlineto{\pgfqpoint{1.176285in}{1.144549in}}%
\pgfpathlineto{\pgfqpoint{1.185595in}{1.155052in}}%
\pgfpathlineto{\pgfqpoint{1.197045in}{1.164862in}}%
\pgfpathlineto{\pgfqpoint{1.210461in}{1.173975in}}%
\pgfpathlineto{\pgfqpoint{1.225785in}{1.182387in}}%
\pgfpathlineto{\pgfqpoint{1.252305in}{1.193689in}}%
\pgfpathlineto{\pgfqpoint{1.283098in}{1.203405in}}%
\pgfpathlineto{\pgfqpoint{1.318324in}{1.211528in}}%
\pgfpathlineto{\pgfqpoint{1.358276in}{1.218054in}}%
\pgfpathlineto{\pgfqpoint{1.403377in}{1.222977in}}%
\pgfpathlineto{\pgfqpoint{1.454034in}{1.226290in}}%
\pgfpathlineto{\pgfqpoint{1.510448in}{1.227947in}}%
\pgfpathlineto{\pgfqpoint{1.573458in}{1.227892in}}%
\pgfpathlineto{\pgfqpoint{1.643859in}{1.226059in}}%
\pgfpathlineto{\pgfqpoint{1.750393in}{1.220721in}}%
\pgfpathlineto{\pgfqpoint{1.872591in}{1.211892in}}%
\pgfpathlineto{\pgfqpoint{2.011424in}{1.199344in}}%
\pgfpathlineto{\pgfqpoint{2.167541in}{1.182822in}}%
\pgfpathlineto{\pgfqpoint{2.341795in}{1.162026in}}%
\pgfpathlineto{\pgfqpoint{2.479502in}{1.143604in}}%
\pgfpathlineto{\pgfqpoint{2.617990in}{1.122874in}}%
\pgfpathlineto{\pgfqpoint{2.753072in}{1.100024in}}%
\pgfpathlineto{\pgfqpoint{2.881047in}{1.075314in}}%
\pgfpathlineto{\pgfqpoint{2.960807in}{1.057969in}}%
\pgfpathlineto{\pgfqpoint{3.035135in}{1.040063in}}%
\pgfpathlineto{\pgfqpoint{3.103295in}{1.021729in}}%
\pgfpathlineto{\pgfqpoint{3.164643in}{1.003113in}}%
\pgfpathlineto{\pgfqpoint{3.218634in}{0.984379in}}%
\pgfpathlineto{\pgfqpoint{3.265037in}{0.965689in}}%
\pgfpathlineto{\pgfqpoint{3.304773in}{0.947145in}}%
\pgfpathlineto{\pgfqpoint{3.338505in}{0.928846in}}%
\pgfpathlineto{\pgfqpoint{3.366783in}{0.910882in}}%
\pgfpathlineto{\pgfqpoint{3.390101in}{0.893332in}}%
\pgfpathlineto{\pgfqpoint{3.408900in}{0.876262in}}%
\pgfpathlineto{\pgfqpoint{3.423638in}{0.859727in}}%
\pgfpathlineto{\pgfqpoint{3.434791in}{0.843766in}}%
\pgfpathlineto{\pgfqpoint{3.442734in}{0.828412in}}%
\pgfpathlineto{\pgfqpoint{3.447768in}{0.813691in}}%
\pgfpathlineto{\pgfqpoint{3.450116in}{0.799622in}}%
\pgfpathlineto{\pgfqpoint{3.449922in}{0.786221in}}%
\pgfpathlineto{\pgfqpoint{3.447278in}{0.773497in}}%
\pgfpathlineto{\pgfqpoint{3.442399in}{0.761457in}}%
\pgfpathlineto{\pgfqpoint{3.435475in}{0.750108in}}%
\pgfpathlineto{\pgfqpoint{3.426642in}{0.739455in}}%
\pgfpathlineto{\pgfqpoint{3.415986in}{0.729500in}}%
\pgfpathlineto{\pgfqpoint{3.403539in}{0.720245in}}%
\pgfpathlineto{\pgfqpoint{3.389281in}{0.711688in}}%
\pgfpathlineto{\pgfqpoint{3.373138in}{0.703828in}}%
\pgfpathlineto{\pgfqpoint{3.345099in}{0.693334in}}%
\pgfpathlineto{\pgfqpoint{3.312148in}{0.684388in}}%
\pgfpathlineto{\pgfqpoint{3.274523in}{0.677020in}}%
\pgfpathlineto{\pgfqpoint{3.231987in}{0.671249in}}%
\pgfpathlineto{\pgfqpoint{3.184188in}{0.667100in}}%
\pgfpathlineto{\pgfqpoint{3.130690in}{0.664604in}}%
\pgfpathlineto{\pgfqpoint{3.070978in}{0.663802in}}%
\pgfpathlineto{\pgfqpoint{3.004457in}{0.664741in}}%
\pgfpathlineto{\pgfqpoint{2.904001in}{0.668796in}}%
\pgfpathlineto{\pgfqpoint{2.788640in}{0.676230in}}%
\pgfpathlineto{\pgfqpoint{2.656449in}{0.687304in}}%
\pgfpathlineto{\pgfqpoint{2.506788in}{0.702266in}}%
\pgfpathlineto{\pgfqpoint{2.340726in}{0.721328in}}%
\pgfpathlineto{\pgfqpoint{2.161132in}{0.744661in}}%
\pgfpathlineto{\pgfqpoint{2.022239in}{0.764933in}}%
\pgfpathlineto{\pgfqpoint{1.884980in}{0.787408in}}%
\pgfpathlineto{\pgfqpoint{1.753769in}{0.811832in}}%
\pgfpathlineto{\pgfqpoint{1.671691in}{0.829029in}}%
\pgfpathlineto{\pgfqpoint{1.595265in}{0.846816in}}%
\pgfpathlineto{\pgfqpoint{1.525568in}{0.865049in}}%
\pgfpathlineto{\pgfqpoint{1.463303in}{0.883574in}}%
\pgfpathlineto{\pgfqpoint{1.408251in}{0.902245in}}%
\pgfpathlineto{\pgfqpoint{1.360023in}{0.920933in}}%
\pgfpathlineto{\pgfqpoint{1.318234in}{0.939518in}}%
\pgfpathlineto{\pgfqpoint{1.282493in}{0.957894in}}%
\pgfpathlineto{\pgfqpoint{1.252412in}{0.975963in}}%
\pgfpathlineto{\pgfqpoint{1.227599in}{0.993643in}}%
\pgfpathlineto{\pgfqpoint{1.207659in}{1.010858in}}%
\pgfpathlineto{\pgfqpoint{1.192042in}{1.027548in}}%
\pgfpathlineto{\pgfqpoint{1.180127in}{1.043671in}}%
\pgfpathlineto{\pgfqpoint{1.171419in}{1.059196in}}%
\pgfpathlineto{\pgfqpoint{1.165544in}{1.074097in}}%
\pgfpathlineto{\pgfqpoint{1.162251in}{1.088351in}}%
\pgfpathlineto{\pgfqpoint{1.161408in}{1.101944in}}%
\pgfpathlineto{\pgfqpoint{1.163001in}{1.114864in}}%
\pgfpathlineto{\pgfqpoint{1.167118in}{1.127104in}}%
\pgfpathlineto{\pgfqpoint{1.173521in}{1.138656in}}%
\pgfpathlineto{\pgfqpoint{1.181982in}{1.149513in}}%
\pgfpathlineto{\pgfqpoint{1.192389in}{1.159674in}}%
\pgfpathlineto{\pgfqpoint{1.204665in}{1.169134in}}%
\pgfpathlineto{\pgfqpoint{1.218773in}{1.177892in}}%
\pgfpathlineto{\pgfqpoint{1.234711in}{1.185949in}}%
\pgfpathlineto{\pgfqpoint{1.262139in}{1.196719in}}%
\pgfpathlineto{\pgfqpoint{1.294051in}{1.205917in}}%
\pgfpathlineto{\pgfqpoint{1.330834in}{1.213552in}}%
\pgfpathlineto{\pgfqpoint{1.372489in}{1.219600in}}%
\pgfpathlineto{\pgfqpoint{1.419338in}{1.224037in}}%
\pgfpathlineto{\pgfqpoint{1.471831in}{1.226831in}}%
\pgfpathlineto{\pgfqpoint{1.530483in}{1.227940in}}%
\pgfpathlineto{\pgfqpoint{1.595864in}{1.227313in}}%
\pgfpathlineto{\pgfqpoint{1.694603in}{1.223672in}}%
\pgfpathlineto{\pgfqpoint{1.808091in}{1.216665in}}%
\pgfpathlineto{\pgfqpoint{1.937966in}{1.206075in}}%
\pgfpathlineto{\pgfqpoint{2.085323in}{1.191630in}}%
\pgfpathlineto{\pgfqpoint{2.249528in}{1.173099in}}%
\pgfpathlineto{\pgfqpoint{2.427566in}{1.150320in}}%
\pgfpathlineto{\pgfqpoint{2.566023in}{1.130456in}}%
\pgfpathlineto{\pgfqpoint{2.704177in}{1.108341in}}%
\pgfpathlineto{\pgfqpoint{2.837237in}{1.084209in}}%
\pgfpathlineto{\pgfqpoint{2.920479in}{1.067143in}}%
\pgfpathlineto{\pgfqpoint{2.998028in}{1.049450in}}%
\pgfpathlineto{\pgfqpoint{3.069235in}{1.031293in}}%
\pgfpathlineto{\pgfqpoint{3.133654in}{1.012825in}}%
\pgfpathlineto{\pgfqpoint{3.191042in}{0.994187in}}%
\pgfpathlineto{\pgfqpoint{3.241354in}{0.975510in}}%
\pgfpathlineto{\pgfqpoint{3.284751in}{0.956912in}}%
\pgfpathlineto{\pgfqpoint{3.321593in}{0.938499in}}%
\pgfpathlineto{\pgfqpoint{3.352442in}{0.920368in}}%
\pgfpathlineto{\pgfqpoint{3.378063in}{0.902601in}}%
\pgfpathlineto{\pgfqpoint{3.398909in}{0.885284in}}%
\pgfpathlineto{\pgfqpoint{3.415248in}{0.868485in}}%
\pgfpathlineto{\pgfqpoint{3.427927in}{0.852237in}}%
\pgfpathlineto{\pgfqpoint{3.437601in}{0.836571in}}%
\pgfpathlineto{\pgfqpoint{3.444726in}{0.821513in}}%
\pgfpathlineto{\pgfqpoint{3.449559in}{0.807082in}}%
\pgfpathlineto{\pgfqpoint{3.452156in}{0.793297in}}%
\pgfpathlineto{\pgfqpoint{3.452375in}{0.780170in}}%
\pgfpathlineto{\pgfqpoint{3.449874in}{0.767710in}}%
\pgfpathlineto{\pgfqpoint{3.444212in}{0.755923in}}%
\pgfpathlineto{\pgfqpoint{3.436163in}{0.744828in}}%
\pgfpathlineto{\pgfqpoint{3.426127in}{0.734436in}}%
\pgfpathlineto{\pgfqpoint{3.414187in}{0.724746in}}%
\pgfpathlineto{\pgfqpoint{3.400391in}{0.715762in}}%
\pgfpathlineto{\pgfqpoint{3.384755in}{0.707483in}}%
\pgfpathlineto{\pgfqpoint{3.357812in}{0.696386in}}%
\pgfpathlineto{\pgfqpoint{3.326502in}{0.686874in}}%
\pgfpathlineto{\pgfqpoint{3.290480in}{0.678938in}}%
\pgfpathlineto{\pgfqpoint{3.249591in}{0.672588in}}%
\pgfpathlineto{\pgfqpoint{3.203607in}{0.667847in}}%
\pgfpathlineto{\pgfqpoint{3.152063in}{0.664746in}}%
\pgfpathlineto{\pgfqpoint{3.094447in}{0.663325in}}%
\pgfpathlineto{\pgfqpoint{3.030191in}{0.663632in}}%
\pgfpathlineto{\pgfqpoint{2.933114in}{0.666836in}}%
\pgfpathlineto{\pgfqpoint{2.821521in}{0.673390in}}%
\pgfpathlineto{\pgfqpoint{2.693726in}{0.683500in}}%
\pgfpathlineto{\pgfqpoint{2.548512in}{0.697432in}}%
\pgfpathlineto{\pgfqpoint{2.386176in}{0.715431in}}%
\pgfpathlineto{\pgfqpoint{2.209427in}{0.737670in}}%
\pgfpathlineto{\pgfqpoint{2.071140in}{0.757144in}}%
\pgfpathlineto{\pgfqpoint{1.932429in}{0.778903in}}%
\pgfpathlineto{\pgfqpoint{1.797947in}{0.802731in}}%
\pgfpathlineto{\pgfqpoint{1.713200in}{0.819616in}}%
\pgfpathlineto{\pgfqpoint{1.634018in}{0.837176in}}%
\pgfpathlineto{\pgfqpoint{1.561089in}{0.855247in}}%
\pgfpathlineto{\pgfqpoint{1.494871in}{0.873671in}}%
\pgfpathlineto{\pgfqpoint{1.435636in}{0.892301in}}%
\pgfpathlineto{\pgfqpoint{1.383463in}{0.911002in}}%
\pgfpathlineto{\pgfqpoint{1.338245in}{0.929652in}}%
\pgfpathlineto{\pgfqpoint{1.299684in}{0.948140in}}%
\pgfpathlineto{\pgfqpoint{1.267292in}{0.966369in}}%
\pgfpathlineto{\pgfqpoint{1.240395in}{0.984253in}}%
\pgfpathlineto{\pgfqpoint{1.218249in}{1.001712in}}%
\pgfpathlineto{\pgfqpoint{1.200786in}{1.018667in}}%
\pgfpathlineto{\pgfqpoint{1.187270in}{1.035075in}}%
\pgfpathlineto{\pgfqpoint{1.176963in}{1.050906in}}%
\pgfpathlineto{\pgfqpoint{1.169327in}{1.066135in}}%
\pgfpathlineto{\pgfqpoint{1.164016in}{1.080739in}}%
\pgfpathlineto{\pgfqpoint{1.160886in}{1.094702in}}%
\pgfpathlineto{\pgfqpoint{1.159986in}{1.108009in}}%
\pgfpathlineto{\pgfqpoint{1.161563in}{1.120650in}}%
\pgfpathlineto{\pgfqpoint{1.166063in}{1.132621in}}%
\pgfpathlineto{\pgfqpoint{1.173531in}{1.143911in}}%
\pgfpathlineto{\pgfqpoint{1.183051in}{1.154501in}}%
\pgfpathlineto{\pgfqpoint{1.194504in}{1.164387in}}%
\pgfpathlineto{\pgfqpoint{1.207830in}{1.173569in}}%
\pgfpathlineto{\pgfqpoint{1.223001in}{1.182045in}}%
\pgfpathlineto{\pgfqpoint{1.249227in}{1.193435in}}%
\pgfpathlineto{\pgfqpoint{1.279753in}{1.203238in}}%
\pgfpathlineto{\pgfqpoint{1.314873in}{1.211460in}}%
\pgfpathlineto{\pgfqpoint{1.354897in}{1.218100in}}%
\pgfpathlineto{\pgfqpoint{1.399929in}{1.223135in}}%
\pgfpathlineto{\pgfqpoint{1.450426in}{1.226538in}}%
\pgfpathlineto{\pgfqpoint{1.506905in}{1.228270in}}%
\pgfpathlineto{\pgfqpoint{1.569930in}{1.228283in}}%
\pgfpathlineto{\pgfqpoint{1.640109in}{1.226518in}}%
\pgfpathlineto{\pgfqpoint{1.745942in}{1.221280in}}%
\pgfpathlineto{\pgfqpoint{1.867327in}{1.212568in}}%
\pgfpathlineto{\pgfqpoint{2.005746in}{1.200142in}}%
\pgfpathlineto{\pgfqpoint{2.161699in}{1.183734in}}%
\pgfpathlineto{\pgfqpoint{2.333411in}{1.163141in}}%
\pgfpathlineto{\pgfqpoint{2.469881in}{1.144888in}}%
\pgfpathlineto{\pgfqpoint{2.608984in}{1.124280in}}%
\pgfpathlineto{\pgfqpoint{2.746223in}{1.101479in}}%
\pgfpathlineto{\pgfqpoint{2.876851in}{1.076754in}}%
\pgfpathlineto{\pgfqpoint{2.957753in}{1.059377in}}%
\pgfpathlineto{\pgfqpoint{3.032276in}{1.041432in}}%
\pgfpathlineto{\pgfqpoint{3.100092in}{1.023082in}}%
\pgfpathlineto{\pgfqpoint{3.161036in}{1.004484in}}%
\pgfpathlineto{\pgfqpoint{3.215071in}{0.985779in}}%
\pgfpathlineto{\pgfqpoint{3.262287in}{0.967094in}}%
\pgfpathlineto{\pgfqpoint{3.302903in}{0.948547in}}%
\pgfpathlineto{\pgfqpoint{3.337268in}{0.930237in}}%
\pgfpathlineto{\pgfqpoint{3.365854in}{0.912254in}}%
\pgfpathlineto{\pgfqpoint{3.389266in}{0.894673in}}%
\pgfpathlineto{\pgfqpoint{3.408194in}{0.877557in}}%
\pgfpathlineto{\pgfqpoint{3.422759in}{0.860981in}}%
\pgfpathlineto{\pgfqpoint{3.433568in}{0.844984in}}%
\pgfpathlineto{\pgfqpoint{3.441345in}{0.829591in}}%
\pgfpathlineto{\pgfqpoint{3.446630in}{0.814823in}}%
\pgfpathlineto{\pgfqpoint{3.449784in}{0.800697in}}%
\pgfpathlineto{\pgfqpoint{3.450985in}{0.787226in}}%
\pgfpathlineto{\pgfqpoint{3.450231in}{0.774419in}}%
\pgfpathlineto{\pgfqpoint{3.447337in}{0.762283in}}%
\pgfpathlineto{\pgfqpoint{3.441940in}{0.750820in}}%
\pgfpathlineto{\pgfqpoint{3.433509in}{0.740029in}}%
\pgfpathlineto{\pgfqpoint{3.422571in}{0.729929in}}%
\pgfpathlineto{\pgfqpoint{3.409703in}{0.720533in}}%
\pgfpathlineto{\pgfqpoint{3.394951in}{0.711841in}}%
\pgfpathlineto{\pgfqpoint{3.369327in}{0.700129in}}%
\pgfpathlineto{\pgfqpoint{3.339459in}{0.690009in}}%
\pgfpathlineto{\pgfqpoint{3.305172in}{0.681482in}}%
\pgfpathlineto{\pgfqpoint{3.266155in}{0.674549in}}%
\pgfpathlineto{\pgfqpoint{3.222028in}{0.669212in}}%
\pgfpathlineto{\pgfqpoint{3.172639in}{0.665497in}}%
\pgfpathlineto{\pgfqpoint{3.117412in}{0.663440in}}%
\pgfpathlineto{\pgfqpoint{3.055693in}{0.663091in}}%
\pgfpathlineto{\pgfqpoint{2.986838in}{0.664506in}}%
\pgfpathlineto{\pgfqpoint{2.882845in}{0.669260in}}%
\pgfpathlineto{\pgfqpoint{2.763584in}{0.677470in}}%
\pgfpathlineto{\pgfqpoint{2.627627in}{0.689365in}}%
\pgfpathlineto{\pgfqpoint{2.473969in}{0.705184in}}%
\pgfpathlineto{\pgfqpoint{2.304064in}{0.725189in}}%
\pgfpathlineto{\pgfqpoint{2.168645in}{0.743019in}}%
\pgfpathlineto{\pgfqpoint{2.029773in}{0.763216in}}%
\pgfpathlineto{\pgfqpoint{1.891607in}{0.785634in}}%
\pgfpathlineto{\pgfqpoint{1.759300in}{0.810041in}}%
\pgfpathlineto{\pgfqpoint{1.676874in}{0.827264in}}%
\pgfpathlineto{\pgfqpoint{1.600238in}{0.845075in}}%
\pgfpathlineto{\pgfqpoint{1.530087in}{0.863318in}}%
\pgfpathlineto{\pgfqpoint{1.466880in}{0.881849in}}%
\pgfpathlineto{\pgfqpoint{1.410835in}{0.900531in}}%
\pgfpathlineto{\pgfqpoint{1.361934in}{0.919238in}}%
\pgfpathlineto{\pgfqpoint{1.319918in}{0.937851in}}%
\pgfpathlineto{\pgfqpoint{1.284292in}{0.956263in}}%
\pgfpathlineto{\pgfqpoint{1.254360in}{0.974373in}}%
\pgfpathlineto{\pgfqpoint{1.229936in}{0.992073in}}%
\pgfpathlineto{\pgfqpoint{1.210246in}{1.009303in}}%
\pgfpathlineto{\pgfqpoint{1.194389in}{1.026023in}}%
\pgfpathlineto{\pgfqpoint{1.181694in}{1.042196in}}%
\pgfpathlineto{\pgfqpoint{1.171722in}{1.057791in}}%
\pgfpathlineto{\pgfqpoint{1.164267in}{1.072780in}}%
\pgfpathlineto{\pgfqpoint{1.159357in}{1.087139in}}%
\pgfpathlineto{\pgfqpoint{1.157249in}{1.100851in}}%
\pgfpathlineto{\pgfqpoint{1.158436in}{1.113901in}}%
\pgfpathlineto{\pgfqpoint{1.162760in}{1.126269in}}%
\pgfpathlineto{\pgfqpoint{1.169223in}{1.137940in}}%
\pgfpathlineto{\pgfqpoint{1.177688in}{1.148908in}}%
\pgfpathlineto{\pgfqpoint{1.188060in}{1.159173in}}%
\pgfpathlineto{\pgfqpoint{1.200284in}{1.168731in}}%
\pgfpathlineto{\pgfqpoint{1.214340in}{1.177584in}}%
\pgfpathlineto{\pgfqpoint{1.230249in}{1.185732in}}%
\pgfpathlineto{\pgfqpoint{1.257723in}{1.196638in}}%
\pgfpathlineto{\pgfqpoint{1.289818in}{1.205971in}}%
\pgfpathlineto{\pgfqpoint{1.326550in}{1.213725in}}%
\pgfpathlineto{\pgfqpoint{1.368111in}{1.219880in}}%
\pgfpathlineto{\pgfqpoint{1.414844in}{1.224416in}}%
\pgfpathlineto{\pgfqpoint{1.467170in}{1.227300in}}%
\pgfpathlineto{\pgfqpoint{1.525590in}{1.228494in}}%
\pgfpathlineto{\pgfqpoint{1.590683in}{1.227952in}}%
\pgfpathlineto{\pgfqpoint{1.689002in}{1.224435in}}%
\pgfpathlineto{\pgfqpoint{1.802006in}{1.217573in}}%
\pgfpathlineto{\pgfqpoint{1.931492in}{1.207114in}}%
\pgfpathlineto{\pgfqpoint{2.078514in}{1.192798in}}%
\pgfpathlineto{\pgfqpoint{2.242339in}{1.174404in}}%
\pgfpathlineto{\pgfqpoint{2.420377in}{1.151752in}}%
\pgfpathlineto{\pgfqpoint{2.559131in}{1.131969in}}%
\pgfpathlineto{\pgfqpoint{2.697393in}{1.109930in}}%
\pgfpathlineto{\pgfqpoint{2.830700in}{1.085865in}}%
\pgfpathlineto{\pgfqpoint{2.914650in}{1.068857in}}%
\pgfpathlineto{\pgfqpoint{2.993170in}{1.051216in}}%
\pgfpathlineto{\pgfqpoint{3.065018in}{1.033086in}}%
\pgfpathlineto{\pgfqpoint{3.129631in}{1.014614in}}%
\pgfpathlineto{\pgfqpoint{3.187144in}{0.995950in}}%
\pgfpathlineto{\pgfqpoint{3.237774in}{0.977227in}}%
\pgfpathlineto{\pgfqpoint{3.281791in}{0.958570in}}%
\pgfpathlineto{\pgfqpoint{3.319521in}{0.940090in}}%
\pgfpathlineto{\pgfqpoint{3.351343in}{0.921889in}}%
\pgfpathlineto{\pgfqpoint{3.377692in}{0.904053in}}%
\pgfpathlineto{\pgfqpoint{3.399053in}{0.886660in}}%
\pgfpathlineto{\pgfqpoint{3.415924in}{0.869779in}}%
\pgfpathlineto{\pgfqpoint{3.428945in}{0.853452in}}%
\pgfpathlineto{\pgfqpoint{3.438679in}{0.837713in}}%
\pgfpathlineto{\pgfqpoint{3.445552in}{0.822590in}}%
\pgfpathlineto{\pgfqpoint{3.449847in}{0.808104in}}%
\pgfpathlineto{\pgfqpoint{3.451705in}{0.794274in}}%
\pgfpathlineto{\pgfqpoint{3.451127in}{0.781113in}}%
\pgfpathlineto{\pgfqpoint{3.447973in}{0.768628in}}%
\pgfpathlineto{\pgfqpoint{3.442269in}{0.756826in}}%
\pgfpathlineto{\pgfqpoint{3.434457in}{0.745719in}}%
\pgfpathlineto{\pgfqpoint{3.424666in}{0.735310in}}%
\pgfpathlineto{\pgfqpoint{3.412980in}{0.725601in}}%
\pgfpathlineto{\pgfqpoint{3.399449in}{0.716594in}}%
\pgfpathlineto{\pgfqpoint{3.384088in}{0.708290in}}%
\pgfpathlineto{\pgfqpoint{3.357556in}{0.697151in}}%
\pgfpathlineto{\pgfqpoint{3.326620in}{0.687588in}}%
\pgfpathlineto{\pgfqpoint{3.290900in}{0.679592in}}%
\pgfpathlineto{\pgfqpoint{3.250332in}{0.673178in}}%
\pgfpathlineto{\pgfqpoint{3.204666in}{0.668370in}}%
\pgfpathlineto{\pgfqpoint{3.153463in}{0.665199in}}%
\pgfpathlineto{\pgfqpoint{3.096223in}{0.663703in}}%
\pgfpathlineto{\pgfqpoint{3.032389in}{0.663932in}}%
\pgfpathlineto{\pgfqpoint{2.935941in}{0.667022in}}%
\pgfpathlineto{\pgfqpoint{2.825035in}{0.673446in}}%
\pgfpathlineto{\pgfqpoint{2.697990in}{0.683412in}}%
\pgfpathlineto{\pgfqpoint{2.553554in}{0.697191in}}%
\pgfpathlineto{\pgfqpoint{2.392004in}{0.715021in}}%
\pgfpathlineto{\pgfqpoint{2.215866in}{0.737083in}}%
\pgfpathlineto{\pgfqpoint{2.077884in}{0.756424in}}%
\pgfpathlineto{\pgfqpoint{1.939221in}{0.778055in}}%
\pgfpathlineto{\pgfqpoint{1.804517in}{0.801766in}}%
\pgfpathlineto{\pgfqpoint{1.719516in}{0.818586in}}%
\pgfpathlineto{\pgfqpoint{1.639942in}{0.836092in}}%
\pgfpathlineto{\pgfqpoint{1.566522in}{0.854115in}}%
\pgfpathlineto{\pgfqpoint{1.499766in}{0.872500in}}%
\pgfpathlineto{\pgfqpoint{1.439983in}{0.891101in}}%
\pgfpathlineto{\pgfqpoint{1.387283in}{0.909784in}}%
\pgfpathlineto{\pgfqpoint{1.341575in}{0.928426in}}%
\pgfpathlineto{\pgfqpoint{1.302567in}{0.946917in}}%
\pgfpathlineto{\pgfqpoint{1.269766in}{0.965158in}}%
\pgfpathlineto{\pgfqpoint{1.242482in}{0.983060in}}%
\pgfpathlineto{\pgfqpoint{1.219997in}{1.000543in}}%
\pgfpathlineto{\pgfqpoint{1.202234in}{1.017525in}}%
\pgfpathlineto{\pgfqpoint{1.188409in}{1.033966in}}%
\pgfpathlineto{\pgfqpoint{1.177796in}{1.049835in}}%
\pgfpathlineto{\pgfqpoint{1.169868in}{1.065104in}}%
\pgfpathlineto{\pgfqpoint{1.164295in}{1.079751in}}%
\pgfpathlineto{\pgfqpoint{1.160947in}{1.093758in}}%
\pgfpathlineto{\pgfqpoint{1.159895in}{1.107110in}}%
\pgfpathlineto{\pgfqpoint{1.161406in}{1.119798in}}%
\pgfpathlineto{\pgfqpoint{1.165947in}{1.131816in}}%
\pgfpathlineto{\pgfqpoint{1.173299in}{1.143149in}}%
\pgfpathlineto{\pgfqpoint{1.182673in}{1.153781in}}%
\pgfpathlineto{\pgfqpoint{1.193977in}{1.163711in}}%
\pgfpathlineto{\pgfqpoint{1.207151in}{1.172936in}}%
\pgfpathlineto{\pgfqpoint{1.222166in}{1.181456in}}%
\pgfpathlineto{\pgfqpoint{1.248160in}{1.192913in}}%
\pgfpathlineto{\pgfqpoint{1.278459in}{1.202785in}}%
\pgfpathlineto{\pgfqpoint{1.313375in}{1.211077in}}%
\pgfpathlineto{\pgfqpoint{1.353167in}{1.217789in}}%
\pgfpathlineto{\pgfqpoint{1.397959in}{1.222897in}}%
\pgfpathlineto{\pgfqpoint{1.448200in}{1.226373in}}%
\pgfpathlineto{\pgfqpoint{1.504393in}{1.228180in}}%
\pgfpathlineto{\pgfqpoint{1.567095in}{1.228270in}}%
\pgfpathlineto{\pgfqpoint{1.636911in}{1.226585in}}%
\pgfpathlineto{\pgfqpoint{1.742208in}{1.221461in}}%
\pgfpathlineto{\pgfqpoint{1.863016in}{1.212873in}}%
\pgfpathlineto{\pgfqpoint{2.000826in}{1.200580in}}%
\pgfpathlineto{\pgfqpoint{2.156164in}{1.184313in}}%
\pgfpathlineto{\pgfqpoint{2.327371in}{1.163869in}}%
\pgfpathlineto{\pgfqpoint{2.463593in}{1.145728in}}%
\pgfpathlineto{\pgfqpoint{2.602613in}{1.125228in}}%
\pgfpathlineto{\pgfqpoint{2.740067in}{1.102525in}}%
\pgfpathlineto{\pgfqpoint{2.871082in}{1.077882in}}%
\pgfpathlineto{\pgfqpoint{2.952258in}{1.060535in}}%
\pgfpathlineto{\pgfqpoint{3.027362in}{1.042619in}}%
\pgfpathlineto{\pgfqpoint{3.095883in}{1.024296in}}%
\pgfpathlineto{\pgfqpoint{3.157497in}{1.005720in}}%
\pgfpathlineto{\pgfqpoint{3.212065in}{0.987027in}}%
\pgfpathlineto{\pgfqpoint{3.259630in}{0.968346in}}%
\pgfpathlineto{\pgfqpoint{3.300425in}{0.949790in}}%
\pgfpathlineto{\pgfqpoint{3.334865in}{0.931463in}}%
\pgfpathlineto{\pgfqpoint{3.363550in}{0.913454in}}%
\pgfpathlineto{\pgfqpoint{3.387266in}{0.895840in}}%
\pgfpathlineto{\pgfqpoint{3.406302in}{0.878706in}}%
\pgfpathlineto{\pgfqpoint{3.421064in}{0.862109in}}%
\pgfpathlineto{\pgfqpoint{3.432363in}{0.846078in}}%
\pgfpathlineto{\pgfqpoint{3.440818in}{0.830642in}}%
\pgfpathlineto{\pgfqpoint{3.446848in}{0.815824in}}%
\pgfpathlineto{\pgfqpoint{3.450681in}{0.801643in}}%
\pgfpathlineto{\pgfqpoint{3.452346in}{0.788113in}}%
\pgfpathlineto{\pgfqpoint{3.451678in}{0.775246in}}%
\pgfpathlineto{\pgfqpoint{3.448316in}{0.763048in}}%
\pgfpathlineto{\pgfqpoint{3.441819in}{0.751524in}}%
\pgfpathlineto{\pgfqpoint{3.432977in}{0.740694in}}%
\pgfpathlineto{\pgfqpoint{3.422172in}{0.730567in}}%
\pgfpathlineto{\pgfqpoint{3.409474in}{0.721144in}}%
\pgfpathlineto{\pgfqpoint{3.394923in}{0.712427in}}%
\pgfpathlineto{\pgfqpoint{3.369635in}{0.700674in}}%
\pgfpathlineto{\pgfqpoint{3.340097in}{0.690510in}}%
\pgfpathlineto{\pgfqpoint{3.306057in}{0.681930in}}%
\pgfpathlineto{\pgfqpoint{3.267155in}{0.674929in}}%
\pgfpathlineto{\pgfqpoint{3.223295in}{0.669527in}}%
\pgfpathlineto{\pgfqpoint{3.174089in}{0.665750in}}%
\pgfpathlineto{\pgfqpoint{3.119021in}{0.663634in}}%
\pgfpathlineto{\pgfqpoint{3.057534in}{0.663225in}}%
\pgfpathlineto{\pgfqpoint{2.989030in}{0.664579in}}%
\pgfpathlineto{\pgfqpoint{2.885664in}{0.669244in}}%
\pgfpathlineto{\pgfqpoint{2.767049in}{0.677350in}}%
\pgfpathlineto{\pgfqpoint{2.631605in}{0.689124in}}%
\pgfpathlineto{\pgfqpoint{2.478587in}{0.704837in}}%
\pgfpathlineto{\pgfqpoint{2.309305in}{0.724708in}}%
\pgfpathlineto{\pgfqpoint{2.174027in}{0.742414in}}%
\pgfpathlineto{\pgfqpoint{2.035102in}{0.762498in}}%
\pgfpathlineto{\pgfqpoint{1.897035in}{0.784822in}}%
\pgfpathlineto{\pgfqpoint{1.764504in}{0.809139in}}%
\pgfpathlineto{\pgfqpoint{1.681582in}{0.826288in}}%
\pgfpathlineto{\pgfqpoint{1.605018in}{0.844051in}}%
\pgfpathlineto{\pgfqpoint{1.535485in}{0.862278in}}%
\pgfpathlineto{\pgfqpoint{1.472821in}{0.880807in}}%
\pgfpathlineto{\pgfqpoint{1.416838in}{0.899491in}}%
\pgfpathlineto{\pgfqpoint{1.367322in}{0.918195in}}%
\pgfpathlineto{\pgfqpoint{1.324032in}{0.936800in}}%
\pgfpathlineto{\pgfqpoint{1.286703in}{0.955199in}}%
\pgfpathlineto{\pgfqpoint{1.255042in}{0.973299in}}%
\pgfpathlineto{\pgfqpoint{1.228731in}{0.991022in}}%
\pgfpathlineto{\pgfqpoint{1.207426in}{1.008304in}}%
\pgfpathlineto{\pgfqpoint{1.190757in}{1.025094in}}%
\pgfpathlineto{\pgfqpoint{1.178329in}{1.041354in}}%
\pgfpathlineto{\pgfqpoint{1.169730in}{1.057019in}}%
\pgfpathlineto{\pgfqpoint{1.164341in}{1.072048in}}%
\pgfpathlineto{\pgfqpoint{1.161590in}{1.086426in}}%
\pgfpathlineto{\pgfqpoint{1.161043in}{1.100137in}}%
\pgfpathlineto{\pgfqpoint{1.162403in}{1.113174in}}%
\pgfpathlineto{\pgfqpoint{1.165507in}{1.125527in}}%
\pgfpathlineto{\pgfqpoint{1.170331in}{1.137194in}}%
\pgfpathlineto{\pgfqpoint{1.176986in}{1.148175in}}%
\pgfpathlineto{\pgfqpoint{1.185722in}{1.158471in}}%
\pgfpathlineto{\pgfqpoint{1.196921in}{1.168090in}}%
\pgfpathlineto{\pgfqpoint{1.210902in}{1.177033in}}%
\pgfpathlineto{\pgfqpoint{1.226942in}{1.185278in}}%
\pgfpathlineto{\pgfqpoint{1.254581in}{1.196321in}}%
\pgfpathlineto{\pgfqpoint{1.286568in}{1.205772in}}%
\pgfpathlineto{\pgfqpoint{1.323061in}{1.213623in}}%
\pgfpathlineto{\pgfqpoint{1.364326in}{1.219864in}}%
\pgfpathlineto{\pgfqpoint{1.410739in}{1.224482in}}%
\pgfpathlineto{\pgfqpoint{1.462778in}{1.227465in}}%
\pgfpathlineto{\pgfqpoint{1.520595in}{1.228790in}}%
\pgfpathlineto{\pgfqpoint{1.584968in}{1.228394in}}%
\pgfpathlineto{\pgfqpoint{1.682756in}{1.225054in}}%
\pgfpathlineto{\pgfqpoint{1.795845in}{1.218321in}}%
\pgfpathlineto{\pgfqpoint{1.925349in}{1.207987in}}%
\pgfpathlineto{\pgfqpoint{2.071671in}{1.193831in}}%
\pgfpathlineto{\pgfqpoint{2.234498in}{1.175619in}}%
\pgfpathlineto{\pgfqpoint{2.412817in}{1.153081in}}%
\pgfpathlineto{\pgfqpoint{2.552428in}{1.133323in}}%
\pgfpathlineto{\pgfqpoint{2.691098in}{1.111331in}}%
\pgfpathlineto{\pgfqpoint{2.824194in}{1.087366in}}%
\pgfpathlineto{\pgfqpoint{2.907889in}{1.070447in}}%
\pgfpathlineto{\pgfqpoint{2.986426in}{1.052899in}}%
\pgfpathlineto{\pgfqpoint{3.059007in}{1.034841in}}%
\pgfpathlineto{\pgfqpoint{3.124988in}{1.016402in}}%
\pgfpathlineto{\pgfqpoint{3.183880in}{0.997724in}}%
\pgfpathlineto{\pgfqpoint{3.235346in}{0.978958in}}%
\pgfpathlineto{\pgfqpoint{3.279419in}{0.960263in}}%
\pgfpathlineto{\pgfqpoint{3.316955in}{0.941753in}}%
\pgfpathlineto{\pgfqpoint{3.348773in}{0.923519in}}%
\pgfpathlineto{\pgfqpoint{3.375531in}{0.905644in}}%
\pgfpathlineto{\pgfqpoint{3.397736in}{0.888197in}}%
\pgfpathlineto{\pgfqpoint{3.415737in}{0.871243in}}%
\pgfpathlineto{\pgfqpoint{3.429731in}{0.854832in}}%
\pgfpathlineto{\pgfqpoint{3.439806in}{0.839009in}}%
\pgfpathlineto{\pgfqpoint{3.446575in}{0.823810in}}%
\pgfpathlineto{\pgfqpoint{3.450508in}{0.809255in}}%
\pgfpathlineto{\pgfqpoint{3.451886in}{0.795361in}}%
\pgfpathlineto{\pgfqpoint{3.450922in}{0.782142in}}%
\pgfpathlineto{\pgfqpoint{3.447754in}{0.769606in}}%
\pgfpathlineto{\pgfqpoint{3.442451in}{0.757760in}}%
\pgfpathlineto{\pgfqpoint{3.435010in}{0.746607in}}%
\pgfpathlineto{\pgfqpoint{3.425433in}{0.736146in}}%
\pgfpathlineto{\pgfqpoint{3.413858in}{0.726382in}}%
\pgfpathlineto{\pgfqpoint{3.400363in}{0.717316in}}%
\pgfpathlineto{\pgfqpoint{3.384996in}{0.708951in}}%
\pgfpathlineto{\pgfqpoint{3.358470in}{0.697722in}}%
\pgfpathlineto{\pgfqpoint{3.327703in}{0.688079in}}%
\pgfpathlineto{\pgfqpoint{3.292480in}{0.680024in}}%
\pgfpathlineto{\pgfqpoint{3.252436in}{0.673556in}}%
\pgfpathlineto{\pgfqpoint{3.207097in}{0.668676in}}%
\pgfpathlineto{\pgfqpoint{3.156358in}{0.665414in}}%
\pgfpathlineto{\pgfqpoint{3.099654in}{0.663817in}}%
\pgfpathlineto{\pgfqpoint{3.036285in}{0.663936in}}%
\pgfpathlineto{\pgfqpoint{2.965585in}{0.665832in}}%
\pgfpathlineto{\pgfqpoint{2.858830in}{0.671253in}}%
\pgfpathlineto{\pgfqpoint{2.736527in}{0.680165in}}%
\pgfpathlineto{\pgfqpoint{2.597392in}{0.692805in}}%
\pgfpathlineto{\pgfqpoint{2.440216in}{0.709399in}}%
\pgfpathlineto{\pgfqpoint{2.267331in}{0.730193in}}%
\pgfpathlineto{\pgfqpoint{2.131091in}{0.748625in}}%
\pgfpathlineto{\pgfqpoint{1.992859in}{0.769418in}}%
\pgfpathlineto{\pgfqpoint{1.856514in}{0.792388in}}%
\pgfpathlineto{\pgfqpoint{1.726489in}{0.817250in}}%
\pgfpathlineto{\pgfqpoint{1.645758in}{0.834684in}}%
\pgfpathlineto{\pgfqpoint{1.571682in}{0.852643in}}%
\pgfpathlineto{\pgfqpoint{1.505288in}{0.870975in}}%
\pgfpathlineto{\pgfqpoint{1.446115in}{0.889540in}}%
\pgfpathlineto{\pgfqpoint{1.393617in}{0.908208in}}%
\pgfpathlineto{\pgfqpoint{1.347322in}{0.926857in}}%
\pgfpathlineto{\pgfqpoint{1.306823in}{0.945380in}}%
\pgfpathlineto{\pgfqpoint{1.271786in}{0.963675in}}%
\pgfpathlineto{\pgfqpoint{1.241943in}{0.981655in}}%
\pgfpathlineto{\pgfqpoint{1.217096in}{0.999240in}}%
\pgfpathlineto{\pgfqpoint{1.197117in}{1.016362in}}%
\pgfpathlineto{\pgfqpoint{1.181945in}{1.032963in}}%
\pgfpathlineto{\pgfqpoint{1.171429in}{1.048992in}}%
\pgfpathlineto{\pgfqpoint{1.164479in}{1.064393in}}%
\pgfpathlineto{\pgfqpoint{1.160431in}{1.079147in}}%
\pgfpathlineto{\pgfqpoint{1.158847in}{1.093236in}}%
\pgfpathlineto{\pgfqpoint{1.159405in}{1.106650in}}%
\pgfpathlineto{\pgfqpoint{1.161900in}{1.119380in}}%
\pgfpathlineto{\pgfqpoint{1.166246in}{1.131419in}}%
\pgfpathlineto{\pgfqpoint{1.172475in}{1.142766in}}%
\pgfpathlineto{\pgfqpoint{1.180734in}{1.153422in}}%
\pgfpathlineto{\pgfqpoint{1.191289in}{1.163390in}}%
\pgfpathlineto{\pgfqpoint{1.204366in}{1.172673in}}%
\pgfpathlineto{\pgfqpoint{1.219457in}{1.181256in}}%
\pgfpathlineto{\pgfqpoint{1.245660in}{1.192810in}}%
\pgfpathlineto{\pgfqpoint{1.276164in}{1.202772in}}%
\pgfpathlineto{\pgfqpoint{1.311096in}{1.211138in}}%
\pgfpathlineto{\pgfqpoint{1.350699in}{1.217898in}}%
\pgfpathlineto{\pgfqpoint{1.395331in}{1.223043in}}%
\pgfpathlineto{\pgfqpoint{1.445462in}{1.226563in}}%
\pgfpathlineto{\pgfqpoint{1.501274in}{1.228437in}}%
\pgfpathlineto{\pgfqpoint{1.563426in}{1.228608in}}%
\pgfpathlineto{\pgfqpoint{1.632890in}{1.227005in}}%
\pgfpathlineto{\pgfqpoint{1.738233in}{1.221977in}}%
\pgfpathlineto{\pgfqpoint{1.859363in}{1.213462in}}%
\pgfpathlineto{\pgfqpoint{1.997115in}{1.201241in}}%
\pgfpathlineto{\pgfqpoint{2.151758in}{1.185076in}}%
\pgfpathlineto{\pgfqpoint{2.323240in}{1.164696in}}%
\pgfpathlineto{\pgfqpoint{2.460697in}{1.146531in}}%
\pgfpathlineto{\pgfqpoint{2.600165in}{1.126019in}}%
\pgfpathlineto{\pgfqpoint{2.736900in}{1.103357in}}%
\pgfpathlineto{\pgfqpoint{2.866788in}{1.078810in}}%
\pgfpathlineto{\pgfqpoint{2.947828in}{1.061559in}}%
\pgfpathlineto{\pgfqpoint{3.023386in}{1.043732in}}%
\pgfpathlineto{\pgfqpoint{3.092708in}{1.025456in}}%
\pgfpathlineto{\pgfqpoint{3.155163in}{1.006873in}}%
\pgfpathlineto{\pgfqpoint{3.210249in}{0.988138in}}%
\pgfpathlineto{\pgfqpoint{3.257799in}{0.969410in}}%
\pgfpathlineto{\pgfqpoint{3.298594in}{0.950808in}}%
\pgfpathlineto{\pgfqpoint{3.333346in}{0.932434in}}%
\pgfpathlineto{\pgfqpoint{3.362650in}{0.914378in}}%
\pgfpathlineto{\pgfqpoint{3.387002in}{0.896720in}}%
\pgfpathlineto{\pgfqpoint{3.406799in}{0.879528in}}%
\pgfpathlineto{\pgfqpoint{3.422338in}{0.862861in}}%
\pgfpathlineto{\pgfqpoint{3.434029in}{0.846767in}}%
\pgfpathlineto{\pgfqpoint{3.442420in}{0.831277in}}%
\pgfpathlineto{\pgfqpoint{3.447869in}{0.816418in}}%
\pgfpathlineto{\pgfqpoint{3.450653in}{0.802210in}}%
\pgfpathlineto{\pgfqpoint{3.450969in}{0.788668in}}%
\pgfpathlineto{\pgfqpoint{3.448941in}{0.775803in}}%
\pgfpathlineto{\pgfqpoint{3.444611in}{0.763623in}}%
\pgfpathlineto{\pgfqpoint{3.438054in}{0.752132in}}%
\pgfpathlineto{\pgfqpoint{3.429426in}{0.741334in}}%
\pgfpathlineto{\pgfqpoint{3.418851in}{0.731232in}}%
\pgfpathlineto{\pgfqpoint{3.406411in}{0.721831in}}%
\pgfpathlineto{\pgfqpoint{3.392153in}{0.713131in}}%
\pgfpathlineto{\pgfqpoint{3.367361in}{0.701398in}}%
\pgfpathlineto{\pgfqpoint{3.338346in}{0.691243in}}%
\pgfpathlineto{\pgfqpoint{3.304777in}{0.682659in}}%
\pgfpathlineto{\pgfqpoint{3.266198in}{0.675638in}}%
\pgfpathlineto{\pgfqpoint{3.222616in}{0.670206in}}%
\pgfpathlineto{\pgfqpoint{3.173692in}{0.666395in}}%
\pgfpathlineto{\pgfqpoint{3.118926in}{0.664239in}}%
\pgfpathlineto{\pgfqpoint{3.057770in}{0.663785in}}%
\pgfpathlineto{\pgfqpoint{2.989630in}{0.665088in}}%
\pgfpathlineto{\pgfqpoint{2.886800in}{0.669674in}}%
\pgfpathlineto{\pgfqpoint{2.768761in}{0.677688in}}%
\pgfpathlineto{\pgfqpoint{2.633946in}{0.689358in}}%
\pgfpathlineto{\pgfqpoint{2.481595in}{0.704956in}}%
\pgfpathlineto{\pgfqpoint{2.312973in}{0.724697in}}%
\pgfpathlineto{\pgfqpoint{2.178077in}{0.742301in}}%
\pgfpathlineto{\pgfqpoint{2.039496in}{0.762280in}}%
\pgfpathlineto{\pgfqpoint{1.901521in}{0.784502in}}%
\pgfpathlineto{\pgfqpoint{1.768862in}{0.808725in}}%
\pgfpathlineto{\pgfqpoint{1.685920in}{0.825825in}}%
\pgfpathlineto{\pgfqpoint{1.608896in}{0.843552in}}%
\pgfpathlineto{\pgfqpoint{1.538329in}{0.861739in}}%
\pgfpathlineto{\pgfqpoint{1.474565in}{0.880231in}}%
\pgfpathlineto{\pgfqpoint{1.417776in}{0.898882in}}%
\pgfpathlineto{\pgfqpoint{1.367968in}{0.917562in}}%
\pgfpathlineto{\pgfqpoint{1.324976in}{0.936151in}}%
\pgfpathlineto{\pgfqpoint{1.288466in}{0.954544in}}%
\pgfpathlineto{\pgfqpoint{1.257934in}{0.972646in}}%
\pgfpathlineto{\pgfqpoint{1.232706in}{0.990375in}}%
\pgfpathlineto{\pgfqpoint{1.212078in}{1.007660in}}%
\pgfpathlineto{\pgfqpoint{1.195985in}{1.024424in}}%
\pgfpathlineto{\pgfqpoint{1.183699in}{1.040627in}}%
\pgfpathlineto{\pgfqpoint{1.174505in}{1.056243in}}%
\pgfpathlineto{\pgfqpoint{1.167884in}{1.071247in}}%
\pgfpathlineto{\pgfqpoint{1.163502in}{1.085620in}}%
\pgfpathlineto{\pgfqpoint{1.161221in}{1.099345in}}%
\pgfpathlineto{\pgfqpoint{1.161090in}{1.112411in}}%
\pgfpathlineto{\pgfqpoint{1.163349in}{1.124811in}}%
\pgfpathlineto{\pgfqpoint{1.168429in}{1.136539in}}%
\pgfpathlineto{\pgfqpoint{1.176526in}{1.147589in}}%
\pgfpathlineto{\pgfqpoint{1.186704in}{1.157939in}}%
\pgfpathlineto{\pgfqpoint{1.198808in}{1.167586in}}%
\pgfpathlineto{\pgfqpoint{1.212785in}{1.176529in}}%
\pgfpathlineto{\pgfqpoint{1.228613in}{1.184766in}}%
\pgfpathlineto{\pgfqpoint{1.255843in}{1.195798in}}%
\pgfpathlineto{\pgfqpoint{1.287399in}{1.205241in}}%
\pgfpathlineto{\pgfqpoint{1.323571in}{1.213099in}}%
\pgfpathlineto{\pgfqpoint{1.364704in}{1.219374in}}%
\pgfpathlineto{\pgfqpoint{1.410916in}{1.224040in}}%
\pgfpathlineto{\pgfqpoint{1.462688in}{1.227067in}}%
\pgfpathlineto{\pgfqpoint{1.520579in}{1.228416in}}%
\pgfpathlineto{\pgfqpoint{1.585177in}{1.228034in}}%
\pgfpathlineto{\pgfqpoint{1.682809in}{1.224725in}}%
\pgfpathlineto{\pgfqpoint{1.795006in}{1.218053in}}%
\pgfpathlineto{\pgfqpoint{1.923400in}{1.207809in}}%
\pgfpathlineto{\pgfqpoint{2.069193in}{1.193743in}}%
\pgfpathlineto{\pgfqpoint{2.232175in}{1.175591in}}%
\pgfpathlineto{\pgfqpoint{2.409302in}{1.153202in}}%
\pgfpathlineto{\pgfqpoint{2.547807in}{1.133615in}}%
\pgfpathlineto{\pgfqpoint{2.686580in}{1.111744in}}%
\pgfpathlineto{\pgfqpoint{2.820578in}{1.087826in}}%
\pgfpathlineto{\pgfqpoint{2.905014in}{1.070902in}}%
\pgfpathlineto{\pgfqpoint{2.984137in}{1.053332in}}%
\pgfpathlineto{\pgfqpoint{3.056878in}{1.035253in}}%
\pgfpathlineto{\pgfqpoint{3.122248in}{1.016820in}}%
\pgfpathlineto{\pgfqpoint{3.180340in}{0.998180in}}%
\pgfpathlineto{\pgfqpoint{3.231576in}{0.979467in}}%
\pgfpathlineto{\pgfqpoint{3.276358in}{0.960803in}}%
\pgfpathlineto{\pgfqpoint{3.315060in}{0.942299in}}%
\pgfpathlineto{\pgfqpoint{3.348037in}{0.924055in}}%
\pgfpathlineto{\pgfqpoint{3.375617in}{0.906159in}}%
\pgfpathlineto{\pgfqpoint{3.398104in}{0.888691in}}%
\pgfpathlineto{\pgfqpoint{3.415782in}{0.871715in}}%
\pgfpathlineto{\pgfqpoint{3.429135in}{0.855293in}}%
\pgfpathlineto{\pgfqpoint{3.438905in}{0.839465in}}%
\pgfpathlineto{\pgfqpoint{3.445651in}{0.824256in}}%
\pgfpathlineto{\pgfqpoint{3.449798in}{0.809687in}}%
\pgfpathlineto{\pgfqpoint{3.451638in}{0.795775in}}%
\pgfpathlineto{\pgfqpoint{3.451331in}{0.782532in}}%
\pgfpathlineto{\pgfqpoint{3.448902in}{0.769967in}}%
\pgfpathlineto{\pgfqpoint{3.444244in}{0.758085in}}%
\pgfpathlineto{\pgfqpoint{3.437115in}{0.746886in}}%
\pgfpathlineto{\pgfqpoint{3.427528in}{0.736375in}}%
\pgfpathlineto{\pgfqpoint{3.415958in}{0.726563in}}%
\pgfpathlineto{\pgfqpoint{3.402479in}{0.717454in}}%
\pgfpathlineto{\pgfqpoint{3.387130in}{0.709047in}}%
\pgfpathlineto{\pgfqpoint{3.360616in}{0.697758in}}%
\pgfpathlineto{\pgfqpoint{3.329833in}{0.688057in}}%
\pgfpathlineto{\pgfqpoint{3.294554in}{0.679945in}}%
\pgfpathlineto{\pgfqpoint{3.254414in}{0.673420in}}%
\pgfpathlineto{\pgfqpoint{3.209102in}{0.668489in}}%
\pgfpathlineto{\pgfqpoint{3.158408in}{0.665186in}}%
\pgfpathlineto{\pgfqpoint{3.101706in}{0.663551in}}%
\pgfpathlineto{\pgfqpoint{3.038362in}{0.663634in}}%
\pgfpathlineto{\pgfqpoint{2.967745in}{0.665495in}}%
\pgfpathlineto{\pgfqpoint{2.861190in}{0.670868in}}%
\pgfpathlineto{\pgfqpoint{2.739114in}{0.679733in}}%
\pgfpathlineto{\pgfqpoint{2.600063in}{0.692323in}}%
\pgfpathlineto{\pgfqpoint{2.443460in}{0.708882in}}%
\pgfpathlineto{\pgfqpoint{2.271185in}{0.729664in}}%
\pgfpathlineto{\pgfqpoint{2.134666in}{0.748059in}}%
\pgfpathlineto{\pgfqpoint{1.995610in}{0.768787in}}%
\pgfpathlineto{\pgfqpoint{1.858554in}{0.791693in}}%
\pgfpathlineto{\pgfqpoint{1.770872in}{0.808068in}}%
\pgfpathlineto{\pgfqpoint{1.687735in}{0.825183in}}%
\pgfpathlineto{\pgfqpoint{1.610215in}{0.842892in}}%
\pgfpathlineto{\pgfqpoint{1.539111in}{0.861053in}}%
\pgfpathlineto{\pgfqpoint{1.474954in}{0.879529in}}%
\pgfpathlineto{\pgfqpoint{1.417998in}{0.898186in}}%
\pgfpathlineto{\pgfqpoint{1.368229in}{0.916896in}}%
\pgfpathlineto{\pgfqpoint{1.325359in}{0.935534in}}%
\pgfpathlineto{\pgfqpoint{1.288944in}{0.953975in}}%
\pgfpathlineto{\pgfqpoint{1.258762in}{0.972094in}}%
\pgfpathlineto{\pgfqpoint{1.233767in}{0.989825in}}%
\pgfpathlineto{\pgfqpoint{1.213007in}{1.007117in}}%
\pgfpathlineto{\pgfqpoint{1.195789in}{1.023921in}}%
\pgfpathlineto{\pgfqpoint{1.181679in}{1.040196in}}%
\pgfpathlineto{\pgfqpoint{1.170499in}{1.055903in}}%
\pgfpathlineto{\pgfqpoint{1.162333in}{1.071008in}}%
\pgfpathlineto{\pgfqpoint{1.157519in}{1.085483in}}%
\pgfpathlineto{\pgfqpoint{1.156552in}{1.099302in}}%
\pgfpathlineto{\pgfqpoint{1.158432in}{1.112439in}}%
\pgfpathlineto{\pgfqpoint{1.162542in}{1.124883in}}%
\pgfpathlineto{\pgfqpoint{1.168718in}{1.136628in}}%
\pgfpathlineto{\pgfqpoint{1.176847in}{1.147671in}}%
\pgfpathlineto{\pgfqpoint{1.186859in}{1.158011in}}%
\pgfpathlineto{\pgfqpoint{1.198731in}{1.167646in}}%
\pgfpathlineto{\pgfqpoint{1.212489in}{1.176577in}}%
\pgfpathlineto{\pgfqpoint{1.228202in}{1.184809in}}%
\pgfpathlineto{\pgfqpoint{1.255590in}{1.195849in}}%
\pgfpathlineto{\pgfqpoint{1.287425in}{1.205312in}}%
\pgfpathlineto{\pgfqpoint{1.323813in}{1.213187in}}%
\pgfpathlineto{\pgfqpoint{1.364983in}{1.219457in}}%
\pgfpathlineto{\pgfqpoint{1.411261in}{1.224102in}}%
\pgfpathlineto{\pgfqpoint{1.463067in}{1.227097in}}%
\pgfpathlineto{\pgfqpoint{1.520919in}{1.228410in}}%
\pgfpathlineto{\pgfqpoint{1.585433in}{1.228007in}}%
\pgfpathlineto{\pgfqpoint{1.682514in}{1.224740in}}%
\pgfpathlineto{\pgfqpoint{1.794341in}{1.218128in}}%
\pgfpathlineto{\pgfqpoint{1.923327in}{1.207868in}}%
\pgfpathlineto{\pgfqpoint{2.070039in}{1.193723in}}%
\pgfpathlineto{\pgfqpoint{2.233179in}{1.175514in}}%
\pgfpathlineto{\pgfqpoint{2.409593in}{1.153128in}}%
\pgfpathlineto{\pgfqpoint{2.547658in}{1.133563in}}%
\pgfpathlineto{\pgfqpoint{2.686461in}{1.111639in}}%
\pgfpathlineto{\pgfqpoint{2.820510in}{1.087662in}}%
\pgfpathlineto{\pgfqpoint{2.904956in}{1.070732in}}%
\pgfpathlineto{\pgfqpoint{2.984184in}{1.053187in}}%
\pgfpathlineto{\pgfqpoint{3.057342in}{1.035149in}}%
\pgfpathlineto{\pgfqpoint{3.123790in}{1.016746in}}%
\pgfpathlineto{\pgfqpoint{3.183109in}{0.998107in}}%
\pgfpathlineto{\pgfqpoint{3.235093in}{0.979363in}}%
\pgfpathlineto{\pgfqpoint{3.279753in}{0.960650in}}%
\pgfpathlineto{\pgfqpoint{3.317354in}{0.942114in}}%
\pgfpathlineto{\pgfqpoint{3.348777in}{0.923875in}}%
\pgfpathlineto{\pgfqpoint{3.375060in}{0.906001in}}%
\pgfpathlineto{\pgfqpoint{3.397010in}{0.888554in}}%
\pgfpathlineto{\pgfqpoint{3.415198in}{0.871588in}}%
\pgfpathlineto{\pgfqpoint{3.429961in}{0.855150in}}%
\pgfpathlineto{\pgfqpoint{3.441403in}{0.839285in}}%
\pgfpathlineto{\pgfqpoint{3.449391in}{0.824028in}}%
\pgfpathlineto{\pgfqpoint{3.453595in}{0.809409in}}%
\pgfpathlineto{\pgfqpoint{3.454720in}{0.795456in}}%
\pgfpathlineto{\pgfqpoint{3.453425in}{0.782184in}}%
\pgfpathlineto{\pgfqpoint{3.449920in}{0.769602in}}%
\pgfpathlineto{\pgfqpoint{3.444360in}{0.757715in}}%
\pgfpathlineto{\pgfqpoint{3.436845in}{0.746527in}}%
\pgfpathlineto{\pgfqpoint{3.427419in}{0.736040in}}%
\pgfpathlineto{\pgfqpoint{3.416072in}{0.726255in}}%
\pgfpathlineto{\pgfqpoint{3.402736in}{0.717168in}}%
\pgfpathlineto{\pgfqpoint{3.387340in}{0.708777in}}%
\pgfpathlineto{\pgfqpoint{3.360590in}{0.697501in}}%
\pgfpathlineto{\pgfqpoint{3.329469in}{0.687805in}}%
\pgfpathlineto{\pgfqpoint{3.293861in}{0.679701in}}%
\pgfpathlineto{\pgfqpoint{3.253544in}{0.673202in}}%
\pgfpathlineto{\pgfqpoint{3.208187in}{0.668323in}}%
\pgfpathlineto{\pgfqpoint{3.157351in}{0.665084in}}%
\pgfpathlineto{\pgfqpoint{3.100490in}{0.663506in}}%
\pgfpathlineto{\pgfqpoint{3.037324in}{0.663606in}}%
\pgfpathlineto{\pgfqpoint{2.967280in}{0.665451in}}%
\pgfpathlineto{\pgfqpoint{2.860888in}{0.670827in}}%
\pgfpathlineto{\pgfqpoint{2.737823in}{0.679763in}}%
\pgfpathlineto{\pgfqpoint{2.597413in}{0.692478in}}%
\pgfpathlineto{\pgfqpoint{2.440380in}{0.709160in}}%
\pgfpathlineto{\pgfqpoint{2.268839in}{0.729969in}}%
\pgfpathlineto{\pgfqpoint{2.132722in}{0.748361in}}%
\pgfpathlineto{\pgfqpoint{1.993380in}{0.769160in}}%
\pgfpathlineto{\pgfqpoint{1.856385in}{0.792142in}}%
\pgfpathlineto{\pgfqpoint{1.768876in}{0.808513in}}%
\pgfpathlineto{\pgfqpoint{1.685879in}{0.825597in}}%
\pgfpathlineto{\pgfqpoint{1.608408in}{0.843278in}}%
\pgfpathlineto{\pgfqpoint{1.537263in}{0.861430in}}%
\pgfpathlineto{\pgfqpoint{1.473038in}{0.879920in}}%
\pgfpathlineto{\pgfqpoint{1.416114in}{0.898608in}}%
\pgfpathlineto{\pgfqpoint{1.366672in}{0.917344in}}%
\pgfpathlineto{\pgfqpoint{1.324409in}{0.935974in}}%
\pgfpathlineto{\pgfqpoint{1.288374in}{0.954398in}}%
\pgfpathlineto{\pgfqpoint{1.257728in}{0.972531in}}%
\pgfpathlineto{\pgfqpoint{1.231828in}{0.990297in}}%
\pgfpathlineto{\pgfqpoint{1.210226in}{1.007627in}}%
\pgfpathlineto{\pgfqpoint{1.192667in}{1.024462in}}%
\pgfpathlineto{\pgfqpoint{1.179090in}{1.040748in}}%
\pgfpathlineto{\pgfqpoint{1.169621in}{1.056441in}}%
\pgfpathlineto{\pgfqpoint{1.163676in}{1.071504in}}%
\pgfpathlineto{\pgfqpoint{1.160528in}{1.085914in}}%
\pgfpathlineto{\pgfqpoint{1.159877in}{1.099657in}}%
\pgfpathlineto{\pgfqpoint{1.161498in}{1.112722in}}%
\pgfpathlineto{\pgfqpoint{1.165237in}{1.125100in}}%
\pgfpathlineto{\pgfqpoint{1.171016in}{1.136784in}}%
\pgfpathlineto{\pgfqpoint{1.178828in}{1.147774in}}%
\pgfpathlineto{\pgfqpoint{1.188742in}{1.158069in}}%
\pgfpathlineto{\pgfqpoint{1.200777in}{1.167671in}}%
\pgfpathlineto{\pgfqpoint{1.214767in}{1.176574in}}%
\pgfpathlineto{\pgfqpoint{1.230662in}{1.184778in}}%
\pgfpathlineto{\pgfqpoint{1.258049in}{1.195766in}}%
\pgfpathlineto{\pgfqpoint{1.289739in}{1.205166in}}%
\pgfpathlineto{\pgfqpoint{1.325909in}{1.212973in}}%
\pgfpathlineto{\pgfqpoint{1.366864in}{1.219179in}}%
\pgfpathlineto{\pgfqpoint{1.413038in}{1.223779in}}%
\pgfpathlineto{\pgfqpoint{1.464836in}{1.226761in}}%
\pgfpathlineto{\pgfqpoint{1.522476in}{1.228082in}}%
\pgfpathlineto{\pgfqpoint{1.586849in}{1.227679in}}%
\pgfpathlineto{\pgfqpoint{1.684558in}{1.224341in}}%
\pgfpathlineto{\pgfqpoint{1.797262in}{1.217631in}}%
\pgfpathlineto{\pgfqpoint{1.926108in}{1.207340in}}%
\pgfpathlineto{\pgfqpoint{2.071815in}{1.193234in}}%
\pgfpathlineto{\pgfqpoint{2.234843in}{1.175048in}}%
\pgfpathlineto{\pgfqpoint{2.414007in}{1.152528in}}%
\pgfpathlineto{\pgfqpoint{2.552809in}{1.132854in}}%
\pgfpathlineto{\pgfqpoint{2.690286in}{1.110959in}}%
\pgfpathlineto{\pgfqpoint{2.822376in}{1.087074in}}%
\pgfpathlineto{\pgfqpoint{2.905665in}{1.070192in}}%
\pgfpathlineto{\pgfqpoint{2.984037in}{1.052667in}}%
\pgfpathlineto{\pgfqpoint{3.056662in}{1.034622in}}%
\pgfpathlineto{\pgfqpoint{3.122810in}{1.016194in}}%
\pgfpathlineto{\pgfqpoint{3.181848in}{0.997536in}}%
\pgfpathlineto{\pgfqpoint{3.233275in}{0.978813in}}%
\pgfpathlineto{\pgfqpoint{3.277529in}{0.960158in}}%
\pgfpathlineto{\pgfqpoint{3.315406in}{0.941678in}}%
\pgfpathlineto{\pgfqpoint{3.347499in}{0.923468in}}%
\pgfpathlineto{\pgfqpoint{3.374329in}{0.905618in}}%
\pgfpathlineto{\pgfqpoint{3.396347in}{0.888200in}}%
\pgfpathlineto{\pgfqpoint{3.413936in}{0.871279in}}%
\pgfpathlineto{\pgfqpoint{3.427536in}{0.854907in}}%
\pgfpathlineto{\pgfqpoint{3.437658in}{0.839121in}}%
\pgfpathlineto{\pgfqpoint{3.444676in}{0.823951in}}%
\pgfpathlineto{\pgfqpoint{3.448884in}{0.809419in}}%
\pgfpathlineto{\pgfqpoint{3.450496in}{0.795546in}}%
\pgfpathlineto{\pgfqpoint{3.449649in}{0.782343in}}%
\pgfpathlineto{\pgfqpoint{3.446411in}{0.769820in}}%
\pgfpathlineto{\pgfqpoint{3.440917in}{0.757983in}}%
\pgfpathlineto{\pgfqpoint{3.433348in}{0.746837in}}%
\pgfpathlineto{\pgfqpoint{3.423839in}{0.736387in}}%
\pgfpathlineto{\pgfqpoint{3.412481in}{0.726636in}}%
\pgfpathlineto{\pgfqpoint{3.399320in}{0.717586in}}%
\pgfpathlineto{\pgfqpoint{3.384358in}{0.709236in}}%
\pgfpathlineto{\pgfqpoint{3.358428in}{0.698022in}}%
\pgfpathlineto{\pgfqpoint{3.327999in}{0.688372in}}%
\pgfpathlineto{\pgfqpoint{3.292633in}{0.680275in}}%
\pgfpathlineto{\pgfqpoint{3.252445in}{0.673759in}}%
\pgfpathlineto{\pgfqpoint{3.207169in}{0.668848in}}%
\pgfpathlineto{\pgfqpoint{3.156391in}{0.665571in}}%
\pgfpathlineto{\pgfqpoint{3.099628in}{0.663966in}}%
\pgfpathlineto{\pgfqpoint{3.036327in}{0.664081in}}%
\pgfpathlineto{\pgfqpoint{2.965868in}{0.665971in}}%
\pgfpathlineto{\pgfqpoint{2.859588in}{0.671366in}}%
\pgfpathlineto{\pgfqpoint{2.737668in}{0.680229in}}%
\pgfpathlineto{\pgfqpoint{2.598594in}{0.692827in}}%
\pgfpathlineto{\pgfqpoint{2.442163in}{0.709403in}}%
\pgfpathlineto{\pgfqpoint{2.270073in}{0.730165in}}%
\pgfpathlineto{\pgfqpoint{2.133580in}{0.748540in}}%
\pgfpathlineto{\pgfqpoint{1.994802in}{0.769258in}}%
\pgfpathlineto{\pgfqpoint{1.858057in}{0.792150in}}%
\pgfpathlineto{\pgfqpoint{1.770381in}{0.808489in}}%
\pgfpathlineto{\pgfqpoint{1.687309in}{0.825570in}}%
\pgfpathlineto{\pgfqpoint{1.610230in}{0.843282in}}%
\pgfpathlineto{\pgfqpoint{1.539657in}{0.861461in}}%
\pgfpathlineto{\pgfqpoint{1.475869in}{0.879949in}}%
\pgfpathlineto{\pgfqpoint{1.419002in}{0.898599in}}%
\pgfpathlineto{\pgfqpoint{1.369042in}{0.917278in}}%
\pgfpathlineto{\pgfqpoint{1.325833in}{0.935867in}}%
\pgfpathlineto{\pgfqpoint{1.289071in}{0.954259in}}%
\pgfpathlineto{\pgfqpoint{1.258308in}{0.972360in}}%
\pgfpathlineto{\pgfqpoint{1.232949in}{0.990090in}}%
\pgfpathlineto{\pgfqpoint{1.212254in}{1.007381in}}%
\pgfpathlineto{\pgfqpoint{1.195929in}{1.024161in}}%
\pgfpathlineto{\pgfqpoint{1.183622in}{1.040377in}}%
\pgfpathlineto{\pgfqpoint{1.174570in}{1.056002in}}%
\pgfpathlineto{\pgfqpoint{1.168190in}{1.071012in}}%
\pgfpathlineto{\pgfqpoint{1.164080in}{1.085389in}}%
\pgfpathlineto{\pgfqpoint{1.162019in}{1.099116in}}%
\pgfpathlineto{\pgfqpoint{1.161964in}{1.112183in}}%
\pgfpathlineto{\pgfqpoint{1.164053in}{1.124582in}}%
\pgfpathlineto{\pgfqpoint{1.168606in}{1.136309in}}%
\pgfpathlineto{\pgfqpoint{1.176118in}{1.147364in}}%
\pgfpathlineto{\pgfqpoint{1.186238in}{1.157732in}}%
\pgfpathlineto{\pgfqpoint{1.198308in}{1.167397in}}%
\pgfpathlineto{\pgfqpoint{1.212272in}{1.176358in}}%
\pgfpathlineto{\pgfqpoint{1.228100in}{1.184614in}}%
\pgfpathlineto{\pgfqpoint{1.255339in}{1.195674in}}%
\pgfpathlineto{\pgfqpoint{1.286883in}{1.205144in}}%
\pgfpathlineto{\pgfqpoint{1.322977in}{1.213023in}}%
\pgfpathlineto{\pgfqpoint{1.364004in}{1.219313in}}%
\pgfpathlineto{\pgfqpoint{1.410153in}{1.224001in}}%
\pgfpathlineto{\pgfqpoint{1.461795in}{1.227053in}}%
\pgfpathlineto{\pgfqpoint{1.519550in}{1.228427in}}%
\pgfpathlineto{\pgfqpoint{1.584039in}{1.228074in}}%
\pgfpathlineto{\pgfqpoint{1.681584in}{1.224805in}}%
\pgfpathlineto{\pgfqpoint{1.793710in}{1.218173in}}%
\pgfpathlineto{\pgfqpoint{1.921921in}{1.207966in}}%
\pgfpathlineto{\pgfqpoint{2.067540in}{1.193945in}}%
\pgfpathlineto{\pgfqpoint{2.230353in}{1.175843in}}%
\pgfpathlineto{\pgfqpoint{2.407405in}{1.153478in}}%
\pgfpathlineto{\pgfqpoint{2.545781in}{1.133923in}}%
\pgfpathlineto{\pgfqpoint{2.684557in}{1.112094in}}%
\pgfpathlineto{\pgfqpoint{2.818810in}{1.088169in}}%
\pgfpathlineto{\pgfqpoint{2.903352in}{1.071238in}}%
\pgfpathlineto{\pgfqpoint{2.982519in}{1.053677in}}%
\pgfpathlineto{\pgfqpoint{3.055426in}{1.035623in}}%
\pgfpathlineto{\pgfqpoint{3.121465in}{1.017208in}}%
\pgfpathlineto{\pgfqpoint{3.180313in}{0.998568in}}%
\pgfpathlineto{\pgfqpoint{3.231928in}{0.979838in}}%
\pgfpathlineto{\pgfqpoint{3.276512in}{0.961154in}}%
\pgfpathlineto{\pgfqpoint{3.314120in}{0.942668in}}%
\pgfpathlineto{\pgfqpoint{3.345747in}{0.924469in}}%
\pgfpathlineto{\pgfqpoint{3.372446in}{0.906619in}}%
\pgfpathlineto{\pgfqpoint{3.395008in}{0.889179in}}%
\pgfpathlineto{\pgfqpoint{3.413958in}{0.872203in}}%
\pgfpathlineto{\pgfqpoint{3.429556in}{0.855741in}}%
\pgfpathlineto{\pgfqpoint{3.441797in}{0.839839in}}%
\pgfpathlineto{\pgfqpoint{3.450411in}{0.824536in}}%
\pgfpathlineto{\pgfqpoint{3.454881in}{0.809870in}}%
\pgfpathlineto{\pgfqpoint{3.455996in}{0.795872in}}%
\pgfpathlineto{\pgfqpoint{3.454691in}{0.782557in}}%
\pgfpathlineto{\pgfqpoint{3.451176in}{0.769934in}}%
\pgfpathlineto{\pgfqpoint{3.445605in}{0.758009in}}%
\pgfpathlineto{\pgfqpoint{3.438080in}{0.746784in}}%
\pgfpathlineto{\pgfqpoint{3.428647in}{0.736263in}}%
\pgfpathlineto{\pgfqpoint{3.417297in}{0.726444in}}%
\pgfpathlineto{\pgfqpoint{3.403970in}{0.717326in}}%
\pgfpathlineto{\pgfqpoint{3.388591in}{0.708905in}}%
\pgfpathlineto{\pgfqpoint{3.361875in}{0.697586in}}%
\pgfpathlineto{\pgfqpoint{3.330794in}{0.687849in}}%
\pgfpathlineto{\pgfqpoint{3.295235in}{0.679706in}}%
\pgfpathlineto{\pgfqpoint{3.254978in}{0.673168in}}%
\pgfpathlineto{\pgfqpoint{3.209691in}{0.668253in}}%
\pgfpathlineto{\pgfqpoint{3.158936in}{0.664978in}}%
\pgfpathlineto{\pgfqpoint{3.102167in}{0.663365in}}%
\pgfpathlineto{\pgfqpoint{3.039087in}{0.663430in}}%
\pgfpathlineto{\pgfqpoint{2.969155in}{0.665240in}}%
\pgfpathlineto{\pgfqpoint{2.862944in}{0.670567in}}%
\pgfpathlineto{\pgfqpoint{2.740077in}{0.679451in}}%
\pgfpathlineto{\pgfqpoint{2.599857in}{0.692113in}}%
\pgfpathlineto{\pgfqpoint{2.442983in}{0.708742in}}%
\pgfpathlineto{\pgfqpoint{2.271548in}{0.729498in}}%
\pgfpathlineto{\pgfqpoint{2.135462in}{0.747851in}}%
\pgfpathlineto{\pgfqpoint{1.996064in}{0.768614in}}%
\pgfpathlineto{\pgfqpoint{1.858918in}{0.791567in}}%
\pgfpathlineto{\pgfqpoint{1.729052in}{0.816375in}}%
\pgfpathlineto{\pgfqpoint{1.648523in}{0.833762in}}%
\pgfpathlineto{\pgfqpoint{1.573927in}{0.851688in}}%
\pgfpathlineto{\pgfqpoint{1.505970in}{0.870023in}}%
\pgfpathlineto{\pgfqpoint{1.445144in}{0.888631in}}%
\pgfpathlineto{\pgfqpoint{1.391734in}{0.907369in}}%
\pgfpathlineto{\pgfqpoint{1.345780in}{0.926076in}}%
\pgfpathlineto{\pgfqpoint{1.306569in}{0.944626in}}%
\pgfpathlineto{\pgfqpoint{1.273166in}{0.962928in}}%
\pgfpathlineto{\pgfqpoint{1.244825in}{0.980903in}}%
\pgfpathlineto{\pgfqpoint{1.220998in}{0.998477in}}%
\pgfpathlineto{\pgfqpoint{1.201327in}{1.015585in}}%
\pgfpathlineto{\pgfqpoint{1.185651in}{1.032171in}}%
\pgfpathlineto{\pgfqpoint{1.174004in}{1.048187in}}%
\pgfpathlineto{\pgfqpoint{1.166360in}{1.063590in}}%
\pgfpathlineto{\pgfqpoint{1.161778in}{1.078350in}}%
\pgfpathlineto{\pgfqpoint{1.159841in}{1.092451in}}%
\pgfpathlineto{\pgfqpoint{1.160284in}{1.105878in}}%
\pgfpathlineto{\pgfqpoint{1.162917in}{1.118622in}}%
\pgfpathlineto{\pgfqpoint{1.167622in}{1.130676in}}%
\pgfpathlineto{\pgfqpoint{1.174356in}{1.142036in}}%
\pgfpathlineto{\pgfqpoint{1.183149in}{1.152700in}}%
\pgfpathlineto{\pgfqpoint{1.194087in}{1.162671in}}%
\pgfpathlineto{\pgfqpoint{1.207053in}{1.171946in}}%
\pgfpathlineto{\pgfqpoint{1.221944in}{1.180522in}}%
\pgfpathlineto{\pgfqpoint{1.247830in}{1.192070in}}%
\pgfpathlineto{\pgfqpoint{1.277986in}{1.202033in}}%
\pgfpathlineto{\pgfqpoint{1.312544in}{1.210404in}}%
\pgfpathlineto{\pgfqpoint{1.351764in}{1.217178in}}%
\pgfpathlineto{\pgfqpoint{1.396038in}{1.222346in}}%
\pgfpathlineto{\pgfqpoint{1.445849in}{1.225905in}}%
\pgfpathlineto{\pgfqpoint{1.501332in}{1.227820in}}%
\pgfpathlineto{\pgfqpoint{1.563231in}{1.228034in}}%
\pgfpathlineto{\pgfqpoint{1.632411in}{1.226480in}}%
\pgfpathlineto{\pgfqpoint{1.737224in}{1.221532in}}%
\pgfpathlineto{\pgfqpoint{1.857633in}{1.213114in}}%
\pgfpathlineto{\pgfqpoint{1.994586in}{1.201004in}}%
\pgfpathlineto{\pgfqpoint{2.148609in}{1.184956in}}%
\pgfpathlineto{\pgfqpoint{2.320298in}{1.164681in}}%
\pgfpathlineto{\pgfqpoint{2.457403in}{1.146626in}}%
\pgfpathlineto{\pgfqpoint{2.596238in}{1.126236in}}%
\pgfpathlineto{\pgfqpoint{2.732413in}{1.103691in}}%
\pgfpathlineto{\pgfqpoint{2.862036in}{1.079241in}}%
\pgfpathlineto{\pgfqpoint{2.943116in}{1.062041in}}%
\pgfpathlineto{\pgfqpoint{3.018883in}{1.044255in}}%
\pgfpathlineto{\pgfqpoint{3.088555in}{1.026011in}}%
\pgfpathlineto{\pgfqpoint{3.151447in}{1.007453in}}%
\pgfpathlineto{\pgfqpoint{3.206972in}{0.988742in}}%
\pgfpathlineto{\pgfqpoint{3.254905in}{0.970038in}}%
\pgfpathlineto{\pgfqpoint{3.296089in}{0.951452in}}%
\pgfpathlineto{\pgfqpoint{3.331179in}{0.933087in}}%
\pgfpathlineto{\pgfqpoint{3.360733in}{0.915036in}}%
\pgfpathlineto{\pgfqpoint{3.385242in}{0.897380in}}%
\pgfpathlineto{\pgfqpoint{3.405126in}{0.880189in}}%
\pgfpathlineto{\pgfqpoint{3.420766in}{0.863520in}}%
\pgfpathlineto{\pgfqpoint{3.432684in}{0.847420in}}%
\pgfpathlineto{\pgfqpoint{3.441310in}{0.831920in}}%
\pgfpathlineto{\pgfqpoint{3.446979in}{0.817047in}}%
\pgfpathlineto{\pgfqpoint{3.449946in}{0.802822in}}%
\pgfpathlineto{\pgfqpoint{3.450388in}{0.789263in}}%
\pgfpathlineto{\pgfqpoint{3.448400in}{0.776379in}}%
\pgfpathlineto{\pgfqpoint{3.444070in}{0.764177in}}%
\pgfpathlineto{\pgfqpoint{3.437580in}{0.752664in}}%
\pgfpathlineto{\pgfqpoint{3.429092in}{0.741845in}}%
\pgfpathlineto{\pgfqpoint{3.418717in}{0.731723in}}%
\pgfpathlineto{\pgfqpoint{3.406524in}{0.722301in}}%
\pgfpathlineto{\pgfqpoint{3.392535in}{0.713580in}}%
\pgfpathlineto{\pgfqpoint{3.376728in}{0.705559in}}%
\pgfpathlineto{\pgfqpoint{3.349443in}{0.694836in}}%
\pgfpathlineto{\pgfqpoint{3.317477in}{0.685670in}}%
\pgfpathlineto{\pgfqpoint{3.280585in}{0.678064in}}%
\pgfpathlineto{\pgfqpoint{3.238844in}{0.672047in}}%
\pgfpathlineto{\pgfqpoint{3.191891in}{0.667643in}}%
\pgfpathlineto{\pgfqpoint{3.139294in}{0.664886in}}%
\pgfpathlineto{\pgfqpoint{3.080549in}{0.663814in}}%
\pgfpathlineto{\pgfqpoint{3.015082in}{0.664479in}}%
\pgfpathlineto{\pgfqpoint{2.916213in}{0.668167in}}%
\pgfpathlineto{\pgfqpoint{2.802563in}{0.675211in}}%
\pgfpathlineto{\pgfqpoint{2.672468in}{0.685852in}}%
\pgfpathlineto{\pgfqpoint{2.524884in}{0.700349in}}%
\pgfpathlineto{\pgfqpoint{2.360536in}{0.718932in}}%
\pgfpathlineto{\pgfqpoint{2.182325in}{0.741765in}}%
\pgfpathlineto{\pgfqpoint{2.043883in}{0.761668in}}%
\pgfpathlineto{\pgfqpoint{1.905921in}{0.783817in}}%
\pgfpathlineto{\pgfqpoint{1.773036in}{0.807975in}}%
\pgfpathlineto{\pgfqpoint{1.689742in}{0.825032in}}%
\pgfpathlineto{\pgfqpoint{1.612490in}{0.842722in}}%
\pgfpathlineto{\pgfqpoint{1.541835in}{0.860889in}}%
\pgfpathlineto{\pgfqpoint{1.477967in}{0.879371in}}%
\pgfpathlineto{\pgfqpoint{1.420961in}{0.898021in}}%
\pgfpathlineto{\pgfqpoint{1.370774in}{0.916704in}}%
\pgfpathlineto{\pgfqpoint{1.327246in}{0.935299in}}%
\pgfpathlineto{\pgfqpoint{1.290099in}{0.953698in}}%
\pgfpathlineto{\pgfqpoint{1.258940in}{0.971808in}}%
\pgfpathlineto{\pgfqpoint{1.233257in}{0.989549in}}%
\pgfpathlineto{\pgfqpoint{1.212423in}{1.006853in}}%
\pgfpathlineto{\pgfqpoint{1.195837in}{1.023661in}}%
\pgfpathlineto{\pgfqpoint{1.183396in}{1.039905in}}%
\pgfpathlineto{\pgfqpoint{1.174401in}{1.055553in}}%
\pgfpathlineto{\pgfqpoint{1.168209in}{1.070584in}}%
\pgfpathlineto{\pgfqpoint{1.164352in}{1.084978in}}%
\pgfpathlineto{\pgfqpoint{1.162530in}{1.098722in}}%
\pgfpathlineto{\pgfqpoint{1.162616in}{1.111803in}}%
\pgfpathlineto{\pgfqpoint{1.164652in}{1.124214in}}%
\pgfpathlineto{\pgfqpoint{1.168854in}{1.135952in}}%
\pgfpathlineto{\pgfqpoint{1.175608in}{1.147015in}}%
\pgfpathlineto{\pgfqpoint{1.185364in}{1.157406in}}%
\pgfpathlineto{\pgfqpoint{1.197371in}{1.167102in}}%
\pgfpathlineto{\pgfqpoint{1.211292in}{1.176094in}}%
\pgfpathlineto{\pgfqpoint{1.227094in}{1.184381in}}%
\pgfpathlineto{\pgfqpoint{1.254312in}{1.195489in}}%
\pgfpathlineto{\pgfqpoint{1.285829in}{1.205004in}}%
\pgfpathlineto{\pgfqpoint{1.321852in}{1.212925in}}%
\pgfpathlineto{\pgfqpoint{1.362720in}{1.219249in}}%
\pgfpathlineto{\pgfqpoint{1.408802in}{1.223972in}}%
\pgfpathlineto{\pgfqpoint{1.460281in}{1.227063in}}%
\pgfpathlineto{\pgfqpoint{1.517818in}{1.228480in}}%
\pgfpathlineto{\pgfqpoint{1.582107in}{1.228171in}}%
\pgfpathlineto{\pgfqpoint{1.679464in}{1.224963in}}%
\pgfpathlineto{\pgfqpoint{1.791464in}{1.218393in}}%
\pgfpathlineto{\pgfqpoint{1.919470in}{1.208251in}}%
\pgfpathlineto{\pgfqpoint{2.064758in}{1.194291in}}%
\pgfpathlineto{\pgfqpoint{2.228080in}{1.176300in}}%
\pgfpathlineto{\pgfqpoint{2.360000in}{1.160004in}}%
\pgfpathlineto{\pgfqpoint{2.496868in}{1.141272in}}%
\pgfpathlineto{\pgfqpoint{2.635281in}{1.120190in}}%
\pgfpathlineto{\pgfqpoint{2.771300in}{1.096953in}}%
\pgfpathlineto{\pgfqpoint{2.900448in}{1.071861in}}%
\pgfpathlineto{\pgfqpoint{2.980277in}{1.054300in}}%
\pgfpathlineto{\pgfqpoint{3.053271in}{1.036239in}}%
\pgfpathlineto{\pgfqpoint{3.118827in}{1.017818in}}%
\pgfpathlineto{\pgfqpoint{3.177266in}{0.999182in}}%
\pgfpathlineto{\pgfqpoint{3.228937in}{0.980467in}}%
\pgfpathlineto{\pgfqpoint{3.274181in}{0.961796in}}%
\pgfpathlineto{\pgfqpoint{3.313334in}{0.943280in}}%
\pgfpathlineto{\pgfqpoint{3.346720in}{0.925019in}}%
\pgfpathlineto{\pgfqpoint{3.374659in}{0.907103in}}%
\pgfpathlineto{\pgfqpoint{3.397462in}{0.889607in}}%
\pgfpathlineto{\pgfqpoint{3.415435in}{0.872599in}}%
\pgfpathlineto{\pgfqpoint{3.428966in}{0.856138in}}%
\pgfpathlineto{\pgfqpoint{3.438772in}{0.840273in}}%
\pgfpathlineto{\pgfqpoint{3.445494in}{0.825028in}}%
\pgfpathlineto{\pgfqpoint{3.449627in}{0.810423in}}%
\pgfpathlineto{\pgfqpoint{3.451522in}{0.796473in}}%
\pgfpathlineto{\pgfqpoint{3.451384in}{0.783192in}}%
\pgfpathlineto{\pgfqpoint{3.449274in}{0.770588in}}%
\pgfpathlineto{\pgfqpoint{3.445108in}{0.758665in}}%
\pgfpathlineto{\pgfqpoint{3.438660in}{0.747424in}}%
\pgfpathlineto{\pgfqpoint{3.429560in}{0.736864in}}%
\pgfpathlineto{\pgfqpoint{3.418064in}{0.726994in}}%
\pgfpathlineto{\pgfqpoint{3.404637in}{0.717826in}}%
\pgfpathlineto{\pgfqpoint{3.389321in}{0.709362in}}%
\pgfpathlineto{\pgfqpoint{3.362833in}{0.697989in}}%
\pgfpathlineto{\pgfqpoint{3.332073in}{0.688207in}}%
\pgfpathlineto{\pgfqpoint{3.296857in}{0.680019in}}%
\pgfpathlineto{\pgfqpoint{3.256872in}{0.673428in}}%
\pgfpathlineto{\pgfqpoint{3.211701in}{0.668435in}}%
\pgfpathlineto{\pgfqpoint{3.161188in}{0.665066in}}%
\pgfpathlineto{\pgfqpoint{3.104761in}{0.663361in}}%
\pgfpathlineto{\pgfqpoint{3.041697in}{0.663373in}}%
\pgfpathlineto{\pgfqpoint{2.971320in}{0.665162in}}%
\pgfpathlineto{\pgfqpoint{2.865014in}{0.670438in}}%
\pgfpathlineto{\pgfqpoint{2.743196in}{0.679204in}}%
\pgfpathlineto{\pgfqpoint{2.604611in}{0.691691in}}%
\pgfpathlineto{\pgfqpoint{2.448081in}{0.708139in}}%
\pgfpathlineto{\pgfqpoint{2.275457in}{0.728777in}}%
\pgfpathlineto{\pgfqpoint{2.139186in}{0.747093in}}%
\pgfpathlineto{\pgfqpoint{2.000753in}{0.767776in}}%
\pgfpathlineto{\pgfqpoint{1.864027in}{0.790652in}}%
\pgfpathlineto{\pgfqpoint{1.733404in}{0.815441in}}%
\pgfpathlineto{\pgfqpoint{1.652123in}{0.832844in}}%
\pgfpathlineto{\pgfqpoint{1.577311in}{0.850786in}}%
\pgfpathlineto{\pgfqpoint{1.509915in}{0.869126in}}%
\pgfpathlineto{\pgfqpoint{1.449698in}{0.887717in}}%
\pgfpathlineto{\pgfqpoint{1.396317in}{0.906422in}}%
\pgfpathlineto{\pgfqpoint{1.349436in}{0.925113in}}%
\pgfpathlineto{\pgfqpoint{1.308725in}{0.943676in}}%
\pgfpathlineto{\pgfqpoint{1.273858in}{0.962008in}}%
\pgfpathlineto{\pgfqpoint{1.244518in}{0.980019in}}%
\pgfpathlineto{\pgfqpoint{1.220394in}{0.997627in}}%
\pgfpathlineto{\pgfqpoint{1.201181in}{1.014765in}}%
\pgfpathlineto{\pgfqpoint{1.186550in}{1.031373in}}%
\pgfpathlineto{\pgfqpoint{1.175871in}{1.047394in}}%
\pgfpathlineto{\pgfqpoint{1.168447in}{1.062800in}}%
\pgfpathlineto{\pgfqpoint{1.163731in}{1.077570in}}%
\pgfpathlineto{\pgfqpoint{1.161322in}{1.091688in}}%
\pgfpathlineto{\pgfqpoint{1.160966in}{1.105141in}}%
\pgfpathlineto{\pgfqpoint{1.162553in}{1.117919in}}%
\pgfpathlineto{\pgfqpoint{1.166122in}{1.130017in}}%
\pgfpathlineto{\pgfqpoint{1.171854in}{1.141432in}}%
\pgfpathlineto{\pgfqpoint{1.180080in}{1.152167in}}%
\pgfpathlineto{\pgfqpoint{1.191002in}{1.162219in}}%
\pgfpathlineto{\pgfqpoint{1.203945in}{1.171572in}}%
\pgfpathlineto{\pgfqpoint{1.218789in}{1.180220in}}%
\pgfpathlineto{\pgfqpoint{1.244576in}{1.191871in}}%
\pgfpathlineto{\pgfqpoint{1.274622in}{1.201931in}}%
\pgfpathlineto{\pgfqpoint{1.309081in}{1.210397in}}%
\pgfpathlineto{\pgfqpoint{1.348235in}{1.217265in}}%
\pgfpathlineto{\pgfqpoint{1.392493in}{1.222532in}}%
\pgfpathlineto{\pgfqpoint{1.442090in}{1.226184in}}%
\pgfpathlineto{\pgfqpoint{1.497441in}{1.228179in}}%
\pgfpathlineto{\pgfqpoint{1.559302in}{1.228467in}}%
\pgfpathlineto{\pgfqpoint{1.628377in}{1.226989in}}%
\pgfpathlineto{\pgfqpoint{1.732815in}{1.222146in}}%
\pgfpathlineto{\pgfqpoint{1.852617in}{1.213842in}}%
\pgfpathlineto{\pgfqpoint{1.989006in}{1.201847in}}%
\pgfpathlineto{\pgfqpoint{2.143209in}{1.185915in}}%
\pgfpathlineto{\pgfqpoint{2.314687in}{1.165832in}}%
\pgfpathlineto{\pgfqpoint{2.450516in}{1.147938in}}%
\pgfpathlineto{\pgfqpoint{2.588671in}{1.127657in}}%
\pgfpathlineto{\pgfqpoint{2.725427in}{1.105138in}}%
\pgfpathlineto{\pgfqpoint{2.856828in}{1.080638in}}%
\pgfpathlineto{\pgfqpoint{2.939397in}{1.063381in}}%
\pgfpathlineto{\pgfqpoint{3.016435in}{1.045536in}}%
\pgfpathlineto{\pgfqpoint{3.086600in}{1.027261in}}%
\pgfpathlineto{\pgfqpoint{3.149014in}{1.008711in}}%
\pgfpathlineto{\pgfqpoint{3.204105in}{0.990014in}}%
\pgfpathlineto{\pgfqpoint{3.252355in}{0.971302in}}%
\pgfpathlineto{\pgfqpoint{3.294210in}{0.952690in}}%
\pgfpathlineto{\pgfqpoint{3.330088in}{0.934287in}}%
\pgfpathlineto{\pgfqpoint{3.360378in}{0.916187in}}%
\pgfpathlineto{\pgfqpoint{3.385437in}{0.898476in}}%
\pgfpathlineto{\pgfqpoint{3.405592in}{0.881225in}}%
\pgfpathlineto{\pgfqpoint{3.421257in}{0.864499in}}%
\pgfpathlineto{\pgfqpoint{3.433134in}{0.848344in}}%
\pgfpathlineto{\pgfqpoint{3.441755in}{0.832791in}}%
\pgfpathlineto{\pgfqpoint{3.447530in}{0.817864in}}%
\pgfpathlineto{\pgfqpoint{3.450751in}{0.803585in}}%
\pgfpathlineto{\pgfqpoint{3.451591in}{0.789968in}}%
\pgfpathlineto{\pgfqpoint{3.450106in}{0.777026in}}%
\pgfpathlineto{\pgfqpoint{3.446234in}{0.764765in}}%
\pgfpathlineto{\pgfqpoint{3.439910in}{0.753190in}}%
\pgfpathlineto{\pgfqpoint{3.431466in}{0.742308in}}%
\pgfpathlineto{\pgfqpoint{3.421044in}{0.732124in}}%
\pgfpathlineto{\pgfqpoint{3.408726in}{0.722640in}}%
\pgfpathlineto{\pgfqpoint{3.394562in}{0.713858in}}%
\pgfpathlineto{\pgfqpoint{3.378564in}{0.705780in}}%
\pgfpathlineto{\pgfqpoint{3.351079in}{0.694980in}}%
\pgfpathlineto{\pgfqpoint{3.319207in}{0.685760in}}%
\pgfpathlineto{\pgfqpoint{3.282567in}{0.678111in}}%
\pgfpathlineto{\pgfqpoint{3.240939in}{0.672042in}}%
\pgfpathlineto{\pgfqpoint{3.194165in}{0.667583in}}%
\pgfpathlineto{\pgfqpoint{3.141751in}{0.664764in}}%
\pgfpathlineto{\pgfqpoint{3.083165in}{0.663628in}}%
\pgfpathlineto{\pgfqpoint{3.017827in}{0.664228in}}%
\pgfpathlineto{\pgfqpoint{2.919127in}{0.667834in}}%
\pgfpathlineto{\pgfqpoint{2.805726in}{0.674813in}}%
\pgfpathlineto{\pgfqpoint{2.675968in}{0.685375in}}%
\pgfpathlineto{\pgfqpoint{2.528748in}{0.699786in}}%
\pgfpathlineto{\pgfqpoint{2.364586in}{0.718288in}}%
\pgfpathlineto{\pgfqpoint{2.186634in}{0.741035in}}%
\pgfpathlineto{\pgfqpoint{2.048065in}{0.760877in}}%
\pgfpathlineto{\pgfqpoint{1.909888in}{0.782974in}}%
\pgfpathlineto{\pgfqpoint{1.776754in}{0.807090in}}%
\pgfpathlineto{\pgfqpoint{1.693196in}{0.824125in}}%
\pgfpathlineto{\pgfqpoint{1.615589in}{0.841796in}}%
\pgfpathlineto{\pgfqpoint{1.544590in}{0.859951in}}%
\pgfpathlineto{\pgfqpoint{1.480394in}{0.878430in}}%
\pgfpathlineto{\pgfqpoint{1.423079in}{0.897083in}}%
\pgfpathlineto{\pgfqpoint{1.372603in}{0.915776in}}%
\pgfpathlineto{\pgfqpoint{1.328808in}{0.934387in}}%
\pgfpathlineto{\pgfqpoint{1.291414in}{0.952808in}}%
\pgfpathlineto{\pgfqpoint{1.260025in}{0.970945in}}%
\pgfpathlineto{\pgfqpoint{1.234128in}{0.988715in}}%
\pgfpathlineto{\pgfqpoint{1.213089in}{1.006050in}}%
\pgfpathlineto{\pgfqpoint{1.196375in}{1.022886in}}%
\pgfpathlineto{\pgfqpoint{1.183839in}{1.039158in}}%
\pgfpathlineto{\pgfqpoint{1.174718in}{1.054836in}}%
\pgfpathlineto{\pgfqpoint{1.168365in}{1.069898in}}%
\pgfpathlineto{\pgfqpoint{1.164314in}{1.084326in}}%
\pgfpathlineto{\pgfqpoint{1.162273in}{1.098104in}}%
\pgfpathlineto{\pgfqpoint{1.162134in}{1.111222in}}%
\pgfpathlineto{\pgfqpoint{1.163964in}{1.123672in}}%
\pgfpathlineto{\pgfqpoint{1.168010in}{1.135449in}}%
\pgfpathlineto{\pgfqpoint{1.174696in}{1.146553in}}%
\pgfpathlineto{\pgfqpoint{1.184447in}{1.156984in}}%
\pgfpathlineto{\pgfqpoint{1.196390in}{1.166718in}}%
\pgfpathlineto{\pgfqpoint{1.210244in}{1.175747in}}%
\pgfpathlineto{\pgfqpoint{1.225974in}{1.184071in}}%
\pgfpathlineto{\pgfqpoint{1.253077in}{1.195231in}}%
\pgfpathlineto{\pgfqpoint{1.284471in}{1.204799in}}%
\pgfpathlineto{\pgfqpoint{1.320367in}{1.212772in}}%
\pgfpathlineto{\pgfqpoint{1.361108in}{1.219149in}}%
\pgfpathlineto{\pgfqpoint{1.407044in}{1.223925in}}%
\pgfpathlineto{\pgfqpoint{1.458371in}{1.227068in}}%
\pgfpathlineto{\pgfqpoint{1.515751in}{1.228538in}}%
\pgfpathlineto{\pgfqpoint{1.579858in}{1.228282in}}%
\pgfpathlineto{\pgfqpoint{1.676917in}{1.225148in}}%
\pgfpathlineto{\pgfqpoint{1.788564in}{1.218656in}}%
\pgfpathlineto{\pgfqpoint{1.916197in}{1.208596in}}%
\pgfpathlineto{\pgfqpoint{2.061147in}{1.194725in}}%
\pgfpathlineto{\pgfqpoint{2.223755in}{1.176815in}}%
\pgfpathlineto{\pgfqpoint{2.400427in}{1.154621in}}%
\pgfpathlineto{\pgfqpoint{2.538365in}{1.135158in}}%
\pgfpathlineto{\pgfqpoint{2.676975in}{1.113407in}}%
\pgfpathlineto{\pgfqpoint{2.811806in}{1.089591in}}%
\pgfpathlineto{\pgfqpoint{2.897037in}{1.072723in}}%
\pgfpathlineto{\pgfqpoint{2.976574in}{1.055193in}}%
\pgfpathlineto{\pgfqpoint{3.049487in}{1.037133in}}%
\pgfpathlineto{\pgfqpoint{3.115580in}{1.018704in}}%
\pgfpathlineto{\pgfqpoint{3.174781in}{1.000061in}}%
\pgfpathlineto{\pgfqpoint{3.227125in}{0.981340in}}%
\pgfpathlineto{\pgfqpoint{3.272754in}{0.962669in}}%
\pgfpathlineto{\pgfqpoint{3.311919in}{0.944159in}}%
\pgfpathlineto{\pgfqpoint{3.344976in}{0.925909in}}%
\pgfpathlineto{\pgfqpoint{3.372389in}{0.908006in}}%
\pgfpathlineto{\pgfqpoint{3.394732in}{0.890522in}}%
\pgfpathlineto{\pgfqpoint{3.412611in}{0.873520in}}%
\pgfpathlineto{\pgfqpoint{3.426151in}{0.857072in}}%
\pgfpathlineto{\pgfqpoint{3.436055in}{0.841212in}}%
\pgfpathlineto{\pgfqpoint{3.443044in}{0.825965in}}%
\pgfpathlineto{\pgfqpoint{3.447652in}{0.811347in}}%
\pgfpathlineto{\pgfqpoint{3.450231in}{0.797375in}}%
\pgfpathlineto{\pgfqpoint{3.450948in}{0.784061in}}%
\pgfpathlineto{\pgfqpoint{3.449787in}{0.771414in}}%
\pgfpathlineto{\pgfqpoint{3.446545in}{0.759437in}}%
\pgfpathlineto{\pgfqpoint{3.440838in}{0.748133in}}%
\pgfpathlineto{\pgfqpoint{3.432096in}{0.737500in}}%
\pgfpathlineto{\pgfqpoint{3.420661in}{0.727553in}}%
\pgfpathlineto{\pgfqpoint{3.407302in}{0.718311in}}%
\pgfpathlineto{\pgfqpoint{3.392059in}{0.709776in}}%
\pgfpathlineto{\pgfqpoint{3.365692in}{0.698300in}}%
\pgfpathlineto{\pgfqpoint{3.335062in}{0.688418in}}%
\pgfpathlineto{\pgfqpoint{3.299986in}{0.680133in}}%
\pgfpathlineto{\pgfqpoint{3.260153in}{0.673446in}}%
\pgfpathlineto{\pgfqpoint{3.215167in}{0.668361in}}%
\pgfpathlineto{\pgfqpoint{3.164870in}{0.664902in}}%
\pgfpathlineto{\pgfqpoint{3.108667in}{0.663108in}}%
\pgfpathlineto{\pgfqpoint{3.045861in}{0.663030in}}%
\pgfpathlineto{\pgfqpoint{2.975787in}{0.664729in}}%
\pgfpathlineto{\pgfqpoint{2.869957in}{0.669882in}}%
\pgfpathlineto{\pgfqpoint{2.748666in}{0.678518in}}%
\pgfpathlineto{\pgfqpoint{2.610598in}{0.690871in}}%
\pgfpathlineto{\pgfqpoint{2.454517in}{0.707172in}}%
\pgfpathlineto{\pgfqpoint{2.282348in}{0.727666in}}%
\pgfpathlineto{\pgfqpoint{2.146209in}{0.745882in}}%
\pgfpathlineto{\pgfqpoint{2.007630in}{0.766474in}}%
\pgfpathlineto{\pgfqpoint{1.870487in}{0.789271in}}%
\pgfpathlineto{\pgfqpoint{1.739263in}{0.813996in}}%
\pgfpathlineto{\pgfqpoint{1.657558in}{0.831364in}}%
\pgfpathlineto{\pgfqpoint{1.582415in}{0.849283in}}%
\pgfpathlineto{\pgfqpoint{1.514814in}{0.867605in}}%
\pgfpathlineto{\pgfqpoint{1.454378in}{0.886186in}}%
\pgfpathlineto{\pgfqpoint{1.400668in}{0.904887in}}%
\pgfpathlineto{\pgfqpoint{1.353284in}{0.923586in}}%
\pgfpathlineto{\pgfqpoint{1.311868in}{0.942167in}}%
\pgfpathlineto{\pgfqpoint{1.276102in}{0.960529in}}%
\pgfpathlineto{\pgfqpoint{1.245706in}{0.978580in}}%
\pgfpathlineto{\pgfqpoint{1.220443in}{0.996243in}}%
\pgfpathlineto{\pgfqpoint{1.200115in}{1.013447in}}%
\pgfpathlineto{\pgfqpoint{1.184564in}{1.030137in}}%
\pgfpathlineto{\pgfqpoint{1.173539in}{1.046260in}}%
\pgfpathlineto{\pgfqpoint{1.166151in}{1.061760in}}%
\pgfpathlineto{\pgfqpoint{1.161719in}{1.076616in}}%
\pgfpathlineto{\pgfqpoint{1.159740in}{1.090813in}}%
\pgfpathlineto{\pgfqpoint{1.159842in}{1.104337in}}%
\pgfpathlineto{\pgfqpoint{1.161793in}{1.117181in}}%
\pgfpathlineto{\pgfqpoint{1.165494in}{1.129339in}}%
\pgfpathlineto{\pgfqpoint{1.170984in}{1.140808in}}%
\pgfpathlineto{\pgfqpoint{1.178436in}{1.151589in}}%
\pgfpathlineto{\pgfqpoint{1.188160in}{1.161687in}}%
\pgfpathlineto{\pgfqpoint{1.200591in}{1.171107in}}%
\pgfpathlineto{\pgfqpoint{1.215366in}{1.179835in}}%
\pgfpathlineto{\pgfqpoint{1.232048in}{1.187859in}}%
\pgfpathlineto{\pgfqpoint{1.260640in}{1.198568in}}%
\pgfpathlineto{\pgfqpoint{1.293591in}{1.207680in}}%
\pgfpathlineto{\pgfqpoint{1.331090in}{1.215189in}}%
\pgfpathlineto{\pgfqpoint{1.373435in}{1.221086in}}%
\pgfpathlineto{\pgfqpoint{1.421037in}{1.225358in}}%
\pgfpathlineto{\pgfqpoint{1.474305in}{1.227995in}}%
\pgfpathlineto{\pgfqpoint{1.533456in}{1.228957in}}%
\pgfpathlineto{\pgfqpoint{1.599477in}{1.228179in}}%
\pgfpathlineto{\pgfqpoint{1.699732in}{1.224303in}}%
\pgfpathlineto{\pgfqpoint{1.815413in}{1.217005in}}%
\pgfpathlineto{\pgfqpoint{1.947536in}{1.206075in}}%
\pgfpathlineto{\pgfqpoint{2.096509in}{1.191284in}}%
\pgfpathlineto{\pgfqpoint{2.262148in}{1.172384in}}%
\pgfpathlineto{\pgfqpoint{2.443070in}{1.149106in}}%
\pgfpathlineto{\pgfqpoint{2.582757in}{1.128858in}}%
\pgfpathlineto{\pgfqpoint{2.720298in}{1.106430in}}%
\pgfpathlineto{\pgfqpoint{2.851413in}{1.082084in}}%
\pgfpathlineto{\pgfqpoint{2.933447in}{1.064946in}}%
\pgfpathlineto{\pgfqpoint{3.010112in}{1.047211in}}%
\pgfpathlineto{\pgfqpoint{3.080642in}{1.029003in}}%
\pgfpathlineto{\pgfqpoint{3.144406in}{1.010461in}}%
\pgfpathlineto{\pgfqpoint{3.200907in}{0.991733in}}%
\pgfpathlineto{\pgfqpoint{3.249847in}{0.972981in}}%
\pgfpathlineto{\pgfqpoint{3.291814in}{0.954337in}}%
\pgfpathlineto{\pgfqpoint{3.327600in}{0.935904in}}%
\pgfpathlineto{\pgfqpoint{3.357842in}{0.917774in}}%
\pgfpathlineto{\pgfqpoint{3.383073in}{0.900029in}}%
\pgfpathlineto{\pgfqpoint{3.403712in}{0.882738in}}%
\pgfpathlineto{\pgfqpoint{3.420073in}{0.865962in}}%
\pgfpathlineto{\pgfqpoint{3.432438in}{0.849749in}}%
\pgfpathlineto{\pgfqpoint{3.441391in}{0.834138in}}%
\pgfpathlineto{\pgfqpoint{3.447333in}{0.819154in}}%
\pgfpathlineto{\pgfqpoint{3.450563in}{0.804818in}}%
\pgfpathlineto{\pgfqpoint{3.451303in}{0.791146in}}%
\pgfpathlineto{\pgfqpoint{3.449698in}{0.778150in}}%
\pgfpathlineto{\pgfqpoint{3.445817in}{0.765839in}}%
\pgfpathlineto{\pgfqpoint{3.439684in}{0.754216in}}%
\pgfpathlineto{\pgfqpoint{3.431434in}{0.743285in}}%
\pgfpathlineto{\pgfqpoint{3.421198in}{0.733050in}}%
\pgfpathlineto{\pgfqpoint{3.409071in}{0.723514in}}%
\pgfpathlineto{\pgfqpoint{3.395109in}{0.714680in}}%
\pgfpathlineto{\pgfqpoint{3.379333in}{0.706548in}}%
\pgfpathlineto{\pgfqpoint{3.352221in}{0.695668in}}%
\pgfpathlineto{\pgfqpoint{3.320769in}{0.686366in}}%
\pgfpathlineto{\pgfqpoint{3.284578in}{0.678634in}}%
\pgfpathlineto{\pgfqpoint{3.243286in}{0.672471in}}%
\pgfpathlineto{\pgfqpoint{3.196872in}{0.667914in}}%
\pgfpathlineto{\pgfqpoint{3.144857in}{0.664995in}}%
\pgfpathlineto{\pgfqpoint{3.086700in}{0.663754in}}%
\pgfpathlineto{\pgfqpoint{3.021822in}{0.664243in}}%
\pgfpathlineto{\pgfqpoint{2.923790in}{0.667695in}}%
\pgfpathlineto{\pgfqpoint{2.811138in}{0.674509in}}%
\pgfpathlineto{\pgfqpoint{2.682215in}{0.684897in}}%
\pgfpathlineto{\pgfqpoint{2.535876in}{0.699118in}}%
\pgfpathlineto{\pgfqpoint{2.372494in}{0.717422in}}%
\pgfpathlineto{\pgfqpoint{2.195156in}{0.739961in}}%
\pgfpathlineto{\pgfqpoint{2.056711in}{0.759653in}}%
\pgfpathlineto{\pgfqpoint{1.918404in}{0.781610in}}%
\pgfpathlineto{\pgfqpoint{1.784903in}{0.805602in}}%
\pgfpathlineto{\pgfqpoint{1.700873in}{0.822566in}}%
\pgfpathlineto{\pgfqpoint{1.622555in}{0.840166in}}%
\pgfpathlineto{\pgfqpoint{1.551364in}{0.858265in}}%
\pgfpathlineto{\pgfqpoint{1.487198in}{0.876704in}}%
\pgfpathlineto{\pgfqpoint{1.429792in}{0.895335in}}%
\pgfpathlineto{\pgfqpoint{1.378876in}{0.914022in}}%
\pgfpathlineto{\pgfqpoint{1.334177in}{0.932643in}}%
\pgfpathlineto{\pgfqpoint{1.295414in}{0.951088in}}%
\pgfpathlineto{\pgfqpoint{1.262305in}{0.969262in}}%
\pgfpathlineto{\pgfqpoint{1.234560in}{0.987081in}}%
\pgfpathlineto{\pgfqpoint{1.211889in}{1.004476in}}%
\pgfpathlineto{\pgfqpoint{1.193992in}{1.021389in}}%
\pgfpathlineto{\pgfqpoint{1.180567in}{1.037778in}}%
\pgfpathlineto{\pgfqpoint{1.171210in}{1.053585in}}%
\pgfpathlineto{\pgfqpoint{1.165229in}{1.068759in}}%
\pgfpathlineto{\pgfqpoint{1.161998in}{1.083284in}}%
\pgfpathlineto{\pgfqpoint{1.161033in}{1.097146in}}%
\pgfpathlineto{\pgfqpoint{1.161992in}{1.110334in}}%
\pgfpathlineto{\pgfqpoint{1.164674in}{1.122841in}}%
\pgfpathlineto{\pgfqpoint{1.169019in}{1.134664in}}%
\pgfpathlineto{\pgfqpoint{1.175110in}{1.145800in}}%
\pgfpathlineto{\pgfqpoint{1.183171in}{1.156251in}}%
\pgfpathlineto{\pgfqpoint{1.193567in}{1.166024in}}%
\pgfpathlineto{\pgfqpoint{1.206794in}{1.175127in}}%
\pgfpathlineto{\pgfqpoint{1.222423in}{1.183540in}}%
\pgfpathlineto{\pgfqpoint{1.249446in}{1.194837in}}%
\pgfpathlineto{\pgfqpoint{1.280805in}{1.204541in}}%
\pgfpathlineto{\pgfqpoint{1.316639in}{1.212645in}}%
\pgfpathlineto{\pgfqpoint{1.357196in}{1.219139in}}%
\pgfpathlineto{\pgfqpoint{1.402831in}{1.224011in}}%
\pgfpathlineto{\pgfqpoint{1.454005in}{1.227246in}}%
\pgfpathlineto{\pgfqpoint{1.510986in}{1.228828in}}%
\pgfpathlineto{\pgfqpoint{1.574288in}{1.228702in}}%
\pgfpathlineto{\pgfqpoint{1.645026in}{1.226790in}}%
\pgfpathlineto{\pgfqpoint{1.752391in}{1.221317in}}%
\pgfpathlineto{\pgfqpoint{1.875916in}{1.212316in}}%
\pgfpathlineto{\pgfqpoint{2.016235in}{1.199571in}}%
\pgfpathlineto{\pgfqpoint{2.173212in}{1.182858in}}%
\pgfpathlineto{\pgfqpoint{2.345993in}{1.161934in}}%
\pgfpathlineto{\pgfqpoint{2.484100in}{1.143336in}}%
\pgfpathlineto{\pgfqpoint{2.623987in}{1.122388in}}%
\pgfpathlineto{\pgfqpoint{2.760522in}{1.099329in}}%
\pgfpathlineto{\pgfqpoint{2.847517in}{1.082928in}}%
\pgfpathlineto{\pgfqpoint{2.930008in}{1.065820in}}%
\pgfpathlineto{\pgfqpoint{3.007079in}{1.048115in}}%
\pgfpathlineto{\pgfqpoint{3.077973in}{1.029935in}}%
\pgfpathlineto{\pgfqpoint{3.142097in}{1.011412in}}%
\pgfpathlineto{\pgfqpoint{3.199017in}{0.992688in}}%
\pgfpathlineto{\pgfqpoint{3.248460in}{0.973917in}}%
\pgfpathlineto{\pgfqpoint{3.290633in}{0.955252in}}%
\pgfpathlineto{\pgfqpoint{3.326462in}{0.936802in}}%
\pgfpathlineto{\pgfqpoint{3.356771in}{0.918653in}}%
\pgfpathlineto{\pgfqpoint{3.382214in}{0.900882in}}%
\pgfpathlineto{\pgfqpoint{3.403277in}{0.883558in}}%
\pgfpathlineto{\pgfqpoint{3.420273in}{0.866738in}}%
\pgfpathlineto{\pgfqpoint{3.433347in}{0.850474in}}%
\pgfpathlineto{\pgfqpoint{3.442525in}{0.834807in}}%
\pgfpathlineto{\pgfqpoint{3.448447in}{0.819771in}}%
\pgfpathlineto{\pgfqpoint{3.451600in}{0.805387in}}%
\pgfpathlineto{\pgfqpoint{3.452259in}{0.791669in}}%
\pgfpathlineto{\pgfqpoint{3.450627in}{0.778628in}}%
\pgfpathlineto{\pgfqpoint{3.446838in}{0.766275in}}%
\pgfpathlineto{\pgfqpoint{3.440956in}{0.754613in}}%
\pgfpathlineto{\pgfqpoint{3.432977in}{0.743646in}}%
\pgfpathlineto{\pgfqpoint{3.422869in}{0.733372in}}%
\pgfpathlineto{\pgfqpoint{3.410760in}{0.723795in}}%
\pgfpathlineto{\pgfqpoint{3.396727in}{0.714916in}}%
\pgfpathlineto{\pgfqpoint{3.380815in}{0.706738in}}%
\pgfpathlineto{\pgfqpoint{3.353444in}{0.695791in}}%
\pgfpathlineto{\pgfqpoint{3.321797in}{0.686430in}}%
\pgfpathlineto{\pgfqpoint{3.285663in}{0.678661in}}%
\pgfpathlineto{\pgfqpoint{3.244686in}{0.672486in}}%
\pgfpathlineto{\pgfqpoint{3.198378in}{0.667906in}}%
\pgfpathlineto{\pgfqpoint{3.146581in}{0.664946in}}%
\pgfpathlineto{\pgfqpoint{3.088767in}{0.663656in}}%
\pgfpathlineto{\pgfqpoint{3.024157in}{0.664093in}}%
\pgfpathlineto{\pgfqpoint{2.926227in}{0.667475in}}%
\pgfpathlineto{\pgfqpoint{2.813490in}{0.674224in}}%
\pgfpathlineto{\pgfqpoint{2.684693in}{0.684553in}}%
\pgfpathlineto{\pgfqpoint{2.538808in}{0.698706in}}%
\pgfpathlineto{\pgfqpoint{2.373150in}{0.717028in}}%
\pgfpathlineto{\pgfqpoint{2.237953in}{0.733642in}}%
\pgfpathlineto{\pgfqpoint{2.099162in}{0.752633in}}%
\pgfpathlineto{\pgfqpoint{1.961373in}{0.773855in}}%
\pgfpathlineto{\pgfqpoint{1.828558in}{0.797103in}}%
\pgfpathlineto{\pgfqpoint{1.744456in}{0.813596in}}%
\pgfpathlineto{\pgfqpoint{1.664899in}{0.830778in}}%
\pgfpathlineto{\pgfqpoint{1.590599in}{0.848542in}}%
\pgfpathlineto{\pgfqpoint{1.522143in}{0.866769in}}%
\pgfpathlineto{\pgfqpoint{1.459996in}{0.885330in}}%
\pgfpathlineto{\pgfqpoint{1.404496in}{0.904081in}}%
\pgfpathlineto{\pgfqpoint{1.355859in}{0.922868in}}%
\pgfpathlineto{\pgfqpoint{1.314177in}{0.941525in}}%
\pgfpathlineto{\pgfqpoint{1.279329in}{0.959886in}}%
\pgfpathlineto{\pgfqpoint{1.250259in}{0.977902in}}%
\pgfpathlineto{\pgfqpoint{1.226235in}{0.995516in}}%
\pgfpathlineto{\pgfqpoint{1.206743in}{1.012661in}}%
\pgfpathlineto{\pgfqpoint{1.191330in}{1.029281in}}%
\pgfpathlineto{\pgfqpoint{1.179579in}{1.045333in}}%
\pgfpathlineto{\pgfqpoint{1.171116in}{1.060783in}}%
\pgfpathlineto{\pgfqpoint{1.165631in}{1.075604in}}%
\pgfpathlineto{\pgfqpoint{1.162877in}{1.089776in}}%
\pgfpathlineto{\pgfqpoint{1.162726in}{1.103283in}}%
\pgfpathlineto{\pgfqpoint{1.165049in}{1.116115in}}%
\pgfpathlineto{\pgfqpoint{1.169586in}{1.128264in}}%
\pgfpathlineto{\pgfqpoint{1.176135in}{1.139722in}}%
\pgfpathlineto{\pgfqpoint{1.184554in}{1.150485in}}%
\pgfpathlineto{\pgfqpoint{1.194762in}{1.160551in}}%
\pgfpathlineto{\pgfqpoint{1.206739in}{1.169918in}}%
\pgfpathlineto{\pgfqpoint{1.220528in}{1.178588in}}%
\pgfpathlineto{\pgfqpoint{1.236230in}{1.186564in}}%
\pgfpathlineto{\pgfqpoint{1.254009in}{1.193853in}}%
\pgfpathlineto{\pgfqpoint{1.284888in}{1.203502in}}%
\pgfpathlineto{\pgfqpoint{1.320424in}{1.211581in}}%
\pgfpathlineto{\pgfqpoint{1.360739in}{1.218070in}}%
\pgfpathlineto{\pgfqpoint{1.406145in}{1.222948in}}%
\pgfpathlineto{\pgfqpoint{1.457037in}{1.226185in}}%
\pgfpathlineto{\pgfqpoint{1.513896in}{1.227747in}}%
\pgfpathlineto{\pgfqpoint{1.577289in}{1.227593in}}%
\pgfpathlineto{\pgfqpoint{1.673098in}{1.224632in}}%
\pgfpathlineto{\pgfqpoint{1.783158in}{1.218369in}}%
\pgfpathlineto{\pgfqpoint{1.909640in}{1.208541in}}%
\pgfpathlineto{\pgfqpoint{2.053601in}{1.194901in}}%
\pgfpathlineto{\pgfqpoint{2.214414in}{1.177230in}}%
\pgfpathlineto{\pgfqpoint{2.389790in}{1.155347in}}%
\pgfpathlineto{\pgfqpoint{2.528122in}{1.136094in}}%
\pgfpathlineto{\pgfqpoint{2.666990in}{1.114546in}}%
\pgfpathlineto{\pgfqpoint{2.801430in}{1.090949in}}%
\pgfpathlineto{\pgfqpoint{2.886479in}{1.074229in}}%
\pgfpathlineto{\pgfqpoint{2.966597in}{1.056842in}}%
\pgfpathlineto{\pgfqpoint{3.040859in}{1.038912in}}%
\pgfpathlineto{\pgfqpoint{3.108487in}{1.020571in}}%
\pgfpathlineto{\pgfqpoint{3.168847in}{1.001969in}}%
\pgfpathlineto{\pgfqpoint{3.221582in}{0.983266in}}%
\pgfpathlineto{\pgfqpoint{3.267228in}{0.964599in}}%
\pgfpathlineto{\pgfqpoint{3.306545in}{0.946077in}}%
\pgfpathlineto{\pgfqpoint{3.340169in}{0.927797in}}%
\pgfpathlineto{\pgfqpoint{3.368611in}{0.909848in}}%
\pgfpathlineto{\pgfqpoint{3.392261in}{0.892306in}}%
\pgfpathlineto{\pgfqpoint{3.411381in}{0.875240in}}%
\pgfpathlineto{\pgfqpoint{3.426114in}{0.858707in}}%
\pgfpathlineto{\pgfqpoint{3.436868in}{0.842756in}}%
\pgfpathlineto{\pgfqpoint{3.444379in}{0.827420in}}%
\pgfpathlineto{\pgfqpoint{3.449037in}{0.812720in}}%
\pgfpathlineto{\pgfqpoint{3.451139in}{0.798676in}}%
\pgfpathlineto{\pgfqpoint{3.450895in}{0.785301in}}%
\pgfpathlineto{\pgfqpoint{3.448425in}{0.772606in}}%
\pgfpathlineto{\pgfqpoint{3.443760in}{0.760598in}}%
\pgfpathlineto{\pgfqpoint{3.436840in}{0.749278in}}%
\pgfpathlineto{\pgfqpoint{3.427741in}{0.738649in}}%
\pgfpathlineto{\pgfqpoint{3.416646in}{0.728717in}}%
\pgfpathlineto{\pgfqpoint{3.403636in}{0.719483in}}%
\pgfpathlineto{\pgfqpoint{3.388761in}{0.710951in}}%
\pgfpathlineto{\pgfqpoint{3.362983in}{0.699470in}}%
\pgfpathlineto{\pgfqpoint{3.332975in}{0.689573in}}%
\pgfpathlineto{\pgfqpoint{3.298512in}{0.681261in}}%
\pgfpathlineto{\pgfqpoint{3.259214in}{0.674529in}}%
\pgfpathlineto{\pgfqpoint{3.214707in}{0.669381in}}%
\pgfpathlineto{\pgfqpoint{3.164871in}{0.665854in}}%
\pgfpathlineto{\pgfqpoint{3.109115in}{0.663985in}}%
\pgfpathlineto{\pgfqpoint{3.046813in}{0.663825in}}%
\pgfpathlineto{\pgfqpoint{2.977337in}{0.665432in}}%
\pgfpathlineto{\pgfqpoint{2.872461in}{0.670446in}}%
\pgfpathlineto{\pgfqpoint{2.752229in}{0.678924in}}%
\pgfpathlineto{\pgfqpoint{2.615161in}{0.691097in}}%
\pgfpathlineto{\pgfqpoint{2.460557in}{0.707212in}}%
\pgfpathlineto{\pgfqpoint{2.289928in}{0.727523in}}%
\pgfpathlineto{\pgfqpoint{2.154177in}{0.745561in}}%
\pgfpathlineto{\pgfqpoint{2.015335in}{0.765943in}}%
\pgfpathlineto{\pgfqpoint{1.877819in}{0.788538in}}%
\pgfpathlineto{\pgfqpoint{1.746676in}{0.813116in}}%
\pgfpathlineto{\pgfqpoint{1.665065in}{0.830393in}}%
\pgfpathlineto{\pgfqpoint{1.589328in}{0.848223in}}%
\pgfpathlineto{\pgfqpoint{1.520196in}{0.866471in}}%
\pgfpathlineto{\pgfqpoint{1.458122in}{0.885002in}}%
\pgfpathlineto{\pgfqpoint{1.403277in}{0.903683in}}%
\pgfpathlineto{\pgfqpoint{1.355554in}{0.922382in}}%
\pgfpathlineto{\pgfqpoint{1.314693in}{0.940963in}}%
\pgfpathlineto{\pgfqpoint{1.280476in}{0.959287in}}%
\pgfpathlineto{\pgfqpoint{1.251791in}{0.977284in}}%
\pgfpathlineto{\pgfqpoint{1.227663in}{0.994893in}}%
\pgfpathlineto{\pgfqpoint{1.207378in}{1.012059in}}%
\pgfpathlineto{\pgfqpoint{1.190487in}{1.028732in}}%
\pgfpathlineto{\pgfqpoint{1.176806in}{1.044867in}}%
\pgfpathlineto{\pgfqpoint{1.166413in}{1.060421in}}%
\pgfpathlineto{\pgfqpoint{1.159651in}{1.075360in}}%
\pgfpathlineto{\pgfqpoint{1.156967in}{1.089651in}}%
\pgfpathlineto{\pgfqpoint{1.157193in}{1.103266in}}%
\pgfpathlineto{\pgfqpoint{1.159740in}{1.116193in}}%
\pgfpathlineto{\pgfqpoint{1.164420in}{1.128426in}}%
\pgfpathlineto{\pgfqpoint{1.171102in}{1.139959in}}%
\pgfpathlineto{\pgfqpoint{1.179704in}{1.150790in}}%
\pgfpathlineto{\pgfqpoint{1.190198in}{1.160918in}}%
\pgfpathlineto{\pgfqpoint{1.202605in}{1.170344in}}%
\pgfpathlineto{\pgfqpoint{1.217000in}{1.179070in}}%
\pgfpathlineto{\pgfqpoint{1.233439in}{1.187100in}}%
\pgfpathlineto{\pgfqpoint{1.261742in}{1.197831in}}%
\pgfpathlineto{\pgfqpoint{1.294457in}{1.206979in}}%
\pgfpathlineto{\pgfqpoint{1.331734in}{1.214532in}}%
\pgfpathlineto{\pgfqpoint{1.373826in}{1.220473in}}%
\pgfpathlineto{\pgfqpoint{1.421095in}{1.224785in}}%
\pgfpathlineto{\pgfqpoint{1.474003in}{1.227446in}}%
\pgfpathlineto{\pgfqpoint{1.533120in}{1.228431in}}%
\pgfpathlineto{\pgfqpoint{1.598708in}{1.227726in}}%
\pgfpathlineto{\pgfqpoint{1.697508in}{1.224019in}}%
\pgfpathlineto{\pgfqpoint{1.812125in}{1.216869in}}%
\pgfpathlineto{\pgfqpoint{1.944160in}{1.206017in}}%
\pgfpathlineto{\pgfqpoint{2.093583in}{1.191256in}}%
\pgfpathlineto{\pgfqpoint{2.258737in}{1.172430in}}%
\pgfpathlineto{\pgfqpoint{2.436336in}{1.149434in}}%
\pgfpathlineto{\pgfqpoint{2.574766in}{1.129416in}}%
\pgfpathlineto{\pgfqpoint{2.713065in}{1.107086in}}%
\pgfpathlineto{\pgfqpoint{2.845630in}{1.082773in}}%
\pgfpathlineto{\pgfqpoint{2.928588in}{1.065657in}}%
\pgfpathlineto{\pgfqpoint{3.005984in}{1.047957in}}%
\pgfpathlineto{\pgfqpoint{3.077027in}{1.029798in}}%
\pgfpathlineto{\pgfqpoint{3.141157in}{1.011311in}}%
\pgfpathlineto{\pgfqpoint{3.198045in}{0.992628in}}%
\pgfpathlineto{\pgfqpoint{3.247593in}{0.973888in}}%
\pgfpathlineto{\pgfqpoint{3.289910in}{0.955237in}}%
\pgfpathlineto{\pgfqpoint{3.325615in}{0.936810in}}%
\pgfpathlineto{\pgfqpoint{3.355734in}{0.918688in}}%
\pgfpathlineto{\pgfqpoint{3.381095in}{0.900939in}}%
\pgfpathlineto{\pgfqpoint{3.402305in}{0.883628in}}%
\pgfpathlineto{\pgfqpoint{3.419749in}{0.866811in}}%
\pgfpathlineto{\pgfqpoint{3.433588in}{0.850538in}}%
\pgfpathlineto{\pgfqpoint{3.443762in}{0.834850in}}%
\pgfpathlineto{\pgfqpoint{3.450021in}{0.819785in}}%
\pgfpathlineto{\pgfqpoint{3.453055in}{0.805373in}}%
\pgfpathlineto{\pgfqpoint{3.453530in}{0.791632in}}%
\pgfpathlineto{\pgfqpoint{3.451691in}{0.778572in}}%
\pgfpathlineto{\pgfqpoint{3.447719in}{0.766201in}}%
\pgfpathlineto{\pgfqpoint{3.441733in}{0.754525in}}%
\pgfpathlineto{\pgfqpoint{3.433790in}{0.743547in}}%
\pgfpathlineto{\pgfqpoint{3.423884in}{0.733268in}}%
\pgfpathlineto{\pgfqpoint{3.411948in}{0.723686in}}%
\pgfpathlineto{\pgfqpoint{3.397931in}{0.714798in}}%
\pgfpathlineto{\pgfqpoint{3.381970in}{0.706609in}}%
\pgfpathlineto{\pgfqpoint{3.354432in}{0.695639in}}%
\pgfpathlineto{\pgfqpoint{3.322544in}{0.686254in}}%
\pgfpathlineto{\pgfqpoint{3.286159in}{0.678463in}}%
\pgfpathlineto{\pgfqpoint{3.245014in}{0.672279in}}%
\pgfpathlineto{\pgfqpoint{3.198731in}{0.667714in}}%
\pgfpathlineto{\pgfqpoint{3.146814in}{0.664782in}}%
\pgfpathlineto{\pgfqpoint{3.088904in}{0.663499in}}%
\pgfpathlineto{\pgfqpoint{3.024644in}{0.663921in}}%
\pgfpathlineto{\pgfqpoint{2.927093in}{0.667287in}}%
\pgfpathlineto{\pgfqpoint{2.814108in}{0.674049in}}%
\pgfpathlineto{\pgfqpoint{2.684517in}{0.684419in}}%
\pgfpathlineto{\pgfqpoint{2.538066in}{0.698611in}}%
\pgfpathlineto{\pgfqpoint{2.375420in}{0.716841in}}%
\pgfpathlineto{\pgfqpoint{2.198151in}{0.739336in}}%
\pgfpathlineto{\pgfqpoint{2.058923in}{0.759078in}}%
\pgfpathlineto{\pgfqpoint{1.920246in}{0.781071in}}%
\pgfpathlineto{\pgfqpoint{1.787009in}{0.805038in}}%
\pgfpathlineto{\pgfqpoint{1.703233in}{0.821953in}}%
\pgfpathlineto{\pgfqpoint{1.624668in}{0.839492in}}%
\pgfpathlineto{\pgfqpoint{1.552127in}{0.857535in}}%
\pgfpathlineto{\pgfqpoint{1.486253in}{0.875955in}}%
\pgfpathlineto{\pgfqpoint{1.427520in}{0.894612in}}%
\pgfpathlineto{\pgfqpoint{1.376234in}{0.913358in}}%
\pgfpathlineto{\pgfqpoint{1.332368in}{0.932035in}}%
\pgfpathlineto{\pgfqpoint{1.295079in}{0.950520in}}%
\pgfpathlineto{\pgfqpoint{1.263463in}{0.968728in}}%
\pgfpathlineto{\pgfqpoint{1.236805in}{0.986579in}}%
\pgfpathlineto{\pgfqpoint{1.214571in}{1.004003in}}%
\pgfpathlineto{\pgfqpoint{1.196412in}{1.020940in}}%
\pgfpathlineto{\pgfqpoint{1.182163in}{1.037337in}}%
\pgfpathlineto{\pgfqpoint{1.171841in}{1.053149in}}%
\pgfpathlineto{\pgfqpoint{1.165190in}{1.068340in}}%
\pgfpathlineto{\pgfqpoint{1.161412in}{1.082882in}}%
\pgfpathlineto{\pgfqpoint{1.160200in}{1.096762in}}%
\pgfpathlineto{\pgfqpoint{1.161320in}{1.109966in}}%
\pgfpathlineto{\pgfqpoint{1.164609in}{1.122485in}}%
\pgfpathlineto{\pgfqpoint{1.169974in}{1.134313in}}%
\pgfpathlineto{\pgfqpoint{1.177390in}{1.145447in}}%
\pgfpathlineto{\pgfqpoint{1.186903in}{1.155886in}}%
\pgfpathlineto{\pgfqpoint{1.198513in}{1.165631in}}%
\pgfpathlineto{\pgfqpoint{1.212079in}{1.174679in}}%
\pgfpathlineto{\pgfqpoint{1.227552in}{1.183028in}}%
\pgfpathlineto{\pgfqpoint{1.254290in}{1.194234in}}%
\pgfpathlineto{\pgfqpoint{1.285304in}{1.203854in}}%
\pgfpathlineto{\pgfqpoint{1.320760in}{1.211882in}}%
\pgfpathlineto{\pgfqpoint{1.360960in}{1.218313in}}%
\pgfpathlineto{\pgfqpoint{1.406337in}{1.223142in}}%
\pgfpathlineto{\pgfqpoint{1.457296in}{1.226360in}}%
\pgfpathlineto{\pgfqpoint{1.514044in}{1.227921in}}%
\pgfpathlineto{\pgfqpoint{1.577430in}{1.227767in}}%
\pgfpathlineto{\pgfqpoint{1.648248in}{1.225832in}}%
\pgfpathlineto{\pgfqpoint{1.755397in}{1.220351in}}%
\pgfpathlineto{\pgfqpoint{1.878263in}{1.211370in}}%
\pgfpathlineto{\pgfqpoint{2.017792in}{1.198661in}}%
\pgfpathlineto{\pgfqpoint{2.174614in}{1.181968in}}%
\pgfpathlineto{\pgfqpoint{2.349403in}{1.160995in}}%
\pgfpathlineto{\pgfqpoint{2.487236in}{1.142443in}}%
\pgfpathlineto{\pgfqpoint{2.625633in}{1.121592in}}%
\pgfpathlineto{\pgfqpoint{2.760426in}{1.098633in}}%
\pgfpathlineto{\pgfqpoint{2.887929in}{1.073831in}}%
\pgfpathlineto{\pgfqpoint{2.967279in}{1.056437in}}%
\pgfpathlineto{\pgfqpoint{3.041125in}{1.038492in}}%
\pgfpathlineto{\pgfqpoint{3.108734in}{1.020132in}}%
\pgfpathlineto{\pgfqpoint{3.169465in}{1.001504in}}%
\pgfpathlineto{\pgfqpoint{3.222774in}{0.982772in}}%
\pgfpathlineto{\pgfqpoint{3.268555in}{0.964095in}}%
\pgfpathlineto{\pgfqpoint{3.307763in}{0.945571in}}%
\pgfpathlineto{\pgfqpoint{3.341022in}{0.927299in}}%
\pgfpathlineto{\pgfqpoint{3.368876in}{0.909370in}}%
\pgfpathlineto{\pgfqpoint{3.391809in}{0.891859in}}%
\pgfpathlineto{\pgfqpoint{3.410256in}{0.874834in}}%
\pgfpathlineto{\pgfqpoint{3.424675in}{0.858347in}}%
\pgfpathlineto{\pgfqpoint{3.435544in}{0.842437in}}%
\pgfpathlineto{\pgfqpoint{3.443231in}{0.827137in}}%
\pgfpathlineto{\pgfqpoint{3.448034in}{0.812471in}}%
\pgfpathlineto{\pgfqpoint{3.450170in}{0.798459in}}%
\pgfpathlineto{\pgfqpoint{3.449779in}{0.785115in}}%
\pgfpathlineto{\pgfqpoint{3.446950in}{0.772449in}}%
\pgfpathlineto{\pgfqpoint{3.441893in}{0.760468in}}%
\pgfpathlineto{\pgfqpoint{3.434794in}{0.749178in}}%
\pgfpathlineto{\pgfqpoint{3.425789in}{0.738584in}}%
\pgfpathlineto{\pgfqpoint{3.414961in}{0.728689in}}%
\pgfpathlineto{\pgfqpoint{3.402341in}{0.719492in}}%
\pgfpathlineto{\pgfqpoint{3.387909in}{0.710995in}}%
\pgfpathlineto{\pgfqpoint{3.371595in}{0.703195in}}%
\pgfpathlineto{\pgfqpoint{3.343311in}{0.692791in}}%
\pgfpathlineto{\pgfqpoint{3.310111in}{0.683936in}}%
\pgfpathlineto{\pgfqpoint{3.272219in}{0.676659in}}%
\pgfpathlineto{\pgfqpoint{3.229399in}{0.670979in}}%
\pgfpathlineto{\pgfqpoint{3.181290in}{0.666922in}}%
\pgfpathlineto{\pgfqpoint{3.127454in}{0.664520in}}%
\pgfpathlineto{\pgfqpoint{3.067371in}{0.663815in}}%
\pgfpathlineto{\pgfqpoint{3.000442in}{0.664854in}}%
\pgfpathlineto{\pgfqpoint{2.899381in}{0.669051in}}%
\pgfpathlineto{\pgfqpoint{2.783334in}{0.676634in}}%
\pgfpathlineto{\pgfqpoint{2.650424in}{0.687868in}}%
\pgfpathlineto{\pgfqpoint{2.500049in}{0.702999in}}%
\pgfpathlineto{\pgfqpoint{2.333328in}{0.722239in}}%
\pgfpathlineto{\pgfqpoint{2.153302in}{0.745753in}}%
\pgfpathlineto{\pgfqpoint{2.014441in}{0.766151in}}%
\pgfpathlineto{\pgfqpoint{1.877404in}{0.788743in}}%
\pgfpathlineto{\pgfqpoint{1.746582in}{0.813270in}}%
\pgfpathlineto{\pgfqpoint{1.664894in}{0.830526in}}%
\pgfpathlineto{\pgfqpoint{1.589017in}{0.848358in}}%
\pgfpathlineto{\pgfqpoint{1.520066in}{0.866621in}}%
\pgfpathlineto{\pgfqpoint{1.458459in}{0.885162in}}%
\pgfpathlineto{\pgfqpoint{1.403941in}{0.903839in}}%
\pgfpathlineto{\pgfqpoint{1.356200in}{0.922524in}}%
\pgfpathlineto{\pgfqpoint{1.314898in}{0.941095in}}%
\pgfpathlineto{\pgfqpoint{1.279669in}{0.959448in}}%
\pgfpathlineto{\pgfqpoint{1.250119in}{0.977486in}}%
\pgfpathlineto{\pgfqpoint{1.225829in}{0.995127in}}%
\pgfpathlineto{\pgfqpoint{1.206344in}{1.012296in}}%
\pgfpathlineto{\pgfqpoint{1.191083in}{1.028935in}}%
\pgfpathlineto{\pgfqpoint{1.179443in}{1.045006in}}%
\pgfpathlineto{\pgfqpoint{1.170940in}{1.060477in}}%
\pgfpathlineto{\pgfqpoint{1.165221in}{1.075323in}}%
\pgfpathlineto{\pgfqpoint{1.162060in}{1.089522in}}%
\pgfpathlineto{\pgfqpoint{1.161359in}{1.103059in}}%
\pgfpathlineto{\pgfqpoint{1.163150in}{1.115923in}}%
\pgfpathlineto{\pgfqpoint{1.167511in}{1.128107in}}%
\pgfpathlineto{\pgfqpoint{1.174102in}{1.139602in}}%
\pgfpathlineto{\pgfqpoint{1.182731in}{1.150402in}}%
\pgfpathlineto{\pgfqpoint{1.193292in}{1.160504in}}%
\pgfpathlineto{\pgfqpoint{1.205714in}{1.169905in}}%
\pgfpathlineto{\pgfqpoint{1.219965in}{1.178605in}}%
\pgfpathlineto{\pgfqpoint{1.236051in}{1.186602in}}%
\pgfpathlineto{\pgfqpoint{1.263724in}{1.197285in}}%
\pgfpathlineto{\pgfqpoint{1.295936in}{1.206399in}}%
\pgfpathlineto{\pgfqpoint{1.333017in}{1.213947in}}%
\pgfpathlineto{\pgfqpoint{1.374960in}{1.219906in}}%
\pgfpathlineto{\pgfqpoint{1.422124in}{1.224252in}}%
\pgfpathlineto{\pgfqpoint{1.474959in}{1.226951in}}%
\pgfpathlineto{\pgfqpoint{1.533976in}{1.227962in}}%
\pgfpathlineto{\pgfqpoint{1.599750in}{1.227234in}}%
\pgfpathlineto{\pgfqpoint{1.699070in}{1.223452in}}%
\pgfpathlineto{\pgfqpoint{1.813221in}{1.216301in}}%
\pgfpathlineto{\pgfqpoint{1.943827in}{1.205553in}}%
\pgfpathlineto{\pgfqpoint{2.091934in}{1.190939in}}%
\pgfpathlineto{\pgfqpoint{2.256806in}{1.172230in}}%
\pgfpathlineto{\pgfqpoint{2.435320in}{1.149269in}}%
\pgfpathlineto{\pgfqpoint{2.573872in}{1.129273in}}%
\pgfpathlineto{\pgfqpoint{2.711861in}{1.107036in}}%
\pgfpathlineto{\pgfqpoint{2.844493in}{1.082799in}}%
\pgfpathlineto{\pgfqpoint{2.927303in}{1.065679in}}%
\pgfpathlineto{\pgfqpoint{3.004310in}{1.047939in}}%
\pgfpathlineto{\pgfqpoint{3.074914in}{1.029748in}}%
\pgfpathlineto{\pgfqpoint{3.138709in}{1.011257in}}%
\pgfpathlineto{\pgfqpoint{3.195482in}{0.992608in}}%
\pgfpathlineto{\pgfqpoint{3.245211in}{0.973932in}}%
\pgfpathlineto{\pgfqpoint{3.288065in}{0.955346in}}%
\pgfpathlineto{\pgfqpoint{3.324408in}{0.936956in}}%
\pgfpathlineto{\pgfqpoint{3.354794in}{0.918855in}}%
\pgfpathlineto{\pgfqpoint{3.379971in}{0.901126in}}%
\pgfpathlineto{\pgfqpoint{3.400491in}{0.883848in}}%
\pgfpathlineto{\pgfqpoint{3.416508in}{0.867093in}}%
\pgfpathlineto{\pgfqpoint{3.428852in}{0.850894in}}%
\pgfpathlineto{\pgfqpoint{3.438186in}{0.835281in}}%
\pgfpathlineto{\pgfqpoint{3.444980in}{0.820278in}}%
\pgfpathlineto{\pgfqpoint{3.449508in}{0.805906in}}%
\pgfpathlineto{\pgfqpoint{3.451851in}{0.792181in}}%
\pgfpathlineto{\pgfqpoint{3.451894in}{0.779115in}}%
\pgfpathlineto{\pgfqpoint{3.449329in}{0.766716in}}%
\pgfpathlineto{\pgfqpoint{3.443674in}{0.754989in}}%
\pgfpathlineto{\pgfqpoint{3.435469in}{0.743952in}}%
\pgfpathlineto{\pgfqpoint{3.425274in}{0.733616in}}%
\pgfpathlineto{\pgfqpoint{3.413170in}{0.723984in}}%
\pgfpathlineto{\pgfqpoint{3.399205in}{0.715056in}}%
\pgfpathlineto{\pgfqpoint{3.383398in}{0.706833in}}%
\pgfpathlineto{\pgfqpoint{3.356195in}{0.695822in}}%
\pgfpathlineto{\pgfqpoint{3.324636in}{0.686396in}}%
\pgfpathlineto{\pgfqpoint{3.288388in}{0.678550in}}%
\pgfpathlineto{\pgfqpoint{3.247223in}{0.672288in}}%
\pgfpathlineto{\pgfqpoint{3.200959in}{0.667637in}}%
\pgfpathlineto{\pgfqpoint{3.149113in}{0.664626in}}%
\pgfpathlineto{\pgfqpoint{3.091155in}{0.663296in}}%
\pgfpathlineto{\pgfqpoint{3.026512in}{0.663697in}}%
\pgfpathlineto{\pgfqpoint{2.928848in}{0.667032in}}%
\pgfpathlineto{\pgfqpoint{2.816608in}{0.673727in}}%
\pgfpathlineto{\pgfqpoint{2.688118in}{0.683990in}}%
\pgfpathlineto{\pgfqpoint{2.542196in}{0.698084in}}%
\pgfpathlineto{\pgfqpoint{2.379172in}{0.716256in}}%
\pgfpathlineto{\pgfqpoint{2.201965in}{0.738670in}}%
\pgfpathlineto{\pgfqpoint{2.063483in}{0.758274in}}%
\pgfpathlineto{\pgfqpoint{1.924896in}{0.780153in}}%
\pgfpathlineto{\pgfqpoint{1.790841in}{0.804086in}}%
\pgfpathlineto{\pgfqpoint{1.706391in}{0.821022in}}%
\pgfpathlineto{\pgfqpoint{1.627689in}{0.838615in}}%
\pgfpathlineto{\pgfqpoint{1.555516in}{0.856719in}}%
\pgfpathlineto{\pgfqpoint{1.490121in}{0.875171in}}%
\pgfpathlineto{\pgfqpoint{1.431625in}{0.893820in}}%
\pgfpathlineto{\pgfqpoint{1.380019in}{0.912530in}}%
\pgfpathlineto{\pgfqpoint{1.335168in}{0.931176in}}%
\pgfpathlineto{\pgfqpoint{1.296807in}{0.949648in}}%
\pgfpathlineto{\pgfqpoint{1.264545in}{0.967850in}}%
\pgfpathlineto{\pgfqpoint{1.237862in}{0.985697in}}%
\pgfpathlineto{\pgfqpoint{1.216109in}{1.003119in}}%
\pgfpathlineto{\pgfqpoint{1.198720in}{1.020050in}}%
\pgfpathlineto{\pgfqpoint{1.185575in}{1.036423in}}%
\pgfpathlineto{\pgfqpoint{1.175908in}{1.052207in}}%
\pgfpathlineto{\pgfqpoint{1.169063in}{1.067380in}}%
\pgfpathlineto{\pgfqpoint{1.164563in}{1.081922in}}%
\pgfpathlineto{\pgfqpoint{1.162110in}{1.095817in}}%
\pgfpathlineto{\pgfqpoint{1.161591in}{1.109053in}}%
\pgfpathlineto{\pgfqpoint{1.163072in}{1.121621in}}%
\pgfpathlineto{\pgfqpoint{1.166800in}{1.133518in}}%
\pgfpathlineto{\pgfqpoint{1.173205in}{1.144743in}}%
\pgfpathlineto{\pgfqpoint{1.182672in}{1.155292in}}%
\pgfpathlineto{\pgfqpoint{1.194299in}{1.165143in}}%
\pgfpathlineto{\pgfqpoint{1.207837in}{1.174289in}}%
\pgfpathlineto{\pgfqpoint{1.223249in}{1.182730in}}%
\pgfpathlineto{\pgfqpoint{1.249864in}{1.194066in}}%
\pgfpathlineto{\pgfqpoint{1.280753in}{1.203810in}}%
\pgfpathlineto{\pgfqpoint{1.316122in}{1.211960in}}%
\pgfpathlineto{\pgfqpoint{1.356315in}{1.218516in}}%
\pgfpathlineto{\pgfqpoint{1.401662in}{1.223474in}}%
\pgfpathlineto{\pgfqpoint{1.452368in}{1.226801in}}%
\pgfpathlineto{\pgfqpoint{1.509068in}{1.228459in}}%
\pgfpathlineto{\pgfqpoint{1.572413in}{1.228398in}}%
\pgfpathlineto{\pgfqpoint{1.643044in}{1.226556in}}%
\pgfpathlineto{\pgfqpoint{1.749633in}{1.221205in}}%
\pgfpathlineto{\pgfqpoint{1.871746in}{1.212360in}}%
\pgfpathlineto{\pgfqpoint{2.010802in}{1.199787in}}%
\pgfpathlineto{\pgfqpoint{2.167532in}{1.183247in}}%
\pgfpathlineto{\pgfqpoint{2.339918in}{1.162481in}}%
\pgfpathlineto{\pgfqpoint{2.476471in}{1.144084in}}%
\pgfpathlineto{\pgfqpoint{2.615586in}{1.123346in}}%
\pgfpathlineto{\pgfqpoint{2.752866in}{1.100435in}}%
\pgfpathlineto{\pgfqpoint{2.840744in}{1.084075in}}%
\pgfpathlineto{\pgfqpoint{2.923941in}{1.066951in}}%
\pgfpathlineto{\pgfqpoint{3.001469in}{1.049217in}}%
\pgfpathlineto{\pgfqpoint{3.072598in}{1.031028in}}%
\pgfpathlineto{\pgfqpoint{3.136836in}{1.012530in}}%
\pgfpathlineto{\pgfqpoint{3.193930in}{0.993862in}}%
\pgfpathlineto{\pgfqpoint{3.243864in}{0.975152in}}%
\pgfpathlineto{\pgfqpoint{3.286860in}{0.956518in}}%
\pgfpathlineto{\pgfqpoint{3.323378in}{0.938070in}}%
\pgfpathlineto{\pgfqpoint{3.354110in}{0.919910in}}%
\pgfpathlineto{\pgfqpoint{3.379321in}{0.902145in}}%
\pgfpathlineto{\pgfqpoint{3.399639in}{0.884844in}}%
\pgfpathlineto{\pgfqpoint{3.416002in}{0.868048in}}%
\pgfpathlineto{\pgfqpoint{3.429118in}{0.851793in}}%
\pgfpathlineto{\pgfqpoint{3.439461in}{0.836113in}}%
\pgfpathlineto{\pgfqpoint{3.447272in}{0.821034in}}%
\pgfpathlineto{\pgfqpoint{3.452562in}{0.806582in}}%
\pgfpathlineto{\pgfqpoint{3.455109in}{0.792775in}}%
\pgfpathlineto{\pgfqpoint{3.454459in}{0.779629in}}%
\pgfpathlineto{\pgfqpoint{3.450508in}{0.767159in}}%
\pgfpathlineto{\pgfqpoint{3.444347in}{0.755387in}}%
\pgfpathlineto{\pgfqpoint{3.436167in}{0.744316in}}%
\pgfpathlineto{\pgfqpoint{3.426068in}{0.733948in}}%
\pgfpathlineto{\pgfqpoint{3.414111in}{0.724287in}}%
\pgfpathlineto{\pgfqpoint{3.400322in}{0.715331in}}%
\pgfpathlineto{\pgfqpoint{3.384685in}{0.707080in}}%
\pgfpathlineto{\pgfqpoint{3.357643in}{0.696021in}}%
\pgfpathlineto{\pgfqpoint{3.326007in}{0.686535in}}%
\pgfpathlineto{\pgfqpoint{3.289730in}{0.678627in}}%
\pgfpathlineto{\pgfqpoint{3.248653in}{0.672315in}}%
\pgfpathlineto{\pgfqpoint{3.202441in}{0.667621in}}%
\pgfpathlineto{\pgfqpoint{3.150680in}{0.664575in}}%
\pgfpathlineto{\pgfqpoint{3.092876in}{0.663214in}}%
\pgfpathlineto{\pgfqpoint{3.028459in}{0.663584in}}%
\pgfpathlineto{\pgfqpoint{2.931149in}{0.666864in}}%
\pgfpathlineto{\pgfqpoint{2.819270in}{0.673475in}}%
\pgfpathlineto{\pgfqpoint{2.691046in}{0.683663in}}%
\pgfpathlineto{\pgfqpoint{2.545302in}{0.697689in}}%
\pgfpathlineto{\pgfqpoint{2.382639in}{0.715779in}}%
\pgfpathlineto{\pgfqpoint{2.205480in}{0.738117in}}%
\pgfpathlineto{\pgfqpoint{2.066926in}{0.757671in}}%
\pgfpathlineto{\pgfqpoint{1.928420in}{0.779499in}}%
\pgfpathlineto{\pgfqpoint{1.794428in}{0.803380in}}%
\pgfpathlineto{\pgfqpoint{1.709785in}{0.820286in}}%
\pgfpathlineto{\pgfqpoint{1.630396in}{0.837842in}}%
\pgfpathlineto{\pgfqpoint{1.557527in}{0.855911in}}%
\pgfpathlineto{\pgfqpoint{1.491868in}{0.874342in}}%
\pgfpathlineto{\pgfqpoint{1.433325in}{0.892987in}}%
\pgfpathlineto{\pgfqpoint{1.381698in}{0.911709in}}%
\pgfpathlineto{\pgfqpoint{1.336728in}{0.930383in}}%
\pgfpathlineto{\pgfqpoint{1.298102in}{0.948894in}}%
\pgfpathlineto{\pgfqpoint{1.265447in}{0.967142in}}%
\pgfpathlineto{\pgfqpoint{1.238335in}{0.985036in}}%
\pgfpathlineto{\pgfqpoint{1.216280in}{1.002498in}}%
\pgfpathlineto{\pgfqpoint{1.198795in}{1.019457in}}%
\pgfpathlineto{\pgfqpoint{1.185256in}{1.035866in}}%
\pgfpathlineto{\pgfqpoint{1.175074in}{1.051693in}}%
\pgfpathlineto{\pgfqpoint{1.167801in}{1.066908in}}%
\pgfpathlineto{\pgfqpoint{1.163134in}{1.081487in}}%
\pgfpathlineto{\pgfqpoint{1.160908in}{1.095414in}}%
\pgfpathlineto{\pgfqpoint{1.161106in}{1.108674in}}%
\pgfpathlineto{\pgfqpoint{1.163849in}{1.121259in}}%
\pgfpathlineto{\pgfqpoint{1.169209in}{1.133162in}}%
\pgfpathlineto{\pgfqpoint{1.176724in}{1.144371in}}%
\pgfpathlineto{\pgfqpoint{1.186235in}{1.154884in}}%
\pgfpathlineto{\pgfqpoint{1.197652in}{1.164696in}}%
\pgfpathlineto{\pgfqpoint{1.210919in}{1.173806in}}%
\pgfpathlineto{\pgfqpoint{1.226017in}{1.182213in}}%
\pgfpathlineto{\pgfqpoint{1.252143in}{1.193506in}}%
\pgfpathlineto{\pgfqpoint{1.282638in}{1.203223in}}%
\pgfpathlineto{\pgfqpoint{1.317881in}{1.211371in}}%
\pgfpathlineto{\pgfqpoint{1.357976in}{1.217941in}}%
\pgfpathlineto{\pgfqpoint{1.403128in}{1.222906in}}%
\pgfpathlineto{\pgfqpoint{1.453772in}{1.226239in}}%
\pgfpathlineto{\pgfqpoint{1.510402in}{1.227900in}}%
\pgfpathlineto{\pgfqpoint{1.573570in}{1.227841in}}%
\pgfpathlineto{\pgfqpoint{1.643890in}{1.226006in}}%
\pgfpathlineto{\pgfqpoint{1.749939in}{1.220679in}}%
\pgfpathlineto{\pgfqpoint{1.871607in}{1.211884in}}%
\pgfpathlineto{\pgfqpoint{2.010356in}{1.199365in}}%
\pgfpathlineto{\pgfqpoint{2.166543in}{1.182867in}}%
\pgfpathlineto{\pgfqpoint{2.338436in}{1.162183in}}%
\pgfpathlineto{\pgfqpoint{2.474861in}{1.143867in}}%
\pgfpathlineto{\pgfqpoint{2.613765in}{1.123202in}}%
\pgfpathlineto{\pgfqpoint{2.750788in}{1.100354in}}%
\pgfpathlineto{\pgfqpoint{2.838642in}{1.084038in}}%
\pgfpathlineto{\pgfqpoint{2.921825in}{1.066956in}}%
\pgfpathlineto{\pgfqpoint{2.999293in}{1.049249in}}%
\pgfpathlineto{\pgfqpoint{3.070405in}{1.031080in}}%
\pgfpathlineto{\pgfqpoint{3.134719in}{1.012602in}}%
\pgfpathlineto{\pgfqpoint{3.191998in}{0.993957in}}%
\pgfpathlineto{\pgfqpoint{3.242201in}{0.975275in}}%
\pgfpathlineto{\pgfqpoint{3.285491in}{0.956674in}}%
\pgfpathlineto{\pgfqpoint{3.322233in}{0.938261in}}%
\pgfpathlineto{\pgfqpoint{3.352990in}{0.920130in}}%
\pgfpathlineto{\pgfqpoint{3.378528in}{0.902366in}}%
\pgfpathlineto{\pgfqpoint{3.399299in}{0.885053in}}%
\pgfpathlineto{\pgfqpoint{3.415571in}{0.868258in}}%
\pgfpathlineto{\pgfqpoint{3.428189in}{0.852016in}}%
\pgfpathlineto{\pgfqpoint{3.437809in}{0.836357in}}%
\pgfpathlineto{\pgfqpoint{3.444886in}{0.821305in}}%
\pgfpathlineto{\pgfqpoint{3.449676in}{0.806883in}}%
\pgfpathlineto{\pgfqpoint{3.452233in}{0.793106in}}%
\pgfpathlineto{\pgfqpoint{3.452417in}{0.779987in}}%
\pgfpathlineto{\pgfqpoint{3.449883in}{0.767535in}}%
\pgfpathlineto{\pgfqpoint{3.444191in}{0.755757in}}%
\pgfpathlineto{\pgfqpoint{3.436109in}{0.744671in}}%
\pgfpathlineto{\pgfqpoint{3.426042in}{0.734287in}}%
\pgfpathlineto{\pgfqpoint{3.414072in}{0.724607in}}%
\pgfpathlineto{\pgfqpoint{3.400245in}{0.715632in}}%
\pgfpathlineto{\pgfqpoint{3.384579in}{0.707362in}}%
\pgfpathlineto{\pgfqpoint{3.357590in}{0.696280in}}%
\pgfpathlineto{\pgfqpoint{3.326234in}{0.686781in}}%
\pgfpathlineto{\pgfqpoint{3.290165in}{0.678860in}}%
\pgfpathlineto{\pgfqpoint{3.249227in}{0.672524in}}%
\pgfpathlineto{\pgfqpoint{3.203190in}{0.667799in}}%
\pgfpathlineto{\pgfqpoint{3.151590in}{0.664713in}}%
\pgfpathlineto{\pgfqpoint{3.093911in}{0.663308in}}%
\pgfpathlineto{\pgfqpoint{3.029586in}{0.663631in}}%
\pgfpathlineto{\pgfqpoint{2.932406in}{0.666858in}}%
\pgfpathlineto{\pgfqpoint{2.820700in}{0.673437in}}%
\pgfpathlineto{\pgfqpoint{2.692780in}{0.683573in}}%
\pgfpathlineto{\pgfqpoint{2.547436in}{0.697534in}}%
\pgfpathlineto{\pgfqpoint{2.384977in}{0.715563in}}%
\pgfpathlineto{\pgfqpoint{2.208138in}{0.737833in}}%
\pgfpathlineto{\pgfqpoint{2.069819in}{0.757330in}}%
\pgfpathlineto{\pgfqpoint{1.931122in}{0.779111in}}%
\pgfpathlineto{\pgfqpoint{1.796702in}{0.802958in}}%
\pgfpathlineto{\pgfqpoint{1.712018in}{0.819853in}}%
\pgfpathlineto{\pgfqpoint{1.632921in}{0.837422in}}%
\pgfpathlineto{\pgfqpoint{1.560091in}{0.855500in}}%
\pgfpathlineto{\pgfqpoint{1.493978in}{0.873928in}}%
\pgfpathlineto{\pgfqpoint{1.434847in}{0.892561in}}%
\pgfpathlineto{\pgfqpoint{1.382775in}{0.911263in}}%
\pgfpathlineto{\pgfqpoint{1.337650in}{0.929912in}}%
\pgfpathlineto{\pgfqpoint{1.299173in}{0.948398in}}%
\pgfpathlineto{\pgfqpoint{1.266859in}{0.966623in}}%
\pgfpathlineto{\pgfqpoint{1.240034in}{0.984500in}}%
\pgfpathlineto{\pgfqpoint{1.217952in}{1.001954in}}%
\pgfpathlineto{\pgfqpoint{1.200545in}{1.018901in}}%
\pgfpathlineto{\pgfqpoint{1.187085in}{1.035302in}}%
\pgfpathlineto{\pgfqpoint{1.176835in}{1.051125in}}%
\pgfpathlineto{\pgfqpoint{1.169253in}{1.066345in}}%
\pgfpathlineto{\pgfqpoint{1.163993in}{1.080940in}}%
\pgfpathlineto{\pgfqpoint{1.160906in}{1.094893in}}%
\pgfpathlineto{\pgfqpoint{1.160038in}{1.108190in}}%
\pgfpathlineto{\pgfqpoint{1.161633in}{1.120821in}}%
\pgfpathlineto{\pgfqpoint{1.166131in}{1.132782in}}%
\pgfpathlineto{\pgfqpoint{1.173618in}{1.144063in}}%
\pgfpathlineto{\pgfqpoint{1.183166in}{1.154643in}}%
\pgfpathlineto{\pgfqpoint{1.194647in}{1.164520in}}%
\pgfpathlineto{\pgfqpoint{1.208001in}{1.173693in}}%
\pgfpathlineto{\pgfqpoint{1.223201in}{1.182159in}}%
\pgfpathlineto{\pgfqpoint{1.249472in}{1.193535in}}%
\pgfpathlineto{\pgfqpoint{1.280042in}{1.203325in}}%
\pgfpathlineto{\pgfqpoint{1.315203in}{1.211531in}}%
\pgfpathlineto{\pgfqpoint{1.355272in}{1.218157in}}%
\pgfpathlineto{\pgfqpoint{1.400352in}{1.223177in}}%
\pgfpathlineto{\pgfqpoint{1.450899in}{1.226565in}}%
\pgfpathlineto{\pgfqpoint{1.507434in}{1.228282in}}%
\pgfpathlineto{\pgfqpoint{1.570523in}{1.228279in}}%
\pgfpathlineto{\pgfqpoint{1.640774in}{1.226498in}}%
\pgfpathlineto{\pgfqpoint{1.746715in}{1.221237in}}%
\pgfpathlineto{\pgfqpoint{1.868213in}{1.212499in}}%
\pgfpathlineto{\pgfqpoint{2.006753in}{1.200047in}}%
\pgfpathlineto{\pgfqpoint{2.162827in}{1.183610in}}%
\pgfpathlineto{\pgfqpoint{2.334639in}{1.162987in}}%
\pgfpathlineto{\pgfqpoint{2.471152in}{1.144712in}}%
\pgfpathlineto{\pgfqpoint{2.610272in}{1.124083in}}%
\pgfpathlineto{\pgfqpoint{2.747465in}{1.101262in}}%
\pgfpathlineto{\pgfqpoint{2.878004in}{1.076521in}}%
\pgfpathlineto{\pgfqpoint{2.958862in}{1.059139in}}%
\pgfpathlineto{\pgfqpoint{3.033261in}{1.041189in}}%
\pgfpathlineto{\pgfqpoint{3.100904in}{1.022835in}}%
\pgfpathlineto{\pgfqpoint{3.161682in}{1.004234in}}%
\pgfpathlineto{\pgfqpoint{3.215594in}{0.985527in}}%
\pgfpathlineto{\pgfqpoint{3.262748in}{0.966842in}}%
\pgfpathlineto{\pgfqpoint{3.303367in}{0.948295in}}%
\pgfpathlineto{\pgfqpoint{3.337779in}{0.929988in}}%
\pgfpathlineto{\pgfqpoint{3.366426in}{0.912009in}}%
\pgfpathlineto{\pgfqpoint{3.389859in}{0.894433in}}%
\pgfpathlineto{\pgfqpoint{3.408738in}{0.877321in}}%
\pgfpathlineto{\pgfqpoint{3.423311in}{0.860745in}}%
\pgfpathlineto{\pgfqpoint{3.434028in}{0.844751in}}%
\pgfpathlineto{\pgfqpoint{3.441640in}{0.829365in}}%
\pgfpathlineto{\pgfqpoint{3.446721in}{0.814605in}}%
\pgfpathlineto{\pgfqpoint{3.449667in}{0.800489in}}%
\pgfpathlineto{\pgfqpoint{3.450696in}{0.787029in}}%
\pgfpathlineto{\pgfqpoint{3.449849in}{0.774235in}}%
\pgfpathlineto{\pgfqpoint{3.446989in}{0.762113in}}%
\pgfpathlineto{\pgfqpoint{3.441803in}{0.750664in}}%
\pgfpathlineto{\pgfqpoint{3.433797in}{0.739888in}}%
\pgfpathlineto{\pgfqpoint{3.422874in}{0.729790in}}%
\pgfpathlineto{\pgfqpoint{3.409972in}{0.720396in}}%
\pgfpathlineto{\pgfqpoint{3.395177in}{0.711707in}}%
\pgfpathlineto{\pgfqpoint{3.369474in}{0.699999in}}%
\pgfpathlineto{\pgfqpoint{3.339522in}{0.689883in}}%
\pgfpathlineto{\pgfqpoint{3.305158in}{0.681363in}}%
\pgfpathlineto{\pgfqpoint{3.266094in}{0.674440in}}%
\pgfpathlineto{\pgfqpoint{3.221922in}{0.669117in}}%
\pgfpathlineto{\pgfqpoint{3.172473in}{0.665412in}}%
\pgfpathlineto{\pgfqpoint{3.117232in}{0.663367in}}%
\pgfpathlineto{\pgfqpoint{3.055491in}{0.663028in}}%
\pgfpathlineto{\pgfqpoint{2.986578in}{0.664455in}}%
\pgfpathlineto{\pgfqpoint{2.882436in}{0.669226in}}%
\pgfpathlineto{\pgfqpoint{2.762988in}{0.677456in}}%
\pgfpathlineto{\pgfqpoint{2.626919in}{0.689373in}}%
\pgfpathlineto{\pgfqpoint{2.472911in}{0.705214in}}%
\pgfpathlineto{\pgfqpoint{2.302200in}{0.725211in}}%
\pgfpathlineto{\pgfqpoint{2.166731in}{0.743052in}}%
\pgfpathlineto{\pgfqpoint{2.028408in}{0.763288in}}%
\pgfpathlineto{\pgfqpoint{1.890971in}{0.785766in}}%
\pgfpathlineto{\pgfqpoint{1.758689in}{0.810224in}}%
\pgfpathlineto{\pgfqpoint{1.675715in}{0.827451in}}%
\pgfpathlineto{\pgfqpoint{1.598706in}{0.845260in}}%
\pgfpathlineto{\pgfqpoint{1.529007in}{0.863505in}}%
\pgfpathlineto{\pgfqpoint{1.466636in}{0.882046in}}%
\pgfpathlineto{\pgfqpoint{1.411241in}{0.900739in}}%
\pgfpathlineto{\pgfqpoint{1.362476in}{0.919456in}}%
\pgfpathlineto{\pgfqpoint{1.319999in}{0.938077in}}%
\pgfpathlineto{\pgfqpoint{1.283480in}{0.956497in}}%
\pgfpathlineto{\pgfqpoint{1.252592in}{0.974620in}}%
\pgfpathlineto{\pgfqpoint{1.227015in}{0.992362in}}%
\pgfpathlineto{\pgfqpoint{1.206437in}{1.009652in}}%
\pgfpathlineto{\pgfqpoint{1.190542in}{1.026427in}}%
\pgfpathlineto{\pgfqpoint{1.178757in}{1.042628in}}%
\pgfpathlineto{\pgfqpoint{1.170371in}{1.058222in}}%
\pgfpathlineto{\pgfqpoint{1.164821in}{1.073187in}}%
\pgfpathlineto{\pgfqpoint{1.161685in}{1.087504in}}%
\pgfpathlineto{\pgfqpoint{1.160689in}{1.101160in}}%
\pgfpathlineto{\pgfqpoint{1.161701in}{1.114143in}}%
\pgfpathlineto{\pgfqpoint{1.164732in}{1.126447in}}%
\pgfpathlineto{\pgfqpoint{1.169938in}{1.138068in}}%
\pgfpathlineto{\pgfqpoint{1.177619in}{1.149008in}}%
\pgfpathlineto{\pgfqpoint{1.187956in}{1.159264in}}%
\pgfpathlineto{\pgfqpoint{1.200314in}{1.168821in}}%
\pgfpathlineto{\pgfqpoint{1.214578in}{1.177674in}}%
\pgfpathlineto{\pgfqpoint{1.230717in}{1.185823in}}%
\pgfpathlineto{\pgfqpoint{1.258440in}{1.196722in}}%
\pgfpathlineto{\pgfqpoint{1.290476in}{1.206030in}}%
\pgfpathlineto{\pgfqpoint{1.327052in}{1.213744in}}%
\pgfpathlineto{\pgfqpoint{1.368528in}{1.219862in}}%
\pgfpathlineto{\pgfqpoint{1.415258in}{1.224379in}}%
\pgfpathlineto{\pgfqpoint{1.467446in}{1.227260in}}%
\pgfpathlineto{\pgfqpoint{1.525780in}{1.228461in}}%
\pgfpathlineto{\pgfqpoint{1.590949in}{1.227929in}}%
\pgfpathlineto{\pgfqpoint{1.689604in}{1.224411in}}%
\pgfpathlineto{\pgfqpoint{1.803033in}{1.217515in}}%
\pgfpathlineto{\pgfqpoint{1.932590in}{1.207024in}}%
\pgfpathlineto{\pgfqpoint{2.079559in}{1.192695in}}%
\pgfpathlineto{\pgfqpoint{2.244407in}{1.174312in}}%
\pgfpathlineto{\pgfqpoint{2.377092in}{1.157711in}}%
\pgfpathlineto{\pgfqpoint{2.514321in}{1.138680in}}%
\pgfpathlineto{\pgfqpoint{2.652627in}{1.117319in}}%
\pgfpathlineto{\pgfqpoint{2.788005in}{1.093834in}}%
\pgfpathlineto{\pgfqpoint{2.874414in}{1.077148in}}%
\pgfpathlineto{\pgfqpoint{2.956041in}{1.059776in}}%
\pgfpathlineto{\pgfqpoint{3.031309in}{1.041858in}}%
\pgfpathlineto{\pgfqpoint{3.099154in}{1.023537in}}%
\pgfpathlineto{\pgfqpoint{3.159771in}{1.004956in}}%
\pgfpathlineto{\pgfqpoint{3.213509in}{0.986255in}}%
\pgfpathlineto{\pgfqpoint{3.260714in}{0.967561in}}%
\pgfpathlineto{\pgfqpoint{3.301720in}{0.948988in}}%
\pgfpathlineto{\pgfqpoint{3.336859in}{0.930640in}}%
\pgfpathlineto{\pgfqpoint{3.366451in}{0.912610in}}%
\pgfpathlineto{\pgfqpoint{3.390811in}{0.894979in}}%
\pgfpathlineto{\pgfqpoint{3.410249in}{0.877815in}}%
\pgfpathlineto{\pgfqpoint{3.425094in}{0.861180in}}%
\pgfpathlineto{\pgfqpoint{3.435990in}{0.845129in}}%
\pgfpathlineto{\pgfqpoint{3.443624in}{0.829691in}}%
\pgfpathlineto{\pgfqpoint{3.448532in}{0.814886in}}%
\pgfpathlineto{\pgfqpoint{3.451105in}{0.800733in}}%
\pgfpathlineto{\pgfqpoint{3.451592in}{0.787245in}}%
\pgfpathlineto{\pgfqpoint{3.450097in}{0.774431in}}%
\pgfpathlineto{\pgfqpoint{3.446579in}{0.762297in}}%
\pgfpathlineto{\pgfqpoint{3.440853in}{0.750845in}}%
\pgfpathlineto{\pgfqpoint{3.432592in}{0.740075in}}%
\pgfpathlineto{\pgfqpoint{3.421732in}{0.729989in}}%
\pgfpathlineto{\pgfqpoint{3.408902in}{0.720605in}}%
\pgfpathlineto{\pgfqpoint{3.394174in}{0.711924in}}%
\pgfpathlineto{\pgfqpoint{3.368567in}{0.700224in}}%
\pgfpathlineto{\pgfqpoint{3.338710in}{0.690114in}}%
\pgfpathlineto{\pgfqpoint{3.304445in}{0.681598in}}%
\pgfpathlineto{\pgfqpoint{3.265484in}{0.674677in}}%
\pgfpathlineto{\pgfqpoint{3.221411in}{0.669353in}}%
\pgfpathlineto{\pgfqpoint{3.172035in}{0.665645in}}%
\pgfpathlineto{\pgfqpoint{3.116891in}{0.663592in}}%
\pgfpathlineto{\pgfqpoint{3.055255in}{0.663245in}}%
\pgfpathlineto{\pgfqpoint{2.986446in}{0.664661in}}%
\pgfpathlineto{\pgfqpoint{2.882436in}{0.669415in}}%
\pgfpathlineto{\pgfqpoint{2.763123in}{0.677625in}}%
\pgfpathlineto{\pgfqpoint{2.627225in}{0.689517in}}%
\pgfpathlineto{\pgfqpoint{2.473349in}{0.705332in}}%
\pgfpathlineto{\pgfqpoint{2.302342in}{0.725285in}}%
\pgfpathlineto{\pgfqpoint{2.166816in}{0.743090in}}%
\pgfpathlineto{\pgfqpoint{2.028759in}{0.763294in}}%
\pgfpathlineto{\pgfqpoint{1.891842in}{0.785749in}}%
\pgfpathlineto{\pgfqpoint{1.760049in}{0.810198in}}%
\pgfpathlineto{\pgfqpoint{1.677142in}{0.827428in}}%
\pgfpathlineto{\pgfqpoint{1.599766in}{0.845247in}}%
\pgfpathlineto{\pgfqpoint{1.529332in}{0.863499in}}%
\pgfpathlineto{\pgfqpoint{1.466631in}{0.882034in}}%
\pgfpathlineto{\pgfqpoint{1.411216in}{0.900721in}}%
\pgfpathlineto{\pgfqpoint{1.362614in}{0.919432in}}%
\pgfpathlineto{\pgfqpoint{1.320387in}{0.938048in}}%
\pgfpathlineto{\pgfqpoint{1.284126in}{0.956463in}}%
\pgfpathlineto{\pgfqpoint{1.253456in}{0.974580in}}%
\pgfpathlineto{\pgfqpoint{1.228035in}{0.992315in}}%
\pgfpathlineto{\pgfqpoint{1.207554in}{1.009594in}}%
\pgfpathlineto{\pgfqpoint{1.191666in}{1.026352in}}%
\pgfpathlineto{\pgfqpoint{1.179663in}{1.042540in}}%
\pgfpathlineto{\pgfqpoint{1.170960in}{1.058126in}}%
\pgfpathlineto{\pgfqpoint{1.165114in}{1.073086in}}%
\pgfpathlineto{\pgfqpoint{1.161804in}{1.087400in}}%
\pgfpathlineto{\pgfqpoint{1.160829in}{1.101052in}}%
\pgfpathlineto{\pgfqpoint{1.162113in}{1.114030in}}%
\pgfpathlineto{\pgfqpoint{1.165700in}{1.126326in}}%
\pgfpathlineto{\pgfqpoint{1.171751in}{1.137939in}}%
\pgfpathlineto{\pgfqpoint{1.180091in}{1.148862in}}%
\pgfpathlineto{\pgfqpoint{1.190434in}{1.159086in}}%
\pgfpathlineto{\pgfqpoint{1.202694in}{1.168611in}}%
\pgfpathlineto{\pgfqpoint{1.216818in}{1.177434in}}%
\pgfpathlineto{\pgfqpoint{1.232784in}{1.185553in}}%
\pgfpathlineto{\pgfqpoint{1.260217in}{1.196412in}}%
\pgfpathlineto{\pgfqpoint{1.291993in}{1.205688in}}%
\pgfpathlineto{\pgfqpoint{1.328438in}{1.213386in}}%
\pgfpathlineto{\pgfqpoint{1.369916in}{1.219505in}}%
\pgfpathlineto{\pgfqpoint{1.416513in}{1.224018in}}%
\pgfpathlineto{\pgfqpoint{1.468718in}{1.226893in}}%
\pgfpathlineto{\pgfqpoint{1.527091in}{1.228086in}}%
\pgfpathlineto{\pgfqpoint{1.592220in}{1.227548in}}%
\pgfpathlineto{\pgfqpoint{1.690640in}{1.224024in}}%
\pgfpathlineto{\pgfqpoint{1.803720in}{1.217130in}}%
\pgfpathlineto{\pgfqpoint{1.933085in}{1.206653in}}%
\pgfpathlineto{\pgfqpoint{2.079868in}{1.192340in}}%
\pgfpathlineto{\pgfqpoint{2.243718in}{1.173931in}}%
\pgfpathlineto{\pgfqpoint{2.421384in}{1.151285in}}%
\pgfpathlineto{\pgfqpoint{2.559962in}{1.131513in}}%
\pgfpathlineto{\pgfqpoint{2.698371in}{1.109476in}}%
\pgfpathlineto{\pgfqpoint{2.831618in}{1.085418in}}%
\pgfpathlineto{\pgfqpoint{2.915369in}{1.068418in}}%
\pgfpathlineto{\pgfqpoint{2.993674in}{1.050790in}}%
\pgfpathlineto{\pgfqpoint{3.065458in}{1.032673in}}%
\pgfpathlineto{\pgfqpoint{3.129824in}{1.014221in}}%
\pgfpathlineto{\pgfqpoint{3.186989in}{0.995580in}}%
\pgfpathlineto{\pgfqpoint{3.237368in}{0.976882in}}%
\pgfpathlineto{\pgfqpoint{3.281353in}{0.958246in}}%
\pgfpathlineto{\pgfqpoint{3.319313in}{0.939784in}}%
\pgfpathlineto{\pgfqpoint{3.351596in}{0.921593in}}%
\pgfpathlineto{\pgfqpoint{3.378525in}{0.903761in}}%
\pgfpathlineto{\pgfqpoint{3.400401in}{0.886363in}}%
\pgfpathlineto{\pgfqpoint{3.417505in}{0.869465in}}%
\pgfpathlineto{\pgfqpoint{3.430369in}{0.853126in}}%
\pgfpathlineto{\pgfqpoint{3.439717in}{0.837386in}}%
\pgfpathlineto{\pgfqpoint{3.446098in}{0.822266in}}%
\pgfpathlineto{\pgfqpoint{3.449925in}{0.807789in}}%
\pgfpathlineto{\pgfqpoint{3.451479in}{0.793970in}}%
\pgfpathlineto{\pgfqpoint{3.450907in}{0.780820in}}%
\pgfpathlineto{\pgfqpoint{3.448222in}{0.768350in}}%
\pgfpathlineto{\pgfqpoint{3.443303in}{0.756562in}}%
\pgfpathlineto{\pgfqpoint{3.435895in}{0.745456in}}%
\pgfpathlineto{\pgfqpoint{3.426046in}{0.735039in}}%
\pgfpathlineto{\pgfqpoint{3.414225in}{0.725321in}}%
\pgfpathlineto{\pgfqpoint{3.400498in}{0.716305in}}%
\pgfpathlineto{\pgfqpoint{3.384901in}{0.707993in}}%
\pgfpathlineto{\pgfqpoint{3.358009in}{0.696845in}}%
\pgfpathlineto{\pgfqpoint{3.326832in}{0.687285in}}%
\pgfpathlineto{\pgfqpoint{3.291139in}{0.679314in}}%
\pgfpathlineto{\pgfqpoint{3.250557in}{0.672931in}}%
\pgfpathlineto{\pgfqpoint{3.204785in}{0.668144in}}%
\pgfpathlineto{\pgfqpoint{3.153593in}{0.664987in}}%
\pgfpathlineto{\pgfqpoint{3.096341in}{0.663502in}}%
\pgfpathlineto{\pgfqpoint{3.032387in}{0.663739in}}%
\pgfpathlineto{\pgfqpoint{2.935593in}{0.666843in}}%
\pgfpathlineto{\pgfqpoint{2.824276in}{0.673298in}}%
\pgfpathlineto{\pgfqpoint{2.696981in}{0.683314in}}%
\pgfpathlineto{\pgfqpoint{2.552346in}{0.697131in}}%
\pgfpathlineto{\pgfqpoint{2.390406in}{0.715001in}}%
\pgfpathlineto{\pgfqpoint{2.214106in}{0.737141in}}%
\pgfpathlineto{\pgfqpoint{2.076078in}{0.756538in}}%
\pgfpathlineto{\pgfqpoint{1.937267in}{0.778212in}}%
\pgfpathlineto{\pgfqpoint{1.802577in}{0.801965in}}%
\pgfpathlineto{\pgfqpoint{1.717663in}{0.818825in}}%
\pgfpathlineto{\pgfqpoint{1.638029in}{0.836341in}}%
\pgfpathlineto{\pgfqpoint{1.564542in}{0.854362in}}%
\pgfpathlineto{\pgfqpoint{1.497808in}{0.872743in}}%
\pgfpathlineto{\pgfqpoint{1.438182in}{0.891347in}}%
\pgfpathlineto{\pgfqpoint{1.385759in}{0.910045in}}%
\pgfpathlineto{\pgfqpoint{1.340379in}{0.928712in}}%
\pgfpathlineto{\pgfqpoint{1.301628in}{0.947233in}}%
\pgfpathlineto{\pgfqpoint{1.268908in}{0.965497in}}%
\pgfpathlineto{\pgfqpoint{1.242032in}{0.983388in}}%
\pgfpathlineto{\pgfqpoint{1.220077in}{1.000844in}}%
\pgfpathlineto{\pgfqpoint{1.202117in}{1.017819in}}%
\pgfpathlineto{\pgfqpoint{1.187475in}{1.034272in}}%
\pgfpathlineto{\pgfqpoint{1.175715in}{1.050167in}}%
\pgfpathlineto{\pgfqpoint{1.166650in}{1.065473in}}%
\pgfpathlineto{\pgfqpoint{1.160335in}{1.080160in}}%
\pgfpathlineto{\pgfqpoint{1.157072in}{1.094207in}}%
\pgfpathlineto{\pgfqpoint{1.157393in}{1.107595in}}%
\pgfpathlineto{\pgfqpoint{1.160661in}{1.120300in}}%
\pgfpathlineto{\pgfqpoint{1.166100in}{1.132308in}}%
\pgfpathlineto{\pgfqpoint{1.173565in}{1.143616in}}%
\pgfpathlineto{\pgfqpoint{1.182952in}{1.154221in}}%
\pgfpathlineto{\pgfqpoint{1.194200in}{1.164121in}}%
\pgfpathlineto{\pgfqpoint{1.207290in}{1.173316in}}%
\pgfpathlineto{\pgfqpoint{1.222244in}{1.181807in}}%
\pgfpathlineto{\pgfqpoint{1.239124in}{1.189596in}}%
\pgfpathlineto{\pgfqpoint{1.268269in}{1.199972in}}%
\pgfpathlineto{\pgfqpoint{1.301964in}{1.208773in}}%
\pgfpathlineto{\pgfqpoint{1.340313in}{1.215984in}}%
\pgfpathlineto{\pgfqpoint{1.383590in}{1.221587in}}%
\pgfpathlineto{\pgfqpoint{1.432157in}{1.225557in}}%
\pgfpathlineto{\pgfqpoint{1.486462in}{1.227861in}}%
\pgfpathlineto{\pgfqpoint{1.547041in}{1.228464in}}%
\pgfpathlineto{\pgfqpoint{1.614518in}{1.227320in}}%
\pgfpathlineto{\pgfqpoint{1.716327in}{1.222987in}}%
\pgfpathlineto{\pgfqpoint{1.833289in}{1.215242in}}%
\pgfpathlineto{\pgfqpoint{1.967373in}{1.203815in}}%
\pgfpathlineto{\pgfqpoint{2.118936in}{1.188469in}}%
\pgfpathlineto{\pgfqpoint{2.286594in}{1.169006in}}%
\pgfpathlineto{\pgfqpoint{2.467260in}{1.145254in}}%
\pgfpathlineto{\pgfqpoint{2.606820in}{1.124642in}}%
\pgfpathlineto{\pgfqpoint{2.744022in}{1.101860in}}%
\pgfpathlineto{\pgfqpoint{2.874246in}{1.077198in}}%
\pgfpathlineto{\pgfqpoint{2.955263in}{1.059882in}}%
\pgfpathlineto{\pgfqpoint{3.030524in}{1.042003in}}%
\pgfpathlineto{\pgfqpoint{3.099252in}{1.023693in}}%
\pgfpathlineto{\pgfqpoint{3.160823in}{1.005096in}}%
\pgfpathlineto{\pgfqpoint{3.214796in}{0.986369in}}%
\pgfpathlineto{\pgfqpoint{3.261444in}{0.967664in}}%
\pgfpathlineto{\pgfqpoint{3.301588in}{0.949092in}}%
\pgfpathlineto{\pgfqpoint{3.335941in}{0.930751in}}%
\pgfpathlineto{\pgfqpoint{3.365076in}{0.912727in}}%
\pgfpathlineto{\pgfqpoint{3.389427in}{0.895101in}}%
\pgfpathlineto{\pgfqpoint{3.409285in}{0.877941in}}%
\pgfpathlineto{\pgfqpoint{3.424804in}{0.861305in}}%
\pgfpathlineto{\pgfqpoint{3.436139in}{0.845244in}}%
\pgfpathlineto{\pgfqpoint{3.444082in}{0.829796in}}%
\pgfpathlineto{\pgfqpoint{3.449100in}{0.814984in}}%
\pgfpathlineto{\pgfqpoint{3.451512in}{0.800826in}}%
\pgfpathlineto{\pgfqpoint{3.451551in}{0.787336in}}%
\pgfpathlineto{\pgfqpoint{3.449365in}{0.774526in}}%
\pgfpathlineto{\pgfqpoint{3.445016in}{0.762404in}}%
\pgfpathlineto{\pgfqpoint{3.438479in}{0.750971in}}%
\pgfpathlineto{\pgfqpoint{3.429713in}{0.740229in}}%
\pgfpathlineto{\pgfqpoint{3.418912in}{0.730183in}}%
\pgfpathlineto{\pgfqpoint{3.406174in}{0.720836in}}%
\pgfpathlineto{\pgfqpoint{3.391558in}{0.712189in}}%
\pgfpathlineto{\pgfqpoint{3.366157in}{0.700536in}}%
\pgfpathlineto{\pgfqpoint{3.336540in}{0.690469in}}%
\pgfpathlineto{\pgfqpoint{3.302517in}{0.681988in}}%
\pgfpathlineto{\pgfqpoint{3.263748in}{0.675093in}}%
\pgfpathlineto{\pgfqpoint{3.219765in}{0.669782in}}%
\pgfpathlineto{\pgfqpoint{3.170332in}{0.666092in}}%
\pgfpathlineto{\pgfqpoint{3.134168in}{0.664551in}}%
\pgfpathlineto{\pgfqpoint{3.134168in}{0.664551in}}%
\pgfusepath{stroke}%
\end{pgfscope}%
\begin{pgfscope}%
\pgfpathrectangle{\pgfqpoint{0.562500in}{0.275000in}}{\pgfqpoint{3.487500in}{1.925000in}}%
\pgfusepath{clip}%
\pgfsetrectcap%
\pgfsetroundjoin%
\pgfsetlinewidth{1.505625pt}%
\definecolor{currentstroke}{rgb}{0.172549,0.627451,0.172549}%
\pgfsetstrokecolor{currentstroke}%
\pgfsetdash{}{0pt}%
\pgfpathmoveto{\pgfqpoint{1.909943in}{2.112500in}}%
\pgfpathlineto{\pgfqpoint{2.066790in}{2.086566in}}%
\pgfpathlineto{\pgfqpoint{2.207268in}{2.059923in}}%
\pgfpathlineto{\pgfqpoint{2.330800in}{2.032738in}}%
\pgfpathlineto{\pgfqpoint{2.437798in}{2.005175in}}%
\pgfpathlineto{\pgfqpoint{2.529571in}{1.977388in}}%
\pgfpathlineto{\pgfqpoint{2.608138in}{1.949525in}}%
\pgfpathlineto{\pgfqpoint{2.675508in}{1.921687in}}%
\pgfpathlineto{\pgfqpoint{2.733537in}{1.893961in}}%
\pgfpathlineto{\pgfqpoint{2.783930in}{1.866425in}}%
\pgfpathlineto{\pgfqpoint{2.828257in}{1.839148in}}%
\pgfpathlineto{\pgfqpoint{2.867761in}{1.812167in}}%
\pgfpathlineto{\pgfqpoint{2.903327in}{1.785508in}}%
\pgfpathlineto{\pgfqpoint{2.935729in}{1.759194in}}%
\pgfpathlineto{\pgfqpoint{2.993555in}{1.707690in}}%
\pgfpathlineto{\pgfqpoint{3.044487in}{1.657727in}}%
\pgfpathlineto{\pgfqpoint{3.090339in}{1.609326in}}%
\pgfpathlineto{\pgfqpoint{3.152526in}{1.539683in}}%
\pgfpathlineto{\pgfqpoint{3.208065in}{1.473501in}}%
\pgfpathlineto{\pgfqpoint{3.273956in}{1.390505in}}%
\pgfpathlineto{\pgfqpoint{3.317904in}{1.332046in}}%
\pgfpathlineto{\pgfqpoint{3.357173in}{1.276685in}}%
\pgfpathlineto{\pgfqpoint{3.403231in}{1.207552in}}%
\pgfpathlineto{\pgfqpoint{3.433285in}{1.159079in}}%
\pgfpathlineto{\pgfqpoint{3.459324in}{1.113342in}}%
\pgfpathlineto{\pgfqpoint{3.481437in}{1.070244in}}%
\pgfpathlineto{\pgfqpoint{3.499933in}{1.029731in}}%
\pgfpathlineto{\pgfqpoint{3.514923in}{0.991735in}}%
\pgfpathlineto{\pgfqpoint{3.526478in}{0.956132in}}%
\pgfpathlineto{\pgfqpoint{3.534706in}{0.922820in}}%
\pgfpathlineto{\pgfqpoint{3.539601in}{0.891722in}}%
\pgfpathlineto{\pgfqpoint{3.541045in}{0.862789in}}%
\pgfpathlineto{\pgfqpoint{3.540008in}{0.844683in}}%
\pgfpathlineto{\pgfqpoint{3.537446in}{0.827496in}}%
\pgfpathlineto{\pgfqpoint{3.533438in}{0.811206in}}%
\pgfpathlineto{\pgfqpoint{3.528027in}{0.795794in}}%
\pgfpathlineto{\pgfqpoint{3.521224in}{0.781241in}}%
\pgfpathlineto{\pgfqpoint{3.513010in}{0.767534in}}%
\pgfpathlineto{\pgfqpoint{3.503334in}{0.754655in}}%
\pgfpathlineto{\pgfqpoint{3.492113in}{0.742594in}}%
\pgfpathlineto{\pgfqpoint{3.479232in}{0.731337in}}%
\pgfpathlineto{\pgfqpoint{3.464621in}{0.720871in}}%
\pgfpathlineto{\pgfqpoint{3.448339in}{0.711187in}}%
\pgfpathlineto{\pgfqpoint{3.420841in}{0.698115in}}%
\pgfpathlineto{\pgfqpoint{3.389631in}{0.686771in}}%
\pgfpathlineto{\pgfqpoint{3.354552in}{0.677142in}}%
\pgfpathlineto{\pgfqpoint{3.315298in}{0.669215in}}%
\pgfpathlineto{\pgfqpoint{3.271419in}{0.662981in}}%
\pgfpathlineto{\pgfqpoint{3.222320in}{0.658434in}}%
\pgfpathlineto{\pgfqpoint{3.167390in}{0.655573in}}%
\pgfpathlineto{\pgfqpoint{3.106512in}{0.654445in}}%
\pgfpathlineto{\pgfqpoint{3.038766in}{0.655114in}}%
\pgfpathlineto{\pgfqpoint{2.963229in}{0.657651in}}%
\pgfpathlineto{\pgfqpoint{2.849105in}{0.664072in}}%
\pgfpathlineto{\pgfqpoint{2.718469in}{0.674160in}}%
\pgfpathlineto{\pgfqpoint{2.570516in}{0.688150in}}%
\pgfpathlineto{\pgfqpoint{2.404931in}{0.706306in}}%
\pgfpathlineto{\pgfqpoint{2.222209in}{0.728916in}}%
\pgfpathlineto{\pgfqpoint{2.079802in}{0.748760in}}%
\pgfpathlineto{\pgfqpoint{1.938371in}{0.770910in}}%
\pgfpathlineto{\pgfqpoint{1.802311in}{0.795128in}}%
\pgfpathlineto{\pgfqpoint{1.716500in}{0.812272in}}%
\pgfpathlineto{\pgfqpoint{1.635775in}{0.830088in}}%
\pgfpathlineto{\pgfqpoint{1.561015in}{0.848449in}}%
\pgfpathlineto{\pgfqpoint{1.492983in}{0.867213in}}%
\pgfpathlineto{\pgfqpoint{1.432328in}{0.886226in}}%
\pgfpathlineto{\pgfqpoint{1.379553in}{0.905315in}}%
\pgfpathlineto{\pgfqpoint{1.334181in}{0.924341in}}%
\pgfpathlineto{\pgfqpoint{1.295387in}{0.943194in}}%
\pgfpathlineto{\pgfqpoint{1.262550in}{0.961773in}}%
\pgfpathlineto{\pgfqpoint{1.235121in}{0.979987in}}%
\pgfpathlineto{\pgfqpoint{1.212622in}{0.997759in}}%
\pgfpathlineto{\pgfqpoint{1.194648in}{1.015026in}}%
\pgfpathlineto{\pgfqpoint{1.180691in}{1.031732in}}%
\pgfpathlineto{\pgfqpoint{1.170238in}{1.047843in}}%
\pgfpathlineto{\pgfqpoint{1.162907in}{1.063329in}}%
\pgfpathlineto{\pgfqpoint{1.158402in}{1.078165in}}%
\pgfpathlineto{\pgfqpoint{1.156510in}{1.092334in}}%
\pgfpathlineto{\pgfqpoint{1.157105in}{1.105822in}}%
\pgfpathlineto{\pgfqpoint{1.160106in}{1.118622in}}%
\pgfpathlineto{\pgfqpoint{1.165328in}{1.130727in}}%
\pgfpathlineto{\pgfqpoint{1.172587in}{1.142132in}}%
\pgfpathlineto{\pgfqpoint{1.181750in}{1.152833in}}%
\pgfpathlineto{\pgfqpoint{1.192731in}{1.162829in}}%
\pgfpathlineto{\pgfqpoint{1.205495in}{1.172119in}}%
\pgfpathlineto{\pgfqpoint{1.220053in}{1.180703in}}%
\pgfpathlineto{\pgfqpoint{1.236466in}{1.188585in}}%
\pgfpathlineto{\pgfqpoint{1.264819in}{1.199098in}}%
\pgfpathlineto{\pgfqpoint{1.298027in}{1.208051in}}%
\pgfpathlineto{\pgfqpoint{1.335923in}{1.215421in}}%
\pgfpathlineto{\pgfqpoint{1.378725in}{1.221187in}}%
\pgfpathlineto{\pgfqpoint{1.426793in}{1.225326in}}%
\pgfpathlineto{\pgfqpoint{1.480569in}{1.227805in}}%
\pgfpathlineto{\pgfqpoint{1.540573in}{1.228584in}}%
\pgfpathlineto{\pgfqpoint{1.607403in}{1.227614in}}%
\pgfpathlineto{\pgfqpoint{1.708309in}{1.223505in}}%
\pgfpathlineto{\pgfqpoint{1.824184in}{1.216008in}}%
\pgfpathlineto{\pgfqpoint{1.956887in}{1.204860in}}%
\pgfpathlineto{\pgfqpoint{2.107116in}{1.189813in}}%
\pgfpathlineto{\pgfqpoint{2.273807in}{1.170651in}}%
\pgfpathlineto{\pgfqpoint{2.453933in}{1.147209in}}%
\pgfpathlineto{\pgfqpoint{2.593026in}{1.126856in}}%
\pgfpathlineto{\pgfqpoint{2.730489in}{1.104295in}}%
\pgfpathlineto{\pgfqpoint{2.861895in}{1.079781in}}%
\pgfpathlineto{\pgfqpoint{2.944008in}{1.062525in}}%
\pgfpathlineto{\pgfqpoint{3.020274in}{1.044686in}}%
\pgfpathlineto{\pgfqpoint{3.089562in}{1.026411in}}%
\pgfpathlineto{\pgfqpoint{3.151578in}{1.007850in}}%
\pgfpathlineto{\pgfqpoint{3.206530in}{0.989146in}}%
\pgfpathlineto{\pgfqpoint{3.254676in}{0.970430in}}%
\pgfpathlineto{\pgfqpoint{3.296322in}{0.951822in}}%
\pgfpathlineto{\pgfqpoint{3.331824in}{0.933430in}}%
\pgfpathlineto{\pgfqpoint{3.361581in}{0.915350in}}%
\pgfpathlineto{\pgfqpoint{3.386045in}{0.897666in}}%
\pgfpathlineto{\pgfqpoint{3.405702in}{0.880452in}}%
\pgfpathlineto{\pgfqpoint{3.421121in}{0.863767in}}%
\pgfpathlineto{\pgfqpoint{3.432923in}{0.847650in}}%
\pgfpathlineto{\pgfqpoint{3.441597in}{0.832130in}}%
\pgfpathlineto{\pgfqpoint{3.447499in}{0.817236in}}%
\pgfpathlineto{\pgfqpoint{3.450850in}{0.802987in}}%
\pgfpathlineto{\pgfqpoint{3.451733in}{0.789400in}}%
\pgfpathlineto{\pgfqpoint{3.450097in}{0.776486in}}%
\pgfpathlineto{\pgfqpoint{3.445842in}{0.764251in}}%
\pgfpathlineto{\pgfqpoint{3.439347in}{0.752707in}}%
\pgfpathlineto{\pgfqpoint{3.430812in}{0.741858in}}%
\pgfpathlineto{\pgfqpoint{3.420346in}{0.731708in}}%
\pgfpathlineto{\pgfqpoint{3.408019in}{0.722259in}}%
\pgfpathlineto{\pgfqpoint{3.393862in}{0.713512in}}%
\pgfpathlineto{\pgfqpoint{3.377871in}{0.705468in}}%
\pgfpathlineto{\pgfqpoint{3.350340in}{0.694716in}}%
\pgfpathlineto{\pgfqpoint{3.318272in}{0.685534in}}%
\pgfpathlineto{\pgfqpoint{3.281355in}{0.677918in}}%
\pgfpathlineto{\pgfqpoint{3.239590in}{0.671892in}}%
\pgfpathlineto{\pgfqpoint{3.192620in}{0.667480in}}%
\pgfpathlineto{\pgfqpoint{3.140000in}{0.664714in}}%
\pgfpathlineto{\pgfqpoint{3.081223in}{0.663635in}}%
\pgfpathlineto{\pgfqpoint{3.015716in}{0.664293in}}%
\pgfpathlineto{\pgfqpoint{2.916792in}{0.667978in}}%
\pgfpathlineto{\pgfqpoint{2.803083in}{0.675025in}}%
\pgfpathlineto{\pgfqpoint{2.672956in}{0.685664in}}%
\pgfpathlineto{\pgfqpoint{2.525330in}{0.700163in}}%
\pgfpathlineto{\pgfqpoint{2.360891in}{0.718750in}}%
\pgfpathlineto{\pgfqpoint{2.182646in}{0.741589in}}%
\pgfpathlineto{\pgfqpoint{2.044133in}{0.761497in}}%
\pgfpathlineto{\pgfqpoint{1.906021in}{0.783654in}}%
\pgfpathlineto{\pgfqpoint{1.773049in}{0.807822in}}%
\pgfpathlineto{\pgfqpoint{1.689876in}{0.824900in}}%
\pgfpathlineto{\pgfqpoint{1.612519in}{0.842609in}}%
\pgfpathlineto{\pgfqpoint{1.541558in}{0.860782in}}%
\pgfpathlineto{\pgfqpoint{1.477385in}{0.879265in}}%
\pgfpathlineto{\pgfqpoint{1.420210in}{0.897914in}}%
\pgfpathlineto{\pgfqpoint{1.370059in}{0.916596in}}%
\pgfpathlineto{\pgfqpoint{1.326776in}{0.935194in}}%
\pgfpathlineto{\pgfqpoint{1.290022in}{0.953600in}}%
\pgfpathlineto{\pgfqpoint{1.259275in}{0.971721in}}%
\pgfpathlineto{\pgfqpoint{1.233831in}{0.989474in}}%
\pgfpathlineto{\pgfqpoint{1.212995in}{1.006785in}}%
\pgfpathlineto{\pgfqpoint{1.196708in}{1.023576in}}%
\pgfpathlineto{\pgfqpoint{1.184208in}{1.039810in}}%
\pgfpathlineto{\pgfqpoint{1.174799in}{1.055459in}}%
\pgfpathlineto{\pgfqpoint{1.167974in}{1.070498in}}%
\pgfpathlineto{\pgfqpoint{1.163417in}{1.084907in}}%
\pgfpathlineto{\pgfqpoint{1.161005in}{1.098668in}}%
\pgfpathlineto{\pgfqpoint{1.160805in}{1.111771in}}%
\pgfpathlineto{\pgfqpoint{1.163075in}{1.124207in}}%
\pgfpathlineto{\pgfqpoint{1.168266in}{1.135971in}}%
\pgfpathlineto{\pgfqpoint{1.176326in}{1.147053in}}%
\pgfpathlineto{\pgfqpoint{1.186402in}{1.157435in}}%
\pgfpathlineto{\pgfqpoint{1.198400in}{1.167113in}}%
\pgfpathlineto{\pgfqpoint{1.212267in}{1.176087in}}%
\pgfpathlineto{\pgfqpoint{1.227980in}{1.184355in}}%
\pgfpathlineto{\pgfqpoint{1.255037in}{1.195435in}}%
\pgfpathlineto{\pgfqpoint{1.286424in}{1.204927in}}%
\pgfpathlineto{\pgfqpoint{1.322446in}{1.212837in}}%
\pgfpathlineto{\pgfqpoint{1.363414in}{1.219164in}}%
\pgfpathlineto{\pgfqpoint{1.409451in}{1.223883in}}%
\pgfpathlineto{\pgfqpoint{1.461044in}{1.226963in}}%
\pgfpathlineto{\pgfqpoint{1.518731in}{1.228365in}}%
\pgfpathlineto{\pgfqpoint{1.583092in}{1.228039in}}%
\pgfpathlineto{\pgfqpoint{1.680356in}{1.224808in}}%
\pgfpathlineto{\pgfqpoint{1.792143in}{1.218220in}}%
\pgfpathlineto{\pgfqpoint{1.920111in}{1.208069in}}%
\pgfpathlineto{\pgfqpoint{2.065474in}{1.194099in}}%
\pgfpathlineto{\pgfqpoint{2.228028in}{1.176053in}}%
\pgfpathlineto{\pgfqpoint{2.404841in}{1.153772in}}%
\pgfpathlineto{\pgfqpoint{2.543281in}{1.134263in}}%
\pgfpathlineto{\pgfqpoint{2.681999in}{1.112472in}}%
\pgfpathlineto{\pgfqpoint{2.816255in}{1.088618in}}%
\pgfpathlineto{\pgfqpoint{2.901001in}{1.071727in}}%
\pgfpathlineto{\pgfqpoint{2.980373in}{1.054184in}}%
\pgfpathlineto{\pgfqpoint{3.052732in}{1.036129in}}%
\pgfpathlineto{\pgfqpoint{3.117841in}{1.017714in}}%
\pgfpathlineto{\pgfqpoint{3.176061in}{0.999088in}}%
\pgfpathlineto{\pgfqpoint{3.227736in}{0.980385in}}%
\pgfpathlineto{\pgfqpoint{3.273193in}{0.961728in}}%
\pgfpathlineto{\pgfqpoint{3.312739in}{0.943228in}}%
\pgfpathlineto{\pgfqpoint{3.346666in}{0.924983in}}%
\pgfpathlineto{\pgfqpoint{3.375246in}{0.907079in}}%
\pgfpathlineto{\pgfqpoint{3.398735in}{0.889589in}}%
\pgfpathlineto{\pgfqpoint{3.417370in}{0.872576in}}%
\pgfpathlineto{\pgfqpoint{3.431370in}{0.856088in}}%
\pgfpathlineto{\pgfqpoint{3.441126in}{0.840180in}}%
\pgfpathlineto{\pgfqpoint{3.447436in}{0.824904in}}%
\pgfpathlineto{\pgfqpoint{3.450945in}{0.810278in}}%
\pgfpathlineto{\pgfqpoint{3.452153in}{0.796314in}}%
\pgfpathlineto{\pgfqpoint{3.451417in}{0.783024in}}%
\pgfpathlineto{\pgfqpoint{3.448951in}{0.770415in}}%
\pgfpathlineto{\pgfqpoint{3.444825in}{0.758491in}}%
\pgfpathlineto{\pgfqpoint{3.438968in}{0.747255in}}%
\pgfpathlineto{\pgfqpoint{3.431163in}{0.736704in}}%
\pgfpathlineto{\pgfqpoint{3.421052in}{0.726833in}}%
\pgfpathlineto{\pgfqpoint{3.408135in}{0.717633in}}%
\pgfpathlineto{\pgfqpoint{3.392732in}{0.709121in}}%
\pgfpathlineto{\pgfqpoint{3.366050in}{0.697678in}}%
\pgfpathlineto{\pgfqpoint{3.335047in}{0.687828in}}%
\pgfpathlineto{\pgfqpoint{3.299590in}{0.679578in}}%
\pgfpathlineto{\pgfqpoint{3.259437in}{0.672940in}}%
\pgfpathlineto{\pgfqpoint{3.214242in}{0.667922in}}%
\pgfpathlineto{\pgfqpoint{3.163548in}{0.664540in}}%
\pgfpathlineto{\pgfqpoint{3.107067in}{0.662810in}}%
\pgfpathlineto{\pgfqpoint{3.044333in}{0.662783in}}%
\pgfpathlineto{\pgfqpoint{2.974241in}{0.664536in}}%
\pgfpathlineto{\pgfqpoint{2.867847in}{0.669786in}}%
\pgfpathlineto{\pgfqpoint{2.745388in}{0.678550in}}%
\pgfpathlineto{\pgfqpoint{2.606165in}{0.691043in}}%
\pgfpathlineto{\pgfqpoint{2.450226in}{0.707493in}}%
\pgfpathlineto{\pgfqpoint{2.278309in}{0.728142in}}%
\pgfpathlineto{\pgfqpoint{2.140518in}{0.746533in}}%
\pgfpathlineto{\pgfqpoint{2.000556in}{0.767290in}}%
\pgfpathlineto{\pgfqpoint{1.863578in}{0.790178in}}%
\pgfpathlineto{\pgfqpoint{1.733936in}{0.814910in}}%
\pgfpathlineto{\pgfqpoint{1.653371in}{0.832256in}}%
\pgfpathlineto{\pgfqpoint{1.578516in}{0.850155in}}%
\pgfpathlineto{\pgfqpoint{1.510068in}{0.868482in}}%
\pgfpathlineto{\pgfqpoint{1.448565in}{0.887101in}}%
\pgfpathlineto{\pgfqpoint{1.394389in}{0.905864in}}%
\pgfpathlineto{\pgfqpoint{1.347719in}{0.924613in}}%
\pgfpathlineto{\pgfqpoint{1.307973in}{0.943211in}}%
\pgfpathlineto{\pgfqpoint{1.274237in}{0.961563in}}%
\pgfpathlineto{\pgfqpoint{1.245768in}{0.979583in}}%
\pgfpathlineto{\pgfqpoint{1.221989in}{0.997200in}}%
\pgfpathlineto{\pgfqpoint{1.202483in}{1.014346in}}%
\pgfpathlineto{\pgfqpoint{1.187001in}{1.030966in}}%
\pgfpathlineto{\pgfqpoint{1.175455in}{1.047014in}}%
\pgfpathlineto{\pgfqpoint{1.167598in}{1.062450in}}%
\pgfpathlineto{\pgfqpoint{1.162716in}{1.077246in}}%
\pgfpathlineto{\pgfqpoint{1.160486in}{1.091386in}}%
\pgfpathlineto{\pgfqpoint{1.160667in}{1.104855in}}%
\pgfpathlineto{\pgfqpoint{1.163085in}{1.117643in}}%
\pgfpathlineto{\pgfqpoint{1.167639in}{1.129742in}}%
\pgfpathlineto{\pgfqpoint{1.174297in}{1.141148in}}%
\pgfpathlineto{\pgfqpoint{1.183094in}{1.151861in}}%
\pgfpathlineto{\pgfqpoint{1.193960in}{1.161879in}}%
\pgfpathlineto{\pgfqpoint{1.206782in}{1.171200in}}%
\pgfpathlineto{\pgfqpoint{1.221498in}{1.179820in}}%
\pgfpathlineto{\pgfqpoint{1.247068in}{1.191434in}}%
\pgfpathlineto{\pgfqpoint{1.276857in}{1.201463in}}%
\pgfpathlineto{\pgfqpoint{1.311029in}{1.209902in}}%
\pgfpathlineto{\pgfqpoint{1.349895in}{1.216750in}}%
\pgfpathlineto{\pgfqpoint{1.393909in}{1.222008in}}%
\pgfpathlineto{\pgfqpoint{1.443364in}{1.225661in}}%
\pgfpathlineto{\pgfqpoint{1.498555in}{1.227662in}}%
\pgfpathlineto{\pgfqpoint{1.560237in}{1.227960in}}%
\pgfpathlineto{\pgfqpoint{1.629116in}{1.226497in}}%
\pgfpathlineto{\pgfqpoint{1.733268in}{1.221681in}}%
\pgfpathlineto{\pgfqpoint{1.852759in}{1.213410in}}%
\pgfpathlineto{\pgfqpoint{1.988813in}{1.201457in}}%
\pgfpathlineto{\pgfqpoint{2.142624in}{1.185575in}}%
\pgfpathlineto{\pgfqpoint{2.313516in}{1.165549in}}%
\pgfpathlineto{\pgfqpoint{2.448994in}{1.147703in}}%
\pgfpathlineto{\pgfqpoint{2.586990in}{1.127471in}}%
\pgfpathlineto{\pgfqpoint{2.723782in}{1.105005in}}%
\pgfpathlineto{\pgfqpoint{2.855332in}{1.080561in}}%
\pgfpathlineto{\pgfqpoint{2.937978in}{1.063342in}}%
\pgfpathlineto{\pgfqpoint{3.014991in}{1.045537in}}%
\pgfpathlineto{\pgfqpoint{3.084961in}{1.027302in}}%
\pgfpathlineto{\pgfqpoint{3.147344in}{1.008780in}}%
\pgfpathlineto{\pgfqpoint{3.202550in}{0.990108in}}%
\pgfpathlineto{\pgfqpoint{3.250991in}{0.971415in}}%
\pgfpathlineto{\pgfqpoint{3.293065in}{0.952820in}}%
\pgfpathlineto{\pgfqpoint{3.329152in}{0.934430in}}%
\pgfpathlineto{\pgfqpoint{3.359618in}{0.916340in}}%
\pgfpathlineto{\pgfqpoint{3.384813in}{0.898634in}}%
\pgfpathlineto{\pgfqpoint{3.405070in}{0.881387in}}%
\pgfpathlineto{\pgfqpoint{3.420797in}{0.864662in}}%
\pgfpathlineto{\pgfqpoint{3.432690in}{0.848508in}}%
\pgfpathlineto{\pgfqpoint{3.441319in}{0.832955in}}%
\pgfpathlineto{\pgfqpoint{3.447124in}{0.818027in}}%
\pgfpathlineto{\pgfqpoint{3.450416in}{0.803745in}}%
\pgfpathlineto{\pgfqpoint{3.451381in}{0.790124in}}%
\pgfpathlineto{\pgfqpoint{3.450080in}{0.777176in}}%
\pgfpathlineto{\pgfqpoint{3.446445in}{0.764908in}}%
\pgfpathlineto{\pgfqpoint{3.440299in}{0.753323in}}%
\pgfpathlineto{\pgfqpoint{3.431891in}{0.742429in}}%
\pgfpathlineto{\pgfqpoint{3.421485in}{0.732232in}}%
\pgfpathlineto{\pgfqpoint{3.409167in}{0.722736in}}%
\pgfpathlineto{\pgfqpoint{3.394986in}{0.713941in}}%
\pgfpathlineto{\pgfqpoint{3.378963in}{0.705849in}}%
\pgfpathlineto{\pgfqpoint{3.351440in}{0.695031in}}%
\pgfpathlineto{\pgfqpoint{3.319561in}{0.685795in}}%
\pgfpathlineto{\pgfqpoint{3.282989in}{0.678137in}}%
\pgfpathlineto{\pgfqpoint{3.241391in}{0.672057in}}%
\pgfpathlineto{\pgfqpoint{3.194667in}{0.667584in}}%
\pgfpathlineto{\pgfqpoint{3.142317in}{0.664752in}}%
\pgfpathlineto{\pgfqpoint{3.083787in}{0.663601in}}%
\pgfpathlineto{\pgfqpoint{3.018490in}{0.664184in}}%
\pgfpathlineto{\pgfqpoint{2.919829in}{0.667768in}}%
\pgfpathlineto{\pgfqpoint{2.806481in}{0.674727in}}%
\pgfpathlineto{\pgfqpoint{2.676812in}{0.685271in}}%
\pgfpathlineto{\pgfqpoint{2.529709in}{0.699658in}}%
\pgfpathlineto{\pgfqpoint{2.365585in}{0.718141in}}%
\pgfpathlineto{\pgfqpoint{2.187733in}{0.740864in}}%
\pgfpathlineto{\pgfqpoint{2.049093in}{0.760693in}}%
\pgfpathlineto{\pgfqpoint{1.910808in}{0.782779in}}%
\pgfpathlineto{\pgfqpoint{1.777743in}{0.806881in}}%
\pgfpathlineto{\pgfqpoint{1.694147in}{0.823906in}}%
\pgfpathlineto{\pgfqpoint{1.616067in}{0.841555in}}%
\pgfpathlineto{\pgfqpoint{1.544832in}{0.859681in}}%
\pgfpathlineto{\pgfqpoint{1.481204in}{0.878125in}}%
\pgfpathlineto{\pgfqpoint{1.424640in}{0.896750in}}%
\pgfpathlineto{\pgfqpoint{1.374616in}{0.915429in}}%
\pgfpathlineto{\pgfqpoint{1.330675in}{0.934046in}}%
\pgfpathlineto{\pgfqpoint{1.292431in}{0.952496in}}%
\pgfpathlineto{\pgfqpoint{1.259567in}{0.970682in}}%
\pgfpathlineto{\pgfqpoint{1.231834in}{0.988519in}}%
\pgfpathlineto{\pgfqpoint{1.209051in}{1.005931in}}%
\pgfpathlineto{\pgfqpoint{1.191110in}{1.022855in}}%
\pgfpathlineto{\pgfqpoint{1.177966in}{1.039234in}}%
\pgfpathlineto{\pgfqpoint{1.168955in}{1.055013in}}%
\pgfpathlineto{\pgfqpoint{1.163130in}{1.070157in}}%
\pgfpathlineto{\pgfqpoint{1.159995in}{1.084648in}}%
\pgfpathlineto{\pgfqpoint{1.159172in}{1.098471in}}%
\pgfpathlineto{\pgfqpoint{1.160403in}{1.111617in}}%
\pgfpathlineto{\pgfqpoint{1.163547in}{1.124076in}}%
\pgfpathlineto{\pgfqpoint{1.168581in}{1.135845in}}%
\pgfpathlineto{\pgfqpoint{1.175602in}{1.146922in}}%
\pgfpathlineto{\pgfqpoint{1.184824in}{1.157311in}}%
\pgfpathlineto{\pgfqpoint{1.196554in}{1.167016in}}%
\pgfpathlineto{\pgfqpoint{1.210456in}{1.176025in}}%
\pgfpathlineto{\pgfqpoint{1.226268in}{1.184330in}}%
\pgfpathlineto{\pgfqpoint{1.253540in}{1.195466in}}%
\pgfpathlineto{\pgfqpoint{1.285126in}{1.205011in}}%
\pgfpathlineto{\pgfqpoint{1.321190in}{1.212956in}}%
\pgfpathlineto{\pgfqpoint{1.362015in}{1.219296in}}%
\pgfpathlineto{\pgfqpoint{1.408000in}{1.224022in}}%
\pgfpathlineto{\pgfqpoint{1.459553in}{1.227124in}}%
\pgfpathlineto{\pgfqpoint{1.516881in}{1.228563in}}%
\pgfpathlineto{\pgfqpoint{1.580884in}{1.228280in}}%
\pgfpathlineto{\pgfqpoint{1.678053in}{1.225103in}}%
\pgfpathlineto{\pgfqpoint{1.790198in}{1.218556in}}%
\pgfpathlineto{\pgfqpoint{1.918481in}{1.208433in}}%
\pgfpathlineto{\pgfqpoint{2.063603in}{1.194500in}}%
\pgfpathlineto{\pgfqpoint{2.225877in}{1.176501in}}%
\pgfpathlineto{\pgfqpoint{2.404602in}{1.154163in}}%
\pgfpathlineto{\pgfqpoint{2.543614in}{1.134608in}}%
\pgfpathlineto{\pgfqpoint{2.681616in}{1.112816in}}%
\pgfpathlineto{\pgfqpoint{2.814441in}{1.089016in}}%
\pgfpathlineto{\pgfqpoint{2.898298in}{1.072179in}}%
\pgfpathlineto{\pgfqpoint{2.977279in}{1.054689in}}%
\pgfpathlineto{\pgfqpoint{3.050543in}{1.036667in}}%
\pgfpathlineto{\pgfqpoint{3.117352in}{1.018250in}}%
\pgfpathlineto{\pgfqpoint{3.177074in}{0.999587in}}%
\pgfpathlineto{\pgfqpoint{3.229197in}{0.980843in}}%
\pgfpathlineto{\pgfqpoint{3.274075in}{0.962158in}}%
\pgfpathlineto{\pgfqpoint{3.312515in}{0.943637in}}%
\pgfpathlineto{\pgfqpoint{3.345128in}{0.925380in}}%
\pgfpathlineto{\pgfqpoint{3.372445in}{0.907474in}}%
\pgfpathlineto{\pgfqpoint{3.394921in}{0.889995in}}%
\pgfpathlineto{\pgfqpoint{3.412931in}{0.873008in}}%
\pgfpathlineto{\pgfqpoint{3.426867in}{0.856566in}}%
\pgfpathlineto{\pgfqpoint{3.437274in}{0.840709in}}%
\pgfpathlineto{\pgfqpoint{3.444546in}{0.825466in}}%
\pgfpathlineto{\pgfqpoint{3.448990in}{0.810860in}}%
\pgfpathlineto{\pgfqpoint{3.450834in}{0.796912in}}%
\pgfpathlineto{\pgfqpoint{3.450225in}{0.783635in}}%
\pgfpathlineto{\pgfqpoint{3.447235in}{0.771037in}}%
\pgfpathlineto{\pgfqpoint{3.441959in}{0.759125in}}%
\pgfpathlineto{\pgfqpoint{3.434573in}{0.747904in}}%
\pgfpathlineto{\pgfqpoint{3.425220in}{0.737379in}}%
\pgfpathlineto{\pgfqpoint{3.413998in}{0.727552in}}%
\pgfpathlineto{\pgfqpoint{3.400964in}{0.718427in}}%
\pgfpathlineto{\pgfqpoint{3.386129in}{0.710003in}}%
\pgfpathlineto{\pgfqpoint{3.360427in}{0.698680in}}%
\pgfpathlineto{\pgfqpoint{3.330314in}{0.688926in}}%
\pgfpathlineto{\pgfqpoint{3.295321in}{0.680728in}}%
\pgfpathlineto{\pgfqpoint{3.255449in}{0.674105in}}%
\pgfpathlineto{\pgfqpoint{3.210522in}{0.669085in}}%
\pgfpathlineto{\pgfqpoint{3.160121in}{0.665696in}}%
\pgfpathlineto{\pgfqpoint{3.103762in}{0.663976in}}%
\pgfpathlineto{\pgfqpoint{3.040897in}{0.663972in}}%
\pgfpathlineto{\pgfqpoint{2.970913in}{0.665740in}}%
\pgfpathlineto{\pgfqpoint{2.865349in}{0.670970in}}%
\pgfpathlineto{\pgfqpoint{2.744218in}{0.679657in}}%
\pgfpathlineto{\pgfqpoint{2.606021in}{0.692063in}}%
\pgfpathlineto{\pgfqpoint{2.450408in}{0.708438in}}%
\pgfpathlineto{\pgfqpoint{2.278997in}{0.728990in}}%
\pgfpathlineto{\pgfqpoint{2.142813in}{0.747208in}}%
\pgfpathlineto{\pgfqpoint{2.004017in}{0.767776in}}%
\pgfpathlineto{\pgfqpoint{1.866916in}{0.790534in}}%
\pgfpathlineto{\pgfqpoint{1.778867in}{0.806796in}}%
\pgfpathlineto{\pgfqpoint{1.695375in}{0.823822in}}%
\pgfpathlineto{\pgfqpoint{1.617616in}{0.841489in}}%
\pgfpathlineto{\pgfqpoint{1.546202in}{0.859631in}}%
\pgfpathlineto{\pgfqpoint{1.481552in}{0.878091in}}%
\pgfpathlineto{\pgfqpoint{1.423897in}{0.896726in}}%
\pgfpathlineto{\pgfqpoint{1.373281in}{0.915403in}}%
\pgfpathlineto{\pgfqpoint{1.329559in}{0.934005in}}%
\pgfpathlineto{\pgfqpoint{1.292401in}{0.952423in}}%
\pgfpathlineto{\pgfqpoint{1.261285in}{0.970562in}}%
\pgfpathlineto{\pgfqpoint{1.235504in}{0.988339in}}%
\pgfpathlineto{\pgfqpoint{1.214366in}{1.005677in}}%
\pgfpathlineto{\pgfqpoint{1.197806in}{1.022500in}}%
\pgfpathlineto{\pgfqpoint{1.185055in}{1.038769in}}%
\pgfpathlineto{\pgfqpoint{1.175414in}{1.054456in}}%
\pgfpathlineto{\pgfqpoint{1.168374in}{1.069534in}}%
\pgfpathlineto{\pgfqpoint{1.163623in}{1.083984in}}%
\pgfpathlineto{\pgfqpoint{1.161037in}{1.097789in}}%
\pgfpathlineto{\pgfqpoint{1.160687in}{1.110936in}}%
\pgfpathlineto{\pgfqpoint{1.162837in}{1.123416in}}%
\pgfpathlineto{\pgfqpoint{1.167942in}{1.135225in}}%
\pgfpathlineto{\pgfqpoint{1.175875in}{1.146350in}}%
\pgfpathlineto{\pgfqpoint{1.185820in}{1.156775in}}%
\pgfpathlineto{\pgfqpoint{1.197687in}{1.166497in}}%
\pgfpathlineto{\pgfqpoint{1.211424in}{1.175515in}}%
\pgfpathlineto{\pgfqpoint{1.227005in}{1.183828in}}%
\pgfpathlineto{\pgfqpoint{1.253861in}{1.194973in}}%
\pgfpathlineto{\pgfqpoint{1.285045in}{1.204533in}}%
\pgfpathlineto{\pgfqpoint{1.320864in}{1.212511in}}%
\pgfpathlineto{\pgfqpoint{1.361615in}{1.218907in}}%
\pgfpathlineto{\pgfqpoint{1.407424in}{1.223695in}}%
\pgfpathlineto{\pgfqpoint{1.458769in}{1.226847in}}%
\pgfpathlineto{\pgfqpoint{1.516182in}{1.228321in}}%
\pgfpathlineto{\pgfqpoint{1.580237in}{1.228070in}}%
\pgfpathlineto{\pgfqpoint{1.677042in}{1.224943in}}%
\pgfpathlineto{\pgfqpoint{1.788316in}{1.218467in}}%
\pgfpathlineto{\pgfqpoint{1.915730in}{1.208437in}}%
\pgfpathlineto{\pgfqpoint{2.060521in}{1.194595in}}%
\pgfpathlineto{\pgfqpoint{2.222528in}{1.176686in}}%
\pgfpathlineto{\pgfqpoint{2.398955in}{1.154544in}}%
\pgfpathlineto{\pgfqpoint{2.537252in}{1.135140in}}%
\pgfpathlineto{\pgfqpoint{2.676019in}{1.113445in}}%
\pgfpathlineto{\pgfqpoint{2.810599in}{1.089675in}}%
\pgfpathlineto{\pgfqpoint{2.895664in}{1.072831in}}%
\pgfpathlineto{\pgfqpoint{2.975244in}{1.055324in}}%
\pgfpathlineto{\pgfqpoint{3.047824in}{1.037284in}}%
\pgfpathlineto{\pgfqpoint{3.113414in}{1.018873in}}%
\pgfpathlineto{\pgfqpoint{3.172222in}{1.000245in}}%
\pgfpathlineto{\pgfqpoint{3.224474in}{0.981537in}}%
\pgfpathlineto{\pgfqpoint{3.270414in}{0.962878in}}%
\pgfpathlineto{\pgfqpoint{3.310306in}{0.944377in}}%
\pgfpathlineto{\pgfqpoint{3.344431in}{0.926135in}}%
\pgfpathlineto{\pgfqpoint{3.373088in}{0.908235in}}%
\pgfpathlineto{\pgfqpoint{3.396595in}{0.890749in}}%
\pgfpathlineto{\pgfqpoint{3.415288in}{0.873736in}}%
\pgfpathlineto{\pgfqpoint{3.429521in}{0.857238in}}%
\pgfpathlineto{\pgfqpoint{3.439675in}{0.841304in}}%
\pgfpathlineto{\pgfqpoint{3.446327in}{0.825996in}}%
\pgfpathlineto{\pgfqpoint{3.450116in}{0.811335in}}%
\pgfpathlineto{\pgfqpoint{3.451541in}{0.797334in}}%
\pgfpathlineto{\pgfqpoint{3.450966in}{0.784005in}}%
\pgfpathlineto{\pgfqpoint{3.448612in}{0.771355in}}%
\pgfpathlineto{\pgfqpoint{3.444566in}{0.759391in}}%
\pgfpathlineto{\pgfqpoint{3.438772in}{0.748114in}}%
\pgfpathlineto{\pgfqpoint{3.431037in}{0.737522in}}%
\pgfpathlineto{\pgfqpoint{3.421030in}{0.727610in}}%
\pgfpathlineto{\pgfqpoint{3.408281in}{0.718371in}}%
\pgfpathlineto{\pgfqpoint{3.393054in}{0.709819in}}%
\pgfpathlineto{\pgfqpoint{3.366631in}{0.698313in}}%
\pgfpathlineto{\pgfqpoint{3.335887in}{0.688398in}}%
\pgfpathlineto{\pgfqpoint{3.300693in}{0.680082in}}%
\pgfpathlineto{\pgfqpoint{3.260816in}{0.673373in}}%
\pgfpathlineto{\pgfqpoint{3.215912in}{0.668284in}}%
\pgfpathlineto{\pgfqpoint{3.165530in}{0.664828in}}%
\pgfpathlineto{\pgfqpoint{3.109368in}{0.663020in}}%
\pgfpathlineto{\pgfqpoint{3.046991in}{0.662912in}}%
\pgfpathlineto{\pgfqpoint{2.977303in}{0.664580in}}%
\pgfpathlineto{\pgfqpoint{2.871513in}{0.669707in}}%
\pgfpathlineto{\pgfqpoint{2.749714in}{0.678338in}}%
\pgfpathlineto{\pgfqpoint{2.611180in}{0.690688in}}%
\pgfpathlineto{\pgfqpoint{2.455917in}{0.706985in}}%
\pgfpathlineto{\pgfqpoint{2.284616in}{0.727470in}}%
\pgfpathlineto{\pgfqpoint{2.147114in}{0.745736in}}%
\pgfpathlineto{\pgfqpoint{2.007212in}{0.766374in}}%
\pgfpathlineto{\pgfqpoint{1.870073in}{0.789154in}}%
\pgfpathlineto{\pgfqpoint{1.740069in}{0.813793in}}%
\pgfpathlineto{\pgfqpoint{1.659163in}{0.831087in}}%
\pgfpathlineto{\pgfqpoint{1.583900in}{0.848943in}}%
\pgfpathlineto{\pgfqpoint{1.514989in}{0.867236in}}%
\pgfpathlineto{\pgfqpoint{1.452980in}{0.885832in}}%
\pgfpathlineto{\pgfqpoint{1.398270in}{0.904582in}}%
\pgfpathlineto{\pgfqpoint{1.351067in}{0.923330in}}%
\pgfpathlineto{\pgfqpoint{1.310840in}{0.941936in}}%
\pgfpathlineto{\pgfqpoint{1.276677in}{0.960301in}}%
\pgfpathlineto{\pgfqpoint{1.247832in}{0.978343in}}%
\pgfpathlineto{\pgfqpoint{1.223721in}{0.995985in}}%
\pgfpathlineto{\pgfqpoint{1.203920in}{1.013162in}}%
\pgfpathlineto{\pgfqpoint{1.188167in}{1.029817in}}%
\pgfpathlineto{\pgfqpoint{1.176359in}{1.045902in}}%
\pgfpathlineto{\pgfqpoint{1.168261in}{1.061378in}}%
\pgfpathlineto{\pgfqpoint{1.163163in}{1.076217in}}%
\pgfpathlineto{\pgfqpoint{1.160736in}{1.090401in}}%
\pgfpathlineto{\pgfqpoint{1.160735in}{1.103915in}}%
\pgfpathlineto{\pgfqpoint{1.162984in}{1.116749in}}%
\pgfpathlineto{\pgfqpoint{1.167379in}{1.128895in}}%
\pgfpathlineto{\pgfqpoint{1.173885in}{1.140349in}}%
\pgfpathlineto{\pgfqpoint{1.182534in}{1.151109in}}%
\pgfpathlineto{\pgfqpoint{1.193256in}{1.161175in}}%
\pgfpathlineto{\pgfqpoint{1.205936in}{1.170544in}}%
\pgfpathlineto{\pgfqpoint{1.220511in}{1.179213in}}%
\pgfpathlineto{\pgfqpoint{1.245866in}{1.190900in}}%
\pgfpathlineto{\pgfqpoint{1.275435in}{1.201002in}}%
\pgfpathlineto{\pgfqpoint{1.309376in}{1.209516in}}%
\pgfpathlineto{\pgfqpoint{1.348000in}{1.216439in}}%
\pgfpathlineto{\pgfqpoint{1.391763in}{1.221773in}}%
\pgfpathlineto{\pgfqpoint{1.440948in}{1.225504in}}%
\pgfpathlineto{\pgfqpoint{1.495851in}{1.227584in}}%
\pgfpathlineto{\pgfqpoint{1.557215in}{1.227964in}}%
\pgfpathlineto{\pgfqpoint{1.625737in}{1.226586in}}%
\pgfpathlineto{\pgfqpoint{1.729351in}{1.221889in}}%
\pgfpathlineto{\pgfqpoint{1.848245in}{1.213747in}}%
\pgfpathlineto{\pgfqpoint{1.983672in}{1.201930in}}%
\pgfpathlineto{\pgfqpoint{2.136824in}{1.186195in}}%
\pgfpathlineto{\pgfqpoint{2.306910in}{1.166321in}}%
\pgfpathlineto{\pgfqpoint{2.442036in}{1.148586in}}%
\pgfpathlineto{\pgfqpoint{2.580068in}{1.128463in}}%
\pgfpathlineto{\pgfqpoint{2.717281in}{1.106100in}}%
\pgfpathlineto{\pgfqpoint{2.849478in}{1.081750in}}%
\pgfpathlineto{\pgfqpoint{2.932538in}{1.064587in}}%
\pgfpathlineto{\pgfqpoint{3.009808in}{1.046833in}}%
\pgfpathlineto{\pgfqpoint{3.079893in}{1.028633in}}%
\pgfpathlineto{\pgfqpoint{3.142626in}{1.010132in}}%
\pgfpathlineto{\pgfqpoint{3.198347in}{0.991469in}}%
\pgfpathlineto{\pgfqpoint{3.247394in}{0.972777in}}%
\pgfpathlineto{\pgfqpoint{3.290105in}{0.954174in}}%
\pgfpathlineto{\pgfqpoint{3.326816in}{0.935767in}}%
\pgfpathlineto{\pgfqpoint{3.357863in}{0.917652in}}%
\pgfpathlineto{\pgfqpoint{3.383578in}{0.899914in}}%
\pgfpathlineto{\pgfqpoint{3.404296in}{0.882626in}}%
\pgfpathlineto{\pgfqpoint{3.420363in}{0.865851in}}%
\pgfpathlineto{\pgfqpoint{3.432381in}{0.849647in}}%
\pgfpathlineto{\pgfqpoint{3.441028in}{0.834045in}}%
\pgfpathlineto{\pgfqpoint{3.446837in}{0.819068in}}%
\pgfpathlineto{\pgfqpoint{3.450201in}{0.804736in}}%
\pgfpathlineto{\pgfqpoint{3.451372in}{0.791064in}}%
\pgfpathlineto{\pgfqpoint{3.450461in}{0.778063in}}%
\pgfpathlineto{\pgfqpoint{3.447436in}{0.765740in}}%
\pgfpathlineto{\pgfqpoint{3.442128in}{0.754098in}}%
\pgfpathlineto{\pgfqpoint{3.434224in}{0.743138in}}%
\pgfpathlineto{\pgfqpoint{3.423932in}{0.732867in}}%
\pgfpathlineto{\pgfqpoint{3.411687in}{0.723297in}}%
\pgfpathlineto{\pgfqpoint{3.397543in}{0.714429in}}%
\pgfpathlineto{\pgfqpoint{3.381533in}{0.706265in}}%
\pgfpathlineto{\pgfqpoint{3.354013in}{0.695341in}}%
\pgfpathlineto{\pgfqpoint{3.322182in}{0.686006in}}%
\pgfpathlineto{\pgfqpoint{3.285793in}{0.678260in}}%
\pgfpathlineto{\pgfqpoint{3.244459in}{0.672104in}}%
\pgfpathlineto{\pgfqpoint{3.197929in}{0.667547in}}%
\pgfpathlineto{\pgfqpoint{3.145900in}{0.664627in}}%
\pgfpathlineto{\pgfqpoint{3.087719in}{0.663386in}}%
\pgfpathlineto{\pgfqpoint{3.022741in}{0.663875in}}%
\pgfpathlineto{\pgfqpoint{2.924435in}{0.667331in}}%
\pgfpathlineto{\pgfqpoint{2.811443in}{0.674159in}}%
\pgfpathlineto{\pgfqpoint{2.682321in}{0.684574in}}%
\pgfpathlineto{\pgfqpoint{2.535750in}{0.698817in}}%
\pgfpathlineto{\pgfqpoint{2.372020in}{0.717140in}}%
\pgfpathlineto{\pgfqpoint{2.194470in}{0.739745in}}%
\pgfpathlineto{\pgfqpoint{2.056075in}{0.759484in}}%
\pgfpathlineto{\pgfqpoint{1.917523in}{0.781476in}}%
\pgfpathlineto{\pgfqpoint{1.783820in}{0.805507in}}%
\pgfpathlineto{\pgfqpoint{1.699963in}{0.822522in}}%
\pgfpathlineto{\pgfqpoint{1.621618in}{0.840165in}}%
\pgfpathlineto{\pgfqpoint{1.549571in}{0.858282in}}%
\pgfpathlineto{\pgfqpoint{1.484362in}{0.876728in}}%
\pgfpathlineto{\pgfqpoint{1.426284in}{0.895366in}}%
\pgfpathlineto{\pgfqpoint{1.375385in}{0.914066in}}%
\pgfpathlineto{\pgfqpoint{1.331464in}{0.932708in}}%
\pgfpathlineto{\pgfqpoint{1.294076in}{0.951180in}}%
\pgfpathlineto{\pgfqpoint{1.262560in}{0.969377in}}%
\pgfpathlineto{\pgfqpoint{1.236720in}{0.987186in}}%
\pgfpathlineto{\pgfqpoint{1.215765in}{1.004543in}}%
\pgfpathlineto{\pgfqpoint{1.198766in}{1.021405in}}%
\pgfpathlineto{\pgfqpoint{1.185034in}{1.037734in}}%
\pgfpathlineto{\pgfqpoint{1.174115in}{1.053496in}}%
\pgfpathlineto{\pgfqpoint{1.165795in}{1.068660in}}%
\pgfpathlineto{\pgfqpoint{1.160097in}{1.083201in}}%
\pgfpathlineto{\pgfqpoint{1.157282in}{1.097100in}}%
\pgfpathlineto{\pgfqpoint{1.157849in}{1.110338in}}%
\pgfpathlineto{\pgfqpoint{1.161567in}{1.122897in}}%
\pgfpathlineto{\pgfqpoint{1.167447in}{1.134758in}}%
\pgfpathlineto{\pgfqpoint{1.175346in}{1.145919in}}%
\pgfpathlineto{\pgfqpoint{1.185164in}{1.156376in}}%
\pgfpathlineto{\pgfqpoint{1.196841in}{1.166128in}}%
\pgfpathlineto{\pgfqpoint{1.210353in}{1.175174in}}%
\pgfpathlineto{\pgfqpoint{1.225719in}{1.183516in}}%
\pgfpathlineto{\pgfqpoint{1.252381in}{1.194713in}}%
\pgfpathlineto{\pgfqpoint{1.283653in}{1.204340in}}%
\pgfpathlineto{\pgfqpoint{1.319519in}{1.212388in}}%
\pgfpathlineto{\pgfqpoint{1.360161in}{1.218840in}}%
\pgfpathlineto{\pgfqpoint{1.405902in}{1.223675in}}%
\pgfpathlineto{\pgfqpoint{1.457143in}{1.226864in}}%
\pgfpathlineto{\pgfqpoint{1.514373in}{1.228371in}}%
\pgfpathlineto{\pgfqpoint{1.578160in}{1.228152in}}%
\pgfpathlineto{\pgfqpoint{1.674548in}{1.225089in}}%
\pgfpathlineto{\pgfqpoint{1.785347in}{1.218712in}}%
\pgfpathlineto{\pgfqpoint{1.912465in}{1.208769in}}%
\pgfpathlineto{\pgfqpoint{2.057092in}{1.195002in}}%
\pgfpathlineto{\pgfqpoint{2.218667in}{1.177187in}}%
\pgfpathlineto{\pgfqpoint{2.394924in}{1.155136in}}%
\pgfpathlineto{\pgfqpoint{2.533431in}{1.135781in}}%
\pgfpathlineto{\pgfqpoint{2.672182in}{1.114141in}}%
\pgfpathlineto{\pgfqpoint{2.806572in}{1.090436in}}%
\pgfpathlineto{\pgfqpoint{2.891637in}{1.073635in}}%
\pgfpathlineto{\pgfqpoint{2.971716in}{1.056166in}}%
\pgfpathlineto{\pgfqpoint{3.045723in}{1.038161in}}%
\pgfpathlineto{\pgfqpoint{3.112655in}{1.019768in}}%
\pgfpathlineto{\pgfqpoint{3.172090in}{1.001144in}}%
\pgfpathlineto{\pgfqpoint{3.224407in}{0.982426in}}%
\pgfpathlineto{\pgfqpoint{3.270038in}{0.963741in}}%
\pgfpathlineto{\pgfqpoint{3.309402in}{0.945202in}}%
\pgfpathlineto{\pgfqpoint{3.342902in}{0.926913in}}%
\pgfpathlineto{\pgfqpoint{3.370924in}{0.908965in}}%
\pgfpathlineto{\pgfqpoint{3.393841in}{0.891437in}}%
\pgfpathlineto{\pgfqpoint{3.412033in}{0.874399in}}%
\pgfpathlineto{\pgfqpoint{3.426118in}{0.857907in}}%
\pgfpathlineto{\pgfqpoint{3.436689in}{0.841997in}}%
\pgfpathlineto{\pgfqpoint{3.444199in}{0.826697in}}%
\pgfpathlineto{\pgfqpoint{3.448985in}{0.812033in}}%
\pgfpathlineto{\pgfqpoint{3.451265in}{0.798022in}}%
\pgfpathlineto{\pgfqpoint{3.451141in}{0.784679in}}%
\pgfpathlineto{\pgfqpoint{3.448597in}{0.772013in}}%
\pgfpathlineto{\pgfqpoint{3.443566in}{0.760030in}}%
\pgfpathlineto{\pgfqpoint{3.436343in}{0.748739in}}%
\pgfpathlineto{\pgfqpoint{3.427103in}{0.738143in}}%
\pgfpathlineto{\pgfqpoint{3.415944in}{0.728246in}}%
\pgfpathlineto{\pgfqpoint{3.402931in}{0.719051in}}%
\pgfpathlineto{\pgfqpoint{3.388091in}{0.710558in}}%
\pgfpathlineto{\pgfqpoint{3.362376in}{0.699135in}}%
\pgfpathlineto{\pgfqpoint{3.332335in}{0.689289in}}%
\pgfpathlineto{\pgfqpoint{3.297601in}{0.681012in}}%
\pgfpathlineto{\pgfqpoint{3.257943in}{0.674307in}}%
\pgfpathlineto{\pgfqpoint{3.213272in}{0.669204in}}%
\pgfpathlineto{\pgfqpoint{3.163151in}{0.665728in}}%
\pgfpathlineto{\pgfqpoint{3.107085in}{0.663920in}}%
\pgfpathlineto{\pgfqpoint{3.044525in}{0.663825in}}%
\pgfpathlineto{\pgfqpoint{2.974865in}{0.665501in}}%
\pgfpathlineto{\pgfqpoint{2.869793in}{0.670607in}}%
\pgfpathlineto{\pgfqpoint{2.749223in}{0.679167in}}%
\pgfpathlineto{\pgfqpoint{2.611665in}{0.691431in}}%
\pgfpathlineto{\pgfqpoint{2.456625in}{0.707659in}}%
\pgfpathlineto{\pgfqpoint{2.285705in}{0.728059in}}%
\pgfpathlineto{\pgfqpoint{2.149711in}{0.746162in}}%
\pgfpathlineto{\pgfqpoint{2.010881in}{0.766622in}}%
\pgfpathlineto{\pgfqpoint{1.873530in}{0.789284in}}%
\pgfpathlineto{\pgfqpoint{1.785202in}{0.805491in}}%
\pgfpathlineto{\pgfqpoint{1.701338in}{0.822474in}}%
\pgfpathlineto{\pgfqpoint{1.623083in}{0.840106in}}%
\pgfpathlineto{\pgfqpoint{1.551110in}{0.858219in}}%
\pgfpathlineto{\pgfqpoint{1.485887in}{0.876661in}}%
\pgfpathlineto{\pgfqpoint{1.427682in}{0.895288in}}%
\pgfpathlineto{\pgfqpoint{1.376560in}{0.913970in}}%
\pgfpathlineto{\pgfqpoint{1.332383in}{0.932586in}}%
\pgfpathlineto{\pgfqpoint{1.294812in}{0.951027in}}%
\pgfpathlineto{\pgfqpoint{1.263305in}{0.969199in}}%
\pgfpathlineto{\pgfqpoint{1.237119in}{0.987014in}}%
\pgfpathlineto{\pgfqpoint{1.215723in}{1.004389in}}%
\pgfpathlineto{\pgfqpoint{1.198918in}{1.021251in}}%
\pgfpathlineto{\pgfqpoint{1.185847in}{1.037565in}}%
\pgfpathlineto{\pgfqpoint{1.175838in}{1.053301in}}%
\pgfpathlineto{\pgfqpoint{1.168413in}{1.068433in}}%
\pgfpathlineto{\pgfqpoint{1.163300in}{1.082939in}}%
\pgfpathlineto{\pgfqpoint{1.160424in}{1.096801in}}%
\pgfpathlineto{\pgfqpoint{1.159913in}{1.110007in}}%
\pgfpathlineto{\pgfqpoint{1.162094in}{1.122547in}}%
\pgfpathlineto{\pgfqpoint{1.167436in}{1.134415in}}%
\pgfpathlineto{\pgfqpoint{1.175253in}{1.145591in}}%
\pgfpathlineto{\pgfqpoint{1.185066in}{1.156066in}}%
\pgfpathlineto{\pgfqpoint{1.196789in}{1.165837in}}%
\pgfpathlineto{\pgfqpoint{1.210371in}{1.174904in}}%
\pgfpathlineto{\pgfqpoint{1.225793in}{1.183265in}}%
\pgfpathlineto{\pgfqpoint{1.252408in}{1.194485in}}%
\pgfpathlineto{\pgfqpoint{1.283372in}{1.204120in}}%
\pgfpathlineto{\pgfqpoint{1.319024in}{1.212179in}}%
\pgfpathlineto{\pgfqpoint{1.359538in}{1.218653in}}%
\pgfpathlineto{\pgfqpoint{1.405116in}{1.223519in}}%
\pgfpathlineto{\pgfqpoint{1.456215in}{1.226749in}}%
\pgfpathlineto{\pgfqpoint{1.513345in}{1.228302in}}%
\pgfpathlineto{\pgfqpoint{1.577067in}{1.228130in}}%
\pgfpathlineto{\pgfqpoint{1.647995in}{1.226174in}}%
\pgfpathlineto{\pgfqpoint{1.754936in}{1.220672in}}%
\pgfpathlineto{\pgfqpoint{1.877582in}{1.211683in}}%
\pgfpathlineto{\pgfqpoint{2.017364in}{1.198957in}}%
\pgfpathlineto{\pgfqpoint{2.174601in}{1.182232in}}%
\pgfpathlineto{\pgfqpoint{2.347379in}{1.161311in}}%
\pgfpathlineto{\pgfqpoint{2.484267in}{1.142812in}}%
\pgfpathlineto{\pgfqpoint{2.623339in}{1.121971in}}%
\pgfpathlineto{\pgfqpoint{2.760163in}{1.098959in}}%
\pgfpathlineto{\pgfqpoint{2.847690in}{1.082546in}}%
\pgfpathlineto{\pgfqpoint{2.930403in}{1.065382in}}%
\pgfpathlineto{\pgfqpoint{3.007260in}{1.047605in}}%
\pgfpathlineto{\pgfqpoint{3.077673in}{1.029380in}}%
\pgfpathlineto{\pgfqpoint{3.141245in}{1.010861in}}%
\pgfpathlineto{\pgfqpoint{3.197774in}{0.992189in}}%
\pgfpathlineto{\pgfqpoint{3.247249in}{0.973495in}}%
\pgfpathlineto{\pgfqpoint{3.289854in}{0.954894in}}%
\pgfpathlineto{\pgfqpoint{3.325964in}{0.936493in}}%
\pgfpathlineto{\pgfqpoint{3.356146in}{0.918387in}}%
\pgfpathlineto{\pgfqpoint{3.381162in}{0.900656in}}%
\pgfpathlineto{\pgfqpoint{3.401467in}{0.883383in}}%
\pgfpathlineto{\pgfqpoint{3.417312in}{0.866635in}}%
\pgfpathlineto{\pgfqpoint{3.429537in}{0.850445in}}%
\pgfpathlineto{\pgfqpoint{3.438798in}{0.834841in}}%
\pgfpathlineto{\pgfqpoint{3.445549in}{0.819848in}}%
\pgfpathlineto{\pgfqpoint{3.450047in}{0.805487in}}%
\pgfpathlineto{\pgfqpoint{3.452352in}{0.791774in}}%
\pgfpathlineto{\pgfqpoint{3.452325in}{0.778720in}}%
\pgfpathlineto{\pgfqpoint{3.449629in}{0.766334in}}%
\pgfpathlineto{\pgfqpoint{3.443791in}{0.754621in}}%
\pgfpathlineto{\pgfqpoint{3.435501in}{0.743600in}}%
\pgfpathlineto{\pgfqpoint{3.425229in}{0.733281in}}%
\pgfpathlineto{\pgfqpoint{3.413054in}{0.723666in}}%
\pgfpathlineto{\pgfqpoint{3.399024in}{0.714757in}}%
\pgfpathlineto{\pgfqpoint{3.383152in}{0.706553in}}%
\pgfpathlineto{\pgfqpoint{3.355852in}{0.695571in}}%
\pgfpathlineto{\pgfqpoint{3.324185in}{0.686174in}}%
\pgfpathlineto{\pgfqpoint{3.287809in}{0.678356in}}%
\pgfpathlineto{\pgfqpoint{3.246533in}{0.672125in}}%
\pgfpathlineto{\pgfqpoint{3.200143in}{0.667505in}}%
\pgfpathlineto{\pgfqpoint{3.148160in}{0.664527in}}%
\pgfpathlineto{\pgfqpoint{3.090057in}{0.663231in}}%
\pgfpathlineto{\pgfqpoint{3.025260in}{0.663669in}}%
\pgfpathlineto{\pgfqpoint{2.927373in}{0.667055in}}%
\pgfpathlineto{\pgfqpoint{2.814882in}{0.673804in}}%
\pgfpathlineto{\pgfqpoint{2.686108in}{0.684125in}}%
\pgfpathlineto{\pgfqpoint{2.539883in}{0.698282in}}%
\pgfpathlineto{\pgfqpoint{2.376592in}{0.716521in}}%
\pgfpathlineto{\pgfqpoint{2.199163in}{0.739005in}}%
\pgfpathlineto{\pgfqpoint{2.060634in}{0.758659in}}%
\pgfpathlineto{\pgfqpoint{1.922068in}{0.780586in}}%
\pgfpathlineto{\pgfqpoint{1.788116in}{0.804562in}}%
\pgfpathlineto{\pgfqpoint{1.703847in}{0.821524in}}%
\pgfpathlineto{\pgfqpoint{1.625353in}{0.839144in}}%
\pgfpathlineto{\pgfqpoint{1.553289in}{0.857266in}}%
\pgfpathlineto{\pgfqpoint{1.487988in}{0.875727in}}%
\pgfpathlineto{\pgfqpoint{1.429629in}{0.894380in}}%
\pgfpathlineto{\pgfqpoint{1.378236in}{0.913089in}}%
\pgfpathlineto{\pgfqpoint{1.333680in}{0.931731in}}%
\pgfpathlineto{\pgfqpoint{1.295674in}{0.950197in}}%
\pgfpathlineto{\pgfqpoint{1.263778in}{0.968390in}}%
\pgfpathlineto{\pgfqpoint{1.237396in}{0.986227in}}%
\pgfpathlineto{\pgfqpoint{1.215778in}{1.003637in}}%
\pgfpathlineto{\pgfqpoint{1.198652in}{1.020543in}}%
\pgfpathlineto{\pgfqpoint{1.185620in}{1.036892in}}%
\pgfpathlineto{\pgfqpoint{1.175898in}{1.052656in}}%
\pgfpathlineto{\pgfqpoint{1.168886in}{1.067811in}}%
\pgfpathlineto{\pgfqpoint{1.164175in}{1.082337in}}%
\pgfpathlineto{\pgfqpoint{1.161540in}{1.096217in}}%
\pgfpathlineto{\pgfqpoint{1.160947in}{1.109440in}}%
\pgfpathlineto{\pgfqpoint{1.162547in}{1.121996in}}%
\pgfpathlineto{\pgfqpoint{1.166680in}{1.133881in}}%
\pgfpathlineto{\pgfqpoint{1.173859in}{1.145093in}}%
\pgfpathlineto{\pgfqpoint{1.183589in}{1.155616in}}%
\pgfpathlineto{\pgfqpoint{1.195272in}{1.165435in}}%
\pgfpathlineto{\pgfqpoint{1.208846in}{1.174550in}}%
\pgfpathlineto{\pgfqpoint{1.224280in}{1.182959in}}%
\pgfpathlineto{\pgfqpoint{1.250913in}{1.194248in}}%
\pgfpathlineto{\pgfqpoint{1.281828in}{1.203946in}}%
\pgfpathlineto{\pgfqpoint{1.317271in}{1.212055in}}%
\pgfpathlineto{\pgfqpoint{1.357621in}{1.218578in}}%
\pgfpathlineto{\pgfqpoint{1.403037in}{1.223499in}}%
\pgfpathlineto{\pgfqpoint{1.453899in}{1.226786in}}%
\pgfpathlineto{\pgfqpoint{1.510791in}{1.228402in}}%
\pgfpathlineto{\pgfqpoint{1.574314in}{1.228297in}}%
\pgfpathlineto{\pgfqpoint{1.645085in}{1.226410in}}%
\pgfpathlineto{\pgfqpoint{1.751820in}{1.220997in}}%
\pgfpathlineto{\pgfqpoint{1.874122in}{1.212089in}}%
\pgfpathlineto{\pgfqpoint{2.013487in}{1.199455in}}%
\pgfpathlineto{\pgfqpoint{2.170361in}{1.182829in}}%
\pgfpathlineto{\pgfqpoint{2.342937in}{1.161991in}}%
\pgfpathlineto{\pgfqpoint{2.479665in}{1.143566in}}%
\pgfpathlineto{\pgfqpoint{2.618887in}{1.122792in}}%
\pgfpathlineto{\pgfqpoint{2.756021in}{1.099817in}}%
\pgfpathlineto{\pgfqpoint{2.843670in}{1.083434in}}%
\pgfpathlineto{\pgfqpoint{2.926750in}{1.066324in}}%
\pgfpathlineto{\pgfqpoint{3.004199in}{1.048611in}}%
\pgfpathlineto{\pgfqpoint{3.075192in}{1.030426in}}%
\pgfpathlineto{\pgfqpoint{3.139140in}{1.011910in}}%
\pgfpathlineto{\pgfqpoint{3.195631in}{0.993220in}}%
\pgfpathlineto{\pgfqpoint{3.244743in}{0.974505in}}%
\pgfpathlineto{\pgfqpoint{3.287253in}{0.955878in}}%
\pgfpathlineto{\pgfqpoint{3.323843in}{0.937442in}}%
\pgfpathlineto{\pgfqpoint{3.355066in}{0.919290in}}%
\pgfpathlineto{\pgfqpoint{3.381342in}{0.901504in}}%
\pgfpathlineto{\pgfqpoint{3.402959in}{0.884159in}}%
\pgfpathlineto{\pgfqpoint{3.420072in}{0.867317in}}%
\pgfpathlineto{\pgfqpoint{3.432764in}{0.851034in}}%
\pgfpathlineto{\pgfqpoint{3.441805in}{0.835355in}}%
\pgfpathlineto{\pgfqpoint{3.447795in}{0.820304in}}%
\pgfpathlineto{\pgfqpoint{3.451102in}{0.805901in}}%
\pgfpathlineto{\pgfqpoint{3.451998in}{0.792163in}}%
\pgfpathlineto{\pgfqpoint{3.450662in}{0.779100in}}%
\pgfpathlineto{\pgfqpoint{3.447176in}{0.766722in}}%
\pgfpathlineto{\pgfqpoint{3.441531in}{0.755034in}}%
\pgfpathlineto{\pgfqpoint{3.433622in}{0.744035in}}%
\pgfpathlineto{\pgfqpoint{3.423537in}{0.733729in}}%
\pgfpathlineto{\pgfqpoint{3.411484in}{0.724121in}}%
\pgfpathlineto{\pgfqpoint{3.397528in}{0.715213in}}%
\pgfpathlineto{\pgfqpoint{3.381705in}{0.707008in}}%
\pgfpathlineto{\pgfqpoint{3.354481in}{0.696019in}}%
\pgfpathlineto{\pgfqpoint{3.322974in}{0.686618in}}%
\pgfpathlineto{\pgfqpoint{3.286941in}{0.678805in}}%
\pgfpathlineto{\pgfqpoint{3.245995in}{0.672579in}}%
\pgfpathlineto{\pgfqpoint{3.199773in}{0.667946in}}%
\pgfpathlineto{\pgfqpoint{3.148105in}{0.664944in}}%
\pgfpathlineto{\pgfqpoint{3.090336in}{0.663615in}}%
\pgfpathlineto{\pgfqpoint{3.025797in}{0.664012in}}%
\pgfpathlineto{\pgfqpoint{2.928103in}{0.667337in}}%
\pgfpathlineto{\pgfqpoint{2.815761in}{0.674024in}}%
\pgfpathlineto{\pgfqpoint{2.687370in}{0.684288in}}%
\pgfpathlineto{\pgfqpoint{2.541602in}{0.698368in}}%
\pgfpathlineto{\pgfqpoint{2.378420in}{0.716501in}}%
\pgfpathlineto{\pgfqpoint{2.201429in}{0.738918in}}%
\pgfpathlineto{\pgfqpoint{2.063433in}{0.758532in}}%
\pgfpathlineto{\pgfqpoint{1.925031in}{0.780413in}}%
\pgfpathlineto{\pgfqpoint{1.790848in}{0.804334in}}%
\pgfpathlineto{\pgfqpoint{1.706413in}{0.821259in}}%
\pgfpathlineto{\pgfqpoint{1.627759in}{0.838845in}}%
\pgfpathlineto{\pgfqpoint{1.555511in}{0.856936in}}%
\pgfpathlineto{\pgfqpoint{1.490019in}{0.875372in}}%
\pgfpathlineto{\pgfqpoint{1.431472in}{0.894005in}}%
\pgfpathlineto{\pgfqpoint{1.379901in}{0.912701in}}%
\pgfpathlineto{\pgfqpoint{1.335179in}{0.931335in}}%
\pgfpathlineto{\pgfqpoint{1.297015in}{0.949798in}}%
\pgfpathlineto{\pgfqpoint{1.264962in}{0.967993in}}%
\pgfpathlineto{\pgfqpoint{1.238414in}{0.985834in}}%
\pgfpathlineto{\pgfqpoint{1.216630in}{1.003248in}}%
\pgfpathlineto{\pgfqpoint{1.199466in}{1.020156in}}%
\pgfpathlineto{\pgfqpoint{1.186335in}{1.036510in}}%
\pgfpathlineto{\pgfqpoint{1.176456in}{1.052282in}}%
\pgfpathlineto{\pgfqpoint{1.169245in}{1.067448in}}%
\pgfpathlineto{\pgfqpoint{1.164311in}{1.081988in}}%
\pgfpathlineto{\pgfqpoint{1.161458in}{1.095885in}}%
\pgfpathlineto{\pgfqpoint{1.160686in}{1.109126in}}%
\pgfpathlineto{\pgfqpoint{1.162187in}{1.121702in}}%
\pgfpathlineto{\pgfqpoint{1.166349in}{1.133608in}}%
\pgfpathlineto{\pgfqpoint{1.173665in}{1.144841in}}%
\pgfpathlineto{\pgfqpoint{1.183358in}{1.155380in}}%
\pgfpathlineto{\pgfqpoint{1.194997in}{1.165216in}}%
\pgfpathlineto{\pgfqpoint{1.208522in}{1.174347in}}%
\pgfpathlineto{\pgfqpoint{1.223901in}{1.182772in}}%
\pgfpathlineto{\pgfqpoint{1.250448in}{1.194084in}}%
\pgfpathlineto{\pgfqpoint{1.281280in}{1.203807in}}%
\pgfpathlineto{\pgfqpoint{1.316655in}{1.211941in}}%
\pgfpathlineto{\pgfqpoint{1.356946in}{1.218491in}}%
\pgfpathlineto{\pgfqpoint{1.402277in}{1.223438in}}%
\pgfpathlineto{\pgfqpoint{1.453068in}{1.226752in}}%
\pgfpathlineto{\pgfqpoint{1.509880in}{1.228394in}}%
\pgfpathlineto{\pgfqpoint{1.573301in}{1.228314in}}%
\pgfpathlineto{\pgfqpoint{1.643944in}{1.226454in}}%
\pgfpathlineto{\pgfqpoint{1.750478in}{1.221079in}}%
\pgfpathlineto{\pgfqpoint{1.872578in}{1.212213in}}%
\pgfpathlineto{\pgfqpoint{2.011744in}{1.199625in}}%
\pgfpathlineto{\pgfqpoint{2.168436in}{1.183043in}}%
\pgfpathlineto{\pgfqpoint{2.340830in}{1.162261in}}%
\pgfpathlineto{\pgfqpoint{2.477482in}{1.143876in}}%
\pgfpathlineto{\pgfqpoint{2.616795in}{1.123129in}}%
\pgfpathlineto{\pgfqpoint{2.753889in}{1.100200in}}%
\pgfpathlineto{\pgfqpoint{2.841581in}{1.083842in}}%
\pgfpathlineto{\pgfqpoint{2.924809in}{1.066747in}}%
\pgfpathlineto{\pgfqpoint{3.002468in}{1.049038in}}%
\pgfpathlineto{\pgfqpoint{3.073610in}{1.030854in}}%
\pgfpathlineto{\pgfqpoint{3.137383in}{1.012350in}}%
\pgfpathlineto{\pgfqpoint{3.193756in}{0.993676in}}%
\pgfpathlineto{\pgfqpoint{3.243257in}{0.974961in}}%
\pgfpathlineto{\pgfqpoint{3.286366in}{0.956323in}}%
\pgfpathlineto{\pgfqpoint{3.323515in}{0.937871in}}%
\pgfpathlineto{\pgfqpoint{3.355084in}{0.919701in}}%
\pgfpathlineto{\pgfqpoint{3.381402in}{0.901900in}}%
\pgfpathlineto{\pgfqpoint{3.402751in}{0.884543in}}%
\pgfpathlineto{\pgfqpoint{3.419369in}{0.867695in}}%
\pgfpathlineto{\pgfqpoint{3.431859in}{0.851413in}}%
\pgfpathlineto{\pgfqpoint{3.440924in}{0.835731in}}%
\pgfpathlineto{\pgfqpoint{3.447050in}{0.820673in}}%
\pgfpathlineto{\pgfqpoint{3.450606in}{0.806261in}}%
\pgfpathlineto{\pgfqpoint{3.451836in}{0.792509in}}%
\pgfpathlineto{\pgfqpoint{3.450864in}{0.779431in}}%
\pgfpathlineto{\pgfqpoint{3.447694in}{0.767034in}}%
\pgfpathlineto{\pgfqpoint{3.442207in}{0.755321in}}%
\pgfpathlineto{\pgfqpoint{3.434247in}{0.744295in}}%
\pgfpathlineto{\pgfqpoint{3.424177in}{0.733963in}}%
\pgfpathlineto{\pgfqpoint{3.412161in}{0.724332in}}%
\pgfpathlineto{\pgfqpoint{3.398258in}{0.715402in}}%
\pgfpathlineto{\pgfqpoint{3.382500in}{0.707176in}}%
\pgfpathlineto{\pgfqpoint{3.355380in}{0.696157in}}%
\pgfpathlineto{\pgfqpoint{3.323960in}{0.686724in}}%
\pgfpathlineto{\pgfqpoint{3.287970in}{0.678876in}}%
\pgfpathlineto{\pgfqpoint{3.246996in}{0.672610in}}%
\pgfpathlineto{\pgfqpoint{3.200883in}{0.667943in}}%
\pgfpathlineto{\pgfqpoint{3.149257in}{0.664910in}}%
\pgfpathlineto{\pgfqpoint{3.091515in}{0.663552in}}%
\pgfpathlineto{\pgfqpoint{3.027045in}{0.663921in}}%
\pgfpathlineto{\pgfqpoint{2.929544in}{0.667208in}}%
\pgfpathlineto{\pgfqpoint{2.817474in}{0.673854in}}%
\pgfpathlineto{\pgfqpoint{2.689304in}{0.684072in}}%
\pgfpathlineto{\pgfqpoint{2.543745in}{0.698102in}}%
\pgfpathlineto{\pgfqpoint{2.381014in}{0.716214in}}%
\pgfpathlineto{\pgfqpoint{2.204067in}{0.738576in}}%
\pgfpathlineto{\pgfqpoint{2.065800in}{0.758123in}}%
\pgfpathlineto{\pgfqpoint{1.927138in}{0.779957in}}%
\pgfpathlineto{\pgfqpoint{1.792997in}{0.803870in}}%
\pgfpathlineto{\pgfqpoint{1.708563in}{0.820781in}}%
\pgfpathlineto{\pgfqpoint{1.629503in}{0.838323in}}%
\pgfpathlineto{\pgfqpoint{1.556699in}{0.856367in}}%
\pgfpathlineto{\pgfqpoint{1.490761in}{0.874779in}}%
\pgfpathlineto{\pgfqpoint{1.432029in}{0.893419in}}%
\pgfpathlineto{\pgfqpoint{1.380608in}{0.912141in}}%
\pgfpathlineto{\pgfqpoint{1.336429in}{0.930792in}}%
\pgfpathlineto{\pgfqpoint{1.298636in}{0.949269in}}%
\pgfpathlineto{\pgfqpoint{1.266392in}{0.967482in}}%
\pgfpathlineto{\pgfqpoint{1.239049in}{0.985351in}}%
\pgfpathlineto{\pgfqpoint{1.216145in}{1.002805in}}%
\pgfpathlineto{\pgfqpoint{1.197405in}{1.019780in}}%
\pgfpathlineto{\pgfqpoint{1.182741in}{1.036220in}}%
\pgfpathlineto{\pgfqpoint{1.172251in}{1.052078in}}%
\pgfpathlineto{\pgfqpoint{1.165444in}{1.067314in}}%
\pgfpathlineto{\pgfqpoint{1.161518in}{1.081904in}}%
\pgfpathlineto{\pgfqpoint{1.160152in}{1.095832in}}%
\pgfpathlineto{\pgfqpoint{1.161102in}{1.109084in}}%
\pgfpathlineto{\pgfqpoint{1.164201in}{1.121652in}}%
\pgfpathlineto{\pgfqpoint{1.169358in}{1.133530in}}%
\pgfpathlineto{\pgfqpoint{1.176560in}{1.144714in}}%
\pgfpathlineto{\pgfqpoint{1.185870in}{1.155204in}}%
\pgfpathlineto{\pgfqpoint{1.197337in}{1.165001in}}%
\pgfpathlineto{\pgfqpoint{1.210775in}{1.174101in}}%
\pgfpathlineto{\pgfqpoint{1.226125in}{1.182501in}}%
\pgfpathlineto{\pgfqpoint{1.252687in}{1.193786in}}%
\pgfpathlineto{\pgfqpoint{1.283529in}{1.203484in}}%
\pgfpathlineto{\pgfqpoint{1.318809in}{1.211590in}}%
\pgfpathlineto{\pgfqpoint{1.358816in}{1.218098in}}%
\pgfpathlineto{\pgfqpoint{1.403971in}{1.223003in}}%
\pgfpathlineto{\pgfqpoint{1.454688in}{1.226296in}}%
\pgfpathlineto{\pgfqpoint{1.511160in}{1.227935in}}%
\pgfpathlineto{\pgfqpoint{1.574230in}{1.227861in}}%
\pgfpathlineto{\pgfqpoint{1.644702in}{1.226009in}}%
\pgfpathlineto{\pgfqpoint{1.751356in}{1.220644in}}%
\pgfpathlineto{\pgfqpoint{1.873696in}{1.211785in}}%
\pgfpathlineto{\pgfqpoint{2.012672in}{1.199206in}}%
\pgfpathlineto{\pgfqpoint{2.168877in}{1.182653in}}%
\pgfpathlineto{\pgfqpoint{2.343064in}{1.161829in}}%
\pgfpathlineto{\pgfqpoint{2.480773in}{1.143383in}}%
\pgfpathlineto{\pgfqpoint{2.619284in}{1.122630in}}%
\pgfpathlineto{\pgfqpoint{2.754391in}{1.099758in}}%
\pgfpathlineto{\pgfqpoint{2.882369in}{1.075029in}}%
\pgfpathlineto{\pgfqpoint{2.962105in}{1.057674in}}%
\pgfpathlineto{\pgfqpoint{3.036379in}{1.039761in}}%
\pgfpathlineto{\pgfqpoint{3.104444in}{1.021423in}}%
\pgfpathlineto{\pgfqpoint{3.165649in}{1.002808in}}%
\pgfpathlineto{\pgfqpoint{3.219437in}{0.984078in}}%
\pgfpathlineto{\pgfqpoint{3.265679in}{0.965393in}}%
\pgfpathlineto{\pgfqpoint{3.305312in}{0.946854in}}%
\pgfpathlineto{\pgfqpoint{3.338965in}{0.928560in}}%
\pgfpathlineto{\pgfqpoint{3.367180in}{0.910602in}}%
\pgfpathlineto{\pgfqpoint{3.390446in}{0.893059in}}%
\pgfpathlineto{\pgfqpoint{3.409192in}{0.875996in}}%
\pgfpathlineto{\pgfqpoint{3.423866in}{0.859469in}}%
\pgfpathlineto{\pgfqpoint{3.434959in}{0.843518in}}%
\pgfpathlineto{\pgfqpoint{3.442850in}{0.828173in}}%
\pgfpathlineto{\pgfqpoint{3.447841in}{0.813462in}}%
\pgfpathlineto{\pgfqpoint{3.450157in}{0.799403in}}%
\pgfpathlineto{\pgfqpoint{3.449948in}{0.786012in}}%
\pgfpathlineto{\pgfqpoint{3.447293in}{0.773299in}}%
\pgfpathlineto{\pgfqpoint{3.442384in}{0.761269in}}%
\pgfpathlineto{\pgfqpoint{3.435411in}{0.749931in}}%
\pgfpathlineto{\pgfqpoint{3.426515in}{0.739287in}}%
\pgfpathlineto{\pgfqpoint{3.415786in}{0.729343in}}%
\pgfpathlineto{\pgfqpoint{3.403264in}{0.720098in}}%
\pgfpathlineto{\pgfqpoint{3.388937in}{0.711552in}}%
\pgfpathlineto{\pgfqpoint{3.372744in}{0.703704in}}%
\pgfpathlineto{\pgfqpoint{3.344696in}{0.693231in}}%
\pgfpathlineto{\pgfqpoint{3.311740in}{0.684307in}}%
\pgfpathlineto{\pgfqpoint{3.274068in}{0.676957in}}%
\pgfpathlineto{\pgfqpoint{3.231483in}{0.671203in}}%
\pgfpathlineto{\pgfqpoint{3.183625in}{0.667070in}}%
\pgfpathlineto{\pgfqpoint{3.130059in}{0.664591in}}%
\pgfpathlineto{\pgfqpoint{3.070269in}{0.663807in}}%
\pgfpathlineto{\pgfqpoint{3.003663in}{0.664764in}}%
\pgfpathlineto{\pgfqpoint{2.903081in}{0.668851in}}%
\pgfpathlineto{\pgfqpoint{2.787566in}{0.676315in}}%
\pgfpathlineto{\pgfqpoint{2.655273in}{0.687419in}}%
\pgfpathlineto{\pgfqpoint{2.505501in}{0.702411in}}%
\pgfpathlineto{\pgfqpoint{2.339289in}{0.721508in}}%
\pgfpathlineto{\pgfqpoint{2.159636in}{0.744873in}}%
\pgfpathlineto{\pgfqpoint{2.020856in}{0.765163in}}%
\pgfpathlineto{\pgfqpoint{1.883617in}{0.787659in}}%
\pgfpathlineto{\pgfqpoint{1.752329in}{0.812108in}}%
\pgfpathlineto{\pgfqpoint{1.670242in}{0.829322in}}%
\pgfpathlineto{\pgfqpoint{1.593972in}{0.847121in}}%
\pgfpathlineto{\pgfqpoint{1.524651in}{0.865358in}}%
\pgfpathlineto{\pgfqpoint{1.462551in}{0.883885in}}%
\pgfpathlineto{\pgfqpoint{1.407488in}{0.902558in}}%
\pgfpathlineto{\pgfqpoint{1.359222in}{0.921246in}}%
\pgfpathlineto{\pgfqpoint{1.317460in}{0.939829in}}%
\pgfpathlineto{\pgfqpoint{1.281852in}{0.958199in}}%
\pgfpathlineto{\pgfqpoint{1.251997in}{0.976260in}}%
\pgfpathlineto{\pgfqpoint{1.227440in}{0.993928in}}%
\pgfpathlineto{\pgfqpoint{1.207687in}{1.011129in}}%
\pgfpathlineto{\pgfqpoint{1.192182in}{1.027803in}}%
\pgfpathlineto{\pgfqpoint{1.180303in}{1.043911in}}%
\pgfpathlineto{\pgfqpoint{1.171554in}{1.059423in}}%
\pgfpathlineto{\pgfqpoint{1.165578in}{1.074313in}}%
\pgfpathlineto{\pgfqpoint{1.162153in}{1.088557in}}%
\pgfpathlineto{\pgfqpoint{1.161195in}{1.102142in}}%
\pgfpathlineto{\pgfqpoint{1.162758in}{1.115055in}}%
\pgfpathlineto{\pgfqpoint{1.166954in}{1.127290in}}%
\pgfpathlineto{\pgfqpoint{1.173407in}{1.138836in}}%
\pgfpathlineto{\pgfqpoint{1.181905in}{1.149686in}}%
\pgfpathlineto{\pgfqpoint{1.192338in}{1.159839in}}%
\pgfpathlineto{\pgfqpoint{1.204637in}{1.169290in}}%
\pgfpathlineto{\pgfqpoint{1.218767in}{1.178040in}}%
\pgfpathlineto{\pgfqpoint{1.234732in}{1.186088in}}%
\pgfpathlineto{\pgfqpoint{1.262224in}{1.196846in}}%
\pgfpathlineto{\pgfqpoint{1.294246in}{1.206035in}}%
\pgfpathlineto{\pgfqpoint{1.331118in}{1.213657in}}%
\pgfpathlineto{\pgfqpoint{1.372838in}{1.219691in}}%
\pgfpathlineto{\pgfqpoint{1.419761in}{1.224112in}}%
\pgfpathlineto{\pgfqpoint{1.472332in}{1.226887in}}%
\pgfpathlineto{\pgfqpoint{1.531057in}{1.227977in}}%
\pgfpathlineto{\pgfqpoint{1.596510in}{1.227329in}}%
\pgfpathlineto{\pgfqpoint{1.695353in}{1.223660in}}%
\pgfpathlineto{\pgfqpoint{1.808972in}{1.216629in}}%
\pgfpathlineto{\pgfqpoint{1.938997in}{1.206008in}}%
\pgfpathlineto{\pgfqpoint{2.086516in}{1.191530in}}%
\pgfpathlineto{\pgfqpoint{2.250854in}{1.172965in}}%
\pgfpathlineto{\pgfqpoint{2.429015in}{1.150150in}}%
\pgfpathlineto{\pgfqpoint{2.567496in}{1.130259in}}%
\pgfpathlineto{\pgfqpoint{2.705608in}{1.108119in}}%
\pgfpathlineto{\pgfqpoint{2.838611in}{1.083966in}}%
\pgfpathlineto{\pgfqpoint{2.921817in}{1.066897in}}%
\pgfpathlineto{\pgfqpoint{2.999232in}{1.049195in}}%
\pgfpathlineto{\pgfqpoint{3.070263in}{1.031027in}}%
\pgfpathlineto{\pgfqpoint{3.134510in}{1.012549in}}%
\pgfpathlineto{\pgfqpoint{3.191756in}{0.993905in}}%
\pgfpathlineto{\pgfqpoint{3.241971in}{0.975225in}}%
\pgfpathlineto{\pgfqpoint{3.285309in}{0.956628in}}%
\pgfpathlineto{\pgfqpoint{3.322112in}{0.938221in}}%
\pgfpathlineto{\pgfqpoint{3.352905in}{0.920098in}}%
\pgfpathlineto{\pgfqpoint{3.378400in}{0.902342in}}%
\pgfpathlineto{\pgfqpoint{3.399275in}{0.885029in}}%
\pgfpathlineto{\pgfqpoint{3.415595in}{0.868235in}}%
\pgfpathlineto{\pgfqpoint{3.428139in}{0.851996in}}%
\pgfpathlineto{\pgfqpoint{3.437599in}{0.836343in}}%
\pgfpathlineto{\pgfqpoint{3.444478in}{0.821298in}}%
\pgfpathlineto{\pgfqpoint{3.449084in}{0.806884in}}%
\pgfpathlineto{\pgfqpoint{3.451535in}{0.793116in}}%
\pgfpathlineto{\pgfqpoint{3.451755in}{0.780007in}}%
\pgfpathlineto{\pgfqpoint{3.449477in}{0.767565in}}%
\pgfpathlineto{\pgfqpoint{3.444241in}{0.755794in}}%
\pgfpathlineto{\pgfqpoint{3.436190in}{0.744706in}}%
\pgfpathlineto{\pgfqpoint{3.426132in}{0.734320in}}%
\pgfpathlineto{\pgfqpoint{3.414154in}{0.724637in}}%
\pgfpathlineto{\pgfqpoint{3.400308in}{0.715658in}}%
\pgfpathlineto{\pgfqpoint{3.384617in}{0.707384in}}%
\pgfpathlineto{\pgfqpoint{3.357594in}{0.696297in}}%
\pgfpathlineto{\pgfqpoint{3.326237in}{0.686796in}}%
\pgfpathlineto{\pgfqpoint{3.290238in}{0.678877in}}%
\pgfpathlineto{\pgfqpoint{3.249299in}{0.672541in}}%
\pgfpathlineto{\pgfqpoint{3.203291in}{0.667813in}}%
\pgfpathlineto{\pgfqpoint{3.151729in}{0.664724in}}%
\pgfpathlineto{\pgfqpoint{3.094076in}{0.663312in}}%
\pgfpathlineto{\pgfqpoint{3.029757in}{0.663630in}}%
\pgfpathlineto{\pgfqpoint{2.932559in}{0.666848in}}%
\pgfpathlineto{\pgfqpoint{2.820846in}{0.673421in}}%
\pgfpathlineto{\pgfqpoint{2.692951in}{0.683556in}}%
\pgfpathlineto{\pgfqpoint{2.547661in}{0.697509in}}%
\pgfpathlineto{\pgfqpoint{2.385186in}{0.715536in}}%
\pgfpathlineto{\pgfqpoint{2.208421in}{0.737799in}}%
\pgfpathlineto{\pgfqpoint{2.070008in}{0.757293in}}%
\pgfpathlineto{\pgfqpoint{1.931301in}{0.779072in}}%
\pgfpathlineto{\pgfqpoint{1.796985in}{0.802914in}}%
\pgfpathlineto{\pgfqpoint{1.712187in}{0.819799in}}%
\pgfpathlineto{\pgfqpoint{1.632870in}{0.837338in}}%
\pgfpathlineto{\pgfqpoint{1.560613in}{0.855397in}}%
\pgfpathlineto{\pgfqpoint{1.495436in}{0.873817in}}%
\pgfpathlineto{\pgfqpoint{1.437070in}{0.892449in}}%
\pgfpathlineto{\pgfqpoint{1.385242in}{0.911156in}}%
\pgfpathlineto{\pgfqpoint{1.339675in}{0.929812in}}%
\pgfpathlineto{\pgfqpoint{1.300090in}{0.948306in}}%
\pgfpathlineto{\pgfqpoint{1.266203in}{0.966541in}}%
\pgfpathlineto{\pgfqpoint{1.237725in}{0.984432in}}%
\pgfpathlineto{\pgfqpoint{1.214366in}{1.001907in}}%
\pgfpathlineto{\pgfqpoint{1.195831in}{1.018907in}}%
\pgfpathlineto{\pgfqpoint{1.181820in}{1.035386in}}%
\pgfpathlineto{\pgfqpoint{1.171958in}{1.051291in}}%
\pgfpathlineto{\pgfqpoint{1.165565in}{1.066567in}}%
\pgfpathlineto{\pgfqpoint{1.161993in}{1.081194in}}%
\pgfpathlineto{\pgfqpoint{1.160737in}{1.095160in}}%
\pgfpathlineto{\pgfqpoint{1.161437in}{1.108453in}}%
\pgfpathlineto{\pgfqpoint{1.163872in}{1.121066in}}%
\pgfpathlineto{\pgfqpoint{1.167967in}{1.132993in}}%
\pgfpathlineto{\pgfqpoint{1.173788in}{1.144234in}}%
\pgfpathlineto{\pgfqpoint{1.181542in}{1.154790in}}%
\pgfpathlineto{\pgfqpoint{1.191582in}{1.164666in}}%
\pgfpathlineto{\pgfqpoint{1.204402in}{1.173871in}}%
\pgfpathlineto{\pgfqpoint{1.219764in}{1.182390in}}%
\pgfpathlineto{\pgfqpoint{1.246386in}{1.193846in}}%
\pgfpathlineto{\pgfqpoint{1.277332in}{1.203709in}}%
\pgfpathlineto{\pgfqpoint{1.312734in}{1.211971in}}%
\pgfpathlineto{\pgfqpoint{1.352828in}{1.218625in}}%
\pgfpathlineto{\pgfqpoint{1.397961in}{1.223657in}}%
\pgfpathlineto{\pgfqpoint{1.448585in}{1.227054in}}%
\pgfpathlineto{\pgfqpoint{1.505006in}{1.228799in}}%
\pgfpathlineto{\pgfqpoint{1.567656in}{1.228843in}}%
\pgfpathlineto{\pgfqpoint{1.637647in}{1.227108in}}%
\pgfpathlineto{\pgfqpoint{1.743901in}{1.221885in}}%
\pgfpathlineto{\pgfqpoint{1.866228in}{1.213148in}}%
\pgfpathlineto{\pgfqpoint{2.005323in}{1.200684in}}%
\pgfpathlineto{\pgfqpoint{2.161123in}{1.184266in}}%
\pgfpathlineto{\pgfqpoint{2.332844in}{1.163655in}}%
\pgfpathlineto{\pgfqpoint{2.470525in}{1.145293in}}%
\pgfpathlineto{\pgfqpoint{2.610489in}{1.124561in}}%
\pgfpathlineto{\pgfqpoint{2.747543in}{1.101694in}}%
\pgfpathlineto{\pgfqpoint{2.877306in}{1.076980in}}%
\pgfpathlineto{\pgfqpoint{2.957966in}{1.059646in}}%
\pgfpathlineto{\pgfqpoint{3.032922in}{1.041756in}}%
\pgfpathlineto{\pgfqpoint{3.101476in}{1.023437in}}%
\pgfpathlineto{\pgfqpoint{3.163089in}{1.004823in}}%
\pgfpathlineto{\pgfqpoint{3.217382in}{0.986063in}}%
\pgfpathlineto{\pgfqpoint{3.264169in}{0.967313in}}%
\pgfpathlineto{\pgfqpoint{3.303999in}{0.948712in}}%
\pgfpathlineto{\pgfqpoint{3.337800in}{0.930358in}}%
\pgfpathlineto{\pgfqpoint{3.366329in}{0.912333in}}%
\pgfpathlineto{\pgfqpoint{3.390175in}{0.894711in}}%
\pgfpathlineto{\pgfqpoint{3.409759in}{0.877558in}}%
\pgfpathlineto{\pgfqpoint{3.425335in}{0.860930in}}%
\pgfpathlineto{\pgfqpoint{3.436990in}{0.844873in}}%
\pgfpathlineto{\pgfqpoint{3.444904in}{0.829426in}}%
\pgfpathlineto{\pgfqpoint{3.449817in}{0.814619in}}%
\pgfpathlineto{\pgfqpoint{3.452069in}{0.800469in}}%
\pgfpathlineto{\pgfqpoint{3.451908in}{0.786990in}}%
\pgfpathlineto{\pgfqpoint{3.449509in}{0.774191in}}%
\pgfpathlineto{\pgfqpoint{3.444980in}{0.762082in}}%
\pgfpathlineto{\pgfqpoint{3.438358in}{0.750666in}}%
\pgfpathlineto{\pgfqpoint{3.429610in}{0.739943in}}%
\pgfpathlineto{\pgfqpoint{3.418776in}{0.729915in}}%
\pgfpathlineto{\pgfqpoint{3.405981in}{0.720584in}}%
\pgfpathlineto{\pgfqpoint{3.391284in}{0.711953in}}%
\pgfpathlineto{\pgfqpoint{3.365737in}{0.700323in}}%
\pgfpathlineto{\pgfqpoint{3.335966in}{0.690279in}}%
\pgfpathlineto{\pgfqpoint{3.301815in}{0.681824in}}%
\pgfpathlineto{\pgfqpoint{3.262986in}{0.674961in}}%
\pgfpathlineto{\pgfqpoint{3.219035in}{0.669691in}}%
\pgfpathlineto{\pgfqpoint{3.169616in}{0.666024in}}%
\pgfpathlineto{\pgfqpoint{3.114491in}{0.664008in}}%
\pgfpathlineto{\pgfqpoint{3.052890in}{0.663693in}}%
\pgfpathlineto{\pgfqpoint{2.984089in}{0.665140in}}%
\pgfpathlineto{\pgfqpoint{2.880021in}{0.669934in}}%
\pgfpathlineto{\pgfqpoint{2.760601in}{0.678181in}}%
\pgfpathlineto{\pgfqpoint{2.624644in}{0.690108in}}%
\pgfpathlineto{\pgfqpoint{2.470712in}{0.705985in}}%
\pgfpathlineto{\pgfqpoint{2.296661in}{0.726153in}}%
\pgfpathlineto{\pgfqpoint{2.158909in}{0.744099in}}%
\pgfpathlineto{\pgfqpoint{2.020308in}{0.764353in}}%
\pgfpathlineto{\pgfqpoint{1.885011in}{0.786745in}}%
\pgfpathlineto{\pgfqpoint{1.756581in}{0.811039in}}%
\pgfpathlineto{\pgfqpoint{1.676267in}{0.828146in}}%
\pgfpathlineto{\pgfqpoint{1.601070in}{0.845860in}}%
\pgfpathlineto{\pgfqpoint{1.531605in}{0.864061in}}%
\pgfpathlineto{\pgfqpoint{1.468372in}{0.882619in}}%
\pgfpathlineto{\pgfqpoint{1.411752in}{0.901389in}}%
\pgfpathlineto{\pgfqpoint{1.362010in}{0.920212in}}%
\pgfpathlineto{\pgfqpoint{1.319291in}{0.938917in}}%
\pgfpathlineto{\pgfqpoint{1.283564in}{0.957330in}}%
\pgfpathlineto{\pgfqpoint{1.253764in}{0.975402in}}%
\pgfpathlineto{\pgfqpoint{1.229088in}{0.993081in}}%
\pgfpathlineto{\pgfqpoint{1.209016in}{1.010298in}}%
\pgfpathlineto{\pgfqpoint{1.193082in}{1.026998in}}%
\pgfpathlineto{\pgfqpoint{1.180860in}{1.043134in}}%
\pgfpathlineto{\pgfqpoint{1.171970in}{1.058672in}}%
\pgfpathlineto{\pgfqpoint{1.166098in}{1.073585in}}%
\pgfpathlineto{\pgfqpoint{1.162995in}{1.087850in}}%
\pgfpathlineto{\pgfqpoint{1.162571in}{1.101452in}}%
\pgfpathlineto{\pgfqpoint{1.164602in}{1.114380in}}%
\pgfpathlineto{\pgfqpoint{1.168822in}{1.126624in}}%
\pgfpathlineto{\pgfqpoint{1.175031in}{1.138179in}}%
\pgfpathlineto{\pgfqpoint{1.183096in}{1.149038in}}%
\pgfpathlineto{\pgfqpoint{1.192949in}{1.159201in}}%
\pgfpathlineto{\pgfqpoint{1.204588in}{1.168667in}}%
\pgfpathlineto{\pgfqpoint{1.218075in}{1.177437in}}%
\pgfpathlineto{\pgfqpoint{1.233541in}{1.185516in}}%
\pgfpathlineto{\pgfqpoint{1.251181in}{1.192910in}}%
\pgfpathlineto{\pgfqpoint{1.281756in}{1.202712in}}%
\pgfpathlineto{\pgfqpoint{1.316881in}{1.210939in}}%
\pgfpathlineto{\pgfqpoint{1.356756in}{1.217574in}}%
\pgfpathlineto{\pgfqpoint{1.401679in}{1.222598in}}%
\pgfpathlineto{\pgfqpoint{1.452040in}{1.225983in}}%
\pgfpathlineto{\pgfqpoint{1.508318in}{1.227697in}}%
\pgfpathlineto{\pgfqpoint{1.571080in}{1.227703in}}%
\pgfpathlineto{\pgfqpoint{1.665860in}{1.224978in}}%
\pgfpathlineto{\pgfqpoint{1.774739in}{1.218967in}}%
\pgfpathlineto{\pgfqpoint{1.900194in}{1.209390in}}%
\pgfpathlineto{\pgfqpoint{2.043225in}{1.195999in}}%
\pgfpathlineto{\pgfqpoint{2.203086in}{1.178591in}}%
\pgfpathlineto{\pgfqpoint{2.377278in}{1.157009in}}%
\pgfpathlineto{\pgfqpoint{2.514796in}{1.138007in}}%
\pgfpathlineto{\pgfqpoint{2.654000in}{1.116636in}}%
\pgfpathlineto{\pgfqpoint{2.789404in}{1.093178in}}%
\pgfpathlineto{\pgfqpoint{2.875247in}{1.076546in}}%
\pgfpathlineto{\pgfqpoint{2.956215in}{1.059246in}}%
\pgfpathlineto{\pgfqpoint{3.031387in}{1.041396in}}%
\pgfpathlineto{\pgfqpoint{3.100042in}{1.023121in}}%
\pgfpathlineto{\pgfqpoint{3.161653in}{1.004557in}}%
\pgfpathlineto{\pgfqpoint{3.215895in}{0.985844in}}%
\pgfpathlineto{\pgfqpoint{3.262652in}{0.967136in}}%
\pgfpathlineto{\pgfqpoint{3.302453in}{0.948578in}}%
\pgfpathlineto{\pgfqpoint{3.336278in}{0.930262in}}%
\pgfpathlineto{\pgfqpoint{3.364939in}{0.912268in}}%
\pgfpathlineto{\pgfqpoint{3.389050in}{0.894670in}}%
\pgfpathlineto{\pgfqpoint{3.409026in}{0.877531in}}%
\pgfpathlineto{\pgfqpoint{3.425084in}{0.860908in}}%
\pgfpathlineto{\pgfqpoint{3.437244in}{0.844849in}}%
\pgfpathlineto{\pgfqpoint{3.445382in}{0.829396in}}%
\pgfpathlineto{\pgfqpoint{3.450245in}{0.814583in}}%
\pgfpathlineto{\pgfqpoint{3.452416in}{0.800427in}}%
\pgfpathlineto{\pgfqpoint{3.452167in}{0.786944in}}%
\pgfpathlineto{\pgfqpoint{3.449701in}{0.774143in}}%
\pgfpathlineto{\pgfqpoint{3.445150in}{0.762031in}}%
\pgfpathlineto{\pgfqpoint{3.438579in}{0.750613in}}%
\pgfpathlineto{\pgfqpoint{3.429979in}{0.739891in}}%
\pgfpathlineto{\pgfqpoint{3.419274in}{0.729863in}}%
\pgfpathlineto{\pgfqpoint{3.406499in}{0.720530in}}%
\pgfpathlineto{\pgfqpoint{3.391786in}{0.711895in}}%
\pgfpathlineto{\pgfqpoint{3.366153in}{0.700256in}}%
\pgfpathlineto{\pgfqpoint{3.336242in}{0.690201in}}%
\pgfpathlineto{\pgfqpoint{3.301938in}{0.681737in}}%
\pgfpathlineto{\pgfqpoint{3.262996in}{0.674871in}}%
\pgfpathlineto{\pgfqpoint{3.219045in}{0.669610in}}%
\pgfpathlineto{\pgfqpoint{3.169588in}{0.665963in}}%
\pgfpathlineto{\pgfqpoint{3.114433in}{0.663951in}}%
\pgfpathlineto{\pgfqpoint{3.052989in}{0.663635in}}%
\pgfpathlineto{\pgfqpoint{2.984318in}{0.665080in}}%
\pgfpathlineto{\pgfqpoint{2.880192in}{0.669880in}}%
\pgfpathlineto{\pgfqpoint{2.760443in}{0.678143in}}%
\pgfpathlineto{\pgfqpoint{2.624161in}{0.690088in}}%
\pgfpathlineto{\pgfqpoint{2.470941in}{0.705954in}}%
\pgfpathlineto{\pgfqpoint{2.300551in}{0.726020in}}%
\pgfpathlineto{\pgfqpoint{2.163751in}{0.743934in}}%
\pgfpathlineto{\pgfqpoint{2.024694in}{0.764197in}}%
\pgfpathlineto{\pgfqpoint{1.887983in}{0.786628in}}%
\pgfpathlineto{\pgfqpoint{1.757667in}{0.810974in}}%
\pgfpathlineto{\pgfqpoint{1.676087in}{0.828112in}}%
\pgfpathlineto{\pgfqpoint{1.599805in}{0.845845in}}%
\pgfpathlineto{\pgfqpoint{1.529609in}{0.864046in}}%
\pgfpathlineto{\pgfqpoint{1.466171in}{0.882573in}}%
\pgfpathlineto{\pgfqpoint{1.410057in}{0.901271in}}%
\pgfpathlineto{\pgfqpoint{1.361522in}{0.919980in}}%
\pgfpathlineto{\pgfqpoint{1.319805in}{0.938579in}}%
\pgfpathlineto{\pgfqpoint{1.284213in}{0.956967in}}%
\pgfpathlineto{\pgfqpoint{1.254163in}{0.975049in}}%
\pgfpathlineto{\pgfqpoint{1.229158in}{0.992745in}}%
\pgfpathlineto{\pgfqpoint{1.208789in}{1.009983in}}%
\pgfpathlineto{\pgfqpoint{1.192729in}{1.026704in}}%
\pgfpathlineto{\pgfqpoint{1.180524in}{1.042859in}}%
\pgfpathlineto{\pgfqpoint{1.171659in}{1.058416in}}%
\pgfpathlineto{\pgfqpoint{1.165777in}{1.073347in}}%
\pgfpathlineto{\pgfqpoint{1.162600in}{1.087631in}}%
\pgfpathlineto{\pgfqpoint{1.161928in}{1.101252in}}%
\pgfpathlineto{\pgfqpoint{1.163643in}{1.114197in}}%
\pgfpathlineto{\pgfqpoint{1.167697in}{1.126460in}}%
\pgfpathlineto{\pgfqpoint{1.173977in}{1.138036in}}%
\pgfpathlineto{\pgfqpoint{1.182322in}{1.148919in}}%
\pgfpathlineto{\pgfqpoint{1.192605in}{1.159105in}}%
\pgfpathlineto{\pgfqpoint{1.204743in}{1.168593in}}%
\pgfpathlineto{\pgfqpoint{1.218692in}{1.177379in}}%
\pgfpathlineto{\pgfqpoint{1.234450in}{1.185463in}}%
\pgfpathlineto{\pgfqpoint{1.261570in}{1.196277in}}%
\pgfpathlineto{\pgfqpoint{1.293154in}{1.205521in}}%
\pgfpathlineto{\pgfqpoint{1.329667in}{1.213207in}}%
\pgfpathlineto{\pgfqpoint{1.371113in}{1.219313in}}%
\pgfpathlineto{\pgfqpoint{1.417731in}{1.223810in}}%
\pgfpathlineto{\pgfqpoint{1.469975in}{1.226666in}}%
\pgfpathlineto{\pgfqpoint{1.528360in}{1.227841in}}%
\pgfpathlineto{\pgfqpoint{1.593454in}{1.227283in}}%
\pgfpathlineto{\pgfqpoint{1.691774in}{1.223737in}}%
\pgfpathlineto{\pgfqpoint{1.804782in}{1.216832in}}%
\pgfpathlineto{\pgfqpoint{1.934132in}{1.206350in}}%
\pgfpathlineto{\pgfqpoint{2.080939in}{1.192024in}}%
\pgfpathlineto{\pgfqpoint{2.244646in}{1.173618in}}%
\pgfpathlineto{\pgfqpoint{2.422308in}{1.150967in}}%
\pgfpathlineto{\pgfqpoint{2.560658in}{1.131198in}}%
\pgfpathlineto{\pgfqpoint{2.698883in}{1.109172in}}%
\pgfpathlineto{\pgfqpoint{2.832205in}{1.085117in}}%
\pgfpathlineto{\pgfqpoint{2.915728in}{1.068095in}}%
\pgfpathlineto{\pgfqpoint{2.993629in}{1.050437in}}%
\pgfpathlineto{\pgfqpoint{3.065231in}{1.032307in}}%
\pgfpathlineto{\pgfqpoint{3.130066in}{1.013858in}}%
\pgfpathlineto{\pgfqpoint{3.187870in}{0.995232in}}%
\pgfpathlineto{\pgfqpoint{3.238588in}{0.976558in}}%
\pgfpathlineto{\pgfqpoint{3.282369in}{0.957955in}}%
\pgfpathlineto{\pgfqpoint{3.319567in}{0.939530in}}%
\pgfpathlineto{\pgfqpoint{3.350745in}{0.921381in}}%
\pgfpathlineto{\pgfqpoint{3.376671in}{0.903592in}}%
\pgfpathlineto{\pgfqpoint{3.397755in}{0.886250in}}%
\pgfpathlineto{\pgfqpoint{3.414323in}{0.869422in}}%
\pgfpathlineto{\pgfqpoint{3.427228in}{0.853144in}}%
\pgfpathlineto{\pgfqpoint{3.437121in}{0.837444in}}%
\pgfpathlineto{\pgfqpoint{3.444456in}{0.822349in}}%
\pgfpathlineto{\pgfqpoint{3.449480in}{0.807881in}}%
\pgfpathlineto{\pgfqpoint{3.452242in}{0.794057in}}%
\pgfpathlineto{\pgfqpoint{3.452586in}{0.780891in}}%
\pgfpathlineto{\pgfqpoint{3.450155in}{0.768390in}}%
\pgfpathlineto{\pgfqpoint{3.444547in}{0.756563in}}%
\pgfpathlineto{\pgfqpoint{3.436608in}{0.745431in}}%
\pgfpathlineto{\pgfqpoint{3.426683in}{0.735000in}}%
\pgfpathlineto{\pgfqpoint{3.414855in}{0.725272in}}%
\pgfpathlineto{\pgfqpoint{3.401172in}{0.716249in}}%
\pgfpathlineto{\pgfqpoint{3.385652in}{0.707931in}}%
\pgfpathlineto{\pgfqpoint{3.358881in}{0.696777in}}%
\pgfpathlineto{\pgfqpoint{3.327740in}{0.687205in}}%
\pgfpathlineto{\pgfqpoint{3.291882in}{0.679210in}}%
\pgfpathlineto{\pgfqpoint{3.251178in}{0.672800in}}%
\pgfpathlineto{\pgfqpoint{3.205387in}{0.667999in}}%
\pgfpathlineto{\pgfqpoint{3.154054in}{0.664836in}}%
\pgfpathlineto{\pgfqpoint{3.096672in}{0.663353in}}%
\pgfpathlineto{\pgfqpoint{3.032679in}{0.663596in}}%
\pgfpathlineto{\pgfqpoint{2.935996in}{0.666710in}}%
\pgfpathlineto{\pgfqpoint{2.824841in}{0.673168in}}%
\pgfpathlineto{\pgfqpoint{2.697521in}{0.683173in}}%
\pgfpathlineto{\pgfqpoint{2.552797in}{0.696995in}}%
\pgfpathlineto{\pgfqpoint{2.390926in}{0.714877in}}%
\pgfpathlineto{\pgfqpoint{2.214510in}{0.736996in}}%
\pgfpathlineto{\pgfqpoint{2.076341in}{0.756381in}}%
\pgfpathlineto{\pgfqpoint{1.937561in}{0.778058in}}%
\pgfpathlineto{\pgfqpoint{1.802825in}{0.801813in}}%
\pgfpathlineto{\pgfqpoint{1.717834in}{0.818661in}}%
\pgfpathlineto{\pgfqpoint{1.638315in}{0.836190in}}%
\pgfpathlineto{\pgfqpoint{1.564984in}{0.854237in}}%
\pgfpathlineto{\pgfqpoint{1.498340in}{0.872642in}}%
\pgfpathlineto{\pgfqpoint{1.438682in}{0.891261in}}%
\pgfpathlineto{\pgfqpoint{1.386111in}{0.909959in}}%
\pgfpathlineto{\pgfqpoint{1.340530in}{0.928614in}}%
\pgfpathlineto{\pgfqpoint{1.301641in}{0.947114in}}%
\pgfpathlineto{\pgfqpoint{1.268951in}{0.965362in}}%
\pgfpathlineto{\pgfqpoint{1.241766in}{0.983268in}}%
\pgfpathlineto{\pgfqpoint{1.219377in}{1.000753in}}%
\pgfpathlineto{\pgfqpoint{1.201706in}{1.017735in}}%
\pgfpathlineto{\pgfqpoint{1.187961in}{1.034174in}}%
\pgfpathlineto{\pgfqpoint{1.177416in}{1.050040in}}%
\pgfpathlineto{\pgfqpoint{1.169544in}{1.065306in}}%
\pgfpathlineto{\pgfqpoint{1.164016in}{1.079948in}}%
\pgfpathlineto{\pgfqpoint{1.160706in}{1.093950in}}%
\pgfpathlineto{\pgfqpoint{1.159685in}{1.107297in}}%
\pgfpathlineto{\pgfqpoint{1.161225in}{1.119980in}}%
\pgfpathlineto{\pgfqpoint{1.165797in}{1.131991in}}%
\pgfpathlineto{\pgfqpoint{1.173190in}{1.143318in}}%
\pgfpathlineto{\pgfqpoint{1.182604in}{1.153944in}}%
\pgfpathlineto{\pgfqpoint{1.193946in}{1.163867in}}%
\pgfpathlineto{\pgfqpoint{1.207157in}{1.173084in}}%
\pgfpathlineto{\pgfqpoint{1.222209in}{1.181597in}}%
\pgfpathlineto{\pgfqpoint{1.248256in}{1.193042in}}%
\pgfpathlineto{\pgfqpoint{1.278610in}{1.202901in}}%
\pgfpathlineto{\pgfqpoint{1.313578in}{1.211180in}}%
\pgfpathlineto{\pgfqpoint{1.353426in}{1.217879in}}%
\pgfpathlineto{\pgfqpoint{1.398275in}{1.222973in}}%
\pgfpathlineto{\pgfqpoint{1.448577in}{1.226434in}}%
\pgfpathlineto{\pgfqpoint{1.504837in}{1.228225in}}%
\pgfpathlineto{\pgfqpoint{1.567611in}{1.228299in}}%
\pgfpathlineto{\pgfqpoint{1.637507in}{1.226598in}}%
\pgfpathlineto{\pgfqpoint{1.742922in}{1.221449in}}%
\pgfpathlineto{\pgfqpoint{1.863860in}{1.212835in}}%
\pgfpathlineto{\pgfqpoint{2.001809in}{1.200513in}}%
\pgfpathlineto{\pgfqpoint{2.157287in}{1.184217in}}%
\pgfpathlineto{\pgfqpoint{2.328614in}{1.163739in}}%
\pgfpathlineto{\pgfqpoint{2.464899in}{1.145574in}}%
\pgfpathlineto{\pgfqpoint{2.603941in}{1.125050in}}%
\pgfpathlineto{\pgfqpoint{2.741368in}{1.102325in}}%
\pgfpathlineto{\pgfqpoint{2.872310in}{1.077663in}}%
\pgfpathlineto{\pgfqpoint{2.953412in}{1.060306in}}%
\pgfpathlineto{\pgfqpoint{3.028422in}{1.042381in}}%
\pgfpathlineto{\pgfqpoint{3.096840in}{1.024052in}}%
\pgfpathlineto{\pgfqpoint{3.158348in}{1.005471in}}%
\pgfpathlineto{\pgfqpoint{3.212813in}{0.986776in}}%
\pgfpathlineto{\pgfqpoint{3.260283in}{0.968094in}}%
\pgfpathlineto{\pgfqpoint{3.300991in}{0.949539in}}%
\pgfpathlineto{\pgfqpoint{3.335351in}{0.931215in}}%
\pgfpathlineto{\pgfqpoint{3.363963in}{0.913210in}}%
\pgfpathlineto{\pgfqpoint{3.387605in}{0.895602in}}%
\pgfpathlineto{\pgfqpoint{3.406588in}{0.878474in}}%
\pgfpathlineto{\pgfqpoint{3.421291in}{0.861882in}}%
\pgfpathlineto{\pgfqpoint{3.432532in}{0.845860in}}%
\pgfpathlineto{\pgfqpoint{3.440927in}{0.830432in}}%
\pgfpathlineto{\pgfqpoint{3.446903in}{0.815623in}}%
\pgfpathlineto{\pgfqpoint{3.450687in}{0.801451in}}%
\pgfpathlineto{\pgfqpoint{3.452314in}{0.787931in}}%
\pgfpathlineto{\pgfqpoint{3.451623in}{0.775073in}}%
\pgfpathlineto{\pgfqpoint{3.448258in}{0.762884in}}%
\pgfpathlineto{\pgfqpoint{3.441761in}{0.751369in}}%
\pgfpathlineto{\pgfqpoint{3.432892in}{0.740549in}}%
\pgfpathlineto{\pgfqpoint{3.422059in}{0.730431in}}%
\pgfpathlineto{\pgfqpoint{3.409331in}{0.721017in}}%
\pgfpathlineto{\pgfqpoint{3.394751in}{0.712308in}}%
\pgfpathlineto{\pgfqpoint{3.369417in}{0.700569in}}%
\pgfpathlineto{\pgfqpoint{3.339834in}{0.690419in}}%
\pgfpathlineto{\pgfqpoint{3.305751in}{0.681853in}}%
\pgfpathlineto{\pgfqpoint{3.266807in}{0.674867in}}%
\pgfpathlineto{\pgfqpoint{3.222899in}{0.669479in}}%
\pgfpathlineto{\pgfqpoint{3.173644in}{0.665717in}}%
\pgfpathlineto{\pgfqpoint{3.118521in}{0.663616in}}%
\pgfpathlineto{\pgfqpoint{3.056973in}{0.663222in}}%
\pgfpathlineto{\pgfqpoint{2.988398in}{0.664592in}}%
\pgfpathlineto{\pgfqpoint{2.884927in}{0.669280in}}%
\pgfpathlineto{\pgfqpoint{2.766198in}{0.677410in}}%
\pgfpathlineto{\pgfqpoint{2.630634in}{0.689210in}}%
\pgfpathlineto{\pgfqpoint{2.477495in}{0.704951in}}%
\pgfpathlineto{\pgfqpoint{2.308105in}{0.724852in}}%
\pgfpathlineto{\pgfqpoint{2.172784in}{0.742580in}}%
\pgfpathlineto{\pgfqpoint{2.033825in}{0.762686in}}%
\pgfpathlineto{\pgfqpoint{1.895785in}{0.785030in}}%
\pgfpathlineto{\pgfqpoint{1.763343in}{0.809364in}}%
\pgfpathlineto{\pgfqpoint{1.680481in}{0.826522in}}%
\pgfpathlineto{\pgfqpoint{1.603872in}{0.844286in}}%
\pgfpathlineto{\pgfqpoint{1.534418in}{0.862511in}}%
\pgfpathlineto{\pgfqpoint{1.471900in}{0.881037in}}%
\pgfpathlineto{\pgfqpoint{1.416075in}{0.899719in}}%
\pgfpathlineto{\pgfqpoint{1.366692in}{0.918422in}}%
\pgfpathlineto{\pgfqpoint{1.323485in}{0.937029in}}%
\pgfpathlineto{\pgfqpoint{1.286179in}{0.955432in}}%
\pgfpathlineto{\pgfqpoint{1.254486in}{0.973539in}}%
\pgfpathlineto{\pgfqpoint{1.228108in}{0.991271in}}%
\pgfpathlineto{\pgfqpoint{1.206734in}{1.008561in}}%
\pgfpathlineto{\pgfqpoint{1.190043in}{1.025358in}}%
\pgfpathlineto{\pgfqpoint{1.177700in}{1.041620in}}%
\pgfpathlineto{\pgfqpoint{1.169240in}{1.057282in}}%
\pgfpathlineto{\pgfqpoint{1.163978in}{1.072306in}}%
\pgfpathlineto{\pgfqpoint{1.161331in}{1.086677in}}%
\pgfpathlineto{\pgfqpoint{1.160853in}{1.100382in}}%
\pgfpathlineto{\pgfqpoint{1.162242in}{1.113411in}}%
\pgfpathlineto{\pgfqpoint{1.165336in}{1.125757in}}%
\pgfpathlineto{\pgfqpoint{1.170113in}{1.137417in}}%
\pgfpathlineto{\pgfqpoint{1.176692in}{1.148391in}}%
\pgfpathlineto{\pgfqpoint{1.185336in}{1.158681in}}%
\pgfpathlineto{\pgfqpoint{1.196445in}{1.168294in}}%
\pgfpathlineto{\pgfqpoint{1.210414in}{1.177234in}}%
\pgfpathlineto{\pgfqpoint{1.226508in}{1.185477in}}%
\pgfpathlineto{\pgfqpoint{1.254229in}{1.196516in}}%
\pgfpathlineto{\pgfqpoint{1.286297in}{1.205962in}}%
\pgfpathlineto{\pgfqpoint{1.322872in}{1.213805in}}%
\pgfpathlineto{\pgfqpoint{1.364220in}{1.220037in}}%
\pgfpathlineto{\pgfqpoint{1.410715in}{1.224645in}}%
\pgfpathlineto{\pgfqpoint{1.462837in}{1.227615in}}%
\pgfpathlineto{\pgfqpoint{1.520750in}{1.228926in}}%
\pgfpathlineto{\pgfqpoint{1.585200in}{1.228516in}}%
\pgfpathlineto{\pgfqpoint{1.683101in}{1.225155in}}%
\pgfpathlineto{\pgfqpoint{1.796346in}{1.218397in}}%
\pgfpathlineto{\pgfqpoint{1.926049in}{1.208033in}}%
\pgfpathlineto{\pgfqpoint{2.072581in}{1.193843in}}%
\pgfpathlineto{\pgfqpoint{2.235566in}{1.175596in}}%
\pgfpathlineto{\pgfqpoint{2.413900in}{1.153028in}}%
\pgfpathlineto{\pgfqpoint{2.553587in}{1.133239in}}%
\pgfpathlineto{\pgfqpoint{2.692352in}{1.111215in}}%
\pgfpathlineto{\pgfqpoint{2.825508in}{1.087221in}}%
\pgfpathlineto{\pgfqpoint{2.909210in}{1.070286in}}%
\pgfpathlineto{\pgfqpoint{2.987722in}{1.052724in}}%
\pgfpathlineto{\pgfqpoint{3.060247in}{1.034654in}}%
\pgfpathlineto{\pgfqpoint{3.126150in}{1.016205in}}%
\pgfpathlineto{\pgfqpoint{3.184950in}{0.997519in}}%
\pgfpathlineto{\pgfqpoint{3.236326in}{0.978747in}}%
\pgfpathlineto{\pgfqpoint{3.280295in}{0.960044in}}%
\pgfpathlineto{\pgfqpoint{3.317699in}{0.941532in}}%
\pgfpathlineto{\pgfqpoint{3.349391in}{0.923298in}}%
\pgfpathlineto{\pgfqpoint{3.376052in}{0.905423in}}%
\pgfpathlineto{\pgfqpoint{3.398199in}{0.887978in}}%
\pgfpathlineto{\pgfqpoint{3.416188in}{0.871025in}}%
\pgfpathlineto{\pgfqpoint{3.430208in}{0.854616in}}%
\pgfpathlineto{\pgfqpoint{3.440301in}{0.838795in}}%
\pgfpathlineto{\pgfqpoint{3.447002in}{0.823597in}}%
\pgfpathlineto{\pgfqpoint{3.450862in}{0.809046in}}%
\pgfpathlineto{\pgfqpoint{3.452166in}{0.795157in}}%
\pgfpathlineto{\pgfqpoint{3.451133in}{0.781943in}}%
\pgfpathlineto{\pgfqpoint{3.447909in}{0.769413in}}%
\pgfpathlineto{\pgfqpoint{3.442571in}{0.757574in}}%
\pgfpathlineto{\pgfqpoint{3.435123in}{0.746428in}}%
\pgfpathlineto{\pgfqpoint{3.425537in}{0.735976in}}%
\pgfpathlineto{\pgfqpoint{3.413932in}{0.726219in}}%
\pgfpathlineto{\pgfqpoint{3.400397in}{0.717161in}}%
\pgfpathlineto{\pgfqpoint{3.384980in}{0.708803in}}%
\pgfpathlineto{\pgfqpoint{3.358364in}{0.697585in}}%
\pgfpathlineto{\pgfqpoint{3.327501in}{0.687953in}}%
\pgfpathlineto{\pgfqpoint{3.292191in}{0.679911in}}%
\pgfpathlineto{\pgfqpoint{3.252090in}{0.673461in}}%
\pgfpathlineto{\pgfqpoint{3.206714in}{0.668601in}}%
\pgfpathlineto{\pgfqpoint{3.155898in}{0.665356in}}%
\pgfpathlineto{\pgfqpoint{3.099160in}{0.663776in}}%
\pgfpathlineto{\pgfqpoint{3.035749in}{0.663911in}}%
\pgfpathlineto{\pgfqpoint{2.964972in}{0.665825in}}%
\pgfpathlineto{\pgfqpoint{2.858048in}{0.671272in}}%
\pgfpathlineto{\pgfqpoint{2.735540in}{0.680214in}}%
\pgfpathlineto{\pgfqpoint{2.596265in}{0.692884in}}%
\pgfpathlineto{\pgfqpoint{2.438970in}{0.709525in}}%
\pgfpathlineto{\pgfqpoint{2.265369in}{0.730350in}}%
\pgfpathlineto{\pgfqpoint{2.128880in}{0.748798in}}%
\pgfpathlineto{\pgfqpoint{1.990900in}{0.769604in}}%
\pgfpathlineto{\pgfqpoint{1.855201in}{0.792592in}}%
\pgfpathlineto{\pgfqpoint{1.725804in}{0.817481in}}%
\pgfpathlineto{\pgfqpoint{1.645135in}{0.834941in}}%
\pgfpathlineto{\pgfqpoint{1.570479in}{0.852933in}}%
\pgfpathlineto{\pgfqpoint{1.503162in}{0.871298in}}%
\pgfpathlineto{\pgfqpoint{1.443464in}{0.889894in}}%
\pgfpathlineto{\pgfqpoint{1.390879in}{0.908589in}}%
\pgfpathlineto{\pgfqpoint{1.344953in}{0.927258in}}%
\pgfpathlineto{\pgfqpoint{1.305256in}{0.945787in}}%
\pgfpathlineto{\pgfqpoint{1.271389in}{0.964075in}}%
\pgfpathlineto{\pgfqpoint{1.242979in}{0.982028in}}%
\pgfpathlineto{\pgfqpoint{1.219682in}{0.999567in}}%
\pgfpathlineto{\pgfqpoint{1.201180in}{1.016623in}}%
\pgfpathlineto{\pgfqpoint{1.186931in}{1.033136in}}%
\pgfpathlineto{\pgfqpoint{1.176270in}{1.049067in}}%
\pgfpathlineto{\pgfqpoint{1.168707in}{1.064387in}}%
\pgfpathlineto{\pgfqpoint{1.163868in}{1.079074in}}%
\pgfpathlineto{\pgfqpoint{1.161499in}{1.093108in}}%
\pgfpathlineto{\pgfqpoint{1.161467in}{1.106475in}}%
\pgfpathlineto{\pgfqpoint{1.163755in}{1.119166in}}%
\pgfpathlineto{\pgfqpoint{1.168468in}{1.131174in}}%
\pgfpathlineto{\pgfqpoint{1.175573in}{1.142496in}}%
\pgfpathlineto{\pgfqpoint{1.184734in}{1.153122in}}%
\pgfpathlineto{\pgfqpoint{1.195846in}{1.163050in}}%
\pgfpathlineto{\pgfqpoint{1.208837in}{1.172278in}}%
\pgfpathlineto{\pgfqpoint{1.223670in}{1.180803in}}%
\pgfpathlineto{\pgfqpoint{1.249374in}{1.192273in}}%
\pgfpathlineto{\pgfqpoint{1.279350in}{1.202162in}}%
\pgfpathlineto{\pgfqpoint{1.313896in}{1.210474in}}%
\pgfpathlineto{\pgfqpoint{1.353403in}{1.217215in}}%
\pgfpathlineto{\pgfqpoint{1.397926in}{1.222360in}}%
\pgfpathlineto{\pgfqpoint{1.447870in}{1.225879in}}%
\pgfpathlineto{\pgfqpoint{1.503756in}{1.227735in}}%
\pgfpathlineto{\pgfqpoint{1.566148in}{1.227879in}}%
\pgfpathlineto{\pgfqpoint{1.635648in}{1.226255in}}%
\pgfpathlineto{\pgfqpoint{1.740489in}{1.221217in}}%
\pgfpathlineto{\pgfqpoint{1.860750in}{1.212720in}}%
\pgfpathlineto{\pgfqpoint{1.997951in}{1.200532in}}%
\pgfpathlineto{\pgfqpoint{2.152682in}{1.184381in}}%
\pgfpathlineto{\pgfqpoint{2.323334in}{1.164061in}}%
\pgfpathlineto{\pgfqpoint{2.459211in}{1.146019in}}%
\pgfpathlineto{\pgfqpoint{2.598122in}{1.125616in}}%
\pgfpathlineto{\pgfqpoint{2.735489in}{1.103006in}}%
\pgfpathlineto{\pgfqpoint{2.866639in}{1.078451in}}%
\pgfpathlineto{\pgfqpoint{2.948276in}{1.061177in}}%
\pgfpathlineto{\pgfqpoint{3.023282in}{1.043322in}}%
\pgfpathlineto{\pgfqpoint{3.091230in}{1.025039in}}%
\pgfpathlineto{\pgfqpoint{3.152325in}{1.006487in}}%
\pgfpathlineto{\pgfqpoint{3.206792in}{0.987808in}}%
\pgfpathlineto{\pgfqpoint{3.254871in}{0.969134in}}%
\pgfpathlineto{\pgfqpoint{3.296821in}{0.950582in}}%
\pgfpathlineto{\pgfqpoint{3.332916in}{0.932253in}}%
\pgfpathlineto{\pgfqpoint{3.363448in}{0.914239in}}%
\pgfpathlineto{\pgfqpoint{3.388725in}{0.896615in}}%
\pgfpathlineto{\pgfqpoint{3.409072in}{0.879444in}}%
\pgfpathlineto{\pgfqpoint{3.424832in}{0.862774in}}%
\pgfpathlineto{\pgfqpoint{3.436364in}{0.846646in}}%
\pgfpathlineto{\pgfqpoint{3.444158in}{0.831128in}}%
\pgfpathlineto{\pgfqpoint{3.448876in}{0.816249in}}%
\pgfpathlineto{\pgfqpoint{3.451069in}{0.802026in}}%
\pgfpathlineto{\pgfqpoint{3.451147in}{0.788471in}}%
\pgfpathlineto{\pgfqpoint{3.449384in}{0.775593in}}%
\pgfpathlineto{\pgfqpoint{3.445915in}{0.763399in}}%
\pgfpathlineto{\pgfqpoint{3.440736in}{0.751891in}}%
\pgfpathlineto{\pgfqpoint{3.433705in}{0.741068in}}%
\pgfpathlineto{\pgfqpoint{3.424541in}{0.730929in}}%
\pgfpathlineto{\pgfqpoint{3.412824in}{0.721465in}}%
\pgfpathlineto{\pgfqpoint{3.398336in}{0.712676in}}%
\pgfpathlineto{\pgfqpoint{3.381856in}{0.704590in}}%
\pgfpathlineto{\pgfqpoint{3.353552in}{0.693783in}}%
\pgfpathlineto{\pgfqpoint{3.320881in}{0.684571in}}%
\pgfpathlineto{\pgfqpoint{3.283671in}{0.676960in}}%
\pgfpathlineto{\pgfqpoint{3.241640in}{0.670962in}}%
\pgfpathlineto{\pgfqpoint{3.194398in}{0.666589in}}%
\pgfpathlineto{\pgfqpoint{3.141470in}{0.663855in}}%
\pgfpathlineto{\pgfqpoint{3.082721in}{0.662786in}}%
\pgfpathlineto{\pgfqpoint{3.017273in}{0.663447in}}%
\pgfpathlineto{\pgfqpoint{2.917842in}{0.667161in}}%
\pgfpathlineto{\pgfqpoint{2.802915in}{0.674292in}}%
\pgfpathlineto{\pgfqpoint{2.671458in}{0.685051in}}%
\pgfpathlineto{\pgfqpoint{2.523182in}{0.699659in}}%
\pgfpathlineto{\pgfqpoint{2.358546in}{0.718348in}}%
\pgfpathlineto{\pgfqpoint{2.178866in}{0.741383in}}%
\pgfpathlineto{\pgfqpoint{2.039031in}{0.761499in}}%
\pgfpathlineto{\pgfqpoint{1.900826in}{0.783814in}}%
\pgfpathlineto{\pgfqpoint{1.768825in}{0.808058in}}%
\pgfpathlineto{\pgfqpoint{1.686166in}{0.825132in}}%
\pgfpathlineto{\pgfqpoint{1.608878in}{0.842809in}}%
\pgfpathlineto{\pgfqpoint{1.537722in}{0.860966in}}%
\pgfpathlineto{\pgfqpoint{1.473302in}{0.879473in}}%
\pgfpathlineto{\pgfqpoint{1.416063in}{0.898187in}}%
\pgfpathlineto{\pgfqpoint{1.366291in}{0.916954in}}%
\pgfpathlineto{\pgfqpoint{1.323801in}{0.935620in}}%
\pgfpathlineto{\pgfqpoint{1.287678in}{0.954077in}}%
\pgfpathlineto{\pgfqpoint{1.257109in}{0.972238in}}%
\pgfpathlineto{\pgfqpoint{1.231445in}{0.990024in}}%
\pgfpathlineto{\pgfqpoint{1.210203in}{1.007366in}}%
\pgfpathlineto{\pgfqpoint{1.193062in}{1.024205in}}%
\pgfpathlineto{\pgfqpoint{1.179867in}{1.040491in}}%
\pgfpathlineto{\pgfqpoint{1.170563in}{1.056180in}}%
\pgfpathlineto{\pgfqpoint{1.164493in}{1.071239in}}%
\pgfpathlineto{\pgfqpoint{1.161199in}{1.085649in}}%
\pgfpathlineto{\pgfqpoint{1.160409in}{1.099394in}}%
\pgfpathlineto{\pgfqpoint{1.161919in}{1.112462in}}%
\pgfpathlineto{\pgfqpoint{1.165598in}{1.124843in}}%
\pgfpathlineto{\pgfqpoint{1.171383in}{1.136534in}}%
\pgfpathlineto{\pgfqpoint{1.179282in}{1.147530in}}%
\pgfpathlineto{\pgfqpoint{1.189309in}{1.157834in}}%
\pgfpathlineto{\pgfqpoint{1.201329in}{1.167441in}}%
\pgfpathlineto{\pgfqpoint{1.215269in}{1.176349in}}%
\pgfpathlineto{\pgfqpoint{1.231084in}{1.184556in}}%
\pgfpathlineto{\pgfqpoint{1.258300in}{1.195548in}}%
\pgfpathlineto{\pgfqpoint{1.289784in}{1.204953in}}%
\pgfpathlineto{\pgfqpoint{1.325754in}{1.212768in}}%
\pgfpathlineto{\pgfqpoint{1.366573in}{1.218991in}}%
\pgfpathlineto{\pgfqpoint{1.412725in}{1.223623in}}%
\pgfpathlineto{\pgfqpoint{1.464337in}{1.226632in}}%
\pgfpathlineto{\pgfqpoint{1.521978in}{1.227972in}}%
\pgfpathlineto{\pgfqpoint{1.586393in}{1.227587in}}%
\pgfpathlineto{\pgfqpoint{1.684001in}{1.224279in}}%
\pgfpathlineto{\pgfqpoint{1.796344in}{1.217609in}}%
\pgfpathlineto{\pgfqpoint{1.924714in}{1.207365in}}%
\pgfpathlineto{\pgfqpoint{2.070219in}{1.193302in}}%
\pgfpathlineto{\pgfqpoint{2.236102in}{1.175106in}}%
\pgfpathlineto{\pgfqpoint{2.371134in}{1.158633in}}%
\pgfpathlineto{\pgfqpoint{2.509679in}{1.139780in}}%
\pgfpathlineto{\pgfqpoint{2.647313in}{1.118666in}}%
\pgfpathlineto{\pgfqpoint{2.780178in}{1.095476in}}%
\pgfpathlineto{\pgfqpoint{2.864456in}{1.078990in}}%
\pgfpathlineto{\pgfqpoint{2.944304in}{1.061791in}}%
\pgfpathlineto{\pgfqpoint{3.018997in}{1.043987in}}%
\pgfpathlineto{\pgfqpoint{3.087924in}{1.025701in}}%
\pgfpathlineto{\pgfqpoint{3.150584in}{1.007071in}}%
\pgfpathlineto{\pgfqpoint{3.206588in}{0.988248in}}%
\pgfpathlineto{\pgfqpoint{3.255658in}{0.969398in}}%
\pgfpathlineto{\pgfqpoint{3.297630in}{0.950698in}}%
\pgfpathlineto{\pgfqpoint{3.332600in}{0.932322in}}%
\pgfpathlineto{\pgfqpoint{3.361772in}{0.914290in}}%
\pgfpathlineto{\pgfqpoint{3.385905in}{0.896658in}}%
\pgfpathlineto{\pgfqpoint{3.405532in}{0.879494in}}%
\pgfpathlineto{\pgfqpoint{3.421098in}{0.862851in}}%
\pgfpathlineto{\pgfqpoint{3.432960in}{0.846776in}}%
\pgfpathlineto{\pgfqpoint{3.441477in}{0.831304in}}%
\pgfpathlineto{\pgfqpoint{3.447005in}{0.816460in}}%
\pgfpathlineto{\pgfqpoint{3.449812in}{0.802266in}}%
\pgfpathlineto{\pgfqpoint{3.450108in}{0.788737in}}%
\pgfpathlineto{\pgfqpoint{3.448049in}{0.775884in}}%
\pgfpathlineto{\pgfqpoint{3.443709in}{0.763715in}}%
\pgfpathlineto{\pgfqpoint{3.437152in}{0.752232in}}%
\pgfpathlineto{\pgfqpoint{3.428545in}{0.741442in}}%
\pgfpathlineto{\pgfqpoint{3.418018in}{0.731348in}}%
\pgfpathlineto{\pgfqpoint{3.405659in}{0.721953in}}%
\pgfpathlineto{\pgfqpoint{3.391512in}{0.713260in}}%
\pgfpathlineto{\pgfqpoint{3.366932in}{0.701535in}}%
\pgfpathlineto{\pgfqpoint{3.338145in}{0.691385in}}%
\pgfpathlineto{\pgfqpoint{3.304752in}{0.682799in}}%
\pgfpathlineto{\pgfqpoint{3.266246in}{0.675768in}}%
\pgfpathlineto{\pgfqpoint{3.222723in}{0.670324in}}%
\pgfpathlineto{\pgfqpoint{3.173857in}{0.666501in}}%
\pgfpathlineto{\pgfqpoint{3.119156in}{0.664333in}}%
\pgfpathlineto{\pgfqpoint{3.058078in}{0.663865in}}%
\pgfpathlineto{\pgfqpoint{2.990029in}{0.665153in}}%
\pgfpathlineto{\pgfqpoint{2.887334in}{0.669717in}}%
\pgfpathlineto{\pgfqpoint{2.769430in}{0.677705in}}%
\pgfpathlineto{\pgfqpoint{2.634762in}{0.689347in}}%
\pgfpathlineto{\pgfqpoint{2.482567in}{0.704914in}}%
\pgfpathlineto{\pgfqpoint{2.314092in}{0.724622in}}%
\pgfpathlineto{\pgfqpoint{2.179278in}{0.742201in}}%
\pgfpathlineto{\pgfqpoint{2.040772in}{0.762154in}}%
\pgfpathlineto{\pgfqpoint{1.902774in}{0.784351in}}%
\pgfpathlineto{\pgfqpoint{1.770066in}{0.808553in}}%
\pgfpathlineto{\pgfqpoint{1.687148in}{0.825653in}}%
\pgfpathlineto{\pgfqpoint{1.609992in}{0.843373in}}%
\pgfpathlineto{\pgfqpoint{1.539221in}{0.861549in}}%
\pgfpathlineto{\pgfqpoint{1.475256in}{0.880028in}}%
\pgfpathlineto{\pgfqpoint{1.418322in}{0.898669in}}%
\pgfpathlineto{\pgfqpoint{1.368447in}{0.917341in}}%
\pgfpathlineto{\pgfqpoint{1.325458in}{0.935928in}}%
\pgfpathlineto{\pgfqpoint{1.288988in}{0.954323in}}%
\pgfpathlineto{\pgfqpoint{1.258469in}{0.972431in}}%
\pgfpathlineto{\pgfqpoint{1.233135in}{0.990169in}}%
\pgfpathlineto{\pgfqpoint{1.212569in}{1.007452in}}%
\pgfpathlineto{\pgfqpoint{1.196474in}{1.024216in}}%
\pgfpathlineto{\pgfqpoint{1.184005in}{1.040425in}}%
\pgfpathlineto{\pgfqpoint{1.174513in}{1.056051in}}%
\pgfpathlineto{\pgfqpoint{1.167548in}{1.071069in}}%
\pgfpathlineto{\pgfqpoint{1.162859in}{1.085458in}}%
\pgfpathlineto{\pgfqpoint{1.160395in}{1.099200in}}%
\pgfpathlineto{\pgfqpoint{1.160303in}{1.112284in}}%
\pgfpathlineto{\pgfqpoint{1.162929in}{1.124701in}}%
\pgfpathlineto{\pgfqpoint{1.168718in}{1.136446in}}%
\pgfpathlineto{\pgfqpoint{1.176890in}{1.147497in}}%
\pgfpathlineto{\pgfqpoint{1.187044in}{1.157846in}}%
\pgfpathlineto{\pgfqpoint{1.199102in}{1.167493in}}%
\pgfpathlineto{\pgfqpoint{1.213016in}{1.176434in}}%
\pgfpathlineto{\pgfqpoint{1.228769in}{1.184670in}}%
\pgfpathlineto{\pgfqpoint{1.255893in}{1.195702in}}%
\pgfpathlineto{\pgfqpoint{1.287388in}{1.205149in}}%
\pgfpathlineto{\pgfqpoint{1.323602in}{1.213019in}}%
\pgfpathlineto{\pgfqpoint{1.364692in}{1.219304in}}%
\pgfpathlineto{\pgfqpoint{1.410893in}{1.223978in}}%
\pgfpathlineto{\pgfqpoint{1.462671in}{1.227011in}}%
\pgfpathlineto{\pgfqpoint{1.520547in}{1.228363in}}%
\pgfpathlineto{\pgfqpoint{1.585089in}{1.227984in}}%
\pgfpathlineto{\pgfqpoint{1.682594in}{1.224680in}}%
\pgfpathlineto{\pgfqpoint{1.794663in}{1.218020in}}%
\pgfpathlineto{\pgfqpoint{1.922979in}{1.207797in}}%
\pgfpathlineto{\pgfqpoint{2.068733in}{1.193743in}}%
\pgfpathlineto{\pgfqpoint{2.231571in}{1.175615in}}%
\pgfpathlineto{\pgfqpoint{2.408680in}{1.153245in}}%
\pgfpathlineto{\pgfqpoint{2.547079in}{1.133674in}}%
\pgfpathlineto{\pgfqpoint{2.685706in}{1.111825in}}%
\pgfpathlineto{\pgfqpoint{2.819902in}{1.087919in}}%
\pgfpathlineto{\pgfqpoint{2.904361in}{1.070991in}}%
\pgfpathlineto{\pgfqpoint{2.983168in}{1.053396in}}%
\pgfpathlineto{\pgfqpoint{3.055667in}{1.035298in}}%
\pgfpathlineto{\pgfqpoint{3.121429in}{1.016857in}}%
\pgfpathlineto{\pgfqpoint{3.180212in}{0.998217in}}%
\pgfpathlineto{\pgfqpoint{3.231954in}{0.979515in}}%
\pgfpathlineto{\pgfqpoint{3.276773in}{0.960871in}}%
\pgfpathlineto{\pgfqpoint{3.314971in}{0.942396in}}%
\pgfpathlineto{\pgfqpoint{3.347031in}{0.924186in}}%
\pgfpathlineto{\pgfqpoint{3.373619in}{0.906328in}}%
\pgfpathlineto{\pgfqpoint{3.395488in}{0.888897in}}%
\pgfpathlineto{\pgfqpoint{3.412712in}{0.871974in}}%
\pgfpathlineto{\pgfqpoint{3.425987in}{0.855600in}}%
\pgfpathlineto{\pgfqpoint{3.436056in}{0.839806in}}%
\pgfpathlineto{\pgfqpoint{3.443464in}{0.824617in}}%
\pgfpathlineto{\pgfqpoint{3.448563in}{0.810054in}}%
\pgfpathlineto{\pgfqpoint{3.451511in}{0.796133in}}%
\pgfpathlineto{\pgfqpoint{3.452270in}{0.782870in}}%
\pgfpathlineto{\pgfqpoint{3.450609in}{0.770272in}}%
\pgfpathlineto{\pgfqpoint{3.446101in}{0.758345in}}%
\pgfpathlineto{\pgfqpoint{3.438555in}{0.747096in}}%
\pgfpathlineto{\pgfqpoint{3.428917in}{0.736548in}}%
\pgfpathlineto{\pgfqpoint{3.417343in}{0.726702in}}%
\pgfpathlineto{\pgfqpoint{3.403894in}{0.717561in}}%
\pgfpathlineto{\pgfqpoint{3.388598in}{0.709126in}}%
\pgfpathlineto{\pgfqpoint{3.362180in}{0.697798in}}%
\pgfpathlineto{\pgfqpoint{3.331464in}{0.688057in}}%
\pgfpathlineto{\pgfqpoint{3.296164in}{0.679900in}}%
\pgfpathlineto{\pgfqpoint{3.255945in}{0.673324in}}%
\pgfpathlineto{\pgfqpoint{3.210705in}{0.668353in}}%
\pgfpathlineto{\pgfqpoint{3.159989in}{0.665016in}}%
\pgfpathlineto{\pgfqpoint{3.103264in}{0.663351in}}%
\pgfpathlineto{\pgfqpoint{3.039960in}{0.663407in}}%
\pgfpathlineto{\pgfqpoint{2.969469in}{0.665243in}}%
\pgfpathlineto{\pgfqpoint{2.863171in}{0.670581in}}%
\pgfpathlineto{\pgfqpoint{2.741291in}{0.679403in}}%
\pgfpathlineto{\pgfqpoint{2.602344in}{0.691944in}}%
\pgfpathlineto{\pgfqpoint{2.445868in}{0.708477in}}%
\pgfpathlineto{\pgfqpoint{2.273721in}{0.729200in}}%
\pgfpathlineto{\pgfqpoint{2.137076in}{0.747550in}}%
\pgfpathlineto{\pgfqpoint{1.997897in}{0.768251in}}%
\pgfpathlineto{\pgfqpoint{1.860858in}{0.791137in}}%
\pgfpathlineto{\pgfqpoint{1.730637in}{0.815931in}}%
\pgfpathlineto{\pgfqpoint{1.649922in}{0.833338in}}%
\pgfpathlineto{\pgfqpoint{1.576033in}{0.851296in}}%
\pgfpathlineto{\pgfqpoint{1.509295in}{0.869652in}}%
\pgfpathlineto{\pgfqpoint{1.449436in}{0.888251in}}%
\pgfpathlineto{\pgfqpoint{1.396178in}{0.906954in}}%
\pgfpathlineto{\pgfqpoint{1.349243in}{0.925634in}}%
\pgfpathlineto{\pgfqpoint{1.308349in}{0.944177in}}%
\pgfpathlineto{\pgfqpoint{1.273212in}{0.962481in}}%
\pgfpathlineto{\pgfqpoint{1.243547in}{0.980458in}}%
\pgfpathlineto{\pgfqpoint{1.219064in}{0.998034in}}%
\pgfpathlineto{\pgfqpoint{1.199473in}{1.015147in}}%
\pgfpathlineto{\pgfqpoint{1.184480in}{1.031749in}}%
\pgfpathlineto{\pgfqpoint{1.173753in}{1.047791in}}%
\pgfpathlineto{\pgfqpoint{1.166657in}{1.063210in}}%
\pgfpathlineto{\pgfqpoint{1.162512in}{1.077985in}}%
\pgfpathlineto{\pgfqpoint{1.160781in}{1.092101in}}%
\pgfpathlineto{\pgfqpoint{1.161070in}{1.105546in}}%
\pgfpathlineto{\pgfqpoint{1.163128in}{1.118312in}}%
\pgfpathlineto{\pgfqpoint{1.166848in}{1.130393in}}%
\pgfpathlineto{\pgfqpoint{1.172265in}{1.141788in}}%
\pgfpathlineto{\pgfqpoint{1.179556in}{1.152498in}}%
\pgfpathlineto{\pgfqpoint{1.189043in}{1.162526in}}%
\pgfpathlineto{\pgfqpoint{1.201190in}{1.171881in}}%
\pgfpathlineto{\pgfqpoint{1.216093in}{1.180558in}}%
\pgfpathlineto{\pgfqpoint{1.232941in}{1.188530in}}%
\pgfpathlineto{\pgfqpoint{1.261797in}{1.199163in}}%
\pgfpathlineto{\pgfqpoint{1.295031in}{1.208199in}}%
\pgfpathlineto{\pgfqpoint{1.332829in}{1.215631in}}%
\pgfpathlineto{\pgfqpoint{1.375486in}{1.221447in}}%
\pgfpathlineto{\pgfqpoint{1.423402in}{1.225634in}}%
\pgfpathlineto{\pgfqpoint{1.477042in}{1.228179in}}%
\pgfpathlineto{\pgfqpoint{1.536545in}{1.229054in}}%
\pgfpathlineto{\pgfqpoint{1.602873in}{1.228187in}}%
\pgfpathlineto{\pgfqpoint{1.703638in}{1.224183in}}%
\pgfpathlineto{\pgfqpoint{1.820036in}{1.216742in}}%
\pgfpathlineto{\pgfqpoint{1.953039in}{1.205651in}}%
\pgfpathlineto{\pgfqpoint{2.102859in}{1.190689in}}%
\pgfpathlineto{\pgfqpoint{2.268946in}{1.171624in}}%
\pgfpathlineto{\pgfqpoint{2.449738in}{1.148195in}}%
\pgfpathlineto{\pgfqpoint{2.589716in}{1.127801in}}%
\pgfpathlineto{\pgfqpoint{2.727505in}{1.105234in}}%
\pgfpathlineto{\pgfqpoint{2.858603in}{1.080774in}}%
\pgfpathlineto{\pgfqpoint{2.940426in}{1.063579in}}%
\pgfpathlineto{\pgfqpoint{3.016724in}{1.045802in}}%
\pgfpathlineto{\pgfqpoint{3.086761in}{1.027566in}}%
\pgfpathlineto{\pgfqpoint{3.149963in}{1.009006in}}%
\pgfpathlineto{\pgfqpoint{3.205916in}{0.990265in}}%
\pgfpathlineto{\pgfqpoint{3.254370in}{0.971498in}}%
\pgfpathlineto{\pgfqpoint{3.295667in}{0.952854in}}%
\pgfpathlineto{\pgfqpoint{3.330744in}{0.934437in}}%
\pgfpathlineto{\pgfqpoint{3.360393in}{0.916331in}}%
\pgfpathlineto{\pgfqpoint{3.385242in}{0.898613in}}%
\pgfpathlineto{\pgfqpoint{3.405749in}{0.881349in}}%
\pgfpathlineto{\pgfqpoint{3.422205in}{0.864599in}}%
\pgfpathlineto{\pgfqpoint{3.434735in}{0.848410in}}%
\pgfpathlineto{\pgfqpoint{3.443417in}{0.832824in}}%
\pgfpathlineto{\pgfqpoint{3.448965in}{0.817873in}}%
\pgfpathlineto{\pgfqpoint{3.451788in}{0.803574in}}%
\pgfpathlineto{\pgfqpoint{3.452149in}{0.789944in}}%
\pgfpathlineto{\pgfqpoint{3.450238in}{0.776992in}}%
\pgfpathlineto{\pgfqpoint{3.446180in}{0.764728in}}%
\pgfpathlineto{\pgfqpoint{3.440025in}{0.753157in}}%
\pgfpathlineto{\pgfqpoint{3.431757in}{0.742279in}}%
\pgfpathlineto{\pgfqpoint{3.421373in}{0.732096in}}%
\pgfpathlineto{\pgfqpoint{3.409010in}{0.722609in}}%
\pgfpathlineto{\pgfqpoint{3.394734in}{0.713821in}}%
\pgfpathlineto{\pgfqpoint{3.378585in}{0.705735in}}%
\pgfpathlineto{\pgfqpoint{3.350863in}{0.694925in}}%
\pgfpathlineto{\pgfqpoint{3.318853in}{0.685703in}}%
\pgfpathlineto{\pgfqpoint{3.282328in}{0.678072in}}%
\pgfpathlineto{\pgfqpoint{3.240911in}{0.672033in}}%
\pgfpathlineto{\pgfqpoint{3.194124in}{0.667589in}}%
\pgfpathlineto{\pgfqpoint{3.141845in}{0.664771in}}%
\pgfpathlineto{\pgfqpoint{3.083461in}{0.663628in}}%
\pgfpathlineto{\pgfqpoint{3.018218in}{0.664218in}}%
\pgfpathlineto{\pgfqpoint{2.919374in}{0.667811in}}%
\pgfpathlineto{\pgfqpoint{2.805655in}{0.674785in}}%
\pgfpathlineto{\pgfqpoint{2.675806in}{0.685352in}}%
\pgfpathlineto{\pgfqpoint{2.528768in}{0.699762in}}%
\pgfpathlineto{\pgfqpoint{2.361396in}{0.718354in}}%
\pgfpathlineto{\pgfqpoint{2.225716in}{0.735143in}}%
\pgfpathlineto{\pgfqpoint{2.086975in}{0.754299in}}%
\pgfpathlineto{\pgfqpoint{1.949610in}{0.775684in}}%
\pgfpathlineto{\pgfqpoint{1.817465in}{0.799094in}}%
\pgfpathlineto{\pgfqpoint{1.733898in}{0.815696in}}%
\pgfpathlineto{\pgfqpoint{1.654927in}{0.832983in}}%
\pgfpathlineto{\pgfqpoint{1.581253in}{0.850847in}}%
\pgfpathlineto{\pgfqpoint{1.513460in}{0.869163in}}%
\pgfpathlineto{\pgfqpoint{1.452015in}{0.887797in}}%
\pgfpathlineto{\pgfqpoint{1.397269in}{0.906600in}}%
\pgfpathlineto{\pgfqpoint{1.349456in}{0.925411in}}%
\pgfpathlineto{\pgfqpoint{1.308692in}{0.944053in}}%
\pgfpathlineto{\pgfqpoint{1.274752in}{0.962370in}}%
\pgfpathlineto{\pgfqpoint{1.246429in}{0.980341in}}%
\pgfpathlineto{\pgfqpoint{1.223078in}{0.997899in}}%
\pgfpathlineto{\pgfqpoint{1.204192in}{1.014978in}}%
\pgfpathlineto{\pgfqpoint{1.189326in}{1.031526in}}%
\pgfpathlineto{\pgfqpoint{1.178063in}{1.047500in}}%
\pgfpathlineto{\pgfqpoint{1.170041in}{1.062868in}}%
\pgfpathlineto{\pgfqpoint{1.164958in}{1.077603in}}%
\pgfpathlineto{\pgfqpoint{1.162576in}{1.091686in}}%
\pgfpathlineto{\pgfqpoint{1.162802in}{1.105102in}}%
\pgfpathlineto{\pgfqpoint{1.165456in}{1.117842in}}%
\pgfpathlineto{\pgfqpoint{1.170284in}{1.129897in}}%
\pgfpathlineto{\pgfqpoint{1.177093in}{1.141260in}}%
\pgfpathlineto{\pgfqpoint{1.185754in}{1.151928in}}%
\pgfpathlineto{\pgfqpoint{1.196196in}{1.161898in}}%
\pgfpathlineto{\pgfqpoint{1.208413in}{1.171170in}}%
\pgfpathlineto{\pgfqpoint{1.222459in}{1.179745in}}%
\pgfpathlineto{\pgfqpoint{1.238449in}{1.187627in}}%
\pgfpathlineto{\pgfqpoint{1.256561in}{1.194823in}}%
\pgfpathlineto{\pgfqpoint{1.287916in}{1.204331in}}%
\pgfpathlineto{\pgfqpoint{1.323889in}{1.212265in}}%
\pgfpathlineto{\pgfqpoint{1.364665in}{1.218606in}}%
\pgfpathlineto{\pgfqpoint{1.410564in}{1.223333in}}%
\pgfpathlineto{\pgfqpoint{1.461988in}{1.226417in}}%
\pgfpathlineto{\pgfqpoint{1.519428in}{1.227821in}}%
\pgfpathlineto{\pgfqpoint{1.583459in}{1.227506in}}%
\pgfpathlineto{\pgfqpoint{1.680198in}{1.224324in}}%
\pgfpathlineto{\pgfqpoint{1.791318in}{1.217824in}}%
\pgfpathlineto{\pgfqpoint{1.919025in}{1.207739in}}%
\pgfpathlineto{\pgfqpoint{2.064233in}{1.193822in}}%
\pgfpathlineto{\pgfqpoint{2.226166in}{1.175862in}}%
\pgfpathlineto{\pgfqpoint{2.402375in}{1.153682in}}%
\pgfpathlineto{\pgfqpoint{2.541003in}{1.134202in}}%
\pgfpathlineto{\pgfqpoint{2.679747in}{1.112442in}}%
\pgfpathlineto{\pgfqpoint{2.813609in}{1.088663in}}%
\pgfpathlineto{\pgfqpoint{2.898036in}{1.071844in}}%
\pgfpathlineto{\pgfqpoint{2.977374in}{1.054379in}}%
\pgfpathlineto{\pgfqpoint{3.050743in}{1.036389in}}%
\pgfpathlineto{\pgfqpoint{3.117423in}{1.018008in}}%
\pgfpathlineto{\pgfqpoint{3.176854in}{0.999381in}}%
\pgfpathlineto{\pgfqpoint{3.228685in}{0.980666in}}%
\pgfpathlineto{\pgfqpoint{3.273348in}{0.962009in}}%
\pgfpathlineto{\pgfqpoint{3.311691in}{0.943515in}}%
\pgfpathlineto{\pgfqpoint{3.344422in}{0.925278in}}%
\pgfpathlineto{\pgfqpoint{3.372102in}{0.907382in}}%
\pgfpathlineto{\pgfqpoint{3.395144in}{0.889902in}}%
\pgfpathlineto{\pgfqpoint{3.413813in}{0.872904in}}%
\pgfpathlineto{\pgfqpoint{3.428223in}{0.856443in}}%
\pgfpathlineto{\pgfqpoint{3.438551in}{0.840568in}}%
\pgfpathlineto{\pgfqpoint{3.445612in}{0.825312in}}%
\pgfpathlineto{\pgfqpoint{3.449830in}{0.810698in}}%
\pgfpathlineto{\pgfqpoint{3.451507in}{0.796742in}}%
\pgfpathlineto{\pgfqpoint{3.450858in}{0.783459in}}%
\pgfpathlineto{\pgfqpoint{3.448019in}{0.770857in}}%
\pgfpathlineto{\pgfqpoint{3.443040in}{0.758943in}}%
\pgfpathlineto{\pgfqpoint{3.435891in}{0.747720in}}%
\pgfpathlineto{\pgfqpoint{3.426537in}{0.737189in}}%
\pgfpathlineto{\pgfqpoint{3.415170in}{0.727353in}}%
\pgfpathlineto{\pgfqpoint{3.401876in}{0.718215in}}%
\pgfpathlineto{\pgfqpoint{3.386703in}{0.709778in}}%
\pgfpathlineto{\pgfqpoint{3.360458in}{0.698441in}}%
\pgfpathlineto{\pgfqpoint{3.329968in}{0.688688in}}%
\pgfpathlineto{\pgfqpoint{3.295029in}{0.680523in}}%
\pgfpathlineto{\pgfqpoint{3.255290in}{0.673946in}}%
\pgfpathlineto{\pgfqpoint{3.210298in}{0.668956in}}%
\pgfpathlineto{\pgfqpoint{3.159939in}{0.665584in}}%
\pgfpathlineto{\pgfqpoint{3.103649in}{0.663874in}}%
\pgfpathlineto{\pgfqpoint{3.040739in}{0.663876in}}%
\pgfpathlineto{\pgfqpoint{2.970547in}{0.665651in}}%
\pgfpathlineto{\pgfqpoint{2.864548in}{0.670902in}}%
\pgfpathlineto{\pgfqpoint{2.743076in}{0.679633in}}%
\pgfpathlineto{\pgfqpoint{2.604814in}{0.692079in}}%
\pgfpathlineto{\pgfqpoint{2.448634in}{0.708470in}}%
\pgfpathlineto{\pgfqpoint{2.276637in}{0.729056in}}%
\pgfpathlineto{\pgfqpoint{2.140664in}{0.747331in}}%
\pgfpathlineto{\pgfqpoint{2.002269in}{0.767967in}}%
\pgfpathlineto{\pgfqpoint{1.865401in}{0.790789in}}%
\pgfpathlineto{\pgfqpoint{1.734699in}{0.815520in}}%
\pgfpathlineto{\pgfqpoint{1.653595in}{0.832882in}}%
\pgfpathlineto{\pgfqpoint{1.579205in}{0.850797in}}%
\pgfpathlineto{\pgfqpoint{1.511922in}{0.869118in}}%
\pgfpathlineto{\pgfqpoint{1.451568in}{0.887692in}}%
\pgfpathlineto{\pgfqpoint{1.397935in}{0.906377in}}%
\pgfpathlineto{\pgfqpoint{1.350786in}{0.925047in}}%
\pgfpathlineto{\pgfqpoint{1.309856in}{0.943585in}}%
\pgfpathlineto{\pgfqpoint{1.274852in}{0.961890in}}%
\pgfpathlineto{\pgfqpoint{1.245449in}{0.979871in}}%
\pgfpathlineto{\pgfqpoint{1.221298in}{0.997453in}}%
\pgfpathlineto{\pgfqpoint{1.202017in}{1.014570in}}%
\pgfpathlineto{\pgfqpoint{1.187211in}{1.031171in}}%
\pgfpathlineto{\pgfqpoint{1.176571in}{1.047188in}}%
\pgfpathlineto{\pgfqpoint{1.169379in}{1.062587in}}%
\pgfpathlineto{\pgfqpoint{1.164938in}{1.077350in}}%
\pgfpathlineto{\pgfqpoint{1.162720in}{1.091462in}}%
\pgfpathlineto{\pgfqpoint{1.162362in}{1.104913in}}%
\pgfpathlineto{\pgfqpoint{1.163673in}{1.117692in}}%
\pgfpathlineto{\pgfqpoint{1.166628in}{1.129796in}}%
\pgfpathlineto{\pgfqpoint{1.171371in}{1.141223in}}%
\pgfpathlineto{\pgfqpoint{1.178215in}{1.151973in}}%
\pgfpathlineto{\pgfqpoint{1.187640in}{1.162052in}}%
\pgfpathlineto{\pgfqpoint{1.200202in}{1.171465in}}%
\pgfpathlineto{\pgfqpoint{1.215075in}{1.180182in}}%
\pgfpathlineto{\pgfqpoint{1.231845in}{1.188193in}}%
\pgfpathlineto{\pgfqpoint{1.260562in}{1.198880in}}%
\pgfpathlineto{\pgfqpoint{1.293634in}{1.207969in}}%
\pgfpathlineto{\pgfqpoint{1.331259in}{1.215453in}}%
\pgfpathlineto{\pgfqpoint{1.373746in}{1.221324in}}%
\pgfpathlineto{\pgfqpoint{1.421514in}{1.225571in}}%
\pgfpathlineto{\pgfqpoint{1.474878in}{1.228182in}}%
\pgfpathlineto{\pgfqpoint{1.534216in}{1.229111in}}%
\pgfpathlineto{\pgfqpoint{1.600496in}{1.228295in}}%
\pgfpathlineto{\pgfqpoint{1.701086in}{1.224366in}}%
\pgfpathlineto{\pgfqpoint{1.817021in}{1.217015in}}%
\pgfpathlineto{\pgfqpoint{1.949337in}{1.206030in}}%
\pgfpathlineto{\pgfqpoint{2.098573in}{1.191176in}}%
\pgfpathlineto{\pgfqpoint{2.264917in}{1.172189in}}%
\pgfpathlineto{\pgfqpoint{2.446365in}{1.148830in}}%
\pgfpathlineto{\pgfqpoint{2.585845in}{1.128543in}}%
\pgfpathlineto{\pgfqpoint{2.723003in}{1.106079in}}%
\pgfpathlineto{\pgfqpoint{2.853751in}{1.081693in}}%
\pgfpathlineto{\pgfqpoint{2.935594in}{1.064524in}}%
\pgfpathlineto{\pgfqpoint{3.012114in}{1.046758in}}%
\pgfpathlineto{\pgfqpoint{3.082521in}{1.028522in}}%
\pgfpathlineto{\pgfqpoint{3.146141in}{1.009957in}}%
\pgfpathlineto{\pgfqpoint{3.202412in}{0.991219in}}%
\pgfpathlineto{\pgfqpoint{3.251079in}{0.972469in}}%
\pgfpathlineto{\pgfqpoint{3.292898in}{0.953829in}}%
\pgfpathlineto{\pgfqpoint{3.328573in}{0.935400in}}%
\pgfpathlineto{\pgfqpoint{3.358690in}{0.917278in}}%
\pgfpathlineto{\pgfqpoint{3.383750in}{0.899544in}}%
\pgfpathlineto{\pgfqpoint{3.404164in}{0.882268in}}%
\pgfpathlineto{\pgfqpoint{3.420262in}{0.865510in}}%
\pgfpathlineto{\pgfqpoint{3.432499in}{0.849318in}}%
\pgfpathlineto{\pgfqpoint{3.441394in}{0.833727in}}%
\pgfpathlineto{\pgfqpoint{3.447304in}{0.818763in}}%
\pgfpathlineto{\pgfqpoint{3.450508in}{0.804446in}}%
\pgfpathlineto{\pgfqpoint{3.451206in}{0.790794in}}%
\pgfpathlineto{\pgfqpoint{3.449517in}{0.777819in}}%
\pgfpathlineto{\pgfqpoint{3.445490in}{0.765527in}}%
\pgfpathlineto{\pgfqpoint{3.439236in}{0.753923in}}%
\pgfpathlineto{\pgfqpoint{3.430921in}{0.743012in}}%
\pgfpathlineto{\pgfqpoint{3.420669in}{0.732799in}}%
\pgfpathlineto{\pgfqpoint{3.408566in}{0.723286in}}%
\pgfpathlineto{\pgfqpoint{3.394654in}{0.714475in}}%
\pgfpathlineto{\pgfqpoint{3.378936in}{0.706366in}}%
\pgfpathlineto{\pgfqpoint{3.351878in}{0.695517in}}%
\pgfpathlineto{\pgfqpoint{3.320356in}{0.686237in}}%
\pgfpathlineto{\pgfqpoint{3.283905in}{0.678516in}}%
\pgfpathlineto{\pgfqpoint{3.242533in}{0.672377in}}%
\pgfpathlineto{\pgfqpoint{3.195996in}{0.667846in}}%
\pgfpathlineto{\pgfqpoint{3.143843in}{0.664957in}}%
\pgfpathlineto{\pgfqpoint{3.085561in}{0.663749in}}%
\pgfpathlineto{\pgfqpoint{3.020583in}{0.664274in}}%
\pgfpathlineto{\pgfqpoint{2.922437in}{0.667773in}}%
\pgfpathlineto{\pgfqpoint{2.809622in}{0.674630in}}%
\pgfpathlineto{\pgfqpoint{2.680482in}{0.685060in}}%
\pgfpathlineto{\pgfqpoint{2.533885in}{0.699334in}}%
\pgfpathlineto{\pgfqpoint{2.370364in}{0.717684in}}%
\pgfpathlineto{\pgfqpoint{2.192814in}{0.740279in}}%
\pgfpathlineto{\pgfqpoint{2.054472in}{0.760008in}}%
\pgfpathlineto{\pgfqpoint{1.916172in}{0.781999in}}%
\pgfpathlineto{\pgfqpoint{1.782681in}{0.806024in}}%
\pgfpathlineto{\pgfqpoint{1.698998in}{0.823030in}}%
\pgfpathlineto{\pgfqpoint{1.620919in}{0.840676in}}%
\pgfpathlineto{\pgfqpoint{1.549125in}{0.858798in}}%
\pgfpathlineto{\pgfqpoint{1.484090in}{0.877244in}}%
\pgfpathlineto{\pgfqpoint{1.426082in}{0.895870in}}%
\pgfpathlineto{\pgfqpoint{1.375163in}{0.914548in}}%
\pgfpathlineto{\pgfqpoint{1.331191in}{0.933158in}}%
\pgfpathlineto{\pgfqpoint{1.293815in}{0.951591in}}%
\pgfpathlineto{\pgfqpoint{1.262481in}{0.969752in}}%
\pgfpathlineto{\pgfqpoint{1.236426in}{0.987555in}}%
\pgfpathlineto{\pgfqpoint{1.215201in}{1.004913in}}%
\pgfpathlineto{\pgfqpoint{1.198521in}{1.021758in}}%
\pgfpathlineto{\pgfqpoint{1.185529in}{1.038055in}}%
\pgfpathlineto{\pgfqpoint{1.175566in}{1.053773in}}%
\pgfpathlineto{\pgfqpoint{1.168174in}{1.068887in}}%
\pgfpathlineto{\pgfqpoint{1.163094in}{1.083374in}}%
\pgfpathlineto{\pgfqpoint{1.160273in}{1.097218in}}%
\pgfpathlineto{\pgfqpoint{1.159858in}{1.110404in}}%
\pgfpathlineto{\pgfqpoint{1.162195in}{1.122924in}}%
\pgfpathlineto{\pgfqpoint{1.167711in}{1.134772in}}%
\pgfpathlineto{\pgfqpoint{1.175593in}{1.145925in}}%
\pgfpathlineto{\pgfqpoint{1.185463in}{1.156377in}}%
\pgfpathlineto{\pgfqpoint{1.197239in}{1.166126in}}%
\pgfpathlineto{\pgfqpoint{1.210871in}{1.175170in}}%
\pgfpathlineto{\pgfqpoint{1.226341in}{1.183508in}}%
\pgfpathlineto{\pgfqpoint{1.253032in}{1.194694in}}%
\pgfpathlineto{\pgfqpoint{1.284087in}{1.204296in}}%
\pgfpathlineto{\pgfqpoint{1.319852in}{1.212323in}}%
\pgfpathlineto{\pgfqpoint{1.360463in}{1.218764in}}%
\pgfpathlineto{\pgfqpoint{1.406153in}{1.223597in}}%
\pgfpathlineto{\pgfqpoint{1.457375in}{1.226791in}}%
\pgfpathlineto{\pgfqpoint{1.514635in}{1.228308in}}%
\pgfpathlineto{\pgfqpoint{1.578496in}{1.228099in}}%
\pgfpathlineto{\pgfqpoint{1.649574in}{1.226105in}}%
\pgfpathlineto{\pgfqpoint{1.756737in}{1.220549in}}%
\pgfpathlineto{\pgfqpoint{1.879635in}{1.211506in}}%
\pgfpathlineto{\pgfqpoint{2.019694in}{1.198720in}}%
\pgfpathlineto{\pgfqpoint{2.177184in}{1.181932in}}%
\pgfpathlineto{\pgfqpoint{2.350177in}{1.160944in}}%
\pgfpathlineto{\pgfqpoint{2.487139in}{1.142397in}}%
\pgfpathlineto{\pgfqpoint{2.626206in}{1.121509in}}%
\pgfpathlineto{\pgfqpoint{2.762938in}{1.098455in}}%
\pgfpathlineto{\pgfqpoint{2.850332in}{1.082017in}}%
\pgfpathlineto{\pgfqpoint{2.932857in}{1.064829in}}%
\pgfpathlineto{\pgfqpoint{3.009536in}{1.047035in}}%
\pgfpathlineto{\pgfqpoint{3.079765in}{1.028800in}}%
\pgfpathlineto{\pgfqpoint{3.143143in}{1.010276in}}%
\pgfpathlineto{\pgfqpoint{3.199462in}{0.991604in}}%
\pgfpathlineto{\pgfqpoint{3.248717in}{0.972911in}}%
\pgfpathlineto{\pgfqpoint{3.291098in}{0.954316in}}%
\pgfpathlineto{\pgfqpoint{3.326994in}{0.935923in}}%
\pgfpathlineto{\pgfqpoint{3.356993in}{0.917826in}}%
\pgfpathlineto{\pgfqpoint{3.381881in}{0.900107in}}%
\pgfpathlineto{\pgfqpoint{3.401995in}{0.882851in}}%
\pgfpathlineto{\pgfqpoint{3.417714in}{0.866121in}}%
\pgfpathlineto{\pgfqpoint{3.429871in}{0.849948in}}%
\pgfpathlineto{\pgfqpoint{3.439099in}{0.834362in}}%
\pgfpathlineto{\pgfqpoint{3.445834in}{0.819387in}}%
\pgfpathlineto{\pgfqpoint{3.450310in}{0.805044in}}%
\pgfpathlineto{\pgfqpoint{3.452562in}{0.791349in}}%
\pgfpathlineto{\pgfqpoint{3.452426in}{0.778314in}}%
\pgfpathlineto{\pgfqpoint{3.449537in}{0.765947in}}%
\pgfpathlineto{\pgfqpoint{3.443489in}{0.754255in}}%
\pgfpathlineto{\pgfqpoint{3.435131in}{0.743257in}}%
\pgfpathlineto{\pgfqpoint{3.424800in}{0.732962in}}%
\pgfpathlineto{\pgfqpoint{3.412572in}{0.723371in}}%
\pgfpathlineto{\pgfqpoint{3.398491in}{0.714486in}}%
\pgfpathlineto{\pgfqpoint{3.382572in}{0.706306in}}%
\pgfpathlineto{\pgfqpoint{3.355195in}{0.695359in}}%
\pgfpathlineto{\pgfqpoint{3.323433in}{0.685996in}}%
\pgfpathlineto{\pgfqpoint{3.286935in}{0.678211in}}%
\pgfpathlineto{\pgfqpoint{3.245559in}{0.672014in}}%
\pgfpathlineto{\pgfqpoint{3.199050in}{0.667429in}}%
\pgfpathlineto{\pgfqpoint{3.146935in}{0.664487in}}%
\pgfpathlineto{\pgfqpoint{3.088692in}{0.663230in}}%
\pgfpathlineto{\pgfqpoint{3.023749in}{0.663706in}}%
\pgfpathlineto{\pgfqpoint{2.925654in}{0.667147in}}%
\pgfpathlineto{\pgfqpoint{2.812922in}{0.673952in}}%
\pgfpathlineto{\pgfqpoint{2.683876in}{0.684331in}}%
\pgfpathlineto{\pgfqpoint{2.537363in}{0.698552in}}%
\pgfpathlineto{\pgfqpoint{2.373828in}{0.716857in}}%
\pgfpathlineto{\pgfqpoint{2.196204in}{0.739410in}}%
\pgfpathlineto{\pgfqpoint{2.057658in}{0.759114in}}%
\pgfpathlineto{\pgfqpoint{1.919131in}{0.781087in}}%
\pgfpathlineto{\pgfqpoint{1.785314in}{0.805103in}}%
\pgfpathlineto{\pgfqpoint{1.701279in}{0.822094in}}%
\pgfpathlineto{\pgfqpoint{1.622972in}{0.839739in}}%
\pgfpathlineto{\pgfqpoint{1.551021in}{0.857874in}}%
\pgfpathlineto{\pgfqpoint{1.485834in}{0.876340in}}%
\pgfpathlineto{\pgfqpoint{1.427639in}{0.894992in}}%
\pgfpathlineto{\pgfqpoint{1.376485in}{0.913696in}}%
\pgfpathlineto{\pgfqpoint{1.332234in}{0.932331in}}%
\pgfpathlineto{\pgfqpoint{1.294571in}{0.950790in}}%
\pgfpathlineto{\pgfqpoint{1.262998in}{0.968974in}}%
\pgfpathlineto{\pgfqpoint{1.236835in}{0.986801in}}%
\pgfpathlineto{\pgfqpoint{1.215355in}{1.004194in}}%
\pgfpathlineto{\pgfqpoint{1.198497in}{1.021074in}}%
\pgfpathlineto{\pgfqpoint{1.185518in}{1.037401in}}%
\pgfpathlineto{\pgfqpoint{1.175694in}{1.053147in}}%
\pgfpathlineto{\pgfqpoint{1.168494in}{1.068285in}}%
\pgfpathlineto{\pgfqpoint{1.163581in}{1.082796in}}%
\pgfpathlineto{\pgfqpoint{1.160814in}{1.096663in}}%
\pgfpathlineto{\pgfqpoint{1.160245in}{1.109872in}}%
\pgfpathlineto{\pgfqpoint{1.162124in}{1.122414in}}%
\pgfpathlineto{\pgfqpoint{1.166891in}{1.134286in}}%
\pgfpathlineto{\pgfqpoint{1.174641in}{1.145477in}}%
\pgfpathlineto{\pgfqpoint{1.184445in}{1.155968in}}%
\pgfpathlineto{\pgfqpoint{1.196178in}{1.165755in}}%
\pgfpathlineto{\pgfqpoint{1.209783in}{1.174838in}}%
\pgfpathlineto{\pgfqpoint{1.225234in}{1.183214in}}%
\pgfpathlineto{\pgfqpoint{1.251887in}{1.194455in}}%
\pgfpathlineto{\pgfqpoint{1.282851in}{1.204108in}}%
\pgfpathlineto{\pgfqpoint{1.318418in}{1.212178in}}%
\pgfpathlineto{\pgfqpoint{1.358912in}{1.218665in}}%
\pgfpathlineto{\pgfqpoint{1.404444in}{1.223546in}}%
\pgfpathlineto{\pgfqpoint{1.455480in}{1.226792in}}%
\pgfpathlineto{\pgfqpoint{1.512556in}{1.228363in}}%
\pgfpathlineto{\pgfqpoint{1.576243in}{1.228210in}}%
\pgfpathlineto{\pgfqpoint{1.647154in}{1.226274in}}%
\pgfpathlineto{\pgfqpoint{1.754068in}{1.220794in}}%
\pgfpathlineto{\pgfqpoint{1.876637in}{1.211823in}}%
\pgfpathlineto{\pgfqpoint{2.016315in}{1.199120in}}%
\pgfpathlineto{\pgfqpoint{2.173503in}{1.182416in}}%
\pgfpathlineto{\pgfqpoint{2.346222in}{1.161515in}}%
\pgfpathlineto{\pgfqpoint{2.483125in}{1.143033in}}%
\pgfpathlineto{\pgfqpoint{2.622311in}{1.122204in}}%
\pgfpathlineto{\pgfqpoint{2.759095in}{1.099204in}}%
\pgfpathlineto{\pgfqpoint{2.846640in}{1.082799in}}%
\pgfpathlineto{\pgfqpoint{2.929660in}{1.065660in}}%
\pgfpathlineto{\pgfqpoint{3.006577in}{1.047915in}}%
\pgfpathlineto{\pgfqpoint{3.076266in}{1.029706in}}%
\pgfpathlineto{\pgfqpoint{3.138954in}{1.011191in}}%
\pgfpathlineto{\pgfqpoint{3.194923in}{0.992515in}}%
\pgfpathlineto{\pgfqpoint{3.244454in}{0.973810in}}%
\pgfpathlineto{\pgfqpoint{3.287830in}{0.955197in}}%
\pgfpathlineto{\pgfqpoint{3.325334in}{0.936782in}}%
\pgfpathlineto{\pgfqpoint{3.357252in}{0.918657in}}%
\pgfpathlineto{\pgfqpoint{3.383870in}{0.900904in}}%
\pgfpathlineto{\pgfqpoint{3.405475in}{0.883590in}}%
\pgfpathlineto{\pgfqpoint{3.422355in}{0.866769in}}%
\pgfpathlineto{\pgfqpoint{3.434802in}{0.850484in}}%
\pgfpathlineto{\pgfqpoint{3.443314in}{0.834801in}}%
\pgfpathlineto{\pgfqpoint{3.448610in}{0.819755in}}%
\pgfpathlineto{\pgfqpoint{3.451287in}{0.805363in}}%
\pgfpathlineto{\pgfqpoint{3.451798in}{0.791637in}}%
\pgfpathlineto{\pgfqpoint{3.450455in}{0.778587in}}%
\pgfpathlineto{\pgfqpoint{3.447425in}{0.766220in}}%
\pgfpathlineto{\pgfqpoint{3.442732in}{0.754538in}}%
\pgfpathlineto{\pgfqpoint{3.436257in}{0.743543in}}%
\pgfpathlineto{\pgfqpoint{3.427738in}{0.733232in}}%
\pgfpathlineto{\pgfqpoint{3.416770in}{0.723597in}}%
\pgfpathlineto{\pgfqpoint{3.402906in}{0.714633in}}%
\pgfpathlineto{\pgfqpoint{3.386851in}{0.706366in}}%
\pgfpathlineto{\pgfqpoint{3.359189in}{0.695291in}}%
\pgfpathlineto{\pgfqpoint{3.327183in}{0.685810in}}%
\pgfpathlineto{\pgfqpoint{3.290675in}{0.677931in}}%
\pgfpathlineto{\pgfqpoint{3.249402in}{0.671665in}}%
\pgfpathlineto{\pgfqpoint{3.202991in}{0.667023in}}%
\pgfpathlineto{\pgfqpoint{3.150965in}{0.664021in}}%
\pgfpathlineto{\pgfqpoint{3.093142in}{0.662676in}}%
\pgfpathlineto{\pgfqpoint{3.028820in}{0.663052in}}%
\pgfpathlineto{\pgfqpoint{2.931117in}{0.666364in}}%
\pgfpathlineto{\pgfqpoint{2.818078in}{0.673073in}}%
\pgfpathlineto{\pgfqpoint{2.688572in}{0.683386in}}%
\pgfpathlineto{\pgfqpoint{2.542231in}{0.697523in}}%
\pgfpathlineto{\pgfqpoint{2.379446in}{0.715714in}}%
\pgfpathlineto{\pgfqpoint{2.201339in}{0.738218in}}%
\pgfpathlineto{\pgfqpoint{2.061658in}{0.757966in}}%
\pgfpathlineto{\pgfqpoint{1.922769in}{0.779956in}}%
\pgfpathlineto{\pgfqpoint{1.789403in}{0.803922in}}%
\pgfpathlineto{\pgfqpoint{1.705529in}{0.820842in}}%
\pgfpathlineto{\pgfqpoint{1.626828in}{0.838391in}}%
\pgfpathlineto{\pgfqpoint{1.554102in}{0.856452in}}%
\pgfpathlineto{\pgfqpoint{1.487992in}{0.874894in}}%
\pgfpathlineto{\pgfqpoint{1.428979in}{0.893579in}}%
\pgfpathlineto{\pgfqpoint{1.377383in}{0.912355in}}%
\pgfpathlineto{\pgfqpoint{1.333216in}{0.931065in}}%
\pgfpathlineto{\pgfqpoint{1.295656in}{0.949588in}}%
\pgfpathlineto{\pgfqpoint{1.263832in}{0.967834in}}%
\pgfpathlineto{\pgfqpoint{1.237050in}{0.985722in}}%
\pgfpathlineto{\pgfqpoint{1.214780in}{1.003183in}}%
\pgfpathlineto{\pgfqpoint{1.196660in}{1.020154in}}%
\pgfpathlineto{\pgfqpoint{1.182497in}{1.036582in}}%
\pgfpathlineto{\pgfqpoint{1.172260in}{1.052424in}}%
\pgfpathlineto{\pgfqpoint{1.165484in}{1.067643in}}%
\pgfpathlineto{\pgfqpoint{1.161563in}{1.082216in}}%
\pgfpathlineto{\pgfqpoint{1.160205in}{1.096127in}}%
\pgfpathlineto{\pgfqpoint{1.161190in}{1.109362in}}%
\pgfpathlineto{\pgfqpoint{1.164365in}{1.121913in}}%
\pgfpathlineto{\pgfqpoint{1.169648in}{1.133774in}}%
\pgfpathlineto{\pgfqpoint{1.177029in}{1.144941in}}%
\pgfpathlineto{\pgfqpoint{1.186546in}{1.155414in}}%
\pgfpathlineto{\pgfqpoint{1.198095in}{1.165191in}}%
\pgfpathlineto{\pgfqpoint{1.211580in}{1.174271in}}%
\pgfpathlineto{\pgfqpoint{1.226952in}{1.182650in}}%
\pgfpathlineto{\pgfqpoint{1.253507in}{1.193900in}}%
\pgfpathlineto{\pgfqpoint{1.284310in}{1.203564in}}%
\pgfpathlineto{\pgfqpoint{1.319551in}{1.211637in}}%
\pgfpathlineto{\pgfqpoint{1.359564in}{1.218118in}}%
\pgfpathlineto{\pgfqpoint{1.404826in}{1.223005in}}%
\pgfpathlineto{\pgfqpoint{1.455553in}{1.226280in}}%
\pgfpathlineto{\pgfqpoint{1.512163in}{1.227893in}}%
\pgfpathlineto{\pgfqpoint{1.575428in}{1.227791in}}%
\pgfpathlineto{\pgfqpoint{1.646058in}{1.225911in}}%
\pgfpathlineto{\pgfqpoint{1.752793in}{1.220512in}}%
\pgfpathlineto{\pgfqpoint{1.875109in}{1.211620in}}%
\pgfpathlineto{\pgfqpoint{2.014152in}{1.199004in}}%
\pgfpathlineto{\pgfqpoint{2.171809in}{1.182375in}}%
\pgfpathlineto{\pgfqpoint{2.348561in}{1.161426in}}%
\pgfpathlineto{\pgfqpoint{2.486929in}{1.142913in}}%
\pgfpathlineto{\pgfqpoint{2.625098in}{1.122128in}}%
\pgfpathlineto{\pgfqpoint{2.759051in}{1.099255in}}%
\pgfpathlineto{\pgfqpoint{2.885370in}{1.074547in}}%
\pgfpathlineto{\pgfqpoint{2.963920in}{1.057210in}}%
\pgfpathlineto{\pgfqpoint{3.037118in}{1.039307in}}%
\pgfpathlineto{\pgfqpoint{3.104389in}{1.020962in}}%
\pgfpathlineto{\pgfqpoint{3.165278in}{1.002311in}}%
\pgfpathlineto{\pgfqpoint{3.219446in}{0.983504in}}%
\pgfpathlineto{\pgfqpoint{3.266670in}{0.964703in}}%
\pgfpathlineto{\pgfqpoint{3.306848in}{0.946084in}}%
\pgfpathlineto{\pgfqpoint{3.340281in}{0.927795in}}%
\pgfpathlineto{\pgfqpoint{3.368174in}{0.909856in}}%
\pgfpathlineto{\pgfqpoint{3.391137in}{0.892336in}}%
\pgfpathlineto{\pgfqpoint{3.409668in}{0.875299in}}%
\pgfpathlineto{\pgfqpoint{3.424218in}{0.858797in}}%
\pgfpathlineto{\pgfqpoint{3.435202in}{0.842873in}}%
\pgfpathlineto{\pgfqpoint{3.442974in}{0.827556in}}%
\pgfpathlineto{\pgfqpoint{3.447825in}{0.812875in}}%
\pgfpathlineto{\pgfqpoint{3.449937in}{0.798847in}}%
\pgfpathlineto{\pgfqpoint{3.449427in}{0.785487in}}%
\pgfpathlineto{\pgfqpoint{3.446586in}{0.772804in}}%
\pgfpathlineto{\pgfqpoint{3.441653in}{0.760807in}}%
\pgfpathlineto{\pgfqpoint{3.434797in}{0.749503in}}%
\pgfpathlineto{\pgfqpoint{3.426123in}{0.738894in}}%
\pgfpathlineto{\pgfqpoint{3.415666in}{0.728982in}}%
\pgfpathlineto{\pgfqpoint{3.403395in}{0.719766in}}%
\pgfpathlineto{\pgfqpoint{3.389211in}{0.711244in}}%
\pgfpathlineto{\pgfqpoint{3.372949in}{0.703409in}}%
\pgfpathlineto{\pgfqpoint{3.344352in}{0.692939in}}%
\pgfpathlineto{\pgfqpoint{3.311180in}{0.684038in}}%
\pgfpathlineto{\pgfqpoint{3.273361in}{0.676722in}}%
\pgfpathlineto{\pgfqpoint{3.230636in}{0.671008in}}%
\pgfpathlineto{\pgfqpoint{3.182653in}{0.666922in}}%
\pgfpathlineto{\pgfqpoint{3.128970in}{0.664492in}}%
\pgfpathlineto{\pgfqpoint{3.069052in}{0.663752in}}%
\pgfpathlineto{\pgfqpoint{3.002277in}{0.664743in}}%
\pgfpathlineto{\pgfqpoint{2.901846in}{0.668837in}}%
\pgfpathlineto{\pgfqpoint{2.786061in}{0.676344in}}%
\pgfpathlineto{\pgfqpoint{2.652967in}{0.687539in}}%
\pgfpathlineto{\pgfqpoint{2.502390in}{0.702647in}}%
\pgfpathlineto{\pgfqpoint{2.335929in}{0.721843in}}%
\pgfpathlineto{\pgfqpoint{2.156965in}{0.745252in}}%
\pgfpathlineto{\pgfqpoint{2.017955in}{0.765623in}}%
\pgfpathlineto{\pgfqpoint{1.880284in}{0.788238in}}%
\pgfpathlineto{\pgfqpoint{1.749103in}{0.812755in}}%
\pgfpathlineto{\pgfqpoint{1.667322in}{0.829973in}}%
\pgfpathlineto{\pgfqpoint{1.591223in}{0.847753in}}%
\pgfpathlineto{\pgfqpoint{1.521554in}{0.865971in}}%
\pgfpathlineto{\pgfqpoint{1.458852in}{0.884496in}}%
\pgfpathlineto{\pgfqpoint{1.403447in}{0.903193in}}%
\pgfpathlineto{\pgfqpoint{1.355463in}{0.921919in}}%
\pgfpathlineto{\pgfqpoint{1.314700in}{0.940517in}}%
\pgfpathlineto{\pgfqpoint{1.280234in}{0.958876in}}%
\pgfpathlineto{\pgfqpoint{1.251115in}{0.976918in}}%
\pgfpathlineto{\pgfqpoint{1.226605in}{0.994573in}}%
\pgfpathlineto{\pgfqpoint{1.206181in}{1.011777in}}%
\pgfpathlineto{\pgfqpoint{1.189536in}{1.028475in}}%
\pgfpathlineto{\pgfqpoint{1.176576in}{1.044618in}}%
\pgfpathlineto{\pgfqpoint{1.167420in}{1.060166in}}%
\pgfpathlineto{\pgfqpoint{1.162089in}{1.075082in}}%
\pgfpathlineto{\pgfqpoint{1.159608in}{1.089340in}}%
\pgfpathlineto{\pgfqpoint{1.159613in}{1.102926in}}%
\pgfpathlineto{\pgfqpoint{1.161879in}{1.115829in}}%
\pgfpathlineto{\pgfqpoint{1.166245in}{1.128044in}}%
\pgfpathlineto{\pgfqpoint{1.172615in}{1.139563in}}%
\pgfpathlineto{\pgfqpoint{1.180956in}{1.150384in}}%
\pgfpathlineto{\pgfqpoint{1.191302in}{1.160509in}}%
\pgfpathlineto{\pgfqpoint{1.203739in}{1.169938in}}%
\pgfpathlineto{\pgfqpoint{1.218186in}{1.178672in}}%
\pgfpathlineto{\pgfqpoint{1.234556in}{1.186707in}}%
\pgfpathlineto{\pgfqpoint{1.262695in}{1.197443in}}%
\pgfpathlineto{\pgfqpoint{1.295186in}{1.206592in}}%
\pgfpathlineto{\pgfqpoint{1.332199in}{1.214145in}}%
\pgfpathlineto{\pgfqpoint{1.374025in}{1.220092in}}%
\pgfpathlineto{\pgfqpoint{1.421072in}{1.224421in}}%
\pgfpathlineto{\pgfqpoint{1.473864in}{1.227120in}}%
\pgfpathlineto{\pgfqpoint{1.532547in}{1.228164in}}%
\pgfpathlineto{\pgfqpoint{1.597887in}{1.227483in}}%
\pgfpathlineto{\pgfqpoint{1.697140in}{1.223752in}}%
\pgfpathlineto{\pgfqpoint{1.811881in}{1.216609in}}%
\pgfpathlineto{\pgfqpoint{1.943149in}{1.205845in}}%
\pgfpathlineto{\pgfqpoint{2.091215in}{1.191238in}}%
\pgfpathlineto{\pgfqpoint{2.255575in}{1.172560in}}%
\pgfpathlineto{\pgfqpoint{2.434866in}{1.149550in}}%
\pgfpathlineto{\pgfqpoint{2.574419in}{1.129457in}}%
\pgfpathlineto{\pgfqpoint{2.712362in}{1.107169in}}%
\pgfpathlineto{\pgfqpoint{2.844123in}{1.082958in}}%
\pgfpathlineto{\pgfqpoint{2.926634in}{1.065909in}}%
\pgfpathlineto{\pgfqpoint{3.003787in}{1.048259in}}%
\pgfpathlineto{\pgfqpoint{3.074823in}{1.030130in}}%
\pgfpathlineto{\pgfqpoint{3.139141in}{1.011652in}}%
\pgfpathlineto{\pgfqpoint{3.196299in}{0.992967in}}%
\pgfpathlineto{\pgfqpoint{3.246014in}{0.974229in}}%
\pgfpathlineto{\pgfqpoint{3.288450in}{0.955590in}}%
\pgfpathlineto{\pgfqpoint{3.324519in}{0.937160in}}%
\pgfpathlineto{\pgfqpoint{3.355043in}{0.919027in}}%
\pgfpathlineto{\pgfqpoint{3.380680in}{0.901267in}}%
\pgfpathlineto{\pgfqpoint{3.401917in}{0.883950in}}%
\pgfpathlineto{\pgfqpoint{3.419076in}{0.867134in}}%
\pgfpathlineto{\pgfqpoint{3.432311in}{0.850871in}}%
\pgfpathlineto{\pgfqpoint{3.441648in}{0.835201in}}%
\pgfpathlineto{\pgfqpoint{3.447701in}{0.820161in}}%
\pgfpathlineto{\pgfqpoint{3.450971in}{0.805770in}}%
\pgfpathlineto{\pgfqpoint{3.451735in}{0.792043in}}%
\pgfpathlineto{\pgfqpoint{3.450199in}{0.778993in}}%
\pgfpathlineto{\pgfqpoint{3.446498in}{0.766629in}}%
\pgfpathlineto{\pgfqpoint{3.440701in}{0.754955in}}%
\pgfpathlineto{\pgfqpoint{3.432805in}{0.743975in}}%
\pgfpathlineto{\pgfqpoint{3.422778in}{0.733688in}}%
\pgfpathlineto{\pgfqpoint{3.410744in}{0.724096in}}%
\pgfpathlineto{\pgfqpoint{3.396783in}{0.715202in}}%
\pgfpathlineto{\pgfqpoint{3.380940in}{0.707009in}}%
\pgfpathlineto{\pgfqpoint{3.353669in}{0.696037in}}%
\pgfpathlineto{\pgfqpoint{3.322124in}{0.686651in}}%
\pgfpathlineto{\pgfqpoint{3.286094in}{0.678856in}}%
\pgfpathlineto{\pgfqpoint{3.245229in}{0.672654in}}%
\pgfpathlineto{\pgfqpoint{3.199042in}{0.668045in}}%
\pgfpathlineto{\pgfqpoint{3.147365in}{0.665056in}}%
\pgfpathlineto{\pgfqpoint{3.089688in}{0.663736in}}%
\pgfpathlineto{\pgfqpoint{3.025229in}{0.664141in}}%
\pgfpathlineto{\pgfqpoint{2.927520in}{0.667477in}}%
\pgfpathlineto{\pgfqpoint{2.815024in}{0.674176in}}%
\pgfpathlineto{\pgfqpoint{2.686487in}{0.684451in}}%
\pgfpathlineto{\pgfqpoint{2.540882in}{0.698546in}}%
\pgfpathlineto{\pgfqpoint{2.375584in}{0.716803in}}%
\pgfpathlineto{\pgfqpoint{2.240532in}{0.733371in}}%
\pgfpathlineto{\pgfqpoint{2.101781in}{0.752317in}}%
\pgfpathlineto{\pgfqpoint{1.963948in}{0.773497in}}%
\pgfpathlineto{\pgfqpoint{1.831023in}{0.796704in}}%
\pgfpathlineto{\pgfqpoint{1.746816in}{0.813173in}}%
\pgfpathlineto{\pgfqpoint{1.667135in}{0.830332in}}%
\pgfpathlineto{\pgfqpoint{1.592694in}{0.848076in}}%
\pgfpathlineto{\pgfqpoint{1.524083in}{0.866286in}}%
\pgfpathlineto{\pgfqpoint{1.461768in}{0.884833in}}%
\pgfpathlineto{\pgfqpoint{1.406089in}{0.903575in}}%
\pgfpathlineto{\pgfqpoint{1.357264in}{0.922358in}}%
\pgfpathlineto{\pgfqpoint{1.315384in}{0.941017in}}%
\pgfpathlineto{\pgfqpoint{1.280344in}{0.959384in}}%
\pgfpathlineto{\pgfqpoint{1.251113in}{0.977409in}}%
\pgfpathlineto{\pgfqpoint{1.226946in}{0.995033in}}%
\pgfpathlineto{\pgfqpoint{1.207326in}{1.012189in}}%
\pgfpathlineto{\pgfqpoint{1.191799in}{1.028823in}}%
\pgfpathlineto{\pgfqpoint{1.179943in}{1.044890in}}%
\pgfpathlineto{\pgfqpoint{1.171384in}{1.060356in}}%
\pgfpathlineto{\pgfqpoint{1.165814in}{1.075194in}}%
\pgfpathlineto{\pgfqpoint{1.162983in}{1.089383in}}%
\pgfpathlineto{\pgfqpoint{1.162783in}{1.102908in}}%
\pgfpathlineto{\pgfqpoint{1.165047in}{1.115758in}}%
\pgfpathlineto{\pgfqpoint{1.169512in}{1.127925in}}%
\pgfpathlineto{\pgfqpoint{1.175978in}{1.139402in}}%
\pgfpathlineto{\pgfqpoint{1.184305in}{1.150184in}}%
\pgfpathlineto{\pgfqpoint{1.194419in}{1.160269in}}%
\pgfpathlineto{\pgfqpoint{1.206308in}{1.169655in}}%
\pgfpathlineto{\pgfqpoint{1.220021in}{1.178345in}}%
\pgfpathlineto{\pgfqpoint{1.235671in}{1.186343in}}%
\pgfpathlineto{\pgfqpoint{1.253435in}{1.193654in}}%
\pgfpathlineto{\pgfqpoint{1.284276in}{1.203336in}}%
\pgfpathlineto{\pgfqpoint{1.319727in}{1.211445in}}%
\pgfpathlineto{\pgfqpoint{1.359951in}{1.217964in}}%
\pgfpathlineto{\pgfqpoint{1.405255in}{1.222870in}}%
\pgfpathlineto{\pgfqpoint{1.456035in}{1.226137in}}%
\pgfpathlineto{\pgfqpoint{1.512770in}{1.227730in}}%
\pgfpathlineto{\pgfqpoint{1.576029in}{1.227608in}}%
\pgfpathlineto{\pgfqpoint{1.671621in}{1.224697in}}%
\pgfpathlineto{\pgfqpoint{1.781418in}{1.218487in}}%
\pgfpathlineto{\pgfqpoint{1.907700in}{1.208711in}}%
\pgfpathlineto{\pgfqpoint{2.051490in}{1.195120in}}%
\pgfpathlineto{\pgfqpoint{2.212105in}{1.177504in}}%
\pgfpathlineto{\pgfqpoint{2.387161in}{1.155687in}}%
\pgfpathlineto{\pgfqpoint{2.525369in}{1.136474in}}%
\pgfpathlineto{\pgfqpoint{2.664449in}{1.114942in}}%
\pgfpathlineto{\pgfqpoint{2.799173in}{1.091362in}}%
\pgfpathlineto{\pgfqpoint{2.884389in}{1.074662in}}%
\pgfpathlineto{\pgfqpoint{2.964656in}{1.057302in}}%
\pgfpathlineto{\pgfqpoint{3.039086in}{1.039400in}}%
\pgfpathlineto{\pgfqpoint{3.106963in}{1.021085in}}%
\pgfpathlineto{\pgfqpoint{3.167750in}{1.002493in}}%
\pgfpathlineto{\pgfqpoint{3.221086in}{0.983774in}}%
\pgfpathlineto{\pgfqpoint{3.266944in}{0.965087in}}%
\pgfpathlineto{\pgfqpoint{3.306125in}{0.946555in}}%
\pgfpathlineto{\pgfqpoint{3.339506in}{0.928270in}}%
\pgfpathlineto{\pgfqpoint{3.367790in}{0.910315in}}%
\pgfpathlineto{\pgfqpoint{3.391500in}{0.892763in}}%
\pgfpathlineto{\pgfqpoint{3.410984in}{0.875680in}}%
\pgfpathlineto{\pgfqpoint{3.426409in}{0.859122in}}%
\pgfpathlineto{\pgfqpoint{3.437768in}{0.843136in}}%
\pgfpathlineto{\pgfqpoint{3.445330in}{0.827764in}}%
\pgfpathlineto{\pgfqpoint{3.449932in}{0.813033in}}%
\pgfpathlineto{\pgfqpoint{3.451906in}{0.798960in}}%
\pgfpathlineto{\pgfqpoint{3.451505in}{0.785559in}}%
\pgfpathlineto{\pgfqpoint{3.448905in}{0.772839in}}%
\pgfpathlineto{\pgfqpoint{3.444209in}{0.760808in}}%
\pgfpathlineto{\pgfqpoint{3.437444in}{0.749470in}}%
\pgfpathlineto{\pgfqpoint{3.428564in}{0.738825in}}%
\pgfpathlineto{\pgfqpoint{3.417545in}{0.728874in}}%
\pgfpathlineto{\pgfqpoint{3.404555in}{0.719619in}}%
\pgfpathlineto{\pgfqpoint{3.389653in}{0.711063in}}%
\pgfpathlineto{\pgfqpoint{3.363774in}{0.699545in}}%
\pgfpathlineto{\pgfqpoint{3.333643in}{0.689612in}}%
\pgfpathlineto{\pgfqpoint{3.299109in}{0.681269in}}%
\pgfpathlineto{\pgfqpoint{3.259889in}{0.674521in}}%
\pgfpathlineto{\pgfqpoint{3.215560in}{0.669372in}}%
\pgfpathlineto{\pgfqpoint{3.165721in}{0.665828in}}%
\pgfpathlineto{\pgfqpoint{3.110178in}{0.663934in}}%
\pgfpathlineto{\pgfqpoint{3.048131in}{0.663745in}}%
\pgfpathlineto{\pgfqpoint{2.978810in}{0.665323in}}%
\pgfpathlineto{\pgfqpoint{2.873908in}{0.670301in}}%
\pgfpathlineto{\pgfqpoint{2.753522in}{0.678746in}}%
\pgfpathlineto{\pgfqpoint{2.616580in}{0.690883in}}%
\pgfpathlineto{\pgfqpoint{2.462190in}{0.706973in}}%
\pgfpathlineto{\pgfqpoint{2.287850in}{0.727445in}}%
\pgfpathlineto{\pgfqpoint{2.149421in}{0.745665in}}%
\pgfpathlineto{\pgfqpoint{2.010261in}{0.766170in}}%
\pgfpathlineto{\pgfqpoint{1.874839in}{0.788754in}}%
\pgfpathlineto{\pgfqpoint{1.746888in}{0.813159in}}%
\pgfpathlineto{\pgfqpoint{1.667250in}{0.830289in}}%
\pgfpathlineto{\pgfqpoint{1.592991in}{0.847987in}}%
\pgfpathlineto{\pgfqpoint{1.524672in}{0.866140in}}%
\pgfpathlineto{\pgfqpoint{1.462714in}{0.884623in}}%
\pgfpathlineto{\pgfqpoint{1.407390in}{0.903302in}}%
\pgfpathlineto{\pgfqpoint{1.358829in}{0.922032in}}%
\pgfpathlineto{\pgfqpoint{1.317015in}{0.940660in}}%
\pgfpathlineto{\pgfqpoint{1.281779in}{0.959023in}}%
\pgfpathlineto{\pgfqpoint{1.252396in}{0.977039in}}%
\pgfpathlineto{\pgfqpoint{1.228085in}{0.994658in}}%
\pgfpathlineto{\pgfqpoint{1.208287in}{1.011814in}}%
\pgfpathlineto{\pgfqpoint{1.192542in}{1.028452in}}%
\pgfpathlineto{\pgfqpoint{1.180485in}{1.044527in}}%
\pgfpathlineto{\pgfqpoint{1.171819in}{1.060002in}}%
\pgfpathlineto{\pgfqpoint{1.166177in}{1.074849in}}%
\pgfpathlineto{\pgfqpoint{1.163273in}{1.089049in}}%
\pgfpathlineto{\pgfqpoint{1.162886in}{1.102586in}}%
\pgfpathlineto{\pgfqpoint{1.164849in}{1.115448in}}%
\pgfpathlineto{\pgfqpoint{1.169049in}{1.127626in}}%
\pgfpathlineto{\pgfqpoint{1.175448in}{1.139118in}}%
\pgfpathlineto{\pgfqpoint{1.183963in}{1.149920in}}%
\pgfpathlineto{\pgfqpoint{1.194457in}{1.160029in}}%
\pgfpathlineto{\pgfqpoint{1.206831in}{1.169440in}}%
\pgfpathlineto{\pgfqpoint{1.221024in}{1.178150in}}%
\pgfpathlineto{\pgfqpoint{1.245691in}{1.189901in}}%
\pgfpathlineto{\pgfqpoint{1.274507in}{1.200073in}}%
\pgfpathlineto{\pgfqpoint{1.307762in}{1.208670in}}%
\pgfpathlineto{\pgfqpoint{1.345945in}{1.215704in}}%
\pgfpathlineto{\pgfqpoint{1.389373in}{1.221167in}}%
\pgfpathlineto{\pgfqpoint{1.438084in}{1.225014in}}%
\pgfpathlineto{\pgfqpoint{1.492617in}{1.227211in}}%
\pgfpathlineto{\pgfqpoint{1.553548in}{1.227711in}}%
\pgfpathlineto{\pgfqpoint{1.621480in}{1.226457in}}%
\pgfpathlineto{\pgfqpoint{1.724036in}{1.221940in}}%
\pgfpathlineto{\pgfqpoint{1.841724in}{1.213996in}}%
\pgfpathlineto{\pgfqpoint{1.976119in}{1.202401in}}%
\pgfpathlineto{\pgfqpoint{2.128025in}{1.186890in}}%
\pgfpathlineto{\pgfqpoint{2.296338in}{1.167232in}}%
\pgfpathlineto{\pgfqpoint{2.430927in}{1.149701in}}%
\pgfpathlineto{\pgfqpoint{2.569498in}{1.129788in}}%
\pgfpathlineto{\pgfqpoint{2.707585in}{1.107625in}}%
\pgfpathlineto{\pgfqpoint{2.840195in}{1.083464in}}%
\pgfpathlineto{\pgfqpoint{2.923369in}{1.066412in}}%
\pgfpathlineto{\pgfqpoint{3.001008in}{1.048744in}}%
\pgfpathlineto{\pgfqpoint{3.072076in}{1.030601in}}%
\pgfpathlineto{\pgfqpoint{3.135712in}{1.012139in}}%
\pgfpathlineto{\pgfqpoint{3.192147in}{0.993502in}}%
\pgfpathlineto{\pgfqpoint{3.241819in}{0.974819in}}%
\pgfpathlineto{\pgfqpoint{3.285136in}{0.956211in}}%
\pgfpathlineto{\pgfqpoint{3.322479in}{0.937785in}}%
\pgfpathlineto{\pgfqpoint{3.354197in}{0.919639in}}%
\pgfpathlineto{\pgfqpoint{3.380613in}{0.901859in}}%
\pgfpathlineto{\pgfqpoint{3.402019in}{0.884520in}}%
\pgfpathlineto{\pgfqpoint{3.418687in}{0.867686in}}%
\pgfpathlineto{\pgfqpoint{3.431197in}{0.851417in}}%
\pgfpathlineto{\pgfqpoint{3.440255in}{0.835747in}}%
\pgfpathlineto{\pgfqpoint{3.446384in}{0.820700in}}%
\pgfpathlineto{\pgfqpoint{3.449978in}{0.806296in}}%
\pgfpathlineto{\pgfqpoint{3.451298in}{0.792552in}}%
\pgfpathlineto{\pgfqpoint{3.450477in}{0.779478in}}%
\pgfpathlineto{\pgfqpoint{3.447516in}{0.767084in}}%
\pgfpathlineto{\pgfqpoint{3.442286in}{0.755373in}}%
\pgfpathlineto{\pgfqpoint{3.434536in}{0.744344in}}%
\pgfpathlineto{\pgfqpoint{3.424486in}{0.734007in}}%
\pgfpathlineto{\pgfqpoint{3.412477in}{0.724368in}}%
\pgfpathlineto{\pgfqpoint{3.398568in}{0.715432in}}%
\pgfpathlineto{\pgfqpoint{3.382793in}{0.707198in}}%
\pgfpathlineto{\pgfqpoint{3.355634in}{0.696168in}}%
\pgfpathlineto{\pgfqpoint{3.324177in}{0.686725in}}%
\pgfpathlineto{\pgfqpoint{3.288176in}{0.678870in}}%
\pgfpathlineto{\pgfqpoint{3.247241in}{0.672602in}}%
\pgfpathlineto{\pgfqpoint{3.201122in}{0.667930in}}%
\pgfpathlineto{\pgfqpoint{3.149532in}{0.664891in}}%
\pgfpathlineto{\pgfqpoint{3.091835in}{0.663527in}}%
\pgfpathlineto{\pgfqpoint{3.027393in}{0.663889in}}%
\pgfpathlineto{\pgfqpoint{2.929893in}{0.667166in}}%
\pgfpathlineto{\pgfqpoint{2.817801in}{0.673804in}}%
\pgfpathlineto{\pgfqpoint{2.689651in}{0.684014in}}%
\pgfpathlineto{\pgfqpoint{2.544107in}{0.698036in}}%
\pgfpathlineto{\pgfqpoint{2.381369in}{0.716129in}}%
\pgfpathlineto{\pgfqpoint{2.204473in}{0.738491in}}%
\pgfpathlineto{\pgfqpoint{2.066244in}{0.758046in}}%
\pgfpathlineto{\pgfqpoint{1.927568in}{0.779866in}}%
\pgfpathlineto{\pgfqpoint{1.793442in}{0.803763in}}%
\pgfpathlineto{\pgfqpoint{1.709023in}{0.820696in}}%
\pgfpathlineto{\pgfqpoint{1.629965in}{0.838262in}}%
\pgfpathlineto{\pgfqpoint{1.557135in}{0.856316in}}%
\pgfpathlineto{\pgfqpoint{1.491131in}{0.874720in}}%
\pgfpathlineto{\pgfqpoint{1.432281in}{0.893338in}}%
\pgfpathlineto{\pgfqpoint{1.380644in}{0.912041in}}%
\pgfpathlineto{\pgfqpoint{1.336008in}{0.930705in}}%
\pgfpathlineto{\pgfqpoint{1.297898in}{0.949209in}}%
\pgfpathlineto{\pgfqpoint{1.266102in}{0.967421in}}%
\pgfpathlineto{\pgfqpoint{1.239863in}{0.985257in}}%
\pgfpathlineto{\pgfqpoint{1.218140in}{1.002665in}}%
\pgfpathlineto{\pgfqpoint{1.200145in}{1.019596in}}%
\pgfpathlineto{\pgfqpoint{1.185348in}{1.036007in}}%
\pgfpathlineto{\pgfqpoint{1.173476in}{1.051860in}}%
\pgfpathlineto{\pgfqpoint{1.164510in}{1.067119in}}%
\pgfpathlineto{\pgfqpoint{1.158688in}{1.081756in}}%
\pgfpathlineto{\pgfqpoint{1.156504in}{1.095746in}}%
\pgfpathlineto{\pgfqpoint{1.157747in}{1.109063in}}%
\pgfpathlineto{\pgfqpoint{1.161287in}{1.121688in}}%
\pgfpathlineto{\pgfqpoint{1.166939in}{1.133616in}}%
\pgfpathlineto{\pgfqpoint{1.174575in}{1.144843in}}%
\pgfpathlineto{\pgfqpoint{1.184109in}{1.155367in}}%
\pgfpathlineto{\pgfqpoint{1.195503in}{1.165186in}}%
\pgfpathlineto{\pgfqpoint{1.208763in}{1.174301in}}%
\pgfpathlineto{\pgfqpoint{1.223942in}{1.182714in}}%
\pgfpathlineto{\pgfqpoint{1.241137in}{1.190430in}}%
\pgfpathlineto{\pgfqpoint{1.270709in}{1.200695in}}%
\pgfpathlineto{\pgfqpoint{1.304758in}{1.209378in}}%
\pgfpathlineto{\pgfqpoint{1.343465in}{1.216466in}}%
\pgfpathlineto{\pgfqpoint{1.387107in}{1.221941in}}%
\pgfpathlineto{\pgfqpoint{1.436053in}{1.225779in}}%
\pgfpathlineto{\pgfqpoint{1.490770in}{1.227951in}}%
\pgfpathlineto{\pgfqpoint{1.551818in}{1.228423in}}%
\pgfpathlineto{\pgfqpoint{1.619830in}{1.227159in}}%
\pgfpathlineto{\pgfqpoint{1.722055in}{1.222682in}}%
\pgfpathlineto{\pgfqpoint{1.839993in}{1.214750in}}%
\pgfpathlineto{\pgfqpoint{1.975433in}{1.203089in}}%
\pgfpathlineto{\pgfqpoint{2.128328in}{1.187477in}}%
\pgfpathlineto{\pgfqpoint{2.296798in}{1.167752in}}%
\pgfpathlineto{\pgfqpoint{2.431207in}{1.150193in}}%
\pgfpathlineto{\pgfqpoint{2.569955in}{1.130226in}}%
\pgfpathlineto{\pgfqpoint{2.708439in}{1.107948in}}%
\pgfpathlineto{\pgfqpoint{2.841285in}{1.083695in}}%
\pgfpathlineto{\pgfqpoint{2.924535in}{1.066615in}}%
\pgfpathlineto{\pgfqpoint{3.002317in}{1.048945in}}%
\pgfpathlineto{\pgfqpoint{3.073828in}{1.030807in}}%
\pgfpathlineto{\pgfqpoint{3.138480in}{1.012329in}}%
\pgfpathlineto{\pgfqpoint{3.195898in}{0.993645in}}%
\pgfpathlineto{\pgfqpoint{3.245922in}{0.974892in}}%
\pgfpathlineto{\pgfqpoint{3.288610in}{0.956217in}}%
\pgfpathlineto{\pgfqpoint{3.324585in}{0.937762in}}%
\pgfpathlineto{\pgfqpoint{3.354893in}{0.919611in}}%
\pgfpathlineto{\pgfqpoint{3.380390in}{0.901833in}}%
\pgfpathlineto{\pgfqpoint{3.401710in}{0.884491in}}%
\pgfpathlineto{\pgfqpoint{3.419261in}{0.867642in}}%
\pgfpathlineto{\pgfqpoint{3.433231in}{0.851335in}}%
\pgfpathlineto{\pgfqpoint{3.443585in}{0.835613in}}%
\pgfpathlineto{\pgfqpoint{3.450070in}{0.820511in}}%
\pgfpathlineto{\pgfqpoint{3.453223in}{0.806062in}}%
\pgfpathlineto{\pgfqpoint{3.453802in}{0.792284in}}%
\pgfpathlineto{\pgfqpoint{3.452053in}{0.779186in}}%
\pgfpathlineto{\pgfqpoint{3.448161in}{0.766778in}}%
\pgfpathlineto{\pgfqpoint{3.442250in}{0.755065in}}%
\pgfpathlineto{\pgfqpoint{3.434382in}{0.744051in}}%
\pgfpathlineto{\pgfqpoint{3.424558in}{0.733735in}}%
\pgfpathlineto{\pgfqpoint{3.412715in}{0.724117in}}%
\pgfpathlineto{\pgfqpoint{3.398794in}{0.715195in}}%
\pgfpathlineto{\pgfqpoint{3.382920in}{0.706970in}}%
\pgfpathlineto{\pgfqpoint{3.355513in}{0.695947in}}%
\pgfpathlineto{\pgfqpoint{3.323756in}{0.686509in}}%
\pgfpathlineto{\pgfqpoint{3.287510in}{0.678665in}}%
\pgfpathlineto{\pgfqpoint{3.246515in}{0.672428in}}%
\pgfpathlineto{\pgfqpoint{3.200398in}{0.667810in}}%
\pgfpathlineto{\pgfqpoint{3.148670in}{0.664825in}}%
\pgfpathlineto{\pgfqpoint{3.090934in}{0.663489in}}%
\pgfpathlineto{\pgfqpoint{3.026898in}{0.663854in}}%
\pgfpathlineto{\pgfqpoint{2.929713in}{0.667138in}}%
\pgfpathlineto{\pgfqpoint{2.817120in}{0.673815in}}%
\pgfpathlineto{\pgfqpoint{2.687910in}{0.684097in}}%
\pgfpathlineto{\pgfqpoint{2.541821in}{0.698198in}}%
\pgfpathlineto{\pgfqpoint{2.379533in}{0.716330in}}%
\pgfpathlineto{\pgfqpoint{2.202658in}{0.738714in}}%
\pgfpathlineto{\pgfqpoint{2.063534in}{0.758376in}}%
\pgfpathlineto{\pgfqpoint{1.924735in}{0.780301in}}%
\pgfpathlineto{\pgfqpoint{1.791210in}{0.804212in}}%
\pgfpathlineto{\pgfqpoint{1.707175in}{0.821096in}}%
\pgfpathlineto{\pgfqpoint{1.628313in}{0.838610in}}%
\pgfpathlineto{\pgfqpoint{1.555449in}{0.856635in}}%
\pgfpathlineto{\pgfqpoint{1.489238in}{0.875042in}}%
\pgfpathlineto{\pgfqpoint{1.430164in}{0.893692in}}%
\pgfpathlineto{\pgfqpoint{1.378539in}{0.912438in}}%
\pgfpathlineto{\pgfqpoint{1.334352in}{0.931121in}}%
\pgfpathlineto{\pgfqpoint{1.296767in}{0.949618in}}%
\pgfpathlineto{\pgfqpoint{1.264880in}{0.967842in}}%
\pgfpathlineto{\pgfqpoint{1.237974in}{0.985713in}}%
\pgfpathlineto{\pgfqpoint{1.215511in}{1.003161in}}%
\pgfpathlineto{\pgfqpoint{1.197142in}{1.020124in}}%
\pgfpathlineto{\pgfqpoint{1.182700in}{1.036550in}}%
\pgfpathlineto{\pgfqpoint{1.172199in}{1.052392in}}%
\pgfpathlineto{\pgfqpoint{1.165404in}{1.067615in}}%
\pgfpathlineto{\pgfqpoint{1.161500in}{1.082190in}}%
\pgfpathlineto{\pgfqpoint{1.160171in}{1.096103in}}%
\pgfpathlineto{\pgfqpoint{1.161181in}{1.109340in}}%
\pgfpathlineto{\pgfqpoint{1.164365in}{1.121894in}}%
\pgfpathlineto{\pgfqpoint{1.169627in}{1.133756in}}%
\pgfpathlineto{\pgfqpoint{1.176941in}{1.144924in}}%
\pgfpathlineto{\pgfqpoint{1.186350in}{1.155397in}}%
\pgfpathlineto{\pgfqpoint{1.197864in}{1.165177in}}%
\pgfpathlineto{\pgfqpoint{1.211339in}{1.174259in}}%
\pgfpathlineto{\pgfqpoint{1.226721in}{1.182642in}}%
\pgfpathlineto{\pgfqpoint{1.253323in}{1.193900in}}%
\pgfpathlineto{\pgfqpoint{1.284198in}{1.203573in}}%
\pgfpathlineto{\pgfqpoint{1.319508in}{1.211653in}}%
\pgfpathlineto{\pgfqpoint{1.359550in}{1.218136in}}%
\pgfpathlineto{\pgfqpoint{1.404755in}{1.223017in}}%
\pgfpathlineto{\pgfqpoint{1.455534in}{1.226288in}}%
\pgfpathlineto{\pgfqpoint{1.512084in}{1.227904in}}%
\pgfpathlineto{\pgfqpoint{1.575246in}{1.227806in}}%
\pgfpathlineto{\pgfqpoint{1.645816in}{1.225930in}}%
\pgfpathlineto{\pgfqpoint{1.752600in}{1.220532in}}%
\pgfpathlineto{\pgfqpoint{1.875070in}{1.211638in}}%
\pgfpathlineto{\pgfqpoint{2.014186in}{1.199023in}}%
\pgfpathlineto{\pgfqpoint{2.170587in}{1.182429in}}%
\pgfpathlineto{\pgfqpoint{2.345046in}{1.161560in}}%
\pgfpathlineto{\pgfqpoint{2.482794in}{1.143084in}}%
\pgfpathlineto{\pgfqpoint{2.621233in}{1.122305in}}%
\pgfpathlineto{\pgfqpoint{2.756185in}{1.099410in}}%
\pgfpathlineto{\pgfqpoint{2.883954in}{1.074663in}}%
\pgfpathlineto{\pgfqpoint{2.963538in}{1.057298in}}%
\pgfpathlineto{\pgfqpoint{3.037661in}{1.039377in}}%
\pgfpathlineto{\pgfqpoint{3.105585in}{1.021033in}}%
\pgfpathlineto{\pgfqpoint{3.166669in}{1.002414in}}%
\pgfpathlineto{\pgfqpoint{3.220369in}{0.983682in}}%
\pgfpathlineto{\pgfqpoint{3.266504in}{0.964999in}}%
\pgfpathlineto{\pgfqpoint{3.306016in}{0.946465in}}%
\pgfpathlineto{\pgfqpoint{3.339548in}{0.928178in}}%
\pgfpathlineto{\pgfqpoint{3.367647in}{0.910230in}}%
\pgfpathlineto{\pgfqpoint{3.390804in}{0.892698in}}%
\pgfpathlineto{\pgfqpoint{3.409455in}{0.875647in}}%
\pgfpathlineto{\pgfqpoint{3.424058in}{0.859133in}}%
\pgfpathlineto{\pgfqpoint{3.435091in}{0.843195in}}%
\pgfpathlineto{\pgfqpoint{3.442926in}{0.827864in}}%
\pgfpathlineto{\pgfqpoint{3.447862in}{0.813167in}}%
\pgfpathlineto{\pgfqpoint{3.450121in}{0.799123in}}%
\pgfpathlineto{\pgfqpoint{3.449845in}{0.785747in}}%
\pgfpathlineto{\pgfqpoint{3.447125in}{0.773048in}}%
\pgfpathlineto{\pgfqpoint{3.442170in}{0.761034in}}%
\pgfpathlineto{\pgfqpoint{3.435171in}{0.749710in}}%
\pgfpathlineto{\pgfqpoint{3.426264in}{0.739083in}}%
\pgfpathlineto{\pgfqpoint{3.415533in}{0.729153in}}%
\pgfpathlineto{\pgfqpoint{3.403010in}{0.719923in}}%
\pgfpathlineto{\pgfqpoint{3.388677in}{0.711392in}}%
\pgfpathlineto{\pgfqpoint{3.372461in}{0.703558in}}%
\pgfpathlineto{\pgfqpoint{3.344320in}{0.693103in}}%
\pgfpathlineto{\pgfqpoint{3.311265in}{0.684197in}}%
\pgfpathlineto{\pgfqpoint{3.273527in}{0.676868in}}%
\pgfpathlineto{\pgfqpoint{3.230871in}{0.671136in}}%
\pgfpathlineto{\pgfqpoint{3.182939in}{0.667026in}}%
\pgfpathlineto{\pgfqpoint{3.129297in}{0.664570in}}%
\pgfpathlineto{\pgfqpoint{3.069427in}{0.663810in}}%
\pgfpathlineto{\pgfqpoint{3.002732in}{0.664792in}}%
\pgfpathlineto{\pgfqpoint{2.902017in}{0.668907in}}%
\pgfpathlineto{\pgfqpoint{2.786363in}{0.676405in}}%
\pgfpathlineto{\pgfqpoint{2.653867in}{0.687548in}}%
\pgfpathlineto{\pgfqpoint{2.503903in}{0.702581in}}%
\pgfpathlineto{\pgfqpoint{2.337559in}{0.721720in}}%
\pgfpathlineto{\pgfqpoint{2.157779in}{0.745130in}}%
\pgfpathlineto{\pgfqpoint{2.018905in}{0.765455in}}%
\pgfpathlineto{\pgfqpoint{1.881743in}{0.787980in}}%
\pgfpathlineto{\pgfqpoint{1.750691in}{0.812448in}}%
\pgfpathlineto{\pgfqpoint{1.668773in}{0.829671in}}%
\pgfpathlineto{\pgfqpoint{1.592577in}{0.847478in}}%
\pgfpathlineto{\pgfqpoint{1.523209in}{0.865724in}}%
\pgfpathlineto{\pgfqpoint{1.461239in}{0.884255in}}%
\pgfpathlineto{\pgfqpoint{1.406412in}{0.902929in}}%
\pgfpathlineto{\pgfqpoint{1.358384in}{0.921615in}}%
\pgfpathlineto{\pgfqpoint{1.316795in}{0.940195in}}%
\pgfpathlineto{\pgfqpoint{1.281272in}{0.958561in}}%
\pgfpathlineto{\pgfqpoint{1.251424in}{0.976617in}}%
\pgfpathlineto{\pgfqpoint{1.226846in}{0.994279in}}%
\pgfpathlineto{\pgfqpoint{1.207112in}{1.011475in}}%
\pgfpathlineto{\pgfqpoint{1.191652in}{1.028143in}}%
\pgfpathlineto{\pgfqpoint{1.179853in}{1.044243in}}%
\pgfpathlineto{\pgfqpoint{1.171229in}{1.059745in}}%
\pgfpathlineto{\pgfqpoint{1.165415in}{1.074621in}}%
\pgfpathlineto{\pgfqpoint{1.162171in}{1.088852in}}%
\pgfpathlineto{\pgfqpoint{1.161382in}{1.102421in}}%
\pgfpathlineto{\pgfqpoint{1.163059in}{1.115317in}}%
\pgfpathlineto{\pgfqpoint{1.167287in}{1.127533in}}%
\pgfpathlineto{\pgfqpoint{1.173772in}{1.139061in}}%
\pgfpathlineto{\pgfqpoint{1.182305in}{1.149894in}}%
\pgfpathlineto{\pgfqpoint{1.192777in}{1.160029in}}%
\pgfpathlineto{\pgfqpoint{1.205113in}{1.169464in}}%
\pgfpathlineto{\pgfqpoint{1.219279in}{1.178197in}}%
\pgfpathlineto{\pgfqpoint{1.235278in}{1.186228in}}%
\pgfpathlineto{\pgfqpoint{1.262809in}{1.196961in}}%
\pgfpathlineto{\pgfqpoint{1.294852in}{1.206123in}}%
\pgfpathlineto{\pgfqpoint{1.331767in}{1.213722in}}%
\pgfpathlineto{\pgfqpoint{1.373545in}{1.219732in}}%
\pgfpathlineto{\pgfqpoint{1.420530in}{1.224130in}}%
\pgfpathlineto{\pgfqpoint{1.473170in}{1.226883in}}%
\pgfpathlineto{\pgfqpoint{1.531978in}{1.227950in}}%
\pgfpathlineto{\pgfqpoint{1.597525in}{1.227280in}}%
\pgfpathlineto{\pgfqpoint{1.696511in}{1.223578in}}%
\pgfpathlineto{\pgfqpoint{1.810282in}{1.216510in}}%
\pgfpathlineto{\pgfqpoint{1.940469in}{1.205852in}}%
\pgfpathlineto{\pgfqpoint{2.088147in}{1.191335in}}%
\pgfpathlineto{\pgfqpoint{2.252637in}{1.172728in}}%
\pgfpathlineto{\pgfqpoint{2.430881in}{1.149871in}}%
\pgfpathlineto{\pgfqpoint{2.569379in}{1.129951in}}%
\pgfpathlineto{\pgfqpoint{2.707463in}{1.107784in}}%
\pgfpathlineto{\pgfqpoint{2.840343in}{1.083606in}}%
\pgfpathlineto{\pgfqpoint{2.923402in}{1.066517in}}%
\pgfpathlineto{\pgfqpoint{3.000718in}{1.048804in}}%
\pgfpathlineto{\pgfqpoint{3.071666in}{1.030632in}}%
\pgfpathlineto{\pgfqpoint{3.135817in}{1.012154in}}%
\pgfpathlineto{\pgfqpoint{3.192941in}{0.993512in}}%
\pgfpathlineto{\pgfqpoint{3.243004in}{0.974835in}}%
\pgfpathlineto{\pgfqpoint{3.286170in}{0.956242in}}%
\pgfpathlineto{\pgfqpoint{3.322799in}{0.937839in}}%
\pgfpathlineto{\pgfqpoint{3.353448in}{0.919721in}}%
\pgfpathlineto{\pgfqpoint{3.378874in}{0.901970in}}%
\pgfpathlineto{\pgfqpoint{3.399589in}{0.884670in}}%
\pgfpathlineto{\pgfqpoint{3.415792in}{0.867889in}}%
\pgfpathlineto{\pgfqpoint{3.428323in}{0.851663in}}%
\pgfpathlineto{\pgfqpoint{3.437843in}{0.836020in}}%
\pgfpathlineto{\pgfqpoint{3.444816in}{0.820985in}}%
\pgfpathlineto{\pgfqpoint{3.449509in}{0.806580in}}%
\pgfpathlineto{\pgfqpoint{3.451992in}{0.792821in}}%
\pgfpathlineto{\pgfqpoint{3.452138in}{0.779721in}}%
\pgfpathlineto{\pgfqpoint{3.449624in}{0.767287in}}%
\pgfpathlineto{\pgfqpoint{3.443980in}{0.755526in}}%
\pgfpathlineto{\pgfqpoint{3.435865in}{0.744456in}}%
\pgfpathlineto{\pgfqpoint{3.425762in}{0.734088in}}%
\pgfpathlineto{\pgfqpoint{3.413752in}{0.724422in}}%
\pgfpathlineto{\pgfqpoint{3.399883in}{0.715462in}}%
\pgfpathlineto{\pgfqpoint{3.384173in}{0.707207in}}%
\pgfpathlineto{\pgfqpoint{3.357118in}{0.696147in}}%
\pgfpathlineto{\pgfqpoint{3.325703in}{0.686671in}}%
\pgfpathlineto{\pgfqpoint{3.289589in}{0.678774in}}%
\pgfpathlineto{\pgfqpoint{3.248582in}{0.672461in}}%
\pgfpathlineto{\pgfqpoint{3.202480in}{0.667759in}}%
\pgfpathlineto{\pgfqpoint{3.150810in}{0.664696in}}%
\pgfpathlineto{\pgfqpoint{3.093049in}{0.663313in}}%
\pgfpathlineto{\pgfqpoint{3.028628in}{0.663660in}}%
\pgfpathlineto{\pgfqpoint{2.931300in}{0.666920in}}%
\pgfpathlineto{\pgfqpoint{2.819433in}{0.673534in}}%
\pgfpathlineto{\pgfqpoint{2.691343in}{0.683708in}}%
\pgfpathlineto{\pgfqpoint{2.545829in}{0.697709in}}%
\pgfpathlineto{\pgfqpoint{2.383199in}{0.715782in}}%
\pgfpathlineto{\pgfqpoint{2.206254in}{0.738095in}}%
\pgfpathlineto{\pgfqpoint{2.067884in}{0.757624in}}%
\pgfpathlineto{\pgfqpoint{1.929227in}{0.779434in}}%
\pgfpathlineto{\pgfqpoint{1.794921in}{0.803307in}}%
\pgfpathlineto{\pgfqpoint{1.710293in}{0.820212in}}%
\pgfpathlineto{\pgfqpoint{1.631328in}{0.837787in}}%
\pgfpathlineto{\pgfqpoint{1.558706in}{0.855874in}}%
\pgfpathlineto{\pgfqpoint{1.492816in}{0.874310in}}%
\pgfpathlineto{\pgfqpoint{1.433879in}{0.892949in}}%
\pgfpathlineto{\pgfqpoint{1.381948in}{0.911653in}}%
\pgfpathlineto{\pgfqpoint{1.336906in}{0.930302in}}%
\pgfpathlineto{\pgfqpoint{1.298471in}{0.948783in}}%
\pgfpathlineto{\pgfqpoint{1.266190in}{0.966999in}}%
\pgfpathlineto{\pgfqpoint{1.239443in}{0.984866in}}%
\pgfpathlineto{\pgfqpoint{1.217456in}{1.002310in}}%
\pgfpathlineto{\pgfqpoint{1.200048in}{1.019251in}}%
\pgfpathlineto{\pgfqpoint{1.186708in}{1.035641in}}%
\pgfpathlineto{\pgfqpoint{1.176662in}{1.051450in}}%
\pgfpathlineto{\pgfqpoint{1.169327in}{1.066654in}}%
\pgfpathlineto{\pgfqpoint{1.164309in}{1.081231in}}%
\pgfpathlineto{\pgfqpoint{1.161405in}{1.095164in}}%
\pgfpathlineto{\pgfqpoint{1.160606in}{1.108441in}}%
\pgfpathlineto{\pgfqpoint{1.162091in}{1.121053in}}%
\pgfpathlineto{\pgfqpoint{1.166229in}{1.132993in}}%
\pgfpathlineto{\pgfqpoint{1.173478in}{1.144260in}}%
\pgfpathlineto{\pgfqpoint{1.183058in}{1.154832in}}%
\pgfpathlineto{\pgfqpoint{1.194586in}{1.164701in}}%
\pgfpathlineto{\pgfqpoint{1.208001in}{1.173865in}}%
\pgfpathlineto{\pgfqpoint{1.223270in}{1.182324in}}%
\pgfpathlineto{\pgfqpoint{1.249649in}{1.193688in}}%
\pgfpathlineto{\pgfqpoint{1.280309in}{1.203462in}}%
\pgfpathlineto{\pgfqpoint{1.315509in}{1.211650in}}%
\pgfpathlineto{\pgfqpoint{1.355621in}{1.218254in}}%
\pgfpathlineto{\pgfqpoint{1.400763in}{1.223256in}}%
\pgfpathlineto{\pgfqpoint{1.451352in}{1.226626in}}%
\pgfpathlineto{\pgfqpoint{1.507942in}{1.228325in}}%
\pgfpathlineto{\pgfqpoint{1.571116in}{1.228306in}}%
\pgfpathlineto{\pgfqpoint{1.641485in}{1.226508in}}%
\pgfpathlineto{\pgfqpoint{1.747613in}{1.221221in}}%
\pgfpathlineto{\pgfqpoint{1.869269in}{1.212449in}}%
\pgfpathlineto{\pgfqpoint{2.007964in}{1.199963in}}%
\pgfpathlineto{\pgfqpoint{2.164195in}{1.183489in}}%
\pgfpathlineto{\pgfqpoint{2.336199in}{1.162821in}}%
\pgfpathlineto{\pgfqpoint{2.472680in}{1.144521in}}%
\pgfpathlineto{\pgfqpoint{2.611970in}{1.123857in}}%
\pgfpathlineto{\pgfqpoint{2.749171in}{1.101004in}}%
\pgfpathlineto{\pgfqpoint{2.879432in}{1.076246in}}%
\pgfpathlineto{\pgfqpoint{2.960317in}{1.058860in}}%
\pgfpathlineto{\pgfqpoint{3.035156in}{1.040921in}}%
\pgfpathlineto{\pgfqpoint{3.102923in}{1.022578in}}%
\pgfpathlineto{\pgfqpoint{3.163239in}{1.003985in}}%
\pgfpathlineto{\pgfqpoint{3.216535in}{0.985279in}}%
\pgfpathlineto{\pgfqpoint{3.263238in}{0.966585in}}%
\pgfpathlineto{\pgfqpoint{3.303747in}{0.948018in}}%
\pgfpathlineto{\pgfqpoint{3.338431in}{0.929682in}}%
\pgfpathlineto{\pgfqpoint{3.367632in}{0.911668in}}%
\pgfpathlineto{\pgfqpoint{3.391660in}{0.894059in}}%
\pgfpathlineto{\pgfqpoint{3.410798in}{0.876923in}}%
\pgfpathlineto{\pgfqpoint{3.425409in}{0.860323in}}%
\pgfpathlineto{\pgfqpoint{3.436243in}{0.844308in}}%
\pgfpathlineto{\pgfqpoint{3.443896in}{0.828904in}}%
\pgfpathlineto{\pgfqpoint{3.448824in}{0.814134in}}%
\pgfpathlineto{\pgfqpoint{3.451354in}{0.800017in}}%
\pgfpathlineto{\pgfqpoint{3.451680in}{0.786565in}}%
\pgfpathlineto{\pgfqpoint{3.449868in}{0.773789in}}%
\pgfpathlineto{\pgfqpoint{3.445850in}{0.761696in}}%
\pgfpathlineto{\pgfqpoint{3.439431in}{0.750286in}}%
\pgfpathlineto{\pgfqpoint{3.430501in}{0.739563in}}%
\pgfpathlineto{\pgfqpoint{3.419538in}{0.729537in}}%
\pgfpathlineto{\pgfqpoint{3.406652in}{0.720213in}}%
\pgfpathlineto{\pgfqpoint{3.391891in}{0.711591in}}%
\pgfpathlineto{\pgfqpoint{3.366266in}{0.699979in}}%
\pgfpathlineto{\pgfqpoint{3.336401in}{0.689954in}}%
\pgfpathlineto{\pgfqpoint{3.302093in}{0.681516in}}%
\pgfpathlineto{\pgfqpoint{3.262997in}{0.674665in}}%
\pgfpathlineto{\pgfqpoint{3.218759in}{0.669403in}}%
\pgfpathlineto{\pgfqpoint{3.169227in}{0.665762in}}%
\pgfpathlineto{\pgfqpoint{3.113811in}{0.663780in}}%
\pgfpathlineto{\pgfqpoint{3.051888in}{0.663504in}}%
\pgfpathlineto{\pgfqpoint{2.982829in}{0.664994in}}%
\pgfpathlineto{\pgfqpoint{2.878569in}{0.669846in}}%
\pgfpathlineto{\pgfqpoint{2.759017in}{0.678157in}}%
\pgfpathlineto{\pgfqpoint{2.622680in}{0.690153in}}%
\pgfpathlineto{\pgfqpoint{2.468782in}{0.706081in}}%
\pgfpathlineto{\pgfqpoint{2.298701in}{0.726201in}}%
\pgfpathlineto{\pgfqpoint{2.163158in}{0.744097in}}%
\pgfpathlineto{\pgfqpoint{2.024281in}{0.764347in}}%
\pgfpathlineto{\pgfqpoint{1.886428in}{0.786822in}}%
\pgfpathlineto{\pgfqpoint{1.754640in}{0.811303in}}%
\pgfpathlineto{\pgfqpoint{1.672466in}{0.828530in}}%
\pgfpathlineto{\pgfqpoint{1.596094in}{0.846324in}}%
\pgfpathlineto{\pgfqpoint{1.526281in}{0.864550in}}%
\pgfpathlineto{\pgfqpoint{1.463506in}{0.883072in}}%
\pgfpathlineto{\pgfqpoint{1.407966in}{0.901757in}}%
\pgfpathlineto{\pgfqpoint{1.359578in}{0.920472in}}%
\pgfpathlineto{\pgfqpoint{1.318101in}{0.939080in}}%
\pgfpathlineto{\pgfqpoint{1.283326in}{0.957440in}}%
\pgfpathlineto{\pgfqpoint{1.254138in}{0.975479in}}%
\pgfpathlineto{\pgfqpoint{1.229553in}{0.993138in}}%
\pgfpathlineto{\pgfqpoint{1.208855in}{1.010359in}}%
\pgfpathlineto{\pgfqpoint{1.191591in}{1.027092in}}%
\pgfpathlineto{\pgfqpoint{1.177571in}{1.043291in}}%
\pgfpathlineto{\pgfqpoint{1.166872in}{1.058912in}}%
\pgfpathlineto{\pgfqpoint{1.159836in}{1.073919in}}%
\pgfpathlineto{\pgfqpoint{1.156909in}{1.088280in}}%
\pgfpathlineto{\pgfqpoint{1.156913in}{1.101965in}}%
\pgfpathlineto{\pgfqpoint{1.159249in}{1.114962in}}%
\pgfpathlineto{\pgfqpoint{1.163730in}{1.127265in}}%
\pgfpathlineto{\pgfqpoint{1.170219in}{1.138869in}}%
\pgfpathlineto{\pgfqpoint{1.178633in}{1.149771in}}%
\pgfpathlineto{\pgfqpoint{1.188942in}{1.159970in}}%
\pgfpathlineto{\pgfqpoint{1.201169in}{1.169466in}}%
\pgfpathlineto{\pgfqpoint{1.215388in}{1.178262in}}%
\pgfpathlineto{\pgfqpoint{1.231649in}{1.186362in}}%
\pgfpathlineto{\pgfqpoint{1.259678in}{1.197198in}}%
\pgfpathlineto{\pgfqpoint{1.292106in}{1.206451in}}%
\pgfpathlineto{\pgfqpoint{1.329078in}{1.214108in}}%
\pgfpathlineto{\pgfqpoint{1.370844in}{1.220155in}}%
\pgfpathlineto{\pgfqpoint{1.417758in}{1.224573in}}%
\pgfpathlineto{\pgfqpoint{1.470283in}{1.227343in}}%
\pgfpathlineto{\pgfqpoint{1.528987in}{1.228440in}}%
\pgfpathlineto{\pgfqpoint{1.594103in}{1.227850in}}%
\pgfpathlineto{\pgfqpoint{1.692249in}{1.224297in}}%
\pgfpathlineto{\pgfqpoint{1.806117in}{1.217312in}}%
\pgfpathlineto{\pgfqpoint{1.937299in}{1.206645in}}%
\pgfpathlineto{\pgfqpoint{2.085830in}{1.192088in}}%
\pgfpathlineto{\pgfqpoint{2.250193in}{1.173477in}}%
\pgfpathlineto{\pgfqpoint{2.427318in}{1.150693in}}%
\pgfpathlineto{\pgfqpoint{2.565742in}{1.130819in}}%
\pgfpathlineto{\pgfqpoint{2.704291in}{1.108627in}}%
\pgfpathlineto{\pgfqpoint{2.837374in}{1.084438in}}%
\pgfpathlineto{\pgfqpoint{2.920805in}{1.067389in}}%
\pgfpathlineto{\pgfqpoint{2.998756in}{1.049741in}}%
\pgfpathlineto{\pgfqpoint{3.070413in}{1.031618in}}%
\pgfpathlineto{\pgfqpoint{3.135187in}{1.013150in}}%
\pgfpathlineto{\pgfqpoint{3.192718in}{0.994473in}}%
\pgfpathlineto{\pgfqpoint{3.242870in}{0.975729in}}%
\pgfpathlineto{\pgfqpoint{3.285733in}{0.957072in}}%
\pgfpathlineto{\pgfqpoint{3.322074in}{0.938624in}}%
\pgfpathlineto{\pgfqpoint{3.352865in}{0.920467in}}%
\pgfpathlineto{\pgfqpoint{3.378866in}{0.902674in}}%
\pgfpathlineto{\pgfqpoint{3.400628in}{0.885312in}}%
\pgfpathlineto{\pgfqpoint{3.418489in}{0.868440in}}%
\pgfpathlineto{\pgfqpoint{3.432576in}{0.852110in}}%
\pgfpathlineto{\pgfqpoint{3.442807in}{0.836365in}}%
\pgfpathlineto{\pgfqpoint{3.449100in}{0.821243in}}%
\pgfpathlineto{\pgfqpoint{3.452410in}{0.806774in}}%
\pgfpathlineto{\pgfqpoint{3.453166in}{0.792973in}}%
\pgfpathlineto{\pgfqpoint{3.451612in}{0.779851in}}%
\pgfpathlineto{\pgfqpoint{3.447923in}{0.767417in}}%
\pgfpathlineto{\pgfqpoint{3.442208in}{0.755677in}}%
\pgfpathlineto{\pgfqpoint{3.434508in}{0.744633in}}%
\pgfpathlineto{\pgfqpoint{3.424798in}{0.734286in}}%
\pgfpathlineto{\pgfqpoint{3.412988in}{0.724634in}}%
\pgfpathlineto{\pgfqpoint{3.399145in}{0.715678in}}%
\pgfpathlineto{\pgfqpoint{3.383375in}{0.707421in}}%
\pgfpathlineto{\pgfqpoint{3.356148in}{0.696351in}}%
\pgfpathlineto{\pgfqpoint{3.324599in}{0.686866in}}%
\pgfpathlineto{\pgfqpoint{3.288579in}{0.678976in}}%
\pgfpathlineto{\pgfqpoint{3.247813in}{0.672689in}}%
\pgfpathlineto{\pgfqpoint{3.201907in}{0.668016in}}%
\pgfpathlineto{\pgfqpoint{3.150348in}{0.664967in}}%
\pgfpathlineto{\pgfqpoint{3.092971in}{0.663566in}}%
\pgfpathlineto{\pgfqpoint{3.029079in}{0.663878in}}%
\pgfpathlineto{\pgfqpoint{2.932038in}{0.667092in}}%
\pgfpathlineto{\pgfqpoint{2.819829in}{0.673681in}}%
\pgfpathlineto{\pgfqpoint{2.691307in}{0.683852in}}%
\pgfpathlineto{\pgfqpoint{2.545996in}{0.697828in}}%
\pgfpathlineto{\pgfqpoint{2.384090in}{0.715847in}}%
\pgfpathlineto{\pgfqpoint{2.206407in}{0.738182in}}%
\pgfpathlineto{\pgfqpoint{2.067203in}{0.757776in}}%
\pgfpathlineto{\pgfqpoint{1.928767in}{0.779606in}}%
\pgfpathlineto{\pgfqpoint{1.795648in}{0.803424in}}%
\pgfpathlineto{\pgfqpoint{1.711776in}{0.820257in}}%
\pgfpathlineto{\pgfqpoint{1.632935in}{0.837729in}}%
\pgfpathlineto{\pgfqpoint{1.559939in}{0.855725in}}%
\pgfpathlineto{\pgfqpoint{1.493454in}{0.874112in}}%
\pgfpathlineto{\pgfqpoint{1.434005in}{0.892749in}}%
\pgfpathlineto{\pgfqpoint{1.381973in}{0.911483in}}%
\pgfpathlineto{\pgfqpoint{1.337319in}{0.930158in}}%
\pgfpathlineto{\pgfqpoint{1.299190in}{0.948660in}}%
\pgfpathlineto{\pgfqpoint{1.266830in}{0.966895in}}%
\pgfpathlineto{\pgfqpoint{1.239618in}{0.984780in}}%
\pgfpathlineto{\pgfqpoint{1.217068in}{1.002242in}}%
\pgfpathlineto{\pgfqpoint{1.198826in}{1.019216in}}%
\pgfpathlineto{\pgfqpoint{1.184672in}{1.035649in}}%
\pgfpathlineto{\pgfqpoint{1.174377in}{1.051496in}}%
\pgfpathlineto{\pgfqpoint{1.167300in}{1.066725in}}%
\pgfpathlineto{\pgfqpoint{1.163067in}{1.081313in}}%
\pgfpathlineto{\pgfqpoint{1.161403in}{1.095244in}}%
\pgfpathlineto{\pgfqpoint{1.162109in}{1.108503in}}%
\pgfpathlineto{\pgfqpoint{1.165057in}{1.121081in}}%
\pgfpathlineto{\pgfqpoint{1.170194in}{1.132970in}}%
\pgfpathlineto{\pgfqpoint{1.177529in}{1.144170in}}%
\pgfpathlineto{\pgfqpoint{1.186960in}{1.154677in}}%
\pgfpathlineto{\pgfqpoint{1.198367in}{1.164488in}}%
\pgfpathlineto{\pgfqpoint{1.211670in}{1.173600in}}%
\pgfpathlineto{\pgfqpoint{1.226825in}{1.182011in}}%
\pgfpathlineto{\pgfqpoint{1.253005in}{1.193309in}}%
\pgfpathlineto{\pgfqpoint{1.283417in}{1.203024in}}%
\pgfpathlineto{\pgfqpoint{1.318315in}{1.211157in}}%
\pgfpathlineto{\pgfqpoint{1.358118in}{1.217712in}}%
\pgfpathlineto{\pgfqpoint{1.403175in}{1.222684in}}%
\pgfpathlineto{\pgfqpoint{1.453613in}{1.226031in}}%
\pgfpathlineto{\pgfqpoint{1.510037in}{1.227715in}}%
\pgfpathlineto{\pgfqpoint{1.573073in}{1.227686in}}%
\pgfpathlineto{\pgfqpoint{1.643346in}{1.225884in}}%
\pgfpathlineto{\pgfqpoint{1.749388in}{1.220596in}}%
\pgfpathlineto{\pgfqpoint{1.870906in}{1.211827in}}%
\pgfpathlineto{\pgfqpoint{2.009372in}{1.199345in}}%
\pgfpathlineto{\pgfqpoint{2.165336in}{1.182899in}}%
\pgfpathlineto{\pgfqpoint{2.337072in}{1.162244in}}%
\pgfpathlineto{\pgfqpoint{2.473317in}{1.143953in}}%
\pgfpathlineto{\pgfqpoint{2.612222in}{1.123328in}}%
\pgfpathlineto{\pgfqpoint{2.749268in}{1.100504in}}%
\pgfpathlineto{\pgfqpoint{2.837101in}{1.084185in}}%
\pgfpathlineto{\pgfqpoint{2.920476in}{1.067132in}}%
\pgfpathlineto{\pgfqpoint{2.998286in}{1.049479in}}%
\pgfpathlineto{\pgfqpoint{3.069703in}{1.031362in}}%
\pgfpathlineto{\pgfqpoint{3.134182in}{1.012913in}}%
\pgfpathlineto{\pgfqpoint{3.191457in}{0.994268in}}%
\pgfpathlineto{\pgfqpoint{3.241541in}{0.975559in}}%
\pgfpathlineto{\pgfqpoint{3.284637in}{0.956927in}}%
\pgfpathlineto{\pgfqpoint{3.320887in}{0.938517in}}%
\pgfpathlineto{\pgfqpoint{3.351380in}{0.920407in}}%
\pgfpathlineto{\pgfqpoint{3.377122in}{0.902659in}}%
\pgfpathlineto{\pgfqpoint{3.398855in}{0.885332in}}%
\pgfpathlineto{\pgfqpoint{3.417055in}{0.868481in}}%
\pgfpathlineto{\pgfqpoint{3.431937in}{0.852152in}}%
\pgfpathlineto{\pgfqpoint{3.443452in}{0.836391in}}%
\pgfpathlineto{\pgfqpoint{3.451286in}{0.821237in}}%
\pgfpathlineto{\pgfqpoint{3.454958in}{0.806723in}}%
\pgfpathlineto{\pgfqpoint{3.455548in}{0.792882in}}%
\pgfpathlineto{\pgfqpoint{3.453763in}{0.779725in}}%
\pgfpathlineto{\pgfqpoint{3.449801in}{0.767261in}}%
\pgfpathlineto{\pgfqpoint{3.443806in}{0.755494in}}%
\pgfpathlineto{\pgfqpoint{3.435871in}{0.744429in}}%
\pgfpathlineto{\pgfqpoint{3.426032in}{0.734066in}}%
\pgfpathlineto{\pgfqpoint{3.414271in}{0.724406in}}%
\pgfpathlineto{\pgfqpoint{3.400519in}{0.715445in}}%
\pgfpathlineto{\pgfqpoint{3.384721in}{0.707181in}}%
\pgfpathlineto{\pgfqpoint{3.357385in}{0.696097in}}%
\pgfpathlineto{\pgfqpoint{3.325669in}{0.686595in}}%
\pgfpathlineto{\pgfqpoint{3.289446in}{0.678688in}}%
\pgfpathlineto{\pgfqpoint{3.248479in}{0.672389in}}%
\pgfpathlineto{\pgfqpoint{3.202425in}{0.667714in}}%
\pgfpathlineto{\pgfqpoint{3.150834in}{0.664683in}}%
\pgfpathlineto{\pgfqpoint{3.093145in}{0.663317in}}%
\pgfpathlineto{\pgfqpoint{3.029122in}{0.663633in}}%
\pgfpathlineto{\pgfqpoint{2.932609in}{0.666809in}}%
\pgfpathlineto{\pgfqpoint{2.820626in}{0.673384in}}%
\pgfpathlineto{\pgfqpoint{2.691535in}{0.683603in}}%
\pgfpathlineto{\pgfqpoint{2.545137in}{0.697677in}}%
\pgfpathlineto{\pgfqpoint{2.382677in}{0.715780in}}%
\pgfpathlineto{\pgfqpoint{2.206836in}{0.738053in}}%
\pgfpathlineto{\pgfqpoint{2.068694in}{0.757562in}}%
\pgfpathlineto{\pgfqpoint{1.929715in}{0.779410in}}%
\pgfpathlineto{\pgfqpoint{1.795492in}{0.803297in}}%
\pgfpathlineto{\pgfqpoint{1.710948in}{0.820179in}}%
\pgfpathlineto{\pgfqpoint{1.631649in}{0.837694in}}%
\pgfpathlineto{\pgfqpoint{1.558462in}{0.855717in}}%
\pgfpathlineto{\pgfqpoint{1.492040in}{0.874121in}}%
\pgfpathlineto{\pgfqpoint{1.432820in}{0.892766in}}%
\pgfpathlineto{\pgfqpoint{1.381025in}{0.911509in}}%
\pgfpathlineto{\pgfqpoint{1.336611in}{0.930190in}}%
\pgfpathlineto{\pgfqpoint{1.298771in}{0.948691in}}%
\pgfpathlineto{\pgfqpoint{1.266581in}{0.966925in}}%
\pgfpathlineto{\pgfqpoint{1.239316in}{0.984815in}}%
\pgfpathlineto{\pgfqpoint{1.216449in}{1.002288in}}%
\pgfpathlineto{\pgfqpoint{1.197652in}{1.019284in}}%
\pgfpathlineto{\pgfqpoint{1.182795in}{1.035748in}}%
\pgfpathlineto{\pgfqpoint{1.171948in}{1.051632in}}%
\pgfpathlineto{\pgfqpoint{1.165020in}{1.066898in}}%
\pgfpathlineto{\pgfqpoint{1.161051in}{1.081516in}}%
\pgfpathlineto{\pgfqpoint{1.159671in}{1.095470in}}%
\pgfpathlineto{\pgfqpoint{1.160629in}{1.108748in}}%
\pgfpathlineto{\pgfqpoint{1.163747in}{1.121342in}}%
\pgfpathlineto{\pgfqpoint{1.168918in}{1.133244in}}%
\pgfpathlineto{\pgfqpoint{1.176105in}{1.144450in}}%
\pgfpathlineto{\pgfqpoint{1.185348in}{1.154961in}}%
\pgfpathlineto{\pgfqpoint{1.196730in}{1.164779in}}%
\pgfpathlineto{\pgfqpoint{1.210121in}{1.173900in}}%
\pgfpathlineto{\pgfqpoint{1.225433in}{1.182322in}}%
\pgfpathlineto{\pgfqpoint{1.251952in}{1.193638in}}%
\pgfpathlineto{\pgfqpoint{1.282759in}{1.203370in}}%
\pgfpathlineto{\pgfqpoint{1.317999in}{1.211508in}}%
\pgfpathlineto{\pgfqpoint{1.357945in}{1.218046in}}%
\pgfpathlineto{\pgfqpoint{1.402998in}{1.222978in}}%
\pgfpathlineto{\pgfqpoint{1.453643in}{1.226295in}}%
\pgfpathlineto{\pgfqpoint{1.510012in}{1.227964in}}%
\pgfpathlineto{\pgfqpoint{1.572898in}{1.227923in}}%
\pgfpathlineto{\pgfqpoint{1.643183in}{1.226105in}}%
\pgfpathlineto{\pgfqpoint{1.749657in}{1.220784in}}%
\pgfpathlineto{\pgfqpoint{1.871903in}{1.211969in}}%
\pgfpathlineto{\pgfqpoint{2.010773in}{1.199435in}}%
\pgfpathlineto{\pgfqpoint{2.166649in}{1.182938in}}%
\pgfpathlineto{\pgfqpoint{2.339798in}{1.162191in}}%
\pgfpathlineto{\pgfqpoint{2.477471in}{1.143782in}}%
\pgfpathlineto{\pgfqpoint{2.616338in}{1.123057in}}%
\pgfpathlineto{\pgfqpoint{2.751970in}{1.100209in}}%
\pgfpathlineto{\pgfqpoint{2.880472in}{1.075504in}}%
\pgfpathlineto{\pgfqpoint{2.960500in}{1.058167in}}%
\pgfpathlineto{\pgfqpoint{3.034992in}{1.040271in}}%
\pgfpathlineto{\pgfqpoint{3.103192in}{1.021950in}}%
\pgfpathlineto{\pgfqpoint{3.164449in}{1.003349in}}%
\pgfpathlineto{\pgfqpoint{3.218216in}{0.984629in}}%
\pgfpathlineto{\pgfqpoint{3.264560in}{0.965942in}}%
\pgfpathlineto{\pgfqpoint{3.304330in}{0.947396in}}%
\pgfpathlineto{\pgfqpoint{3.338157in}{0.929092in}}%
\pgfpathlineto{\pgfqpoint{3.366590in}{0.911120in}}%
\pgfpathlineto{\pgfqpoint{3.390099in}{0.893558in}}%
\pgfpathlineto{\pgfqpoint{3.409074in}{0.876474in}}%
\pgfpathlineto{\pgfqpoint{3.423861in}{0.859924in}}%
\pgfpathlineto{\pgfqpoint{3.435000in}{0.843951in}}%
\pgfpathlineto{\pgfqpoint{3.442922in}{0.828585in}}%
\pgfpathlineto{\pgfqpoint{3.447956in}{0.813851in}}%
\pgfpathlineto{\pgfqpoint{3.450349in}{0.799770in}}%
\pgfpathlineto{\pgfqpoint{3.450270in}{0.786357in}}%
\pgfpathlineto{\pgfqpoint{3.447808in}{0.773622in}}%
\pgfpathlineto{\pgfqpoint{3.443026in}{0.761570in}}%
\pgfpathlineto{\pgfqpoint{3.436086in}{0.750208in}}%
\pgfpathlineto{\pgfqpoint{3.427141in}{0.739540in}}%
\pgfpathlineto{\pgfqpoint{3.416300in}{0.729569in}}%
\pgfpathlineto{\pgfqpoint{3.403633in}{0.720299in}}%
\pgfpathlineto{\pgfqpoint{3.389164in}{0.711730in}}%
\pgfpathlineto{\pgfqpoint{3.364035in}{0.700191in}}%
\pgfpathlineto{\pgfqpoint{3.334573in}{0.690223in}}%
\pgfpathlineto{\pgfqpoint{3.300330in}{0.681813in}}%
\pgfpathlineto{\pgfqpoint{3.261141in}{0.674971in}}%
\pgfpathlineto{\pgfqpoint{3.216957in}{0.669728in}}%
\pgfpathlineto{\pgfqpoint{3.167360in}{0.666110in}}%
\pgfpathlineto{\pgfqpoint{3.111873in}{0.664155in}}%
\pgfpathlineto{\pgfqpoint{3.049955in}{0.663908in}}%
\pgfpathlineto{\pgfqpoint{2.981005in}{0.665425in}}%
\pgfpathlineto{\pgfqpoint{2.876975in}{0.670305in}}%
\pgfpathlineto{\pgfqpoint{2.757557in}{0.678621in}}%
\pgfpathlineto{\pgfqpoint{2.621218in}{0.690623in}}%
\pgfpathlineto{\pgfqpoint{2.467402in}{0.706571in}}%
\pgfpathlineto{\pgfqpoint{2.297520in}{0.726678in}}%
\pgfpathlineto{\pgfqpoint{2.162033in}{0.744559in}}%
\pgfpathlineto{\pgfqpoint{2.023364in}{0.764805in}}%
\pgfpathlineto{\pgfqpoint{1.885741in}{0.787269in}}%
\pgfpathlineto{\pgfqpoint{1.754066in}{0.811703in}}%
\pgfpathlineto{\pgfqpoint{1.672187in}{0.828936in}}%
\pgfpathlineto{\pgfqpoint{1.596184in}{0.846757in}}%
\pgfpathlineto{\pgfqpoint{1.526651in}{0.865002in}}%
\pgfpathlineto{\pgfqpoint{1.463979in}{0.883522in}}%
\pgfpathlineto{\pgfqpoint{1.408360in}{0.902178in}}%
\pgfpathlineto{\pgfqpoint{1.359788in}{0.920843in}}%
\pgfpathlineto{\pgfqpoint{1.318057in}{0.939400in}}%
\pgfpathlineto{\pgfqpoint{1.282760in}{0.957747in}}%
\pgfpathlineto{\pgfqpoint{1.253292in}{0.975791in}}%
\pgfpathlineto{\pgfqpoint{1.228854in}{0.993452in}}%
\pgfpathlineto{\pgfqpoint{1.209184in}{1.010642in}}%
\pgfpathlineto{\pgfqpoint{1.193825in}{1.027305in}}%
\pgfpathlineto{\pgfqpoint{1.181967in}{1.043407in}}%
\pgfpathlineto{\pgfqpoint{1.172996in}{1.058921in}}%
\pgfpathlineto{\pgfqpoint{1.166498in}{1.073822in}}%
\pgfpathlineto{\pgfqpoint{1.162254in}{1.088090in}}%
\pgfpathlineto{\pgfqpoint{1.160245in}{1.101709in}}%
\pgfpathlineto{\pgfqpoint{1.160648in}{1.114667in}}%
\pgfpathlineto{\pgfqpoint{1.163839in}{1.126957in}}%
\pgfpathlineto{\pgfqpoint{1.170156in}{1.138572in}}%
\pgfpathlineto{\pgfqpoint{1.178704in}{1.149491in}}%
\pgfpathlineto{\pgfqpoint{1.189219in}{1.159707in}}%
\pgfpathlineto{\pgfqpoint{1.201626in}{1.169221in}}%
\pgfpathlineto{\pgfqpoint{1.215884in}{1.178029in}}%
\pgfpathlineto{\pgfqpoint{1.231983in}{1.186132in}}%
\pgfpathlineto{\pgfqpoint{1.259639in}{1.196964in}}%
\pgfpathlineto{\pgfqpoint{1.291699in}{1.206213in}}%
\pgfpathlineto{\pgfqpoint{1.328520in}{1.213885in}}%
\pgfpathlineto{\pgfqpoint{1.370225in}{1.219969in}}%
\pgfpathlineto{\pgfqpoint{1.417094in}{1.224439in}}%
\pgfpathlineto{\pgfqpoint{1.469603in}{1.227264in}}%
\pgfpathlineto{\pgfqpoint{1.528279in}{1.228402in}}%
\pgfpathlineto{\pgfqpoint{1.593698in}{1.227802in}}%
\pgfpathlineto{\pgfqpoint{1.692498in}{1.224191in}}%
\pgfpathlineto{\pgfqpoint{1.806021in}{1.217206in}}%
\pgfpathlineto{\pgfqpoint{1.935933in}{1.206635in}}%
\pgfpathlineto{\pgfqpoint{2.083338in}{1.192208in}}%
\pgfpathlineto{\pgfqpoint{2.247673in}{1.173688in}}%
\pgfpathlineto{\pgfqpoint{2.425855in}{1.150916in}}%
\pgfpathlineto{\pgfqpoint{2.564520in}{1.131052in}}%
\pgfpathlineto{\pgfqpoint{2.702853in}{1.108931in}}%
\pgfpathlineto{\pgfqpoint{2.836136in}{1.084789in}}%
\pgfpathlineto{\pgfqpoint{2.919620in}{1.067727in}}%
\pgfpathlineto{\pgfqpoint{2.997278in}{1.050027in}}%
\pgfpathlineto{\pgfqpoint{3.068524in}{1.031853in}}%
\pgfpathlineto{\pgfqpoint{3.132976in}{1.013363in}}%
\pgfpathlineto{\pgfqpoint{3.190429in}{0.994701in}}%
\pgfpathlineto{\pgfqpoint{3.240859in}{0.976001in}}%
\pgfpathlineto{\pgfqpoint{3.284416in}{0.957381in}}%
\pgfpathlineto{\pgfqpoint{3.321435in}{0.938949in}}%
\pgfpathlineto{\pgfqpoint{3.352424in}{0.920800in}}%
\pgfpathlineto{\pgfqpoint{3.378073in}{0.903017in}}%
\pgfpathlineto{\pgfqpoint{3.399078in}{0.885676in}}%
\pgfpathlineto{\pgfqpoint{3.415504in}{0.868853in}}%
\pgfpathlineto{\pgfqpoint{3.428115in}{0.852586in}}%
\pgfpathlineto{\pgfqpoint{3.437621in}{0.836904in}}%
\pgfpathlineto{\pgfqpoint{3.444541in}{0.821832in}}%
\pgfpathlineto{\pgfqpoint{3.449199in}{0.807389in}}%
\pgfpathlineto{\pgfqpoint{3.451726in}{0.793593in}}%
\pgfpathlineto{\pgfqpoint{3.452058in}{0.780455in}}%
\pgfpathlineto{\pgfqpoint{3.449937in}{0.767984in}}%
\pgfpathlineto{\pgfqpoint{3.444915in}{0.756184in}}%
\pgfpathlineto{\pgfqpoint{3.436943in}{0.745066in}}%
\pgfpathlineto{\pgfqpoint{3.426936in}{0.734648in}}%
\pgfpathlineto{\pgfqpoint{3.415005in}{0.724934in}}%
\pgfpathlineto{\pgfqpoint{3.401203in}{0.715925in}}%
\pgfpathlineto{\pgfqpoint{3.385556in}{0.707621in}}%
\pgfpathlineto{\pgfqpoint{3.358602in}{0.696490in}}%
\pgfpathlineto{\pgfqpoint{3.327328in}{0.686947in}}%
\pgfpathlineto{\pgfqpoint{3.291432in}{0.678987in}}%
\pgfpathlineto{\pgfqpoint{3.250597in}{0.672611in}}%
\pgfpathlineto{\pgfqpoint{3.204702in}{0.667843in}}%
\pgfpathlineto{\pgfqpoint{3.153269in}{0.664712in}}%
\pgfpathlineto{\pgfqpoint{3.095757in}{0.663258in}}%
\pgfpathlineto{\pgfqpoint{3.031588in}{0.663532in}}%
\pgfpathlineto{\pgfqpoint{2.934608in}{0.666691in}}%
\pgfpathlineto{\pgfqpoint{2.823138in}{0.673203in}}%
\pgfpathlineto{\pgfqpoint{2.695521in}{0.683274in}}%
\pgfpathlineto{\pgfqpoint{2.550523in}{0.697158in}}%
\pgfpathlineto{\pgfqpoint{2.388298in}{0.715115in}}%
\pgfpathlineto{\pgfqpoint{2.211715in}{0.737305in}}%
\pgfpathlineto{\pgfqpoint{2.073332in}{0.756747in}}%
\pgfpathlineto{\pgfqpoint{1.934526in}{0.778478in}}%
\pgfpathlineto{\pgfqpoint{1.800056in}{0.802279in}}%
\pgfpathlineto{\pgfqpoint{1.715097in}{0.819141in}}%
\pgfpathlineto{\pgfqpoint{1.635436in}{0.836660in}}%
\pgfpathlineto{\pgfqpoint{1.562699in}{0.854698in}}%
\pgfpathlineto{\pgfqpoint{1.497271in}{0.873101in}}%
\pgfpathlineto{\pgfqpoint{1.438769in}{0.891722in}}%
\pgfpathlineto{\pgfqpoint{1.386832in}{0.910426in}}%
\pgfpathlineto{\pgfqpoint{1.341123in}{0.929089in}}%
\pgfpathlineto{\pgfqpoint{1.301325in}{0.947602in}}%
\pgfpathlineto{\pgfqpoint{1.267146in}{0.965865in}}%
\pgfpathlineto{\pgfqpoint{1.238312in}{0.983790in}}%
\pgfpathlineto{\pgfqpoint{1.214575in}{1.001305in}}%
\pgfpathlineto{\pgfqpoint{1.195707in}{1.018346in}}%
\pgfpathlineto{\pgfqpoint{1.181503in}{1.034862in}}%
\pgfpathlineto{\pgfqpoint{1.171589in}{1.050799in}}%
\pgfpathlineto{\pgfqpoint{1.165143in}{1.066105in}}%
\pgfpathlineto{\pgfqpoint{1.161520in}{1.080763in}}%
\pgfpathlineto{\pgfqpoint{1.160216in}{1.094759in}}%
\pgfpathlineto{\pgfqpoint{1.160874in}{1.108080in}}%
\pgfpathlineto{\pgfqpoint{1.163276in}{1.120721in}}%
\pgfpathlineto{\pgfqpoint{1.167348in}{1.132676in}}%
\pgfpathlineto{\pgfqpoint{1.173158in}{1.143943in}}%
\pgfpathlineto{\pgfqpoint{1.180916in}{1.154525in}}%
\pgfpathlineto{\pgfqpoint{1.190973in}{1.164426in}}%
\pgfpathlineto{\pgfqpoint{1.203822in}{1.173655in}}%
\pgfpathlineto{\pgfqpoint{1.219144in}{1.182196in}}%
\pgfpathlineto{\pgfqpoint{1.245703in}{1.193683in}}%
\pgfpathlineto{\pgfqpoint{1.276580in}{1.203577in}}%
\pgfpathlineto{\pgfqpoint{1.311905in}{1.211870in}}%
\pgfpathlineto{\pgfqpoint{1.351916in}{1.218553in}}%
\pgfpathlineto{\pgfqpoint{1.396961in}{1.223616in}}%
\pgfpathlineto{\pgfqpoint{1.447493in}{1.227044in}}%
\pgfpathlineto{\pgfqpoint{1.503801in}{1.228821in}}%
\pgfpathlineto{\pgfqpoint{1.566349in}{1.228896in}}%
\pgfpathlineto{\pgfqpoint{1.636232in}{1.227193in}}%
\pgfpathlineto{\pgfqpoint{1.742307in}{1.222014in}}%
\pgfpathlineto{\pgfqpoint{1.864406in}{1.213327in}}%
\pgfpathlineto{\pgfqpoint{2.003250in}{1.200915in}}%
\pgfpathlineto{\pgfqpoint{2.158830in}{1.184552in}}%
\pgfpathlineto{\pgfqpoint{2.330474in}{1.163992in}}%
\pgfpathlineto{\pgfqpoint{2.468159in}{1.145668in}}%
\pgfpathlineto{\pgfqpoint{2.608108in}{1.124977in}}%
\pgfpathlineto{\pgfqpoint{2.745181in}{1.102148in}}%
\pgfpathlineto{\pgfqpoint{2.875031in}{1.077465in}}%
\pgfpathlineto{\pgfqpoint{2.955793in}{1.060146in}}%
\pgfpathlineto{\pgfqpoint{3.030886in}{1.042268in}}%
\pgfpathlineto{\pgfqpoint{3.099606in}{1.023956in}}%
\pgfpathlineto{\pgfqpoint{3.161404in}{1.005347in}}%
\pgfpathlineto{\pgfqpoint{3.215887in}{0.986588in}}%
\pgfpathlineto{\pgfqpoint{3.262860in}{0.967837in}}%
\pgfpathlineto{\pgfqpoint{3.302893in}{0.949232in}}%
\pgfpathlineto{\pgfqpoint{3.336891in}{0.930869in}}%
\pgfpathlineto{\pgfqpoint{3.365588in}{0.912833in}}%
\pgfpathlineto{\pgfqpoint{3.389559in}{0.895200in}}%
\pgfpathlineto{\pgfqpoint{3.409220in}{0.878034in}}%
\pgfpathlineto{\pgfqpoint{3.424830in}{0.861392in}}%
\pgfpathlineto{\pgfqpoint{3.436486in}{0.845322in}}%
\pgfpathlineto{\pgfqpoint{3.444478in}{0.829862in}}%
\pgfpathlineto{\pgfqpoint{3.449493in}{0.815041in}}%
\pgfpathlineto{\pgfqpoint{3.451848in}{0.800875in}}%
\pgfpathlineto{\pgfqpoint{3.451785in}{0.787379in}}%
\pgfpathlineto{\pgfqpoint{3.449474in}{0.774563in}}%
\pgfpathlineto{\pgfqpoint{3.445017in}{0.762436in}}%
\pgfpathlineto{\pgfqpoint{3.438441in}{0.751001in}}%
\pgfpathlineto{\pgfqpoint{3.429717in}{0.740259in}}%
\pgfpathlineto{\pgfqpoint{3.418932in}{0.730212in}}%
\pgfpathlineto{\pgfqpoint{3.406195in}{0.720863in}}%
\pgfpathlineto{\pgfqpoint{3.391567in}{0.712213in}}%
\pgfpathlineto{\pgfqpoint{3.366139in}{0.700557in}}%
\pgfpathlineto{\pgfqpoint{3.336500in}{0.690485in}}%
\pgfpathlineto{\pgfqpoint{3.302481in}{0.682002in}}%
\pgfpathlineto{\pgfqpoint{3.263765in}{0.675108in}}%
\pgfpathlineto{\pgfqpoint{3.219887in}{0.669802in}}%
\pgfpathlineto{\pgfqpoint{3.170606in}{0.666101in}}%
\pgfpathlineto{\pgfqpoint{3.115578in}{0.664052in}}%
\pgfpathlineto{\pgfqpoint{3.054072in}{0.663705in}}%
\pgfpathlineto{\pgfqpoint{2.985402in}{0.665117in}}%
\pgfpathlineto{\pgfqpoint{2.881587in}{0.669861in}}%
\pgfpathlineto{\pgfqpoint{2.762484in}{0.678054in}}%
\pgfpathlineto{\pgfqpoint{2.626817in}{0.689924in}}%
\pgfpathlineto{\pgfqpoint{2.473208in}{0.705710in}}%
\pgfpathlineto{\pgfqpoint{2.302459in}{0.725627in}}%
\pgfpathlineto{\pgfqpoint{2.167126in}{0.743399in}}%
\pgfpathlineto{\pgfqpoint{2.029261in}{0.763569in}}%
\pgfpathlineto{\pgfqpoint{1.892523in}{0.785987in}}%
\pgfpathlineto{\pgfqpoint{1.760872in}{0.810398in}}%
\pgfpathlineto{\pgfqpoint{1.678021in}{0.827604in}}%
\pgfpathlineto{\pgfqpoint{1.600657in}{0.845401in}}%
\pgfpathlineto{\pgfqpoint{1.530174in}{0.863632in}}%
\pgfpathlineto{\pgfqpoint{1.467428in}{0.882146in}}%
\pgfpathlineto{\pgfqpoint{1.411980in}{0.900813in}}%
\pgfpathlineto{\pgfqpoint{1.363350in}{0.919506in}}%
\pgfpathlineto{\pgfqpoint{1.321089in}{0.938106in}}%
\pgfpathlineto{\pgfqpoint{1.284787in}{0.956505in}}%
\pgfpathlineto{\pgfqpoint{1.254067in}{0.974609in}}%
\pgfpathlineto{\pgfqpoint{1.228586in}{0.992332in}}%
\pgfpathlineto{\pgfqpoint{1.208039in}{1.009601in}}%
\pgfpathlineto{\pgfqpoint{1.192092in}{1.026351in}}%
\pgfpathlineto{\pgfqpoint{1.180044in}{1.042532in}}%
\pgfpathlineto{\pgfqpoint{1.171308in}{1.058112in}}%
\pgfpathlineto{\pgfqpoint{1.165439in}{1.073067in}}%
\pgfpathlineto{\pgfqpoint{1.162113in}{1.087376in}}%
\pgfpathlineto{\pgfqpoint{1.161128in}{1.101023in}}%
\pgfpathlineto{\pgfqpoint{1.162402in}{1.113997in}}%
\pgfpathlineto{\pgfqpoint{1.165974in}{1.126290in}}%
\pgfpathlineto{\pgfqpoint{1.171998in}{1.137900in}}%
\pgfpathlineto{\pgfqpoint{1.180318in}{1.148820in}}%
\pgfpathlineto{\pgfqpoint{1.190643in}{1.159043in}}%
\pgfpathlineto{\pgfqpoint{1.202888in}{1.168566in}}%
\pgfpathlineto{\pgfqpoint{1.217000in}{1.177388in}}%
\pgfpathlineto{\pgfqpoint{1.232954in}{1.185506in}}%
\pgfpathlineto{\pgfqpoint{1.260371in}{1.196365in}}%
\pgfpathlineto{\pgfqpoint{1.292128in}{1.205641in}}%
\pgfpathlineto{\pgfqpoint{1.328551in}{1.213338in}}%
\pgfpathlineto{\pgfqpoint{1.370010in}{1.219458in}}%
\pgfpathlineto{\pgfqpoint{1.416588in}{1.223972in}}%
\pgfpathlineto{\pgfqpoint{1.468773in}{1.226848in}}%
\pgfpathlineto{\pgfqpoint{1.527125in}{1.228043in}}%
\pgfpathlineto{\pgfqpoint{1.592234in}{1.227507in}}%
\pgfpathlineto{\pgfqpoint{1.690628in}{1.223988in}}%
\pgfpathlineto{\pgfqpoint{1.803679in}{1.217098in}}%
\pgfpathlineto{\pgfqpoint{1.933007in}{1.206626in}}%
\pgfpathlineto{\pgfqpoint{2.079749in}{1.192319in}}%
\pgfpathlineto{\pgfqpoint{2.243560in}{1.173917in}}%
\pgfpathlineto{\pgfqpoint{2.421194in}{1.151278in}}%
\pgfpathlineto{\pgfqpoint{2.559740in}{1.131513in}}%
\pgfpathlineto{\pgfqpoint{2.698157in}{1.109480in}}%
\pgfpathlineto{\pgfqpoint{2.831402in}{1.085427in}}%
\pgfpathlineto{\pgfqpoint{2.915144in}{1.068432in}}%
\pgfpathlineto{\pgfqpoint{2.993452in}{1.050808in}}%
\pgfpathlineto{\pgfqpoint{3.065297in}{1.032693in}}%
\pgfpathlineto{\pgfqpoint{3.129745in}{1.014243in}}%
\pgfpathlineto{\pgfqpoint{3.186922in}{0.995603in}}%
\pgfpathlineto{\pgfqpoint{3.237276in}{0.976907in}}%
\pgfpathlineto{\pgfqpoint{3.281225in}{0.958273in}}%
\pgfpathlineto{\pgfqpoint{3.319154in}{0.939811in}}%
\pgfpathlineto{\pgfqpoint{3.351422in}{0.921621in}}%
\pgfpathlineto{\pgfqpoint{3.378355in}{0.903789in}}%
\pgfpathlineto{\pgfqpoint{3.400249in}{0.886391in}}%
\pgfpathlineto{\pgfqpoint{3.417372in}{0.869494in}}%
\pgfpathlineto{\pgfqpoint{3.430268in}{0.853156in}}%
\pgfpathlineto{\pgfqpoint{3.439659in}{0.837415in}}%
\pgfpathlineto{\pgfqpoint{3.446078in}{0.822295in}}%
\pgfpathlineto{\pgfqpoint{3.449926in}{0.807817in}}%
\pgfpathlineto{\pgfqpoint{3.451473in}{0.793998in}}%
\pgfpathlineto{\pgfqpoint{3.450861in}{0.780848in}}%
\pgfpathlineto{\pgfqpoint{3.448104in}{0.768378in}}%
\pgfpathlineto{\pgfqpoint{3.443083in}{0.756590in}}%
\pgfpathlineto{\pgfqpoint{3.435554in}{0.745486in}}%
\pgfpathlineto{\pgfqpoint{3.425711in}{0.735073in}}%
\pgfpathlineto{\pgfqpoint{3.413903in}{0.725359in}}%
\pgfpathlineto{\pgfqpoint{3.400194in}{0.716346in}}%
\pgfpathlineto{\pgfqpoint{3.384619in}{0.708037in}}%
\pgfpathlineto{\pgfqpoint{3.357766in}{0.696894in}}%
\pgfpathlineto{\pgfqpoint{3.326627in}{0.687337in}}%
\pgfpathlineto{\pgfqpoint{3.290962in}{0.679369in}}%
\pgfpathlineto{\pgfqpoint{3.250386in}{0.672985in}}%
\pgfpathlineto{\pgfqpoint{3.204637in}{0.668197in}}%
\pgfpathlineto{\pgfqpoint{3.153452in}{0.665041in}}%
\pgfpathlineto{\pgfqpoint{3.096200in}{0.663555in}}%
\pgfpathlineto{\pgfqpoint{3.032255in}{0.663792in}}%
\pgfpathlineto{\pgfqpoint{2.935494in}{0.666895in}}%
\pgfpathlineto{\pgfqpoint{2.824229in}{0.673348in}}%
\pgfpathlineto{\pgfqpoint{2.696975in}{0.683361in}}%
\pgfpathlineto{\pgfqpoint{2.552376in}{0.697173in}}%
\pgfpathlineto{\pgfqpoint{2.390525in}{0.715045in}}%
\pgfpathlineto{\pgfqpoint{2.214235in}{0.737177in}}%
\pgfpathlineto{\pgfqpoint{2.076174in}{0.756562in}}%
\pgfpathlineto{\pgfqpoint{1.937372in}{0.778224in}}%
\pgfpathlineto{\pgfqpoint{1.802757in}{0.801991in}}%
\pgfpathlineto{\pgfqpoint{1.717796in}{0.818844in}}%
\pgfpathlineto{\pgfqpoint{1.638085in}{0.836342in}}%
\pgfpathlineto{\pgfqpoint{1.564535in}{0.854345in}}%
\pgfpathlineto{\pgfqpoint{1.497779in}{0.872716in}}%
\pgfpathlineto{\pgfqpoint{1.438174in}{0.891321in}}%
\pgfpathlineto{\pgfqpoint{1.385796in}{0.910029in}}%
\pgfpathlineto{\pgfqpoint{1.340447in}{0.928711in}}%
\pgfpathlineto{\pgfqpoint{1.301800in}{0.947233in}}%
\pgfpathlineto{\pgfqpoint{1.269590in}{0.965466in}}%
\pgfpathlineto{\pgfqpoint{1.242705in}{0.983342in}}%
\pgfpathlineto{\pgfqpoint{1.220193in}{1.000806in}}%
\pgfpathlineto{\pgfqpoint{1.201366in}{1.017805in}}%
\pgfpathlineto{\pgfqpoint{1.185795in}{1.034293in}}%
\pgfpathlineto{\pgfqpoint{1.173319in}{1.050228in}}%
\pgfpathlineto{\pgfqpoint{1.164035in}{1.065571in}}%
\pgfpathlineto{\pgfqpoint{1.158306in}{1.080290in}}%
\pgfpathlineto{\pgfqpoint{1.156558in}{1.094356in}}%
\pgfpathlineto{\pgfqpoint{1.157601in}{1.107742in}}%
\pgfpathlineto{\pgfqpoint{1.160912in}{1.120437in}}%
\pgfpathlineto{\pgfqpoint{1.166319in}{1.132435in}}%
\pgfpathlineto{\pgfqpoint{1.173699in}{1.143732in}}%
\pgfpathlineto{\pgfqpoint{1.182981in}{1.154326in}}%
\pgfpathlineto{\pgfqpoint{1.194140in}{1.164217in}}%
\pgfpathlineto{\pgfqpoint{1.207205in}{1.173404in}}%
\pgfpathlineto{\pgfqpoint{1.222250in}{1.181893in}}%
\pgfpathlineto{\pgfqpoint{1.248617in}{1.193317in}}%
\pgfpathlineto{\pgfqpoint{1.279382in}{1.203165in}}%
\pgfpathlineto{\pgfqpoint{1.314642in}{1.211424in}}%
\pgfpathlineto{\pgfqpoint{1.354603in}{1.218080in}}%
\pgfpathlineto{\pgfqpoint{1.399572in}{1.223116in}}%
\pgfpathlineto{\pgfqpoint{1.449959in}{1.226508in}}%
\pgfpathlineto{\pgfqpoint{1.506275in}{1.228232in}}%
\pgfpathlineto{\pgfqpoint{1.569120in}{1.228258in}}%
\pgfpathlineto{\pgfqpoint{1.663404in}{1.225627in}}%
\pgfpathlineto{\pgfqpoint{1.772354in}{1.219679in}}%
\pgfpathlineto{\pgfqpoint{1.898467in}{1.210094in}}%
\pgfpathlineto{\pgfqpoint{2.042391in}{1.196627in}}%
\pgfpathlineto{\pgfqpoint{2.202937in}{1.179110in}}%
\pgfpathlineto{\pgfqpoint{2.377066in}{1.157449in}}%
\pgfpathlineto{\pgfqpoint{2.513710in}{1.138471in}}%
\pgfpathlineto{\pgfqpoint{2.652418in}{1.117167in}}%
\pgfpathlineto{\pgfqpoint{2.787893in}{1.093663in}}%
\pgfpathlineto{\pgfqpoint{2.873894in}{1.076974in}}%
\pgfpathlineto{\pgfqpoint{2.955093in}{1.059624in}}%
\pgfpathlineto{\pgfqpoint{3.030575in}{1.041744in}}%
\pgfpathlineto{\pgfqpoint{3.099639in}{1.023466in}}%
\pgfpathlineto{\pgfqpoint{3.161797in}{1.004915in}}%
\pgfpathlineto{\pgfqpoint{3.216777in}{0.986219in}}%
\pgfpathlineto{\pgfqpoint{3.264519in}{0.967502in}}%
\pgfpathlineto{\pgfqpoint{3.305179in}{0.948886in}}%
\pgfpathlineto{\pgfqpoint{3.339092in}{0.930498in}}%
\pgfpathlineto{\pgfqpoint{3.366818in}{0.912478in}}%
\pgfpathlineto{\pgfqpoint{3.389529in}{0.894884in}}%
\pgfpathlineto{\pgfqpoint{3.408236in}{0.877761in}}%
\pgfpathlineto{\pgfqpoint{3.423674in}{0.861148in}}%
\pgfpathlineto{\pgfqpoint{3.436305in}{0.845081in}}%
\pgfpathlineto{\pgfqpoint{3.446319in}{0.829593in}}%
\pgfpathlineto{\pgfqpoint{3.453633in}{0.814714in}}%
\pgfpathlineto{\pgfqpoint{3.457890in}{0.800469in}}%
\pgfpathlineto{\pgfqpoint{3.458460in}{0.786882in}}%
\pgfpathlineto{\pgfqpoint{3.455512in}{0.773977in}}%
\pgfpathlineto{\pgfqpoint{3.450364in}{0.761774in}}%
\pgfpathlineto{\pgfqpoint{3.443180in}{0.750275in}}%
\pgfpathlineto{\pgfqpoint{3.434068in}{0.739483in}}%
\pgfpathlineto{\pgfqpoint{3.423092in}{0.729400in}}%
\pgfpathlineto{\pgfqpoint{3.410281in}{0.720025in}}%
\pgfpathlineto{\pgfqpoint{3.395618in}{0.711358in}}%
\pgfpathlineto{\pgfqpoint{3.379048in}{0.703396in}}%
\pgfpathlineto{\pgfqpoint{3.350417in}{0.692767in}}%
\pgfpathlineto{\pgfqpoint{3.317259in}{0.683717in}}%
\pgfpathlineto{\pgfqpoint{3.279488in}{0.676260in}}%
\pgfpathlineto{\pgfqpoint{3.236842in}{0.670412in}}%
\pgfpathlineto{\pgfqpoint{3.188971in}{0.666198in}}%
\pgfpathlineto{\pgfqpoint{3.135436in}{0.663646in}}%
\pgfpathlineto{\pgfqpoint{3.075709in}{0.662794in}}%
\pgfpathlineto{\pgfqpoint{3.009173in}{0.663683in}}%
\pgfpathlineto{\pgfqpoint{2.908757in}{0.667665in}}%
\pgfpathlineto{\pgfqpoint{2.793405in}{0.675035in}}%
\pgfpathlineto{\pgfqpoint{2.661021in}{0.686067in}}%
\pgfpathlineto{\pgfqpoint{2.511080in}{0.701000in}}%
\pgfpathlineto{\pgfqpoint{2.344822in}{0.720036in}}%
\pgfpathlineto{\pgfqpoint{2.165196in}{0.743343in}}%
\pgfpathlineto{\pgfqpoint{2.025531in}{0.763657in}}%
\pgfpathlineto{\pgfqpoint{1.887467in}{0.786181in}}%
\pgfpathlineto{\pgfqpoint{1.755832in}{0.810626in}}%
\pgfpathlineto{\pgfqpoint{1.673639in}{0.827821in}}%
\pgfpathlineto{\pgfqpoint{1.597040in}{0.845602in}}%
\pgfpathlineto{\pgfqpoint{1.526824in}{0.863842in}}%
\pgfpathlineto{\pgfqpoint{1.463605in}{0.882403in}}%
\pgfpathlineto{\pgfqpoint{1.407829in}{0.901137in}}%
\pgfpathlineto{\pgfqpoint{1.359626in}{0.919882in}}%
\pgfpathlineto{\pgfqpoint{1.318275in}{0.938508in}}%
\pgfpathlineto{\pgfqpoint{1.282924in}{0.956920in}}%
\pgfpathlineto{\pgfqpoint{1.252881in}{0.975028in}}%
\pgfpathlineto{\pgfqpoint{1.227619in}{0.992755in}}%
\pgfpathlineto{\pgfqpoint{1.206776in}{1.010031in}}%
\pgfpathlineto{\pgfqpoint{1.190150in}{1.026797in}}%
\pgfpathlineto{\pgfqpoint{1.177708in}{1.043000in}}%
\pgfpathlineto{\pgfqpoint{1.169109in}{1.058598in}}%
\pgfpathlineto{\pgfqpoint{1.163554in}{1.073561in}}%
\pgfpathlineto{\pgfqpoint{1.160696in}{1.087872in}}%
\pgfpathlineto{\pgfqpoint{1.160268in}{1.101515in}}%
\pgfpathlineto{\pgfqpoint{1.162083in}{1.114479in}}%
\pgfpathlineto{\pgfqpoint{1.166034in}{1.126756in}}%
\pgfpathlineto{\pgfqpoint{1.172094in}{1.138341in}}%
\pgfpathlineto{\pgfqpoint{1.180315in}{1.149234in}}%
\pgfpathlineto{\pgfqpoint{1.190685in}{1.159433in}}%
\pgfpathlineto{\pgfqpoint{1.203026in}{1.168934in}}%
\pgfpathlineto{\pgfqpoint{1.217276in}{1.177736in}}%
\pgfpathlineto{\pgfqpoint{1.233397in}{1.185836in}}%
\pgfpathlineto{\pgfqpoint{1.261078in}{1.196666in}}%
\pgfpathlineto{\pgfqpoint{1.293051in}{1.205908in}}%
\pgfpathlineto{\pgfqpoint{1.329551in}{1.213559in}}%
\pgfpathlineto{\pgfqpoint{1.370957in}{1.219619in}}%
\pgfpathlineto{\pgfqpoint{1.417712in}{1.224084in}}%
\pgfpathlineto{\pgfqpoint{1.469940in}{1.226921in}}%
\pgfpathlineto{\pgfqpoint{1.528298in}{1.228080in}}%
\pgfpathlineto{\pgfqpoint{1.593506in}{1.227508in}}%
\pgfpathlineto{\pgfqpoint{1.692260in}{1.223938in}}%
\pgfpathlineto{\pgfqpoint{1.805839in}{1.216991in}}%
\pgfpathlineto{\pgfqpoint{1.935544in}{1.206450in}}%
\pgfpathlineto{\pgfqpoint{2.082530in}{1.192068in}}%
\pgfpathlineto{\pgfqpoint{2.247434in}{1.173628in}}%
\pgfpathlineto{\pgfqpoint{2.380202in}{1.156994in}}%
\pgfpathlineto{\pgfqpoint{2.517382in}{1.137939in}}%
\pgfpathlineto{\pgfqpoint{2.655454in}{1.116562in}}%
\pgfpathlineto{\pgfqpoint{2.790451in}{1.093070in}}%
\pgfpathlineto{\pgfqpoint{2.876580in}{1.076382in}}%
\pgfpathlineto{\pgfqpoint{2.957974in}{1.059010in}}%
\pgfpathlineto{\pgfqpoint{3.033129in}{1.041093in}}%
\pgfpathlineto{\pgfqpoint{3.100889in}{1.022777in}}%
\pgfpathlineto{\pgfqpoint{3.161368in}{1.004205in}}%
\pgfpathlineto{\pgfqpoint{3.214914in}{0.985516in}}%
\pgfpathlineto{\pgfqpoint{3.261871in}{0.966836in}}%
\pgfpathlineto{\pgfqpoint{3.302585in}{0.948281in}}%
\pgfpathlineto{\pgfqpoint{3.337397in}{0.929955in}}%
\pgfpathlineto{\pgfqpoint{3.366648in}{0.911950in}}%
\pgfpathlineto{\pgfqpoint{3.390677in}{0.894347in}}%
\pgfpathlineto{\pgfqpoint{3.409820in}{0.877216in}}%
\pgfpathlineto{\pgfqpoint{3.424498in}{0.860620in}}%
\pgfpathlineto{\pgfqpoint{3.435404in}{0.844606in}}%
\pgfpathlineto{\pgfqpoint{3.443146in}{0.829202in}}%
\pgfpathlineto{\pgfqpoint{3.448191in}{0.814429in}}%
\pgfpathlineto{\pgfqpoint{3.450870in}{0.800306in}}%
\pgfpathlineto{\pgfqpoint{3.451372in}{0.786846in}}%
\pgfpathlineto{\pgfqpoint{3.449750in}{0.774061in}}%
\pgfpathlineto{\pgfqpoint{3.445915in}{0.761956in}}%
\pgfpathlineto{\pgfqpoint{3.439642in}{0.750532in}}%
\pgfpathlineto{\pgfqpoint{3.430776in}{0.739793in}}%
\pgfpathlineto{\pgfqpoint{3.419858in}{0.729751in}}%
\pgfpathlineto{\pgfqpoint{3.407014in}{0.720410in}}%
\pgfpathlineto{\pgfqpoint{3.392293in}{0.711772in}}%
\pgfpathlineto{\pgfqpoint{3.366725in}{0.700136in}}%
\pgfpathlineto{\pgfqpoint{3.336916in}{0.690086in}}%
\pgfpathlineto{\pgfqpoint{3.302665in}{0.681624in}}%
\pgfpathlineto{\pgfqpoint{3.263632in}{0.674749in}}%
\pgfpathlineto{\pgfqpoint{3.219468in}{0.669463in}}%
\pgfpathlineto{\pgfqpoint{3.170017in}{0.665797in}}%
\pgfpathlineto{\pgfqpoint{3.114689in}{0.663790in}}%
\pgfpathlineto{\pgfqpoint{3.052863in}{0.663489in}}%
\pgfpathlineto{\pgfqpoint{2.983912in}{0.664952in}}%
\pgfpathlineto{\pgfqpoint{2.879811in}{0.669768in}}%
\pgfpathlineto{\pgfqpoint{2.760434in}{0.678040in}}%
\pgfpathlineto{\pgfqpoint{2.624286in}{0.689994in}}%
\pgfpathlineto{\pgfqpoint{2.470578in}{0.705878in}}%
\pgfpathlineto{\pgfqpoint{2.300653in}{0.725951in}}%
\pgfpathlineto{\pgfqpoint{2.165181in}{0.743813in}}%
\pgfpathlineto{\pgfqpoint{2.026319in}{0.764028in}}%
\pgfpathlineto{\pgfqpoint{1.888414in}{0.786474in}}%
\pgfpathlineto{\pgfqpoint{1.756495in}{0.810928in}}%
\pgfpathlineto{\pgfqpoint{1.674202in}{0.828140in}}%
\pgfpathlineto{\pgfqpoint{1.597691in}{0.845923in}}%
\pgfpathlineto{\pgfqpoint{1.527729in}{0.864140in}}%
\pgfpathlineto{\pgfqpoint{1.464798in}{0.882658in}}%
\pgfpathlineto{\pgfqpoint{1.409104in}{0.901342in}}%
\pgfpathlineto{\pgfqpoint{1.360566in}{0.920060in}}%
\pgfpathlineto{\pgfqpoint{1.318983in}{0.938669in}}%
\pgfpathlineto{\pgfqpoint{1.284095in}{0.957033in}}%
\pgfpathlineto{\pgfqpoint{1.254762in}{0.975080in}}%
\pgfpathlineto{\pgfqpoint{1.230017in}{0.992750in}}%
\pgfpathlineto{\pgfqpoint{1.209161in}{1.009984in}}%
\pgfpathlineto{\pgfqpoint{1.191756in}{1.026732in}}%
\pgfpathlineto{\pgfqpoint{1.177630in}{1.042946in}}%
\pgfpathlineto{\pgfqpoint{1.166877in}{1.058584in}}%
\pgfpathlineto{\pgfqpoint{1.159853in}{1.073607in}}%
\pgfpathlineto{\pgfqpoint{1.156951in}{1.087982in}}%
\pgfpathlineto{\pgfqpoint{1.156907in}{1.101681in}}%
\pgfpathlineto{\pgfqpoint{1.159193in}{1.114693in}}%
\pgfpathlineto{\pgfqpoint{1.163622in}{1.127011in}}%
\pgfpathlineto{\pgfqpoint{1.170059in}{1.138629in}}%
\pgfpathlineto{\pgfqpoint{1.178423in}{1.149546in}}%
\pgfpathlineto{\pgfqpoint{1.188687in}{1.159760in}}%
\pgfpathlineto{\pgfqpoint{1.200875in}{1.169271in}}%
\pgfpathlineto{\pgfqpoint{1.215068in}{1.178083in}}%
\pgfpathlineto{\pgfqpoint{1.231295in}{1.186198in}}%
\pgfpathlineto{\pgfqpoint{1.259265in}{1.197057in}}%
\pgfpathlineto{\pgfqpoint{1.291628in}{1.206331in}}%
\pgfpathlineto{\pgfqpoint{1.328528in}{1.214010in}}%
\pgfpathlineto{\pgfqpoint{1.370215in}{1.220079in}}%
\pgfpathlineto{\pgfqpoint{1.417046in}{1.224520in}}%
\pgfpathlineto{\pgfqpoint{1.469485in}{1.227313in}}%
\pgfpathlineto{\pgfqpoint{1.528103in}{1.228437in}}%
\pgfpathlineto{\pgfqpoint{1.593076in}{1.227874in}}%
\pgfpathlineto{\pgfqpoint{1.691115in}{1.224348in}}%
\pgfpathlineto{\pgfqpoint{1.804865in}{1.217395in}}%
\pgfpathlineto{\pgfqpoint{1.935856in}{1.206768in}}%
\pgfpathlineto{\pgfqpoint{2.084132in}{1.192261in}}%
\pgfpathlineto{\pgfqpoint{2.248255in}{1.173707in}}%
\pgfpathlineto{\pgfqpoint{2.425303in}{1.150973in}}%
\pgfpathlineto{\pgfqpoint{2.563838in}{1.131122in}}%
\pgfpathlineto{\pgfqpoint{2.702456in}{1.108959in}}%
\pgfpathlineto{\pgfqpoint{2.835624in}{1.084802in}}%
\pgfpathlineto{\pgfqpoint{2.919139in}{1.067770in}}%
\pgfpathlineto{\pgfqpoint{2.997197in}{1.050134in}}%
\pgfpathlineto{\pgfqpoint{3.068978in}{1.032017in}}%
\pgfpathlineto{\pgfqpoint{3.133882in}{1.013550in}}%
\pgfpathlineto{\pgfqpoint{3.191527in}{0.994872in}}%
\pgfpathlineto{\pgfqpoint{3.241752in}{0.976127in}}%
\pgfpathlineto{\pgfqpoint{3.284683in}{0.957474in}}%
\pgfpathlineto{\pgfqpoint{3.321202in}{0.939022in}}%
\pgfpathlineto{\pgfqpoint{3.352220in}{0.920856in}}%
\pgfpathlineto{\pgfqpoint{3.378445in}{0.903052in}}%
\pgfpathlineto{\pgfqpoint{3.400381in}{0.885677in}}%
\pgfpathlineto{\pgfqpoint{3.418332in}{0.868791in}}%
\pgfpathlineto{\pgfqpoint{3.432396in}{0.852448in}}%
\pgfpathlineto{\pgfqpoint{3.442470in}{0.836692in}}%
\pgfpathlineto{\pgfqpoint{3.448718in}{0.821562in}}%
\pgfpathlineto{\pgfqpoint{3.452096in}{0.807084in}}%
\pgfpathlineto{\pgfqpoint{3.452935in}{0.793272in}}%
\pgfpathlineto{\pgfqpoint{3.451475in}{0.780139in}}%
\pgfpathlineto{\pgfqpoint{3.447883in}{0.767693in}}%
\pgfpathlineto{\pgfqpoint{3.442257in}{0.755939in}}%
\pgfpathlineto{\pgfqpoint{3.434623in}{0.744882in}}%
\pgfpathlineto{\pgfqpoint{3.424939in}{0.734520in}}%
\pgfpathlineto{\pgfqpoint{3.413125in}{0.724853in}}%
\pgfpathlineto{\pgfqpoint{3.399317in}{0.715882in}}%
\pgfpathlineto{\pgfqpoint{3.383592in}{0.707611in}}%
\pgfpathlineto{\pgfqpoint{3.356449in}{0.696520in}}%
\pgfpathlineto{\pgfqpoint{3.324998in}{0.687016in}}%
\pgfpathlineto{\pgfqpoint{3.289083in}{0.679106in}}%
\pgfpathlineto{\pgfqpoint{3.248417in}{0.672797in}}%
\pgfpathlineto{\pgfqpoint{3.202591in}{0.668099in}}%
\pgfpathlineto{\pgfqpoint{3.151121in}{0.665019in}}%
\pgfpathlineto{\pgfqpoint{3.093871in}{0.663592in}}%
\pgfpathlineto{\pgfqpoint{3.030006in}{0.663882in}}%
\pgfpathlineto{\pgfqpoint{2.933024in}{0.667064in}}%
\pgfpathlineto{\pgfqpoint{2.821012in}{0.673613in}}%
\pgfpathlineto{\pgfqpoint{2.692823in}{0.683736in}}%
\pgfpathlineto{\pgfqpoint{2.547837in}{0.697661in}}%
\pgfpathlineto{\pgfqpoint{2.385962in}{0.715638in}}%
\pgfpathlineto{\pgfqpoint{2.207782in}{0.737946in}}%
\pgfpathlineto{\pgfqpoint{2.068834in}{0.757487in}}%
\pgfpathlineto{\pgfqpoint{1.930796in}{0.779262in}}%
\pgfpathlineto{\pgfqpoint{1.797949in}{0.803038in}}%
\pgfpathlineto{\pgfqpoint{1.714119in}{0.819854in}}%
\pgfpathlineto{\pgfqpoint{1.635200in}{0.837320in}}%
\pgfpathlineto{\pgfqpoint{1.562028in}{0.855315in}}%
\pgfpathlineto{\pgfqpoint{1.495319in}{0.873705in}}%
\pgfpathlineto{\pgfqpoint{1.435677in}{0.892344in}}%
\pgfpathlineto{\pgfqpoint{1.383577in}{0.911068in}}%
\pgfpathlineto{\pgfqpoint{1.338732in}{0.929736in}}%
\pgfpathlineto{\pgfqpoint{1.300324in}{0.948239in}}%
\pgfpathlineto{\pgfqpoint{1.267717in}{0.966479in}}%
\pgfpathlineto{\pgfqpoint{1.240366in}{0.984371in}}%
\pgfpathlineto{\pgfqpoint{1.217820in}{1.001837in}}%
\pgfpathlineto{\pgfqpoint{1.199716in}{1.018813in}}%
\pgfpathlineto{\pgfqpoint{1.185745in}{1.035246in}}%
\pgfpathlineto{\pgfqpoint{1.175354in}{1.051094in}}%
\pgfpathlineto{\pgfqpoint{1.168110in}{1.066327in}}%
\pgfpathlineto{\pgfqpoint{1.163693in}{1.080923in}}%
\pgfpathlineto{\pgfqpoint{1.161862in}{1.094862in}}%
\pgfpathlineto{\pgfqpoint{1.162451in}{1.108132in}}%
\pgfpathlineto{\pgfqpoint{1.165379in}{1.120721in}}%
\pgfpathlineto{\pgfqpoint{1.170604in}{1.132625in}}%
\pgfpathlineto{\pgfqpoint{1.177981in}{1.143839in}}%
\pgfpathlineto{\pgfqpoint{1.187366in}{1.154358in}}%
\pgfpathlineto{\pgfqpoint{1.198653in}{1.164180in}}%
\pgfpathlineto{\pgfqpoint{1.211774in}{1.173301in}}%
\pgfpathlineto{\pgfqpoint{1.226704in}{1.181721in}}%
\pgfpathlineto{\pgfqpoint{1.252527in}{1.193035in}}%
\pgfpathlineto{\pgfqpoint{1.282665in}{1.202775in}}%
\pgfpathlineto{\pgfqpoint{1.317527in}{1.210949in}}%
\pgfpathlineto{\pgfqpoint{1.357423in}{1.217559in}}%
\pgfpathlineto{\pgfqpoint{1.402351in}{1.222568in}}%
\pgfpathlineto{\pgfqpoint{1.452753in}{1.225948in}}%
\pgfpathlineto{\pgfqpoint{1.509132in}{1.227661in}}%
\pgfpathlineto{\pgfqpoint{1.572043in}{1.227659in}}%
\pgfpathlineto{\pgfqpoint{1.642092in}{1.225884in}}%
\pgfpathlineto{\pgfqpoint{1.747742in}{1.220640in}}%
\pgfpathlineto{\pgfqpoint{1.868947in}{1.211932in}}%
\pgfpathlineto{\pgfqpoint{2.007172in}{1.199512in}}%
\pgfpathlineto{\pgfqpoint{2.162862in}{1.183119in}}%
\pgfpathlineto{\pgfqpoint{2.334296in}{1.162549in}}%
\pgfpathlineto{\pgfqpoint{2.470502in}{1.144319in}}%
\pgfpathlineto{\pgfqpoint{2.609320in}{1.123740in}}%
\pgfpathlineto{\pgfqpoint{2.746393in}{1.100970in}}%
\pgfpathlineto{\pgfqpoint{2.834389in}{1.084702in}}%
\pgfpathlineto{\pgfqpoint{2.917814in}{1.067668in}}%
\pgfpathlineto{\pgfqpoint{2.995543in}{1.049998in}}%
\pgfpathlineto{\pgfqpoint{3.066937in}{1.031856in}}%
\pgfpathlineto{\pgfqpoint{3.131555in}{1.013398in}}%
\pgfpathlineto{\pgfqpoint{3.189154in}{0.994766in}}%
\pgfpathlineto{\pgfqpoint{3.239689in}{0.976090in}}%
\pgfpathlineto{\pgfqpoint{3.283313in}{0.957489in}}%
\pgfpathlineto{\pgfqpoint{3.320375in}{0.939072in}}%
\pgfpathlineto{\pgfqpoint{3.351424in}{0.920932in}}%
\pgfpathlineto{\pgfqpoint{3.377203in}{0.903155in}}%
\pgfpathlineto{\pgfqpoint{3.398229in}{0.885824in}}%
\pgfpathlineto{\pgfqpoint{3.414705in}{0.869010in}}%
\pgfpathlineto{\pgfqpoint{3.427480in}{0.852746in}}%
\pgfpathlineto{\pgfqpoint{3.437224in}{0.837064in}}%
\pgfpathlineto{\pgfqpoint{3.444406in}{0.821987in}}%
\pgfpathlineto{\pgfqpoint{3.449298in}{0.807538in}}%
\pgfpathlineto{\pgfqpoint{3.451971in}{0.793734in}}%
\pgfpathlineto{\pgfqpoint{3.452298in}{0.780588in}}%
\pgfpathlineto{\pgfqpoint{3.449954in}{0.768108in}}%
\pgfpathlineto{\pgfqpoint{3.444463in}{0.756299in}}%
\pgfpathlineto{\pgfqpoint{3.436482in}{0.745182in}}%
\pgfpathlineto{\pgfqpoint{3.426510in}{0.734765in}}%
\pgfpathlineto{\pgfqpoint{3.414629in}{0.725053in}}%
\pgfpathlineto{\pgfqpoint{3.400888in}{0.716045in}}%
\pgfpathlineto{\pgfqpoint{3.385307in}{0.707742in}}%
\pgfpathlineto{\pgfqpoint{3.358448in}{0.696610in}}%
\pgfpathlineto{\pgfqpoint{3.327236in}{0.687063in}}%
\pgfpathlineto{\pgfqpoint{3.291331in}{0.679094in}}%
\pgfpathlineto{\pgfqpoint{3.250546in}{0.672709in}}%
\pgfpathlineto{\pgfqpoint{3.204680in}{0.667933in}}%
\pgfpathlineto{\pgfqpoint{3.153267in}{0.664795in}}%
\pgfpathlineto{\pgfqpoint{3.095791in}{0.663336in}}%
\pgfpathlineto{\pgfqpoint{3.031685in}{0.663604in}}%
\pgfpathlineto{\pgfqpoint{2.934825in}{0.666754in}}%
\pgfpathlineto{\pgfqpoint{2.823480in}{0.673252in}}%
\pgfpathlineto{\pgfqpoint{2.695962in}{0.683302in}}%
\pgfpathlineto{\pgfqpoint{2.551041in}{0.697169in}}%
\pgfpathlineto{\pgfqpoint{2.388963in}{0.715100in}}%
\pgfpathlineto{\pgfqpoint{2.212417in}{0.737269in}}%
\pgfpathlineto{\pgfqpoint{2.074175in}{0.756691in}}%
\pgfpathlineto{\pgfqpoint{1.935429in}{0.778403in}}%
\pgfpathlineto{\pgfqpoint{1.800809in}{0.802189in}}%
\pgfpathlineto{\pgfqpoint{1.715872in}{0.819046in}}%
\pgfpathlineto{\pgfqpoint{1.636498in}{0.836585in}}%
\pgfpathlineto{\pgfqpoint{1.563389in}{0.854642in}}%
\pgfpathlineto{\pgfqpoint{1.496985in}{0.873058in}}%
\pgfpathlineto{\pgfqpoint{1.437542in}{0.891685in}}%
\pgfpathlineto{\pgfqpoint{1.385139in}{0.910388in}}%
\pgfpathlineto{\pgfqpoint{1.339673in}{0.929042in}}%
\pgfpathlineto{\pgfqpoint{1.300861in}{0.947538in}}%
\pgfpathlineto{\pgfqpoint{1.268240in}{0.965776in}}%
\pgfpathlineto{\pgfqpoint{1.241167in}{0.983670in}}%
\pgfpathlineto{\pgfqpoint{1.218864in}{1.001145in}}%
\pgfpathlineto{\pgfqpoint{1.201214in}{1.018119in}}%
\pgfpathlineto{\pgfqpoint{1.187605in}{1.034546in}}%
\pgfpathlineto{\pgfqpoint{1.177272in}{1.050396in}}%
\pgfpathlineto{\pgfqpoint{1.169648in}{1.065643in}}%
\pgfpathlineto{\pgfqpoint{1.164358in}{1.080266in}}%
\pgfpathlineto{\pgfqpoint{1.161220in}{1.094247in}}%
\pgfpathlineto{\pgfqpoint{1.160249in}{1.107572in}}%
\pgfpathlineto{\pgfqpoint{1.161653in}{1.120233in}}%
\pgfpathlineto{\pgfqpoint{1.165834in}{1.132223in}}%
\pgfpathlineto{\pgfqpoint{1.173110in}{1.143536in}}%
\pgfpathlineto{\pgfqpoint{1.182562in}{1.154152in}}%
\pgfpathlineto{\pgfqpoint{1.193957in}{1.164064in}}%
\pgfpathlineto{\pgfqpoint{1.207232in}{1.173272in}}%
\pgfpathlineto{\pgfqpoint{1.222357in}{1.181774in}}%
\pgfpathlineto{\pgfqpoint{1.248515in}{1.193204in}}%
\pgfpathlineto{\pgfqpoint{1.278956in}{1.203045in}}%
\pgfpathlineto{\pgfqpoint{1.313955in}{1.211302in}}%
\pgfpathlineto{\pgfqpoint{1.353860in}{1.217977in}}%
\pgfpathlineto{\pgfqpoint{1.398772in}{1.223050in}}%
\pgfpathlineto{\pgfqpoint{1.449126in}{1.226492in}}%
\pgfpathlineto{\pgfqpoint{1.505454in}{1.228264in}}%
\pgfpathlineto{\pgfqpoint{1.568325in}{1.228318in}}%
\pgfpathlineto{\pgfqpoint{1.638347in}{1.226597in}}%
\pgfpathlineto{\pgfqpoint{1.743954in}{1.221419in}}%
\pgfpathlineto{\pgfqpoint{1.865060in}{1.212766in}}%
\pgfpathlineto{\pgfqpoint{2.003178in}{1.200407in}}%
\pgfpathlineto{\pgfqpoint{2.158845in}{1.184068in}}%
\pgfpathlineto{\pgfqpoint{2.330340in}{1.163546in}}%
\pgfpathlineto{\pgfqpoint{2.466675in}{1.145348in}}%
\pgfpathlineto{\pgfqpoint{2.605848in}{1.124790in}}%
\pgfpathlineto{\pgfqpoint{2.743160in}{1.102035in}}%
\pgfpathlineto{\pgfqpoint{2.873933in}{1.077351in}}%
\pgfpathlineto{\pgfqpoint{2.955214in}{1.060004in}}%
\pgfpathlineto{\pgfqpoint{3.029922in}{1.042095in}}%
\pgfpathlineto{\pgfqpoint{3.097340in}{1.023775in}}%
\pgfpathlineto{\pgfqpoint{3.157795in}{1.005197in}}%
\pgfpathlineto{\pgfqpoint{3.211608in}{0.986501in}}%
\pgfpathlineto{\pgfqpoint{3.259088in}{0.967814in}}%
\pgfpathlineto{\pgfqpoint{3.300537in}{0.949250in}}%
\pgfpathlineto{\pgfqpoint{3.336245in}{0.930914in}}%
\pgfpathlineto{\pgfqpoint{3.366493in}{0.912893in}}%
\pgfpathlineto{\pgfqpoint{3.391553in}{0.895265in}}%
\pgfpathlineto{\pgfqpoint{3.411686in}{0.878095in}}%
\pgfpathlineto{\pgfqpoint{3.427143in}{0.861435in}}%
\pgfpathlineto{\pgfqpoint{3.438214in}{0.845331in}}%
\pgfpathlineto{\pgfqpoint{3.445578in}{0.829848in}}%
\pgfpathlineto{\pgfqpoint{3.449942in}{0.815009in}}%
\pgfpathlineto{\pgfqpoint{3.451856in}{0.800828in}}%
\pgfpathlineto{\pgfqpoint{3.451729in}{0.787318in}}%
\pgfpathlineto{\pgfqpoint{3.449824in}{0.774486in}}%
\pgfpathlineto{\pgfqpoint{3.446262in}{0.762338in}}%
\pgfpathlineto{\pgfqpoint{3.441018in}{0.750877in}}%
\pgfpathlineto{\pgfqpoint{3.433924in}{0.740102in}}%
\pgfpathlineto{\pgfqpoint{3.424668in}{0.730009in}}%
\pgfpathlineto{\pgfqpoint{3.412794in}{0.720590in}}%
\pgfpathlineto{\pgfqpoint{3.398093in}{0.711847in}}%
\pgfpathlineto{\pgfqpoint{3.381409in}{0.703807in}}%
\pgfpathlineto{\pgfqpoint{3.352804in}{0.693073in}}%
\pgfpathlineto{\pgfqpoint{3.319829in}{0.683936in}}%
\pgfpathlineto{\pgfqpoint{3.282306in}{0.676404in}}%
\pgfpathlineto{\pgfqpoint{3.239946in}{0.670487in}}%
\pgfpathlineto{\pgfqpoint{3.192355in}{0.666197in}}%
\pgfpathlineto{\pgfqpoint{3.139060in}{0.663550in}}%
\pgfpathlineto{\pgfqpoint{3.079925in}{0.662571in}}%
\pgfpathlineto{\pgfqpoint{3.014037in}{0.663329in}}%
\pgfpathlineto{\pgfqpoint{2.913939in}{0.667179in}}%
\pgfpathlineto{\pgfqpoint{2.798270in}{0.674459in}}%
\pgfpathlineto{\pgfqpoint{2.666022in}{0.685379in}}%
\pgfpathlineto{\pgfqpoint{2.516947in}{0.700160in}}%
\pgfpathlineto{\pgfqpoint{2.351560in}{0.719033in}}%
\pgfpathlineto{\pgfqpoint{2.171299in}{0.742262in}}%
\pgfpathlineto{\pgfqpoint{2.031333in}{0.762517in}}%
\pgfpathlineto{\pgfqpoint{1.893257in}{0.784961in}}%
\pgfpathlineto{\pgfqpoint{1.761624in}{0.809317in}}%
\pgfpathlineto{\pgfqpoint{1.679329in}{0.826456in}}%
\pgfpathlineto{\pgfqpoint{1.602486in}{0.844186in}}%
\pgfpathlineto{\pgfqpoint{1.531845in}{0.862387in}}%
\pgfpathlineto{\pgfqpoint{1.467994in}{0.880926in}}%
\pgfpathlineto{\pgfqpoint{1.411363in}{0.899658in}}%
\pgfpathlineto{\pgfqpoint{1.362219in}{0.918432in}}%
\pgfpathlineto{\pgfqpoint{1.320307in}{0.937093in}}%
\pgfpathlineto{\pgfqpoint{1.284696in}{0.955536in}}%
\pgfpathlineto{\pgfqpoint{1.254578in}{0.973676in}}%
\pgfpathlineto{\pgfqpoint{1.229312in}{0.991434in}}%
\pgfpathlineto{\pgfqpoint{1.208423in}{1.008743in}}%
\pgfpathlineto{\pgfqpoint{1.191607in}{1.025544in}}%
\pgfpathlineto{\pgfqpoint{1.178722in}{1.041787in}}%
\pgfpathlineto{\pgfqpoint{1.169713in}{1.057430in}}%
\pgfpathlineto{\pgfqpoint{1.163901in}{1.072442in}}%
\pgfpathlineto{\pgfqpoint{1.160840in}{1.086802in}}%
\pgfpathlineto{\pgfqpoint{1.160263in}{1.100495in}}%
\pgfpathlineto{\pgfqpoint{1.161972in}{1.113510in}}%
\pgfpathlineto{\pgfqpoint{1.165837in}{1.125838in}}%
\pgfpathlineto{\pgfqpoint{1.171800in}{1.137473in}}%
\pgfpathlineto{\pgfqpoint{1.179873in}{1.148415in}}%
\pgfpathlineto{\pgfqpoint{1.190070in}{1.158663in}}%
\pgfpathlineto{\pgfqpoint{1.202255in}{1.168215in}}%
\pgfpathlineto{\pgfqpoint{1.216357in}{1.177067in}}%
\pgfpathlineto{\pgfqpoint{1.232335in}{1.185218in}}%
\pgfpathlineto{\pgfqpoint{1.259798in}{1.196126in}}%
\pgfpathlineto{\pgfqpoint{1.291538in}{1.205446in}}%
\pgfpathlineto{\pgfqpoint{1.327777in}{1.213175in}}%
\pgfpathlineto{\pgfqpoint{1.368880in}{1.219311in}}%
\pgfpathlineto{\pgfqpoint{1.415332in}{1.223854in}}%
\pgfpathlineto{\pgfqpoint{1.467263in}{1.226774in}}%
\pgfpathlineto{\pgfqpoint{1.525251in}{1.228020in}}%
\pgfpathlineto{\pgfqpoint{1.590054in}{1.227540in}}%
\pgfpathlineto{\pgfqpoint{1.688248in}{1.224097in}}%
\pgfpathlineto{\pgfqpoint{1.801251in}{1.217284in}}%
\pgfpathlineto{\pgfqpoint{1.930332in}{1.206887in}}%
\pgfpathlineto{\pgfqpoint{2.076560in}{1.192660in}}%
\pgfpathlineto{\pgfqpoint{2.243011in}{1.174270in}}%
\pgfpathlineto{\pgfqpoint{2.378347in}{1.157632in}}%
\pgfpathlineto{\pgfqpoint{2.517025in}{1.138622in}}%
\pgfpathlineto{\pgfqpoint{2.654581in}{1.117371in}}%
\pgfpathlineto{\pgfqpoint{2.787141in}{1.094077in}}%
\pgfpathlineto{\pgfqpoint{2.871092in}{1.077542in}}%
\pgfpathlineto{\pgfqpoint{2.950522in}{1.060311in}}%
\pgfpathlineto{\pgfqpoint{3.024719in}{1.042492in}}%
\pgfpathlineto{\pgfqpoint{3.093086in}{1.024207in}}%
\pgfpathlineto{\pgfqpoint{3.155144in}{1.005592in}}%
\pgfpathlineto{\pgfqpoint{3.210529in}{0.986793in}}%
\pgfpathlineto{\pgfqpoint{3.258993in}{0.967973in}}%
\pgfpathlineto{\pgfqpoint{3.300406in}{0.949306in}}%
\pgfpathlineto{\pgfqpoint{3.334918in}{0.930957in}}%
\pgfpathlineto{\pgfqpoint{3.363720in}{0.912952in}}%
\pgfpathlineto{\pgfqpoint{3.387501in}{0.895353in}}%
\pgfpathlineto{\pgfqpoint{3.406775in}{0.878227in}}%
\pgfpathlineto{\pgfqpoint{3.421990in}{0.861628in}}%
\pgfpathlineto{\pgfqpoint{3.433564in}{0.845600in}}%
\pgfpathlineto{\pgfqpoint{3.441867in}{0.830176in}}%
\pgfpathlineto{\pgfqpoint{3.447206in}{0.815382in}}%
\pgfpathlineto{\pgfqpoint{3.449826in}{0.801239in}}%
\pgfpathlineto{\pgfqpoint{3.449848in}{0.787761in}}%
\pgfpathlineto{\pgfqpoint{3.447408in}{0.774959in}}%
\pgfpathlineto{\pgfqpoint{3.442765in}{0.762840in}}%
\pgfpathlineto{\pgfqpoint{3.436119in}{0.751413in}}%
\pgfpathlineto{\pgfqpoint{3.427607in}{0.740681in}}%
\pgfpathlineto{\pgfqpoint{3.417309in}{0.730647in}}%
\pgfpathlineto{\pgfqpoint{3.405239in}{0.721312in}}%
\pgfpathlineto{\pgfqpoint{3.391355in}{0.712674in}}%
\pgfpathlineto{\pgfqpoint{3.375552in}{0.704729in}}%
\pgfpathlineto{\pgfqpoint{3.357662in}{0.697473in}}%
\pgfpathlineto{\pgfqpoint{3.326622in}{0.687871in}}%
\pgfpathlineto{\pgfqpoint{3.290935in}{0.679841in}}%
\pgfpathlineto{\pgfqpoint{3.250460in}{0.673402in}}%
\pgfpathlineto{\pgfqpoint{3.204883in}{0.668576in}}%
\pgfpathlineto{\pgfqpoint{3.153807in}{0.665391in}}%
\pgfpathlineto{\pgfqpoint{3.096746in}{0.663882in}}%
\pgfpathlineto{\pgfqpoint{3.033132in}{0.664092in}}%
\pgfpathlineto{\pgfqpoint{2.936998in}{0.667129in}}%
\pgfpathlineto{\pgfqpoint{2.826568in}{0.673474in}}%
\pgfpathlineto{\pgfqpoint{2.699670in}{0.683389in}}%
\pgfpathlineto{\pgfqpoint{2.555287in}{0.697124in}}%
\pgfpathlineto{\pgfqpoint{2.394086in}{0.714893in}}%
\pgfpathlineto{\pgfqpoint{2.218395in}{0.736880in}}%
\pgfpathlineto{\pgfqpoint{2.079956in}{0.756210in}}%
\pgfpathlineto{\pgfqpoint{1.941148in}{0.777828in}}%
\pgfpathlineto{\pgfqpoint{1.806915in}{0.801487in}}%
\pgfpathlineto{\pgfqpoint{1.722075in}{0.818242in}}%
\pgfpathlineto{\pgfqpoint{1.642214in}{0.835657in}}%
\pgfpathlineto{\pgfqpoint{1.568250in}{0.853611in}}%
\pgfpathlineto{\pgfqpoint{1.500954in}{0.871968in}}%
\pgfpathlineto{\pgfqpoint{1.440950in}{0.890580in}}%
\pgfpathlineto{\pgfqpoint{1.388568in}{0.909285in}}%
\pgfpathlineto{\pgfqpoint{1.343241in}{0.927947in}}%
\pgfpathlineto{\pgfqpoint{1.304213in}{0.946461in}}%
\pgfpathlineto{\pgfqpoint{1.270850in}{0.964727in}}%
\pgfpathlineto{\pgfqpoint{1.242646in}{0.982659in}}%
\pgfpathlineto{\pgfqpoint{1.219216in}{1.000180in}}%
\pgfpathlineto{\pgfqpoint{1.200303in}{1.017223in}}%
\pgfpathlineto{\pgfqpoint{1.185772in}{1.033730in}}%
\pgfpathlineto{\pgfqpoint{1.175183in}{1.049653in}}%
\pgfpathlineto{\pgfqpoint{1.167817in}{1.064960in}}%
\pgfpathlineto{\pgfqpoint{1.163289in}{1.079628in}}%
\pgfpathlineto{\pgfqpoint{1.161305in}{1.093641in}}%
\pgfpathlineto{\pgfqpoint{1.161660in}{1.106984in}}%
\pgfpathlineto{\pgfqpoint{1.164236in}{1.119647in}}%
\pgfpathlineto{\pgfqpoint{1.169007in}{1.131623in}}%
\pgfpathlineto{\pgfqpoint{1.176033in}{1.142911in}}%
\pgfpathlineto{\pgfqpoint{1.185230in}{1.153506in}}%
\pgfpathlineto{\pgfqpoint{1.196418in}{1.163406in}}%
\pgfpathlineto{\pgfqpoint{1.209518in}{1.172606in}}%
\pgfpathlineto{\pgfqpoint{1.224482in}{1.181105in}}%
\pgfpathlineto{\pgfqpoint{1.250394in}{1.192536in}}%
\pgfpathlineto{\pgfqpoint{1.280541in}{1.202383in}}%
\pgfpathlineto{\pgfqpoint{1.315152in}{1.210645in}}%
\pgfpathlineto{\pgfqpoint{1.354609in}{1.217327in}}%
\pgfpathlineto{\pgfqpoint{1.399282in}{1.222424in}}%
\pgfpathlineto{\pgfqpoint{1.449296in}{1.225900in}}%
\pgfpathlineto{\pgfqpoint{1.505249in}{1.227714in}}%
\pgfpathlineto{\pgfqpoint{1.567768in}{1.227820in}}%
\pgfpathlineto{\pgfqpoint{1.637482in}{1.226156in}}%
\pgfpathlineto{\pgfqpoint{1.742708in}{1.221062in}}%
\pgfpathlineto{\pgfqpoint{1.863325in}{1.212499in}}%
\pgfpathlineto{\pgfqpoint{2.000808in}{1.200235in}}%
\pgfpathlineto{\pgfqpoint{2.155826in}{1.184024in}}%
\pgfpathlineto{\pgfqpoint{2.326792in}{1.163611in}}%
\pgfpathlineto{\pgfqpoint{2.462696in}{1.145497in}}%
\pgfpathlineto{\pgfqpoint{2.601569in}{1.125042in}}%
\pgfpathlineto{\pgfqpoint{2.738963in}{1.102380in}}%
\pgfpathlineto{\pgfqpoint{2.869820in}{1.077745in}}%
\pgfpathlineto{\pgfqpoint{2.951170in}{1.060437in}}%
\pgfpathlineto{\pgfqpoint{3.026607in}{1.042583in}}%
\pgfpathlineto{\pgfqpoint{3.095412in}{1.024319in}}%
\pgfpathlineto{\pgfqpoint{3.157147in}{1.005779in}}%
\pgfpathlineto{\pgfqpoint{3.211653in}{0.987095in}}%
\pgfpathlineto{\pgfqpoint{3.259048in}{0.968399in}}%
\pgfpathlineto{\pgfqpoint{3.299624in}{0.949825in}}%
\pgfpathlineto{\pgfqpoint{3.333576in}{0.931512in}}%
\pgfpathlineto{\pgfqpoint{3.361993in}{0.913533in}}%
\pgfpathlineto{\pgfqpoint{3.385861in}{0.895945in}}%
\pgfpathlineto{\pgfqpoint{3.405901in}{0.878804in}}%
\pgfpathlineto{\pgfqpoint{3.422570in}{0.862159in}}%
\pgfpathlineto{\pgfqpoint{3.436063in}{0.846054in}}%
\pgfpathlineto{\pgfqpoint{3.446310in}{0.830531in}}%
\pgfpathlineto{\pgfqpoint{3.452980in}{0.815623in}}%
\pgfpathlineto{\pgfqpoint{3.455596in}{0.801364in}}%
\pgfpathlineto{\pgfqpoint{3.455259in}{0.787781in}}%
\pgfpathlineto{\pgfqpoint{3.452606in}{0.774886in}}%
\pgfpathlineto{\pgfqpoint{3.447822in}{0.762685in}}%
\pgfpathlineto{\pgfqpoint{3.441039in}{0.751185in}}%
\pgfpathlineto{\pgfqpoint{3.432338in}{0.740386in}}%
\pgfpathlineto{\pgfqpoint{3.421751in}{0.730291in}}%
\pgfpathlineto{\pgfqpoint{3.409256in}{0.720898in}}%
\pgfpathlineto{\pgfqpoint{3.394780in}{0.712206in}}%
\pgfpathlineto{\pgfqpoint{3.378262in}{0.704210in}}%
\pgfpathlineto{\pgfqpoint{3.349850in}{0.693534in}}%
\pgfpathlineto{\pgfqpoint{3.317009in}{0.684444in}}%
\pgfpathlineto{\pgfqpoint{3.279559in}{0.676949in}}%
\pgfpathlineto{\pgfqpoint{3.237228in}{0.671060in}}%
\pgfpathlineto{\pgfqpoint{3.189652in}{0.666793in}}%
\pgfpathlineto{\pgfqpoint{3.172552in}{0.665734in}}%
\pgfpathlineto{\pgfqpoint{3.172552in}{0.665734in}}%
\pgfusepath{stroke}%
\end{pgfscope}%
\begin{pgfscope}%
\pgfpathrectangle{\pgfqpoint{0.562500in}{0.275000in}}{\pgfqpoint{3.487500in}{1.925000in}}%
\pgfusepath{clip}%
\pgfsetrectcap%
\pgfsetroundjoin%
\pgfsetlinewidth{1.505625pt}%
\definecolor{currentstroke}{rgb}{0.839216,0.152941,0.156863}%
\pgfsetstrokecolor{currentstroke}%
\pgfsetdash{}{0pt}%
\pgfpathmoveto{\pgfqpoint{1.117330in}{1.529167in}}%
\pgfpathlineto{\pgfqpoint{1.309063in}{1.516088in}}%
\pgfpathlineto{\pgfqpoint{1.518900in}{1.498617in}}%
\pgfpathlineto{\pgfqpoint{1.749525in}{1.476532in}}%
\pgfpathlineto{\pgfqpoint{1.998753in}{1.449670in}}%
\pgfpathlineto{\pgfqpoint{2.171911in}{1.429054in}}%
\pgfpathlineto{\pgfqpoint{2.346380in}{1.406273in}}%
\pgfpathlineto{\pgfqpoint{2.516049in}{1.381519in}}%
\pgfpathlineto{\pgfqpoint{2.675556in}{1.355051in}}%
\pgfpathlineto{\pgfqpoint{2.820781in}{1.327162in}}%
\pgfpathlineto{\pgfqpoint{2.887090in}{1.312784in}}%
\pgfpathlineto{\pgfqpoint{2.948844in}{1.298176in}}%
\pgfpathlineto{\pgfqpoint{3.005877in}{1.283383in}}%
\pgfpathlineto{\pgfqpoint{3.058102in}{1.268452in}}%
\pgfpathlineto{\pgfqpoint{3.105557in}{1.253437in}}%
\pgfpathlineto{\pgfqpoint{3.148648in}{1.238387in}}%
\pgfpathlineto{\pgfqpoint{3.187749in}{1.223338in}}%
\pgfpathlineto{\pgfqpoint{3.223191in}{1.208317in}}%
\pgfpathlineto{\pgfqpoint{3.255279in}{1.193355in}}%
\pgfpathlineto{\pgfqpoint{3.284293in}{1.178479in}}%
\pgfpathlineto{\pgfqpoint{3.310487in}{1.163716in}}%
\pgfpathlineto{\pgfqpoint{3.334091in}{1.149091in}}%
\pgfpathlineto{\pgfqpoint{3.374425in}{1.120349in}}%
\pgfpathlineto{\pgfqpoint{3.407445in}{1.092364in}}%
\pgfpathlineto{\pgfqpoint{3.434678in}{1.065198in}}%
\pgfpathlineto{\pgfqpoint{3.457201in}{1.038905in}}%
\pgfpathlineto{\pgfqpoint{3.475731in}{1.013531in}}%
\pgfpathlineto{\pgfqpoint{3.490892in}{0.989105in}}%
\pgfpathlineto{\pgfqpoint{3.503234in}{0.965625in}}%
\pgfpathlineto{\pgfqpoint{3.513141in}{0.943084in}}%
\pgfpathlineto{\pgfqpoint{3.520874in}{0.921476in}}%
\pgfpathlineto{\pgfqpoint{3.526575in}{0.900800in}}%
\pgfpathlineto{\pgfqpoint{3.530265in}{0.881052in}}%
\pgfpathlineto{\pgfqpoint{3.532019in}{0.862225in}}%
\pgfpathlineto{\pgfqpoint{3.532048in}{0.844302in}}%
\pgfpathlineto{\pgfqpoint{3.530444in}{0.827267in}}%
\pgfpathlineto{\pgfqpoint{3.527266in}{0.811104in}}%
\pgfpathlineto{\pgfqpoint{3.522537in}{0.795799in}}%
\pgfpathlineto{\pgfqpoint{3.516248in}{0.781339in}}%
\pgfpathlineto{\pgfqpoint{3.508354in}{0.767713in}}%
\pgfpathlineto{\pgfqpoint{3.498771in}{0.754907in}}%
\pgfpathlineto{\pgfqpoint{3.487513in}{0.742908in}}%
\pgfpathlineto{\pgfqpoint{3.474670in}{0.731708in}}%
\pgfpathlineto{\pgfqpoint{3.460299in}{0.721297in}}%
\pgfpathlineto{\pgfqpoint{3.435908in}{0.707143in}}%
\pgfpathlineto{\pgfqpoint{3.408022in}{0.694721in}}%
\pgfpathlineto{\pgfqpoint{3.376365in}{0.684005in}}%
\pgfpathlineto{\pgfqpoint{3.340476in}{0.674970in}}%
\pgfpathlineto{\pgfqpoint{3.299717in}{0.667593in}}%
\pgfpathlineto{\pgfqpoint{3.253933in}{0.661882in}}%
\pgfpathlineto{\pgfqpoint{3.202924in}{0.657868in}}%
\pgfpathlineto{\pgfqpoint{3.146113in}{0.655578in}}%
\pgfpathlineto{\pgfqpoint{3.082886in}{0.655054in}}%
\pgfpathlineto{\pgfqpoint{3.012592in}{0.656348in}}%
\pgfpathlineto{\pgfqpoint{2.906676in}{0.661014in}}%
\pgfpathlineto{\pgfqpoint{2.785225in}{0.669217in}}%
\pgfpathlineto{\pgfqpoint{2.646533in}{0.681185in}}%
\pgfpathlineto{\pgfqpoint{2.489776in}{0.697196in}}%
\pgfpathlineto{\pgfqpoint{2.316276in}{0.717476in}}%
\pgfpathlineto{\pgfqpoint{2.177636in}{0.735564in}}%
\pgfpathlineto{\pgfqpoint{2.035427in}{0.756092in}}%
\pgfpathlineto{\pgfqpoint{1.894329in}{0.778911in}}%
\pgfpathlineto{\pgfqpoint{1.759284in}{0.803763in}}%
\pgfpathlineto{\pgfqpoint{1.675198in}{0.821283in}}%
\pgfpathlineto{\pgfqpoint{1.597603in}{0.839424in}}%
\pgfpathlineto{\pgfqpoint{1.526921in}{0.858016in}}%
\pgfpathlineto{\pgfqpoint{1.463312in}{0.876897in}}%
\pgfpathlineto{\pgfqpoint{1.406815in}{0.895919in}}%
\pgfpathlineto{\pgfqpoint{1.357341in}{0.914946in}}%
\pgfpathlineto{\pgfqpoint{1.314680in}{0.933856in}}%
\pgfpathlineto{\pgfqpoint{1.278494in}{0.952544in}}%
\pgfpathlineto{\pgfqpoint{1.248320in}{0.970914in}}%
\pgfpathlineto{\pgfqpoint{1.223573in}{0.988889in}}%
\pgfpathlineto{\pgfqpoint{1.203541in}{1.006401in}}%
\pgfpathlineto{\pgfqpoint{1.187950in}{1.023377in}}%
\pgfpathlineto{\pgfqpoint{1.176345in}{1.039765in}}%
\pgfpathlineto{\pgfqpoint{1.167915in}{1.055542in}}%
\pgfpathlineto{\pgfqpoint{1.162045in}{1.070690in}}%
\pgfpathlineto{\pgfqpoint{1.158314in}{1.085191in}}%
\pgfpathlineto{\pgfqpoint{1.156495in}{1.099033in}}%
\pgfpathlineto{\pgfqpoint{1.156555in}{1.112205in}}%
\pgfpathlineto{\pgfqpoint{1.158654in}{1.124702in}}%
\pgfpathlineto{\pgfqpoint{1.163147in}{1.136520in}}%
\pgfpathlineto{\pgfqpoint{1.170583in}{1.147658in}}%
\pgfpathlineto{\pgfqpoint{1.180948in}{1.158108in}}%
\pgfpathlineto{\pgfqpoint{1.193261in}{1.167847in}}%
\pgfpathlineto{\pgfqpoint{1.207449in}{1.176874in}}%
\pgfpathlineto{\pgfqpoint{1.223486in}{1.185190in}}%
\pgfpathlineto{\pgfqpoint{1.251011in}{1.196328in}}%
\pgfpathlineto{\pgfqpoint{1.282809in}{1.205863in}}%
\pgfpathlineto{\pgfqpoint{1.319124in}{1.213796in}}%
\pgfpathlineto{\pgfqpoint{1.360334in}{1.220130in}}%
\pgfpathlineto{\pgfqpoint{1.406677in}{1.224853in}}%
\pgfpathlineto{\pgfqpoint{1.458484in}{1.227933in}}%
\pgfpathlineto{\pgfqpoint{1.516394in}{1.229330in}}%
\pgfpathlineto{\pgfqpoint{1.581048in}{1.228993in}}%
\pgfpathlineto{\pgfqpoint{1.678840in}{1.225737in}}%
\pgfpathlineto{\pgfqpoint{1.791248in}{1.219108in}}%
\pgfpathlineto{\pgfqpoint{1.919763in}{1.208893in}}%
\pgfpathlineto{\pgfqpoint{2.065735in}{1.194854in}}%
\pgfpathlineto{\pgfqpoint{2.228976in}{1.176727in}}%
\pgfpathlineto{\pgfqpoint{2.406469in}{1.154317in}}%
\pgfpathlineto{\pgfqpoint{2.545183in}{1.134716in}}%
\pgfpathlineto{\pgfqpoint{2.684330in}{1.112840in}}%
\pgfpathlineto{\pgfqpoint{2.818882in}{1.088874in}}%
\pgfpathlineto{\pgfqpoint{2.903566in}{1.071896in}}%
\pgfpathlineto{\pgfqpoint{2.982854in}{1.054285in}}%
\pgfpathlineto{\pgfqpoint{3.055868in}{1.036184in}}%
\pgfpathlineto{\pgfqpoint{3.122006in}{1.017731in}}%
\pgfpathlineto{\pgfqpoint{3.180942in}{0.999062in}}%
\pgfpathlineto{\pgfqpoint{3.232629in}{0.980307in}}%
\pgfpathlineto{\pgfqpoint{3.277292in}{0.961592in}}%
\pgfpathlineto{\pgfqpoint{3.315370in}{0.943045in}}%
\pgfpathlineto{\pgfqpoint{3.347041in}{0.924798in}}%
\pgfpathlineto{\pgfqpoint{3.373315in}{0.906923in}}%
\pgfpathlineto{\pgfqpoint{3.395187in}{0.889473in}}%
\pgfpathlineto{\pgfqpoint{3.413393in}{0.872498in}}%
\pgfpathlineto{\pgfqpoint{3.428402in}{0.856043in}}%
\pgfpathlineto{\pgfqpoint{3.440424in}{0.840147in}}%
\pgfpathlineto{\pgfqpoint{3.449406in}{0.824846in}}%
\pgfpathlineto{\pgfqpoint{3.455029in}{0.810172in}}%
\pgfpathlineto{\pgfqpoint{3.456764in}{0.796150in}}%
\pgfpathlineto{\pgfqpoint{3.455481in}{0.782809in}}%
\pgfpathlineto{\pgfqpoint{3.451934in}{0.770160in}}%
\pgfpathlineto{\pgfqpoint{3.446293in}{0.758210in}}%
\pgfpathlineto{\pgfqpoint{3.438682in}{0.746962in}}%
\pgfpathlineto{\pgfqpoint{3.429178in}{0.736419in}}%
\pgfpathlineto{\pgfqpoint{3.417809in}{0.726581in}}%
\pgfpathlineto{\pgfqpoint{3.404556in}{0.717447in}}%
\pgfpathlineto{\pgfqpoint{3.389352in}{0.709014in}}%
\pgfpathlineto{\pgfqpoint{3.372106in}{0.701279in}}%
\pgfpathlineto{\pgfqpoint{3.342539in}{0.690989in}}%
\pgfpathlineto{\pgfqpoint{3.308509in}{0.682283in}}%
\pgfpathlineto{\pgfqpoint{3.269837in}{0.675175in}}%
\pgfpathlineto{\pgfqpoint{3.226246in}{0.669683in}}%
\pgfpathlineto{\pgfqpoint{3.177361in}{0.665829in}}%
\pgfpathlineto{\pgfqpoint{3.122707in}{0.663639in}}%
\pgfpathlineto{\pgfqpoint{3.061712in}{0.663144in}}%
\pgfpathlineto{\pgfqpoint{2.969774in}{0.665148in}}%
\pgfpathlineto{\pgfqpoint{2.864197in}{0.670342in}}%
\pgfpathlineto{\pgfqpoint{2.741872in}{0.679112in}}%
\pgfpathlineto{\pgfqpoint{2.601543in}{0.691755in}}%
\pgfpathlineto{\pgfqpoint{2.443992in}{0.708467in}}%
\pgfpathlineto{\pgfqpoint{2.272039in}{0.729338in}}%
\pgfpathlineto{\pgfqpoint{2.136469in}{0.747717in}}%
\pgfpathlineto{\pgfqpoint{1.998291in}{0.768386in}}%
\pgfpathlineto{\pgfqpoint{1.861324in}{0.791268in}}%
\pgfpathlineto{\pgfqpoint{1.773567in}{0.807633in}}%
\pgfpathlineto{\pgfqpoint{1.690275in}{0.824725in}}%
\pgfpathlineto{\pgfqpoint{1.612506in}{0.842407in}}%
\pgfpathlineto{\pgfqpoint{1.541065in}{0.860543in}}%
\pgfpathlineto{\pgfqpoint{1.476502in}{0.879001in}}%
\pgfpathlineto{\pgfqpoint{1.419114in}{0.897650in}}%
\pgfpathlineto{\pgfqpoint{1.368941in}{0.916362in}}%
\pgfpathlineto{\pgfqpoint{1.325771in}{0.935012in}}%
\pgfpathlineto{\pgfqpoint{1.289227in}{0.953472in}}%
\pgfpathlineto{\pgfqpoint{1.258986in}{0.971609in}}%
\pgfpathlineto{\pgfqpoint{1.233948in}{0.989358in}}%
\pgfpathlineto{\pgfqpoint{1.213136in}{1.006668in}}%
\pgfpathlineto{\pgfqpoint{1.195840in}{1.023493in}}%
\pgfpathlineto{\pgfqpoint{1.181613in}{1.039790in}}%
\pgfpathlineto{\pgfqpoint{1.170277in}{1.055521in}}%
\pgfpathlineto{\pgfqpoint{1.161917in}{1.070653in}}%
\pgfpathlineto{\pgfqpoint{1.156884in}{1.085156in}}%
\pgfpathlineto{\pgfqpoint{1.155748in}{1.099004in}}%
\pgfpathlineto{\pgfqpoint{1.157623in}{1.112170in}}%
\pgfpathlineto{\pgfqpoint{1.161729in}{1.124641in}}%
\pgfpathlineto{\pgfqpoint{1.167906in}{1.136413in}}%
\pgfpathlineto{\pgfqpoint{1.176036in}{1.147481in}}%
\pgfpathlineto{\pgfqpoint{1.186048in}{1.157845in}}%
\pgfpathlineto{\pgfqpoint{1.197919in}{1.167503in}}%
\pgfpathlineto{\pgfqpoint{1.211667in}{1.176456in}}%
\pgfpathlineto{\pgfqpoint{1.227359in}{1.184708in}}%
\pgfpathlineto{\pgfqpoint{1.254707in}{1.195778in}}%
\pgfpathlineto{\pgfqpoint{1.286505in}{1.205270in}}%
\pgfpathlineto{\pgfqpoint{1.322853in}{1.213173in}}%
\pgfpathlineto{\pgfqpoint{1.363978in}{1.219471in}}%
\pgfpathlineto{\pgfqpoint{1.410204in}{1.224143in}}%
\pgfpathlineto{\pgfqpoint{1.461952in}{1.227164in}}%
\pgfpathlineto{\pgfqpoint{1.519736in}{1.228503in}}%
\pgfpathlineto{\pgfqpoint{1.584169in}{1.228125in}}%
\pgfpathlineto{\pgfqpoint{1.681187in}{1.224887in}}%
\pgfpathlineto{\pgfqpoint{1.792875in}{1.218308in}}%
\pgfpathlineto{\pgfqpoint{1.921671in}{1.208088in}}%
\pgfpathlineto{\pgfqpoint{2.068189in}{1.193987in}}%
\pgfpathlineto{\pgfqpoint{2.231187in}{1.175824in}}%
\pgfpathlineto{\pgfqpoint{2.407569in}{1.153480in}}%
\pgfpathlineto{\pgfqpoint{2.545713in}{1.133938in}}%
\pgfpathlineto{\pgfqpoint{2.684628in}{1.112036in}}%
\pgfpathlineto{\pgfqpoint{2.818824in}{1.088083in}}%
\pgfpathlineto{\pgfqpoint{2.903379in}{1.071167in}}%
\pgfpathlineto{\pgfqpoint{2.982721in}{1.053630in}}%
\pgfpathlineto{\pgfqpoint{3.055990in}{1.035597in}}%
\pgfpathlineto{\pgfqpoint{3.122545in}{1.017194in}}%
\pgfpathlineto{\pgfqpoint{3.181960in}{0.998551in}}%
\pgfpathlineto{\pgfqpoint{3.234023in}{0.979802in}}%
\pgfpathlineto{\pgfqpoint{3.278741in}{0.961083in}}%
\pgfpathlineto{\pgfqpoint{3.316411in}{0.942546in}}%
\pgfpathlineto{\pgfqpoint{3.347993in}{0.924301in}}%
\pgfpathlineto{\pgfqpoint{3.374481in}{0.906416in}}%
\pgfpathlineto{\pgfqpoint{3.396640in}{0.888955in}}%
\pgfpathlineto{\pgfqpoint{3.415004in}{0.871973in}}%
\pgfpathlineto{\pgfqpoint{3.429876in}{0.855520in}}%
\pgfpathlineto{\pgfqpoint{3.441327in}{0.839640in}}%
\pgfpathlineto{\pgfqpoint{3.449200in}{0.824369in}}%
\pgfpathlineto{\pgfqpoint{3.453263in}{0.809738in}}%
\pgfpathlineto{\pgfqpoint{3.454453in}{0.795774in}}%
\pgfpathlineto{\pgfqpoint{3.453232in}{0.782491in}}%
\pgfpathlineto{\pgfqpoint{3.449811in}{0.769896in}}%
\pgfpathlineto{\pgfqpoint{3.444338in}{0.757996in}}%
\pgfpathlineto{\pgfqpoint{3.436908in}{0.746795in}}%
\pgfpathlineto{\pgfqpoint{3.427554in}{0.736294in}}%
\pgfpathlineto{\pgfqpoint{3.416254in}{0.726494in}}%
\pgfpathlineto{\pgfqpoint{3.402928in}{0.717391in}}%
\pgfpathlineto{\pgfqpoint{3.387556in}{0.708984in}}%
\pgfpathlineto{\pgfqpoint{3.360876in}{0.697684in}}%
\pgfpathlineto{\pgfqpoint{3.329838in}{0.687967in}}%
\pgfpathlineto{\pgfqpoint{3.294327in}{0.679841in}}%
\pgfpathlineto{\pgfqpoint{3.254117in}{0.673321in}}%
\pgfpathlineto{\pgfqpoint{3.208869in}{0.668419in}}%
\pgfpathlineto{\pgfqpoint{3.158136in}{0.665155in}}%
\pgfpathlineto{\pgfqpoint{3.101362in}{0.663545in}}%
\pgfpathlineto{\pgfqpoint{3.038420in}{0.663612in}}%
\pgfpathlineto{\pgfqpoint{2.968421in}{0.665438in}}%
\pgfpathlineto{\pgfqpoint{2.862051in}{0.670790in}}%
\pgfpathlineto{\pgfqpoint{2.739165in}{0.679685in}}%
\pgfpathlineto{\pgfqpoint{2.599132in}{0.692338in}}%
\pgfpathlineto{\pgfqpoint{2.442531in}{0.708947in}}%
\pgfpathlineto{\pgfqpoint{2.271152in}{0.729692in}}%
\pgfpathlineto{\pgfqpoint{2.134723in}{0.748067in}}%
\pgfpathlineto{\pgfqpoint{1.995231in}{0.768850in}}%
\pgfpathlineto{\pgfqpoint{1.858252in}{0.791796in}}%
\pgfpathlineto{\pgfqpoint{1.728595in}{0.816590in}}%
\pgfpathlineto{\pgfqpoint{1.648167in}{0.833968in}}%
\pgfpathlineto{\pgfqpoint{1.573627in}{0.851888in}}%
\pgfpathlineto{\pgfqpoint{1.505691in}{0.870219in}}%
\pgfpathlineto{\pgfqpoint{1.444882in}{0.888826in}}%
\pgfpathlineto{\pgfqpoint{1.391535in}{0.907559in}}%
\pgfpathlineto{\pgfqpoint{1.345658in}{0.926259in}}%
\pgfpathlineto{\pgfqpoint{1.306447in}{0.944804in}}%
\pgfpathlineto{\pgfqpoint{1.273015in}{0.963102in}}%
\pgfpathlineto{\pgfqpoint{1.244658in}{0.981073in}}%
\pgfpathlineto{\pgfqpoint{1.220855in}{0.998641in}}%
\pgfpathlineto{\pgfqpoint{1.201271in}{1.015741in}}%
\pgfpathlineto{\pgfqpoint{1.185749in}{1.032318in}}%
\pgfpathlineto{\pgfqpoint{1.174321in}{1.048321in}}%
\pgfpathlineto{\pgfqpoint{1.166755in}{1.063711in}}%
\pgfpathlineto{\pgfqpoint{1.162164in}{1.078459in}}%
\pgfpathlineto{\pgfqpoint{1.160200in}{1.092547in}}%
\pgfpathlineto{\pgfqpoint{1.160608in}{1.105964in}}%
\pgfpathlineto{\pgfqpoint{1.163206in}{1.118698in}}%
\pgfpathlineto{\pgfqpoint{1.167890in}{1.130743in}}%
\pgfpathlineto{\pgfqpoint{1.174629in}{1.142094in}}%
\pgfpathlineto{\pgfqpoint{1.183470in}{1.152751in}}%
\pgfpathlineto{\pgfqpoint{1.194465in}{1.162715in}}%
\pgfpathlineto{\pgfqpoint{1.207445in}{1.171982in}}%
\pgfpathlineto{\pgfqpoint{1.222339in}{1.180549in}}%
\pgfpathlineto{\pgfqpoint{1.248213in}{1.192085in}}%
\pgfpathlineto{\pgfqpoint{1.278344in}{1.202035in}}%
\pgfpathlineto{\pgfqpoint{1.312875in}{1.210394in}}%
\pgfpathlineto{\pgfqpoint{1.352084in}{1.217157in}}%
\pgfpathlineto{\pgfqpoint{1.396383in}{1.222320in}}%
\pgfpathlineto{\pgfqpoint{1.446200in}{1.225875in}}%
\pgfpathlineto{\pgfqpoint{1.501703in}{1.227782in}}%
\pgfpathlineto{\pgfqpoint{1.563690in}{1.227986in}}%
\pgfpathlineto{\pgfqpoint{1.632951in}{1.226422in}}%
\pgfpathlineto{\pgfqpoint{1.737796in}{1.221463in}}%
\pgfpathlineto{\pgfqpoint{1.858149in}{1.213036in}}%
\pgfpathlineto{\pgfqpoint{1.995047in}{1.200918in}}%
\pgfpathlineto{\pgfqpoint{2.149221in}{1.184854in}}%
\pgfpathlineto{\pgfqpoint{2.322017in}{1.164535in}}%
\pgfpathlineto{\pgfqpoint{2.459315in}{1.146462in}}%
\pgfpathlineto{\pgfqpoint{2.597913in}{1.126066in}}%
\pgfpathlineto{\pgfqpoint{2.733583in}{1.103521in}}%
\pgfpathlineto{\pgfqpoint{2.862602in}{1.079076in}}%
\pgfpathlineto{\pgfqpoint{2.943308in}{1.061879in}}%
\pgfpathlineto{\pgfqpoint{3.018785in}{1.044093in}}%
\pgfpathlineto{\pgfqpoint{3.088304in}{1.025845in}}%
\pgfpathlineto{\pgfqpoint{3.151232in}{1.007277in}}%
\pgfpathlineto{\pgfqpoint{3.207037in}{0.988547in}}%
\pgfpathlineto{\pgfqpoint{3.255291in}{0.969825in}}%
\pgfpathlineto{\pgfqpoint{3.296460in}{0.951240in}}%
\pgfpathlineto{\pgfqpoint{3.331470in}{0.932880in}}%
\pgfpathlineto{\pgfqpoint{3.360891in}{0.914836in}}%
\pgfpathlineto{\pgfqpoint{3.385241in}{0.897189in}}%
\pgfpathlineto{\pgfqpoint{3.404992in}{0.880010in}}%
\pgfpathlineto{\pgfqpoint{3.420595in}{0.863353in}}%
\pgfpathlineto{\pgfqpoint{3.432524in}{0.847263in}}%
\pgfpathlineto{\pgfqpoint{3.441162in}{0.831773in}}%
\pgfpathlineto{\pgfqpoint{3.446820in}{0.816910in}}%
\pgfpathlineto{\pgfqpoint{3.449737in}{0.802697in}}%
\pgfpathlineto{\pgfqpoint{3.450061in}{0.789147in}}%
\pgfpathlineto{\pgfqpoint{3.447902in}{0.776273in}}%
\pgfpathlineto{\pgfqpoint{3.443500in}{0.764082in}}%
\pgfpathlineto{\pgfqpoint{3.437054in}{0.752580in}}%
\pgfpathlineto{\pgfqpoint{3.428704in}{0.741774in}}%
\pgfpathlineto{\pgfqpoint{3.418537in}{0.731666in}}%
\pgfpathlineto{\pgfqpoint{3.406578in}{0.722256in}}%
\pgfpathlineto{\pgfqpoint{3.392801in}{0.713544in}}%
\pgfpathlineto{\pgfqpoint{3.377117in}{0.705527in}}%
\pgfpathlineto{\pgfqpoint{3.359385in}{0.698200in}}%
\pgfpathlineto{\pgfqpoint{3.328658in}{0.688496in}}%
\pgfpathlineto{\pgfqpoint{3.293291in}{0.680363in}}%
\pgfpathlineto{\pgfqpoint{3.253155in}{0.673819in}}%
\pgfpathlineto{\pgfqpoint{3.207941in}{0.668885in}}%
\pgfpathlineto{\pgfqpoint{3.157257in}{0.665591in}}%
\pgfpathlineto{\pgfqpoint{3.100624in}{0.663970in}}%
\pgfpathlineto{\pgfqpoint{3.037479in}{0.664064in}}%
\pgfpathlineto{\pgfqpoint{2.942032in}{0.666941in}}%
\pgfpathlineto{\pgfqpoint{2.832380in}{0.673115in}}%
\pgfpathlineto{\pgfqpoint{2.706368in}{0.682844in}}%
\pgfpathlineto{\pgfqpoint{2.562884in}{0.696380in}}%
\pgfpathlineto{\pgfqpoint{2.402505in}{0.713940in}}%
\pgfpathlineto{\pgfqpoint{2.227481in}{0.735709in}}%
\pgfpathlineto{\pgfqpoint{2.089282in}{0.754875in}}%
\pgfpathlineto{\pgfqpoint{1.950366in}{0.776343in}}%
\pgfpathlineto{\pgfqpoint{1.815707in}{0.799871in}}%
\pgfpathlineto{\pgfqpoint{1.730428in}{0.816552in}}%
\pgfpathlineto{\pgfqpoint{1.650023in}{0.833905in}}%
\pgfpathlineto{\pgfqpoint{1.575429in}{0.851810in}}%
\pgfpathlineto{\pgfqpoint{1.507439in}{0.870132in}}%
\pgfpathlineto{\pgfqpoint{1.446702in}{0.888724in}}%
\pgfpathlineto{\pgfqpoint{1.393592in}{0.907423in}}%
\pgfpathlineto{\pgfqpoint{1.347582in}{0.926093in}}%
\pgfpathlineto{\pgfqpoint{1.307925in}{0.944624in}}%
\pgfpathlineto{\pgfqpoint{1.273992in}{0.962919in}}%
\pgfpathlineto{\pgfqpoint{1.245275in}{0.980887in}}%
\pgfpathlineto{\pgfqpoint{1.221384in}{0.998451in}}%
\pgfpathlineto{\pgfqpoint{1.202051in}{1.015543in}}%
\pgfpathlineto{\pgfqpoint{1.187128in}{1.032104in}}%
\pgfpathlineto{\pgfqpoint{1.176192in}{1.048085in}}%
\pgfpathlineto{\pgfqpoint{1.168514in}{1.063453in}}%
\pgfpathlineto{\pgfqpoint{1.163704in}{1.078186in}}%
\pgfpathlineto{\pgfqpoint{1.161459in}{1.092264in}}%
\pgfpathlineto{\pgfqpoint{1.161569in}{1.105674in}}%
\pgfpathlineto{\pgfqpoint{1.163913in}{1.118404in}}%
\pgfpathlineto{\pgfqpoint{1.168461in}{1.130449in}}%
\pgfpathlineto{\pgfqpoint{1.175272in}{1.141805in}}%
\pgfpathlineto{\pgfqpoint{1.184266in}{1.152470in}}%
\pgfpathlineto{\pgfqpoint{1.195258in}{1.162439in}}%
\pgfpathlineto{\pgfqpoint{1.208164in}{1.171709in}}%
\pgfpathlineto{\pgfqpoint{1.222936in}{1.180278in}}%
\pgfpathlineto{\pgfqpoint{1.248556in}{1.191814in}}%
\pgfpathlineto{\pgfqpoint{1.278402in}{1.201766in}}%
\pgfpathlineto{\pgfqpoint{1.312697in}{1.210134in}}%
\pgfpathlineto{\pgfqpoint{1.351824in}{1.216922in}}%
\pgfpathlineto{\pgfqpoint{1.396147in}{1.222127in}}%
\pgfpathlineto{\pgfqpoint{1.445788in}{1.225713in}}%
\pgfpathlineto{\pgfqpoint{1.501331in}{1.227640in}}%
\pgfpathlineto{\pgfqpoint{1.563396in}{1.227861in}}%
\pgfpathlineto{\pgfqpoint{1.632606in}{1.226318in}}%
\pgfpathlineto{\pgfqpoint{1.737084in}{1.221392in}}%
\pgfpathlineto{\pgfqpoint{1.856879in}{1.213008in}}%
\pgfpathlineto{\pgfqpoint{1.993489in}{1.200937in}}%
\pgfpathlineto{\pgfqpoint{2.147622in}{1.184928in}}%
\pgfpathlineto{\pgfqpoint{2.317873in}{1.164730in}}%
\pgfpathlineto{\pgfqpoint{2.453456in}{1.146780in}}%
\pgfpathlineto{\pgfqpoint{2.592261in}{1.126481in}}%
\pgfpathlineto{\pgfqpoint{2.729908in}{1.103955in}}%
\pgfpathlineto{\pgfqpoint{2.861382in}{1.079444in}}%
\pgfpathlineto{\pgfqpoint{2.943286in}{1.062207in}}%
\pgfpathlineto{\pgfqpoint{3.019355in}{1.044409in}}%
\pgfpathlineto{\pgfqpoint{3.088842in}{1.026180in}}%
\pgfpathlineto{\pgfqpoint{3.151281in}{1.007657in}}%
\pgfpathlineto{\pgfqpoint{3.206484in}{0.988973in}}%
\pgfpathlineto{\pgfqpoint{3.254544in}{0.970267in}}%
\pgfpathlineto{\pgfqpoint{3.295453in}{0.951695in}}%
\pgfpathlineto{\pgfqpoint{3.329943in}{0.933367in}}%
\pgfpathlineto{\pgfqpoint{3.359136in}{0.915350in}}%
\pgfpathlineto{\pgfqpoint{3.383895in}{0.897708in}}%
\pgfpathlineto{\pgfqpoint{3.404817in}{0.880500in}}%
\pgfpathlineto{\pgfqpoint{3.422238in}{0.863780in}}%
\pgfpathlineto{\pgfqpoint{3.436232in}{0.847598in}}%
\pgfpathlineto{\pgfqpoint{3.446609in}{0.831997in}}%
\pgfpathlineto{\pgfqpoint{3.452919in}{0.817016in}}%
\pgfpathlineto{\pgfqpoint{3.455352in}{0.802693in}}%
\pgfpathlineto{\pgfqpoint{3.455234in}{0.789047in}}%
\pgfpathlineto{\pgfqpoint{3.452820in}{0.776088in}}%
\pgfpathlineto{\pgfqpoint{3.448290in}{0.763823in}}%
\pgfpathlineto{\pgfqpoint{3.441763in}{0.752257in}}%
\pgfpathlineto{\pgfqpoint{3.433307in}{0.741393in}}%
\pgfpathlineto{\pgfqpoint{3.422931in}{0.731231in}}%
\pgfpathlineto{\pgfqpoint{3.410587in}{0.721769in}}%
\pgfpathlineto{\pgfqpoint{3.396188in}{0.713005in}}%
\pgfpathlineto{\pgfqpoint{3.379815in}{0.704940in}}%
\pgfpathlineto{\pgfqpoint{3.351644in}{0.694157in}}%
\pgfpathlineto{\pgfqpoint{3.319092in}{0.684961in}}%
\pgfpathlineto{\pgfqpoint{3.282006in}{0.677362in}}%
\pgfpathlineto{\pgfqpoint{3.240125in}{0.671374in}}%
\pgfpathlineto{\pgfqpoint{3.193073in}{0.667013in}}%
\pgfpathlineto{\pgfqpoint{3.140364in}{0.664295in}}%
\pgfpathlineto{\pgfqpoint{3.081484in}{0.663240in}}%
\pgfpathlineto{\pgfqpoint{3.016313in}{0.663880in}}%
\pgfpathlineto{\pgfqpoint{2.917569in}{0.667544in}}%
\pgfpathlineto{\pgfqpoint{2.803063in}{0.674640in}}%
\pgfpathlineto{\pgfqpoint{2.671539in}{0.685393in}}%
\pgfpathlineto{\pgfqpoint{2.522975in}{0.700009in}}%
\pgfpathlineto{\pgfqpoint{2.358591in}{0.718674in}}%
\pgfpathlineto{\pgfqpoint{2.180846in}{0.741558in}}%
\pgfpathlineto{\pgfqpoint{2.041734in}{0.761562in}}%
\pgfpathlineto{\pgfqpoint{1.903277in}{0.783828in}}%
\pgfpathlineto{\pgfqpoint{1.770643in}{0.808055in}}%
\pgfpathlineto{\pgfqpoint{1.687566in}{0.825124in}}%
\pgfpathlineto{\pgfqpoint{1.609963in}{0.842795in}}%
\pgfpathlineto{\pgfqpoint{1.538642in}{0.860944in}}%
\pgfpathlineto{\pgfqpoint{1.474221in}{0.879437in}}%
\pgfpathlineto{\pgfqpoint{1.417124in}{0.898132in}}%
\pgfpathlineto{\pgfqpoint{1.367573in}{0.916873in}}%
\pgfpathlineto{\pgfqpoint{1.325161in}{0.935512in}}%
\pgfpathlineto{\pgfqpoint{1.288944in}{0.953948in}}%
\pgfpathlineto{\pgfqpoint{1.258127in}{0.972096in}}%
\pgfpathlineto{\pgfqpoint{1.232098in}{0.989878in}}%
\pgfpathlineto{\pgfqpoint{1.210429in}{1.007224in}}%
\pgfpathlineto{\pgfqpoint{1.192876in}{1.024073in}}%
\pgfpathlineto{\pgfqpoint{1.179380in}{1.040372in}}%
\pgfpathlineto{\pgfqpoint{1.170005in}{1.056077in}}%
\pgfpathlineto{\pgfqpoint{1.163988in}{1.071151in}}%
\pgfpathlineto{\pgfqpoint{1.160755in}{1.085574in}}%
\pgfpathlineto{\pgfqpoint{1.160012in}{1.099331in}}%
\pgfpathlineto{\pgfqpoint{1.161539in}{1.112410in}}%
\pgfpathlineto{\pgfqpoint{1.165193in}{1.124802in}}%
\pgfpathlineto{\pgfqpoint{1.170905in}{1.136502in}}%
\pgfpathlineto{\pgfqpoint{1.178684in}{1.147508in}}%
\pgfpathlineto{\pgfqpoint{1.188611in}{1.157820in}}%
\pgfpathlineto{\pgfqpoint{1.200615in}{1.167438in}}%
\pgfpathlineto{\pgfqpoint{1.214560in}{1.176357in}}%
\pgfpathlineto{\pgfqpoint{1.230401in}{1.184575in}}%
\pgfpathlineto{\pgfqpoint{1.257691in}{1.195585in}}%
\pgfpathlineto{\pgfqpoint{1.289271in}{1.205008in}}%
\pgfpathlineto{\pgfqpoint{1.325328in}{1.212837in}}%
\pgfpathlineto{\pgfqpoint{1.366184in}{1.219068in}}%
\pgfpathlineto{\pgfqpoint{1.412293in}{1.223698in}}%
\pgfpathlineto{\pgfqpoint{1.463960in}{1.226711in}}%
\pgfpathlineto{\pgfqpoint{1.521528in}{1.228058in}}%
\pgfpathlineto{\pgfqpoint{1.585853in}{1.227681in}}%
\pgfpathlineto{\pgfqpoint{1.683431in}{1.224380in}}%
\pgfpathlineto{\pgfqpoint{1.795884in}{1.217715in}}%
\pgfpathlineto{\pgfqpoint{1.924406in}{1.207473in}}%
\pgfpathlineto{\pgfqpoint{2.069868in}{1.193416in}}%
\pgfpathlineto{\pgfqpoint{2.233396in}{1.175262in}}%
\pgfpathlineto{\pgfqpoint{2.413004in}{1.152777in}}%
\pgfpathlineto{\pgfqpoint{2.551664in}{1.133141in}}%
\pgfpathlineto{\pgfqpoint{2.688780in}{1.111288in}}%
\pgfpathlineto{\pgfqpoint{2.820452in}{1.087443in}}%
\pgfpathlineto{\pgfqpoint{2.903509in}{1.070583in}}%
\pgfpathlineto{\pgfqpoint{2.981739in}{1.053075in}}%
\pgfpathlineto{\pgfqpoint{3.054355in}{1.035039in}}%
\pgfpathlineto{\pgfqpoint{3.120670in}{1.016612in}}%
\pgfpathlineto{\pgfqpoint{3.180092in}{0.997943in}}%
\pgfpathlineto{\pgfqpoint{3.232130in}{0.979197in}}%
\pgfpathlineto{\pgfqpoint{3.276660in}{0.960535in}}%
\pgfpathlineto{\pgfqpoint{3.314679in}{0.942049in}}%
\pgfpathlineto{\pgfqpoint{3.346835in}{0.923835in}}%
\pgfpathlineto{\pgfqpoint{3.373668in}{0.905980in}}%
\pgfpathlineto{\pgfqpoint{3.395672in}{0.888558in}}%
\pgfpathlineto{\pgfqpoint{3.413292in}{0.871632in}}%
\pgfpathlineto{\pgfqpoint{3.427006in}{0.855253in}}%
\pgfpathlineto{\pgfqpoint{3.437246in}{0.839458in}}%
\pgfpathlineto{\pgfqpoint{3.444362in}{0.824277in}}%
\pgfpathlineto{\pgfqpoint{3.448628in}{0.809736in}}%
\pgfpathlineto{\pgfqpoint{3.450247in}{0.795852in}}%
\pgfpathlineto{\pgfqpoint{3.449316in}{0.782638in}}%
\pgfpathlineto{\pgfqpoint{3.446012in}{0.770103in}}%
\pgfpathlineto{\pgfqpoint{3.440564in}{0.758255in}}%
\pgfpathlineto{\pgfqpoint{3.433141in}{0.747099in}}%
\pgfpathlineto{\pgfqpoint{3.423860in}{0.736641in}}%
\pgfpathlineto{\pgfqpoint{3.412780in}{0.726880in}}%
\pgfpathlineto{\pgfqpoint{3.399904in}{0.717819in}}%
\pgfpathlineto{\pgfqpoint{3.385180in}{0.709454in}}%
\pgfpathlineto{\pgfqpoint{3.368498in}{0.701783in}}%
\pgfpathlineto{\pgfqpoint{3.339440in}{0.691565in}}%
\pgfpathlineto{\pgfqpoint{3.305614in}{0.682905in}}%
\pgfpathlineto{\pgfqpoint{3.267102in}{0.675828in}}%
\pgfpathlineto{\pgfqpoint{3.223623in}{0.670354in}}%
\pgfpathlineto{\pgfqpoint{3.174813in}{0.666508in}}%
\pgfpathlineto{\pgfqpoint{3.120222in}{0.664323in}}%
\pgfpathlineto{\pgfqpoint{3.059316in}{0.663839in}}%
\pgfpathlineto{\pgfqpoint{2.991478in}{0.665103in}}%
\pgfpathlineto{\pgfqpoint{2.889130in}{0.669601in}}%
\pgfpathlineto{\pgfqpoint{2.771550in}{0.677518in}}%
\pgfpathlineto{\pgfqpoint{2.636869in}{0.689119in}}%
\pgfpathlineto{\pgfqpoint{2.484790in}{0.704638in}}%
\pgfpathlineto{\pgfqpoint{2.316700in}{0.724280in}}%
\pgfpathlineto{\pgfqpoint{2.135725in}{0.748211in}}%
\pgfpathlineto{\pgfqpoint{1.996541in}{0.768925in}}%
\pgfpathlineto{\pgfqpoint{1.860014in}{0.791789in}}%
\pgfpathlineto{\pgfqpoint{1.730577in}{0.816524in}}%
\pgfpathlineto{\pgfqpoint{1.650126in}{0.833882in}}%
\pgfpathlineto{\pgfqpoint{1.575494in}{0.851794in}}%
\pgfpathlineto{\pgfqpoint{1.507523in}{0.870120in}}%
\pgfpathlineto{\pgfqpoint{1.446925in}{0.888709in}}%
\pgfpathlineto{\pgfqpoint{1.393893in}{0.907402in}}%
\pgfpathlineto{\pgfqpoint{1.347796in}{0.926071in}}%
\pgfpathlineto{\pgfqpoint{1.307996in}{0.944604in}}%
\pgfpathlineto{\pgfqpoint{1.273947in}{0.962901in}}%
\pgfpathlineto{\pgfqpoint{1.245195in}{0.980871in}}%
\pgfpathlineto{\pgfqpoint{1.221377in}{0.998435in}}%
\pgfpathlineto{\pgfqpoint{1.202225in}{1.015524in}}%
\pgfpathlineto{\pgfqpoint{1.187517in}{1.032080in}}%
\pgfpathlineto{\pgfqpoint{1.176583in}{1.048054in}}%
\pgfpathlineto{\pgfqpoint{1.168864in}{1.063418in}}%
\pgfpathlineto{\pgfqpoint{1.163978in}{1.078147in}}%
\pgfpathlineto{\pgfqpoint{1.161641in}{1.092224in}}%
\pgfpathlineto{\pgfqpoint{1.161664in}{1.105632in}}%
\pgfpathlineto{\pgfqpoint{1.163957in}{1.118362in}}%
\pgfpathlineto{\pgfqpoint{1.168528in}{1.130408in}}%
\pgfpathlineto{\pgfqpoint{1.175427in}{1.141766in}}%
\pgfpathlineto{\pgfqpoint{1.184434in}{1.152432in}}%
\pgfpathlineto{\pgfqpoint{1.195416in}{1.162401in}}%
\pgfpathlineto{\pgfqpoint{1.208293in}{1.171670in}}%
\pgfpathlineto{\pgfqpoint{1.223019in}{1.180237in}}%
\pgfpathlineto{\pgfqpoint{1.248557in}{1.191771in}}%
\pgfpathlineto{\pgfqpoint{1.278331in}{1.201723in}}%
\pgfpathlineto{\pgfqpoint{1.312604in}{1.210095in}}%
\pgfpathlineto{\pgfqpoint{1.351801in}{1.216895in}}%
\pgfpathlineto{\pgfqpoint{1.396094in}{1.222106in}}%
\pgfpathlineto{\pgfqpoint{1.445749in}{1.225695in}}%
\pgfpathlineto{\pgfqpoint{1.501321in}{1.227625in}}%
\pgfpathlineto{\pgfqpoint{1.563393in}{1.227847in}}%
\pgfpathlineto{\pgfqpoint{1.632572in}{1.226305in}}%
\pgfpathlineto{\pgfqpoint{1.736961in}{1.221381in}}%
\pgfpathlineto{\pgfqpoint{1.856680in}{1.213001in}}%
\pgfpathlineto{\pgfqpoint{1.993271in}{1.200940in}}%
\pgfpathlineto{\pgfqpoint{2.147383in}{1.184927in}}%
\pgfpathlineto{\pgfqpoint{2.317592in}{1.164743in}}%
\pgfpathlineto{\pgfqpoint{2.453140in}{1.146812in}}%
\pgfpathlineto{\pgfqpoint{2.592050in}{1.126510in}}%
\pgfpathlineto{\pgfqpoint{2.729696in}{1.103987in}}%
\pgfpathlineto{\pgfqpoint{2.861047in}{1.079518in}}%
\pgfpathlineto{\pgfqpoint{2.942979in}{1.062295in}}%
\pgfpathlineto{\pgfqpoint{3.019130in}{1.044490in}}%
\pgfpathlineto{\pgfqpoint{3.088577in}{1.026244in}}%
\pgfpathlineto{\pgfqpoint{3.150538in}{1.007714in}}%
\pgfpathlineto{\pgfqpoint{3.205254in}{0.989045in}}%
\pgfpathlineto{\pgfqpoint{3.253235in}{0.970362in}}%
\pgfpathlineto{\pgfqpoint{3.294942in}{0.951781in}}%
\pgfpathlineto{\pgfqpoint{3.330790in}{0.933407in}}%
\pgfpathlineto{\pgfqpoint{3.361142in}{0.915336in}}%
\pgfpathlineto{\pgfqpoint{3.386316in}{0.897651in}}%
\pgfpathlineto{\pgfqpoint{3.406579in}{0.880424in}}%
\pgfpathlineto{\pgfqpoint{3.422191in}{0.863718in}}%
\pgfpathlineto{\pgfqpoint{3.433850in}{0.847587in}}%
\pgfpathlineto{\pgfqpoint{3.442200in}{0.832061in}}%
\pgfpathlineto{\pgfqpoint{3.447704in}{0.817164in}}%
\pgfpathlineto{\pgfqpoint{3.450706in}{0.802916in}}%
\pgfpathlineto{\pgfqpoint{3.451427in}{0.789331in}}%
\pgfpathlineto{\pgfqpoint{3.449968in}{0.776421in}}%
\pgfpathlineto{\pgfqpoint{3.446305in}{0.764192in}}%
\pgfpathlineto{\pgfqpoint{3.440296in}{0.752647in}}%
\pgfpathlineto{\pgfqpoint{3.431828in}{0.741789in}}%
\pgfpathlineto{\pgfqpoint{3.421302in}{0.731628in}}%
\pgfpathlineto{\pgfqpoint{3.408842in}{0.722165in}}%
\pgfpathlineto{\pgfqpoint{3.394502in}{0.713405in}}%
\pgfpathlineto{\pgfqpoint{3.378306in}{0.705348in}}%
\pgfpathlineto{\pgfqpoint{3.350515in}{0.694582in}}%
\pgfpathlineto{\pgfqpoint{3.318394in}{0.685403in}}%
\pgfpathlineto{\pgfqpoint{3.281653in}{0.677809in}}%
\pgfpathlineto{\pgfqpoint{3.239877in}{0.671796in}}%
\pgfpathlineto{\pgfqpoint{3.192936in}{0.667388in}}%
\pgfpathlineto{\pgfqpoint{3.140395in}{0.664619in}}%
\pgfpathlineto{\pgfqpoint{3.081642in}{0.663532in}}%
\pgfpathlineto{\pgfqpoint{3.016058in}{0.664180in}}%
\pgfpathlineto{\pgfqpoint{2.916905in}{0.667856in}}%
\pgfpathlineto{\pgfqpoint{2.803000in}{0.674915in}}%
\pgfpathlineto{\pgfqpoint{2.672827in}{0.685574in}}%
\pgfpathlineto{\pgfqpoint{2.525185in}{0.700073in}}%
\pgfpathlineto{\pgfqpoint{2.360560in}{0.718683in}}%
\pgfpathlineto{\pgfqpoint{2.182307in}{0.741553in}}%
\pgfpathlineto{\pgfqpoint{2.043698in}{0.761470in}}%
\pgfpathlineto{\pgfqpoint{1.905446in}{0.783645in}}%
\pgfpathlineto{\pgfqpoint{1.772518in}{0.807859in}}%
\pgfpathlineto{\pgfqpoint{1.689264in}{0.824936in}}%
\pgfpathlineto{\pgfqpoint{1.611620in}{0.842611in}}%
\pgfpathlineto{\pgfqpoint{1.540403in}{0.860751in}}%
\pgfpathlineto{\pgfqpoint{1.476152in}{0.879222in}}%
\pgfpathlineto{\pgfqpoint{1.419125in}{0.897888in}}%
\pgfpathlineto{\pgfqpoint{1.369303in}{0.916609in}}%
\pgfpathlineto{\pgfqpoint{1.326746in}{0.935226in}}%
\pgfpathlineto{\pgfqpoint{1.290756in}{0.953625in}}%
\pgfpathlineto{\pgfqpoint{1.260201in}{0.971734in}}%
\pgfpathlineto{\pgfqpoint{1.234212in}{0.989487in}}%
\pgfpathlineto{\pgfqpoint{1.212179in}{1.006822in}}%
\pgfpathlineto{\pgfqpoint{1.193756in}{1.023683in}}%
\pgfpathlineto{\pgfqpoint{1.178853in}{1.040018in}}%
\pgfpathlineto{\pgfqpoint{1.167643in}{1.055779in}}%
\pgfpathlineto{\pgfqpoint{1.160560in}{1.070926in}}%
\pgfpathlineto{\pgfqpoint{1.157461in}{1.085420in}}%
\pgfpathlineto{\pgfqpoint{1.156974in}{1.099239in}}%
\pgfpathlineto{\pgfqpoint{1.158820in}{1.112373in}}%
\pgfpathlineto{\pgfqpoint{1.162813in}{1.124814in}}%
\pgfpathlineto{\pgfqpoint{1.168823in}{1.136558in}}%
\pgfpathlineto{\pgfqpoint{1.176777in}{1.147600in}}%
\pgfpathlineto{\pgfqpoint{1.186663in}{1.157941in}}%
\pgfpathlineto{\pgfqpoint{1.198526in}{1.167581in}}%
\pgfpathlineto{\pgfqpoint{1.212451in}{1.176523in}}%
\pgfpathlineto{\pgfqpoint{1.228348in}{1.184766in}}%
\pgfpathlineto{\pgfqpoint{1.255795in}{1.195816in}}%
\pgfpathlineto{\pgfqpoint{1.287595in}{1.205279in}}%
\pgfpathlineto{\pgfqpoint{1.323885in}{1.213146in}}%
\pgfpathlineto{\pgfqpoint{1.364916in}{1.219404in}}%
\pgfpathlineto{\pgfqpoint{1.411054in}{1.224040in}}%
\pgfpathlineto{\pgfqpoint{1.462776in}{1.227038in}}%
\pgfpathlineto{\pgfqpoint{1.520577in}{1.228381in}}%
\pgfpathlineto{\pgfqpoint{1.584601in}{1.228032in}}%
\pgfpathlineto{\pgfqpoint{1.681649in}{1.224773in}}%
\pgfpathlineto{\pgfqpoint{1.794169in}{1.218116in}}%
\pgfpathlineto{\pgfqpoint{1.923459in}{1.207841in}}%
\pgfpathlineto{\pgfqpoint{2.069731in}{1.193740in}}%
\pgfpathlineto{\pgfqpoint{2.232111in}{1.175609in}}%
\pgfpathlineto{\pgfqpoint{2.408646in}{1.153256in}}%
\pgfpathlineto{\pgfqpoint{2.547592in}{1.133627in}}%
\pgfpathlineto{\pgfqpoint{2.686530in}{1.111713in}}%
\pgfpathlineto{\pgfqpoint{2.820333in}{1.087796in}}%
\pgfpathlineto{\pgfqpoint{2.904562in}{1.070902in}}%
\pgfpathlineto{\pgfqpoint{2.983588in}{1.053377in}}%
\pgfpathlineto{\pgfqpoint{3.056565in}{1.035339in}}%
\pgfpathlineto{\pgfqpoint{3.122824in}{1.016921in}}%
\pgfpathlineto{\pgfqpoint{3.181878in}{0.998261in}}%
\pgfpathlineto{\pgfqpoint{3.233416in}{0.979511in}}%
\pgfpathlineto{\pgfqpoint{3.277575in}{0.960828in}}%
\pgfpathlineto{\pgfqpoint{3.315250in}{0.942327in}}%
\pgfpathlineto{\pgfqpoint{3.347298in}{0.924095in}}%
\pgfpathlineto{\pgfqpoint{3.374393in}{0.906213in}}%
\pgfpathlineto{\pgfqpoint{3.397026in}{0.888752in}}%
\pgfpathlineto{\pgfqpoint{3.415511in}{0.871775in}}%
\pgfpathlineto{\pgfqpoint{3.429975in}{0.855335in}}%
\pgfpathlineto{\pgfqpoint{3.440368in}{0.839480in}}%
\pgfpathlineto{\pgfqpoint{3.447158in}{0.824247in}}%
\pgfpathlineto{\pgfqpoint{3.451078in}{0.809660in}}%
\pgfpathlineto{\pgfqpoint{3.452436in}{0.795737in}}%
\pgfpathlineto{\pgfqpoint{3.451469in}{0.782488in}}%
\pgfpathlineto{\pgfqpoint{3.448336in}{0.769924in}}%
\pgfpathlineto{\pgfqpoint{3.443126in}{0.758051in}}%
\pgfpathlineto{\pgfqpoint{3.435851in}{0.746871in}}%
\pgfpathlineto{\pgfqpoint{3.426450in}{0.736386in}}%
\pgfpathlineto{\pgfqpoint{3.414944in}{0.726595in}}%
\pgfpathlineto{\pgfqpoint{3.401485in}{0.717501in}}%
\pgfpathlineto{\pgfqpoint{3.386124in}{0.709108in}}%
\pgfpathlineto{\pgfqpoint{3.359558in}{0.697835in}}%
\pgfpathlineto{\pgfqpoint{3.328726in}{0.688148in}}%
\pgfpathlineto{\pgfqpoint{3.293457in}{0.680053in}}%
\pgfpathlineto{\pgfqpoint{3.253450in}{0.673555in}}%
\pgfpathlineto{\pgfqpoint{3.208264in}{0.668656in}}%
\pgfpathlineto{\pgfqpoint{3.157539in}{0.665368in}}%
\pgfpathlineto{\pgfqpoint{3.101021in}{0.663739in}}%
\pgfpathlineto{\pgfqpoint{3.037878in}{0.663822in}}%
\pgfpathlineto{\pgfqpoint{2.967343in}{0.665684in}}%
\pgfpathlineto{\pgfqpoint{2.860654in}{0.671061in}}%
\pgfpathlineto{\pgfqpoint{2.738324in}{0.679931in}}%
\pgfpathlineto{\pgfqpoint{2.599336in}{0.692523in}}%
\pgfpathlineto{\pgfqpoint{2.442826in}{0.709098in}}%
\pgfpathlineto{\pgfqpoint{2.267800in}{0.729967in}}%
\pgfpathlineto{\pgfqpoint{2.129892in}{0.748439in}}%
\pgfpathlineto{\pgfqpoint{1.991521in}{0.769207in}}%
\pgfpathlineto{\pgfqpoint{1.856820in}{0.792081in}}%
\pgfpathlineto{\pgfqpoint{1.729402in}{0.816804in}}%
\pgfpathlineto{\pgfqpoint{1.650048in}{0.834152in}}%
\pgfpathlineto{\pgfqpoint{1.576097in}{0.852060in}}%
\pgfpathlineto{\pgfqpoint{1.508231in}{0.870397in}}%
\pgfpathlineto{\pgfqpoint{1.447027in}{0.889020in}}%
\pgfpathlineto{\pgfqpoint{1.392960in}{0.907771in}}%
\pgfpathlineto{\pgfqpoint{1.346395in}{0.926478in}}%
\pgfpathlineto{\pgfqpoint{1.306802in}{0.945016in}}%
\pgfpathlineto{\pgfqpoint{1.273239in}{0.963306in}}%
\pgfpathlineto{\pgfqpoint{1.245139in}{0.981257in}}%
\pgfpathlineto{\pgfqpoint{1.221985in}{0.998792in}}%
\pgfpathlineto{\pgfqpoint{1.203310in}{1.015846in}}%
\pgfpathlineto{\pgfqpoint{1.188654in}{1.032365in}}%
\pgfpathlineto{\pgfqpoint{1.177570in}{1.048308in}}%
\pgfpathlineto{\pgfqpoint{1.169697in}{1.063644in}}%
\pgfpathlineto{\pgfqpoint{1.164743in}{1.078347in}}%
\pgfpathlineto{\pgfqpoint{1.162485in}{1.092396in}}%
\pgfpathlineto{\pgfqpoint{1.162823in}{1.105778in}}%
\pgfpathlineto{\pgfqpoint{1.165572in}{1.118483in}}%
\pgfpathlineto{\pgfqpoint{1.170489in}{1.130502in}}%
\pgfpathlineto{\pgfqpoint{1.177394in}{1.141831in}}%
\pgfpathlineto{\pgfqpoint{1.186164in}{1.152463in}}%
\pgfpathlineto{\pgfqpoint{1.196733in}{1.162397in}}%
\pgfpathlineto{\pgfqpoint{1.209097in}{1.171633in}}%
\pgfpathlineto{\pgfqpoint{1.223310in}{1.180173in}}%
\pgfpathlineto{\pgfqpoint{1.239483in}{1.188020in}}%
\pgfpathlineto{\pgfqpoint{1.257790in}{1.195180in}}%
\pgfpathlineto{\pgfqpoint{1.289336in}{1.204629in}}%
\pgfpathlineto{\pgfqpoint{1.325477in}{1.212502in}}%
\pgfpathlineto{\pgfqpoint{1.366430in}{1.218782in}}%
\pgfpathlineto{\pgfqpoint{1.412517in}{1.223446in}}%
\pgfpathlineto{\pgfqpoint{1.464145in}{1.226465in}}%
\pgfpathlineto{\pgfqpoint{1.521807in}{1.227805in}}%
\pgfpathlineto{\pgfqpoint{1.586085in}{1.227424in}}%
\pgfpathlineto{\pgfqpoint{1.683175in}{1.224154in}}%
\pgfpathlineto{\pgfqpoint{1.794704in}{1.217560in}}%
\pgfpathlineto{\pgfqpoint{1.922902in}{1.207371in}}%
\pgfpathlineto{\pgfqpoint{2.068613in}{1.193342in}}%
\pgfpathlineto{\pgfqpoint{2.230978in}{1.175267in}}%
\pgfpathlineto{\pgfqpoint{2.407447in}{1.152971in}}%
\pgfpathlineto{\pgfqpoint{2.546150in}{1.133400in}}%
\pgfpathlineto{\pgfqpoint{2.684862in}{1.111551in}}%
\pgfpathlineto{\pgfqpoint{2.818514in}{1.087697in}}%
\pgfpathlineto{\pgfqpoint{2.902698in}{1.070841in}}%
\pgfpathlineto{\pgfqpoint{2.981725in}{1.053349in}}%
\pgfpathlineto{\pgfqpoint{3.054739in}{1.035342in}}%
\pgfpathlineto{\pgfqpoint{3.121058in}{1.016951in}}%
\pgfpathlineto{\pgfqpoint{3.180172in}{0.998318in}}%
\pgfpathlineto{\pgfqpoint{3.231750in}{0.979595in}}%
\pgfpathlineto{\pgfqpoint{3.276022in}{0.960939in}}%
\pgfpathlineto{\pgfqpoint{3.313888in}{0.942456in}}%
\pgfpathlineto{\pgfqpoint{3.346149in}{0.924239in}}%
\pgfpathlineto{\pgfqpoint{3.373435in}{0.906367in}}%
\pgfpathlineto{\pgfqpoint{3.396207in}{0.888915in}}%
\pgfpathlineto{\pgfqpoint{3.414755in}{0.871945in}}%
\pgfpathlineto{\pgfqpoint{3.429199in}{0.855512in}}%
\pgfpathlineto{\pgfqpoint{3.439521in}{0.839663in}}%
\pgfpathlineto{\pgfqpoint{3.446391in}{0.824436in}}%
\pgfpathlineto{\pgfqpoint{3.450401in}{0.809853in}}%
\pgfpathlineto{\pgfqpoint{3.451856in}{0.795931in}}%
\pgfpathlineto{\pgfqpoint{3.450987in}{0.782682in}}%
\pgfpathlineto{\pgfqpoint{3.447947in}{0.770117in}}%
\pgfpathlineto{\pgfqpoint{3.442811in}{0.758241in}}%
\pgfpathlineto{\pgfqpoint{3.435580in}{0.747058in}}%
\pgfpathlineto{\pgfqpoint{3.426182in}{0.736567in}}%
\pgfpathlineto{\pgfqpoint{3.414711in}{0.726770in}}%
\pgfpathlineto{\pgfqpoint{3.401294in}{0.717671in}}%
\pgfpathlineto{\pgfqpoint{3.385982in}{0.709273in}}%
\pgfpathlineto{\pgfqpoint{3.359501in}{0.697991in}}%
\pgfpathlineto{\pgfqpoint{3.328760in}{0.688296in}}%
\pgfpathlineto{\pgfqpoint{3.293580in}{0.680190in}}%
\pgfpathlineto{\pgfqpoint{3.253643in}{0.673679in}}%
\pgfpathlineto{\pgfqpoint{3.208493in}{0.668762in}}%
\pgfpathlineto{\pgfqpoint{3.157866in}{0.665457in}}%
\pgfpathlineto{\pgfqpoint{3.101394in}{0.663812in}}%
\pgfpathlineto{\pgfqpoint{3.038282in}{0.663881in}}%
\pgfpathlineto{\pgfqpoint{2.967809in}{0.665727in}}%
\pgfpathlineto{\pgfqpoint{2.861280in}{0.671078in}}%
\pgfpathlineto{\pgfqpoint{2.739173in}{0.679919in}}%
\pgfpathlineto{\pgfqpoint{2.600377in}{0.692480in}}%
\pgfpathlineto{\pgfqpoint{2.443337in}{0.709047in}}%
\pgfpathlineto{\pgfqpoint{2.266915in}{0.729972in}}%
\pgfpathlineto{\pgfqpoint{2.128459in}{0.748478in}}%
\pgfpathlineto{\pgfqpoint{1.990085in}{0.769250in}}%
\pgfpathlineto{\pgfqpoint{1.855912in}{0.792091in}}%
\pgfpathlineto{\pgfqpoint{1.729426in}{0.816747in}}%
\pgfpathlineto{\pgfqpoint{1.650805in}{0.834039in}}%
\pgfpathlineto{\pgfqpoint{1.577568in}{0.851891in}}%
\pgfpathlineto{\pgfqpoint{1.510275in}{0.870182in}}%
\pgfpathlineto{\pgfqpoint{1.449364in}{0.888781in}}%
\pgfpathlineto{\pgfqpoint{1.395146in}{0.907543in}}%
\pgfpathlineto{\pgfqpoint{1.347805in}{0.926313in}}%
\pgfpathlineto{\pgfqpoint{1.307401in}{0.944923in}}%
\pgfpathlineto{\pgfqpoint{1.273695in}{0.963219in}}%
\pgfpathlineto{\pgfqpoint{1.245581in}{0.981166in}}%
\pgfpathlineto{\pgfqpoint{1.222418in}{0.998697in}}%
\pgfpathlineto{\pgfqpoint{1.203704in}{1.015747in}}%
\pgfpathlineto{\pgfqpoint{1.188990in}{1.032265in}}%
\pgfpathlineto{\pgfqpoint{1.177860in}{1.048208in}}%
\pgfpathlineto{\pgfqpoint{1.169956in}{1.063543in}}%
\pgfpathlineto{\pgfqpoint{1.164985in}{1.078246in}}%
\pgfpathlineto{\pgfqpoint{1.162734in}{1.092296in}}%
\pgfpathlineto{\pgfqpoint{1.163120in}{1.105680in}}%
\pgfpathlineto{\pgfqpoint{1.165872in}{1.118386in}}%
\pgfpathlineto{\pgfqpoint{1.170746in}{1.130407in}}%
\pgfpathlineto{\pgfqpoint{1.177561in}{1.141737in}}%
\pgfpathlineto{\pgfqpoint{1.186204in}{1.152371in}}%
\pgfpathlineto{\pgfqpoint{1.196626in}{1.162308in}}%
\pgfpathlineto{\pgfqpoint{1.208845in}{1.171548in}}%
\pgfpathlineto{\pgfqpoint{1.222942in}{1.180094in}}%
\pgfpathlineto{\pgfqpoint{1.239066in}{1.187950in}}%
\pgfpathlineto{\pgfqpoint{1.257431in}{1.195124in}}%
\pgfpathlineto{\pgfqpoint{1.288969in}{1.204589in}}%
\pgfpathlineto{\pgfqpoint{1.325085in}{1.212476in}}%
\pgfpathlineto{\pgfqpoint{1.366007in}{1.218767in}}%
\pgfpathlineto{\pgfqpoint{1.412055in}{1.223442in}}%
\pgfpathlineto{\pgfqpoint{1.463636in}{1.226472in}}%
\pgfpathlineto{\pgfqpoint{1.521248in}{1.227824in}}%
\pgfpathlineto{\pgfqpoint{1.585477in}{1.227458in}}%
\pgfpathlineto{\pgfqpoint{1.682385in}{1.224220in}}%
\pgfpathlineto{\pgfqpoint{1.793783in}{1.217655in}}%
\pgfpathlineto{\pgfqpoint{1.922015in}{1.207482in}}%
\pgfpathlineto{\pgfqpoint{2.067812in}{1.193461in}}%
\pgfpathlineto{\pgfqpoint{2.230166in}{1.175395in}}%
\pgfpathlineto{\pgfqpoint{2.406324in}{1.153132in}}%
\pgfpathlineto{\pgfqpoint{2.544690in}{1.133605in}}%
\pgfpathlineto{\pgfqpoint{2.683509in}{1.111764in}}%
\pgfpathlineto{\pgfqpoint{2.817440in}{1.087906in}}%
\pgfpathlineto{\pgfqpoint{2.901811in}{1.071049in}}%
\pgfpathlineto{\pgfqpoint{2.980994in}{1.053560in}}%
\pgfpathlineto{\pgfqpoint{3.054131in}{1.035560in}}%
\pgfpathlineto{\pgfqpoint{3.120562in}{1.017177in}}%
\pgfpathlineto{\pgfqpoint{3.179819in}{0.998549in}}%
\pgfpathlineto{\pgfqpoint{3.231630in}{0.979821in}}%
\pgfpathlineto{\pgfqpoint{3.276029in}{0.961151in}}%
\pgfpathlineto{\pgfqpoint{3.313828in}{0.942663in}}%
\pgfpathlineto{\pgfqpoint{3.345953in}{0.924443in}}%
\pgfpathlineto{\pgfqpoint{3.373134in}{0.906569in}}%
\pgfpathlineto{\pgfqpoint{3.395905in}{0.889112in}}%
\pgfpathlineto{\pgfqpoint{3.414607in}{0.872133in}}%
\pgfpathlineto{\pgfqpoint{3.429387in}{0.855687in}}%
\pgfpathlineto{\pgfqpoint{3.440194in}{0.839821in}}%
\pgfpathlineto{\pgfqpoint{3.447121in}{0.824572in}}%
\pgfpathlineto{\pgfqpoint{3.451094in}{0.809970in}}%
\pgfpathlineto{\pgfqpoint{3.452475in}{0.796031in}}%
\pgfpathlineto{\pgfqpoint{3.451514in}{0.782767in}}%
\pgfpathlineto{\pgfqpoint{3.448387in}{0.770188in}}%
\pgfpathlineto{\pgfqpoint{3.443200in}{0.758300in}}%
\pgfpathlineto{\pgfqpoint{3.435989in}{0.747106in}}%
\pgfpathlineto{\pgfqpoint{3.426718in}{0.736607in}}%
\pgfpathlineto{\pgfqpoint{3.415313in}{0.726802in}}%
\pgfpathlineto{\pgfqpoint{3.401910in}{0.717693in}}%
\pgfpathlineto{\pgfqpoint{3.386587in}{0.709283in}}%
\pgfpathlineto{\pgfqpoint{3.360052in}{0.697984in}}%
\pgfpathlineto{\pgfqpoint{3.329232in}{0.688270in}}%
\pgfpathlineto{\pgfqpoint{3.293976in}{0.680148in}}%
\pgfpathlineto{\pgfqpoint{3.254010in}{0.673625in}}%
\pgfpathlineto{\pgfqpoint{3.208929in}{0.668708in}}%
\pgfpathlineto{\pgfqpoint{3.158253in}{0.665405in}}%
\pgfpathlineto{\pgfqpoint{3.101847in}{0.663750in}}%
\pgfpathlineto{\pgfqpoint{3.038908in}{0.663805in}}%
\pgfpathlineto{\pgfqpoint{2.968564in}{0.665638in}}%
\pgfpathlineto{\pgfqpoint{2.862017in}{0.670979in}}%
\pgfpathlineto{\pgfqpoint{2.739709in}{0.679815in}}%
\pgfpathlineto{\pgfqpoint{2.600776in}{0.692369in}}%
\pgfpathlineto{\pgfqpoint{2.444813in}{0.708889in}}%
\pgfpathlineto{\pgfqpoint{2.271495in}{0.729661in}}%
\pgfpathlineto{\pgfqpoint{2.133823in}{0.748086in}}%
\pgfpathlineto{\pgfqpoint{1.995032in}{0.768824in}}%
\pgfpathlineto{\pgfqpoint{1.859513in}{0.791683in}}%
\pgfpathlineto{\pgfqpoint{1.731133in}{0.816397in}}%
\pgfpathlineto{\pgfqpoint{1.651184in}{0.833740in}}%
\pgfpathlineto{\pgfqpoint{1.576767in}{0.851640in}}%
\pgfpathlineto{\pgfqpoint{1.508639in}{0.869965in}}%
\pgfpathlineto{\pgfqpoint{1.447454in}{0.888567in}}%
\pgfpathlineto{\pgfqpoint{1.393766in}{0.907285in}}%
\pgfpathlineto{\pgfqpoint{1.347497in}{0.925969in}}%
\pgfpathlineto{\pgfqpoint{1.307793in}{0.944511in}}%
\pgfpathlineto{\pgfqpoint{1.274030in}{0.962810in}}%
\pgfpathlineto{\pgfqpoint{1.245662in}{0.980775in}}%
\pgfpathlineto{\pgfqpoint{1.222221in}{0.998329in}}%
\pgfpathlineto{\pgfqpoint{1.203313in}{1.015405in}}%
\pgfpathlineto{\pgfqpoint{1.188576in}{1.031946in}}%
\pgfpathlineto{\pgfqpoint{1.177470in}{1.047910in}}%
\pgfpathlineto{\pgfqpoint{1.169574in}{1.063267in}}%
\pgfpathlineto{\pgfqpoint{1.164565in}{1.077991in}}%
\pgfpathlineto{\pgfqpoint{1.162198in}{1.092063in}}%
\pgfpathlineto{\pgfqpoint{1.162308in}{1.105467in}}%
\pgfpathlineto{\pgfqpoint{1.164810in}{1.118193in}}%
\pgfpathlineto{\pgfqpoint{1.169630in}{1.130236in}}%
\pgfpathlineto{\pgfqpoint{1.176596in}{1.141589in}}%
\pgfpathlineto{\pgfqpoint{1.185556in}{1.152247in}}%
\pgfpathlineto{\pgfqpoint{1.196403in}{1.162208in}}%
\pgfpathlineto{\pgfqpoint{1.209073in}{1.171468in}}%
\pgfpathlineto{\pgfqpoint{1.223541in}{1.180028in}}%
\pgfpathlineto{\pgfqpoint{1.248677in}{1.191554in}}%
\pgfpathlineto{\pgfqpoint{1.278171in}{1.201510in}}%
\pgfpathlineto{\pgfqpoint{1.312487in}{1.209909in}}%
\pgfpathlineto{\pgfqpoint{1.351711in}{1.216739in}}%
\pgfpathlineto{\pgfqpoint{1.395939in}{1.221970in}}%
\pgfpathlineto{\pgfqpoint{1.445583in}{1.225574in}}%
\pgfpathlineto{\pgfqpoint{1.501119in}{1.227515in}}%
\pgfpathlineto{\pgfqpoint{1.563085in}{1.227747in}}%
\pgfpathlineto{\pgfqpoint{1.632087in}{1.226215in}}%
\pgfpathlineto{\pgfqpoint{1.736194in}{1.221316in}}%
\pgfpathlineto{\pgfqpoint{1.855701in}{1.212980in}}%
\pgfpathlineto{\pgfqpoint{1.992149in}{1.200955in}}%
\pgfpathlineto{\pgfqpoint{2.146062in}{1.184983in}}%
\pgfpathlineto{\pgfqpoint{2.316027in}{1.164852in}}%
\pgfpathlineto{\pgfqpoint{2.451560in}{1.146952in}}%
\pgfpathlineto{\pgfqpoint{2.590227in}{1.126689in}}%
\pgfpathlineto{\pgfqpoint{2.727805in}{1.104208in}}%
\pgfpathlineto{\pgfqpoint{2.859435in}{1.079763in}}%
\pgfpathlineto{\pgfqpoint{2.941260in}{1.062526in}}%
\pgfpathlineto{\pgfqpoint{3.017154in}{1.044700in}}%
\pgfpathlineto{\pgfqpoint{3.086561in}{1.026449in}}%
\pgfpathlineto{\pgfqpoint{3.149115in}{1.007924in}}%
\pgfpathlineto{\pgfqpoint{3.204637in}{0.989265in}}%
\pgfpathlineto{\pgfqpoint{3.253140in}{0.970601in}}%
\pgfpathlineto{\pgfqpoint{3.294826in}{0.952046in}}%
\pgfpathlineto{\pgfqpoint{3.330087in}{0.933704in}}%
\pgfpathlineto{\pgfqpoint{3.359502in}{0.915668in}}%
\pgfpathlineto{\pgfqpoint{3.383843in}{0.898016in}}%
\pgfpathlineto{\pgfqpoint{3.403553in}{0.880829in}}%
\pgfpathlineto{\pgfqpoint{3.418874in}{0.864172in}}%
\pgfpathlineto{\pgfqpoint{3.430630in}{0.848077in}}%
\pgfpathlineto{\pgfqpoint{3.439458in}{0.832573in}}%
\pgfpathlineto{\pgfqpoint{3.445802in}{0.817683in}}%
\pgfpathlineto{\pgfqpoint{3.449912in}{0.803427in}}%
\pgfpathlineto{\pgfqpoint{3.451846in}{0.789820in}}%
\pgfpathlineto{\pgfqpoint{3.451469in}{0.776875in}}%
\pgfpathlineto{\pgfqpoint{3.448451in}{0.764598in}}%
\pgfpathlineto{\pgfqpoint{3.442330in}{0.752993in}}%
\pgfpathlineto{\pgfqpoint{3.433773in}{0.742080in}}%
\pgfpathlineto{\pgfqpoint{3.423238in}{0.731869in}}%
\pgfpathlineto{\pgfqpoint{3.410802in}{0.722361in}}%
\pgfpathlineto{\pgfqpoint{3.396509in}{0.713558in}}%
\pgfpathlineto{\pgfqpoint{3.380372in}{0.705460in}}%
\pgfpathlineto{\pgfqpoint{3.352664in}{0.694636in}}%
\pgfpathlineto{\pgfqpoint{3.320574in}{0.685397in}}%
\pgfpathlineto{\pgfqpoint{3.283761in}{0.677738in}}%
\pgfpathlineto{\pgfqpoint{3.242017in}{0.671665in}}%
\pgfpathlineto{\pgfqpoint{3.195128in}{0.667206in}}%
\pgfpathlineto{\pgfqpoint{3.142598in}{0.664390in}}%
\pgfpathlineto{\pgfqpoint{3.083888in}{0.663261in}}%
\pgfpathlineto{\pgfqpoint{3.018418in}{0.663869in}}%
\pgfpathlineto{\pgfqpoint{2.919528in}{0.667492in}}%
\pgfpathlineto{\pgfqpoint{2.805918in}{0.674494in}}%
\pgfpathlineto{\pgfqpoint{2.675928in}{0.685084in}}%
\pgfpathlineto{\pgfqpoint{2.528456in}{0.699529in}}%
\pgfpathlineto{\pgfqpoint{2.364019in}{0.718071in}}%
\pgfpathlineto{\pgfqpoint{2.185816in}{0.740862in}}%
\pgfpathlineto{\pgfqpoint{2.047069in}{0.760742in}}%
\pgfpathlineto{\pgfqpoint{1.908775in}{0.782876in}}%
\pgfpathlineto{\pgfqpoint{1.775599in}{0.807030in}}%
\pgfpathlineto{\pgfqpoint{1.692021in}{0.824088in}}%
\pgfpathlineto{\pgfqpoint{1.614412in}{0.841776in}}%
\pgfpathlineto{\pgfqpoint{1.543550in}{0.859950in}}%
\pgfpathlineto{\pgfqpoint{1.479535in}{0.878446in}}%
\pgfpathlineto{\pgfqpoint{1.422372in}{0.897116in}}%
\pgfpathlineto{\pgfqpoint{1.371978in}{0.915825in}}%
\pgfpathlineto{\pgfqpoint{1.328173in}{0.934451in}}%
\pgfpathlineto{\pgfqpoint{1.290688in}{0.952885in}}%
\pgfpathlineto{\pgfqpoint{1.259158in}{0.971032in}}%
\pgfpathlineto{\pgfqpoint{1.233129in}{0.988811in}}%
\pgfpathlineto{\pgfqpoint{1.212052in}{1.006154in}}%
\pgfpathlineto{\pgfqpoint{1.195339in}{1.023003in}}%
\pgfpathlineto{\pgfqpoint{1.182836in}{1.039288in}}%
\pgfpathlineto{\pgfqpoint{1.173880in}{1.054975in}}%
\pgfpathlineto{\pgfqpoint{1.167780in}{1.070041in}}%
\pgfpathlineto{\pgfqpoint{1.164020in}{1.084469in}}%
\pgfpathlineto{\pgfqpoint{1.162257in}{1.098246in}}%
\pgfpathlineto{\pgfqpoint{1.162325in}{1.111359in}}%
\pgfpathlineto{\pgfqpoint{1.164227in}{1.123803in}}%
\pgfpathlineto{\pgfqpoint{1.168145in}{1.135572in}}%
\pgfpathlineto{\pgfqpoint{1.174432in}{1.146667in}}%
\pgfpathlineto{\pgfqpoint{1.183617in}{1.157091in}}%
\pgfpathlineto{\pgfqpoint{1.195583in}{1.166832in}}%
\pgfpathlineto{\pgfqpoint{1.209478in}{1.175869in}}%
\pgfpathlineto{\pgfqpoint{1.225260in}{1.184200in}}%
\pgfpathlineto{\pgfqpoint{1.252456in}{1.195370in}}%
\pgfpathlineto{\pgfqpoint{1.283947in}{1.204947in}}%
\pgfpathlineto{\pgfqpoint{1.319924in}{1.212925in}}%
\pgfpathlineto{\pgfqpoint{1.360700in}{1.219301in}}%
\pgfpathlineto{\pgfqpoint{1.406691in}{1.224071in}}%
\pgfpathlineto{\pgfqpoint{1.458060in}{1.227212in}}%
\pgfpathlineto{\pgfqpoint{1.515413in}{1.228681in}}%
\pgfpathlineto{\pgfqpoint{1.579506in}{1.228423in}}%
\pgfpathlineto{\pgfqpoint{1.676641in}{1.225285in}}%
\pgfpathlineto{\pgfqpoint{1.788473in}{1.218785in}}%
\pgfpathlineto{\pgfqpoint{1.916296in}{1.208714in}}%
\pgfpathlineto{\pgfqpoint{2.061216in}{1.194829in}}%
\pgfpathlineto{\pgfqpoint{2.226142in}{1.176804in}}%
\pgfpathlineto{\pgfqpoint{2.361051in}{1.160433in}}%
\pgfpathlineto{\pgfqpoint{2.499872in}{1.141673in}}%
\pgfpathlineto{\pgfqpoint{2.638055in}{1.120647in}}%
\pgfpathlineto{\pgfqpoint{2.771644in}{1.097546in}}%
\pgfpathlineto{\pgfqpoint{2.856465in}{1.081118in}}%
\pgfpathlineto{\pgfqpoint{2.936886in}{1.063973in}}%
\pgfpathlineto{\pgfqpoint{3.012173in}{1.046219in}}%
\pgfpathlineto{\pgfqpoint{3.081709in}{1.027976in}}%
\pgfpathlineto{\pgfqpoint{3.144993in}{1.009376in}}%
\pgfpathlineto{\pgfqpoint{3.201642in}{0.990566in}}%
\pgfpathlineto{\pgfqpoint{3.251390in}{0.971705in}}%
\pgfpathlineto{\pgfqpoint{3.294087in}{0.952966in}}%
\pgfpathlineto{\pgfqpoint{3.329767in}{0.934526in}}%
\pgfpathlineto{\pgfqpoint{3.359514in}{0.916431in}}%
\pgfpathlineto{\pgfqpoint{3.384140in}{0.898730in}}%
\pgfpathlineto{\pgfqpoint{3.404166in}{0.881493in}}%
\pgfpathlineto{\pgfqpoint{3.420058in}{0.864776in}}%
\pgfpathlineto{\pgfqpoint{3.432239in}{0.848623in}}%
\pgfpathlineto{\pgfqpoint{3.441091in}{0.833070in}}%
\pgfpathlineto{\pgfqpoint{3.446929in}{0.818144in}}%
\pgfpathlineto{\pgfqpoint{3.450009in}{0.803867in}}%
\pgfpathlineto{\pgfqpoint{3.450447in}{0.790254in}}%
\pgfpathlineto{\pgfqpoint{3.448403in}{0.777316in}}%
\pgfpathlineto{\pgfqpoint{3.444147in}{0.765061in}}%
\pgfpathlineto{\pgfqpoint{3.437882in}{0.753498in}}%
\pgfpathlineto{\pgfqpoint{3.429749in}{0.742630in}}%
\pgfpathlineto{\pgfqpoint{3.419828in}{0.732459in}}%
\pgfpathlineto{\pgfqpoint{3.408133in}{0.722988in}}%
\pgfpathlineto{\pgfqpoint{3.394615in}{0.714213in}}%
\pgfpathlineto{\pgfqpoint{3.379161in}{0.706132in}}%
\pgfpathlineto{\pgfqpoint{3.361596in}{0.698737in}}%
\pgfpathlineto{\pgfqpoint{3.331063in}{0.688931in}}%
\pgfpathlineto{\pgfqpoint{3.295936in}{0.680699in}}%
\pgfpathlineto{\pgfqpoint{3.256059in}{0.674057in}}%
\pgfpathlineto{\pgfqpoint{3.211130in}{0.669026in}}%
\pgfpathlineto{\pgfqpoint{3.160761in}{0.665634in}}%
\pgfpathlineto{\pgfqpoint{3.104476in}{0.663914in}}%
\pgfpathlineto{\pgfqpoint{3.041712in}{0.663905in}}%
\pgfpathlineto{\pgfqpoint{2.946863in}{0.666633in}}%
\pgfpathlineto{\pgfqpoint{2.837928in}{0.672648in}}%
\pgfpathlineto{\pgfqpoint{2.712557in}{0.682219in}}%
\pgfpathlineto{\pgfqpoint{2.569663in}{0.695595in}}%
\pgfpathlineto{\pgfqpoint{2.409879in}{0.712988in}}%
\pgfpathlineto{\pgfqpoint{2.235553in}{0.734568in}}%
\pgfpathlineto{\pgfqpoint{2.097716in}{0.753599in}}%
\pgfpathlineto{\pgfqpoint{1.958428in}{0.774985in}}%
\pgfpathlineto{\pgfqpoint{1.823080in}{0.798442in}}%
\pgfpathlineto{\pgfqpoint{1.737280in}{0.815070in}}%
\pgfpathlineto{\pgfqpoint{1.656325in}{0.832364in}}%
\pgfpathlineto{\pgfqpoint{1.581121in}{0.850210in}}%
\pgfpathlineto{\pgfqpoint{1.512386in}{0.868481in}}%
\pgfpathlineto{\pgfqpoint{1.450653in}{0.887046in}}%
\pgfpathlineto{\pgfqpoint{1.396269in}{0.905764in}}%
\pgfpathlineto{\pgfqpoint{1.349386in}{0.924485in}}%
\pgfpathlineto{\pgfqpoint{1.309572in}{0.943056in}}%
\pgfpathlineto{\pgfqpoint{1.275825in}{0.961380in}}%
\pgfpathlineto{\pgfqpoint{1.247291in}{0.979378in}}%
\pgfpathlineto{\pgfqpoint{1.223314in}{0.996980in}}%
\pgfpathlineto{\pgfqpoint{1.203442in}{1.014120in}}%
\pgfpathlineto{\pgfqpoint{1.187422in}{1.030745in}}%
\pgfpathlineto{\pgfqpoint{1.175206in}{1.046805in}}%
\pgfpathlineto{\pgfqpoint{1.166934in}{1.062261in}}%
\pgfpathlineto{\pgfqpoint{1.162067in}{1.077079in}}%
\pgfpathlineto{\pgfqpoint{1.159906in}{1.091237in}}%
\pgfpathlineto{\pgfqpoint{1.160177in}{1.104723in}}%
\pgfpathlineto{\pgfqpoint{1.162675in}{1.117527in}}%
\pgfpathlineto{\pgfqpoint{1.167260in}{1.129641in}}%
\pgfpathlineto{\pgfqpoint{1.173862in}{1.141060in}}%
\pgfpathlineto{\pgfqpoint{1.182476in}{1.151783in}}%
\pgfpathlineto{\pgfqpoint{1.193164in}{1.161811in}}%
\pgfpathlineto{\pgfqpoint{1.205935in}{1.171144in}}%
\pgfpathlineto{\pgfqpoint{1.220648in}{1.179780in}}%
\pgfpathlineto{\pgfqpoint{1.246290in}{1.191420in}}%
\pgfpathlineto{\pgfqpoint{1.276216in}{1.201477in}}%
\pgfpathlineto{\pgfqpoint{1.310537in}{1.209942in}}%
\pgfpathlineto{\pgfqpoint{1.349492in}{1.216809in}}%
\pgfpathlineto{\pgfqpoint{1.393444in}{1.222069in}}%
\pgfpathlineto{\pgfqpoint{1.442882in}{1.225713in}}%
\pgfpathlineto{\pgfqpoint{1.498072in}{1.227724in}}%
\pgfpathlineto{\pgfqpoint{1.559488in}{1.228042in}}%
\pgfpathlineto{\pgfqpoint{1.628117in}{1.226597in}}%
\pgfpathlineto{\pgfqpoint{1.732227in}{1.221798in}}%
\pgfpathlineto{\pgfqpoint{1.852024in}{1.213533in}}%
\pgfpathlineto{\pgfqpoint{1.988381in}{1.201585in}}%
\pgfpathlineto{\pgfqpoint{2.141605in}{1.185718in}}%
\pgfpathlineto{\pgfqpoint{2.311661in}{1.165663in}}%
\pgfpathlineto{\pgfqpoint{2.448428in}{1.147747in}}%
\pgfpathlineto{\pgfqpoint{2.587656in}{1.127474in}}%
\pgfpathlineto{\pgfqpoint{2.724576in}{1.105035in}}%
\pgfpathlineto{\pgfqpoint{2.855037in}{1.080686in}}%
\pgfpathlineto{\pgfqpoint{2.936655in}{1.063550in}}%
\pgfpathlineto{\pgfqpoint{3.012928in}{1.045821in}}%
\pgfpathlineto{\pgfqpoint{3.083085in}{1.027626in}}%
\pgfpathlineto{\pgfqpoint{3.146479in}{1.009102in}}%
\pgfpathlineto{\pgfqpoint{3.202582in}{0.990404in}}%
\pgfpathlineto{\pgfqpoint{3.251128in}{0.971692in}}%
\pgfpathlineto{\pgfqpoint{3.292811in}{0.953091in}}%
\pgfpathlineto{\pgfqpoint{3.328369in}{0.934702in}}%
\pgfpathlineto{\pgfqpoint{3.358406in}{0.916618in}}%
\pgfpathlineto{\pgfqpoint{3.383433in}{0.898921in}}%
\pgfpathlineto{\pgfqpoint{3.403858in}{0.881680in}}%
\pgfpathlineto{\pgfqpoint{3.419991in}{0.864956in}}%
\pgfpathlineto{\pgfqpoint{3.432190in}{0.848796in}}%
\pgfpathlineto{\pgfqpoint{3.441028in}{0.833237in}}%
\pgfpathlineto{\pgfqpoint{3.446875in}{0.818304in}}%
\pgfpathlineto{\pgfqpoint{3.450022in}{0.804018in}}%
\pgfpathlineto{\pgfqpoint{3.450678in}{0.790395in}}%
\pgfpathlineto{\pgfqpoint{3.448977in}{0.777449in}}%
\pgfpathlineto{\pgfqpoint{3.444972in}{0.765185in}}%
\pgfpathlineto{\pgfqpoint{3.438719in}{0.753609in}}%
\pgfpathlineto{\pgfqpoint{3.430372in}{0.742725in}}%
\pgfpathlineto{\pgfqpoint{3.420059in}{0.732537in}}%
\pgfpathlineto{\pgfqpoint{3.407870in}{0.723047in}}%
\pgfpathlineto{\pgfqpoint{3.393855in}{0.714259in}}%
\pgfpathlineto{\pgfqpoint{3.369430in}{0.702392in}}%
\pgfpathlineto{\pgfqpoint{3.340803in}{0.692104in}}%
\pgfpathlineto{\pgfqpoint{3.307664in}{0.683387in}}%
\pgfpathlineto{\pgfqpoint{3.269546in}{0.676233in}}%
\pgfpathlineto{\pgfqpoint{3.226419in}{0.670664in}}%
\pgfpathlineto{\pgfqpoint{3.177992in}{0.666712in}}%
\pgfpathlineto{\pgfqpoint{3.123768in}{0.664411in}}%
\pgfpathlineto{\pgfqpoint{3.063202in}{0.663807in}}%
\pgfpathlineto{\pgfqpoint{2.995705in}{0.664955in}}%
\pgfpathlineto{\pgfqpoint{2.893820in}{0.669324in}}%
\pgfpathlineto{\pgfqpoint{2.776844in}{0.677108in}}%
\pgfpathlineto{\pgfqpoint{2.643181in}{0.688530in}}%
\pgfpathlineto{\pgfqpoint{2.491987in}{0.703862in}}%
\pgfpathlineto{\pgfqpoint{2.324362in}{0.723325in}}%
\pgfpathlineto{\pgfqpoint{2.189994in}{0.740718in}}%
\pgfpathlineto{\pgfqpoint{2.051581in}{0.760491in}}%
\pgfpathlineto{\pgfqpoint{1.913396in}{0.782522in}}%
\pgfpathlineto{\pgfqpoint{1.780101in}{0.806580in}}%
\pgfpathlineto{\pgfqpoint{1.696444in}{0.823584in}}%
\pgfpathlineto{\pgfqpoint{1.618623in}{0.841234in}}%
\pgfpathlineto{\pgfqpoint{1.547234in}{0.859369in}}%
\pgfpathlineto{\pgfqpoint{1.482611in}{0.877828in}}%
\pgfpathlineto{\pgfqpoint{1.424931in}{0.896465in}}%
\pgfpathlineto{\pgfqpoint{1.374211in}{0.915148in}}%
\pgfpathlineto{\pgfqpoint{1.330308in}{0.933754in}}%
\pgfpathlineto{\pgfqpoint{1.292920in}{0.952176in}}%
\pgfpathlineto{\pgfqpoint{1.261587in}{0.970318in}}%
\pgfpathlineto{\pgfqpoint{1.235688in}{0.988098in}}%
\pgfpathlineto{\pgfqpoint{1.214465in}{1.005445in}}%
\pgfpathlineto{\pgfqpoint{1.197766in}{1.022281in}}%
\pgfpathlineto{\pgfqpoint{1.185049in}{1.038559in}}%
\pgfpathlineto{\pgfqpoint{1.175559in}{1.054252in}}%
\pgfpathlineto{\pgfqpoint{1.168725in}{1.069335in}}%
\pgfpathlineto{\pgfqpoint{1.164165in}{1.083788in}}%
\pgfpathlineto{\pgfqpoint{1.161686in}{1.097595in}}%
\pgfpathlineto{\pgfqpoint{1.161282in}{1.110744in}}%
\pgfpathlineto{\pgfqpoint{1.163135in}{1.123226in}}%
\pgfpathlineto{\pgfqpoint{1.167614in}{1.135038in}}%
\pgfpathlineto{\pgfqpoint{1.175185in}{1.146177in}}%
\pgfpathlineto{\pgfqpoint{1.185101in}{1.156622in}}%
\pgfpathlineto{\pgfqpoint{1.196961in}{1.166364in}}%
\pgfpathlineto{\pgfqpoint{1.210706in}{1.175401in}}%
\pgfpathlineto{\pgfqpoint{1.226308in}{1.183734in}}%
\pgfpathlineto{\pgfqpoint{1.253198in}{1.194908in}}%
\pgfpathlineto{\pgfqpoint{1.284386in}{1.204493in}}%
\pgfpathlineto{\pgfqpoint{1.320134in}{1.212490in}}%
\pgfpathlineto{\pgfqpoint{1.360819in}{1.218903in}}%
\pgfpathlineto{\pgfqpoint{1.406572in}{1.223711in}}%
\pgfpathlineto{\pgfqpoint{1.457819in}{1.226885in}}%
\pgfpathlineto{\pgfqpoint{1.515137in}{1.228383in}}%
\pgfpathlineto{\pgfqpoint{1.579122in}{1.228157in}}%
\pgfpathlineto{\pgfqpoint{1.650390in}{1.226145in}}%
\pgfpathlineto{\pgfqpoint{1.757849in}{1.220558in}}%
\pgfpathlineto{\pgfqpoint{1.880960in}{1.211467in}}%
\pgfpathlineto{\pgfqpoint{2.021197in}{1.198638in}}%
\pgfpathlineto{\pgfqpoint{2.178919in}{1.181801in}}%
\pgfpathlineto{\pgfqpoint{2.352144in}{1.160752in}}%
\pgfpathlineto{\pgfqpoint{2.489131in}{1.142169in}}%
\pgfpathlineto{\pgfqpoint{2.628391in}{1.121238in}}%
\pgfpathlineto{\pgfqpoint{2.765121in}{1.098134in}}%
\pgfpathlineto{\pgfqpoint{2.852319in}{1.081678in}}%
\pgfpathlineto{\pgfqpoint{2.934860in}{1.064504in}}%
\pgfpathlineto{\pgfqpoint{3.011698in}{1.046736in}}%
\pgfpathlineto{\pgfqpoint{3.081996in}{1.028509in}}%
\pgfpathlineto{\pgfqpoint{3.145112in}{1.009971in}}%
\pgfpathlineto{\pgfqpoint{3.200640in}{0.991283in}}%
\pgfpathlineto{\pgfqpoint{3.249116in}{0.972576in}}%
\pgfpathlineto{\pgfqpoint{3.291189in}{0.953963in}}%
\pgfpathlineto{\pgfqpoint{3.327409in}{0.935549in}}%
\pgfpathlineto{\pgfqpoint{3.358228in}{0.917427in}}%
\pgfpathlineto{\pgfqpoint{3.384000in}{0.899682in}}%
\pgfpathlineto{\pgfqpoint{3.404980in}{0.882387in}}%
\pgfpathlineto{\pgfqpoint{3.421328in}{0.865605in}}%
\pgfpathlineto{\pgfqpoint{3.433443in}{0.849391in}}%
\pgfpathlineto{\pgfqpoint{3.442144in}{0.833783in}}%
\pgfpathlineto{\pgfqpoint{3.447889in}{0.818803in}}%
\pgfpathlineto{\pgfqpoint{3.451030in}{0.804472in}}%
\pgfpathlineto{\pgfqpoint{3.451811in}{0.790806in}}%
\pgfpathlineto{\pgfqpoint{3.450375in}{0.777815in}}%
\pgfpathlineto{\pgfqpoint{3.446756in}{0.765508in}}%
\pgfpathlineto{\pgfqpoint{3.440886in}{0.753889in}}%
\pgfpathlineto{\pgfqpoint{3.432662in}{0.742958in}}%
\pgfpathlineto{\pgfqpoint{3.422359in}{0.732722in}}%
\pgfpathlineto{\pgfqpoint{3.410111in}{0.723185in}}%
\pgfpathlineto{\pgfqpoint{3.395978in}{0.714350in}}%
\pgfpathlineto{\pgfqpoint{3.379990in}{0.706217in}}%
\pgfpathlineto{\pgfqpoint{3.352522in}{0.695338in}}%
\pgfpathlineto{\pgfqpoint{3.320749in}{0.686046in}}%
\pgfpathlineto{\pgfqpoint{3.284394in}{0.678338in}}%
\pgfpathlineto{\pgfqpoint{3.243032in}{0.672213in}}%
\pgfpathlineto{\pgfqpoint{3.196479in}{0.667686in}}%
\pgfpathlineto{\pgfqpoint{3.144388in}{0.664795in}}%
\pgfpathlineto{\pgfqpoint{3.086133in}{0.663583in}}%
\pgfpathlineto{\pgfqpoint{3.021086in}{0.664100in}}%
\pgfpathlineto{\pgfqpoint{2.922708in}{0.667594in}}%
\pgfpathlineto{\pgfqpoint{2.809652in}{0.674459in}}%
\pgfpathlineto{\pgfqpoint{2.680430in}{0.684911in}}%
\pgfpathlineto{\pgfqpoint{2.533769in}{0.699190in}}%
\pgfpathlineto{\pgfqpoint{2.370037in}{0.717561in}}%
\pgfpathlineto{\pgfqpoint{2.192423in}{0.740197in}}%
\pgfpathlineto{\pgfqpoint{2.053989in}{0.759946in}}%
\pgfpathlineto{\pgfqpoint{1.915552in}{0.781951in}}%
\pgfpathlineto{\pgfqpoint{1.782103in}{0.806023in}}%
\pgfpathlineto{\pgfqpoint{1.698306in}{0.823036in}}%
\pgfpathlineto{\pgfqpoint{1.620001in}{0.840659in}}%
\pgfpathlineto{\pgfqpoint{1.548034in}{0.858755in}}%
\pgfpathlineto{\pgfqpoint{1.482975in}{0.877190in}}%
\pgfpathlineto{\pgfqpoint{1.425111in}{0.895830in}}%
\pgfpathlineto{\pgfqpoint{1.374451in}{0.914542in}}%
\pgfpathlineto{\pgfqpoint{1.330726in}{0.933197in}}%
\pgfpathlineto{\pgfqpoint{1.293827in}{0.951646in}}%
\pgfpathlineto{\pgfqpoint{1.262982in}{0.969792in}}%
\pgfpathlineto{\pgfqpoint{1.237092in}{0.987573in}}%
\pgfpathlineto{\pgfqpoint{1.215322in}{1.004932in}}%
\pgfpathlineto{\pgfqpoint{1.197100in}{1.021818in}}%
\pgfpathlineto{\pgfqpoint{1.182120in}{1.038184in}}%
\pgfpathlineto{\pgfqpoint{1.170340in}{1.053986in}}%
\pgfpathlineto{\pgfqpoint{1.161978in}{1.069188in}}%
\pgfpathlineto{\pgfqpoint{1.157522in}{1.083755in}}%
\pgfpathlineto{\pgfqpoint{1.156706in}{1.097656in}}%
\pgfpathlineto{\pgfqpoint{1.158314in}{1.110874in}}%
\pgfpathlineto{\pgfqpoint{1.162129in}{1.123399in}}%
\pgfpathlineto{\pgfqpoint{1.167997in}{1.135226in}}%
\pgfpathlineto{\pgfqpoint{1.175812in}{1.146353in}}%
\pgfpathlineto{\pgfqpoint{1.185523in}{1.156775in}}%
\pgfpathlineto{\pgfqpoint{1.197128in}{1.166495in}}%
\pgfpathlineto{\pgfqpoint{1.210678in}{1.175514in}}%
\pgfpathlineto{\pgfqpoint{1.226267in}{1.183835in}}%
\pgfpathlineto{\pgfqpoint{1.253355in}{1.195007in}}%
\pgfpathlineto{\pgfqpoint{1.284833in}{1.204598in}}%
\pgfpathlineto{\pgfqpoint{1.320822in}{1.212598in}}%
\pgfpathlineto{\pgfqpoint{1.361547in}{1.218991in}}%
\pgfpathlineto{\pgfqpoint{1.407337in}{1.223760in}}%
\pgfpathlineto{\pgfqpoint{1.458626in}{1.226884in}}%
\pgfpathlineto{\pgfqpoint{1.515951in}{1.228338in}}%
\pgfpathlineto{\pgfqpoint{1.579775in}{1.228103in}}%
\pgfpathlineto{\pgfqpoint{1.675620in}{1.225081in}}%
\pgfpathlineto{\pgfqpoint{1.786714in}{1.218679in}}%
\pgfpathlineto{\pgfqpoint{1.915050in}{1.208621in}}%
\pgfpathlineto{\pgfqpoint{2.060955in}{1.194688in}}%
\pgfpathlineto{\pgfqpoint{2.223092in}{1.176714in}}%
\pgfpathlineto{\pgfqpoint{2.398458in}{1.154590in}}%
\pgfpathlineto{\pgfqpoint{2.535921in}{1.135238in}}%
\pgfpathlineto{\pgfqpoint{2.674831in}{1.113531in}}%
\pgfpathlineto{\pgfqpoint{2.809483in}{1.089727in}}%
\pgfpathlineto{\pgfqpoint{2.894485in}{1.072888in}}%
\pgfpathlineto{\pgfqpoint{2.974332in}{1.055415in}}%
\pgfpathlineto{\pgfqpoint{3.048138in}{1.037433in}}%
\pgfpathlineto{\pgfqpoint{3.115246in}{1.019069in}}%
\pgfpathlineto{\pgfqpoint{3.175232in}{1.000455in}}%
\pgfpathlineto{\pgfqpoint{3.227904in}{0.981724in}}%
\pgfpathlineto{\pgfqpoint{3.273304in}{0.963013in}}%
\pgfpathlineto{\pgfqpoint{3.311688in}{0.944475in}}%
\pgfpathlineto{\pgfqpoint{3.343965in}{0.926213in}}%
\pgfpathlineto{\pgfqpoint{3.371111in}{0.908301in}}%
\pgfpathlineto{\pgfqpoint{3.393874in}{0.890802in}}%
\pgfpathlineto{\pgfqpoint{3.412774in}{0.873775in}}%
\pgfpathlineto{\pgfqpoint{3.428105in}{0.857270in}}%
\pgfpathlineto{\pgfqpoint{3.439933in}{0.841333in}}%
\pgfpathlineto{\pgfqpoint{3.448095in}{0.826001in}}%
\pgfpathlineto{\pgfqpoint{3.452430in}{0.811308in}}%
\pgfpathlineto{\pgfqpoint{3.453927in}{0.797280in}}%
\pgfpathlineto{\pgfqpoint{3.452998in}{0.783930in}}%
\pgfpathlineto{\pgfqpoint{3.449856in}{0.771267in}}%
\pgfpathlineto{\pgfqpoint{3.444653in}{0.759297in}}%
\pgfpathlineto{\pgfqpoint{3.437483in}{0.748024in}}%
\pgfpathlineto{\pgfqpoint{3.428378in}{0.737451in}}%
\pgfpathlineto{\pgfqpoint{3.417311in}{0.727578in}}%
\pgfpathlineto{\pgfqpoint{3.404196in}{0.718400in}}%
\pgfpathlineto{\pgfqpoint{3.389044in}{0.709919in}}%
\pgfpathlineto{\pgfqpoint{3.362699in}{0.698507in}}%
\pgfpathlineto{\pgfqpoint{3.332008in}{0.688677in}}%
\pgfpathlineto{\pgfqpoint{3.296863in}{0.680439in}}%
\pgfpathlineto{\pgfqpoint{3.257040in}{0.673803in}}%
\pgfpathlineto{\pgfqpoint{3.212205in}{0.668785in}}%
\pgfpathlineto{\pgfqpoint{3.161908in}{0.665399in}}%
\pgfpathlineto{\pgfqpoint{3.105608in}{0.663663in}}%
\pgfpathlineto{\pgfqpoint{3.043202in}{0.663602in}}%
\pgfpathlineto{\pgfqpoint{2.973716in}{0.665301in}}%
\pgfpathlineto{\pgfqpoint{2.868128in}{0.670471in}}%
\pgfpathlineto{\pgfqpoint{2.746199in}{0.679164in}}%
\pgfpathlineto{\pgfqpoint{2.607256in}{0.691592in}}%
\pgfpathlineto{\pgfqpoint{2.451710in}{0.707961in}}%
\pgfpathlineto{\pgfqpoint{2.281058in}{0.728466in}}%
\pgfpathlineto{\pgfqpoint{2.144760in}{0.746678in}}%
\pgfpathlineto{\pgfqpoint{2.005218in}{0.767303in}}%
\pgfpathlineto{\pgfqpoint{1.867972in}{0.790098in}}%
\pgfpathlineto{\pgfqpoint{1.737759in}{0.814758in}}%
\pgfpathlineto{\pgfqpoint{1.656788in}{0.832061in}}%
\pgfpathlineto{\pgfqpoint{1.581576in}{0.849920in}}%
\pgfpathlineto{\pgfqpoint{1.512856in}{0.868208in}}%
\pgfpathlineto{\pgfqpoint{1.451181in}{0.886789in}}%
\pgfpathlineto{\pgfqpoint{1.396924in}{0.905516in}}%
\pgfpathlineto{\pgfqpoint{1.350194in}{0.924231in}}%
\pgfpathlineto{\pgfqpoint{1.310295in}{0.942800in}}%
\pgfpathlineto{\pgfqpoint{1.276307in}{0.961131in}}%
\pgfpathlineto{\pgfqpoint{1.247491in}{0.979140in}}%
\pgfpathlineto{\pgfqpoint{1.223290in}{0.996754in}}%
\pgfpathlineto{\pgfqpoint{1.203328in}{1.013906in}}%
\pgfpathlineto{\pgfqpoint{1.187410in}{1.030539in}}%
\pgfpathlineto{\pgfqpoint{1.175524in}{1.046604in}}%
\pgfpathlineto{\pgfqpoint{1.167560in}{1.062060in}}%
\pgfpathlineto{\pgfqpoint{1.162645in}{1.076876in}}%
\pgfpathlineto{\pgfqpoint{1.160394in}{1.091036in}}%
\pgfpathlineto{\pgfqpoint{1.160547in}{1.104525in}}%
\pgfpathlineto{\pgfqpoint{1.162914in}{1.117332in}}%
\pgfpathlineto{\pgfqpoint{1.167382in}{1.129451in}}%
\pgfpathlineto{\pgfqpoint{1.173909in}{1.140877in}}%
\pgfpathlineto{\pgfqpoint{1.182530in}{1.151608in}}%
\pgfpathlineto{\pgfqpoint{1.193299in}{1.161646in}}%
\pgfpathlineto{\pgfqpoint{1.206063in}{1.170988in}}%
\pgfpathlineto{\pgfqpoint{1.220745in}{1.179631in}}%
\pgfpathlineto{\pgfqpoint{1.246300in}{1.191280in}}%
\pgfpathlineto{\pgfqpoint{1.276105in}{1.201344in}}%
\pgfpathlineto{\pgfqpoint{1.310293in}{1.209818in}}%
\pgfpathlineto{\pgfqpoint{1.349136in}{1.216696in}}%
\pgfpathlineto{\pgfqpoint{1.393037in}{1.221975in}}%
\pgfpathlineto{\pgfqpoint{1.442443in}{1.225649in}}%
\pgfpathlineto{\pgfqpoint{1.497503in}{1.227678in}}%
\pgfpathlineto{\pgfqpoint{1.558987in}{1.228008in}}%
\pgfpathlineto{\pgfqpoint{1.627691in}{1.226576in}}%
\pgfpathlineto{\pgfqpoint{1.731713in}{1.221800in}}%
\pgfpathlineto{\pgfqpoint{1.851167in}{1.213570in}}%
\pgfpathlineto{\pgfqpoint{1.987118in}{1.201662in}}%
\pgfpathlineto{\pgfqpoint{2.140324in}{1.185822in}}%
\pgfpathlineto{\pgfqpoint{2.312321in}{1.165737in}}%
\pgfpathlineto{\pgfqpoint{2.449395in}{1.147836in}}%
\pgfpathlineto{\pgfqpoint{2.588039in}{1.127604in}}%
\pgfpathlineto{\pgfqpoint{2.723994in}{1.105211in}}%
\pgfpathlineto{\pgfqpoint{2.853514in}{1.080899in}}%
\pgfpathlineto{\pgfqpoint{2.934669in}{1.063779in}}%
\pgfpathlineto{\pgfqpoint{3.010688in}{1.046055in}}%
\pgfpathlineto{\pgfqpoint{3.080841in}{1.027853in}}%
\pgfpathlineto{\pgfqpoint{3.144503in}{1.009315in}}%
\pgfpathlineto{\pgfqpoint{3.201147in}{0.990592in}}%
\pgfpathlineto{\pgfqpoint{3.250350in}{0.971855in}}%
\pgfpathlineto{\pgfqpoint{3.292251in}{0.953252in}}%
\pgfpathlineto{\pgfqpoint{3.327905in}{0.934863in}}%
\pgfpathlineto{\pgfqpoint{3.357902in}{0.916781in}}%
\pgfpathlineto{\pgfqpoint{3.382774in}{0.899089in}}%
\pgfpathlineto{\pgfqpoint{3.403001in}{0.881857in}}%
\pgfpathlineto{\pgfqpoint{3.419037in}{0.865142in}}%
\pgfpathlineto{\pgfqpoint{3.431352in}{0.848989in}}%
\pgfpathlineto{\pgfqpoint{3.440336in}{0.833432in}}%
\pgfpathlineto{\pgfqpoint{3.446306in}{0.818501in}}%
\pgfpathlineto{\pgfqpoint{3.449509in}{0.804216in}}%
\pgfpathlineto{\pgfqpoint{3.450096in}{0.790594in}}%
\pgfpathlineto{\pgfqpoint{3.448182in}{0.777646in}}%
\pgfpathlineto{\pgfqpoint{3.444017in}{0.765381in}}%
\pgfpathlineto{\pgfqpoint{3.437802in}{0.753805in}}%
\pgfpathlineto{\pgfqpoint{3.429683in}{0.742923in}}%
\pgfpathlineto{\pgfqpoint{3.419746in}{0.732739in}}%
\pgfpathlineto{\pgfqpoint{3.408018in}{0.723254in}}%
\pgfpathlineto{\pgfqpoint{3.394468in}{0.714466in}}%
\pgfpathlineto{\pgfqpoint{3.379006in}{0.706372in}}%
\pgfpathlineto{\pgfqpoint{3.361486in}{0.698968in}}%
\pgfpathlineto{\pgfqpoint{3.331071in}{0.689148in}}%
\pgfpathlineto{\pgfqpoint{3.296040in}{0.680898in}}%
\pgfpathlineto{\pgfqpoint{3.256260in}{0.674238in}}%
\pgfpathlineto{\pgfqpoint{3.211429in}{0.669187in}}%
\pgfpathlineto{\pgfqpoint{3.161162in}{0.665772in}}%
\pgfpathlineto{\pgfqpoint{3.104986in}{0.664029in}}%
\pgfpathlineto{\pgfqpoint{3.042341in}{0.663996in}}%
\pgfpathlineto{\pgfqpoint{2.947641in}{0.666696in}}%
\pgfpathlineto{\pgfqpoint{2.838863in}{0.672678in}}%
\pgfpathlineto{\pgfqpoint{2.713759in}{0.682206in}}%
\pgfpathlineto{\pgfqpoint{2.571172in}{0.695532in}}%
\pgfpathlineto{\pgfqpoint{2.411648in}{0.712870in}}%
\pgfpathlineto{\pgfqpoint{2.237436in}{0.734400in}}%
\pgfpathlineto{\pgfqpoint{2.099561in}{0.753394in}}%
\pgfpathlineto{\pgfqpoint{1.960378in}{0.774721in}}%
\pgfpathlineto{\pgfqpoint{1.825131in}{0.798122in}}%
\pgfpathlineto{\pgfqpoint{1.739355in}{0.814723in}}%
\pgfpathlineto{\pgfqpoint{1.658386in}{0.832002in}}%
\pgfpathlineto{\pgfqpoint{1.583141in}{0.849841in}}%
\pgfpathlineto{\pgfqpoint{1.514371in}{0.868112in}}%
\pgfpathlineto{\pgfqpoint{1.452657in}{0.886677in}}%
\pgfpathlineto{\pgfqpoint{1.398410in}{0.905385in}}%
\pgfpathlineto{\pgfqpoint{1.351647in}{0.924077in}}%
\pgfpathlineto{\pgfqpoint{1.311565in}{0.942631in}}%
\pgfpathlineto{\pgfqpoint{1.277332in}{0.960952in}}%
\pgfpathlineto{\pgfqpoint{1.248282in}{0.978957in}}%
\pgfpathlineto{\pgfqpoint{1.223915in}{0.996567in}}%
\pgfpathlineto{\pgfqpoint{1.203891in}{1.013716in}}%
\pgfpathlineto{\pgfqpoint{1.188040in}{1.030345in}}%
\pgfpathlineto{\pgfqpoint{1.176352in}{1.046406in}}%
\pgfpathlineto{\pgfqpoint{1.168370in}{1.061856in}}%
\pgfpathlineto{\pgfqpoint{1.163366in}{1.076669in}}%
\pgfpathlineto{\pgfqpoint{1.161007in}{1.090827in}}%
\pgfpathlineto{\pgfqpoint{1.161040in}{1.104316in}}%
\pgfpathlineto{\pgfqpoint{1.163291in}{1.117126in}}%
\pgfpathlineto{\pgfqpoint{1.167663in}{1.129248in}}%
\pgfpathlineto{\pgfqpoint{1.174138in}{1.140678in}}%
\pgfpathlineto{\pgfqpoint{1.182778in}{1.151416in}}%
\pgfpathlineto{\pgfqpoint{1.193542in}{1.161462in}}%
\pgfpathlineto{\pgfqpoint{1.206270in}{1.170810in}}%
\pgfpathlineto{\pgfqpoint{1.220901in}{1.179459in}}%
\pgfpathlineto{\pgfqpoint{1.246359in}{1.191116in}}%
\pgfpathlineto{\pgfqpoint{1.276049in}{1.201189in}}%
\pgfpathlineto{\pgfqpoint{1.310125in}{1.209672in}}%
\pgfpathlineto{\pgfqpoint{1.348882in}{1.216565in}}%
\pgfpathlineto{\pgfqpoint{1.392755in}{1.221864in}}%
\pgfpathlineto{\pgfqpoint{1.442062in}{1.225560in}}%
\pgfpathlineto{\pgfqpoint{1.497076in}{1.227605in}}%
\pgfpathlineto{\pgfqpoint{1.558559in}{1.227950in}}%
\pgfpathlineto{\pgfqpoint{1.627228in}{1.226535in}}%
\pgfpathlineto{\pgfqpoint{1.731086in}{1.221786in}}%
\pgfpathlineto{\pgfqpoint{1.850267in}{1.213587in}}%
\pgfpathlineto{\pgfqpoint{1.985975in}{1.201711in}}%
\pgfpathlineto{\pgfqpoint{2.139777in}{1.185890in}}%
\pgfpathlineto{\pgfqpoint{2.313743in}{1.165808in}}%
\pgfpathlineto{\pgfqpoint{2.451390in}{1.147932in}}%
\pgfpathlineto{\pgfqpoint{2.589927in}{1.127739in}}%
\pgfpathlineto{\pgfqpoint{2.725240in}{1.105391in}}%
\pgfpathlineto{\pgfqpoint{2.853790in}{1.081118in}}%
\pgfpathlineto{\pgfqpoint{2.934243in}{1.064013in}}%
\pgfpathlineto{\pgfqpoint{3.009623in}{1.046291in}}%
\pgfpathlineto{\pgfqpoint{3.079303in}{1.028073in}}%
\pgfpathlineto{\pgfqpoint{3.142773in}{1.009493in}}%
\pgfpathlineto{\pgfqpoint{3.199633in}{0.990698in}}%
\pgfpathlineto{\pgfqpoint{3.249600in}{0.971851in}}%
\pgfpathlineto{\pgfqpoint{3.292499in}{0.953127in}}%
\pgfpathlineto{\pgfqpoint{3.328344in}{0.934705in}}%
\pgfpathlineto{\pgfqpoint{3.358233in}{0.916626in}}%
\pgfpathlineto{\pgfqpoint{3.383000in}{0.898938in}}%
\pgfpathlineto{\pgfqpoint{3.403179in}{0.881710in}}%
\pgfpathlineto{\pgfqpoint{3.419228in}{0.864997in}}%
\pgfpathlineto{\pgfqpoint{3.431529in}{0.848847in}}%
\pgfpathlineto{\pgfqpoint{3.440467in}{0.833296in}}%
\pgfpathlineto{\pgfqpoint{3.446382in}{0.818369in}}%
\pgfpathlineto{\pgfqpoint{3.449542in}{0.804090in}}%
\pgfpathlineto{\pgfqpoint{3.450159in}{0.790474in}}%
\pgfpathlineto{\pgfqpoint{3.448378in}{0.777533in}}%
\pgfpathlineto{\pgfqpoint{3.444242in}{0.765273in}}%
\pgfpathlineto{\pgfqpoint{3.437923in}{0.753700in}}%
\pgfpathlineto{\pgfqpoint{3.429598in}{0.742820in}}%
\pgfpathlineto{\pgfqpoint{3.419395in}{0.732638in}}%
\pgfpathlineto{\pgfqpoint{3.407393in}{0.723155in}}%
\pgfpathlineto{\pgfqpoint{3.393621in}{0.714373in}}%
\pgfpathlineto{\pgfqpoint{3.378062in}{0.706291in}}%
\pgfpathlineto{\pgfqpoint{3.351212in}{0.695477in}}%
\pgfpathlineto{\pgfqpoint{3.319744in}{0.686221in}}%
\pgfpathlineto{\pgfqpoint{3.283205in}{0.678514in}}%
\pgfpathlineto{\pgfqpoint{3.241794in}{0.672392in}}%
\pgfpathlineto{\pgfqpoint{3.195202in}{0.667881in}}%
\pgfpathlineto{\pgfqpoint{3.142993in}{0.665012in}}%
\pgfpathlineto{\pgfqpoint{3.084665in}{0.663826in}}%
\pgfpathlineto{\pgfqpoint{3.019650in}{0.664372in}}%
\pgfpathlineto{\pgfqpoint{2.921454in}{0.667897in}}%
\pgfpathlineto{\pgfqpoint{2.808559in}{0.674773in}}%
\pgfpathlineto{\pgfqpoint{2.679318in}{0.685229in}}%
\pgfpathlineto{\pgfqpoint{2.532607in}{0.699530in}}%
\pgfpathlineto{\pgfqpoint{2.369028in}{0.717907in}}%
\pgfpathlineto{\pgfqpoint{2.191396in}{0.740529in}}%
\pgfpathlineto{\pgfqpoint{2.053082in}{0.760278in}}%
\pgfpathlineto{\pgfqpoint{1.914890in}{0.782285in}}%
\pgfpathlineto{\pgfqpoint{1.781483in}{0.806322in}}%
\pgfpathlineto{\pgfqpoint{1.697789in}{0.823318in}}%
\pgfpathlineto{\pgfqpoint{1.619874in}{0.840964in}}%
\pgfpathlineto{\pgfqpoint{1.548330in}{0.859092in}}%
\pgfpathlineto{\pgfqpoint{1.483538in}{0.877546in}}%
\pgfpathlineto{\pgfqpoint{1.425708in}{0.896179in}}%
\pgfpathlineto{\pgfqpoint{1.374874in}{0.914859in}}%
\pgfpathlineto{\pgfqpoint{1.330902in}{0.933464in}}%
\pgfpathlineto{\pgfqpoint{1.293480in}{0.951887in}}%
\pgfpathlineto{\pgfqpoint{1.262128in}{0.970032in}}%
\pgfpathlineto{\pgfqpoint{1.236191in}{0.987817in}}%
\pgfpathlineto{\pgfqpoint{1.214895in}{1.005170in}}%
\pgfpathlineto{\pgfqpoint{1.198145in}{1.022012in}}%
\pgfpathlineto{\pgfqpoint{1.185332in}{1.038298in}}%
\pgfpathlineto{\pgfqpoint{1.175720in}{1.054000in}}%
\pgfpathlineto{\pgfqpoint{1.168761in}{1.069093in}}%
\pgfpathlineto{\pgfqpoint{1.164093in}{1.083558in}}%
\pgfpathlineto{\pgfqpoint{1.161545in}{1.097377in}}%
\pgfpathlineto{\pgfqpoint{1.161130in}{1.110537in}}%
\pgfpathlineto{\pgfqpoint{1.163054in}{1.123032in}}%
\pgfpathlineto{\pgfqpoint{1.167705in}{1.134855in}}%
\pgfpathlineto{\pgfqpoint{1.175412in}{1.146002in}}%
\pgfpathlineto{\pgfqpoint{1.185291in}{1.156452in}}%
\pgfpathlineto{\pgfqpoint{1.197108in}{1.166200in}}%
\pgfpathlineto{\pgfqpoint{1.210805in}{1.175243in}}%
\pgfpathlineto{\pgfqpoint{1.226354in}{1.183582in}}%
\pgfpathlineto{\pgfqpoint{1.253162in}{1.194766in}}%
\pgfpathlineto{\pgfqpoint{1.284274in}{1.204362in}}%
\pgfpathlineto{\pgfqpoint{1.319968in}{1.212372in}}%
\pgfpathlineto{\pgfqpoint{1.360605in}{1.218800in}}%
\pgfpathlineto{\pgfqpoint{1.406295in}{1.223623in}}%
\pgfpathlineto{\pgfqpoint{1.457493in}{1.226811in}}%
\pgfpathlineto{\pgfqpoint{1.514754in}{1.228325in}}%
\pgfpathlineto{\pgfqpoint{1.578663in}{1.228114in}}%
\pgfpathlineto{\pgfqpoint{1.649833in}{1.226118in}}%
\pgfpathlineto{\pgfqpoint{1.757139in}{1.220555in}}%
\pgfpathlineto{\pgfqpoint{1.880105in}{1.211491in}}%
\pgfpathlineto{\pgfqpoint{2.020196in}{1.198692in}}%
\pgfpathlineto{\pgfqpoint{2.177784in}{1.181885in}}%
\pgfpathlineto{\pgfqpoint{2.350852in}{1.160875in}}%
\pgfpathlineto{\pgfqpoint{2.487794in}{1.142318in}}%
\pgfpathlineto{\pgfqpoint{2.627077in}{1.121410in}}%
\pgfpathlineto{\pgfqpoint{2.763711in}{1.098341in}}%
\pgfpathlineto{\pgfqpoint{2.850952in}{1.081900in}}%
\pgfpathlineto{\pgfqpoint{2.933629in}{1.064735in}}%
\pgfpathlineto{\pgfqpoint{3.010609in}{1.046970in}}%
\pgfpathlineto{\pgfqpoint{3.080838in}{1.028749in}}%
\pgfpathlineto{\pgfqpoint{3.143596in}{1.010228in}}%
\pgfpathlineto{\pgfqpoint{3.199212in}{0.991549in}}%
\pgfpathlineto{\pgfqpoint{3.248108in}{0.972841in}}%
\pgfpathlineto{\pgfqpoint{3.290684in}{0.954223in}}%
\pgfpathlineto{\pgfqpoint{3.327312in}{0.935801in}}%
\pgfpathlineto{\pgfqpoint{3.358339in}{0.917672in}}%
\pgfpathlineto{\pgfqpoint{3.384084in}{0.899920in}}%
\pgfpathlineto{\pgfqpoint{3.404843in}{0.882620in}}%
\pgfpathlineto{\pgfqpoint{3.420910in}{0.865834in}}%
\pgfpathlineto{\pgfqpoint{3.432933in}{0.849620in}}%
\pgfpathlineto{\pgfqpoint{3.441588in}{0.834010in}}%
\pgfpathlineto{\pgfqpoint{3.447381in}{0.819026in}}%
\pgfpathlineto{\pgfqpoint{3.450689in}{0.804688in}}%
\pgfpathlineto{\pgfqpoint{3.451753in}{0.791012in}}%
\pgfpathlineto{\pgfqpoint{3.450684in}{0.778009in}}%
\pgfpathlineto{\pgfqpoint{3.447462in}{0.765686in}}%
\pgfpathlineto{\pgfqpoint{3.441933in}{0.754046in}}%
\pgfpathlineto{\pgfqpoint{3.433844in}{0.743090in}}%
\pgfpathlineto{\pgfqpoint{3.423544in}{0.732828in}}%
\pgfpathlineto{\pgfqpoint{3.411295in}{0.723266in}}%
\pgfpathlineto{\pgfqpoint{3.397156in}{0.714406in}}%
\pgfpathlineto{\pgfqpoint{3.381155in}{0.706250in}}%
\pgfpathlineto{\pgfqpoint{3.353657in}{0.695338in}}%
\pgfpathlineto{\pgfqpoint{3.321848in}{0.686014in}}%
\pgfpathlineto{\pgfqpoint{3.285467in}{0.678278in}}%
\pgfpathlineto{\pgfqpoint{3.244111in}{0.672127in}}%
\pgfpathlineto{\pgfqpoint{3.197582in}{0.667578in}}%
\pgfpathlineto{\pgfqpoint{3.145528in}{0.664665in}}%
\pgfpathlineto{\pgfqpoint{3.087315in}{0.663431in}}%
\pgfpathlineto{\pgfqpoint{3.022312in}{0.663928in}}%
\pgfpathlineto{\pgfqpoint{2.923995in}{0.667394in}}%
\pgfpathlineto{\pgfqpoint{2.811005in}{0.674232in}}%
\pgfpathlineto{\pgfqpoint{2.681861in}{0.684657in}}%
\pgfpathlineto{\pgfqpoint{2.535272in}{0.698908in}}%
\pgfpathlineto{\pgfqpoint{2.371586in}{0.717249in}}%
\pgfpathlineto{\pgfqpoint{2.193989in}{0.739861in}}%
\pgfpathlineto{\pgfqpoint{2.055527in}{0.759596in}}%
\pgfpathlineto{\pgfqpoint{1.916995in}{0.781586in}}%
\pgfpathlineto{\pgfqpoint{1.783425in}{0.805643in}}%
\pgfpathlineto{\pgfqpoint{1.699537in}{0.822656in}}%
\pgfpathlineto{\pgfqpoint{1.621127in}{0.840282in}}%
\pgfpathlineto{\pgfqpoint{1.549042in}{0.858382in}}%
\pgfpathlineto{\pgfqpoint{1.483852in}{0.876819in}}%
\pgfpathlineto{\pgfqpoint{1.425852in}{0.895459in}}%
\pgfpathlineto{\pgfqpoint{1.375059in}{0.914173in}}%
\pgfpathlineto{\pgfqpoint{1.331217in}{0.932832in}}%
\pgfpathlineto{\pgfqpoint{1.293989in}{0.951305in}}%
\pgfpathlineto{\pgfqpoint{1.263045in}{0.969467in}}%
\pgfpathlineto{\pgfqpoint{1.237264in}{0.987257in}}%
\pgfpathlineto{\pgfqpoint{1.215725in}{1.004621in}}%
\pgfpathlineto{\pgfqpoint{1.197765in}{1.021507in}}%
\pgfpathlineto{\pgfqpoint{1.182983in}{1.037872in}}%
\pgfpathlineto{\pgfqpoint{1.171242in}{1.053675in}}%
\pgfpathlineto{\pgfqpoint{1.162666in}{1.068879in}}%
\pgfpathlineto{\pgfqpoint{1.157640in}{1.083454in}}%
\pgfpathlineto{\pgfqpoint{1.156528in}{1.097372in}}%
\pgfpathlineto{\pgfqpoint{1.158087in}{1.110606in}}%
\pgfpathlineto{\pgfqpoint{1.161882in}{1.123148in}}%
\pgfpathlineto{\pgfqpoint{1.167747in}{1.134993in}}%
\pgfpathlineto{\pgfqpoint{1.175568in}{1.146135in}}%
\pgfpathlineto{\pgfqpoint{1.185280in}{1.156575in}}%
\pgfpathlineto{\pgfqpoint{1.196864in}{1.166310in}}%
\pgfpathlineto{\pgfqpoint{1.210353in}{1.175343in}}%
\pgfpathlineto{\pgfqpoint{1.225826in}{1.183676in}}%
\pgfpathlineto{\pgfqpoint{1.252833in}{1.194869in}}%
\pgfpathlineto{\pgfqpoint{1.284251in}{1.204483in}}%
\pgfpathlineto{\pgfqpoint{1.320193in}{1.212508in}}%
\pgfpathlineto{\pgfqpoint{1.360879in}{1.218929in}}%
\pgfpathlineto{\pgfqpoint{1.406630in}{1.223725in}}%
\pgfpathlineto{\pgfqpoint{1.457866in}{1.226875in}}%
\pgfpathlineto{\pgfqpoint{1.515106in}{1.228350in}}%
\pgfpathlineto{\pgfqpoint{1.578962in}{1.228120in}}%
\pgfpathlineto{\pgfqpoint{1.674747in}{1.225143in}}%
\pgfpathlineto{\pgfqpoint{1.785420in}{1.218828in}}%
\pgfpathlineto{\pgfqpoint{1.913449in}{1.208838in}}%
\pgfpathlineto{\pgfqpoint{2.059354in}{1.194924in}}%
\pgfpathlineto{\pgfqpoint{2.221716in}{1.176927in}}%
\pgfpathlineto{\pgfqpoint{2.397173in}{1.154775in}}%
\pgfpathlineto{\pgfqpoint{2.534240in}{1.135442in}}%
\pgfpathlineto{\pgfqpoint{2.672704in}{1.113820in}}%
\pgfpathlineto{\pgfqpoint{2.807382in}{1.090052in}}%
\pgfpathlineto{\pgfqpoint{2.892446in}{1.073215in}}%
\pgfpathlineto{\pgfqpoint{2.972361in}{1.055741in}}%
\pgfpathlineto{\pgfqpoint{3.046227in}{1.037762in}}%
\pgfpathlineto{\pgfqpoint{3.113397in}{1.019410in}}%
\pgfpathlineto{\pgfqpoint{3.173472in}{1.000815in}}%
\pgfpathlineto{\pgfqpoint{3.226303in}{0.982107in}}%
\pgfpathlineto{\pgfqpoint{3.271992in}{0.963414in}}%
\pgfpathlineto{\pgfqpoint{3.310838in}{0.944866in}}%
\pgfpathlineto{\pgfqpoint{3.343117in}{0.926609in}}%
\pgfpathlineto{\pgfqpoint{3.369923in}{0.908716in}}%
\pgfpathlineto{\pgfqpoint{3.392294in}{0.891238in}}%
\pgfpathlineto{\pgfqpoint{3.410996in}{0.874225in}}%
\pgfpathlineto{\pgfqpoint{3.426522in}{0.857722in}}%
\pgfpathlineto{\pgfqpoint{3.439093in}{0.841770in}}%
\pgfpathlineto{\pgfqpoint{3.448658in}{0.826404in}}%
\pgfpathlineto{\pgfqpoint{3.454893in}{0.811658in}}%
\pgfpathlineto{\pgfqpoint{3.457204in}{0.797561in}}%
\pgfpathlineto{\pgfqpoint{3.456142in}{0.784141in}}%
\pgfpathlineto{\pgfqpoint{3.452795in}{0.771414in}}%
\pgfpathlineto{\pgfqpoint{3.447337in}{0.759386in}}%
\pgfpathlineto{\pgfqpoint{3.439897in}{0.748060in}}%
\pgfpathlineto{\pgfqpoint{3.430557in}{0.737440in}}%
\pgfpathlineto{\pgfqpoint{3.419352in}{0.727525in}}%
\pgfpathlineto{\pgfqpoint{3.406272in}{0.718315in}}%
\pgfpathlineto{\pgfqpoint{3.391257in}{0.709808in}}%
\pgfpathlineto{\pgfqpoint{3.374212in}{0.701999in}}%
\pgfpathlineto{\pgfqpoint{3.344931in}{0.691598in}}%
\pgfpathlineto{\pgfqpoint{3.311195in}{0.682780in}}%
\pgfpathlineto{\pgfqpoint{3.272832in}{0.675560in}}%
\pgfpathlineto{\pgfqpoint{3.229573in}{0.669955in}}%
\pgfpathlineto{\pgfqpoint{3.181048in}{0.665987in}}%
\pgfpathlineto{\pgfqpoint{3.126792in}{0.663681in}}%
\pgfpathlineto{\pgfqpoint{3.066239in}{0.663070in}}%
\pgfpathlineto{\pgfqpoint{2.974880in}{0.664922in}}%
\pgfpathlineto{\pgfqpoint{2.870074in}{0.669938in}}%
\pgfpathlineto{\pgfqpoint{2.748656in}{0.678509in}}%
\pgfpathlineto{\pgfqpoint{2.609227in}{0.690947in}}%
\pgfpathlineto{\pgfqpoint{2.452453in}{0.707456in}}%
\pgfpathlineto{\pgfqpoint{2.281070in}{0.728129in}}%
\pgfpathlineto{\pgfqpoint{2.145755in}{0.746361in}}%
\pgfpathlineto{\pgfqpoint{2.007669in}{0.766884in}}%
\pgfpathlineto{\pgfqpoint{1.870508in}{0.789621in}}%
\pgfpathlineto{\pgfqpoint{1.782471in}{0.805916in}}%
\pgfpathlineto{\pgfqpoint{1.698760in}{0.822958in}}%
\pgfpathlineto{\pgfqpoint{1.620394in}{0.840601in}}%
\pgfpathlineto{\pgfqpoint{1.548163in}{0.858705in}}%
\pgfpathlineto{\pgfqpoint{1.482638in}{0.877137in}}%
\pgfpathlineto{\pgfqpoint{1.424162in}{0.895765in}}%
\pgfpathlineto{\pgfqpoint{1.372856in}{0.914467in}}%
\pgfpathlineto{\pgfqpoint{1.328618in}{0.933123in}}%
\pgfpathlineto{\pgfqpoint{1.291118in}{0.951618in}}%
\pgfpathlineto{\pgfqpoint{1.259836in}{0.969843in}}%
\pgfpathlineto{\pgfqpoint{1.234370in}{0.987671in}}%
\pgfpathlineto{\pgfqpoint{1.213747in}{1.005042in}}%
\pgfpathlineto{\pgfqpoint{1.197016in}{1.021913in}}%
\pgfpathlineto{\pgfqpoint{1.183478in}{1.038247in}}%
\pgfpathlineto{\pgfqpoint{1.172677in}{1.054012in}}%
\pgfpathlineto{\pgfqpoint{1.164407in}{1.069179in}}%
\pgfpathlineto{\pgfqpoint{1.158711in}{1.083722in}}%
\pgfpathlineto{\pgfqpoint{1.155877in}{1.097621in}}%
\pgfpathlineto{\pgfqpoint{1.156441in}{1.110859in}}%
\pgfpathlineto{\pgfqpoint{1.160316in}{1.123415in}}%
\pgfpathlineto{\pgfqpoint{1.166357in}{1.135272in}}%
\pgfpathlineto{\pgfqpoint{1.174410in}{1.146426in}}%
\pgfpathlineto{\pgfqpoint{1.184376in}{1.156874in}}%
\pgfpathlineto{\pgfqpoint{1.196195in}{1.166614in}}%
\pgfpathlineto{\pgfqpoint{1.209846in}{1.175647in}}%
\pgfpathlineto{\pgfqpoint{1.225345in}{1.183975in}}%
\pgfpathlineto{\pgfqpoint{1.252194in}{1.195146in}}%
\pgfpathlineto{\pgfqpoint{1.283638in}{1.204744in}}%
\pgfpathlineto{\pgfqpoint{1.319679in}{1.212760in}}%
\pgfpathlineto{\pgfqpoint{1.360500in}{1.219179in}}%
\pgfpathlineto{\pgfqpoint{1.406427in}{1.223978in}}%
\pgfpathlineto{\pgfqpoint{1.457866in}{1.227129in}}%
\pgfpathlineto{\pgfqpoint{1.515309in}{1.228595in}}%
\pgfpathlineto{\pgfqpoint{1.579326in}{1.228332in}}%
\pgfpathlineto{\pgfqpoint{1.676048in}{1.225205in}}%
\pgfpathlineto{\pgfqpoint{1.787235in}{1.218757in}}%
\pgfpathlineto{\pgfqpoint{1.914762in}{1.208736in}}%
\pgfpathlineto{\pgfqpoint{2.059813in}{1.194885in}}%
\pgfpathlineto{\pgfqpoint{2.221814in}{1.176979in}}%
\pgfpathlineto{\pgfqpoint{2.398456in}{1.154830in}}%
\pgfpathlineto{\pgfqpoint{2.537076in}{1.135404in}}%
\pgfpathlineto{\pgfqpoint{2.675827in}{1.113695in}}%
\pgfpathlineto{\pgfqpoint{2.810141in}{1.089925in}}%
\pgfpathlineto{\pgfqpoint{2.895106in}{1.073084in}}%
\pgfpathlineto{\pgfqpoint{2.975019in}{1.055579in}}%
\pgfpathlineto{\pgfqpoint{3.048758in}{1.037545in}}%
\pgfpathlineto{\pgfqpoint{3.115310in}{1.019132in}}%
\pgfpathlineto{\pgfqpoint{3.174449in}{1.000493in}}%
\pgfpathlineto{\pgfqpoint{3.226539in}{0.981766in}}%
\pgfpathlineto{\pgfqpoint{3.271965in}{0.963076in}}%
\pgfpathlineto{\pgfqpoint{3.311110in}{0.944537in}}%
\pgfpathlineto{\pgfqpoint{3.344363in}{0.926253in}}%
\pgfpathlineto{\pgfqpoint{3.372113in}{0.908314in}}%
\pgfpathlineto{\pgfqpoint{3.394753in}{0.890800in}}%
\pgfpathlineto{\pgfqpoint{3.412715in}{0.873780in}}%
\pgfpathlineto{\pgfqpoint{3.426633in}{0.857307in}}%
\pgfpathlineto{\pgfqpoint{3.437081in}{0.841418in}}%
\pgfpathlineto{\pgfqpoint{3.444507in}{0.826139in}}%
\pgfpathlineto{\pgfqpoint{3.449235in}{0.811497in}}%
\pgfpathlineto{\pgfqpoint{3.451466in}{0.797508in}}%
\pgfpathlineto{\pgfqpoint{3.451277in}{0.784188in}}%
\pgfpathlineto{\pgfqpoint{3.448625in}{0.771545in}}%
\pgfpathlineto{\pgfqpoint{3.443472in}{0.759585in}}%
\pgfpathlineto{\pgfqpoint{3.436159in}{0.748317in}}%
\pgfpathlineto{\pgfqpoint{3.426840in}{0.737746in}}%
\pgfpathlineto{\pgfqpoint{3.415612in}{0.727874in}}%
\pgfpathlineto{\pgfqpoint{3.402535in}{0.718704in}}%
\pgfpathlineto{\pgfqpoint{3.387632in}{0.710236in}}%
\pgfpathlineto{\pgfqpoint{3.361814in}{0.698852in}}%
\pgfpathlineto{\pgfqpoint{3.331647in}{0.689043in}}%
\pgfpathlineto{\pgfqpoint{3.296745in}{0.680801in}}%
\pgfpathlineto{\pgfqpoint{3.256963in}{0.674135in}}%
\pgfpathlineto{\pgfqpoint{3.212149in}{0.669071in}}%
\pgfpathlineto{\pgfqpoint{3.161872in}{0.665638in}}%
\pgfpathlineto{\pgfqpoint{3.105641in}{0.663872in}}%
\pgfpathlineto{\pgfqpoint{3.042907in}{0.663822in}}%
\pgfpathlineto{\pgfqpoint{2.973061in}{0.665545in}}%
\pgfpathlineto{\pgfqpoint{2.867713in}{0.670715in}}%
\pgfpathlineto{\pgfqpoint{2.746827in}{0.679343in}}%
\pgfpathlineto{\pgfqpoint{2.608919in}{0.691680in}}%
\pgfpathlineto{\pgfqpoint{2.453548in}{0.707986in}}%
\pgfpathlineto{\pgfqpoint{2.282341in}{0.728468in}}%
\pgfpathlineto{\pgfqpoint{2.146222in}{0.746633in}}%
\pgfpathlineto{\pgfqpoint{2.007370in}{0.767153in}}%
\pgfpathlineto{\pgfqpoint{1.870108in}{0.789869in}}%
\pgfpathlineto{\pgfqpoint{1.781913in}{0.806107in}}%
\pgfpathlineto{\pgfqpoint{1.698249in}{0.823120in}}%
\pgfpathlineto{\pgfqpoint{1.620220in}{0.840774in}}%
\pgfpathlineto{\pgfqpoint{1.548491in}{0.858904in}}%
\pgfpathlineto{\pgfqpoint{1.483525in}{0.877357in}}%
\pgfpathlineto{\pgfqpoint{1.425584in}{0.895989in}}%
\pgfpathlineto{\pgfqpoint{1.374725in}{0.914671in}}%
\pgfpathlineto{\pgfqpoint{1.330804in}{0.933282in}}%
\pgfpathlineto{\pgfqpoint{1.293474in}{0.951716in}}%
\pgfpathlineto{\pgfqpoint{1.262188in}{0.969875in}}%
\pgfpathlineto{\pgfqpoint{1.236192in}{0.987676in}}%
\pgfpathlineto{\pgfqpoint{1.214978in}{1.005033in}}%
\pgfpathlineto{\pgfqpoint{1.198327in}{1.021876in}}%
\pgfpathlineto{\pgfqpoint{1.185388in}{1.038170in}}%
\pgfpathlineto{\pgfqpoint{1.175492in}{1.053884in}}%
\pgfpathlineto{\pgfqpoint{1.168170in}{1.068993in}}%
\pgfpathlineto{\pgfqpoint{1.163153in}{1.083475in}}%
\pgfpathlineto{\pgfqpoint{1.160374in}{1.097312in}}%
\pgfpathlineto{\pgfqpoint{1.159963in}{1.110493in}}%
\pgfpathlineto{\pgfqpoint{1.162251in}{1.123007in}}%
\pgfpathlineto{\pgfqpoint{1.167699in}{1.134849in}}%
\pgfpathlineto{\pgfqpoint{1.175592in}{1.145998in}}%
\pgfpathlineto{\pgfqpoint{1.185478in}{1.156447in}}%
\pgfpathlineto{\pgfqpoint{1.197272in}{1.166192in}}%
\pgfpathlineto{\pgfqpoint{1.210925in}{1.175232in}}%
\pgfpathlineto{\pgfqpoint{1.226416in}{1.183566in}}%
\pgfpathlineto{\pgfqpoint{1.253138in}{1.194746in}}%
\pgfpathlineto{\pgfqpoint{1.284215in}{1.204342in}}%
\pgfpathlineto{\pgfqpoint{1.319988in}{1.212360in}}%
\pgfpathlineto{\pgfqpoint{1.360624in}{1.218794in}}%
\pgfpathlineto{\pgfqpoint{1.406334in}{1.223620in}}%
\pgfpathlineto{\pgfqpoint{1.457578in}{1.226807in}}%
\pgfpathlineto{\pgfqpoint{1.514866in}{1.228318in}}%
\pgfpathlineto{\pgfqpoint{1.578762in}{1.228102in}}%
\pgfpathlineto{\pgfqpoint{1.649880in}{1.226100in}}%
\pgfpathlineto{\pgfqpoint{1.757102in}{1.220534in}}%
\pgfpathlineto{\pgfqpoint{1.880059in}{1.211478in}}%
\pgfpathlineto{\pgfqpoint{2.020173in}{1.198678in}}%
\pgfpathlineto{\pgfqpoint{2.177728in}{1.181876in}}%
\pgfpathlineto{\pgfqpoint{2.350763in}{1.160874in}}%
\pgfpathlineto{\pgfqpoint{2.487757in}{1.142316in}}%
\pgfpathlineto{\pgfqpoint{2.626829in}{1.121417in}}%
\pgfpathlineto{\pgfqpoint{2.763532in}{1.098354in}}%
\pgfpathlineto{\pgfqpoint{2.850909in}{1.081911in}}%
\pgfpathlineto{\pgfqpoint{2.933416in}{1.064721in}}%
\pgfpathlineto{\pgfqpoint{3.010037in}{1.046925in}}%
\pgfpathlineto{\pgfqpoint{3.080192in}{1.028686in}}%
\pgfpathlineto{\pgfqpoint{3.143497in}{1.010159in}}%
\pgfpathlineto{\pgfqpoint{3.199757in}{0.991483in}}%
\pgfpathlineto{\pgfqpoint{3.248969in}{0.972790in}}%
\pgfpathlineto{\pgfqpoint{3.291325in}{0.954195in}}%
\pgfpathlineto{\pgfqpoint{3.327203in}{0.935804in}}%
\pgfpathlineto{\pgfqpoint{3.357178in}{0.917710in}}%
\pgfpathlineto{\pgfqpoint{3.382014in}{0.899995in}}%
\pgfpathlineto{\pgfqpoint{3.402145in}{0.882741in}}%
\pgfpathlineto{\pgfqpoint{3.417839in}{0.866013in}}%
\pgfpathlineto{\pgfqpoint{3.429936in}{0.849845in}}%
\pgfpathlineto{\pgfqpoint{3.439085in}{0.834264in}}%
\pgfpathlineto{\pgfqpoint{3.445736in}{0.819295in}}%
\pgfpathlineto{\pgfqpoint{3.450141in}{0.804959in}}%
\pgfpathlineto{\pgfqpoint{3.452358in}{0.791270in}}%
\pgfpathlineto{\pgfqpoint{3.452243in}{0.778242in}}%
\pgfpathlineto{\pgfqpoint{3.449457in}{0.765882in}}%
\pgfpathlineto{\pgfqpoint{3.443530in}{0.754194in}}%
\pgfpathlineto{\pgfqpoint{3.435164in}{0.743200in}}%
\pgfpathlineto{\pgfqpoint{3.424819in}{0.732907in}}%
\pgfpathlineto{\pgfqpoint{3.412573in}{0.723319in}}%
\pgfpathlineto{\pgfqpoint{3.398471in}{0.714435in}}%
\pgfpathlineto{\pgfqpoint{3.382528in}{0.706258in}}%
\pgfpathlineto{\pgfqpoint{3.355119in}{0.695315in}}%
\pgfpathlineto{\pgfqpoint{3.323338in}{0.685958in}}%
\pgfpathlineto{\pgfqpoint{3.286842in}{0.678180in}}%
\pgfpathlineto{\pgfqpoint{3.245443in}{0.671988in}}%
\pgfpathlineto{\pgfqpoint{3.198920in}{0.667409in}}%
\pgfpathlineto{\pgfqpoint{3.146792in}{0.664473in}}%
\pgfpathlineto{\pgfqpoint{3.088528in}{0.663219in}}%
\pgfpathlineto{\pgfqpoint{3.023555in}{0.663701in}}%
\pgfpathlineto{\pgfqpoint{2.925408in}{0.667148in}}%
\pgfpathlineto{\pgfqpoint{2.812624in}{0.673963in}}%
\pgfpathlineto{\pgfqpoint{2.683531in}{0.684353in}}%
\pgfpathlineto{\pgfqpoint{2.536975in}{0.698584in}}%
\pgfpathlineto{\pgfqpoint{2.373380in}{0.716902in}}%
\pgfpathlineto{\pgfqpoint{2.195736in}{0.739466in}}%
\pgfpathlineto{\pgfqpoint{2.057145in}{0.759179in}}%
\pgfpathlineto{\pgfqpoint{1.918637in}{0.781161in}}%
\pgfpathlineto{\pgfqpoint{1.784870in}{0.805184in}}%
\pgfpathlineto{\pgfqpoint{1.700779in}{0.822172in}}%
\pgfpathlineto{\pgfqpoint{1.622518in}{0.839813in}}%
\pgfpathlineto{\pgfqpoint{1.550731in}{0.857949in}}%
\pgfpathlineto{\pgfqpoint{1.485720in}{0.876421in}}%
\pgfpathlineto{\pgfqpoint{1.427643in}{0.895079in}}%
\pgfpathlineto{\pgfqpoint{1.376508in}{0.913789in}}%
\pgfpathlineto{\pgfqpoint{1.332176in}{0.932427in}}%
\pgfpathlineto{\pgfqpoint{1.294363in}{0.950885in}}%
\pgfpathlineto{\pgfqpoint{1.262636in}{0.969066in}}%
\pgfpathlineto{\pgfqpoint{1.236417in}{0.986888in}}%
\pgfpathlineto{\pgfqpoint{1.214977in}{1.004280in}}%
\pgfpathlineto{\pgfqpoint{1.197975in}{1.021169in}}%
\pgfpathlineto{\pgfqpoint{1.185090in}{1.037498in}}%
\pgfpathlineto{\pgfqpoint{1.175530in}{1.053240in}}%
\pgfpathlineto{\pgfqpoint{1.168686in}{1.068372in}}%
\pgfpathlineto{\pgfqpoint{1.164137in}{1.082873in}}%
\pgfpathlineto{\pgfqpoint{1.161645in}{1.096727in}}%
\pgfpathlineto{\pgfqpoint{1.161160in}{1.109923in}}%
\pgfpathlineto{\pgfqpoint{1.162817in}{1.122451in}}%
\pgfpathlineto{\pgfqpoint{1.166935in}{1.134309in}}%
\pgfpathlineto{\pgfqpoint{1.174020in}{1.145494in}}%
\pgfpathlineto{\pgfqpoint{1.183821in}{1.155993in}}%
\pgfpathlineto{\pgfqpoint{1.195578in}{1.165789in}}%
\pgfpathlineto{\pgfqpoint{1.209230in}{1.174880in}}%
\pgfpathlineto{\pgfqpoint{1.224745in}{1.183266in}}%
\pgfpathlineto{\pgfqpoint{1.251504in}{1.194519in}}%
\pgfpathlineto{\pgfqpoint{1.282546in}{1.204181in}}%
\pgfpathlineto{\pgfqpoint{1.318109in}{1.212253in}}%
\pgfpathlineto{\pgfqpoint{1.358569in}{1.218736in}}%
\pgfpathlineto{\pgfqpoint{1.404123in}{1.223617in}}%
\pgfpathlineto{\pgfqpoint{1.455115in}{1.226866in}}%
\pgfpathlineto{\pgfqpoint{1.512149in}{1.228441in}}%
\pgfpathlineto{\pgfqpoint{1.575837in}{1.228295in}}%
\pgfpathlineto{\pgfqpoint{1.646804in}{1.226365in}}%
\pgfpathlineto{\pgfqpoint{1.753839in}{1.220892in}}%
\pgfpathlineto{\pgfqpoint{1.876458in}{1.211918in}}%
\pgfpathlineto{\pgfqpoint{2.016148in}{1.199213in}}%
\pgfpathlineto{\pgfqpoint{2.173337in}{1.182515in}}%
\pgfpathlineto{\pgfqpoint{2.346173in}{1.161595in}}%
\pgfpathlineto{\pgfqpoint{2.483023in}{1.143107in}}%
\pgfpathlineto{\pgfqpoint{2.622222in}{1.122282in}}%
\pgfpathlineto{\pgfqpoint{2.759224in}{1.099254in}}%
\pgfpathlineto{\pgfqpoint{2.846780in}{1.082829in}}%
\pgfpathlineto{\pgfqpoint{2.929706in}{1.065687in}}%
\pgfpathlineto{\pgfqpoint{3.006933in}{1.047954in}}%
\pgfpathlineto{\pgfqpoint{3.077666in}{1.029762in}}%
\pgfpathlineto{\pgfqpoint{3.141385in}{1.011247in}}%
\pgfpathlineto{\pgfqpoint{3.197841in}{0.992549in}}%
\pgfpathlineto{\pgfqpoint{3.246941in}{0.973820in}}%
\pgfpathlineto{\pgfqpoint{3.288753in}{0.955216in}}%
\pgfpathlineto{\pgfqpoint{3.324392in}{0.936825in}}%
\pgfpathlineto{\pgfqpoint{3.354852in}{0.918720in}}%
\pgfpathlineto{\pgfqpoint{3.380870in}{0.900970in}}%
\pgfpathlineto{\pgfqpoint{3.402931in}{0.883641in}}%
\pgfpathlineto{\pgfqpoint{3.421263in}{0.866790in}}%
\pgfpathlineto{\pgfqpoint{3.435844in}{0.850473in}}%
\pgfpathlineto{\pgfqpoint{3.446395in}{0.834739in}}%
\pgfpathlineto{\pgfqpoint{3.452489in}{0.819631in}}%
\pgfpathlineto{\pgfqpoint{3.455309in}{0.805185in}}%
\pgfpathlineto{\pgfqpoint{3.455602in}{0.791414in}}%
\pgfpathlineto{\pgfqpoint{3.453604in}{0.778329in}}%
\pgfpathlineto{\pgfqpoint{3.449489in}{0.765937in}}%
\pgfpathlineto{\pgfqpoint{3.443369in}{0.754244in}}%
\pgfpathlineto{\pgfqpoint{3.435292in}{0.743252in}}%
\pgfpathlineto{\pgfqpoint{3.425244in}{0.732961in}}%
\pgfpathlineto{\pgfqpoint{3.413152in}{0.723370in}}%
\pgfpathlineto{\pgfqpoint{3.399016in}{0.714476in}}%
\pgfpathlineto{\pgfqpoint{3.382954in}{0.706282in}}%
\pgfpathlineto{\pgfqpoint{3.355284in}{0.695311in}}%
\pgfpathlineto{\pgfqpoint{3.323283in}{0.685929in}}%
\pgfpathlineto{\pgfqpoint{3.286796in}{0.678146in}}%
\pgfpathlineto{\pgfqpoint{3.245552in}{0.671972in}}%
\pgfpathlineto{\pgfqpoint{3.199159in}{0.667417in}}%
\pgfpathlineto{\pgfqpoint{3.147106in}{0.664495in}}%
\pgfpathlineto{\pgfqpoint{3.089116in}{0.663223in}}%
\pgfpathlineto{\pgfqpoint{3.024681in}{0.663665in}}%
\pgfpathlineto{\pgfqpoint{2.926835in}{0.667064in}}%
\pgfpathlineto{\pgfqpoint{2.813600in}{0.673859in}}%
\pgfpathlineto{\pgfqpoint{2.683839in}{0.684262in}}%
\pgfpathlineto{\pgfqpoint{2.537239in}{0.698489in}}%
\pgfpathlineto{\pgfqpoint{2.374315in}{0.716765in}}%
\pgfpathlineto{\pgfqpoint{2.196382in}{0.739340in}}%
\pgfpathlineto{\pgfqpoint{2.056897in}{0.759138in}}%
\pgfpathlineto{\pgfqpoint{1.918252in}{0.781172in}}%
\pgfpathlineto{\pgfqpoint{1.785209in}{0.805171in}}%
\pgfpathlineto{\pgfqpoint{1.701601in}{0.822104in}}%
\pgfpathlineto{\pgfqpoint{1.623209in}{0.839659in}}%
\pgfpathlineto{\pgfqpoint{1.550826in}{0.857718in}}%
\pgfpathlineto{\pgfqpoint{1.485087in}{0.876152in}}%
\pgfpathlineto{\pgfqpoint{1.426459in}{0.894822in}}%
\pgfpathlineto{\pgfqpoint{1.375247in}{0.913578in}}%
\pgfpathlineto{\pgfqpoint{1.331449in}{0.932263in}}%
\pgfpathlineto{\pgfqpoint{1.294238in}{0.950754in}}%
\pgfpathlineto{\pgfqpoint{1.262723in}{0.968965in}}%
\pgfpathlineto{\pgfqpoint{1.236192in}{0.986816in}}%
\pgfpathlineto{\pgfqpoint{1.214110in}{1.004238in}}%
\pgfpathlineto{\pgfqpoint{1.196119in}{1.021169in}}%
\pgfpathlineto{\pgfqpoint{1.182036in}{1.037557in}}%
\pgfpathlineto{\pgfqpoint{1.171856in}{1.053360in}}%
\pgfpathlineto{\pgfqpoint{1.165244in}{1.068539in}}%
\pgfpathlineto{\pgfqpoint{1.161486in}{1.083071in}}%
\pgfpathlineto{\pgfqpoint{1.160287in}{1.096940in}}%
\pgfpathlineto{\pgfqpoint{1.161421in}{1.110133in}}%
\pgfpathlineto{\pgfqpoint{1.164732in}{1.122642in}}%
\pgfpathlineto{\pgfqpoint{1.170131in}{1.134459in}}%
\pgfpathlineto{\pgfqpoint{1.177599in}{1.145583in}}%
\pgfpathlineto{\pgfqpoint{1.187183in}{1.156012in}}%
\pgfpathlineto{\pgfqpoint{1.198828in}{1.165747in}}%
\pgfpathlineto{\pgfqpoint{1.212421in}{1.174784in}}%
\pgfpathlineto{\pgfqpoint{1.227912in}{1.183122in}}%
\pgfpathlineto{\pgfqpoint{1.254665in}{1.194311in}}%
\pgfpathlineto{\pgfqpoint{1.285685in}{1.203914in}}%
\pgfpathlineto{\pgfqpoint{1.321148in}{1.211925in}}%
\pgfpathlineto{\pgfqpoint{1.361370in}{1.218341in}}%
\pgfpathlineto{\pgfqpoint{1.406804in}{1.223158in}}%
\pgfpathlineto{\pgfqpoint{1.457789in}{1.226364in}}%
\pgfpathlineto{\pgfqpoint{1.514607in}{1.227910in}}%
\pgfpathlineto{\pgfqpoint{1.578090in}{1.227739in}}%
\pgfpathlineto{\pgfqpoint{1.674400in}{1.224729in}}%
\pgfpathlineto{\pgfqpoint{1.785430in}{1.218372in}}%
\pgfpathlineto{\pgfqpoint{1.912418in}{1.208460in}}%
\pgfpathlineto{\pgfqpoint{2.056307in}{1.194756in}}%
\pgfpathlineto{\pgfqpoint{2.218696in}{1.176942in}}%
\pgfpathlineto{\pgfqpoint{2.398944in}{1.154671in}}%
\pgfpathlineto{\pgfqpoint{2.538440in}{1.135187in}}%
\pgfpathlineto{\pgfqpoint{2.676169in}{1.113523in}}%
\pgfpathlineto{\pgfqpoint{2.808028in}{1.089917in}}%
\pgfpathlineto{\pgfqpoint{2.890984in}{1.073240in}}%
\pgfpathlineto{\pgfqpoint{2.969014in}{1.055924in}}%
\pgfpathlineto{\pgfqpoint{3.041455in}{1.038076in}}%
\pgfpathlineto{\pgfqpoint{3.107792in}{1.019812in}}%
\pgfpathlineto{\pgfqpoint{3.167658in}{1.001259in}}%
\pgfpathlineto{\pgfqpoint{3.220835in}{0.982554in}}%
\pgfpathlineto{\pgfqpoint{3.267250in}{0.963844in}}%
\pgfpathlineto{\pgfqpoint{3.306978in}{0.945283in}}%
\pgfpathlineto{\pgfqpoint{3.340300in}{0.927027in}}%
\pgfpathlineto{\pgfqpoint{3.368045in}{0.909128in}}%
\pgfpathlineto{\pgfqpoint{3.390905in}{0.891648in}}%
\pgfpathlineto{\pgfqpoint{3.409418in}{0.874648in}}%
\pgfpathlineto{\pgfqpoint{3.424021in}{0.858181in}}%
\pgfpathlineto{\pgfqpoint{3.435044in}{0.842288in}}%
\pgfpathlineto{\pgfqpoint{3.442752in}{0.827004in}}%
\pgfpathlineto{\pgfqpoint{3.447516in}{0.812355in}}%
\pgfpathlineto{\pgfqpoint{3.449612in}{0.798359in}}%
\pgfpathlineto{\pgfqpoint{3.449251in}{0.785029in}}%
\pgfpathlineto{\pgfqpoint{3.446591in}{0.772378in}}%
\pgfpathlineto{\pgfqpoint{3.441735in}{0.760412in}}%
\pgfpathlineto{\pgfqpoint{3.434728in}{0.749135in}}%
\pgfpathlineto{\pgfqpoint{3.425617in}{0.738548in}}%
\pgfpathlineto{\pgfqpoint{3.414524in}{0.728654in}}%
\pgfpathlineto{\pgfqpoint{3.401540in}{0.719457in}}%
\pgfpathlineto{\pgfqpoint{3.386721in}{0.710960in}}%
\pgfpathlineto{\pgfqpoint{3.361084in}{0.699531in}}%
\pgfpathlineto{\pgfqpoint{3.331268in}{0.689683in}}%
\pgfpathlineto{\pgfqpoint{3.297009in}{0.681414in}}%
\pgfpathlineto{\pgfqpoint{3.257857in}{0.674716in}}%
\pgfpathlineto{\pgfqpoint{3.213375in}{0.669592in}}%
\pgfpathlineto{\pgfqpoint{3.163538in}{0.666085in}}%
\pgfpathlineto{\pgfqpoint{3.107780in}{0.664236in}}%
\pgfpathlineto{\pgfqpoint{3.045479in}{0.664093in}}%
\pgfpathlineto{\pgfqpoint{2.976008in}{0.665715in}}%
\pgfpathlineto{\pgfqpoint{2.871142in}{0.670749in}}%
\pgfpathlineto{\pgfqpoint{2.750917in}{0.679244in}}%
\pgfpathlineto{\pgfqpoint{2.613845in}{0.691432in}}%
\pgfpathlineto{\pgfqpoint{2.459273in}{0.707563in}}%
\pgfpathlineto{\pgfqpoint{2.288693in}{0.727887in}}%
\pgfpathlineto{\pgfqpoint{2.152999in}{0.745927in}}%
\pgfpathlineto{\pgfqpoint{2.014251in}{0.766310in}}%
\pgfpathlineto{\pgfqpoint{1.876856in}{0.788913in}}%
\pgfpathlineto{\pgfqpoint{1.745834in}{0.813484in}}%
\pgfpathlineto{\pgfqpoint{1.664318in}{0.830749in}}%
\pgfpathlineto{\pgfqpoint{1.588683in}{0.848567in}}%
\pgfpathlineto{\pgfqpoint{1.519658in}{0.866808in}}%
\pgfpathlineto{\pgfqpoint{1.457694in}{0.885335in}}%
\pgfpathlineto{\pgfqpoint{1.402961in}{0.904014in}}%
\pgfpathlineto{\pgfqpoint{1.355357in}{0.922706in}}%
\pgfpathlineto{\pgfqpoint{1.314907in}{0.941254in}}%
\pgfpathlineto{\pgfqpoint{1.280778in}{0.959556in}}%
\pgfpathlineto{\pgfqpoint{1.251870in}{0.977542in}}%
\pgfpathlineto{\pgfqpoint{1.227347in}{0.995151in}}%
\pgfpathlineto{\pgfqpoint{1.206632in}{1.012322in}}%
\pgfpathlineto{\pgfqpoint{1.189412in}{1.029003in}}%
\pgfpathlineto{\pgfqpoint{1.175634in}{1.045143in}}%
\pgfpathlineto{\pgfqpoint{1.165508in}{1.060700in}}%
\pgfpathlineto{\pgfqpoint{1.159504in}{1.075634in}}%
\pgfpathlineto{\pgfqpoint{1.157255in}{1.089909in}}%
\pgfpathlineto{\pgfqpoint{1.157522in}{1.103506in}}%
\pgfpathlineto{\pgfqpoint{1.160071in}{1.116416in}}%
\pgfpathlineto{\pgfqpoint{1.164727in}{1.128632in}}%
\pgfpathlineto{\pgfqpoint{1.171374in}{1.140149in}}%
\pgfpathlineto{\pgfqpoint{1.179949in}{1.150965in}}%
\pgfpathlineto{\pgfqpoint{1.190447in}{1.161078in}}%
\pgfpathlineto{\pgfqpoint{1.202921in}{1.170491in}}%
\pgfpathlineto{\pgfqpoint{1.217452in}{1.179207in}}%
\pgfpathlineto{\pgfqpoint{1.233949in}{1.187225in}}%
\pgfpathlineto{\pgfqpoint{1.262307in}{1.197936in}}%
\pgfpathlineto{\pgfqpoint{1.295051in}{1.207060in}}%
\pgfpathlineto{\pgfqpoint{1.332336in}{1.214587in}}%
\pgfpathlineto{\pgfqpoint{1.374433in}{1.220502in}}%
\pgfpathlineto{\pgfqpoint{1.421721in}{1.224790in}}%
\pgfpathlineto{\pgfqpoint{1.474690in}{1.227433in}}%
\pgfpathlineto{\pgfqpoint{1.533837in}{1.228414in}}%
\pgfpathlineto{\pgfqpoint{1.599306in}{1.227696in}}%
\pgfpathlineto{\pgfqpoint{1.698545in}{1.223921in}}%
\pgfpathlineto{\pgfqpoint{1.813602in}{1.216706in}}%
\pgfpathlineto{\pgfqpoint{1.945690in}{1.205828in}}%
\pgfpathlineto{\pgfqpoint{2.094788in}{1.191082in}}%
\pgfpathlineto{\pgfqpoint{2.259647in}{1.172280in}}%
\pgfpathlineto{\pgfqpoint{2.437787in}{1.149254in}}%
\pgfpathlineto{\pgfqpoint{2.576997in}{1.129143in}}%
\pgfpathlineto{\pgfqpoint{2.715287in}{1.106784in}}%
\pgfpathlineto{\pgfqpoint{2.847544in}{1.082480in}}%
\pgfpathlineto{\pgfqpoint{2.930278in}{1.065368in}}%
\pgfpathlineto{\pgfqpoint{3.007482in}{1.047662in}}%
\pgfpathlineto{\pgfqpoint{3.078359in}{1.029487in}}%
\pgfpathlineto{\pgfqpoint{3.142305in}{1.010976in}}%
\pgfpathlineto{\pgfqpoint{3.198905in}{0.992275in}}%
\pgfpathlineto{\pgfqpoint{3.247956in}{0.973537in}}%
\pgfpathlineto{\pgfqpoint{3.289928in}{0.954909in}}%
\pgfpathlineto{\pgfqpoint{3.325762in}{0.936490in}}%
\pgfpathlineto{\pgfqpoint{3.356242in}{0.918365in}}%
\pgfpathlineto{\pgfqpoint{3.381967in}{0.900610in}}%
\pgfpathlineto{\pgfqpoint{3.403353in}{0.883297in}}%
\pgfpathlineto{\pgfqpoint{3.420631in}{0.866483in}}%
\pgfpathlineto{\pgfqpoint{3.433850in}{0.850223in}}%
\pgfpathlineto{\pgfqpoint{3.442964in}{0.834560in}}%
\pgfpathlineto{\pgfqpoint{3.448786in}{0.819529in}}%
\pgfpathlineto{\pgfqpoint{3.451846in}{0.805151in}}%
\pgfpathlineto{\pgfqpoint{3.452434in}{0.791440in}}%
\pgfpathlineto{\pgfqpoint{3.450763in}{0.778407in}}%
\pgfpathlineto{\pgfqpoint{3.446971in}{0.766061in}}%
\pgfpathlineto{\pgfqpoint{3.441122in}{0.754408in}}%
\pgfpathlineto{\pgfqpoint{3.433202in}{0.743449in}}%
\pgfpathlineto{\pgfqpoint{3.423129in}{0.733183in}}%
\pgfpathlineto{\pgfqpoint{3.410993in}{0.723612in}}%
\pgfpathlineto{\pgfqpoint{3.396922in}{0.714740in}}%
\pgfpathlineto{\pgfqpoint{3.380955in}{0.706569in}}%
\pgfpathlineto{\pgfqpoint{3.353478in}{0.695630in}}%
\pgfpathlineto{\pgfqpoint{3.321705in}{0.686279in}}%
\pgfpathlineto{\pgfqpoint{3.285443in}{0.678521in}}%
\pgfpathlineto{\pgfqpoint{3.244365in}{0.672361in}}%
\pgfpathlineto{\pgfqpoint{3.198009in}{0.667803in}}%
\pgfpathlineto{\pgfqpoint{3.146096in}{0.664863in}}%
\pgfpathlineto{\pgfqpoint{3.088244in}{0.663592in}}%
\pgfpathlineto{\pgfqpoint{3.023599in}{0.664047in}}%
\pgfpathlineto{\pgfqpoint{2.925546in}{0.667457in}}%
\pgfpathlineto{\pgfqpoint{2.812576in}{0.674239in}}%
\pgfpathlineto{\pgfqpoint{2.683504in}{0.684604in}}%
\pgfpathlineto{\pgfqpoint{2.537466in}{0.698793in}}%
\pgfpathlineto{\pgfqpoint{2.373273in}{0.717090in}}%
\pgfpathlineto{\pgfqpoint{2.193287in}{0.739721in}}%
\pgfpathlineto{\pgfqpoint{2.054606in}{0.759460in}}%
\pgfpathlineto{\pgfqpoint{1.917666in}{0.781406in}}%
\pgfpathlineto{\pgfqpoint{1.786342in}{0.805333in}}%
\pgfpathlineto{\pgfqpoint{1.703599in}{0.822239in}}%
\pgfpathlineto{\pgfqpoint{1.625738in}{0.839785in}}%
\pgfpathlineto{\pgfqpoint{1.553535in}{0.857851in}}%
\pgfpathlineto{\pgfqpoint{1.487670in}{0.876299in}}%
\pgfpathlineto{\pgfqpoint{1.428722in}{0.894979in}}%
\pgfpathlineto{\pgfqpoint{1.377171in}{0.913726in}}%
\pgfpathlineto{\pgfqpoint{1.333100in}{0.932378in}}%
\pgfpathlineto{\pgfqpoint{1.295491in}{0.950849in}}%
\pgfpathlineto{\pgfqpoint{1.263711in}{0.969042in}}%
\pgfpathlineto{\pgfqpoint{1.237218in}{0.986870in}}%
\pgfpathlineto{\pgfqpoint{1.215524in}{1.004260in}}%
\pgfpathlineto{\pgfqpoint{1.198182in}{1.021150in}}%
\pgfpathlineto{\pgfqpoint{1.184712in}{1.037491in}}%
\pgfpathlineto{\pgfqpoint{1.174690in}{1.053246in}}%
\pgfpathlineto{\pgfqpoint{1.167772in}{1.068384in}}%
\pgfpathlineto{\pgfqpoint{1.163690in}{1.082882in}}%
\pgfpathlineto{\pgfqpoint{1.162246in}{1.096721in}}%
\pgfpathlineto{\pgfqpoint{1.163350in}{1.109890in}}%
\pgfpathlineto{\pgfqpoint{1.166805in}{1.122379in}}%
\pgfpathlineto{\pgfqpoint{1.172388in}{1.134181in}}%
\pgfpathlineto{\pgfqpoint{1.179932in}{1.145289in}}%
\pgfpathlineto{\pgfqpoint{1.189326in}{1.155701in}}%
\pgfpathlineto{\pgfqpoint{1.200517in}{1.165414in}}%
\pgfpathlineto{\pgfqpoint{1.213507in}{1.174429in}}%
\pgfpathlineto{\pgfqpoint{1.228354in}{1.182747in}}%
\pgfpathlineto{\pgfqpoint{1.245174in}{1.190372in}}%
\pgfpathlineto{\pgfqpoint{1.274469in}{1.200523in}}%
\pgfpathlineto{\pgfqpoint{1.308500in}{1.209113in}}%
\pgfpathlineto{\pgfqpoint{1.347228in}{1.216119in}}%
\pgfpathlineto{\pgfqpoint{1.390936in}{1.221521in}}%
\pgfpathlineto{\pgfqpoint{1.439993in}{1.225293in}}%
\pgfpathlineto{\pgfqpoint{1.494853in}{1.227402in}}%
\pgfpathlineto{\pgfqpoint{1.556053in}{1.227809in}}%
\pgfpathlineto{\pgfqpoint{1.624216in}{1.226467in}}%
\pgfpathlineto{\pgfqpoint{1.727029in}{1.221859in}}%
\pgfpathlineto{\pgfqpoint{1.845160in}{1.213823in}}%
\pgfpathlineto{\pgfqpoint{1.980440in}{1.202095in}}%
\pgfpathlineto{\pgfqpoint{2.133098in}{1.186440in}}%
\pgfpathlineto{\pgfqpoint{2.301671in}{1.166657in}}%
\pgfpathlineto{\pgfqpoint{2.482954in}{1.142581in}}%
\pgfpathlineto{\pgfqpoint{2.622164in}{1.121762in}}%
\pgfpathlineto{\pgfqpoint{2.758491in}{1.098805in}}%
\pgfpathlineto{\pgfqpoint{2.887505in}{1.073995in}}%
\pgfpathlineto{\pgfqpoint{2.967573in}{1.056598in}}%
\pgfpathlineto{\pgfqpoint{3.041769in}{1.038658in}}%
\pgfpathlineto{\pgfqpoint{3.109285in}{1.020311in}}%
\pgfpathlineto{\pgfqpoint{3.169447in}{1.001708in}}%
\pgfpathlineto{\pgfqpoint{3.222016in}{0.983011in}}%
\pgfpathlineto{\pgfqpoint{3.267627in}{0.964347in}}%
\pgfpathlineto{\pgfqpoint{3.306961in}{0.945828in}}%
\pgfpathlineto{\pgfqpoint{3.340596in}{0.927552in}}%
\pgfpathlineto{\pgfqpoint{3.369003in}{0.909607in}}%
\pgfpathlineto{\pgfqpoint{3.392548in}{0.892072in}}%
\pgfpathlineto{\pgfqpoint{3.411489in}{0.875015in}}%
\pgfpathlineto{\pgfqpoint{3.426001in}{0.858494in}}%
\pgfpathlineto{\pgfqpoint{3.436705in}{0.842556in}}%
\pgfpathlineto{\pgfqpoint{3.444202in}{0.827232in}}%
\pgfpathlineto{\pgfqpoint{3.448873in}{0.812545in}}%
\pgfpathlineto{\pgfqpoint{3.451004in}{0.798511in}}%
\pgfpathlineto{\pgfqpoint{3.450788in}{0.785147in}}%
\pgfpathlineto{\pgfqpoint{3.448325in}{0.772463in}}%
\pgfpathlineto{\pgfqpoint{3.443620in}{0.760463in}}%
\pgfpathlineto{\pgfqpoint{3.436609in}{0.749152in}}%
\pgfpathlineto{\pgfqpoint{3.427472in}{0.738533in}}%
\pgfpathlineto{\pgfqpoint{3.416355in}{0.728610in}}%
\pgfpathlineto{\pgfqpoint{3.403339in}{0.719387in}}%
\pgfpathlineto{\pgfqpoint{3.388468in}{0.710866in}}%
\pgfpathlineto{\pgfqpoint{3.362706in}{0.699401in}}%
\pgfpathlineto{\pgfqpoint{3.332708in}{0.689520in}}%
\pgfpathlineto{\pgfqpoint{3.298222in}{0.681219in}}%
\pgfpathlineto{\pgfqpoint{3.258835in}{0.674495in}}%
\pgfpathlineto{\pgfqpoint{3.214297in}{0.669358in}}%
\pgfpathlineto{\pgfqpoint{3.164400in}{0.665843in}}%
\pgfpathlineto{\pgfqpoint{3.108564in}{0.663989in}}%
\pgfpathlineto{\pgfqpoint{3.046191in}{0.663844in}}%
\pgfpathlineto{\pgfqpoint{2.976666in}{0.665467in}}%
\pgfpathlineto{\pgfqpoint{2.871752in}{0.670503in}}%
\pgfpathlineto{\pgfqpoint{2.751465in}{0.679003in}}%
\pgfpathlineto{\pgfqpoint{2.614284in}{0.691196in}}%
\pgfpathlineto{\pgfqpoint{2.459600in}{0.707343in}}%
\pgfpathlineto{\pgfqpoint{2.288885in}{0.727677in}}%
\pgfpathlineto{\pgfqpoint{2.153092in}{0.745720in}}%
\pgfpathlineto{\pgfqpoint{2.014227in}{0.766119in}}%
\pgfpathlineto{\pgfqpoint{1.876659in}{0.788749in}}%
\pgfpathlineto{\pgfqpoint{1.745627in}{0.813310in}}%
\pgfpathlineto{\pgfqpoint{1.664105in}{0.830573in}}%
\pgfpathlineto{\pgfqpoint{1.588444in}{0.848403in}}%
\pgfpathlineto{\pgfqpoint{1.519402in}{0.866668in}}%
\pgfpathlineto{\pgfqpoint{1.457500in}{0.885223in}}%
\pgfpathlineto{\pgfqpoint{1.403111in}{0.903909in}}%
\pgfpathlineto{\pgfqpoint{1.355908in}{0.922587in}}%
\pgfpathlineto{\pgfqpoint{1.315107in}{0.941148in}}%
\pgfpathlineto{\pgfqpoint{1.280052in}{0.959493in}}%
\pgfpathlineto{\pgfqpoint{1.250223in}{0.977531in}}%
\pgfpathlineto{\pgfqpoint{1.225233in}{0.995183in}}%
\pgfpathlineto{\pgfqpoint{1.204829in}{1.012378in}}%
\pgfpathlineto{\pgfqpoint{1.188892in}{1.029054in}}%
\pgfpathlineto{\pgfqpoint{1.177271in}{1.045160in}}%
\pgfpathlineto{\pgfqpoint{1.169101in}{1.060656in}}%
\pgfpathlineto{\pgfqpoint{1.163879in}{1.075517in}}%
\pgfpathlineto{\pgfqpoint{1.161263in}{1.089726in}}%
\pgfpathlineto{\pgfqpoint{1.161002in}{1.103268in}}%
\pgfpathlineto{\pgfqpoint{1.162940in}{1.116131in}}%
\pgfpathlineto{\pgfqpoint{1.167015in}{1.128309in}}%
\pgfpathlineto{\pgfqpoint{1.173257in}{1.139798in}}%
\pgfpathlineto{\pgfqpoint{1.181776in}{1.150597in}}%
\pgfpathlineto{\pgfqpoint{1.192408in}{1.160702in}}%
\pgfpathlineto{\pgfqpoint{1.204992in}{1.170109in}}%
\pgfpathlineto{\pgfqpoint{1.219470in}{1.178816in}}%
\pgfpathlineto{\pgfqpoint{1.244682in}{1.190559in}}%
\pgfpathlineto{\pgfqpoint{1.274119in}{1.200717in}}%
\pgfpathlineto{\pgfqpoint{1.307945in}{1.209287in}}%
\pgfpathlineto{\pgfqpoint{1.346472in}{1.216269in}}%
\pgfpathlineto{\pgfqpoint{1.390148in}{1.221664in}}%
\pgfpathlineto{\pgfqpoint{1.439134in}{1.225449in}}%
\pgfpathlineto{\pgfqpoint{1.493882in}{1.227583in}}%
\pgfpathlineto{\pgfqpoint{1.555082in}{1.228018in}}%
\pgfpathlineto{\pgfqpoint{1.623397in}{1.226696in}}%
\pgfpathlineto{\pgfqpoint{1.726649in}{1.222079in}}%
\pgfpathlineto{\pgfqpoint{1.845112in}{1.214020in}}%
\pgfpathlineto{\pgfqpoint{1.980132in}{1.202293in}}%
\pgfpathlineto{\pgfqpoint{2.132934in}{1.186660in}}%
\pgfpathlineto{\pgfqpoint{2.302263in}{1.166877in}}%
\pgfpathlineto{\pgfqpoint{2.437102in}{1.149207in}}%
\pgfpathlineto{\pgfqpoint{2.575324in}{1.129151in}}%
\pgfpathlineto{\pgfqpoint{2.713111in}{1.106857in}}%
\pgfpathlineto{\pgfqpoint{2.845897in}{1.082576in}}%
\pgfpathlineto{\pgfqpoint{2.929050in}{1.065456in}}%
\pgfpathlineto{\pgfqpoint{3.005981in}{1.047728in}}%
\pgfpathlineto{\pgfqpoint{3.076045in}{1.029531in}}%
\pgfpathlineto{\pgfqpoint{3.139220in}{1.011026in}}%
\pgfpathlineto{\pgfqpoint{3.195561in}{0.992360in}}%
\pgfpathlineto{\pgfqpoint{3.245197in}{0.973667in}}%
\pgfpathlineto{\pgfqpoint{3.288333in}{0.955066in}}%
\pgfpathlineto{\pgfqpoint{3.325248in}{0.936666in}}%
\pgfpathlineto{\pgfqpoint{3.356296in}{0.918560in}}%
\pgfpathlineto{\pgfqpoint{3.381906in}{0.900829in}}%
\pgfpathlineto{\pgfqpoint{3.402581in}{0.883541in}}%
\pgfpathlineto{\pgfqpoint{3.418845in}{0.866753in}}%
\pgfpathlineto{\pgfqpoint{3.430880in}{0.850536in}}%
\pgfpathlineto{\pgfqpoint{3.439401in}{0.834921in}}%
\pgfpathlineto{\pgfqpoint{3.445113in}{0.819930in}}%
\pgfpathlineto{\pgfqpoint{3.448546in}{0.805578in}}%
\pgfpathlineto{\pgfqpoint{3.450050in}{0.791880in}}%
\pgfpathlineto{\pgfqpoint{3.449797in}{0.778845in}}%
\pgfpathlineto{\pgfqpoint{3.447782in}{0.766479in}}%
\pgfpathlineto{\pgfqpoint{3.443821in}{0.754787in}}%
\pgfpathlineto{\pgfqpoint{3.437552in}{0.743768in}}%
\pgfpathlineto{\pgfqpoint{3.428438in}{0.733419in}}%
\pgfpathlineto{\pgfqpoint{3.416296in}{0.723746in}}%
\pgfpathlineto{\pgfqpoint{3.402164in}{0.714776in}}%
\pgfpathlineto{\pgfqpoint{3.386143in}{0.706512in}}%
\pgfpathlineto{\pgfqpoint{3.358582in}{0.695444in}}%
\pgfpathlineto{\pgfqpoint{3.326712in}{0.685973in}}%
\pgfpathlineto{\pgfqpoint{3.290345in}{0.678102in}}%
\pgfpathlineto{\pgfqpoint{3.249171in}{0.671836in}}%
\pgfpathlineto{\pgfqpoint{3.202767in}{0.667182in}}%
\pgfpathlineto{\pgfqpoint{3.150947in}{0.664157in}}%
\pgfpathlineto{\pgfqpoint{3.093143in}{0.662808in}}%
\pgfpathlineto{\pgfqpoint{3.028549in}{0.663190in}}%
\pgfpathlineto{\pgfqpoint{2.930621in}{0.666505in}}%
\pgfpathlineto{\pgfqpoint{2.817849in}{0.673195in}}%
\pgfpathlineto{\pgfqpoint{2.689000in}{0.683470in}}%
\pgfpathlineto{\pgfqpoint{2.543098in}{0.697573in}}%
\pgfpathlineto{\pgfqpoint{2.377955in}{0.715851in}}%
\pgfpathlineto{\pgfqpoint{2.242501in}{0.732483in}}%
\pgfpathlineto{\pgfqpoint{2.103074in}{0.751511in}}%
\pgfpathlineto{\pgfqpoint{1.964545in}{0.772768in}}%
\pgfpathlineto{\pgfqpoint{1.831076in}{0.796034in}}%
\pgfpathlineto{\pgfqpoint{1.746645in}{0.812528in}}%
\pgfpathlineto{\pgfqpoint{1.666864in}{0.829701in}}%
\pgfpathlineto{\pgfqpoint{1.592444in}{0.847446in}}%
\pgfpathlineto{\pgfqpoint{1.523954in}{0.865650in}}%
\pgfpathlineto{\pgfqpoint{1.461827in}{0.884185in}}%
\pgfpathlineto{\pgfqpoint{1.406354in}{0.902917in}}%
\pgfpathlineto{\pgfqpoint{1.357686in}{0.921696in}}%
\pgfpathlineto{\pgfqpoint{1.315837in}{0.940366in}}%
\pgfpathlineto{\pgfqpoint{1.280659in}{0.958762in}}%
\pgfpathlineto{\pgfqpoint{1.251328in}{0.976812in}}%
\pgfpathlineto{\pgfqpoint{1.227070in}{0.994461in}}%
\pgfpathlineto{\pgfqpoint{1.207348in}{1.011644in}}%
\pgfpathlineto{\pgfqpoint{1.191707in}{1.028306in}}%
\pgfpathlineto{\pgfqpoint{1.179769in}{1.044401in}}%
\pgfpathlineto{\pgfqpoint{1.171160in}{1.059894in}}%
\pgfpathlineto{\pgfqpoint{1.165545in}{1.074759in}}%
\pgfpathlineto{\pgfqpoint{1.162661in}{1.088975in}}%
\pgfpathlineto{\pgfqpoint{1.162301in}{1.102526in}}%
\pgfpathlineto{\pgfqpoint{1.164318in}{1.115400in}}%
\pgfpathlineto{\pgfqpoint{1.168676in}{1.127593in}}%
\pgfpathlineto{\pgfqpoint{1.175209in}{1.139098in}}%
\pgfpathlineto{\pgfqpoint{1.183744in}{1.149910in}}%
\pgfpathlineto{\pgfqpoint{1.194155in}{1.160024in}}%
\pgfpathlineto{\pgfqpoint{1.206363in}{1.169438in}}%
\pgfpathlineto{\pgfqpoint{1.220339in}{1.178150in}}%
\pgfpathlineto{\pgfqpoint{1.244670in}{1.189906in}}%
\pgfpathlineto{\pgfqpoint{1.273296in}{1.200094in}}%
\pgfpathlineto{\pgfqpoint{1.306723in}{1.208728in}}%
\pgfpathlineto{\pgfqpoint{1.345210in}{1.215805in}}%
\pgfpathlineto{\pgfqpoint{1.388644in}{1.221285in}}%
\pgfpathlineto{\pgfqpoint{1.437426in}{1.225145in}}%
\pgfpathlineto{\pgfqpoint{1.492020in}{1.227347in}}%
\pgfpathlineto{\pgfqpoint{1.552957in}{1.227848in}}%
\pgfpathlineto{\pgfqpoint{1.620831in}{1.226594in}}%
\pgfpathlineto{\pgfqpoint{1.723267in}{1.222082in}}%
\pgfpathlineto{\pgfqpoint{1.840908in}{1.214158in}}%
\pgfpathlineto{\pgfqpoint{1.975337in}{1.202575in}}%
\pgfpathlineto{\pgfqpoint{2.127263in}{1.187073in}}%
\pgfpathlineto{\pgfqpoint{2.295517in}{1.167433in}}%
\pgfpathlineto{\pgfqpoint{2.430234in}{1.149905in}}%
\pgfpathlineto{\pgfqpoint{2.568684in}{1.130000in}}%
\pgfpathlineto{\pgfqpoint{2.706746in}{1.107847in}}%
\pgfpathlineto{\pgfqpoint{2.839650in}{1.083685in}}%
\pgfpathlineto{\pgfqpoint{2.922749in}{1.066607in}}%
\pgfpathlineto{\pgfqpoint{3.000088in}{1.048901in}}%
\pgfpathlineto{\pgfqpoint{3.071057in}{1.030734in}}%
\pgfpathlineto{\pgfqpoint{3.135239in}{1.012259in}}%
\pgfpathlineto{\pgfqpoint{3.192408in}{0.993619in}}%
\pgfpathlineto{\pgfqpoint{3.242531in}{0.974945in}}%
\pgfpathlineto{\pgfqpoint{3.285767in}{0.956354in}}%
\pgfpathlineto{\pgfqpoint{3.322466in}{0.937953in}}%
\pgfpathlineto{\pgfqpoint{3.353170in}{0.919837in}}%
\pgfpathlineto{\pgfqpoint{3.378615in}{0.902087in}}%
\pgfpathlineto{\pgfqpoint{3.399409in}{0.884784in}}%
\pgfpathlineto{\pgfqpoint{3.415664in}{0.868000in}}%
\pgfpathlineto{\pgfqpoint{3.428195in}{0.851772in}}%
\pgfpathlineto{\pgfqpoint{3.437678in}{0.836127in}}%
\pgfpathlineto{\pgfqpoint{3.444595in}{0.821091in}}%
\pgfpathlineto{\pgfqpoint{3.449236in}{0.806685in}}%
\pgfpathlineto{\pgfqpoint{3.451693in}{0.792925in}}%
\pgfpathlineto{\pgfqpoint{3.451869in}{0.779824in}}%
\pgfpathlineto{\pgfqpoint{3.449469in}{0.767389in}}%
\pgfpathlineto{\pgfqpoint{3.444012in}{0.755626in}}%
\pgfpathlineto{\pgfqpoint{3.435925in}{0.744551in}}%
\pgfpathlineto{\pgfqpoint{3.425841in}{0.734177in}}%
\pgfpathlineto{\pgfqpoint{3.413843in}{0.724506in}}%
\pgfpathlineto{\pgfqpoint{3.399982in}{0.715539in}}%
\pgfpathlineto{\pgfqpoint{3.384278in}{0.707278in}}%
\pgfpathlineto{\pgfqpoint{3.357234in}{0.696209in}}%
\pgfpathlineto{\pgfqpoint{3.325844in}{0.686725in}}%
\pgfpathlineto{\pgfqpoint{3.289782in}{0.678821in}}%
\pgfpathlineto{\pgfqpoint{3.248800in}{0.672501in}}%
\pgfpathlineto{\pgfqpoint{3.202737in}{0.667789in}}%
\pgfpathlineto{\pgfqpoint{3.151110in}{0.664717in}}%
\pgfpathlineto{\pgfqpoint{3.093391in}{0.663324in}}%
\pgfpathlineto{\pgfqpoint{3.029008in}{0.663660in}}%
\pgfpathlineto{\pgfqpoint{2.931727in}{0.666904in}}%
\pgfpathlineto{\pgfqpoint{2.819916in}{0.673503in}}%
\pgfpathlineto{\pgfqpoint{2.691901in}{0.683664in}}%
\pgfpathlineto{\pgfqpoint{2.546475in}{0.697647in}}%
\pgfpathlineto{\pgfqpoint{2.383902in}{0.715703in}}%
\pgfpathlineto{\pgfqpoint{2.207033in}{0.737997in}}%
\pgfpathlineto{\pgfqpoint{2.068640in}{0.757513in}}%
\pgfpathlineto{\pgfqpoint{1.929978in}{0.779311in}}%
\pgfpathlineto{\pgfqpoint{1.795679in}{0.803172in}}%
\pgfpathlineto{\pgfqpoint{1.710956in}{0.820068in}}%
\pgfpathlineto{\pgfqpoint{1.631888in}{0.837626in}}%
\pgfpathlineto{\pgfqpoint{1.559369in}{0.855705in}}%
\pgfpathlineto{\pgfqpoint{1.493629in}{0.874141in}}%
\pgfpathlineto{\pgfqpoint{1.434775in}{0.892783in}}%
\pgfpathlineto{\pgfqpoint{1.382793in}{0.911493in}}%
\pgfpathlineto{\pgfqpoint{1.337550in}{0.930146in}}%
\pgfpathlineto{\pgfqpoint{1.298792in}{0.948631in}}%
\pgfpathlineto{\pgfqpoint{1.266144in}{0.966850in}}%
\pgfpathlineto{\pgfqpoint{1.239113in}{0.984719in}}%
\pgfpathlineto{\pgfqpoint{1.217083in}{1.002165in}}%
\pgfpathlineto{\pgfqpoint{1.199432in}{1.019128in}}%
\pgfpathlineto{\pgfqpoint{1.186058in}{1.035536in}}%
\pgfpathlineto{\pgfqpoint{1.176250in}{1.051355in}}%
\pgfpathlineto{\pgfqpoint{1.169327in}{1.066563in}}%
\pgfpathlineto{\pgfqpoint{1.164785in}{1.081139in}}%
\pgfpathlineto{\pgfqpoint{1.162299in}{1.095069in}}%
\pgfpathlineto{\pgfqpoint{1.161726in}{1.108341in}}%
\pgfpathlineto{\pgfqpoint{1.163101in}{1.120945in}}%
\pgfpathlineto{\pgfqpoint{1.166638in}{1.132877in}}%
\pgfpathlineto{\pgfqpoint{1.172732in}{1.144136in}}%
\pgfpathlineto{\pgfqpoint{1.181890in}{1.154723in}}%
\pgfpathlineto{\pgfqpoint{1.193409in}{1.164616in}}%
\pgfpathlineto{\pgfqpoint{1.206847in}{1.173805in}}%
\pgfpathlineto{\pgfqpoint{1.222165in}{1.182288in}}%
\pgfpathlineto{\pgfqpoint{1.248643in}{1.193688in}}%
\pgfpathlineto{\pgfqpoint{1.279389in}{1.203495in}}%
\pgfpathlineto{\pgfqpoint{1.314600in}{1.211707in}}%
\pgfpathlineto{\pgfqpoint{1.354600in}{1.218323in}}%
\pgfpathlineto{\pgfqpoint{1.399763in}{1.223341in}}%
\pgfpathlineto{\pgfqpoint{1.450254in}{1.226731in}}%
\pgfpathlineto{\pgfqpoint{1.506701in}{1.228454in}}%
\pgfpathlineto{\pgfqpoint{1.569773in}{1.228459in}}%
\pgfpathlineto{\pgfqpoint{1.640123in}{1.226687in}}%
\pgfpathlineto{\pgfqpoint{1.746327in}{1.221432in}}%
\pgfpathlineto{\pgfqpoint{1.868019in}{1.212690in}}%
\pgfpathlineto{\pgfqpoint{2.006566in}{1.200225in}}%
\pgfpathlineto{\pgfqpoint{2.162948in}{1.183805in}}%
\pgfpathlineto{\pgfqpoint{2.335084in}{1.163170in}}%
\pgfpathlineto{\pgfqpoint{2.471381in}{1.144860in}}%
\pgfpathlineto{\pgfqpoint{2.610277in}{1.124193in}}%
\pgfpathlineto{\pgfqpoint{2.747631in}{1.101343in}}%
\pgfpathlineto{\pgfqpoint{2.835908in}{1.085028in}}%
\pgfpathlineto{\pgfqpoint{2.919603in}{1.067964in}}%
\pgfpathlineto{\pgfqpoint{2.997318in}{1.050262in}}%
\pgfpathlineto{\pgfqpoint{3.068517in}{1.032080in}}%
\pgfpathlineto{\pgfqpoint{3.132900in}{1.013578in}}%
\pgfpathlineto{\pgfqpoint{3.190318in}{0.994901in}}%
\pgfpathlineto{\pgfqpoint{3.240777in}{0.976185in}}%
\pgfpathlineto{\pgfqpoint{3.284434in}{0.957549in}}%
\pgfpathlineto{\pgfqpoint{3.321599in}{0.939103in}}%
\pgfpathlineto{\pgfqpoint{3.352735in}{0.920942in}}%
\pgfpathlineto{\pgfqpoint{3.378456in}{0.903148in}}%
\pgfpathlineto{\pgfqpoint{3.399485in}{0.885793in}}%
\pgfpathlineto{\pgfqpoint{3.415937in}{0.868956in}}%
\pgfpathlineto{\pgfqpoint{3.428449in}{0.852679in}}%
\pgfpathlineto{\pgfqpoint{3.437787in}{0.836991in}}%
\pgfpathlineto{\pgfqpoint{3.444521in}{0.821913in}}%
\pgfpathlineto{\pgfqpoint{3.449030in}{0.807467in}}%
\pgfpathlineto{\pgfqpoint{3.451495in}{0.793667in}}%
\pgfpathlineto{\pgfqpoint{3.451909in}{0.780525in}}%
\pgfpathlineto{\pgfqpoint{3.450067in}{0.768050in}}%
\pgfpathlineto{\pgfqpoint{3.445571in}{0.756245in}}%
\pgfpathlineto{\pgfqpoint{3.437921in}{0.745114in}}%
\pgfpathlineto{\pgfqpoint{3.427904in}{0.734678in}}%
\pgfpathlineto{\pgfqpoint{3.415947in}{0.724945in}}%
\pgfpathlineto{\pgfqpoint{3.402109in}{0.715919in}}%
\pgfpathlineto{\pgfqpoint{3.386416in}{0.707598in}}%
\pgfpathlineto{\pgfqpoint{3.359394in}{0.696444in}}%
\pgfpathlineto{\pgfqpoint{3.328073in}{0.686881in}}%
\pgfpathlineto{\pgfqpoint{3.292192in}{0.678908in}}%
\pgfpathlineto{\pgfqpoint{3.251373in}{0.672521in}}%
\pgfpathlineto{\pgfqpoint{3.205485in}{0.667740in}}%
\pgfpathlineto{\pgfqpoint{3.154093in}{0.664596in}}%
\pgfpathlineto{\pgfqpoint{3.096619in}{0.663128in}}%
\pgfpathlineto{\pgfqpoint{3.032465in}{0.663388in}}%
\pgfpathlineto{\pgfqpoint{2.935463in}{0.666527in}}%
\pgfpathlineto{\pgfqpoint{2.823962in}{0.673023in}}%
\pgfpathlineto{\pgfqpoint{2.696380in}{0.683083in}}%
\pgfpathlineto{\pgfqpoint{2.551419in}{0.696951in}}%
\pgfpathlineto{\pgfqpoint{2.389190in}{0.714897in}}%
\pgfpathlineto{\pgfqpoint{2.212537in}{0.737085in}}%
\pgfpathlineto{\pgfqpoint{2.074246in}{0.756513in}}%
\pgfpathlineto{\pgfqpoint{1.935203in}{0.778251in}}%
\pgfpathlineto{\pgfqpoint{1.800575in}{0.802061in}}%
\pgfpathlineto{\pgfqpoint{1.715676in}{0.818919in}}%
\pgfpathlineto{\pgfqpoint{1.636025in}{0.836429in}}%
\pgfpathlineto{\pgfqpoint{1.562552in}{0.854464in}}%
\pgfpathlineto{\pgfqpoint{1.495973in}{0.872881in}}%
\pgfpathlineto{\pgfqpoint{1.436847in}{0.891525in}}%
\pgfpathlineto{\pgfqpoint{1.385145in}{0.910244in}}%
\pgfpathlineto{\pgfqpoint{1.340169in}{0.928920in}}%
\pgfpathlineto{\pgfqpoint{1.301300in}{0.947445in}}%
\pgfpathlineto{\pgfqpoint{1.268023in}{0.965719in}}%
\pgfpathlineto{\pgfqpoint{1.239928in}{0.983654in}}%
\pgfpathlineto{\pgfqpoint{1.216708in}{1.001174in}}%
\pgfpathlineto{\pgfqpoint{1.198162in}{1.018208in}}%
\pgfpathlineto{\pgfqpoint{1.184141in}{1.034700in}}%
\pgfpathlineto{\pgfqpoint{1.173900in}{1.050600in}}%
\pgfpathlineto{\pgfqpoint{1.166822in}{1.065882in}}%
\pgfpathlineto{\pgfqpoint{1.162509in}{1.080523in}}%
\pgfpathlineto{\pgfqpoint{1.160663in}{1.094507in}}%
\pgfpathlineto{\pgfqpoint{1.161092in}{1.107819in}}%
\pgfpathlineto{\pgfqpoint{1.163705in}{1.120450in}}%
\pgfpathlineto{\pgfqpoint{1.168514in}{1.132394in}}%
\pgfpathlineto{\pgfqpoint{1.175634in}{1.143651in}}%
\pgfpathlineto{\pgfqpoint{1.184993in}{1.154215in}}%
\pgfpathlineto{\pgfqpoint{1.196336in}{1.164082in}}%
\pgfpathlineto{\pgfqpoint{1.209591in}{1.173249in}}%
\pgfpathlineto{\pgfqpoint{1.224711in}{1.181713in}}%
\pgfpathlineto{\pgfqpoint{1.250867in}{1.193090in}}%
\pgfpathlineto{\pgfqpoint{1.281271in}{1.202882in}}%
\pgfpathlineto{\pgfqpoint{1.316149in}{1.211087in}}%
\pgfpathlineto{\pgfqpoint{1.355880in}{1.217708in}}%
\pgfpathlineto{\pgfqpoint{1.400809in}{1.222741in}}%
\pgfpathlineto{\pgfqpoint{1.451093in}{1.226149in}}%
\pgfpathlineto{\pgfqpoint{1.507342in}{1.227894in}}%
\pgfpathlineto{\pgfqpoint{1.570187in}{1.227927in}}%
\pgfpathlineto{\pgfqpoint{1.640258in}{1.226186in}}%
\pgfpathlineto{\pgfqpoint{1.746010in}{1.220983in}}%
\pgfpathlineto{\pgfqpoint{1.867206in}{1.212301in}}%
\pgfpathlineto{\pgfqpoint{2.005310in}{1.199909in}}%
\pgfpathlineto{\pgfqpoint{2.160960in}{1.183561in}}%
\pgfpathlineto{\pgfqpoint{2.332453in}{1.163003in}}%
\pgfpathlineto{\pgfqpoint{2.468611in}{1.144778in}}%
\pgfpathlineto{\pgfqpoint{2.607571in}{1.124214in}}%
\pgfpathlineto{\pgfqpoint{2.744847in}{1.101456in}}%
\pgfpathlineto{\pgfqpoint{2.832891in}{1.085168in}}%
\pgfpathlineto{\pgfqpoint{2.916530in}{1.068132in}}%
\pgfpathlineto{\pgfqpoint{2.994653in}{1.050488in}}%
\pgfpathlineto{\pgfqpoint{3.066425in}{1.032377in}}%
\pgfpathlineto{\pgfqpoint{3.131286in}{1.013935in}}%
\pgfpathlineto{\pgfqpoint{3.188953in}{0.995294in}}%
\pgfpathlineto{\pgfqpoint{3.239423in}{0.976586in}}%
\pgfpathlineto{\pgfqpoint{3.282964in}{0.957937in}}%
\pgfpathlineto{\pgfqpoint{3.319875in}{0.939483in}}%
\pgfpathlineto{\pgfqpoint{3.350563in}{0.921342in}}%
\pgfpathlineto{\pgfqpoint{3.376157in}{0.903575in}}%
\pgfpathlineto{\pgfqpoint{3.397558in}{0.886235in}}%
\pgfpathlineto{\pgfqpoint{3.415408in}{0.869374in}}%
\pgfpathlineto{\pgfqpoint{3.430084in}{0.853036in}}%
\pgfpathlineto{\pgfqpoint{3.441703in}{0.837262in}}%
\pgfpathlineto{\pgfqpoint{3.450119in}{0.822087in}}%
\pgfpathlineto{\pgfqpoint{3.454925in}{0.807544in}}%
\pgfpathlineto{\pgfqpoint{3.455850in}{0.793659in}}%
\pgfpathlineto{\pgfqpoint{3.454186in}{0.780459in}}%
\pgfpathlineto{\pgfqpoint{3.450298in}{0.767951in}}%
\pgfpathlineto{\pgfqpoint{3.444348in}{0.756141in}}%
\pgfpathlineto{\pgfqpoint{3.436445in}{0.745032in}}%
\pgfpathlineto{\pgfqpoint{3.426654in}{0.734627in}}%
\pgfpathlineto{\pgfqpoint{3.414987in}{0.724925in}}%
\pgfpathlineto{\pgfqpoint{3.401410in}{0.715926in}}%
\pgfpathlineto{\pgfqpoint{3.385837in}{0.707626in}}%
\pgfpathlineto{\pgfqpoint{3.358691in}{0.696483in}}%
\pgfpathlineto{\pgfqpoint{3.327138in}{0.686919in}}%
\pgfpathlineto{\pgfqpoint{3.291056in}{0.678944in}}%
\pgfpathlineto{\pgfqpoint{3.250222in}{0.672575in}}%
\pgfpathlineto{\pgfqpoint{3.204313in}{0.667830in}}%
\pgfpathlineto{\pgfqpoint{3.152903in}{0.664733in}}%
\pgfpathlineto{\pgfqpoint{3.095469in}{0.663311in}}%
\pgfpathlineto{\pgfqpoint{3.031412in}{0.663593in}}%
\pgfpathlineto{\pgfqpoint{2.935344in}{0.666644in}}%
\pgfpathlineto{\pgfqpoint{2.824266in}{0.673047in}}%
\pgfpathlineto{\pgfqpoint{2.695807in}{0.683130in}}%
\pgfpathlineto{\pgfqpoint{2.549511in}{0.697136in}}%
\pgfpathlineto{\pgfqpoint{2.386839in}{0.715223in}}%
\pgfpathlineto{\pgfqpoint{2.211164in}{0.737462in}}%
\pgfpathlineto{\pgfqpoint{2.073987in}{0.756863in}}%
\pgfpathlineto{\pgfqpoint{1.935495in}{0.778559in}}%
\pgfpathlineto{\pgfqpoint{1.801022in}{0.802390in}}%
\pgfpathlineto{\pgfqpoint{1.716170in}{0.819258in}}%
\pgfpathlineto{\pgfqpoint{1.636514in}{0.836757in}}%
\pgfpathlineto{\pgfqpoint{1.562940in}{0.854755in}}%
\pgfpathlineto{\pgfqpoint{1.496086in}{0.873121in}}%
\pgfpathlineto{\pgfqpoint{1.436341in}{0.891725in}}%
\pgfpathlineto{\pgfqpoint{1.383848in}{0.910437in}}%
\pgfpathlineto{\pgfqpoint{1.338497in}{0.929129in}}%
\pgfpathlineto{\pgfqpoint{1.300017in}{0.947665in}}%
\pgfpathlineto{\pgfqpoint{1.268058in}{0.965903in}}%
\pgfpathlineto{\pgfqpoint{1.241467in}{0.983778in}}%
\pgfpathlineto{\pgfqpoint{1.219230in}{1.001236in}}%
\pgfpathlineto{\pgfqpoint{1.200612in}{1.018228in}}%
\pgfpathlineto{\pgfqpoint{1.185151in}{1.034710in}}%
\pgfpathlineto{\pgfqpoint{1.172657in}{1.050640in}}%
\pgfpathlineto{\pgfqpoint{1.163218in}{1.065981in}}%
\pgfpathlineto{\pgfqpoint{1.157195in}{1.080701in}}%
\pgfpathlineto{\pgfqpoint{1.155196in}{1.094769in}}%
\pgfpathlineto{\pgfqpoint{1.156383in}{1.108157in}}%
\pgfpathlineto{\pgfqpoint{1.159841in}{1.120852in}}%
\pgfpathlineto{\pgfqpoint{1.165398in}{1.132848in}}%
\pgfpathlineto{\pgfqpoint{1.172930in}{1.144140in}}%
\pgfpathlineto{\pgfqpoint{1.182358in}{1.154728in}}%
\pgfpathlineto{\pgfqpoint{1.193652in}{1.164609in}}%
\pgfpathlineto{\pgfqpoint{1.206829in}{1.173786in}}%
\pgfpathlineto{\pgfqpoint{1.221951in}{1.182261in}}%
\pgfpathlineto{\pgfqpoint{1.239110in}{1.190038in}}%
\pgfpathlineto{\pgfqpoint{1.268539in}{1.200389in}}%
\pgfpathlineto{\pgfqpoint{1.302425in}{1.209156in}}%
\pgfpathlineto{\pgfqpoint{1.340941in}{1.216324in}}%
\pgfpathlineto{\pgfqpoint{1.384362in}{1.221876in}}%
\pgfpathlineto{\pgfqpoint{1.433061in}{1.225791in}}%
\pgfpathlineto{\pgfqpoint{1.487509in}{1.228042in}}%
\pgfpathlineto{\pgfqpoint{1.548278in}{1.228600in}}%
\pgfpathlineto{\pgfqpoint{1.639887in}{1.226680in}}%
\pgfpathlineto{\pgfqpoint{1.745075in}{1.221572in}}%
\pgfpathlineto{\pgfqpoint{1.866959in}{1.212898in}}%
\pgfpathlineto{\pgfqpoint{2.006831in}{1.200358in}}%
\pgfpathlineto{\pgfqpoint{2.163959in}{1.183759in}}%
\pgfpathlineto{\pgfqpoint{2.335589in}{1.163004in}}%
\pgfpathlineto{\pgfqpoint{2.471033in}{1.144711in}}%
\pgfpathlineto{\pgfqpoint{2.609230in}{1.124121in}}%
\pgfpathlineto{\pgfqpoint{2.746390in}{1.101307in}}%
\pgfpathlineto{\pgfqpoint{2.876780in}{1.076536in}}%
\pgfpathlineto{\pgfqpoint{2.957740in}{1.059161in}}%
\pgfpathlineto{\pgfqpoint{3.032787in}{1.041258in}}%
\pgfpathlineto{\pgfqpoint{3.101229in}{1.022959in}}%
\pgfpathlineto{\pgfqpoint{3.162627in}{1.004396in}}%
\pgfpathlineto{\pgfqpoint{3.216796in}{0.985700in}}%
\pgfpathlineto{\pgfqpoint{3.263806in}{0.966995in}}%
\pgfpathlineto{\pgfqpoint{3.303978in}{0.948408in}}%
\pgfpathlineto{\pgfqpoint{3.337572in}{0.930079in}}%
\pgfpathlineto{\pgfqpoint{3.365340in}{0.912107in}}%
\pgfpathlineto{\pgfqpoint{3.388400in}{0.894545in}}%
\pgfpathlineto{\pgfqpoint{3.407602in}{0.877442in}}%
\pgfpathlineto{\pgfqpoint{3.423528in}{0.860844in}}%
\pgfpathlineto{\pgfqpoint{3.436495in}{0.844791in}}%
\pgfpathlineto{\pgfqpoint{3.446551in}{0.829319in}}%
\pgfpathlineto{\pgfqpoint{3.453479in}{0.814460in}}%
\pgfpathlineto{\pgfqpoint{3.456793in}{0.800243in}}%
\pgfpathlineto{\pgfqpoint{3.456323in}{0.786694in}}%
\pgfpathlineto{\pgfqpoint{3.453420in}{0.773836in}}%
\pgfpathlineto{\pgfqpoint{3.448366in}{0.761675in}}%
\pgfpathlineto{\pgfqpoint{3.441302in}{0.750216in}}%
\pgfpathlineto{\pgfqpoint{3.432323in}{0.739461in}}%
\pgfpathlineto{\pgfqpoint{3.421478in}{0.729411in}}%
\pgfpathlineto{\pgfqpoint{3.408771in}{0.720067in}}%
\pgfpathlineto{\pgfqpoint{3.394158in}{0.711426in}}%
\pgfpathlineto{\pgfqpoint{3.377551in}{0.703485in}}%
\pgfpathlineto{\pgfqpoint{3.348857in}{0.692882in}}%
\pgfpathlineto{\pgfqpoint{3.315716in}{0.683861in}}%
\pgfpathlineto{\pgfqpoint{3.277970in}{0.676433in}}%
\pgfpathlineto{\pgfqpoint{3.235362in}{0.670617in}}%
\pgfpathlineto{\pgfqpoint{3.187539in}{0.666434in}}%
\pgfpathlineto{\pgfqpoint{3.134051in}{0.663911in}}%
\pgfpathlineto{\pgfqpoint{3.074352in}{0.663079in}}%
\pgfpathlineto{\pgfqpoint{3.007810in}{0.663975in}}%
\pgfpathlineto{\pgfqpoint{2.907836in}{0.667926in}}%
\pgfpathlineto{\pgfqpoint{2.792463in}{0.675292in}}%
\pgfpathlineto{\pgfqpoint{2.659653in}{0.686363in}}%
\pgfpathlineto{\pgfqpoint{2.509231in}{0.701363in}}%
\pgfpathlineto{\pgfqpoint{2.342883in}{0.720458in}}%
\pgfpathlineto{\pgfqpoint{2.164157in}{0.743748in}}%
\pgfpathlineto{\pgfqpoint{2.025285in}{0.764001in}}%
\pgfpathlineto{\pgfqpoint{1.887259in}{0.786554in}}%
\pgfpathlineto{\pgfqpoint{1.755394in}{0.811049in}}%
\pgfpathlineto{\pgfqpoint{1.673034in}{0.828260in}}%
\pgfpathlineto{\pgfqpoint{1.596287in}{0.846036in}}%
\pgfpathlineto{\pgfqpoint{1.525916in}{0.864250in}}%
\pgfpathlineto{\pgfqpoint{1.462466in}{0.882774in}}%
\pgfpathlineto{\pgfqpoint{1.406266in}{0.901480in}}%
\pgfpathlineto{\pgfqpoint{1.357427in}{0.920232in}}%
\pgfpathlineto{\pgfqpoint{1.315847in}{0.938893in}}%
\pgfpathlineto{\pgfqpoint{1.281024in}{0.957315in}}%
\pgfpathlineto{\pgfqpoint{1.251895in}{0.975410in}}%
\pgfpathlineto{\pgfqpoint{1.227526in}{0.993112in}}%
\pgfpathlineto{\pgfqpoint{1.207217in}{1.010365in}}%
\pgfpathlineto{\pgfqpoint{1.190505in}{1.027115in}}%
\pgfpathlineto{\pgfqpoint{1.177157in}{1.043317in}}%
\pgfpathlineto{\pgfqpoint{1.167176in}{1.058930in}}%
\pgfpathlineto{\pgfqpoint{1.160800in}{1.073921in}}%
\pgfpathlineto{\pgfqpoint{1.158098in}{1.088261in}}%
\pgfpathlineto{\pgfqpoint{1.158032in}{1.101926in}}%
\pgfpathlineto{\pgfqpoint{1.160287in}{1.114907in}}%
\pgfpathlineto{\pgfqpoint{1.164677in}{1.127195in}}%
\pgfpathlineto{\pgfqpoint{1.171072in}{1.138786in}}%
\pgfpathlineto{\pgfqpoint{1.179396in}{1.149677in}}%
\pgfpathlineto{\pgfqpoint{1.189632in}{1.159866in}}%
\pgfpathlineto{\pgfqpoint{1.201816in}{1.169355in}}%
\pgfpathlineto{\pgfqpoint{1.216039in}{1.178147in}}%
\pgfpathlineto{\pgfqpoint{1.232277in}{1.186242in}}%
\pgfpathlineto{\pgfqpoint{1.260260in}{1.197072in}}%
\pgfpathlineto{\pgfqpoint{1.292634in}{1.206319in}}%
\pgfpathlineto{\pgfqpoint{1.329544in}{1.213970in}}%
\pgfpathlineto{\pgfqpoint{1.371245in}{1.220012in}}%
\pgfpathlineto{\pgfqpoint{1.418100in}{1.224429in}}%
\pgfpathlineto{\pgfqpoint{1.470581in}{1.227201in}}%
\pgfpathlineto{\pgfqpoint{1.529249in}{1.228307in}}%
\pgfpathlineto{\pgfqpoint{1.594194in}{1.227725in}}%
\pgfpathlineto{\pgfqpoint{1.692413in}{1.224159in}}%
\pgfpathlineto{\pgfqpoint{1.806364in}{1.217165in}}%
\pgfpathlineto{\pgfqpoint{1.937440in}{1.206509in}}%
\pgfpathlineto{\pgfqpoint{2.085669in}{1.191986in}}%
\pgfpathlineto{\pgfqpoint{2.249722in}{1.173418in}}%
\pgfpathlineto{\pgfqpoint{2.426907in}{1.150658in}}%
\pgfpathlineto{\pgfqpoint{2.565665in}{1.130766in}}%
\pgfpathlineto{\pgfqpoint{2.704213in}{1.108583in}}%
\pgfpathlineto{\pgfqpoint{2.837181in}{1.084421in}}%
\pgfpathlineto{\pgfqpoint{2.920546in}{1.067388in}}%
\pgfpathlineto{\pgfqpoint{2.998462in}{1.049748in}}%
\pgfpathlineto{\pgfqpoint{3.070109in}{1.031627in}}%
\pgfpathlineto{\pgfqpoint{3.134876in}{1.013156in}}%
\pgfpathlineto{\pgfqpoint{3.192355in}{0.994476in}}%
\pgfpathlineto{\pgfqpoint{3.242341in}{0.975738in}}%
\pgfpathlineto{\pgfqpoint{3.285109in}{0.957097in}}%
\pgfpathlineto{\pgfqpoint{3.321601in}{0.938654in}}%
\pgfpathlineto{\pgfqpoint{3.352653in}{0.920495in}}%
\pgfpathlineto{\pgfqpoint{3.378913in}{0.902698in}}%
\pgfpathlineto{\pgfqpoint{3.400835in}{0.885331in}}%
\pgfpathlineto{\pgfqpoint{3.418679in}{0.868457in}}%
\pgfpathlineto{\pgfqpoint{3.432518in}{0.852128in}}%
\pgfpathlineto{\pgfqpoint{3.442237in}{0.836390in}}%
\pgfpathlineto{\pgfqpoint{3.448408in}{0.821279in}}%
\pgfpathlineto{\pgfqpoint{3.451761in}{0.806820in}}%
\pgfpathlineto{\pgfqpoint{3.452602in}{0.793027in}}%
\pgfpathlineto{\pgfqpoint{3.451159in}{0.779912in}}%
\pgfpathlineto{\pgfqpoint{3.447589in}{0.767483in}}%
\pgfpathlineto{\pgfqpoint{3.441972in}{0.755746in}}%
\pgfpathlineto{\pgfqpoint{3.434315in}{0.744704in}}%
\pgfpathlineto{\pgfqpoint{3.424551in}{0.734356in}}%
\pgfpathlineto{\pgfqpoint{3.412662in}{0.724702in}}%
\pgfpathlineto{\pgfqpoint{3.398818in}{0.715746in}}%
\pgfpathlineto{\pgfqpoint{3.383068in}{0.707490in}}%
\pgfpathlineto{\pgfqpoint{3.355903in}{0.696423in}}%
\pgfpathlineto{\pgfqpoint{3.324444in}{0.686943in}}%
\pgfpathlineto{\pgfqpoint{3.288516in}{0.679056in}}%
\pgfpathlineto{\pgfqpoint{3.247818in}{0.672770in}}%
\pgfpathlineto{\pgfqpoint{3.201916in}{0.668088in}}%
\pgfpathlineto{\pgfqpoint{3.150411in}{0.665021in}}%
\pgfpathlineto{\pgfqpoint{3.093081in}{0.663616in}}%
\pgfpathlineto{\pgfqpoint{3.029051in}{0.663931in}}%
\pgfpathlineto{\pgfqpoint{2.931868in}{0.667146in}}%
\pgfpathlineto{\pgfqpoint{2.819770in}{0.673724in}}%
\pgfpathlineto{\pgfqpoint{2.691585in}{0.683873in}}%
\pgfpathlineto{\pgfqpoint{2.546542in}{0.697830in}}%
\pgfpathlineto{\pgfqpoint{2.384053in}{0.715859in}}%
\pgfpathlineto{\pgfqpoint{2.205214in}{0.738217in}}%
\pgfpathlineto{\pgfqpoint{2.066515in}{0.757772in}}%
\pgfpathlineto{\pgfqpoint{1.928976in}{0.779556in}}%
\pgfpathlineto{\pgfqpoint{1.796635in}{0.803345in}}%
\pgfpathlineto{\pgfqpoint{1.713067in}{0.820174in}}%
\pgfpathlineto{\pgfqpoint{1.634327in}{0.837656in}}%
\pgfpathlineto{\pgfqpoint{1.561248in}{0.855668in}}%
\pgfpathlineto{\pgfqpoint{1.494567in}{0.874074in}}%
\pgfpathlineto{\pgfqpoint{1.434926in}{0.892724in}}%
\pgfpathlineto{\pgfqpoint{1.382861in}{0.911450in}}%
\pgfpathlineto{\pgfqpoint{1.338084in}{0.930113in}}%
\pgfpathlineto{\pgfqpoint{1.299741in}{0.948610in}}%
\pgfpathlineto{\pgfqpoint{1.267241in}{0.966842in}}%
\pgfpathlineto{\pgfqpoint{1.240055in}{0.984722in}}%
\pgfpathlineto{\pgfqpoint{1.217721in}{1.002173in}}%
\pgfpathlineto{\pgfqpoint{1.199837in}{1.019131in}}%
\pgfpathlineto{\pgfqpoint{1.185951in}{1.035545in}}%
\pgfpathlineto{\pgfqpoint{1.175566in}{1.051375in}}%
\pgfpathlineto{\pgfqpoint{1.168312in}{1.066591in}}%
\pgfpathlineto{\pgfqpoint{1.163894in}{1.081171in}}%
\pgfpathlineto{\pgfqpoint{1.162097in}{1.095094in}}%
\pgfpathlineto{\pgfqpoint{1.162783in}{1.108346in}}%
\pgfpathlineto{\pgfqpoint{1.165880in}{1.120920in}}%
\pgfpathlineto{\pgfqpoint{1.171223in}{1.132808in}}%
\pgfpathlineto{\pgfqpoint{1.178630in}{1.144005in}}%
\pgfpathlineto{\pgfqpoint{1.187965in}{1.154506in}}%
\pgfpathlineto{\pgfqpoint{1.199139in}{1.164308in}}%
\pgfpathlineto{\pgfqpoint{1.212109in}{1.173410in}}%
\pgfpathlineto{\pgfqpoint{1.226883in}{1.181812in}}%
\pgfpathlineto{\pgfqpoint{1.252553in}{1.193106in}}%
\pgfpathlineto{\pgfqpoint{1.282786in}{1.202839in}}%
\pgfpathlineto{\pgfqpoint{1.317953in}{1.211020in}}%
\pgfpathlineto{\pgfqpoint{1.357897in}{1.217618in}}%
\pgfpathlineto{\pgfqpoint{1.402913in}{1.222612in}}%
\pgfpathlineto{\pgfqpoint{1.453404in}{1.225971in}}%
\pgfpathlineto{\pgfqpoint{1.509847in}{1.227659in}}%
\pgfpathlineto{\pgfqpoint{1.572790in}{1.227631in}}%
\pgfpathlineto{\pgfqpoint{1.642855in}{1.225832in}}%
\pgfpathlineto{\pgfqpoint{1.748564in}{1.220568in}}%
\pgfpathlineto{\pgfqpoint{1.869836in}{1.211844in}}%
\pgfpathlineto{\pgfqpoint{2.008270in}{1.199393in}}%
\pgfpathlineto{\pgfqpoint{2.164060in}{1.182972in}}%
\pgfpathlineto{\pgfqpoint{2.335595in}{1.162374in}}%
\pgfpathlineto{\pgfqpoint{2.471944in}{1.144119in}}%
\pgfpathlineto{\pgfqpoint{2.610697in}{1.123517in}}%
\pgfpathlineto{\pgfqpoint{2.747537in}{1.100730in}}%
\pgfpathlineto{\pgfqpoint{2.877881in}{1.076029in}}%
\pgfpathlineto{\pgfqpoint{2.958779in}{1.058678in}}%
\pgfpathlineto{\pgfqpoint{3.033214in}{1.040766in}}%
\pgfpathlineto{\pgfqpoint{3.100773in}{1.022448in}}%
\pgfpathlineto{\pgfqpoint{3.161421in}{1.003877in}}%
\pgfpathlineto{\pgfqpoint{3.215218in}{0.985194in}}%
\pgfpathlineto{\pgfqpoint{3.262310in}{0.966528in}}%
\pgfpathlineto{\pgfqpoint{3.302934in}{0.947995in}}%
\pgfpathlineto{\pgfqpoint{3.337416in}{0.929698in}}%
\pgfpathlineto{\pgfqpoint{3.366172in}{0.911726in}}%
\pgfpathlineto{\pgfqpoint{3.389708in}{0.894158in}}%
\pgfpathlineto{\pgfqpoint{3.408600in}{0.877058in}}%
\pgfpathlineto{\pgfqpoint{3.423124in}{0.860498in}}%
\pgfpathlineto{\pgfqpoint{3.433916in}{0.844517in}}%
\pgfpathlineto{\pgfqpoint{3.441662in}{0.829141in}}%
\pgfpathlineto{\pgfqpoint{3.446876in}{0.814392in}}%
\pgfpathlineto{\pgfqpoint{3.449899in}{0.800285in}}%
\pgfpathlineto{\pgfqpoint{3.450904in}{0.786836in}}%
\pgfpathlineto{\pgfqpoint{3.449888in}{0.774054in}}%
\pgfpathlineto{\pgfqpoint{3.446680in}{0.761945in}}%
\pgfpathlineto{\pgfqpoint{3.440938in}{0.750511in}}%
\pgfpathlineto{\pgfqpoint{3.432240in}{0.739751in}}%
\pgfpathlineto{\pgfqpoint{3.421276in}{0.729687in}}%
\pgfpathlineto{\pgfqpoint{3.408385in}{0.720325in}}%
\pgfpathlineto{\pgfqpoint{3.393613in}{0.711668in}}%
\pgfpathlineto{\pgfqpoint{3.367960in}{0.700004in}}%
\pgfpathlineto{\pgfqpoint{3.338061in}{0.689931in}}%
\pgfpathlineto{\pgfqpoint{3.303729in}{0.681449in}}%
\pgfpathlineto{\pgfqpoint{3.264641in}{0.674558in}}%
\pgfpathlineto{\pgfqpoint{3.220437in}{0.669260in}}%
\pgfpathlineto{\pgfqpoint{3.170959in}{0.665584in}}%
\pgfpathlineto{\pgfqpoint{3.115615in}{0.663568in}}%
\pgfpathlineto{\pgfqpoint{3.053769in}{0.663259in}}%
\pgfpathlineto{\pgfqpoint{2.984786in}{0.664715in}}%
\pgfpathlineto{\pgfqpoint{2.880621in}{0.669524in}}%
\pgfpathlineto{\pgfqpoint{2.761175in}{0.677790in}}%
\pgfpathlineto{\pgfqpoint{2.624984in}{0.689744in}}%
\pgfpathlineto{\pgfqpoint{2.471166in}{0.705625in}}%
\pgfpathlineto{\pgfqpoint{2.301145in}{0.725698in}}%
\pgfpathlineto{\pgfqpoint{2.165627in}{0.743569in}}%
\pgfpathlineto{\pgfqpoint{2.026708in}{0.763799in}}%
\pgfpathlineto{\pgfqpoint{1.888674in}{0.786248in}}%
\pgfpathlineto{\pgfqpoint{1.756701in}{0.810704in}}%
\pgfpathlineto{\pgfqpoint{1.674403in}{0.827938in}}%
\pgfpathlineto{\pgfqpoint{1.597879in}{0.845743in}}%
\pgfpathlineto{\pgfqpoint{1.527881in}{0.863976in}}%
\pgfpathlineto{\pgfqpoint{1.464889in}{0.882500in}}%
\pgfpathlineto{\pgfqpoint{1.409116in}{0.901182in}}%
\pgfpathlineto{\pgfqpoint{1.360506in}{0.919893in}}%
\pgfpathlineto{\pgfqpoint{1.318732in}{0.938513in}}%
\pgfpathlineto{\pgfqpoint{1.283308in}{0.956920in}}%
\pgfpathlineto{\pgfqpoint{1.254021in}{0.974989in}}%
\pgfpathlineto{\pgfqpoint{1.229847in}{0.992657in}}%
\pgfpathlineto{\pgfqpoint{1.209840in}{1.009874in}}%
\pgfpathlineto{\pgfqpoint{1.193312in}{1.026595in}}%
\pgfpathlineto{\pgfqpoint{1.179829in}{1.042777in}}%
\pgfpathlineto{\pgfqpoint{1.169213in}{1.058385in}}%
\pgfpathlineto{\pgfqpoint{1.161542in}{1.073387in}}%
\pgfpathlineto{\pgfqpoint{1.157150in}{1.087756in}}%
\pgfpathlineto{\pgfqpoint{1.156547in}{1.101467in}}%
\pgfpathlineto{\pgfqpoint{1.158800in}{1.114495in}}%
\pgfpathlineto{\pgfqpoint{1.163264in}{1.126828in}}%
\pgfpathlineto{\pgfqpoint{1.169782in}{1.138463in}}%
\pgfpathlineto{\pgfqpoint{1.178242in}{1.149395in}}%
\pgfpathlineto{\pgfqpoint{1.188579in}{1.159622in}}%
\pgfpathlineto{\pgfqpoint{1.200770in}{1.169145in}}%
\pgfpathlineto{\pgfqpoint{1.214840in}{1.177964in}}%
\pgfpathlineto{\pgfqpoint{1.230857in}{1.186082in}}%
\pgfpathlineto{\pgfqpoint{1.258706in}{1.196953in}}%
\pgfpathlineto{\pgfqpoint{1.291023in}{1.206247in}}%
\pgfpathlineto{\pgfqpoint{1.327918in}{1.213952in}}%
\pgfpathlineto{\pgfqpoint{1.369631in}{1.220051in}}%
\pgfpathlineto{\pgfqpoint{1.416497in}{1.224522in}}%
\pgfpathlineto{\pgfqpoint{1.468943in}{1.227339in}}%
\pgfpathlineto{\pgfqpoint{1.527492in}{1.228470in}}%
\pgfpathlineto{\pgfqpoint{1.592761in}{1.227877in}}%
\pgfpathlineto{\pgfqpoint{1.691077in}{1.224333in}}%
\pgfpathlineto{\pgfqpoint{1.804231in}{1.217425in}}%
\pgfpathlineto{\pgfqpoint{1.934525in}{1.206864in}}%
\pgfpathlineto{\pgfqpoint{2.082465in}{1.192415in}}%
\pgfpathlineto{\pgfqpoint{2.246735in}{1.173893in}}%
\pgfpathlineto{\pgfqpoint{2.424192in}{1.151171in}}%
\pgfpathlineto{\pgfqpoint{2.562892in}{1.131320in}}%
\pgfpathlineto{\pgfqpoint{2.701559in}{1.109150in}}%
\pgfpathlineto{\pgfqpoint{2.834800in}{1.084992in}}%
\pgfpathlineto{\pgfqpoint{2.918412in}{1.067965in}}%
\pgfpathlineto{\pgfqpoint{2.996620in}{1.050335in}}%
\pgfpathlineto{\pgfqpoint{3.068605in}{1.032226in}}%
\pgfpathlineto{\pgfqpoint{3.133762in}{1.013767in}}%
\pgfpathlineto{\pgfqpoint{3.191694in}{0.995089in}}%
\pgfpathlineto{\pgfqpoint{3.242218in}{0.976335in}}%
\pgfpathlineto{\pgfqpoint{3.285378in}{0.957652in}}%
\pgfpathlineto{\pgfqpoint{3.321837in}{0.939181in}}%
\pgfpathlineto{\pgfqpoint{3.352622in}{0.921004in}}%
\pgfpathlineto{\pgfqpoint{3.378564in}{0.903193in}}%
\pgfpathlineto{\pgfqpoint{3.400274in}{0.885814in}}%
\pgfpathlineto{\pgfqpoint{3.418145in}{0.868924in}}%
\pgfpathlineto{\pgfqpoint{3.432354in}{0.852573in}}%
\pgfpathlineto{\pgfqpoint{3.442860in}{0.836806in}}%
\pgfpathlineto{\pgfqpoint{3.449442in}{0.821658in}}%
\pgfpathlineto{\pgfqpoint{3.452810in}{0.807162in}}%
\pgfpathlineto{\pgfqpoint{3.453598in}{0.793336in}}%
\pgfpathlineto{\pgfqpoint{3.452054in}{0.780188in}}%
\pgfpathlineto{\pgfqpoint{3.448363in}{0.767730in}}%
\pgfpathlineto{\pgfqpoint{3.442644in}{0.755965in}}%
\pgfpathlineto{\pgfqpoint{3.434955in}{0.744898in}}%
\pgfpathlineto{\pgfqpoint{3.425289in}{0.734530in}}%
\pgfpathlineto{\pgfqpoint{3.413577in}{0.724858in}}%
\pgfpathlineto{\pgfqpoint{3.399791in}{0.715882in}}%
\pgfpathlineto{\pgfqpoint{3.384068in}{0.707604in}}%
\pgfpathlineto{\pgfqpoint{3.356896in}{0.696501in}}%
\pgfpathlineto{\pgfqpoint{3.325391in}{0.686983in}}%
\pgfpathlineto{\pgfqpoint{3.289414in}{0.679060in}}%
\pgfpathlineto{\pgfqpoint{3.248703in}{0.672742in}}%
\pgfpathlineto{\pgfqpoint{3.202880in}{0.668040in}}%
\pgfpathlineto{\pgfqpoint{3.151447in}{0.664966in}}%
\pgfpathlineto{\pgfqpoint{3.094091in}{0.663538in}}%
\pgfpathlineto{\pgfqpoint{3.030381in}{0.663815in}}%
\pgfpathlineto{\pgfqpoint{2.933644in}{0.666979in}}%
\pgfpathlineto{\pgfqpoint{2.821649in}{0.673521in}}%
\pgfpathlineto{\pgfqpoint{2.693208in}{0.683651in}}%
\pgfpathlineto{\pgfqpoint{2.547945in}{0.697586in}}%
\pgfpathlineto{\pgfqpoint{2.386293in}{0.715551in}}%
\pgfpathlineto{\pgfqpoint{2.209439in}{0.737795in}}%
\pgfpathlineto{\pgfqpoint{2.070251in}{0.757352in}}%
\pgfpathlineto{\pgfqpoint{1.931423in}{0.779166in}}%
\pgfpathlineto{\pgfqpoint{1.797771in}{0.802971in}}%
\pgfpathlineto{\pgfqpoint{1.713553in}{0.819793in}}%
\pgfpathlineto{\pgfqpoint{1.634408in}{0.837253in}}%
\pgfpathlineto{\pgfqpoint{1.561155in}{0.855235in}}%
\pgfpathlineto{\pgfqpoint{1.494448in}{0.873611in}}%
\pgfpathlineto{\pgfqpoint{1.434781in}{0.892244in}}%
\pgfpathlineto{\pgfqpoint{1.382483in}{0.910986in}}%
\pgfpathlineto{\pgfqpoint{1.337639in}{0.929679in}}%
\pgfpathlineto{\pgfqpoint{1.299521in}{0.948195in}}%
\pgfpathlineto{\pgfqpoint{1.267216in}{0.966441in}}%
\pgfpathlineto{\pgfqpoint{1.239990in}{0.984340in}}%
\pgfpathlineto{\pgfqpoint{1.217286in}{1.001819in}}%
\pgfpathlineto{\pgfqpoint{1.198717in}{1.018817in}}%
\pgfpathlineto{\pgfqpoint{1.184076in}{1.035279in}}%
\pgfpathlineto{\pgfqpoint{1.173327in}{1.051161in}}%
\pgfpathlineto{\pgfqpoint{1.166250in}{1.066426in}}%
\pgfpathlineto{\pgfqpoint{1.162087in}{1.081046in}}%
\pgfpathlineto{\pgfqpoint{1.160522in}{1.095006in}}%
\pgfpathlineto{\pgfqpoint{1.161320in}{1.108293in}}%
\pgfpathlineto{\pgfqpoint{1.164316in}{1.120896in}}%
\pgfpathlineto{\pgfqpoint{1.169410in}{1.132809in}}%
\pgfpathlineto{\pgfqpoint{1.176575in}{1.144029in}}%
\pgfpathlineto{\pgfqpoint{1.185850in}{1.154555in}}%
\pgfpathlineto{\pgfqpoint{1.197206in}{1.164387in}}%
\pgfpathlineto{\pgfqpoint{1.210519in}{1.173522in}}%
\pgfpathlineto{\pgfqpoint{1.225734in}{1.181958in}}%
\pgfpathlineto{\pgfqpoint{1.252075in}{1.193295in}}%
\pgfpathlineto{\pgfqpoint{1.282672in}{1.203046in}}%
\pgfpathlineto{\pgfqpoint{1.317691in}{1.211208in}}%
\pgfpathlineto{\pgfqpoint{1.357435in}{1.217774in}}%
\pgfpathlineto{\pgfqpoint{1.402347in}{1.222742in}}%
\pgfpathlineto{\pgfqpoint{1.452798in}{1.226103in}}%
\pgfpathlineto{\pgfqpoint{1.509021in}{1.227808in}}%
\pgfpathlineto{\pgfqpoint{1.571835in}{1.227803in}}%
\pgfpathlineto{\pgfqpoint{1.642004in}{1.226024in}}%
\pgfpathlineto{\pgfqpoint{1.748144in}{1.220763in}}%
\pgfpathlineto{\pgfqpoint{1.869871in}{1.212019in}}%
\pgfpathlineto{\pgfqpoint{2.008229in}{1.199561in}}%
\pgfpathlineto{\pgfqpoint{2.164201in}{1.183123in}}%
\pgfpathlineto{\pgfqpoint{2.339922in}{1.162278in}}%
\pgfpathlineto{\pgfqpoint{2.478675in}{1.143795in}}%
\pgfpathlineto{\pgfqpoint{2.617630in}{1.123046in}}%
\pgfpathlineto{\pgfqpoint{2.752394in}{1.100246in}}%
\pgfpathlineto{\pgfqpoint{2.879309in}{1.075658in}}%
\pgfpathlineto{\pgfqpoint{2.958086in}{1.058427in}}%
\pgfpathlineto{\pgfqpoint{3.031377in}{1.040646in}}%
\pgfpathlineto{\pgfqpoint{3.098644in}{1.022431in}}%
\pgfpathlineto{\pgfqpoint{3.159493in}{1.003908in}}%
\pgfpathlineto{\pgfqpoint{3.213677in}{0.985213in}}%
\pgfpathlineto{\pgfqpoint{3.261094in}{0.966491in}}%
\pgfpathlineto{\pgfqpoint{3.301785in}{0.947898in}}%
\pgfpathlineto{\pgfqpoint{3.335972in}{0.929593in}}%
\pgfpathlineto{\pgfqpoint{3.364464in}{0.911640in}}%
\pgfpathlineto{\pgfqpoint{3.387986in}{0.894095in}}%
\pgfpathlineto{\pgfqpoint{3.407079in}{0.877023in}}%
\pgfpathlineto{\pgfqpoint{3.422191in}{0.860477in}}%
\pgfpathlineto{\pgfqpoint{3.433670in}{0.844502in}}%
\pgfpathlineto{\pgfqpoint{3.441808in}{0.829131in}}%
\pgfpathlineto{\pgfqpoint{3.446974in}{0.814391in}}%
\pgfpathlineto{\pgfqpoint{3.449442in}{0.800301in}}%
\pgfpathlineto{\pgfqpoint{3.449425in}{0.786877in}}%
\pgfpathlineto{\pgfqpoint{3.447083in}{0.774130in}}%
\pgfpathlineto{\pgfqpoint{3.442521in}{0.762067in}}%
\pgfpathlineto{\pgfqpoint{3.435770in}{0.750691in}}%
\pgfpathlineto{\pgfqpoint{3.426921in}{0.740005in}}%
\pgfpathlineto{\pgfqpoint{3.416104in}{0.730014in}}%
\pgfpathlineto{\pgfqpoint{3.403414in}{0.720720in}}%
\pgfpathlineto{\pgfqpoint{3.388905in}{0.712126in}}%
\pgfpathlineto{\pgfqpoint{3.363758in}{0.700551in}}%
\pgfpathlineto{\pgfqpoint{3.334443in}{0.690556in}}%
\pgfpathlineto{\pgfqpoint{3.300661in}{0.682135in}}%
\pgfpathlineto{\pgfqpoint{3.261918in}{0.675278in}}%
\pgfpathlineto{\pgfqpoint{3.217917in}{0.669995in}}%
\pgfpathlineto{\pgfqpoint{3.168589in}{0.666331in}}%
\pgfpathlineto{\pgfqpoint{3.113379in}{0.664322in}}%
\pgfpathlineto{\pgfqpoint{3.051700in}{0.664015in}}%
\pgfpathlineto{\pgfqpoint{2.982944in}{0.665469in}}%
\pgfpathlineto{\pgfqpoint{2.879170in}{0.670265in}}%
\pgfpathlineto{\pgfqpoint{2.760133in}{0.678507in}}%
\pgfpathlineto{\pgfqpoint{2.624287in}{0.690420in}}%
\pgfpathlineto{\pgfqpoint{2.470934in}{0.706269in}}%
\pgfpathlineto{\pgfqpoint{2.301381in}{0.726284in}}%
\pgfpathlineto{\pgfqpoint{2.166182in}{0.744084in}}%
\pgfpathlineto{\pgfqpoint{2.027453in}{0.764256in}}%
\pgfpathlineto{\pgfqpoint{1.889677in}{0.786661in}}%
\pgfpathlineto{\pgfqpoint{1.757963in}{0.811027in}}%
\pgfpathlineto{\pgfqpoint{1.675686in}{0.828189in}}%
\pgfpathlineto{\pgfqpoint{1.599101in}{0.845944in}}%
\pgfpathlineto{\pgfqpoint{1.529106in}{0.864152in}}%
\pgfpathlineto{\pgfqpoint{1.466492in}{0.882658in}}%
\pgfpathlineto{\pgfqpoint{1.411260in}{0.901313in}}%
\pgfpathlineto{\pgfqpoint{1.362849in}{0.919990in}}%
\pgfpathlineto{\pgfqpoint{1.320750in}{0.938573in}}%
\pgfpathlineto{\pgfqpoint{1.284518in}{0.956956in}}%
\pgfpathlineto{\pgfqpoint{1.253768in}{0.975045in}}%
\pgfpathlineto{\pgfqpoint{1.228176in}{0.992755in}}%
\pgfpathlineto{\pgfqpoint{1.207482in}{1.010013in}}%
\pgfpathlineto{\pgfqpoint{1.191469in}{1.026755in}}%
\pgfpathlineto{\pgfqpoint{1.179504in}{1.042927in}}%
\pgfpathlineto{\pgfqpoint{1.170902in}{1.058496in}}%
\pgfpathlineto{\pgfqpoint{1.165196in}{1.073437in}}%
\pgfpathlineto{\pgfqpoint{1.162038in}{1.087731in}}%
\pgfpathlineto{\pgfqpoint{1.161193in}{1.101362in}}%
\pgfpathlineto{\pgfqpoint{1.162546in}{1.114319in}}%
\pgfpathlineto{\pgfqpoint{1.166097in}{1.126595in}}%
\pgfpathlineto{\pgfqpoint{1.171962in}{1.138186in}}%
\pgfpathlineto{\pgfqpoint{1.180260in}{1.149090in}}%
\pgfpathlineto{\pgfqpoint{1.190630in}{1.159299in}}%
\pgfpathlineto{\pgfqpoint{1.202940in}{1.168809in}}%
\pgfpathlineto{\pgfqpoint{1.217133in}{1.177618in}}%
\pgfpathlineto{\pgfqpoint{1.233183in}{1.185723in}}%
\pgfpathlineto{\pgfqpoint{1.260749in}{1.196562in}}%
\pgfpathlineto{\pgfqpoint{1.292636in}{1.205815in}}%
\pgfpathlineto{\pgfqpoint{1.329124in}{1.213484in}}%
\pgfpathlineto{\pgfqpoint{1.370634in}{1.219572in}}%
\pgfpathlineto{\pgfqpoint{1.417307in}{1.224058in}}%
\pgfpathlineto{\pgfqpoint{1.469550in}{1.226907in}}%
\pgfpathlineto{\pgfqpoint{1.527975in}{1.228077in}}%
\pgfpathlineto{\pgfqpoint{1.593200in}{1.227514in}}%
\pgfpathlineto{\pgfqpoint{1.691827in}{1.223958in}}%
\pgfpathlineto{\pgfqpoint{1.805151in}{1.217028in}}%
\pgfpathlineto{\pgfqpoint{1.934684in}{1.206506in}}%
\pgfpathlineto{\pgfqpoint{2.081661in}{1.192154in}}%
\pgfpathlineto{\pgfqpoint{2.245679in}{1.173701in}}%
\pgfpathlineto{\pgfqpoint{2.423507in}{1.150989in}}%
\pgfpathlineto{\pgfqpoint{2.561999in}{1.131188in}}%
\pgfpathlineto{\pgfqpoint{2.700362in}{1.109124in}}%
\pgfpathlineto{\pgfqpoint{2.833653in}{1.085003in}}%
\pgfpathlineto{\pgfqpoint{2.917272in}{1.067976in}}%
\pgfpathlineto{\pgfqpoint{2.995354in}{1.050342in}}%
\pgfpathlineto{\pgfqpoint{3.067060in}{1.032234in}}%
\pgfpathlineto{\pgfqpoint{3.131834in}{1.013785in}}%
\pgfpathlineto{\pgfqpoint{3.189396in}{0.995131in}}%
\pgfpathlineto{\pgfqpoint{3.239745in}{0.976411in}}%
\pgfpathlineto{\pgfqpoint{3.282876in}{0.957780in}}%
\pgfpathlineto{\pgfqpoint{3.319307in}{0.939364in}}%
\pgfpathlineto{\pgfqpoint{3.350197in}{0.921233in}}%
\pgfpathlineto{\pgfqpoint{3.376449in}{0.903455in}}%
\pgfpathlineto{\pgfqpoint{3.398707in}{0.886090in}}%
\pgfpathlineto{\pgfqpoint{3.417354in}{0.869196in}}%
\pgfpathlineto{\pgfqpoint{3.432509in}{0.852824in}}%
\pgfpathlineto{\pgfqpoint{3.444033in}{0.837022in}}%
\pgfpathlineto{\pgfqpoint{3.451523in}{0.821830in}}%
\pgfpathlineto{\pgfqpoint{3.454893in}{0.807288in}}%
\pgfpathlineto{\pgfqpoint{3.455563in}{0.793419in}}%
\pgfpathlineto{\pgfqpoint{3.453879in}{0.780235in}}%
\pgfpathlineto{\pgfqpoint{3.450034in}{0.767743in}}%
\pgfpathlineto{\pgfqpoint{3.444164in}{0.755948in}}%
\pgfpathlineto{\pgfqpoint{3.436346in}{0.744854in}}%
\pgfpathlineto{\pgfqpoint{3.426602in}{0.734462in}}%
\pgfpathlineto{\pgfqpoint{3.414893in}{0.724772in}}%
\pgfpathlineto{\pgfqpoint{3.401132in}{0.715779in}}%
\pgfpathlineto{\pgfqpoint{3.385380in}{0.707484in}}%
\pgfpathlineto{\pgfqpoint{3.358145in}{0.696357in}}%
\pgfpathlineto{\pgfqpoint{3.326558in}{0.686816in}}%
\pgfpathlineto{\pgfqpoint{3.290489in}{0.678870in}}%
\pgfpathlineto{\pgfqpoint{3.249692in}{0.672533in}}%
\pgfpathlineto{\pgfqpoint{3.203811in}{0.667817in}}%
\pgfpathlineto{\pgfqpoint{3.152376in}{0.664739in}}%
\pgfpathlineto{\pgfqpoint{3.094862in}{0.663317in}}%
\pgfpathlineto{\pgfqpoint{3.031162in}{0.663581in}}%
\pgfpathlineto{\pgfqpoint{2.934681in}{0.666716in}}%
\pgfpathlineto{\pgfqpoint{2.822783in}{0.673243in}}%
\pgfpathlineto{\pgfqpoint{2.694103in}{0.683383in}}%
\pgfpathlineto{\pgfqpoint{2.548398in}{0.697344in}}%
\pgfpathlineto{\pgfqpoint{2.386553in}{0.715326in}}%
\pgfpathlineto{\pgfqpoint{2.210576in}{0.737517in}}%
\pgfpathlineto{\pgfqpoint{2.071858in}{0.757019in}}%
\pgfpathlineto{\pgfqpoint{1.932789in}{0.778825in}}%
\pgfpathlineto{\pgfqpoint{1.798592in}{0.802651in}}%
\pgfpathlineto{\pgfqpoint{1.713995in}{0.819496in}}%
\pgfpathlineto{\pgfqpoint{1.634540in}{0.836981in}}%
\pgfpathlineto{\pgfqpoint{1.561093in}{0.854986in}}%
\pgfpathlineto{\pgfqpoint{1.494333in}{0.873382in}}%
\pgfpathlineto{\pgfqpoint{1.434763in}{0.892029in}}%
\pgfpathlineto{\pgfqpoint{1.382700in}{0.910777in}}%
\pgfpathlineto{\pgfqpoint{1.338042in}{0.929466in}}%
\pgfpathlineto{\pgfqpoint{1.299910in}{0.947982in}}%
\pgfpathlineto{\pgfqpoint{1.267446in}{0.966236in}}%
\pgfpathlineto{\pgfqpoint{1.239973in}{0.984146in}}%
\pgfpathlineto{\pgfqpoint{1.216992in}{1.001641in}}%
\pgfpathlineto{\pgfqpoint{1.198189in}{1.018656in}}%
\pgfpathlineto{\pgfqpoint{1.183430in}{1.035138in}}%
\pgfpathlineto{\pgfqpoint{1.172759in}{1.051038in}}%
\pgfpathlineto{\pgfqpoint{1.165747in}{1.066318in}}%
\pgfpathlineto{\pgfqpoint{1.161629in}{1.080952in}}%
\pgfpathlineto{\pgfqpoint{1.160088in}{1.094925in}}%
\pgfpathlineto{\pgfqpoint{1.160886in}{1.108224in}}%
\pgfpathlineto{\pgfqpoint{1.163856in}{1.120839in}}%
\pgfpathlineto{\pgfqpoint{1.168907in}{1.132763in}}%
\pgfpathlineto{\pgfqpoint{1.176024in}{1.143994in}}%
\pgfpathlineto{\pgfqpoint{1.185264in}{1.154531in}}%
\pgfpathlineto{\pgfqpoint{1.196624in}{1.164374in}}%
\pgfpathlineto{\pgfqpoint{1.209941in}{1.173519in}}%
\pgfpathlineto{\pgfqpoint{1.225163in}{1.181965in}}%
\pgfpathlineto{\pgfqpoint{1.251520in}{1.193316in}}%
\pgfpathlineto{\pgfqpoint{1.282139in}{1.203081in}}%
\pgfpathlineto{\pgfqpoint{1.317183in}{1.211254in}}%
\pgfpathlineto{\pgfqpoint{1.356950in}{1.217831in}}%
\pgfpathlineto{\pgfqpoint{1.401872in}{1.222809in}}%
\pgfpathlineto{\pgfqpoint{1.452324in}{1.226178in}}%
\pgfpathlineto{\pgfqpoint{1.508539in}{1.227891in}}%
\pgfpathlineto{\pgfqpoint{1.571343in}{1.227892in}}%
\pgfpathlineto{\pgfqpoint{1.641502in}{1.226119in}}%
\pgfpathlineto{\pgfqpoint{1.747637in}{1.220867in}}%
\pgfpathlineto{\pgfqpoint{1.869364in}{1.212130in}}%
\pgfpathlineto{\pgfqpoint{2.007720in}{1.199680in}}%
\pgfpathlineto{\pgfqpoint{2.163624in}{1.183251in}}%
\pgfpathlineto{\pgfqpoint{2.339294in}{1.162411in}}%
\pgfpathlineto{\pgfqpoint{2.478137in}{1.143924in}}%
\pgfpathlineto{\pgfqpoint{2.617220in}{1.123170in}}%
\pgfpathlineto{\pgfqpoint{2.752107in}{1.100366in}}%
\pgfpathlineto{\pgfqpoint{2.879110in}{1.075776in}}%
\pgfpathlineto{\pgfqpoint{2.957918in}{1.058546in}}%
\pgfpathlineto{\pgfqpoint{3.031218in}{1.040767in}}%
\pgfpathlineto{\pgfqpoint{3.098475in}{1.022555in}}%
\pgfpathlineto{\pgfqpoint{3.159302in}{1.004036in}}%
\pgfpathlineto{\pgfqpoint{3.213459in}{0.985344in}}%
\pgfpathlineto{\pgfqpoint{3.260857in}{0.966623in}}%
\pgfpathlineto{\pgfqpoint{3.301554in}{0.948029in}}%
\pgfpathlineto{\pgfqpoint{3.335780in}{0.929720in}}%
\pgfpathlineto{\pgfqpoint{3.364295in}{0.911765in}}%
\pgfpathlineto{\pgfqpoint{3.387836in}{0.894217in}}%
\pgfpathlineto{\pgfqpoint{3.406962in}{0.877142in}}%
\pgfpathlineto{\pgfqpoint{3.422125in}{0.860591in}}%
\pgfpathlineto{\pgfqpoint{3.433671in}{0.844610in}}%
\pgfpathlineto{\pgfqpoint{3.441856in}{0.829233in}}%
\pgfpathlineto{\pgfqpoint{3.447032in}{0.814487in}}%
\pgfpathlineto{\pgfqpoint{3.449504in}{0.800392in}}%
\pgfpathlineto{\pgfqpoint{3.449491in}{0.786963in}}%
\pgfpathlineto{\pgfqpoint{3.447161in}{0.774210in}}%
\pgfpathlineto{\pgfqpoint{3.442628in}{0.762141in}}%
\pgfpathlineto{\pgfqpoint{3.435951in}{0.750762in}}%
\pgfpathlineto{\pgfqpoint{3.427154in}{0.740072in}}%
\pgfpathlineto{\pgfqpoint{3.416350in}{0.730074in}}%
\pgfpathlineto{\pgfqpoint{3.403634in}{0.720773in}}%
\pgfpathlineto{\pgfqpoint{3.389070in}{0.712172in}}%
\pgfpathlineto{\pgfqpoint{3.363807in}{0.700585in}}%
\pgfpathlineto{\pgfqpoint{3.334381in}{0.690579in}}%
\pgfpathlineto{\pgfqpoint{3.300557in}{0.682154in}}%
\pgfpathlineto{\pgfqpoint{3.261926in}{0.675304in}}%
\pgfpathlineto{\pgfqpoint{3.217982in}{0.670026in}}%
\pgfpathlineto{\pgfqpoint{3.168698in}{0.666359in}}%
\pgfpathlineto{\pgfqpoint{3.113566in}{0.664344in}}%
\pgfpathlineto{\pgfqpoint{3.051948in}{0.664030in}}%
\pgfpathlineto{\pgfqpoint{2.983212in}{0.665475in}}%
\pgfpathlineto{\pgfqpoint{2.879403in}{0.670260in}}%
\pgfpathlineto{\pgfqpoint{2.760345in}{0.678492in}}%
\pgfpathlineto{\pgfqpoint{2.624573in}{0.690400in}}%
\pgfpathlineto{\pgfqpoint{2.471246in}{0.706226in}}%
\pgfpathlineto{\pgfqpoint{2.301729in}{0.726232in}}%
\pgfpathlineto{\pgfqpoint{2.166561in}{0.744043in}}%
\pgfpathlineto{\pgfqpoint{2.027952in}{0.764207in}}%
\pgfpathlineto{\pgfqpoint{1.890188in}{0.786589in}}%
\pgfpathlineto{\pgfqpoint{1.758367in}{0.810983in}}%
\pgfpathlineto{\pgfqpoint{1.676091in}{0.828169in}}%
\pgfpathlineto{\pgfqpoint{1.599550in}{0.845927in}}%
\pgfpathlineto{\pgfqpoint{1.529508in}{0.864117in}}%
\pgfpathlineto{\pgfqpoint{1.466453in}{0.882604in}}%
\pgfpathlineto{\pgfqpoint{1.410603in}{0.901255in}}%
\pgfpathlineto{\pgfqpoint{1.361900in}{0.919942in}}%
\pgfpathlineto{\pgfqpoint{1.320012in}{0.938540in}}%
\pgfpathlineto{\pgfqpoint{1.284650in}{0.956914in}}%
\pgfpathlineto{\pgfqpoint{1.255323in}{0.974956in}}%
\pgfpathlineto{\pgfqpoint{1.230917in}{0.992608in}}%
\pgfpathlineto{\pgfqpoint{1.210564in}{1.009817in}}%
\pgfpathlineto{\pgfqpoint{1.193654in}{1.026535in}}%
\pgfpathlineto{\pgfqpoint{1.179838in}{1.042720in}}%
\pgfpathlineto{\pgfqpoint{1.169026in}{1.058332in}}%
\pgfpathlineto{\pgfqpoint{1.161385in}{1.073337in}}%
\pgfpathlineto{\pgfqpoint{1.157346in}{1.087707in}}%
\pgfpathlineto{\pgfqpoint{1.157081in}{1.101415in}}%
\pgfpathlineto{\pgfqpoint{1.159307in}{1.114436in}}%
\pgfpathlineto{\pgfqpoint{1.163724in}{1.126764in}}%
\pgfpathlineto{\pgfqpoint{1.170180in}{1.138394in}}%
\pgfpathlineto{\pgfqpoint{1.178571in}{1.149322in}}%
\pgfpathlineto{\pgfqpoint{1.188841in}{1.159546in}}%
\pgfpathlineto{\pgfqpoint{1.200983in}{1.169067in}}%
\pgfpathlineto{\pgfqpoint{1.215035in}{1.177886in}}%
\pgfpathlineto{\pgfqpoint{1.231087in}{1.186006in}}%
\pgfpathlineto{\pgfqpoint{1.258953in}{1.196880in}}%
\pgfpathlineto{\pgfqpoint{1.291248in}{1.206175in}}%
\pgfpathlineto{\pgfqpoint{1.328110in}{1.213879in}}%
\pgfpathlineto{\pgfqpoint{1.369777in}{1.219976in}}%
\pgfpathlineto{\pgfqpoint{1.416588in}{1.224446in}}%
\pgfpathlineto{\pgfqpoint{1.468979in}{1.227263in}}%
\pgfpathlineto{\pgfqpoint{1.527484in}{1.228398in}}%
\pgfpathlineto{\pgfqpoint{1.592714in}{1.227819in}}%
\pgfpathlineto{\pgfqpoint{1.690542in}{1.224371in}}%
\pgfpathlineto{\pgfqpoint{1.803621in}{1.217549in}}%
\pgfpathlineto{\pgfqpoint{1.934282in}{1.207001in}}%
\pgfpathlineto{\pgfqpoint{2.082824in}{1.192477in}}%
\pgfpathlineto{\pgfqpoint{2.247512in}{1.173832in}}%
\pgfpathlineto{\pgfqpoint{2.424577in}{1.151024in}}%
\pgfpathlineto{\pgfqpoint{2.562085in}{1.131218in}}%
\pgfpathlineto{\pgfqpoint{2.700081in}{1.109164in}}%
\pgfpathlineto{\pgfqpoint{2.833281in}{1.085025in}}%
\pgfpathlineto{\pgfqpoint{2.916874in}{1.067982in}}%
\pgfpathlineto{\pgfqpoint{2.995011in}{1.050340in}}%
\pgfpathlineto{\pgfqpoint{3.066871in}{1.032236in}}%
\pgfpathlineto{\pgfqpoint{3.131885in}{1.013801in}}%
\pgfpathlineto{\pgfqpoint{3.189741in}{0.995167in}}%
\pgfpathlineto{\pgfqpoint{3.240381in}{0.976461in}}%
\pgfpathlineto{\pgfqpoint{3.283998in}{0.957808in}}%
\pgfpathlineto{\pgfqpoint{3.320973in}{0.939336in}}%
\pgfpathlineto{\pgfqpoint{3.351596in}{0.921180in}}%
\pgfpathlineto{\pgfqpoint{3.376945in}{0.903409in}}%
\pgfpathlineto{\pgfqpoint{3.398013in}{0.886073in}}%
\pgfpathlineto{\pgfqpoint{3.415527in}{0.869219in}}%
\pgfpathlineto{\pgfqpoint{3.429949in}{0.852890in}}%
\pgfpathlineto{\pgfqpoint{3.441477in}{0.837124in}}%
\pgfpathlineto{\pgfqpoint{3.450042in}{0.821955in}}%
\pgfpathlineto{\pgfqpoint{3.455311in}{0.807413in}}%
\pgfpathlineto{\pgfqpoint{3.456706in}{0.793524in}}%
\pgfpathlineto{\pgfqpoint{3.454986in}{0.780316in}}%
\pgfpathlineto{\pgfqpoint{3.451021in}{0.767802in}}%
\pgfpathlineto{\pgfqpoint{3.444977in}{0.755987in}}%
\pgfpathlineto{\pgfqpoint{3.436972in}{0.744874in}}%
\pgfpathlineto{\pgfqpoint{3.427082in}{0.734467in}}%
\pgfpathlineto{\pgfqpoint{3.415333in}{0.724765in}}%
\pgfpathlineto{\pgfqpoint{3.401709in}{0.715767in}}%
\pgfpathlineto{\pgfqpoint{3.386147in}{0.707471in}}%
\pgfpathlineto{\pgfqpoint{3.358996in}{0.696335in}}%
\pgfpathlineto{\pgfqpoint{3.327391in}{0.686777in}}%
\pgfpathlineto{\pgfqpoint{3.291243in}{0.678806in}}%
\pgfpathlineto{\pgfqpoint{3.250329in}{0.672440in}}%
\pgfpathlineto{\pgfqpoint{3.204330in}{0.667699in}}%
\pgfpathlineto{\pgfqpoint{3.152829in}{0.664608in}}%
\pgfpathlineto{\pgfqpoint{3.095315in}{0.663197in}}%
\pgfpathlineto{\pgfqpoint{3.031180in}{0.663501in}}%
\pgfpathlineto{\pgfqpoint{2.934581in}{0.666638in}}%
\pgfpathlineto{\pgfqpoint{2.823411in}{0.673106in}}%
\pgfpathlineto{\pgfqpoint{2.695186in}{0.683206in}}%
\pgfpathlineto{\pgfqpoint{2.549232in}{0.697179in}}%
\pgfpathlineto{\pgfqpoint{2.386729in}{0.715209in}}%
\pgfpathlineto{\pgfqpoint{2.210711in}{0.737418in}}%
\pgfpathlineto{\pgfqpoint{2.072709in}{0.756858in}}%
\pgfpathlineto{\pgfqpoint{1.933711in}{0.778664in}}%
\pgfpathlineto{\pgfqpoint{1.799218in}{0.802536in}}%
\pgfpathlineto{\pgfqpoint{1.714367in}{0.819409in}}%
\pgfpathlineto{\pgfqpoint{1.634669in}{0.836908in}}%
\pgfpathlineto{\pgfqpoint{1.560994in}{0.854913in}}%
\pgfpathlineto{\pgfqpoint{1.494000in}{0.873296in}}%
\pgfpathlineto{\pgfqpoint{1.434126in}{0.891927in}}%
\pgfpathlineto{\pgfqpoint{1.381598in}{0.910674in}}%
\pgfpathlineto{\pgfqpoint{1.336422in}{0.929398in}}%
\pgfpathlineto{\pgfqpoint{1.298328in}{0.947950in}}%
\pgfpathlineto{\pgfqpoint{1.266387in}{0.966215in}}%
\pgfpathlineto{\pgfqpoint{1.239591in}{0.984125in}}%
\pgfpathlineto{\pgfqpoint{1.217165in}{1.001614in}}%
\pgfpathlineto{\pgfqpoint{1.198564in}{1.018628in}}%
\pgfpathlineto{\pgfqpoint{1.183471in}{1.035116in}}%
\pgfpathlineto{\pgfqpoint{1.171803in}{1.051034in}}%
\pgfpathlineto{\pgfqpoint{1.163706in}{1.066344in}}%
\pgfpathlineto{\pgfqpoint{1.159431in}{1.081017in}}%
\pgfpathlineto{\pgfqpoint{1.158078in}{1.095023in}}%
\pgfpathlineto{\pgfqpoint{1.159149in}{1.108349in}}%
\pgfpathlineto{\pgfqpoint{1.162433in}{1.120988in}}%
\pgfpathlineto{\pgfqpoint{1.167774in}{1.132932in}}%
\pgfpathlineto{\pgfqpoint{1.175079in}{1.144177in}}%
\pgfpathlineto{\pgfqpoint{1.184307in}{1.154722in}}%
\pgfpathlineto{\pgfqpoint{1.195481in}{1.164567in}}%
\pgfpathlineto{\pgfqpoint{1.208677in}{1.173714in}}%
\pgfpathlineto{\pgfqpoint{1.223924in}{1.182166in}}%
\pgfpathlineto{\pgfqpoint{1.250419in}{1.193532in}}%
\pgfpathlineto{\pgfqpoint{1.281269in}{1.203317in}}%
\pgfpathlineto{\pgfqpoint{1.316583in}{1.211510in}}%
\pgfpathlineto{\pgfqpoint{1.356586in}{1.218098in}}%
\pgfpathlineto{\pgfqpoint{1.401608in}{1.223069in}}%
\pgfpathlineto{\pgfqpoint{1.452097in}{1.226403in}}%
\pgfpathlineto{\pgfqpoint{1.508605in}{1.228084in}}%
\pgfpathlineto{\pgfqpoint{1.571272in}{1.228090in}}%
\pgfpathlineto{\pgfqpoint{1.640949in}{1.226340in}}%
\pgfpathlineto{\pgfqpoint{1.746825in}{1.221097in}}%
\pgfpathlineto{\pgfqpoint{1.869168in}{1.212318in}}%
\pgfpathlineto{\pgfqpoint{2.008644in}{1.199789in}}%
\pgfpathlineto{\pgfqpoint{2.164722in}{1.183311in}}%
\pgfpathlineto{\pgfqpoint{2.335672in}{1.162701in}}%
\pgfpathlineto{\pgfqpoint{2.471891in}{1.144428in}}%
\pgfpathlineto{\pgfqpoint{2.611378in}{1.123742in}}%
\pgfpathlineto{\pgfqpoint{2.748560in}{1.100882in}}%
\pgfpathlineto{\pgfqpoint{2.878603in}{1.076160in}}%
\pgfpathlineto{\pgfqpoint{2.959374in}{1.058822in}}%
\pgfpathlineto{\pgfqpoint{3.034312in}{1.040933in}}%
\pgfpathlineto{\pgfqpoint{3.102689in}{1.022624in}}%
\pgfpathlineto{\pgfqpoint{3.163969in}{1.004031in}}%
\pgfpathlineto{\pgfqpoint{3.217802in}{0.985301in}}%
\pgfpathlineto{\pgfqpoint{3.264140in}{0.966594in}}%
\pgfpathlineto{\pgfqpoint{3.303749in}{0.948036in}}%
\pgfpathlineto{\pgfqpoint{3.337525in}{0.929719in}}%
\pgfpathlineto{\pgfqpoint{3.366184in}{0.911726in}}%
\pgfpathlineto{\pgfqpoint{3.390259in}{0.894130in}}%
\pgfpathlineto{\pgfqpoint{3.410098in}{0.876998in}}%
\pgfpathlineto{\pgfqpoint{3.425869in}{0.860387in}}%
\pgfpathlineto{\pgfqpoint{3.437556in}{0.844346in}}%
\pgfpathlineto{\pgfqpoint{3.445336in}{0.828917in}}%
\pgfpathlineto{\pgfqpoint{3.450107in}{0.814128in}}%
\pgfpathlineto{\pgfqpoint{3.452232in}{0.799997in}}%
\pgfpathlineto{\pgfqpoint{3.451971in}{0.786538in}}%
\pgfpathlineto{\pgfqpoint{3.449510in}{0.773760in}}%
\pgfpathlineto{\pgfqpoint{3.444957in}{0.761672in}}%
\pgfpathlineto{\pgfqpoint{3.438348in}{0.750277in}}%
\pgfpathlineto{\pgfqpoint{3.429642in}{0.739576in}}%
\pgfpathlineto{\pgfqpoint{3.418778in}{0.729568in}}%
\pgfpathlineto{\pgfqpoint{3.405922in}{0.720256in}}%
\pgfpathlineto{\pgfqpoint{3.391149in}{0.711644in}}%
\pgfpathlineto{\pgfqpoint{3.365457in}{0.700042in}}%
\pgfpathlineto{\pgfqpoint{3.335511in}{0.690025in}}%
\pgfpathlineto{\pgfqpoint{3.301174in}{0.681598in}}%
\pgfpathlineto{\pgfqpoint{3.262173in}{0.674767in}}%
\pgfpathlineto{\pgfqpoint{3.218101in}{0.669536in}}%
\pgfpathlineto{\pgfqpoint{3.168520in}{0.665912in}}%
\pgfpathlineto{\pgfqpoint{3.113274in}{0.663934in}}%
\pgfpathlineto{\pgfqpoint{3.051578in}{0.663657in}}%
\pgfpathlineto{\pgfqpoint{2.982639in}{0.665144in}}%
\pgfpathlineto{\pgfqpoint{2.878269in}{0.669996in}}%
\pgfpathlineto{\pgfqpoint{2.758433in}{0.678309in}}%
\pgfpathlineto{\pgfqpoint{2.622080in}{0.690306in}}%
\pgfpathlineto{\pgfqpoint{2.468466in}{0.706239in}}%
\pgfpathlineto{\pgfqpoint{2.296147in}{0.726422in}}%
\pgfpathlineto{\pgfqpoint{2.158982in}{0.744395in}}%
\pgfpathlineto{\pgfqpoint{2.020352in}{0.764696in}}%
\pgfpathlineto{\pgfqpoint{1.884504in}{0.787154in}}%
\pgfpathlineto{\pgfqpoint{1.755175in}{0.811523in}}%
\pgfpathlineto{\pgfqpoint{1.674195in}{0.828676in}}%
\pgfpathlineto{\pgfqpoint{1.598390in}{0.846428in}}%
\pgfpathlineto{\pgfqpoint{1.528492in}{0.864650in}}%
\pgfpathlineto{\pgfqpoint{1.465133in}{0.883202in}}%
\pgfpathlineto{\pgfqpoint{1.408842in}{0.901928in}}%
\pgfpathlineto{\pgfqpoint{1.360052in}{0.920658in}}%
\pgfpathlineto{\pgfqpoint{1.318466in}{0.939253in}}%
\pgfpathlineto{\pgfqpoint{1.283084in}{0.957631in}}%
\pgfpathlineto{\pgfqpoint{1.253328in}{0.975698in}}%
\pgfpathlineto{\pgfqpoint{1.228675in}{0.993371in}}%
\pgfpathlineto{\pgfqpoint{1.208648in}{1.010582in}}%
\pgfpathlineto{\pgfqpoint{1.192794in}{1.027272in}}%
\pgfpathlineto{\pgfqpoint{1.180642in}{1.043400in}}%
\pgfpathlineto{\pgfqpoint{1.171804in}{1.058929in}}%
\pgfpathlineto{\pgfqpoint{1.165965in}{1.073832in}}%
\pgfpathlineto{\pgfqpoint{1.162882in}{1.088087in}}%
\pgfpathlineto{\pgfqpoint{1.162403in}{1.101678in}}%
\pgfpathlineto{\pgfqpoint{1.164418in}{1.114596in}}%
\pgfpathlineto{\pgfqpoint{1.168683in}{1.126830in}}%
\pgfpathlineto{\pgfqpoint{1.174997in}{1.138375in}}%
\pgfpathlineto{\pgfqpoint{1.183217in}{1.149225in}}%
\pgfpathlineto{\pgfqpoint{1.193255in}{1.159378in}}%
\pgfpathlineto{\pgfqpoint{1.205085in}{1.168832in}}%
\pgfpathlineto{\pgfqpoint{1.218733in}{1.177588in}}%
\pgfpathlineto{\pgfqpoint{1.234289in}{1.185650in}}%
\pgfpathlineto{\pgfqpoint{1.251895in}{1.193021in}}%
\pgfpathlineto{\pgfqpoint{1.282435in}{1.202792in}}%
\pgfpathlineto{\pgfqpoint{1.317606in}{1.210993in}}%
\pgfpathlineto{\pgfqpoint{1.357535in}{1.217605in}}%
\pgfpathlineto{\pgfqpoint{1.402526in}{1.222607in}}%
\pgfpathlineto{\pgfqpoint{1.452968in}{1.225972in}}%
\pgfpathlineto{\pgfqpoint{1.509336in}{1.227664in}}%
\pgfpathlineto{\pgfqpoint{1.572191in}{1.227643in}}%
\pgfpathlineto{\pgfqpoint{1.667205in}{1.224869in}}%
\pgfpathlineto{\pgfqpoint{1.776354in}{1.218807in}}%
\pgfpathlineto{\pgfqpoint{1.901834in}{1.209194in}}%
\pgfpathlineto{\pgfqpoint{2.044788in}{1.195781in}}%
\pgfpathlineto{\pgfqpoint{2.204665in}{1.178351in}}%
\pgfpathlineto{\pgfqpoint{2.379225in}{1.156721in}}%
\pgfpathlineto{\pgfqpoint{2.517264in}{1.137653in}}%
\pgfpathlineto{\pgfqpoint{2.656324in}{1.116271in}}%
\pgfpathlineto{\pgfqpoint{2.791303in}{1.092819in}}%
\pgfpathlineto{\pgfqpoint{2.876860in}{1.076185in}}%
\pgfpathlineto{\pgfqpoint{2.957586in}{1.058875in}}%
\pgfpathlineto{\pgfqpoint{3.032554in}{1.041006in}}%
\pgfpathlineto{\pgfqpoint{3.100995in}{1.022711in}}%
\pgfpathlineto{\pgfqpoint{3.162293in}{1.004134in}}%
\pgfpathlineto{\pgfqpoint{3.216012in}{0.985430in}}%
\pgfpathlineto{\pgfqpoint{3.262422in}{0.966751in}}%
\pgfpathlineto{\pgfqpoint{3.302348in}{0.948208in}}%
\pgfpathlineto{\pgfqpoint{3.336506in}{0.929897in}}%
\pgfpathlineto{\pgfqpoint{3.365471in}{0.911907in}}%
\pgfpathlineto{\pgfqpoint{3.389675in}{0.894316in}}%
\pgfpathlineto{\pgfqpoint{3.409412in}{0.877190in}}%
\pgfpathlineto{\pgfqpoint{3.424832in}{0.860590in}}%
\pgfpathlineto{\pgfqpoint{3.436076in}{0.844564in}}%
\pgfpathlineto{\pgfqpoint{3.443927in}{0.829151in}}%
\pgfpathlineto{\pgfqpoint{3.448858in}{0.814374in}}%
\pgfpathlineto{\pgfqpoint{3.451187in}{0.800250in}}%
\pgfpathlineto{\pgfqpoint{3.451146in}{0.786796in}}%
\pgfpathlineto{\pgfqpoint{3.448884in}{0.774020in}}%
\pgfpathlineto{\pgfqpoint{3.444463in}{0.761931in}}%
\pgfpathlineto{\pgfqpoint{3.437863in}{0.750532in}}%
\pgfpathlineto{\pgfqpoint{3.429036in}{0.739824in}}%
\pgfpathlineto{\pgfqpoint{3.418171in}{0.729810in}}%
\pgfpathlineto{\pgfqpoint{3.405367in}{0.720495in}}%
\pgfpathlineto{\pgfqpoint{3.390682in}{0.711880in}}%
\pgfpathlineto{\pgfqpoint{3.365173in}{0.700274in}}%
\pgfpathlineto{\pgfqpoint{3.335442in}{0.690253in}}%
\pgfpathlineto{\pgfqpoint{3.301301in}{0.681818in}}%
\pgfpathlineto{\pgfqpoint{3.262410in}{0.674969in}}%
\pgfpathlineto{\pgfqpoint{3.218317in}{0.669705in}}%
\pgfpathlineto{\pgfqpoint{3.168907in}{0.666053in}}%
\pgfpathlineto{\pgfqpoint{3.113662in}{0.664056in}}%
\pgfpathlineto{\pgfqpoint{3.051912in}{0.663762in}}%
\pgfpathlineto{\pgfqpoint{2.983006in}{0.665231in}}%
\pgfpathlineto{\pgfqpoint{2.878911in}{0.670052in}}%
\pgfpathlineto{\pgfqpoint{2.759533in}{0.678325in}}%
\pgfpathlineto{\pgfqpoint{2.623490in}{0.690282in}}%
\pgfpathlineto{\pgfqpoint{2.469715in}{0.706156in}}%
\pgfpathlineto{\pgfqpoint{2.299769in}{0.726208in}}%
\pgfpathlineto{\pgfqpoint{2.164530in}{0.744073in}}%
\pgfpathlineto{\pgfqpoint{2.026010in}{0.764306in}}%
\pgfpathlineto{\pgfqpoint{1.888247in}{0.786751in}}%
\pgfpathlineto{\pgfqpoint{1.756187in}{0.811160in}}%
\pgfpathlineto{\pgfqpoint{1.674050in}{0.828367in}}%
\pgfpathlineto{\pgfqpoint{1.597893in}{0.846174in}}%
\pgfpathlineto{\pgfqpoint{1.528252in}{0.864416in}}%
\pgfpathlineto{\pgfqpoint{1.465475in}{0.882940in}}%
\pgfpathlineto{\pgfqpoint{1.409730in}{0.901605in}}%
\pgfpathlineto{\pgfqpoint{1.360996in}{0.920283in}}%
\pgfpathlineto{\pgfqpoint{1.319069in}{0.938857in}}%
\pgfpathlineto{\pgfqpoint{1.283558in}{0.957220in}}%
\pgfpathlineto{\pgfqpoint{1.253891in}{0.975281in}}%
\pgfpathlineto{\pgfqpoint{1.229308in}{0.992958in}}%
\pgfpathlineto{\pgfqpoint{1.209540in}{1.010164in}}%
\pgfpathlineto{\pgfqpoint{1.194127in}{1.026843in}}%
\pgfpathlineto{\pgfqpoint{1.182229in}{1.042962in}}%
\pgfpathlineto{\pgfqpoint{1.173211in}{1.058494in}}%
\pgfpathlineto{\pgfqpoint{1.166638in}{1.073413in}}%
\pgfpathlineto{\pgfqpoint{1.162278in}{1.087701in}}%
\pgfpathlineto{\pgfqpoint{1.160106in}{1.101342in}}%
\pgfpathlineto{\pgfqpoint{1.160295in}{1.114323in}}%
\pgfpathlineto{\pgfqpoint{1.163225in}{1.126637in}}%
\pgfpathlineto{\pgfqpoint{1.169362in}{1.138278in}}%
\pgfpathlineto{\pgfqpoint{1.177877in}{1.149226in}}%
\pgfpathlineto{\pgfqpoint{1.188364in}{1.159470in}}%
\pgfpathlineto{\pgfqpoint{1.200748in}{1.169010in}}%
\pgfpathlineto{\pgfqpoint{1.214985in}{1.177845in}}%
\pgfpathlineto{\pgfqpoint{1.231062in}{1.185974in}}%
\pgfpathlineto{\pgfqpoint{1.258680in}{1.196844in}}%
\pgfpathlineto{\pgfqpoint{1.290683in}{1.206128in}}%
\pgfpathlineto{\pgfqpoint{1.327420in}{1.213833in}}%
\pgfpathlineto{\pgfqpoint{1.369055in}{1.219950in}}%
\pgfpathlineto{\pgfqpoint{1.415836in}{1.224453in}}%
\pgfpathlineto{\pgfqpoint{1.468248in}{1.227312in}}%
\pgfpathlineto{\pgfqpoint{1.526819in}{1.228484in}}%
\pgfpathlineto{\pgfqpoint{1.592127in}{1.227920in}}%
\pgfpathlineto{\pgfqpoint{1.690769in}{1.224356in}}%
\pgfpathlineto{\pgfqpoint{1.804108in}{1.217419in}}%
\pgfpathlineto{\pgfqpoint{1.933812in}{1.206898in}}%
\pgfpathlineto{\pgfqpoint{2.081002in}{1.192524in}}%
\pgfpathlineto{\pgfqpoint{2.245169in}{1.174058in}}%
\pgfpathlineto{\pgfqpoint{2.423231in}{1.151341in}}%
\pgfpathlineto{\pgfqpoint{2.561922in}{1.131516in}}%
\pgfpathlineto{\pgfqpoint{2.700322in}{1.109431in}}%
\pgfpathlineto{\pgfqpoint{2.833756in}{1.085318in}}%
\pgfpathlineto{\pgfqpoint{2.917470in}{1.068278in}}%
\pgfpathlineto{\pgfqpoint{2.995290in}{1.050595in}}%
\pgfpathlineto{\pgfqpoint{3.066630in}{1.032428in}}%
\pgfpathlineto{\pgfqpoint{3.131178in}{1.013938in}}%
\pgfpathlineto{\pgfqpoint{3.188777in}{0.995271in}}%
\pgfpathlineto{\pgfqpoint{3.239421in}{0.976563in}}%
\pgfpathlineto{\pgfqpoint{3.283258in}{0.957934in}}%
\pgfpathlineto{\pgfqpoint{3.320589in}{0.939492in}}%
\pgfpathlineto{\pgfqpoint{3.351868in}{0.921333in}}%
\pgfpathlineto{\pgfqpoint{3.377701in}{0.903538in}}%
\pgfpathlineto{\pgfqpoint{3.398840in}{0.886178in}}%
\pgfpathlineto{\pgfqpoint{3.415488in}{0.869330in}}%
\pgfpathlineto{\pgfqpoint{3.428122in}{0.853042in}}%
\pgfpathlineto{\pgfqpoint{3.437511in}{0.837342in}}%
\pgfpathlineto{\pgfqpoint{3.444237in}{0.822254in}}%
\pgfpathlineto{\pgfqpoint{3.448695in}{0.807797in}}%
\pgfpathlineto{\pgfqpoint{3.451093in}{0.793988in}}%
\pgfpathlineto{\pgfqpoint{3.451454in}{0.780837in}}%
\pgfpathlineto{\pgfqpoint{3.449609in}{0.768354in}}%
\pgfpathlineto{\pgfqpoint{3.445208in}{0.756542in}}%
\pgfpathlineto{\pgfqpoint{3.437740in}{0.745403in}}%
\pgfpathlineto{\pgfqpoint{3.427795in}{0.734955in}}%
\pgfpathlineto{\pgfqpoint{3.415904in}{0.725211in}}%
\pgfpathlineto{\pgfqpoint{3.402125in}{0.716172in}}%
\pgfpathlineto{\pgfqpoint{3.386487in}{0.707838in}}%
\pgfpathlineto{\pgfqpoint{3.359545in}{0.696663in}}%
\pgfpathlineto{\pgfqpoint{3.328308in}{0.687078in}}%
\pgfpathlineto{\pgfqpoint{3.292526in}{0.679083in}}%
\pgfpathlineto{\pgfqpoint{3.251814in}{0.672675in}}%
\pgfpathlineto{\pgfqpoint{3.206023in}{0.667870in}}%
\pgfpathlineto{\pgfqpoint{3.154749in}{0.664701in}}%
\pgfpathlineto{\pgfqpoint{3.097403in}{0.663207in}}%
\pgfpathlineto{\pgfqpoint{3.033380in}{0.663438in}}%
\pgfpathlineto{\pgfqpoint{2.962060in}{0.665455in}}%
\pgfpathlineto{\pgfqpoint{2.854517in}{0.671051in}}%
\pgfpathlineto{\pgfqpoint{2.731326in}{0.680153in}}%
\pgfpathlineto{\pgfqpoint{2.591012in}{0.692995in}}%
\pgfpathlineto{\pgfqpoint{2.433216in}{0.709844in}}%
\pgfpathlineto{\pgfqpoint{2.259904in}{0.730913in}}%
\pgfpathlineto{\pgfqpoint{2.122863in}{0.749511in}}%
\pgfpathlineto{\pgfqpoint{1.983637in}{0.770449in}}%
\pgfpathlineto{\pgfqpoint{1.846851in}{0.793573in}}%
\pgfpathlineto{\pgfqpoint{1.759620in}{0.810043in}}%
\pgfpathlineto{\pgfqpoint{1.677094in}{0.827225in}}%
\pgfpathlineto{\pgfqpoint{1.600311in}{0.844995in}}%
\pgfpathlineto{\pgfqpoint{1.530060in}{0.863220in}}%
\pgfpathlineto{\pgfqpoint{1.466879in}{0.881761in}}%
\pgfpathlineto{\pgfqpoint{1.411101in}{0.900465in}}%
\pgfpathlineto{\pgfqpoint{1.362734in}{0.919178in}}%
\pgfpathlineto{\pgfqpoint{1.320987in}{0.937788in}}%
\pgfpathlineto{\pgfqpoint{1.285118in}{0.956195in}}%
\pgfpathlineto{\pgfqpoint{1.254535in}{0.974308in}}%
\pgfpathlineto{\pgfqpoint{1.228792in}{0.992047in}}%
\pgfpathlineto{\pgfqpoint{1.207595in}{1.009340in}}%
\pgfpathlineto{\pgfqpoint{1.190798in}{1.026125in}}%
\pgfpathlineto{\pgfqpoint{1.178383in}{1.042349in}}%
\pgfpathlineto{\pgfqpoint{1.169681in}{1.057966in}}%
\pgfpathlineto{\pgfqpoint{1.164016in}{1.072952in}}%
\pgfpathlineto{\pgfqpoint{1.161024in}{1.087286in}}%
\pgfpathlineto{\pgfqpoint{1.160437in}{1.100954in}}%
\pgfpathlineto{\pgfqpoint{1.162071in}{1.113944in}}%
\pgfpathlineto{\pgfqpoint{1.165833in}{1.126249in}}%
\pgfpathlineto{\pgfqpoint{1.171721in}{1.137863in}}%
\pgfpathlineto{\pgfqpoint{1.179821in}{1.148786in}}%
\pgfpathlineto{\pgfqpoint{1.190110in}{1.159016in}}%
\pgfpathlineto{\pgfqpoint{1.202370in}{1.168549in}}%
\pgfpathlineto{\pgfqpoint{1.216538in}{1.177381in}}%
\pgfpathlineto{\pgfqpoint{1.232577in}{1.185512in}}%
\pgfpathlineto{\pgfqpoint{1.260136in}{1.196388in}}%
\pgfpathlineto{\pgfqpoint{1.291990in}{1.205676in}}%
\pgfpathlineto{\pgfqpoint{1.328374in}{1.213374in}}%
\pgfpathlineto{\pgfqpoint{1.369666in}{1.219482in}}%
\pgfpathlineto{\pgfqpoint{1.416277in}{1.223995in}}%
\pgfpathlineto{\pgfqpoint{1.468349in}{1.226878in}}%
\pgfpathlineto{\pgfqpoint{1.526549in}{1.228085in}}%
\pgfpathlineto{\pgfqpoint{1.591577in}{1.227562in}}%
\pgfpathlineto{\pgfqpoint{1.690040in}{1.224060in}}%
\pgfpathlineto{\pgfqpoint{1.803273in}{1.217185in}}%
\pgfpathlineto{\pgfqpoint{1.932609in}{1.206722in}}%
\pgfpathlineto{\pgfqpoint{2.079299in}{1.192424in}}%
\pgfpathlineto{\pgfqpoint{2.244114in}{1.174085in}}%
\pgfpathlineto{\pgfqpoint{2.376785in}{1.157524in}}%
\pgfpathlineto{\pgfqpoint{2.513833in}{1.138534in}}%
\pgfpathlineto{\pgfqpoint{2.651782in}{1.117210in}}%
\pgfpathlineto{\pgfqpoint{2.786757in}{1.093754in}}%
\pgfpathlineto{\pgfqpoint{2.872995in}{1.077084in}}%
\pgfpathlineto{\pgfqpoint{2.954658in}{1.059722in}}%
\pgfpathlineto{\pgfqpoint{3.030305in}{1.041811in}}%
\pgfpathlineto{\pgfqpoint{3.098578in}{1.023504in}}%
\pgfpathlineto{\pgfqpoint{3.159376in}{1.004938in}}%
\pgfpathlineto{\pgfqpoint{3.213127in}{0.986250in}}%
\pgfpathlineto{\pgfqpoint{3.260237in}{0.967567in}}%
\pgfpathlineto{\pgfqpoint{3.301092in}{0.949003in}}%
\pgfpathlineto{\pgfqpoint{3.336062in}{0.930665in}}%
\pgfpathlineto{\pgfqpoint{3.365496in}{0.912644in}}%
\pgfpathlineto{\pgfqpoint{3.389726in}{0.895024in}}%
\pgfpathlineto{\pgfqpoint{3.409066in}{0.877873in}}%
\pgfpathlineto{\pgfqpoint{3.423930in}{0.861256in}}%
\pgfpathlineto{\pgfqpoint{3.435037in}{0.845220in}}%
\pgfpathlineto{\pgfqpoint{3.442962in}{0.829790in}}%
\pgfpathlineto{\pgfqpoint{3.448146in}{0.814992in}}%
\pgfpathlineto{\pgfqpoint{3.450900in}{0.800843in}}%
\pgfpathlineto{\pgfqpoint{3.451406in}{0.787359in}}%
\pgfpathlineto{\pgfqpoint{3.449713in}{0.774549in}}%
\pgfpathlineto{\pgfqpoint{3.445745in}{0.762421in}}%
\pgfpathlineto{\pgfqpoint{3.439291in}{0.750975in}}%
\pgfpathlineto{\pgfqpoint{3.430457in}{0.740218in}}%
\pgfpathlineto{\pgfqpoint{3.419629in}{0.730159in}}%
\pgfpathlineto{\pgfqpoint{3.406886in}{0.720801in}}%
\pgfpathlineto{\pgfqpoint{3.392275in}{0.712145in}}%
\pgfpathlineto{\pgfqpoint{3.366886in}{0.700481in}}%
\pgfpathlineto{\pgfqpoint{3.337261in}{0.690403in}}%
\pgfpathlineto{\pgfqpoint{3.303181in}{0.681908in}}%
\pgfpathlineto{\pgfqpoint{3.264276in}{0.674995in}}%
\pgfpathlineto{\pgfqpoint{3.220270in}{0.669670in}}%
\pgfpathlineto{\pgfqpoint{3.170963in}{0.665966in}}%
\pgfpathlineto{\pgfqpoint{3.115780in}{0.663919in}}%
\pgfpathlineto{\pgfqpoint{3.054126in}{0.663577in}}%
\pgfpathlineto{\pgfqpoint{2.985389in}{0.664998in}}%
\pgfpathlineto{\pgfqpoint{2.881635in}{0.669753in}}%
\pgfpathlineto{\pgfqpoint{2.762634in}{0.677956in}}%
\pgfpathlineto{\pgfqpoint{2.626838in}{0.689834in}}%
\pgfpathlineto{\pgfqpoint{2.473519in}{0.705646in}}%
\pgfpathlineto{\pgfqpoint{2.303913in}{0.725633in}}%
\pgfpathlineto{\pgfqpoint{2.168611in}{0.743418in}}%
\pgfpathlineto{\pgfqpoint{2.029817in}{0.763567in}}%
\pgfpathlineto{\pgfqpoint{1.891767in}{0.785971in}}%
\pgfpathlineto{\pgfqpoint{1.759709in}{0.810345in}}%
\pgfpathlineto{\pgfqpoint{1.677264in}{0.827509in}}%
\pgfpathlineto{\pgfqpoint{1.600534in}{0.845262in}}%
\pgfpathlineto{\pgfqpoint{1.530322in}{0.863473in}}%
\pgfpathlineto{\pgfqpoint{1.467191in}{0.881999in}}%
\pgfpathlineto{\pgfqpoint{1.411550in}{0.900681in}}%
\pgfpathlineto{\pgfqpoint{1.363196in}{0.919376in}}%
\pgfpathlineto{\pgfqpoint{1.321352in}{0.937971in}}%
\pgfpathlineto{\pgfqpoint{1.285355in}{0.956366in}}%
\pgfpathlineto{\pgfqpoint{1.254670in}{0.974469in}}%
\pgfpathlineto{\pgfqpoint{1.228897in}{0.992198in}}%
\pgfpathlineto{\pgfqpoint{1.207762in}{1.009480in}}%
\pgfpathlineto{\pgfqpoint{1.191126in}{1.026252in}}%
\pgfpathlineto{\pgfqpoint{1.178872in}{1.042461in}}%
\pgfpathlineto{\pgfqpoint{1.170175in}{1.058063in}}%
\pgfpathlineto{\pgfqpoint{1.164484in}{1.073036in}}%
\pgfpathlineto{\pgfqpoint{1.161444in}{1.087359in}}%
\pgfpathlineto{\pgfqpoint{1.160790in}{1.101017in}}%
\pgfpathlineto{\pgfqpoint{1.162355in}{1.113998in}}%
\pgfpathlineto{\pgfqpoint{1.166065in}{1.126294in}}%
\pgfpathlineto{\pgfqpoint{1.171941in}{1.137901in}}%
\pgfpathlineto{\pgfqpoint{1.180094in}{1.148819in}}%
\pgfpathlineto{\pgfqpoint{1.190395in}{1.159044in}}%
\pgfpathlineto{\pgfqpoint{1.202657in}{1.168570in}}%
\pgfpathlineto{\pgfqpoint{1.216818in}{1.177397in}}%
\pgfpathlineto{\pgfqpoint{1.232845in}{1.185521in}}%
\pgfpathlineto{\pgfqpoint{1.260380in}{1.196388in}}%
\pgfpathlineto{\pgfqpoint{1.292213in}{1.205668in}}%
\pgfpathlineto{\pgfqpoint{1.328595in}{1.213360in}}%
\pgfpathlineto{\pgfqpoint{1.369923in}{1.219464in}}%
\pgfpathlineto{\pgfqpoint{1.416537in}{1.223975in}}%
\pgfpathlineto{\pgfqpoint{1.468633in}{1.226852in}}%
\pgfpathlineto{\pgfqpoint{1.526881in}{1.228052in}}%
\pgfpathlineto{\pgfqpoint{1.591949in}{1.227522in}}%
\pgfpathlineto{\pgfqpoint{1.690424in}{1.224012in}}%
\pgfpathlineto{\pgfqpoint{1.803630in}{1.217129in}}%
\pgfpathlineto{\pgfqpoint{1.932959in}{1.206657in}}%
\pgfpathlineto{\pgfqpoint{2.079727in}{1.192355in}}%
\pgfpathlineto{\pgfqpoint{2.243830in}{1.173987in}}%
\pgfpathlineto{\pgfqpoint{2.421422in}{1.151329in}}%
\pgfpathlineto{\pgfqpoint{2.559552in}{1.131539in}}%
\pgfpathlineto{\pgfqpoint{2.697766in}{1.109495in}}%
\pgfpathlineto{\pgfqpoint{2.831412in}{1.085432in}}%
\pgfpathlineto{\pgfqpoint{2.915292in}{1.068429in}}%
\pgfpathlineto{\pgfqpoint{2.993240in}{1.050782in}}%
\pgfpathlineto{\pgfqpoint{3.064686in}{1.032646in}}%
\pgfpathlineto{\pgfqpoint{3.129343in}{1.014180in}}%
\pgfpathlineto{\pgfqpoint{3.187067in}{0.995533in}}%
\pgfpathlineto{\pgfqpoint{3.237863in}{0.976839in}}%
\pgfpathlineto{\pgfqpoint{3.281877in}{0.958219in}}%
\pgfpathlineto{\pgfqpoint{3.319403in}{0.939782in}}%
\pgfpathlineto{\pgfqpoint{3.350881in}{0.921625in}}%
\pgfpathlineto{\pgfqpoint{3.376894in}{0.903831in}}%
\pgfpathlineto{\pgfqpoint{3.398165in}{0.886469in}}%
\pgfpathlineto{\pgfqpoint{3.414900in}{0.869618in}}%
\pgfpathlineto{\pgfqpoint{3.427595in}{0.853327in}}%
\pgfpathlineto{\pgfqpoint{3.437036in}{0.837623in}}%
\pgfpathlineto{\pgfqpoint{3.443820in}{0.822529in}}%
\pgfpathlineto{\pgfqpoint{3.448354in}{0.808064in}}%
\pgfpathlineto{\pgfqpoint{3.450853in}{0.794246in}}%
\pgfpathlineto{\pgfqpoint{3.451340in}{0.781085in}}%
\pgfpathlineto{\pgfqpoint{3.449650in}{0.768590in}}%
\pgfpathlineto{\pgfqpoint{3.445426in}{0.756765in}}%
\pgfpathlineto{\pgfqpoint{3.438130in}{0.745611in}}%
\pgfpathlineto{\pgfqpoint{3.428219in}{0.735147in}}%
\pgfpathlineto{\pgfqpoint{3.416359in}{0.725385in}}%
\pgfpathlineto{\pgfqpoint{3.402608in}{0.716329in}}%
\pgfpathlineto{\pgfqpoint{3.386997in}{0.707979in}}%
\pgfpathlineto{\pgfqpoint{3.360093in}{0.696779in}}%
\pgfpathlineto{\pgfqpoint{3.328900in}{0.687170in}}%
\pgfpathlineto{\pgfqpoint{3.293170in}{0.679151in}}%
\pgfpathlineto{\pgfqpoint{3.252524in}{0.672721in}}%
\pgfpathlineto{\pgfqpoint{3.206794in}{0.667894in}}%
\pgfpathlineto{\pgfqpoint{3.155597in}{0.664702in}}%
\pgfpathlineto{\pgfqpoint{3.098336in}{0.663184in}}%
\pgfpathlineto{\pgfqpoint{3.034402in}{0.663390in}}%
\pgfpathlineto{\pgfqpoint{2.963175in}{0.665383in}}%
\pgfpathlineto{\pgfqpoint{2.855763in}{0.670944in}}%
\pgfpathlineto{\pgfqpoint{2.732722in}{0.680010in}}%
\pgfpathlineto{\pgfqpoint{2.592574in}{0.692814in}}%
\pgfpathlineto{\pgfqpoint{2.434938in}{0.709621in}}%
\pgfpathlineto{\pgfqpoint{2.261744in}{0.730650in}}%
\pgfpathlineto{\pgfqpoint{2.124743in}{0.749218in}}%
\pgfpathlineto{\pgfqpoint{1.985527in}{0.770124in}}%
\pgfpathlineto{\pgfqpoint{1.848662in}{0.793225in}}%
\pgfpathlineto{\pgfqpoint{1.761323in}{0.809685in}}%
\pgfpathlineto{\pgfqpoint{1.678674in}{0.826858in}}%
\pgfpathlineto{\pgfqpoint{1.601765in}{0.844617in}}%
\pgfpathlineto{\pgfqpoint{1.531379in}{0.862833in}}%
\pgfpathlineto{\pgfqpoint{1.468037in}{0.881366in}}%
\pgfpathlineto{\pgfqpoint{1.411995in}{0.900073in}}%
\pgfpathlineto{\pgfqpoint{1.363392in}{0.918795in}}%
\pgfpathlineto{\pgfqpoint{1.321627in}{0.937408in}}%
\pgfpathlineto{\pgfqpoint{1.285835in}{0.955816in}}%
\pgfpathlineto{\pgfqpoint{1.255323in}{0.973931in}}%
\pgfpathlineto{\pgfqpoint{1.229569in}{0.991674in}}%
\pgfpathlineto{\pgfqpoint{1.208226in}{1.008975in}}%
\pgfpathlineto{\pgfqpoint{1.191118in}{1.025772in}}%
\pgfpathlineto{\pgfqpoint{1.178244in}{1.042013in}}%
\pgfpathlineto{\pgfqpoint{1.169422in}{1.057651in}}%
\pgfpathlineto{\pgfqpoint{1.163718in}{1.072657in}}%
\pgfpathlineto{\pgfqpoint{1.160736in}{1.087010in}}%
\pgfpathlineto{\pgfqpoint{1.160190in}{1.100695in}}%
\pgfpathlineto{\pgfqpoint{1.161878in}{1.113703in}}%
\pgfpathlineto{\pgfqpoint{1.165679in}{1.126024in}}%
\pgfpathlineto{\pgfqpoint{1.171552in}{1.137653in}}%
\pgfpathlineto{\pgfqpoint{1.179540in}{1.148588in}}%
\pgfpathlineto{\pgfqpoint{1.189729in}{1.158832in}}%
\pgfpathlineto{\pgfqpoint{1.201944in}{1.168379in}}%
\pgfpathlineto{\pgfqpoint{1.216085in}{1.177228in}}%
\pgfpathlineto{\pgfqpoint{1.232111in}{1.185375in}}%
\pgfpathlineto{\pgfqpoint{1.259668in}{1.196276in}}%
\pgfpathlineto{\pgfqpoint{1.291520in}{1.205589in}}%
\pgfpathlineto{\pgfqpoint{1.327877in}{1.213310in}}%
\pgfpathlineto{\pgfqpoint{1.369082in}{1.219434in}}%
\pgfpathlineto{\pgfqpoint{1.415608in}{1.223961in}}%
\pgfpathlineto{\pgfqpoint{1.467617in}{1.226864in}}%
\pgfpathlineto{\pgfqpoint{1.525661in}{1.228095in}}%
\pgfpathlineto{\pgfqpoint{1.590530in}{1.227597in}}%
\pgfpathlineto{\pgfqpoint{1.688857in}{1.224129in}}%
\pgfpathlineto{\pgfqpoint{1.802047in}{1.217287in}}%
\pgfpathlineto{\pgfqpoint{1.931333in}{1.206858in}}%
\pgfpathlineto{\pgfqpoint{2.077701in}{1.192599in}}%
\pgfpathlineto{\pgfqpoint{2.243667in}{1.174154in}}%
\pgfpathlineto{\pgfqpoint{2.379137in}{1.157428in}}%
\pgfpathlineto{\pgfqpoint{2.518140in}{1.138325in}}%
\pgfpathlineto{\pgfqpoint{2.655992in}{1.117005in}}%
\pgfpathlineto{\pgfqpoint{2.788671in}{1.093682in}}%
\pgfpathlineto{\pgfqpoint{2.872563in}{1.077152in}}%
\pgfpathlineto{\pgfqpoint{2.951818in}{1.059948in}}%
\pgfpathlineto{\pgfqpoint{3.025735in}{1.042174in}}%
\pgfpathlineto{\pgfqpoint{3.093744in}{1.023948in}}%
\pgfpathlineto{\pgfqpoint{3.155406in}{1.005399in}}%
\pgfpathlineto{\pgfqpoint{3.210413in}{0.986666in}}%
\pgfpathlineto{\pgfqpoint{3.258587in}{0.967901in}}%
\pgfpathlineto{\pgfqpoint{3.299882in}{0.949266in}}%
\pgfpathlineto{\pgfqpoint{3.334456in}{0.930925in}}%
\pgfpathlineto{\pgfqpoint{3.363296in}{0.912929in}}%
\pgfpathlineto{\pgfqpoint{3.387112in}{0.895339in}}%
\pgfpathlineto{\pgfqpoint{3.406416in}{0.878220in}}%
\pgfpathlineto{\pgfqpoint{3.421662in}{0.861628in}}%
\pgfpathlineto{\pgfqpoint{3.433264in}{0.845606in}}%
\pgfpathlineto{\pgfqpoint{3.441593in}{0.830186in}}%
\pgfpathlineto{\pgfqpoint{3.446955in}{0.815396in}}%
\pgfpathlineto{\pgfqpoint{3.449594in}{0.801257in}}%
\pgfpathlineto{\pgfqpoint{3.449611in}{0.787782in}}%
\pgfpathlineto{\pgfqpoint{3.447179in}{0.774982in}}%
\pgfpathlineto{\pgfqpoint{3.442556in}{0.762867in}}%
\pgfpathlineto{\pgfqpoint{3.435941in}{0.751442in}}%
\pgfpathlineto{\pgfqpoint{3.427469in}{0.740712in}}%
\pgfpathlineto{\pgfqpoint{3.417213in}{0.730679in}}%
\pgfpathlineto{\pgfqpoint{3.405184in}{0.721345in}}%
\pgfpathlineto{\pgfqpoint{3.391329in}{0.712707in}}%
\pgfpathlineto{\pgfqpoint{3.375536in}{0.704761in}}%
\pgfpathlineto{\pgfqpoint{3.357628in}{0.697502in}}%
\pgfpathlineto{\pgfqpoint{3.326568in}{0.687898in}}%
\pgfpathlineto{\pgfqpoint{3.290889in}{0.679867in}}%
\pgfpathlineto{\pgfqpoint{3.250425in}{0.673428in}}%
\pgfpathlineto{\pgfqpoint{3.204860in}{0.668601in}}%
\pgfpathlineto{\pgfqpoint{3.153798in}{0.665415in}}%
\pgfpathlineto{\pgfqpoint{3.096754in}{0.663906in}}%
\pgfpathlineto{\pgfqpoint{3.033156in}{0.664113in}}%
\pgfpathlineto{\pgfqpoint{2.937060in}{0.667144in}}%
\pgfpathlineto{\pgfqpoint{2.826679in}{0.673483in}}%
\pgfpathlineto{\pgfqpoint{2.699775in}{0.683395in}}%
\pgfpathlineto{\pgfqpoint{2.555368in}{0.697129in}}%
\pgfpathlineto{\pgfqpoint{2.394178in}{0.714896in}}%
\pgfpathlineto{\pgfqpoint{2.218610in}{0.736873in}}%
\pgfpathlineto{\pgfqpoint{2.080194in}{0.756205in}}%
\pgfpathlineto{\pgfqpoint{1.941267in}{0.777836in}}%
\pgfpathlineto{\pgfqpoint{1.806940in}{0.801501in}}%
\pgfpathlineto{\pgfqpoint{1.722083in}{0.818253in}}%
\pgfpathlineto{\pgfqpoint{1.642235in}{0.835659in}}%
\pgfpathlineto{\pgfqpoint{1.568280in}{0.853601in}}%
\pgfpathlineto{\pgfqpoint{1.500939in}{0.871947in}}%
\pgfpathlineto{\pgfqpoint{1.440761in}{0.890555in}}%
\pgfpathlineto{\pgfqpoint{1.388124in}{0.909273in}}%
\pgfpathlineto{\pgfqpoint{1.342831in}{0.927943in}}%
\pgfpathlineto{\pgfqpoint{1.304010in}{0.946456in}}%
\pgfpathlineto{\pgfqpoint{1.270881in}{0.964720in}}%
\pgfpathlineto{\pgfqpoint{1.242822in}{0.982649in}}%
\pgfpathlineto{\pgfqpoint{1.219376in}{1.000169in}}%
\pgfpathlineto{\pgfqpoint{1.200243in}{1.017215in}}%
\pgfpathlineto{\pgfqpoint{1.185289in}{1.033729in}}%
\pgfpathlineto{\pgfqpoint{1.174493in}{1.049665in}}%
\pgfpathlineto{\pgfqpoint{1.167160in}{1.064983in}}%
\pgfpathlineto{\pgfqpoint{1.162723in}{1.079661in}}%
\pgfpathlineto{\pgfqpoint{1.160869in}{1.093680in}}%
\pgfpathlineto{\pgfqpoint{1.161362in}{1.107029in}}%
\pgfpathlineto{\pgfqpoint{1.164047in}{1.119695in}}%
\pgfpathlineto{\pgfqpoint{1.168849in}{1.131674in}}%
\pgfpathlineto{\pgfqpoint{1.175770in}{1.142960in}}%
\pgfpathlineto{\pgfqpoint{1.184881in}{1.153555in}}%
\pgfpathlineto{\pgfqpoint{1.196061in}{1.163456in}}%
\pgfpathlineto{\pgfqpoint{1.209187in}{1.172659in}}%
\pgfpathlineto{\pgfqpoint{1.224206in}{1.181162in}}%
\pgfpathlineto{\pgfqpoint{1.250236in}{1.192599in}}%
\pgfpathlineto{\pgfqpoint{1.280508in}{1.202451in}}%
\pgfpathlineto{\pgfqpoint{1.315202in}{1.210715in}}%
\pgfpathlineto{\pgfqpoint{1.354635in}{1.217387in}}%
\pgfpathlineto{\pgfqpoint{1.399270in}{1.222468in}}%
\pgfpathlineto{\pgfqpoint{1.449316in}{1.225939in}}%
\pgfpathlineto{\pgfqpoint{1.505189in}{1.227752in}}%
\pgfpathlineto{\pgfqpoint{1.567637in}{1.227857in}}%
\pgfpathlineto{\pgfqpoint{1.637357in}{1.226192in}}%
\pgfpathlineto{\pgfqpoint{1.742731in}{1.221095in}}%
\pgfpathlineto{\pgfqpoint{1.863550in}{1.212526in}}%
\pgfpathlineto{\pgfqpoint{2.001026in}{1.200255in}}%
\pgfpathlineto{\pgfqpoint{2.156361in}{1.184036in}}%
\pgfpathlineto{\pgfqpoint{2.328318in}{1.163654in}}%
\pgfpathlineto{\pgfqpoint{2.464208in}{1.145537in}}%
\pgfpathlineto{\pgfqpoint{2.602279in}{1.125048in}}%
\pgfpathlineto{\pgfqpoint{2.738767in}{1.102348in}}%
\pgfpathlineto{\pgfqpoint{2.869546in}{1.077708in}}%
\pgfpathlineto{\pgfqpoint{2.951358in}{1.060385in}}%
\pgfpathlineto{\pgfqpoint{3.027228in}{1.042503in}}%
\pgfpathlineto{\pgfqpoint{3.095803in}{1.024214in}}%
\pgfpathlineto{\pgfqpoint{3.156922in}{1.005660in}}%
\pgfpathlineto{\pgfqpoint{3.210984in}{0.986977in}}%
\pgfpathlineto{\pgfqpoint{3.258375in}{0.968294in}}%
\pgfpathlineto{\pgfqpoint{3.299472in}{0.949727in}}%
\pgfpathlineto{\pgfqpoint{3.334639in}{0.931381in}}%
\pgfpathlineto{\pgfqpoint{3.364233in}{0.913350in}}%
\pgfpathlineto{\pgfqpoint{3.388599in}{0.895717in}}%
\pgfpathlineto{\pgfqpoint{3.408073in}{0.878552in}}%
\pgfpathlineto{\pgfqpoint{3.423099in}{0.861919in}}%
\pgfpathlineto{\pgfqpoint{3.434378in}{0.845863in}}%
\pgfpathlineto{\pgfqpoint{3.442474in}{0.830412in}}%
\pgfpathlineto{\pgfqpoint{3.447819in}{0.815591in}}%
\pgfpathlineto{\pgfqpoint{3.450718in}{0.801418in}}%
\pgfpathlineto{\pgfqpoint{3.451342in}{0.787908in}}%
\pgfpathlineto{\pgfqpoint{3.449733in}{0.775073in}}%
\pgfpathlineto{\pgfqpoint{3.445802in}{0.762917in}}%
\pgfpathlineto{\pgfqpoint{3.439341in}{0.751444in}}%
\pgfpathlineto{\pgfqpoint{3.430590in}{0.740662in}}%
\pgfpathlineto{\pgfqpoint{3.419851in}{0.730579in}}%
\pgfpathlineto{\pgfqpoint{3.407201in}{0.721195in}}%
\pgfpathlineto{\pgfqpoint{3.392689in}{0.712514in}}%
\pgfpathlineto{\pgfqpoint{3.367454in}{0.700811in}}%
\pgfpathlineto{\pgfqpoint{3.337985in}{0.690693in}}%
\pgfpathlineto{\pgfqpoint{3.304050in}{0.682157in}}%
\pgfpathlineto{\pgfqpoint{3.265263in}{0.675199in}}%
\pgfpathlineto{\pgfqpoint{3.221416in}{0.669830in}}%
\pgfpathlineto{\pgfqpoint{3.172259in}{0.666081in}}%
\pgfpathlineto{\pgfqpoint{3.117235in}{0.663990in}}%
\pgfpathlineto{\pgfqpoint{3.055767in}{0.663601in}}%
\pgfpathlineto{\pgfqpoint{2.987248in}{0.664974in}}%
\pgfpathlineto{\pgfqpoint{2.883830in}{0.669661in}}%
\pgfpathlineto{\pgfqpoint{2.765191in}{0.677791in}}%
\pgfpathlineto{\pgfqpoint{2.629766in}{0.689588in}}%
\pgfpathlineto{\pgfqpoint{2.476808in}{0.705318in}}%
\pgfpathlineto{\pgfqpoint{2.307539in}{0.725212in}}%
\pgfpathlineto{\pgfqpoint{2.172417in}{0.742924in}}%
\pgfpathlineto{\pgfqpoint{2.033594in}{0.763014in}}%
\pgfpathlineto{\pgfqpoint{1.895534in}{0.785349in}}%
\pgfpathlineto{\pgfqpoint{1.763342in}{0.809660in}}%
\pgfpathlineto{\pgfqpoint{1.680654in}{0.826796in}}%
\pgfpathlineto{\pgfqpoint{1.603611in}{0.844534in}}%
\pgfpathlineto{\pgfqpoint{1.533139in}{0.862735in}}%
\pgfpathlineto{\pgfqpoint{1.470067in}{0.881241in}}%
\pgfpathlineto{\pgfqpoint{1.414358in}{0.899906in}}%
\pgfpathlineto{\pgfqpoint{1.365474in}{0.918601in}}%
\pgfpathlineto{\pgfqpoint{1.322934in}{0.937209in}}%
\pgfpathlineto{\pgfqpoint{1.286306in}{0.955623in}}%
\pgfpathlineto{\pgfqpoint{1.255213in}{0.973748in}}%
\pgfpathlineto{\pgfqpoint{1.229328in}{0.991497in}}%
\pgfpathlineto{\pgfqpoint{1.208380in}{1.008798in}}%
\pgfpathlineto{\pgfqpoint{1.192135in}{1.025585in}}%
\pgfpathlineto{\pgfqpoint{1.179961in}{1.041804in}}%
\pgfpathlineto{\pgfqpoint{1.171171in}{1.057420in}}%
\pgfpathlineto{\pgfqpoint{1.165289in}{1.072411in}}%
\pgfpathlineto{\pgfqpoint{1.161958in}{1.086755in}}%
\pgfpathlineto{\pgfqpoint{1.160939in}{1.100437in}}%
\pgfpathlineto{\pgfqpoint{1.162116in}{1.113446in}}%
\pgfpathlineto{\pgfqpoint{1.165491in}{1.125773in}}%
\pgfpathlineto{\pgfqpoint{1.171187in}{1.137416in}}%
\pgfpathlineto{\pgfqpoint{1.179343in}{1.148373in}}%
\pgfpathlineto{\pgfqpoint{1.189589in}{1.158634in}}%
\pgfpathlineto{\pgfqpoint{1.201778in}{1.168196in}}%
\pgfpathlineto{\pgfqpoint{1.215852in}{1.177057in}}%
\pgfpathlineto{\pgfqpoint{1.231781in}{1.185214in}}%
\pgfpathlineto{\pgfqpoint{1.259164in}{1.196129in}}%
\pgfpathlineto{\pgfqpoint{1.290858in}{1.205459in}}%
\pgfpathlineto{\pgfqpoint{1.327141in}{1.213204in}}%
\pgfpathlineto{\pgfqpoint{1.368428in}{1.219367in}}%
\pgfpathlineto{\pgfqpoint{1.414867in}{1.223930in}}%
\pgfpathlineto{\pgfqpoint{1.466852in}{1.226857in}}%
\pgfpathlineto{\pgfqpoint{1.524991in}{1.228106in}}%
\pgfpathlineto{\pgfqpoint{1.589900in}{1.227625in}}%
\pgfpathlineto{\pgfqpoint{1.688055in}{1.224182in}}%
\pgfpathlineto{\pgfqpoint{1.800853in}{1.217371in}}%
\pgfpathlineto{\pgfqpoint{1.929816in}{1.206977in}}%
\pgfpathlineto{\pgfqpoint{2.076201in}{1.192760in}}%
\pgfpathlineto{\pgfqpoint{2.239685in}{1.174451in}}%
\pgfpathlineto{\pgfqpoint{2.417155in}{1.151886in}}%
\pgfpathlineto{\pgfqpoint{2.555563in}{1.132191in}}%
\pgfpathlineto{\pgfqpoint{2.694062in}{1.110222in}}%
\pgfpathlineto{\pgfqpoint{2.827723in}{1.086188in}}%
\pgfpathlineto{\pgfqpoint{2.911692in}{1.069209in}}%
\pgfpathlineto{\pgfqpoint{2.990190in}{1.051613in}}%
\pgfpathlineto{\pgfqpoint{3.062360in}{1.033531in}}%
\pgfpathlineto{\pgfqpoint{3.127623in}{1.015097in}}%
\pgfpathlineto{\pgfqpoint{3.185676in}{0.996447in}}%
\pgfpathlineto{\pgfqpoint{3.236485in}{0.977725in}}%
\pgfpathlineto{\pgfqpoint{3.279920in}{0.959094in}}%
\pgfpathlineto{\pgfqpoint{3.316788in}{0.940663in}}%
\pgfpathlineto{\pgfqpoint{3.348192in}{0.922506in}}%
\pgfpathlineto{\pgfqpoint{3.374978in}{0.904693in}}%
\pgfpathlineto{\pgfqpoint{3.397735in}{0.887287in}}%
\pgfpathlineto{\pgfqpoint{3.416794in}{0.870346in}}%
\pgfpathlineto{\pgfqpoint{3.432230in}{0.853926in}}%
\pgfpathlineto{\pgfqpoint{3.443860in}{0.838074in}}%
\pgfpathlineto{\pgfqpoint{3.451246in}{0.822836in}}%
\pgfpathlineto{\pgfqpoint{3.454695in}{0.808249in}}%
\pgfpathlineto{\pgfqpoint{3.455530in}{0.794336in}}%
\pgfpathlineto{\pgfqpoint{3.454013in}{0.781106in}}%
\pgfpathlineto{\pgfqpoint{3.450338in}{0.768567in}}%
\pgfpathlineto{\pgfqpoint{3.444635in}{0.756725in}}%
\pgfpathlineto{\pgfqpoint{3.436975in}{0.745584in}}%
\pgfpathlineto{\pgfqpoint{3.427370in}{0.735145in}}%
\pgfpathlineto{\pgfqpoint{3.415769in}{0.725406in}}%
\pgfpathlineto{\pgfqpoint{3.402098in}{0.716364in}}%
\pgfpathlineto{\pgfqpoint{3.386465in}{0.708021in}}%
\pgfpathlineto{\pgfqpoint{3.359424in}{0.696823in}}%
\pgfpathlineto{\pgfqpoint{3.328050in}{0.687212in}}%
\pgfpathlineto{\pgfqpoint{3.292213in}{0.679197in}}%
\pgfpathlineto{\pgfqpoint{3.251664in}{0.672789in}}%
\pgfpathlineto{\pgfqpoint{3.206039in}{0.668001in}}%
\pgfpathlineto{\pgfqpoint{3.154859in}{0.664846in}}%
\pgfpathlineto{\pgfqpoint{3.097677in}{0.663340in}}%
\pgfpathlineto{\pgfqpoint{3.034282in}{0.663527in}}%
\pgfpathlineto{\pgfqpoint{2.963524in}{0.665496in}}%
\pgfpathlineto{\pgfqpoint{2.856057in}{0.671047in}}%
\pgfpathlineto{\pgfqpoint{2.732255in}{0.680136in}}%
\pgfpathlineto{\pgfqpoint{2.591542in}{0.692976in}}%
\pgfpathlineto{\pgfqpoint{2.434292in}{0.709781in}}%
\pgfpathlineto{\pgfqpoint{2.261823in}{0.730765in}}%
\pgfpathlineto{\pgfqpoint{2.124226in}{0.749383in}}%
\pgfpathlineto{\pgfqpoint{1.984545in}{0.770369in}}%
\pgfpathlineto{\pgfqpoint{1.848109in}{0.793468in}}%
\pgfpathlineto{\pgfqpoint{1.761193in}{0.809893in}}%
\pgfpathlineto{\pgfqpoint{1.678828in}{0.827020in}}%
\pgfpathlineto{\pgfqpoint{1.601952in}{0.844736in}}%
\pgfpathlineto{\pgfqpoint{1.531329in}{0.862919in}}%
\pgfpathlineto{\pgfqpoint{1.467556in}{0.881436in}}%
\pgfpathlineto{\pgfqpoint{1.411055in}{0.900146in}}%
\pgfpathlineto{\pgfqpoint{1.362076in}{0.918896in}}%
\pgfpathlineto{\pgfqpoint{1.320316in}{0.937533in}}%
\pgfpathlineto{\pgfqpoint{1.284821in}{0.955951in}}%
\pgfpathlineto{\pgfqpoint{1.254753in}{0.974067in}}%
\pgfpathlineto{\pgfqpoint{1.229458in}{0.991804in}}%
\pgfpathlineto{\pgfqpoint{1.208466in}{1.009095in}}%
\pgfpathlineto{\pgfqpoint{1.191490in}{1.025882in}}%
\pgfpathlineto{\pgfqpoint{1.178429in}{1.042113in}}%
\pgfpathlineto{\pgfqpoint{1.169339in}{1.057748in}}%
\pgfpathlineto{\pgfqpoint{1.163605in}{1.072750in}}%
\pgfpathlineto{\pgfqpoint{1.160635in}{1.087099in}}%
\pgfpathlineto{\pgfqpoint{1.160149in}{1.100781in}}%
\pgfpathlineto{\pgfqpoint{1.161936in}{1.113784in}}%
\pgfpathlineto{\pgfqpoint{1.165855in}{1.126099in}}%
\pgfpathlineto{\pgfqpoint{1.171836in}{1.137722in}}%
\pgfpathlineto{\pgfqpoint{1.179877in}{1.148650in}}%
\pgfpathlineto{\pgfqpoint{1.190042in}{1.158884in}}%
\pgfpathlineto{\pgfqpoint{1.202251in}{1.168422in}}%
\pgfpathlineto{\pgfqpoint{1.216397in}{1.177263in}}%
\pgfpathlineto{\pgfqpoint{1.232438in}{1.185403in}}%
\pgfpathlineto{\pgfqpoint{1.260026in}{1.196295in}}%
\pgfpathlineto{\pgfqpoint{1.291911in}{1.205600in}}%
\pgfpathlineto{\pgfqpoint{1.328287in}{1.213311in}}%
\pgfpathlineto{\pgfqpoint{1.369485in}{1.219424in}}%
\pgfpathlineto{\pgfqpoint{1.415969in}{1.223934in}}%
\pgfpathlineto{\pgfqpoint{1.468040in}{1.226828in}}%
\pgfpathlineto{\pgfqpoint{1.526052in}{1.228052in}}%
\pgfpathlineto{\pgfqpoint{1.590873in}{1.227548in}}%
\pgfpathlineto{\pgfqpoint{1.689199in}{1.224071in}}%
\pgfpathlineto{\pgfqpoint{1.802492in}{1.217217in}}%
\pgfpathlineto{\pgfqpoint{1.931917in}{1.206776in}}%
\pgfpathlineto{\pgfqpoint{2.078298in}{1.192507in}}%
\pgfpathlineto{\pgfqpoint{2.242642in}{1.174129in}}%
\pgfpathlineto{\pgfqpoint{2.422641in}{1.151420in}}%
\pgfpathlineto{\pgfqpoint{2.561321in}{1.131623in}}%
\pgfpathlineto{\pgfqpoint{2.698244in}{1.109620in}}%
\pgfpathlineto{\pgfqpoint{2.829518in}{1.085641in}}%
\pgfpathlineto{\pgfqpoint{2.912192in}{1.068705in}}%
\pgfpathlineto{\pgfqpoint{2.989941in}{1.051133in}}%
\pgfpathlineto{\pgfqpoint{3.061971in}{1.033048in}}%
\pgfpathlineto{\pgfqpoint{3.127584in}{1.014589in}}%
\pgfpathlineto{\pgfqpoint{3.186179in}{0.995909in}}%
\pgfpathlineto{\pgfqpoint{3.237249in}{0.977176in}}%
\pgfpathlineto{\pgfqpoint{3.280986in}{0.958533in}}%
\pgfpathlineto{\pgfqpoint{3.318351in}{0.940073in}}%
\pgfpathlineto{\pgfqpoint{3.349927in}{0.921894in}}%
\pgfpathlineto{\pgfqpoint{3.376245in}{0.904082in}}%
\pgfpathlineto{\pgfqpoint{3.397778in}{0.886711in}}%
\pgfpathlineto{\pgfqpoint{3.414960in}{0.869842in}}%
\pgfpathlineto{\pgfqpoint{3.428276in}{0.853524in}}%
\pgfpathlineto{\pgfqpoint{3.438160in}{0.837794in}}%
\pgfpathlineto{\pgfqpoint{3.444958in}{0.822681in}}%
\pgfpathlineto{\pgfqpoint{3.448939in}{0.808209in}}%
\pgfpathlineto{\pgfqpoint{3.450299in}{0.794396in}}%
\pgfpathlineto{\pgfqpoint{3.449142in}{0.781254in}}%
\pgfpathlineto{\pgfqpoint{3.445613in}{0.768793in}}%
\pgfpathlineto{\pgfqpoint{3.439931in}{0.757018in}}%
\pgfpathlineto{\pgfqpoint{3.432267in}{0.745937in}}%
\pgfpathlineto{\pgfqpoint{3.422737in}{0.735552in}}%
\pgfpathlineto{\pgfqpoint{3.411403in}{0.725866in}}%
\pgfpathlineto{\pgfqpoint{3.398277in}{0.716880in}}%
\pgfpathlineto{\pgfqpoint{3.383313in}{0.708591in}}%
\pgfpathlineto{\pgfqpoint{3.366418in}{0.700997in}}%
\pgfpathlineto{\pgfqpoint{3.337109in}{0.690898in}}%
\pgfpathlineto{\pgfqpoint{3.302957in}{0.682354in}}%
\pgfpathlineto{\pgfqpoint{3.264104in}{0.675392in}}%
\pgfpathlineto{\pgfqpoint{3.220258in}{0.670034in}}%
\pgfpathlineto{\pgfqpoint{3.171045in}{0.666305in}}%
\pgfpathlineto{\pgfqpoint{3.116011in}{0.664239in}}%
\pgfpathlineto{\pgfqpoint{3.054619in}{0.663878in}}%
\pgfpathlineto{\pgfqpoint{2.986251in}{0.665271in}}%
\pgfpathlineto{\pgfqpoint{2.883078in}{0.669955in}}%
\pgfpathlineto{\pgfqpoint{2.764617in}{0.678067in}}%
\pgfpathlineto{\pgfqpoint{2.629061in}{0.689870in}}%
\pgfpathlineto{\pgfqpoint{2.476113in}{0.705603in}}%
\pgfpathlineto{\pgfqpoint{2.307177in}{0.725474in}}%
\pgfpathlineto{\pgfqpoint{2.171817in}{0.743192in}}%
\pgfpathlineto{\pgfqpoint{2.032964in}{0.763287in}}%
\pgfpathlineto{\pgfqpoint{1.895414in}{0.785600in}}%
\pgfpathlineto{\pgfqpoint{1.763612in}{0.809883in}}%
\pgfpathlineto{\pgfqpoint{1.680975in}{0.827002in}}%
\pgfpathlineto{\pgfqpoint{1.603844in}{0.844726in}}%
\pgfpathlineto{\pgfqpoint{1.533276in}{0.862915in}}%
\pgfpathlineto{\pgfqpoint{1.470110in}{0.881414in}}%
\pgfpathlineto{\pgfqpoint{1.414255in}{0.900075in}}%
\pgfpathlineto{\pgfqpoint{1.365289in}{0.918767in}}%
\pgfpathlineto{\pgfqpoint{1.322799in}{0.937370in}}%
\pgfpathlineto{\pgfqpoint{1.286380in}{0.955774in}}%
\pgfpathlineto{\pgfqpoint{1.255641in}{0.973884in}}%
\pgfpathlineto{\pgfqpoint{1.230196in}{0.991613in}}%
\pgfpathlineto{\pgfqpoint{1.209671in}{1.008887in}}%
\pgfpathlineto{\pgfqpoint{1.193576in}{1.025643in}}%
\pgfpathlineto{\pgfqpoint{1.181261in}{1.041834in}}%
\pgfpathlineto{\pgfqpoint{1.172220in}{1.057431in}}%
\pgfpathlineto{\pgfqpoint{1.166059in}{1.072405in}}%
\pgfpathlineto{\pgfqpoint{1.162508in}{1.086735in}}%
\pgfpathlineto{\pgfqpoint{1.161408in}{1.100405in}}%
\pgfpathlineto{\pgfqpoint{1.162724in}{1.113403in}}%
\pgfpathlineto{\pgfqpoint{1.166529in}{1.125721in}}%
\pgfpathlineto{\pgfqpoint{1.172679in}{1.137353in}}%
\pgfpathlineto{\pgfqpoint{1.180907in}{1.148291in}}%
\pgfpathlineto{\pgfqpoint{1.191097in}{1.158532in}}%
\pgfpathlineto{\pgfqpoint{1.203166in}{1.168074in}}%
\pgfpathlineto{\pgfqpoint{1.217070in}{1.176913in}}%
\pgfpathlineto{\pgfqpoint{1.232802in}{1.185051in}}%
\pgfpathlineto{\pgfqpoint{1.259901in}{1.195943in}}%
\pgfpathlineto{\pgfqpoint{1.291443in}{1.205261in}}%
\pgfpathlineto{\pgfqpoint{1.327829in}{1.213015in}}%
\pgfpathlineto{\pgfqpoint{1.369100in}{1.219186in}}%
\pgfpathlineto{\pgfqpoint{1.415527in}{1.223749in}}%
\pgfpathlineto{\pgfqpoint{1.467563in}{1.226671in}}%
\pgfpathlineto{\pgfqpoint{1.525719in}{1.227913in}}%
\pgfpathlineto{\pgfqpoint{1.590560in}{1.227423in}}%
\pgfpathlineto{\pgfqpoint{1.688505in}{1.223971in}}%
\pgfpathlineto{\pgfqpoint{1.801090in}{1.217165in}}%
\pgfpathlineto{\pgfqpoint{1.929977in}{1.206789in}}%
\pgfpathlineto{\pgfqpoint{2.076314in}{1.192575in}}%
\pgfpathlineto{\pgfqpoint{2.239601in}{1.174286in}}%
\pgfpathlineto{\pgfqpoint{2.416982in}{1.151754in}}%
\pgfpathlineto{\pgfqpoint{2.555286in}{1.132071in}}%
\pgfpathlineto{\pgfqpoint{2.693627in}{1.110123in}}%
\pgfpathlineto{\pgfqpoint{2.827264in}{1.086136in}}%
\pgfpathlineto{\pgfqpoint{2.911112in}{1.069154in}}%
\pgfpathlineto{\pgfqpoint{2.989362in}{1.051525in}}%
\pgfpathlineto{\pgfqpoint{3.061335in}{1.033414in}}%
\pgfpathlineto{\pgfqpoint{3.126561in}{1.014975in}}%
\pgfpathlineto{\pgfqpoint{3.184770in}{0.996350in}}%
\pgfpathlineto{\pgfqpoint{3.235895in}{0.977672in}}%
\pgfpathlineto{\pgfqpoint{3.280073in}{0.959058in}}%
\pgfpathlineto{\pgfqpoint{3.317644in}{0.940619in}}%
\pgfpathlineto{\pgfqpoint{3.349151in}{0.922449in}}%
\pgfpathlineto{\pgfqpoint{3.375340in}{0.904636in}}%
\pgfpathlineto{\pgfqpoint{3.396734in}{0.887263in}}%
\pgfpathlineto{\pgfqpoint{3.413542in}{0.870402in}}%
\pgfpathlineto{\pgfqpoint{3.426618in}{0.854089in}}%
\pgfpathlineto{\pgfqpoint{3.436635in}{0.838354in}}%
\pgfpathlineto{\pgfqpoint{3.444066in}{0.823223in}}%
\pgfpathlineto{\pgfqpoint{3.449184in}{0.808718in}}%
\pgfpathlineto{\pgfqpoint{3.452060in}{0.794856in}}%
\pgfpathlineto{\pgfqpoint{3.452565in}{0.781652in}}%
\pgfpathlineto{\pgfqpoint{3.450370in}{0.769113in}}%
\pgfpathlineto{\pgfqpoint{3.445012in}{0.757246in}}%
\pgfpathlineto{\pgfqpoint{3.437195in}{0.746071in}}%
\pgfpathlineto{\pgfqpoint{3.427383in}{0.735597in}}%
\pgfpathlineto{\pgfqpoint{3.415661in}{0.725827in}}%
\pgfpathlineto{\pgfqpoint{3.402080in}{0.716761in}}%
\pgfpathlineto{\pgfqpoint{3.386661in}{0.708401in}}%
\pgfpathlineto{\pgfqpoint{3.360048in}{0.697182in}}%
\pgfpathlineto{\pgfqpoint{3.329084in}{0.687548in}}%
\pgfpathlineto{\pgfqpoint{3.293429in}{0.679491in}}%
\pgfpathlineto{\pgfqpoint{3.252864in}{0.673021in}}%
\pgfpathlineto{\pgfqpoint{3.207198in}{0.668164in}}%
\pgfpathlineto{\pgfqpoint{3.156027in}{0.664947in}}%
\pgfpathlineto{\pgfqpoint{3.098870in}{0.663405in}}%
\pgfpathlineto{\pgfqpoint{3.035164in}{0.663583in}}%
\pgfpathlineto{\pgfqpoint{2.988741in}{0.664683in}}%
\pgfpathlineto{\pgfqpoint{2.988741in}{0.664683in}}%
\pgfusepath{stroke}%
\end{pgfscope}%
\begin{pgfscope}%
\pgfpathrectangle{\pgfqpoint{0.562500in}{0.275000in}}{\pgfqpoint{3.487500in}{1.925000in}}%
\pgfusepath{clip}%
\pgfsetrectcap%
\pgfsetroundjoin%
\pgfsetlinewidth{1.505625pt}%
\definecolor{currentstroke}{rgb}{0.580392,0.403922,0.741176}%
\pgfsetstrokecolor{currentstroke}%
\pgfsetdash{}{0pt}%
\pgfpathmoveto{\pgfqpoint{0.721023in}{0.362500in}}%
\pgfpathlineto{\pgfqpoint{0.899983in}{0.390425in}}%
\pgfpathlineto{\pgfqpoint{1.010919in}{0.416380in}}%
\pgfpathlineto{\pgfqpoint{1.080022in}{0.440877in}}%
\pgfpathlineto{\pgfqpoint{1.127330in}{0.464257in}}%
\pgfpathlineto{\pgfqpoint{1.159459in}{0.486707in}}%
\pgfpathlineto{\pgfqpoint{1.180136in}{0.508361in}}%
\pgfpathlineto{\pgfqpoint{1.193088in}{0.529328in}}%
\pgfpathlineto{\pgfqpoint{1.200831in}{0.549692in}}%
\pgfpathlineto{\pgfqpoint{1.204437in}{0.569502in}}%
\pgfpathlineto{\pgfqpoint{1.204850in}{0.588802in}}%
\pgfpathlineto{\pgfqpoint{1.202926in}{0.607629in}}%
\pgfpathlineto{\pgfqpoint{1.194961in}{0.643991in}}%
\pgfpathlineto{\pgfqpoint{1.183713in}{0.678763in}}%
\pgfpathlineto{\pgfqpoint{1.163770in}{0.728173in}}%
\pgfpathlineto{\pgfqpoint{1.113532in}{0.845643in}}%
\pgfpathlineto{\pgfqpoint{1.098843in}{0.884802in}}%
\pgfpathlineto{\pgfqpoint{1.087123in}{0.921482in}}%
\pgfpathlineto{\pgfqpoint{1.078715in}{0.955749in}}%
\pgfpathlineto{\pgfqpoint{1.073357in}{0.987719in}}%
\pgfpathlineto{\pgfqpoint{1.070801in}{1.017506in}}%
\pgfpathlineto{\pgfqpoint{1.071035in}{1.045198in}}%
\pgfpathlineto{\pgfqpoint{1.074285in}{1.070861in}}%
\pgfpathlineto{\pgfqpoint{1.078345in}{1.086864in}}%
\pgfpathlineto{\pgfqpoint{1.084139in}{1.101990in}}%
\pgfpathlineto{\pgfqpoint{1.091840in}{1.116244in}}%
\pgfpathlineto{\pgfqpoint{1.101172in}{1.129649in}}%
\pgfpathlineto{\pgfqpoint{1.112000in}{1.142223in}}%
\pgfpathlineto{\pgfqpoint{1.124273in}{1.153976in}}%
\pgfpathlineto{\pgfqpoint{1.137974in}{1.164922in}}%
\pgfpathlineto{\pgfqpoint{1.153121in}{1.175073in}}%
\pgfpathlineto{\pgfqpoint{1.178671in}{1.188830in}}%
\pgfpathlineto{\pgfqpoint{1.207909in}{1.200854in}}%
\pgfpathlineto{\pgfqpoint{1.241324in}{1.211178in}}%
\pgfpathlineto{\pgfqpoint{1.279348in}{1.219821in}}%
\pgfpathlineto{\pgfqpoint{1.321932in}{1.226776in}}%
\pgfpathlineto{\pgfqpoint{1.369438in}{1.232032in}}%
\pgfpathlineto{\pgfqpoint{1.422331in}{1.235570in}}%
\pgfpathlineto{\pgfqpoint{1.481148in}{1.237359in}}%
\pgfpathlineto{\pgfqpoint{1.546502in}{1.237359in}}%
\pgfpathlineto{\pgfqpoint{1.619080in}{1.235520in}}%
\pgfpathlineto{\pgfqpoint{1.728414in}{1.230099in}}%
\pgfpathlineto{\pgfqpoint{1.853761in}{1.221102in}}%
\pgfpathlineto{\pgfqpoint{1.996820in}{1.208253in}}%
\pgfpathlineto{\pgfqpoint{2.157848in}{1.191297in}}%
\pgfpathlineto{\pgfqpoint{2.334998in}{1.170025in}}%
\pgfpathlineto{\pgfqpoint{2.475599in}{1.151175in}}%
\pgfpathlineto{\pgfqpoint{2.618256in}{1.129912in}}%
\pgfpathlineto{\pgfqpoint{2.758303in}{1.106418in}}%
\pgfpathlineto{\pgfqpoint{2.847840in}{1.089657in}}%
\pgfpathlineto{\pgfqpoint{2.932605in}{1.072146in}}%
\pgfpathlineto{\pgfqpoint{3.011082in}{1.054020in}}%
\pgfpathlineto{\pgfqpoint{3.082352in}{1.035427in}}%
\pgfpathlineto{\pgfqpoint{3.146368in}{1.016528in}}%
\pgfpathlineto{\pgfqpoint{3.203209in}{0.997472in}}%
\pgfpathlineto{\pgfqpoint{3.253041in}{0.978396in}}%
\pgfpathlineto{\pgfqpoint{3.296115in}{0.959422in}}%
\pgfpathlineto{\pgfqpoint{3.332771in}{0.940661in}}%
\pgfpathlineto{\pgfqpoint{3.363435in}{0.922212in}}%
\pgfpathlineto{\pgfqpoint{3.388621in}{0.904160in}}%
\pgfpathlineto{\pgfqpoint{3.408899in}{0.886577in}}%
\pgfpathlineto{\pgfqpoint{3.424673in}{0.869537in}}%
\pgfpathlineto{\pgfqpoint{3.436669in}{0.853078in}}%
\pgfpathlineto{\pgfqpoint{3.445555in}{0.837226in}}%
\pgfpathlineto{\pgfqpoint{3.451820in}{0.822005in}}%
\pgfpathlineto{\pgfqpoint{3.455774in}{0.807433in}}%
\pgfpathlineto{\pgfqpoint{3.457552in}{0.793526in}}%
\pgfpathlineto{\pgfqpoint{3.457111in}{0.780294in}}%
\pgfpathlineto{\pgfqpoint{3.454227in}{0.767744in}}%
\pgfpathlineto{\pgfqpoint{3.448506in}{0.755880in}}%
\pgfpathlineto{\pgfqpoint{3.440263in}{0.744713in}}%
\pgfpathlineto{\pgfqpoint{3.430045in}{0.734254in}}%
\pgfpathlineto{\pgfqpoint{3.417932in}{0.724505in}}%
\pgfpathlineto{\pgfqpoint{3.403975in}{0.715466in}}%
\pgfpathlineto{\pgfqpoint{3.388192in}{0.707137in}}%
\pgfpathlineto{\pgfqpoint{3.361061in}{0.695975in}}%
\pgfpathlineto{\pgfqpoint{3.329615in}{0.686407in}}%
\pgfpathlineto{\pgfqpoint{3.293524in}{0.678427in}}%
\pgfpathlineto{\pgfqpoint{3.252514in}{0.672036in}}%
\pgfpathlineto{\pgfqpoint{3.206441in}{0.667260in}}%
\pgfpathlineto{\pgfqpoint{3.154825in}{0.664127in}}%
\pgfpathlineto{\pgfqpoint{3.097134in}{0.662677in}}%
\pgfpathlineto{\pgfqpoint{3.032792in}{0.662960in}}%
\pgfpathlineto{\pgfqpoint{2.961181in}{0.665035in}}%
\pgfpathlineto{\pgfqpoint{2.853239in}{0.670713in}}%
\pgfpathlineto{\pgfqpoint{2.729505in}{0.679901in}}%
\pgfpathlineto{\pgfqpoint{2.588564in}{0.692843in}}%
\pgfpathlineto{\pgfqpoint{2.430130in}{0.709808in}}%
\pgfpathlineto{\pgfqpoint{2.256331in}{0.730985in}}%
\pgfpathlineto{\pgfqpoint{2.118851in}{0.749681in}}%
\pgfpathlineto{\pgfqpoint{1.979504in}{0.770716in}}%
\pgfpathlineto{\pgfqpoint{1.842852in}{0.793909in}}%
\pgfpathlineto{\pgfqpoint{1.755610in}{0.810432in}}%
\pgfpathlineto{\pgfqpoint{1.673213in}{0.827677in}}%
\pgfpathlineto{\pgfqpoint{1.597050in}{0.845527in}}%
\pgfpathlineto{\pgfqpoint{1.527557in}{0.863821in}}%
\pgfpathlineto{\pgfqpoint{1.464926in}{0.882398in}}%
\pgfpathlineto{\pgfqpoint{1.409221in}{0.901114in}}%
\pgfpathlineto{\pgfqpoint{1.360379in}{0.919838in}}%
\pgfpathlineto{\pgfqpoint{1.318210in}{0.938451in}}%
\pgfpathlineto{\pgfqpoint{1.282394in}{0.956848in}}%
\pgfpathlineto{\pgfqpoint{1.252486in}{0.974937in}}%
\pgfpathlineto{\pgfqpoint{1.227913in}{0.992639in}}%
\pgfpathlineto{\pgfqpoint{1.207974in}{1.009890in}}%
\pgfpathlineto{\pgfqpoint{1.192406in}{1.026617in}}%
\pgfpathlineto{\pgfqpoint{1.180793in}{1.042770in}}%
\pgfpathlineto{\pgfqpoint{1.172366in}{1.058324in}}%
\pgfpathlineto{\pgfqpoint{1.166538in}{1.073256in}}%
\pgfpathlineto{\pgfqpoint{1.162905in}{1.087550in}}%
\pgfpathlineto{\pgfqpoint{1.161245in}{1.101191in}}%
\pgfpathlineto{\pgfqpoint{1.161518in}{1.114168in}}%
\pgfpathlineto{\pgfqpoint{1.163866in}{1.126476in}}%
\pgfpathlineto{\pgfqpoint{1.168614in}{1.138110in}}%
\pgfpathlineto{\pgfqpoint{1.176267in}{1.149071in}}%
\pgfpathlineto{\pgfqpoint{1.186731in}{1.159349in}}%
\pgfpathlineto{\pgfqpoint{1.199146in}{1.168923in}}%
\pgfpathlineto{\pgfqpoint{1.213453in}{1.177792in}}%
\pgfpathlineto{\pgfqpoint{1.229625in}{1.185954in}}%
\pgfpathlineto{\pgfqpoint{1.257386in}{1.196870in}}%
\pgfpathlineto{\pgfqpoint{1.289461in}{1.206193in}}%
\pgfpathlineto{\pgfqpoint{1.326095in}{1.213921in}}%
\pgfpathlineto{\pgfqpoint{1.367664in}{1.220055in}}%
\pgfpathlineto{\pgfqpoint{1.414408in}{1.224583in}}%
\pgfpathlineto{\pgfqpoint{1.466663in}{1.227471in}}%
\pgfpathlineto{\pgfqpoint{1.525086in}{1.228677in}}%
\pgfpathlineto{\pgfqpoint{1.590327in}{1.228147in}}%
\pgfpathlineto{\pgfqpoint{1.689017in}{1.224631in}}%
\pgfpathlineto{\pgfqpoint{1.802436in}{1.217734in}}%
\pgfpathlineto{\pgfqpoint{1.932024in}{1.207240in}}%
\pgfpathlineto{\pgfqpoint{2.079103in}{1.192911in}}%
\pgfpathlineto{\pgfqpoint{2.243345in}{1.174492in}}%
\pgfpathlineto{\pgfqpoint{2.421307in}{1.151782in}}%
\pgfpathlineto{\pgfqpoint{2.559900in}{1.131962in}}%
\pgfpathlineto{\pgfqpoint{2.698534in}{1.109891in}}%
\pgfpathlineto{\pgfqpoint{2.832194in}{1.085788in}}%
\pgfpathlineto{\pgfqpoint{2.915918in}{1.068730in}}%
\pgfpathlineto{\pgfqpoint{2.994064in}{1.051047in}}%
\pgfpathlineto{\pgfqpoint{3.065871in}{1.032895in}}%
\pgfpathlineto{\pgfqpoint{3.130821in}{1.014420in}}%
\pgfpathlineto{\pgfqpoint{3.188634in}{0.995760in}}%
\pgfpathlineto{\pgfqpoint{3.239272in}{0.977046in}}%
\pgfpathlineto{\pgfqpoint{3.282937in}{0.958396in}}%
\pgfpathlineto{\pgfqpoint{3.320071in}{0.939922in}}%
\pgfpathlineto{\pgfqpoint{3.351355in}{0.921726in}}%
\pgfpathlineto{\pgfqpoint{3.377071in}{0.903917in}}%
\pgfpathlineto{\pgfqpoint{3.397829in}{0.886566in}}%
\pgfpathlineto{\pgfqpoint{3.414578in}{0.869715in}}%
\pgfpathlineto{\pgfqpoint{3.428039in}{0.853402in}}%
\pgfpathlineto{\pgfqpoint{3.438694in}{0.837660in}}%
\pgfpathlineto{\pgfqpoint{3.446794in}{0.822517in}}%
\pgfpathlineto{\pgfqpoint{3.452357in}{0.807998in}}%
\pgfpathlineto{\pgfqpoint{3.455163in}{0.794123in}}%
\pgfpathlineto{\pgfqpoint{3.454763in}{0.780907in}}%
\pgfpathlineto{\pgfqpoint{3.451032in}{0.768368in}}%
\pgfpathlineto{\pgfqpoint{3.445075in}{0.756525in}}%
\pgfpathlineto{\pgfqpoint{3.437091in}{0.745384in}}%
\pgfpathlineto{\pgfqpoint{3.427184in}{0.734946in}}%
\pgfpathlineto{\pgfqpoint{3.415417in}{0.725214in}}%
\pgfpathlineto{\pgfqpoint{3.401815in}{0.716188in}}%
\pgfpathlineto{\pgfqpoint{3.386368in}{0.707866in}}%
\pgfpathlineto{\pgfqpoint{3.359614in}{0.696702in}}%
\pgfpathlineto{\pgfqpoint{3.328272in}{0.687110in}}%
\pgfpathlineto{\pgfqpoint{3.292300in}{0.679097in}}%
\pgfpathlineto{\pgfqpoint{3.251548in}{0.672678in}}%
\pgfpathlineto{\pgfqpoint{3.205686in}{0.667875in}}%
\pgfpathlineto{\pgfqpoint{3.154308in}{0.664719in}}%
\pgfpathlineto{\pgfqpoint{3.096924in}{0.663247in}}%
\pgfpathlineto{\pgfqpoint{3.032968in}{0.663501in}}%
\pgfpathlineto{\pgfqpoint{2.936343in}{0.666619in}}%
\pgfpathlineto{\pgfqpoint{2.825238in}{0.673057in}}%
\pgfpathlineto{\pgfqpoint{2.697858in}{0.683061in}}%
\pgfpathlineto{\pgfqpoint{2.552972in}{0.696891in}}%
\pgfpathlineto{\pgfqpoint{2.391110in}{0.714774in}}%
\pgfpathlineto{\pgfqpoint{2.214577in}{0.736899in}}%
\pgfpathlineto{\pgfqpoint{2.076132in}{0.756299in}}%
\pgfpathlineto{\pgfqpoint{1.937475in}{0.777983in}}%
\pgfpathlineto{\pgfqpoint{1.803097in}{0.801735in}}%
\pgfpathlineto{\pgfqpoint{1.718042in}{0.818568in}}%
\pgfpathlineto{\pgfqpoint{1.638073in}{0.836065in}}%
\pgfpathlineto{\pgfqpoint{1.564413in}{0.854089in}}%
\pgfpathlineto{\pgfqpoint{1.497985in}{0.872490in}}%
\pgfpathlineto{\pgfqpoint{1.438770in}{0.891121in}}%
\pgfpathlineto{\pgfqpoint{1.386518in}{0.909841in}}%
\pgfpathlineto{\pgfqpoint{1.340941in}{0.928526in}}%
\pgfpathlineto{\pgfqpoint{1.301711in}{0.947059in}}%
\pgfpathlineto{\pgfqpoint{1.268457in}{0.965339in}}%
\pgfpathlineto{\pgfqpoint{1.240771in}{0.983273in}}%
\pgfpathlineto{\pgfqpoint{1.218202in}{1.000783in}}%
\pgfpathlineto{\pgfqpoint{1.200262in}{1.017797in}}%
\pgfpathlineto{\pgfqpoint{1.186335in}{1.034264in}}%
\pgfpathlineto{\pgfqpoint{1.175835in}{1.050152in}}%
\pgfpathlineto{\pgfqpoint{1.168311in}{1.065429in}}%
\pgfpathlineto{\pgfqpoint{1.163448in}{1.080073in}}%
\pgfpathlineto{\pgfqpoint{1.161064in}{1.094065in}}%
\pgfpathlineto{\pgfqpoint{1.161115in}{1.107389in}}%
\pgfpathlineto{\pgfqpoint{1.163692in}{1.120039in}}%
\pgfpathlineto{\pgfqpoint{1.168839in}{1.132006in}}%
\pgfpathlineto{\pgfqpoint{1.176143in}{1.143281in}}%
\pgfpathlineto{\pgfqpoint{1.185450in}{1.153859in}}%
\pgfpathlineto{\pgfqpoint{1.196666in}{1.163737in}}%
\pgfpathlineto{\pgfqpoint{1.209731in}{1.172914in}}%
\pgfpathlineto{\pgfqpoint{1.224626in}{1.181387in}}%
\pgfpathlineto{\pgfqpoint{1.250438in}{1.192781in}}%
\pgfpathlineto{\pgfqpoint{1.280613in}{1.202600in}}%
\pgfpathlineto{\pgfqpoint{1.315531in}{1.210851in}}%
\pgfpathlineto{\pgfqpoint{1.355296in}{1.217525in}}%
\pgfpathlineto{\pgfqpoint{1.400096in}{1.222597in}}%
\pgfpathlineto{\pgfqpoint{1.450357in}{1.226037in}}%
\pgfpathlineto{\pgfqpoint{1.506568in}{1.227809in}}%
\pgfpathlineto{\pgfqpoint{1.569275in}{1.227865in}}%
\pgfpathlineto{\pgfqpoint{1.639088in}{1.226149in}}%
\pgfpathlineto{\pgfqpoint{1.744391in}{1.220989in}}%
\pgfpathlineto{\pgfqpoint{1.865230in}{1.212374in}}%
\pgfpathlineto{\pgfqpoint{2.003097in}{1.200047in}}%
\pgfpathlineto{\pgfqpoint{2.158421in}{1.183754in}}%
\pgfpathlineto{\pgfqpoint{2.329596in}{1.163284in}}%
\pgfpathlineto{\pgfqpoint{2.465704in}{1.145128in}}%
\pgfpathlineto{\pgfqpoint{2.604554in}{1.124617in}}%
\pgfpathlineto{\pgfqpoint{2.741834in}{1.101908in}}%
\pgfpathlineto{\pgfqpoint{2.830059in}{1.085674in}}%
\pgfpathlineto{\pgfqpoint{2.913769in}{1.068667in}}%
\pgfpathlineto{\pgfqpoint{2.991827in}{1.051016in}}%
\pgfpathlineto{\pgfqpoint{3.063580in}{1.032887in}}%
\pgfpathlineto{\pgfqpoint{3.128573in}{1.014434in}}%
\pgfpathlineto{\pgfqpoint{3.186552in}{0.995800in}}%
\pgfpathlineto{\pgfqpoint{3.237458in}{0.977118in}}%
\pgfpathlineto{\pgfqpoint{3.281432in}{0.958505in}}%
\pgfpathlineto{\pgfqpoint{3.318814in}{0.940071in}}%
\pgfpathlineto{\pgfqpoint{3.350141in}{0.921910in}}%
\pgfpathlineto{\pgfqpoint{3.376149in}{0.904108in}}%
\pgfpathlineto{\pgfqpoint{3.397428in}{0.886746in}}%
\pgfpathlineto{\pgfqpoint{3.414118in}{0.869899in}}%
\pgfpathlineto{\pgfqpoint{3.427053in}{0.853602in}}%
\pgfpathlineto{\pgfqpoint{3.436917in}{0.837885in}}%
\pgfpathlineto{\pgfqpoint{3.444191in}{0.822774in}}%
\pgfpathlineto{\pgfqpoint{3.449164in}{0.808289in}}%
\pgfpathlineto{\pgfqpoint{3.451924in}{0.794450in}}%
\pgfpathlineto{\pgfqpoint{3.452362in}{0.781267in}}%
\pgfpathlineto{\pgfqpoint{3.450172in}{0.768751in}}%
\pgfpathlineto{\pgfqpoint{3.444871in}{0.756907in}}%
\pgfpathlineto{\pgfqpoint{3.436999in}{0.745751in}}%
\pgfpathlineto{\pgfqpoint{3.427128in}{0.735297in}}%
\pgfpathlineto{\pgfqpoint{3.415343in}{0.725547in}}%
\pgfpathlineto{\pgfqpoint{3.401697in}{0.716501in}}%
\pgfpathlineto{\pgfqpoint{3.386210in}{0.708160in}}%
\pgfpathlineto{\pgfqpoint{3.359496in}{0.696972in}}%
\pgfpathlineto{\pgfqpoint{3.328441in}{0.687368in}}%
\pgfpathlineto{\pgfqpoint{3.292713in}{0.679344in}}%
\pgfpathlineto{\pgfqpoint{3.252097in}{0.672902in}}%
\pgfpathlineto{\pgfqpoint{3.206419in}{0.668068in}}%
\pgfpathlineto{\pgfqpoint{3.155211in}{0.664872in}}%
\pgfpathlineto{\pgfqpoint{3.097958in}{0.663352in}}%
\pgfpathlineto{\pgfqpoint{3.034095in}{0.663558in}}%
\pgfpathlineto{\pgfqpoint{2.963007in}{0.665549in}}%
\pgfpathlineto{\pgfqpoint{2.855831in}{0.671102in}}%
\pgfpathlineto{\pgfqpoint{2.732935in}{0.680146in}}%
\pgfpathlineto{\pgfqpoint{2.592896in}{0.692931in}}%
\pgfpathlineto{\pgfqpoint{2.435397in}{0.709718in}}%
\pgfpathlineto{\pgfqpoint{2.262423in}{0.730704in}}%
\pgfpathlineto{\pgfqpoint{2.125433in}{0.749251in}}%
\pgfpathlineto{\pgfqpoint{1.986355in}{0.770138in}}%
\pgfpathlineto{\pgfqpoint{1.849649in}{0.793191in}}%
\pgfpathlineto{\pgfqpoint{1.762234in}{0.809628in}}%
\pgfpathlineto{\pgfqpoint{1.679637in}{0.826804in}}%
\pgfpathlineto{\pgfqpoint{1.603008in}{0.844595in}}%
\pgfpathlineto{\pgfqpoint{1.532874in}{0.862832in}}%
\pgfpathlineto{\pgfqpoint{1.469579in}{0.881362in}}%
\pgfpathlineto{\pgfqpoint{1.413291in}{0.900040in}}%
\pgfpathlineto{\pgfqpoint{1.364005in}{0.918736in}}%
\pgfpathlineto{\pgfqpoint{1.321539in}{0.937333in}}%
\pgfpathlineto{\pgfqpoint{1.285537in}{0.955725in}}%
\pgfpathlineto{\pgfqpoint{1.255466in}{0.973819in}}%
\pgfpathlineto{\pgfqpoint{1.230622in}{0.991534in}}%
\pgfpathlineto{\pgfqpoint{1.210371in}{1.008796in}}%
\pgfpathlineto{\pgfqpoint{1.194625in}{1.025531in}}%
\pgfpathlineto{\pgfqpoint{1.182588in}{1.041703in}}%
\pgfpathlineto{\pgfqpoint{1.173571in}{1.057286in}}%
\pgfpathlineto{\pgfqpoint{1.167080in}{1.072256in}}%
\pgfpathlineto{\pgfqpoint{1.162812in}{1.086592in}}%
\pgfpathlineto{\pgfqpoint{1.160658in}{1.100280in}}%
\pgfpathlineto{\pgfqpoint{1.160700in}{1.113308in}}%
\pgfpathlineto{\pgfqpoint{1.163215in}{1.125667in}}%
\pgfpathlineto{\pgfqpoint{1.168671in}{1.137356in}}%
\pgfpathlineto{\pgfqpoint{1.176995in}{1.148361in}}%
\pgfpathlineto{\pgfqpoint{1.187324in}{1.158664in}}%
\pgfpathlineto{\pgfqpoint{1.199570in}{1.168263in}}%
\pgfpathlineto{\pgfqpoint{1.213681in}{1.177158in}}%
\pgfpathlineto{\pgfqpoint{1.229639in}{1.185346in}}%
\pgfpathlineto{\pgfqpoint{1.257067in}{1.196305in}}%
\pgfpathlineto{\pgfqpoint{1.288836in}{1.205676in}}%
\pgfpathlineto{\pgfqpoint{1.325255in}{1.213462in}}%
\pgfpathlineto{\pgfqpoint{1.366634in}{1.219663in}}%
\pgfpathlineto{\pgfqpoint{1.413108in}{1.224255in}}%
\pgfpathlineto{\pgfqpoint{1.465175in}{1.227204in}}%
\pgfpathlineto{\pgfqpoint{1.523384in}{1.228472in}}%
\pgfpathlineto{\pgfqpoint{1.588320in}{1.228006in}}%
\pgfpathlineto{\pgfqpoint{1.686439in}{1.224581in}}%
\pgfpathlineto{\pgfqpoint{1.799180in}{1.217785in}}%
\pgfpathlineto{\pgfqpoint{1.928191in}{1.207411in}}%
\pgfpathlineto{\pgfqpoint{2.074634in}{1.193202in}}%
\pgfpathlineto{\pgfqpoint{2.238198in}{1.174902in}}%
\pgfpathlineto{\pgfqpoint{2.415741in}{1.152360in}}%
\pgfpathlineto{\pgfqpoint{2.554414in}{1.132659in}}%
\pgfpathlineto{\pgfqpoint{2.693011in}{1.110687in}}%
\pgfpathlineto{\pgfqpoint{2.826698in}{1.086677in}}%
\pgfpathlineto{\pgfqpoint{2.910855in}{1.069699in}}%
\pgfpathlineto{\pgfqpoint{2.989563in}{1.052085in}}%
\pgfpathlineto{\pgfqpoint{3.061378in}{1.033979in}}%
\pgfpathlineto{\pgfqpoint{3.125678in}{1.015539in}}%
\pgfpathlineto{\pgfqpoint{3.182966in}{0.996903in}}%
\pgfpathlineto{\pgfqpoint{3.233704in}{0.978203in}}%
\pgfpathlineto{\pgfqpoint{3.278304in}{0.959557in}}%
\pgfpathlineto{\pgfqpoint{3.317124in}{0.941073in}}%
\pgfpathlineto{\pgfqpoint{3.350473in}{0.922847in}}%
\pgfpathlineto{\pgfqpoint{3.378607in}{0.904967in}}%
\pgfpathlineto{\pgfqpoint{3.401729in}{0.887506in}}%
\pgfpathlineto{\pgfqpoint{3.419993in}{0.870529in}}%
\pgfpathlineto{\pgfqpoint{3.433499in}{0.854091in}}%
\pgfpathlineto{\pgfqpoint{3.442802in}{0.838247in}}%
\pgfpathlineto{\pgfqpoint{3.448832in}{0.823037in}}%
\pgfpathlineto{\pgfqpoint{3.452140in}{0.808479in}}%
\pgfpathlineto{\pgfqpoint{3.453149in}{0.794587in}}%
\pgfpathlineto{\pgfqpoint{3.452151in}{0.781371in}}%
\pgfpathlineto{\pgfqpoint{3.449311in}{0.768840in}}%
\pgfpathlineto{\pgfqpoint{3.444667in}{0.756998in}}%
\pgfpathlineto{\pgfqpoint{3.438125in}{0.745846in}}%
\pgfpathlineto{\pgfqpoint{3.429463in}{0.735382in}}%
\pgfpathlineto{\pgfqpoint{3.418332in}{0.725600in}}%
\pgfpathlineto{\pgfqpoint{3.404640in}{0.716502in}}%
\pgfpathlineto{\pgfqpoint{3.389004in}{0.708108in}}%
\pgfpathlineto{\pgfqpoint{3.361987in}{0.696839in}}%
\pgfpathlineto{\pgfqpoint{3.330655in}{0.687163in}}%
\pgfpathlineto{\pgfqpoint{3.294857in}{0.679087in}}%
\pgfpathlineto{\pgfqpoint{3.254330in}{0.672619in}}%
\pgfpathlineto{\pgfqpoint{3.208696in}{0.667768in}}%
\pgfpathlineto{\pgfqpoint{3.157496in}{0.664546in}}%
\pgfpathlineto{\pgfqpoint{3.100582in}{0.662979in}}%
\pgfpathlineto{\pgfqpoint{3.037128in}{0.663128in}}%
\pgfpathlineto{\pgfqpoint{2.966199in}{0.665066in}}%
\pgfpathlineto{\pgfqpoint{2.858713in}{0.670564in}}%
\pgfpathlineto{\pgfqpoint{2.735295in}{0.679579in}}%
\pgfpathlineto{\pgfqpoint{2.595185in}{0.692332in}}%
\pgfpathlineto{\pgfqpoint{2.438173in}{0.709065in}}%
\pgfpathlineto{\pgfqpoint{2.264355in}{0.730057in}}%
\pgfpathlineto{\pgfqpoint{2.126109in}{0.748670in}}%
\pgfpathlineto{\pgfqpoint{1.986793in}{0.769599in}}%
\pgfpathlineto{\pgfqpoint{1.851020in}{0.792635in}}%
\pgfpathlineto{\pgfqpoint{1.722780in}{0.817500in}}%
\pgfpathlineto{\pgfqpoint{1.643161in}{0.834927in}}%
\pgfpathlineto{\pgfqpoint{1.569244in}{0.852896in}}%
\pgfpathlineto{\pgfqpoint{1.501752in}{0.871276in}}%
\pgfpathlineto{\pgfqpoint{1.441282in}{0.889921in}}%
\pgfpathlineto{\pgfqpoint{1.388311in}{0.908673in}}%
\pgfpathlineto{\pgfqpoint{1.342714in}{0.927380in}}%
\pgfpathlineto{\pgfqpoint{1.303652in}{0.945931in}}%
\pgfpathlineto{\pgfqpoint{1.270460in}{0.964230in}}%
\pgfpathlineto{\pgfqpoint{1.242572in}{0.982186in}}%
\pgfpathlineto{\pgfqpoint{1.219522in}{0.999724in}}%
\pgfpathlineto{\pgfqpoint{1.200939in}{1.016778in}}%
\pgfpathlineto{\pgfqpoint{1.186537in}{1.033292in}}%
\pgfpathlineto{\pgfqpoint{1.175800in}{1.049224in}}%
\pgfpathlineto{\pgfqpoint{1.168253in}{1.064543in}}%
\pgfpathlineto{\pgfqpoint{1.163563in}{1.079226in}}%
\pgfpathlineto{\pgfqpoint{1.161476in}{1.093253in}}%
\pgfpathlineto{\pgfqpoint{1.161817in}{1.106610in}}%
\pgfpathlineto{\pgfqpoint{1.164487in}{1.119287in}}%
\pgfpathlineto{\pgfqpoint{1.169455in}{1.131279in}}%
\pgfpathlineto{\pgfqpoint{1.176602in}{1.142580in}}%
\pgfpathlineto{\pgfqpoint{1.185778in}{1.153187in}}%
\pgfpathlineto{\pgfqpoint{1.196872in}{1.163096in}}%
\pgfpathlineto{\pgfqpoint{1.209811in}{1.172305in}}%
\pgfpathlineto{\pgfqpoint{1.224561in}{1.180811in}}%
\pgfpathlineto{\pgfqpoint{1.250103in}{1.192255in}}%
\pgfpathlineto{\pgfqpoint{1.279920in}{1.202121in}}%
\pgfpathlineto{\pgfqpoint{1.314386in}{1.210418in}}%
\pgfpathlineto{\pgfqpoint{1.353898in}{1.217152in}}%
\pgfpathlineto{\pgfqpoint{1.398428in}{1.222289in}}%
\pgfpathlineto{\pgfqpoint{1.448390in}{1.225799in}}%
\pgfpathlineto{\pgfqpoint{1.504291in}{1.227646in}}%
\pgfpathlineto{\pgfqpoint{1.566684in}{1.227782in}}%
\pgfpathlineto{\pgfqpoint{1.636173in}{1.226150in}}%
\pgfpathlineto{\pgfqpoint{1.740993in}{1.221103in}}%
\pgfpathlineto{\pgfqpoint{1.861264in}{1.212601in}}%
\pgfpathlineto{\pgfqpoint{1.998480in}{1.200405in}}%
\pgfpathlineto{\pgfqpoint{2.153205in}{1.184249in}}%
\pgfpathlineto{\pgfqpoint{2.323830in}{1.163924in}}%
\pgfpathlineto{\pgfqpoint{2.459703in}{1.145878in}}%
\pgfpathlineto{\pgfqpoint{2.598501in}{1.125475in}}%
\pgfpathlineto{\pgfqpoint{2.735880in}{1.102864in}}%
\pgfpathlineto{\pgfqpoint{2.867016in}{1.078308in}}%
\pgfpathlineto{\pgfqpoint{2.948382in}{1.061015in}}%
\pgfpathlineto{\pgfqpoint{3.023704in}{1.043143in}}%
\pgfpathlineto{\pgfqpoint{3.092475in}{1.024858in}}%
\pgfpathlineto{\pgfqpoint{3.154369in}{1.006310in}}%
\pgfpathlineto{\pgfqpoint{3.209240in}{0.987640in}}%
\pgfpathlineto{\pgfqpoint{3.257126in}{0.968975in}}%
\pgfpathlineto{\pgfqpoint{3.298243in}{0.950430in}}%
\pgfpathlineto{\pgfqpoint{3.332990in}{0.932108in}}%
\pgfpathlineto{\pgfqpoint{3.361949in}{0.914100in}}%
\pgfpathlineto{\pgfqpoint{3.385878in}{0.896483in}}%
\pgfpathlineto{\pgfqpoint{3.405178in}{0.879339in}}%
\pgfpathlineto{\pgfqpoint{3.420133in}{0.862729in}}%
\pgfpathlineto{\pgfqpoint{3.431569in}{0.846684in}}%
\pgfpathlineto{\pgfqpoint{3.440119in}{0.831233in}}%
\pgfpathlineto{\pgfqpoint{3.446223in}{0.816398in}}%
\pgfpathlineto{\pgfqpoint{3.450126in}{0.802199in}}%
\pgfpathlineto{\pgfqpoint{3.451880in}{0.788650in}}%
\pgfpathlineto{\pgfqpoint{3.451344in}{0.775763in}}%
\pgfpathlineto{\pgfqpoint{3.448182in}{0.763545in}}%
\pgfpathlineto{\pgfqpoint{3.441910in}{0.751998in}}%
\pgfpathlineto{\pgfqpoint{3.433168in}{0.741143in}}%
\pgfpathlineto{\pgfqpoint{3.422454in}{0.730990in}}%
\pgfpathlineto{\pgfqpoint{3.409840in}{0.721541in}}%
\pgfpathlineto{\pgfqpoint{3.395369in}{0.712796in}}%
\pgfpathlineto{\pgfqpoint{3.370198in}{0.701003in}}%
\pgfpathlineto{\pgfqpoint{3.340783in}{0.690797in}}%
\pgfpathlineto{\pgfqpoint{3.306885in}{0.682177in}}%
\pgfpathlineto{\pgfqpoint{3.268136in}{0.675136in}}%
\pgfpathlineto{\pgfqpoint{3.224425in}{0.669691in}}%
\pgfpathlineto{\pgfqpoint{3.175386in}{0.665870in}}%
\pgfpathlineto{\pgfqpoint{3.120500in}{0.663708in}}%
\pgfpathlineto{\pgfqpoint{3.059207in}{0.663252in}}%
\pgfpathlineto{\pgfqpoint{2.990908in}{0.664556in}}%
\pgfpathlineto{\pgfqpoint{2.887841in}{0.669152in}}%
\pgfpathlineto{\pgfqpoint{2.769567in}{0.677185in}}%
\pgfpathlineto{\pgfqpoint{2.634499in}{0.688879in}}%
\pgfpathlineto{\pgfqpoint{2.481860in}{0.704507in}}%
\pgfpathlineto{\pgfqpoint{2.312890in}{0.724290in}}%
\pgfpathlineto{\pgfqpoint{2.177805in}{0.741928in}}%
\pgfpathlineto{\pgfqpoint{2.038903in}{0.761947in}}%
\pgfpathlineto{\pgfqpoint{1.900760in}{0.784210in}}%
\pgfpathlineto{\pgfqpoint{1.768067in}{0.808474in}}%
\pgfpathlineto{\pgfqpoint{1.684921in}{0.825593in}}%
\pgfpathlineto{\pgfqpoint{1.607787in}{0.843321in}}%
\pgfpathlineto{\pgfqpoint{1.537900in}{0.861516in}}%
\pgfpathlineto{\pgfqpoint{1.475052in}{0.880021in}}%
\pgfpathlineto{\pgfqpoint{1.418942in}{0.898691in}}%
\pgfpathlineto{\pgfqpoint{1.369272in}{0.917394in}}%
\pgfpathlineto{\pgfqpoint{1.325750in}{0.936010in}}%
\pgfpathlineto{\pgfqpoint{1.288088in}{0.954432in}}%
\pgfpathlineto{\pgfqpoint{1.256001in}{0.972567in}}%
\pgfpathlineto{\pgfqpoint{1.229209in}{0.990333in}}%
\pgfpathlineto{\pgfqpoint{1.207436in}{1.007662in}}%
\pgfpathlineto{\pgfqpoint{1.190408in}{1.024499in}}%
\pgfpathlineto{\pgfqpoint{1.177856in}{1.040799in}}%
\pgfpathlineto{\pgfqpoint{1.169271in}{1.056499in}}%
\pgfpathlineto{\pgfqpoint{1.163918in}{1.071562in}}%
\pgfpathlineto{\pgfqpoint{1.161195in}{1.085971in}}%
\pgfpathlineto{\pgfqpoint{1.160642in}{1.099714in}}%
\pgfpathlineto{\pgfqpoint{1.161946in}{1.112781in}}%
\pgfpathlineto{\pgfqpoint{1.164934in}{1.125165in}}%
\pgfpathlineto{\pgfqpoint{1.169580in}{1.136864in}}%
\pgfpathlineto{\pgfqpoint{1.175998in}{1.147876in}}%
\pgfpathlineto{\pgfqpoint{1.184450in}{1.158205in}}%
\pgfpathlineto{\pgfqpoint{1.195339in}{1.167856in}}%
\pgfpathlineto{\pgfqpoint{1.209129in}{1.176837in}}%
\pgfpathlineto{\pgfqpoint{1.225143in}{1.185122in}}%
\pgfpathlineto{\pgfqpoint{1.252741in}{1.196225in}}%
\pgfpathlineto{\pgfqpoint{1.284682in}{1.205733in}}%
\pgfpathlineto{\pgfqpoint{1.321121in}{1.213639in}}%
\pgfpathlineto{\pgfqpoint{1.362321in}{1.219932in}}%
\pgfpathlineto{\pgfqpoint{1.408650in}{1.224600in}}%
\pgfpathlineto{\pgfqpoint{1.460586in}{1.227630in}}%
\pgfpathlineto{\pgfqpoint{1.518321in}{1.229001in}}%
\pgfpathlineto{\pgfqpoint{1.582531in}{1.228655in}}%
\pgfpathlineto{\pgfqpoint{1.680062in}{1.225382in}}%
\pgfpathlineto{\pgfqpoint{1.792914in}{1.218715in}}%
\pgfpathlineto{\pgfqpoint{1.922228in}{1.208446in}}%
\pgfpathlineto{\pgfqpoint{2.068381in}{1.194356in}}%
\pgfpathlineto{\pgfqpoint{2.230981in}{1.176214in}}%
\pgfpathlineto{\pgfqpoint{2.408916in}{1.153763in}}%
\pgfpathlineto{\pgfqpoint{2.548573in}{1.134052in}}%
\pgfpathlineto{\pgfqpoint{2.687528in}{1.112095in}}%
\pgfpathlineto{\pgfqpoint{2.821032in}{1.088157in}}%
\pgfpathlineto{\pgfqpoint{2.905029in}{1.071252in}}%
\pgfpathlineto{\pgfqpoint{2.983872in}{1.053716in}}%
\pgfpathlineto{\pgfqpoint{3.056755in}{1.035665in}}%
\pgfpathlineto{\pgfqpoint{3.123033in}{1.017230in}}%
\pgfpathlineto{\pgfqpoint{3.182223in}{0.998549in}}%
\pgfpathlineto{\pgfqpoint{3.234001in}{0.979772in}}%
\pgfpathlineto{\pgfqpoint{3.278338in}{0.961057in}}%
\pgfpathlineto{\pgfqpoint{3.316043in}{0.942528in}}%
\pgfpathlineto{\pgfqpoint{3.347990in}{0.924274in}}%
\pgfpathlineto{\pgfqpoint{3.374884in}{0.906375in}}%
\pgfpathlineto{\pgfqpoint{3.397258in}{0.888902in}}%
\pgfpathlineto{\pgfqpoint{3.415478in}{0.871918in}}%
\pgfpathlineto{\pgfqpoint{3.429742in}{0.855476in}}%
\pgfpathlineto{\pgfqpoint{3.440080in}{0.839618in}}%
\pgfpathlineto{\pgfqpoint{3.446927in}{0.824383in}}%
\pgfpathlineto{\pgfqpoint{3.450910in}{0.809794in}}%
\pgfpathlineto{\pgfqpoint{3.452322in}{0.795867in}}%
\pgfpathlineto{\pgfqpoint{3.451389in}{0.782614in}}%
\pgfpathlineto{\pgfqpoint{3.448262in}{0.770046in}}%
\pgfpathlineto{\pgfqpoint{3.443028in}{0.758168in}}%
\pgfpathlineto{\pgfqpoint{3.435699in}{0.746984in}}%
\pgfpathlineto{\pgfqpoint{3.426235in}{0.736495in}}%
\pgfpathlineto{\pgfqpoint{3.414734in}{0.726700in}}%
\pgfpathlineto{\pgfqpoint{3.401293in}{0.717603in}}%
\pgfpathlineto{\pgfqpoint{3.385964in}{0.709207in}}%
\pgfpathlineto{\pgfqpoint{3.359473in}{0.697931in}}%
\pgfpathlineto{\pgfqpoint{3.328734in}{0.688241in}}%
\pgfpathlineto{\pgfqpoint{3.293563in}{0.680142in}}%
\pgfpathlineto{\pgfqpoint{3.253628in}{0.673636in}}%
\pgfpathlineto{\pgfqpoint{3.208457in}{0.668723in}}%
\pgfpathlineto{\pgfqpoint{3.157813in}{0.665422in}}%
\pgfpathlineto{\pgfqpoint{3.101306in}{0.663782in}}%
\pgfpathlineto{\pgfqpoint{3.038156in}{0.663856in}}%
\pgfpathlineto{\pgfqpoint{2.967650in}{0.665707in}}%
\pgfpathlineto{\pgfqpoint{2.861090in}{0.671066in}}%
\pgfpathlineto{\pgfqpoint{2.738960in}{0.679914in}}%
\pgfpathlineto{\pgfqpoint{2.600120in}{0.692485in}}%
\pgfpathlineto{\pgfqpoint{2.442789in}{0.709063in}}%
\pgfpathlineto{\pgfqpoint{2.266233in}{0.729971in}}%
\pgfpathlineto{\pgfqpoint{2.127869in}{0.748456in}}%
\pgfpathlineto{\pgfqpoint{1.989628in}{0.769213in}}%
\pgfpathlineto{\pgfqpoint{1.855553in}{0.792056in}}%
\pgfpathlineto{\pgfqpoint{1.729087in}{0.816731in}}%
\pgfpathlineto{\pgfqpoint{1.650430in}{0.834047in}}%
\pgfpathlineto{\pgfqpoint{1.577120in}{0.851929in}}%
\pgfpathlineto{\pgfqpoint{1.509732in}{0.870254in}}%
\pgfpathlineto{\pgfqpoint{1.448719in}{0.888889in}}%
\pgfpathlineto{\pgfqpoint{1.394415in}{0.907683in}}%
\pgfpathlineto{\pgfqpoint{1.347037in}{0.926479in}}%
\pgfpathlineto{\pgfqpoint{1.306681in}{0.945101in}}%
\pgfpathlineto{\pgfqpoint{1.273078in}{0.963398in}}%
\pgfpathlineto{\pgfqpoint{1.245042in}{0.981346in}}%
\pgfpathlineto{\pgfqpoint{1.221954in}{0.998877in}}%
\pgfpathlineto{\pgfqpoint{1.203313in}{1.015927in}}%
\pgfpathlineto{\pgfqpoint{1.188661in}{1.032442in}}%
\pgfpathlineto{\pgfqpoint{1.177582in}{1.048382in}}%
\pgfpathlineto{\pgfqpoint{1.169726in}{1.063714in}}%
\pgfpathlineto{\pgfqpoint{1.164804in}{1.078411in}}%
\pgfpathlineto{\pgfqpoint{1.162701in}{1.092456in}}%
\pgfpathlineto{\pgfqpoint{1.163173in}{1.105833in}}%
\pgfpathlineto{\pgfqpoint{1.165920in}{1.118533in}}%
\pgfpathlineto{\pgfqpoint{1.170716in}{1.130546in}}%
\pgfpathlineto{\pgfqpoint{1.177406in}{1.141868in}}%
\pgfpathlineto{\pgfqpoint{1.185910in}{1.152495in}}%
\pgfpathlineto{\pgfqpoint{1.196218in}{1.162426in}}%
\pgfpathlineto{\pgfqpoint{1.208395in}{1.171663in}}%
\pgfpathlineto{\pgfqpoint{1.222579in}{1.180210in}}%
\pgfpathlineto{\pgfqpoint{1.238977in}{1.188074in}}%
\pgfpathlineto{\pgfqpoint{1.267614in}{1.198577in}}%
\pgfpathlineto{\pgfqpoint{1.300719in}{1.207507in}}%
\pgfpathlineto{\pgfqpoint{1.338455in}{1.214848in}}%
\pgfpathlineto{\pgfqpoint{1.381082in}{1.220585in}}%
\pgfpathlineto{\pgfqpoint{1.428953in}{1.224694in}}%
\pgfpathlineto{\pgfqpoint{1.482518in}{1.227150in}}%
\pgfpathlineto{\pgfqpoint{1.542323in}{1.227921in}}%
\pgfpathlineto{\pgfqpoint{1.632477in}{1.226298in}}%
\pgfpathlineto{\pgfqpoint{1.735994in}{1.221512in}}%
\pgfpathlineto{\pgfqpoint{1.856017in}{1.213191in}}%
\pgfpathlineto{\pgfqpoint{1.993924in}{1.201043in}}%
\pgfpathlineto{\pgfqpoint{2.149134in}{1.184867in}}%
\pgfpathlineto{\pgfqpoint{2.319112in}{1.164559in}}%
\pgfpathlineto{\pgfqpoint{2.453657in}{1.146606in}}%
\pgfpathlineto{\pgfqpoint{2.591404in}{1.126351in}}%
\pgfpathlineto{\pgfqpoint{2.728790in}{1.103853in}}%
\pgfpathlineto{\pgfqpoint{2.860122in}{1.079353in}}%
\pgfpathlineto{\pgfqpoint{2.942022in}{1.062129in}}%
\pgfpathlineto{\pgfqpoint{3.018204in}{1.044350in}}%
\pgfpathlineto{\pgfqpoint{3.087926in}{1.026150in}}%
\pgfpathlineto{\pgfqpoint{3.150697in}{1.007658in}}%
\pgfpathlineto{\pgfqpoint{3.206275in}{0.989005in}}%
\pgfpathlineto{\pgfqpoint{3.254670in}{0.970318in}}%
\pgfpathlineto{\pgfqpoint{3.296142in}{0.951722in}}%
\pgfpathlineto{\pgfqpoint{3.330971in}{0.933357in}}%
\pgfpathlineto{\pgfqpoint{3.359779in}{0.915334in}}%
\pgfpathlineto{\pgfqpoint{3.383713in}{0.897710in}}%
\pgfpathlineto{\pgfqpoint{3.403666in}{0.880533in}}%
\pgfpathlineto{\pgfqpoint{3.420260in}{0.863850in}}%
\pgfpathlineto{\pgfqpoint{3.433852in}{0.847703in}}%
\pgfpathlineto{\pgfqpoint{3.444531in}{0.832129in}}%
\pgfpathlineto{\pgfqpoint{3.452118in}{0.817161in}}%
\pgfpathlineto{\pgfqpoint{3.456166in}{0.802830in}}%
\pgfpathlineto{\pgfqpoint{3.456297in}{0.789160in}}%
\pgfpathlineto{\pgfqpoint{3.453832in}{0.776178in}}%
\pgfpathlineto{\pgfqpoint{3.449184in}{0.763892in}}%
\pgfpathlineto{\pgfqpoint{3.442503in}{0.752307in}}%
\pgfpathlineto{\pgfqpoint{3.433891in}{0.741424in}}%
\pgfpathlineto{\pgfqpoint{3.423406in}{0.731247in}}%
\pgfpathlineto{\pgfqpoint{3.411059in}{0.721775in}}%
\pgfpathlineto{\pgfqpoint{3.396816in}{0.713006in}}%
\pgfpathlineto{\pgfqpoint{3.380595in}{0.704937in}}%
\pgfpathlineto{\pgfqpoint{3.352463in}{0.694142in}}%
\pgfpathlineto{\pgfqpoint{3.319891in}{0.684927in}}%
\pgfpathlineto{\pgfqpoint{3.282738in}{0.677303in}}%
\pgfpathlineto{\pgfqpoint{3.240760in}{0.671288in}}%
\pgfpathlineto{\pgfqpoint{3.193617in}{0.666903in}}%
\pgfpathlineto{\pgfqpoint{3.140868in}{0.664174in}}%
\pgfpathlineto{\pgfqpoint{3.081981in}{0.663131in}}%
\pgfpathlineto{\pgfqpoint{3.016325in}{0.663811in}}%
\pgfpathlineto{\pgfqpoint{2.917646in}{0.667458in}}%
\pgfpathlineto{\pgfqpoint{2.803831in}{0.674493in}}%
\pgfpathlineto{\pgfqpoint{2.672678in}{0.685208in}}%
\pgfpathlineto{\pgfqpoint{2.523848in}{0.699836in}}%
\pgfpathlineto{\pgfqpoint{2.358869in}{0.718546in}}%
\pgfpathlineto{\pgfqpoint{2.181135in}{0.741442in}}%
\pgfpathlineto{\pgfqpoint{2.042598in}{0.761392in}}%
\pgfpathlineto{\pgfqpoint{1.904188in}{0.783673in}}%
\pgfpathlineto{\pgfqpoint{1.771326in}{0.807948in}}%
\pgfpathlineto{\pgfqpoint{1.688026in}{0.825041in}}%
\pgfpathlineto{\pgfqpoint{1.610168in}{0.842719in}}%
\pgfpathlineto{\pgfqpoint{1.538552in}{0.860859in}}%
\pgfpathlineto{\pgfqpoint{1.473764in}{0.879333in}}%
\pgfpathlineto{\pgfqpoint{1.416173in}{0.898012in}}%
\pgfpathlineto{\pgfqpoint{1.365932in}{0.916763in}}%
\pgfpathlineto{\pgfqpoint{1.322976in}{0.935451in}}%
\pgfpathlineto{\pgfqpoint{1.286949in}{0.953928in}}%
\pgfpathlineto{\pgfqpoint{1.256860in}{0.972089in}}%
\pgfpathlineto{\pgfqpoint{1.231712in}{0.989868in}}%
\pgfpathlineto{\pgfqpoint{1.210743in}{1.007207in}}%
\pgfpathlineto{\pgfqpoint{1.193427in}{1.024052in}}%
\pgfpathlineto{\pgfqpoint{1.179471in}{1.040357in}}%
\pgfpathlineto{\pgfqpoint{1.168821in}{1.056082in}}%
\pgfpathlineto{\pgfqpoint{1.161657in}{1.071190in}}%
\pgfpathlineto{\pgfqpoint{1.158268in}{1.085655in}}%
\pgfpathlineto{\pgfqpoint{1.157750in}{1.099448in}}%
\pgfpathlineto{\pgfqpoint{1.159600in}{1.112558in}}%
\pgfpathlineto{\pgfqpoint{1.163621in}{1.124976in}}%
\pgfpathlineto{\pgfqpoint{1.169669in}{1.136698in}}%
\pgfpathlineto{\pgfqpoint{1.177656in}{1.147720in}}%
\pgfpathlineto{\pgfqpoint{1.187547in}{1.158039in}}%
\pgfpathlineto{\pgfqpoint{1.199364in}{1.167658in}}%
\pgfpathlineto{\pgfqpoint{1.213183in}{1.176579in}}%
\pgfpathlineto{\pgfqpoint{1.229055in}{1.184804in}}%
\pgfpathlineto{\pgfqpoint{1.256501in}{1.195829in}}%
\pgfpathlineto{\pgfqpoint{1.288334in}{1.205273in}}%
\pgfpathlineto{\pgfqpoint{1.324683in}{1.213124in}}%
\pgfpathlineto{\pgfqpoint{1.365788in}{1.219367in}}%
\pgfpathlineto{\pgfqpoint{1.411995in}{1.223986in}}%
\pgfpathlineto{\pgfqpoint{1.463757in}{1.226962in}}%
\pgfpathlineto{\pgfqpoint{1.521635in}{1.228272in}}%
\pgfpathlineto{\pgfqpoint{1.585841in}{1.227900in}}%
\pgfpathlineto{\pgfqpoint{1.682670in}{1.224645in}}%
\pgfpathlineto{\pgfqpoint{1.795025in}{1.217984in}}%
\pgfpathlineto{\pgfqpoint{1.924507in}{1.207675in}}%
\pgfpathlineto{\pgfqpoint{2.071276in}{1.193509in}}%
\pgfpathlineto{\pgfqpoint{2.234054in}{1.175310in}}%
\pgfpathlineto{\pgfqpoint{2.410121in}{1.152939in}}%
\pgfpathlineto{\pgfqpoint{2.548334in}{1.133358in}}%
\pgfpathlineto{\pgfqpoint{2.687205in}{1.111444in}}%
\pgfpathlineto{\pgfqpoint{2.821163in}{1.087502in}}%
\pgfpathlineto{\pgfqpoint{2.905460in}{1.070591in}}%
\pgfpathlineto{\pgfqpoint{2.984469in}{1.053053in}}%
\pgfpathlineto{\pgfqpoint{3.057332in}{1.035012in}}%
\pgfpathlineto{\pgfqpoint{3.123407in}{1.016598in}}%
\pgfpathlineto{\pgfqpoint{3.182269in}{0.997949in}}%
\pgfpathlineto{\pgfqpoint{3.233705in}{0.979209in}}%
\pgfpathlineto{\pgfqpoint{3.277779in}{0.960538in}}%
\pgfpathlineto{\pgfqpoint{3.315327in}{0.942052in}}%
\pgfpathlineto{\pgfqpoint{3.347263in}{0.923835in}}%
\pgfpathlineto{\pgfqpoint{3.374305in}{0.905968in}}%
\pgfpathlineto{\pgfqpoint{3.396971in}{0.888519in}}%
\pgfpathlineto{\pgfqpoint{3.415580in}{0.871550in}}%
\pgfpathlineto{\pgfqpoint{3.430253in}{0.855116in}}%
\pgfpathlineto{\pgfqpoint{3.440914in}{0.839263in}}%
\pgfpathlineto{\pgfqpoint{3.447684in}{0.824030in}}%
\pgfpathlineto{\pgfqpoint{3.451523in}{0.809446in}}%
\pgfpathlineto{\pgfqpoint{3.452785in}{0.795525in}}%
\pgfpathlineto{\pgfqpoint{3.451718in}{0.782281in}}%
\pgfpathlineto{\pgfqpoint{3.448499in}{0.769723in}}%
\pgfpathlineto{\pgfqpoint{3.443230in}{0.757856in}}%
\pgfpathlineto{\pgfqpoint{3.435945in}{0.746684in}}%
\pgfpathlineto{\pgfqpoint{3.426603in}{0.736208in}}%
\pgfpathlineto{\pgfqpoint{3.415121in}{0.726425in}}%
\pgfpathlineto{\pgfqpoint{3.401637in}{0.717339in}}%
\pgfpathlineto{\pgfqpoint{3.386233in}{0.708952in}}%
\pgfpathlineto{\pgfqpoint{3.359575in}{0.697688in}}%
\pgfpathlineto{\pgfqpoint{3.328626in}{0.688009in}}%
\pgfpathlineto{\pgfqpoint{3.293236in}{0.679924in}}%
\pgfpathlineto{\pgfqpoint{3.253129in}{0.673439in}}%
\pgfpathlineto{\pgfqpoint{3.207902in}{0.668560in}}%
\pgfpathlineto{\pgfqpoint{3.157072in}{0.665297in}}%
\pgfpathlineto{\pgfqpoint{3.100505in}{0.663684in}}%
\pgfpathlineto{\pgfqpoint{3.037395in}{0.663781in}}%
\pgfpathlineto{\pgfqpoint{2.966857in}{0.665659in}}%
\pgfpathlineto{\pgfqpoint{2.860011in}{0.671063in}}%
\pgfpathlineto{\pgfqpoint{2.737361in}{0.679968in}}%
\pgfpathlineto{\pgfqpoint{2.598070in}{0.692596in}}%
\pgfpathlineto{\pgfqpoint{2.441782in}{0.709194in}}%
\pgfpathlineto{\pgfqpoint{2.268276in}{0.730047in}}%
\pgfpathlineto{\pgfqpoint{2.130487in}{0.748535in}}%
\pgfpathlineto{\pgfqpoint{1.991644in}{0.769334in}}%
\pgfpathlineto{\pgfqpoint{1.856170in}{0.792247in}}%
\pgfpathlineto{\pgfqpoint{1.727951in}{0.817007in}}%
\pgfpathlineto{\pgfqpoint{1.648175in}{0.834374in}}%
\pgfpathlineto{\pgfqpoint{1.573978in}{0.852294in}}%
\pgfpathlineto{\pgfqpoint{1.506111in}{0.870632in}}%
\pgfpathlineto{\pgfqpoint{1.445217in}{0.889242in}}%
\pgfpathlineto{\pgfqpoint{1.391834in}{0.907962in}}%
\pgfpathlineto{\pgfqpoint{1.345832in}{0.926645in}}%
\pgfpathlineto{\pgfqpoint{1.306371in}{0.945181in}}%
\pgfpathlineto{\pgfqpoint{1.272823in}{0.963471in}}%
\pgfpathlineto{\pgfqpoint{1.244640in}{0.981424in}}%
\pgfpathlineto{\pgfqpoint{1.221357in}{0.998964in}}%
\pgfpathlineto{\pgfqpoint{1.202591in}{1.016023in}}%
\pgfpathlineto{\pgfqpoint{1.187994in}{1.032546in}}%
\pgfpathlineto{\pgfqpoint{1.177025in}{1.048490in}}%
\pgfpathlineto{\pgfqpoint{1.169255in}{1.063825in}}%
\pgfpathlineto{\pgfqpoint{1.164359in}{1.078527in}}%
\pgfpathlineto{\pgfqpoint{1.162093in}{1.092575in}}%
\pgfpathlineto{\pgfqpoint{1.162289in}{1.105955in}}%
\pgfpathlineto{\pgfqpoint{1.164862in}{1.118657in}}%
\pgfpathlineto{\pgfqpoint{1.169750in}{1.130675in}}%
\pgfpathlineto{\pgfqpoint{1.176791in}{1.142003in}}%
\pgfpathlineto{\pgfqpoint{1.185835in}{1.152637in}}%
\pgfpathlineto{\pgfqpoint{1.196772in}{1.162573in}}%
\pgfpathlineto{\pgfqpoint{1.209536in}{1.171809in}}%
\pgfpathlineto{\pgfqpoint{1.224102in}{1.180343in}}%
\pgfpathlineto{\pgfqpoint{1.249379in}{1.191831in}}%
\pgfpathlineto{\pgfqpoint{1.278992in}{1.201747in}}%
\pgfpathlineto{\pgfqpoint{1.313387in}{1.210104in}}%
\pgfpathlineto{\pgfqpoint{1.352736in}{1.216894in}}%
\pgfpathlineto{\pgfqpoint{1.397091in}{1.222084in}}%
\pgfpathlineto{\pgfqpoint{1.446871in}{1.225648in}}%
\pgfpathlineto{\pgfqpoint{1.502561in}{1.227548in}}%
\pgfpathlineto{\pgfqpoint{1.564703in}{1.227738in}}%
\pgfpathlineto{\pgfqpoint{1.633899in}{1.226162in}}%
\pgfpathlineto{\pgfqpoint{1.738291in}{1.221199in}}%
\pgfpathlineto{\pgfqpoint{1.858113in}{1.212794in}}%
\pgfpathlineto{\pgfqpoint{1.994883in}{1.200697in}}%
\pgfpathlineto{\pgfqpoint{2.149123in}{1.184648in}}%
\pgfpathlineto{\pgfqpoint{2.319365in}{1.164436in}}%
\pgfpathlineto{\pgfqpoint{2.455017in}{1.146476in}}%
\pgfpathlineto{\pgfqpoint{2.593725in}{1.126154in}}%
\pgfpathlineto{\pgfqpoint{2.731238in}{1.103620in}}%
\pgfpathlineto{\pgfqpoint{2.862615in}{1.079126in}}%
\pgfpathlineto{\pgfqpoint{2.944220in}{1.061858in}}%
\pgfpathlineto{\pgfqpoint{3.019924in}{1.044013in}}%
\pgfpathlineto{\pgfqpoint{3.089137in}{1.025751in}}%
\pgfpathlineto{\pgfqpoint{3.151472in}{1.007222in}}%
\pgfpathlineto{\pgfqpoint{3.206742in}{0.988563in}}%
\pgfpathlineto{\pgfqpoint{3.254965in}{0.969902in}}%
\pgfpathlineto{\pgfqpoint{3.296358in}{0.951352in}}%
\pgfpathlineto{\pgfqpoint{3.331343in}{0.933017in}}%
\pgfpathlineto{\pgfqpoint{3.360541in}{0.914989in}}%
\pgfpathlineto{\pgfqpoint{3.384756in}{0.897349in}}%
\pgfpathlineto{\pgfqpoint{3.404163in}{0.880185in}}%
\pgfpathlineto{\pgfqpoint{3.419331in}{0.863549in}}%
\pgfpathlineto{\pgfqpoint{3.431055in}{0.847475in}}%
\pgfpathlineto{\pgfqpoint{3.439930in}{0.831990in}}%
\pgfpathlineto{\pgfqpoint{3.446352in}{0.817118in}}%
\pgfpathlineto{\pgfqpoint{3.450518in}{0.802881in}}%
\pgfpathlineto{\pgfqpoint{3.452424in}{0.789293in}}%
\pgfpathlineto{\pgfqpoint{3.451871in}{0.776366in}}%
\pgfpathlineto{\pgfqpoint{3.448457in}{0.764108in}}%
\pgfpathlineto{\pgfqpoint{3.441955in}{0.752527in}}%
\pgfpathlineto{\pgfqpoint{3.433303in}{0.741644in}}%
\pgfpathlineto{\pgfqpoint{3.422692in}{0.731462in}}%
\pgfpathlineto{\pgfqpoint{3.410195in}{0.721985in}}%
\pgfpathlineto{\pgfqpoint{3.395849in}{0.713212in}}%
\pgfpathlineto{\pgfqpoint{3.379663in}{0.705145in}}%
\pgfpathlineto{\pgfqpoint{3.351871in}{0.694366in}}%
\pgfpathlineto{\pgfqpoint{3.319655in}{0.685170in}}%
\pgfpathlineto{\pgfqpoint{3.282659in}{0.677550in}}%
\pgfpathlineto{\pgfqpoint{3.240789in}{0.671520in}}%
\pgfpathlineto{\pgfqpoint{3.193734in}{0.667105in}}%
\pgfpathlineto{\pgfqpoint{3.141022in}{0.664336in}}%
\pgfpathlineto{\pgfqpoint{3.082128in}{0.663256in}}%
\pgfpathlineto{\pgfqpoint{3.016475in}{0.663915in}}%
\pgfpathlineto{\pgfqpoint{2.917332in}{0.667610in}}%
\pgfpathlineto{\pgfqpoint{2.803419in}{0.674682in}}%
\pgfpathlineto{\pgfqpoint{2.673075in}{0.685347in}}%
\pgfpathlineto{\pgfqpoint{2.525224in}{0.699874in}}%
\pgfpathlineto{\pgfqpoint{2.360497in}{0.718499in}}%
\pgfpathlineto{\pgfqpoint{2.182031in}{0.741379in}}%
\pgfpathlineto{\pgfqpoint{2.043317in}{0.761321in}}%
\pgfpathlineto{\pgfqpoint{1.905065in}{0.783514in}}%
\pgfpathlineto{\pgfqpoint{1.772035in}{0.807719in}}%
\pgfpathlineto{\pgfqpoint{1.688854in}{0.824821in}}%
\pgfpathlineto{\pgfqpoint{1.611488in}{0.842551in}}%
\pgfpathlineto{\pgfqpoint{1.540531in}{0.860744in}}%
\pgfpathlineto{\pgfqpoint{1.476387in}{0.879244in}}%
\pgfpathlineto{\pgfqpoint{1.419268in}{0.897908in}}%
\pgfpathlineto{\pgfqpoint{1.369200in}{0.916606in}}%
\pgfpathlineto{\pgfqpoint{1.326017in}{0.935218in}}%
\pgfpathlineto{\pgfqpoint{1.289364in}{0.953638in}}%
\pgfpathlineto{\pgfqpoint{1.258699in}{0.971772in}}%
\pgfpathlineto{\pgfqpoint{1.233287in}{0.989536in}}%
\pgfpathlineto{\pgfqpoint{1.212557in}{1.006851in}}%
\pgfpathlineto{\pgfqpoint{1.196365in}{1.023645in}}%
\pgfpathlineto{\pgfqpoint{1.183879in}{1.039883in}}%
\pgfpathlineto{\pgfqpoint{1.174420in}{1.055536in}}%
\pgfpathlineto{\pgfqpoint{1.167508in}{1.070580in}}%
\pgfpathlineto{\pgfqpoint{1.162857in}{1.084993in}}%
\pgfpathlineto{\pgfqpoint{1.160381in}{1.098760in}}%
\pgfpathlineto{\pgfqpoint{1.160188in}{1.111868in}}%
\pgfpathlineto{\pgfqpoint{1.162586in}{1.124309in}}%
\pgfpathlineto{\pgfqpoint{1.168068in}{1.136079in}}%
\pgfpathlineto{\pgfqpoint{1.176183in}{1.147159in}}%
\pgfpathlineto{\pgfqpoint{1.186290in}{1.157538in}}%
\pgfpathlineto{\pgfqpoint{1.198309in}{1.167213in}}%
\pgfpathlineto{\pgfqpoint{1.212189in}{1.176183in}}%
\pgfpathlineto{\pgfqpoint{1.227911in}{1.184447in}}%
\pgfpathlineto{\pgfqpoint{1.254981in}{1.195520in}}%
\pgfpathlineto{\pgfqpoint{1.286396in}{1.205007in}}%
\pgfpathlineto{\pgfqpoint{1.322484in}{1.212913in}}%
\pgfpathlineto{\pgfqpoint{1.363487in}{1.219235in}}%
\pgfpathlineto{\pgfqpoint{1.409572in}{1.223946in}}%
\pgfpathlineto{\pgfqpoint{1.461220in}{1.227018in}}%
\pgfpathlineto{\pgfqpoint{1.518961in}{1.228411in}}%
\pgfpathlineto{\pgfqpoint{1.583366in}{1.228074in}}%
\pgfpathlineto{\pgfqpoint{1.680677in}{1.224827in}}%
\pgfpathlineto{\pgfqpoint{1.792521in}{1.218224in}}%
\pgfpathlineto{\pgfqpoint{1.920576in}{1.208060in}}%
\pgfpathlineto{\pgfqpoint{2.066047in}{1.194071in}}%
\pgfpathlineto{\pgfqpoint{2.228665in}{1.176008in}}%
\pgfpathlineto{\pgfqpoint{2.405584in}{1.153706in}}%
\pgfpathlineto{\pgfqpoint{2.544002in}{1.134185in}}%
\pgfpathlineto{\pgfqpoint{2.682687in}{1.112380in}}%
\pgfpathlineto{\pgfqpoint{2.817007in}{1.088513in}}%
\pgfpathlineto{\pgfqpoint{2.901730in}{1.071613in}}%
\pgfpathlineto{\pgfqpoint{2.980782in}{1.054049in}}%
\pgfpathlineto{\pgfqpoint{3.053319in}{1.035965in}}%
\pgfpathlineto{\pgfqpoint{3.119086in}{1.017526in}}%
\pgfpathlineto{\pgfqpoint{3.177956in}{0.998881in}}%
\pgfpathlineto{\pgfqpoint{3.229932in}{0.980169in}}%
\pgfpathlineto{\pgfqpoint{3.275141in}{0.961515in}}%
\pgfpathlineto{\pgfqpoint{3.313842in}{0.943028in}}%
\pgfpathlineto{\pgfqpoint{3.346419in}{0.924807in}}%
\pgfpathlineto{\pgfqpoint{3.373386in}{0.906938in}}%
\pgfpathlineto{\pgfqpoint{3.395383in}{0.889490in}}%
\pgfpathlineto{\pgfqpoint{3.413005in}{0.872528in}}%
\pgfpathlineto{\pgfqpoint{3.426358in}{0.856123in}}%
\pgfpathlineto{\pgfqpoint{3.436191in}{0.840305in}}%
\pgfpathlineto{\pgfqpoint{3.443171in}{0.825098in}}%
\pgfpathlineto{\pgfqpoint{3.447787in}{0.810520in}}%
\pgfpathlineto{\pgfqpoint{3.450344in}{0.796589in}}%
\pgfpathlineto{\pgfqpoint{3.450967in}{0.783316in}}%
\pgfpathlineto{\pgfqpoint{3.449599in}{0.770710in}}%
\pgfpathlineto{\pgfqpoint{3.446002in}{0.758776in}}%
\pgfpathlineto{\pgfqpoint{3.439757in}{0.747515in}}%
\pgfpathlineto{\pgfqpoint{3.430439in}{0.736928in}}%
\pgfpathlineto{\pgfqpoint{3.418916in}{0.727038in}}%
\pgfpathlineto{\pgfqpoint{3.405479in}{0.717853in}}%
\pgfpathlineto{\pgfqpoint{3.390167in}{0.709373in}}%
\pgfpathlineto{\pgfqpoint{3.363702in}{0.697979in}}%
\pgfpathlineto{\pgfqpoint{3.332970in}{0.688177in}}%
\pgfpathlineto{\pgfqpoint{3.297766in}{0.679969in}}%
\pgfpathlineto{\pgfqpoint{3.257755in}{0.673355in}}%
\pgfpathlineto{\pgfqpoint{3.212593in}{0.668339in}}%
\pgfpathlineto{\pgfqpoint{3.162091in}{0.664951in}}%
\pgfpathlineto{\pgfqpoint{3.105619in}{0.663232in}}%
\pgfpathlineto{\pgfqpoint{3.042524in}{0.663230in}}%
\pgfpathlineto{\pgfqpoint{2.972168in}{0.665006in}}%
\pgfpathlineto{\pgfqpoint{2.865978in}{0.670265in}}%
\pgfpathlineto{\pgfqpoint{2.744304in}{0.679012in}}%
\pgfpathlineto{\pgfqpoint{2.605724in}{0.691481in}}%
\pgfpathlineto{\pgfqpoint{2.449459in}{0.707909in}}%
\pgfpathlineto{\pgfqpoint{2.277440in}{0.728559in}}%
\pgfpathlineto{\pgfqpoint{2.141048in}{0.746869in}}%
\pgfpathlineto{\pgfqpoint{2.001954in}{0.767525in}}%
\pgfpathlineto{\pgfqpoint{1.864517in}{0.790361in}}%
\pgfpathlineto{\pgfqpoint{1.776420in}{0.806677in}}%
\pgfpathlineto{\pgfqpoint{1.692939in}{0.823763in}}%
\pgfpathlineto{\pgfqpoint{1.615083in}{0.841469in}}%
\pgfpathlineto{\pgfqpoint{1.543592in}{0.859638in}}%
\pgfpathlineto{\pgfqpoint{1.478970in}{0.878122in}}%
\pgfpathlineto{\pgfqpoint{1.421485in}{0.896785in}}%
\pgfpathlineto{\pgfqpoint{1.371165in}{0.915496in}}%
\pgfpathlineto{\pgfqpoint{1.327801in}{0.934137in}}%
\pgfpathlineto{\pgfqpoint{1.290947in}{0.952598in}}%
\pgfpathlineto{\pgfqpoint{1.259919in}{0.970776in}}%
\pgfpathlineto{\pgfqpoint{1.234406in}{0.988565in}}%
\pgfpathlineto{\pgfqpoint{1.213849in}{1.005893in}}%
\pgfpathlineto{\pgfqpoint{1.197294in}{1.022718in}}%
\pgfpathlineto{\pgfqpoint{1.184023in}{1.039003in}}%
\pgfpathlineto{\pgfqpoint{1.173545in}{1.054715in}}%
\pgfpathlineto{\pgfqpoint{1.165604in}{1.069827in}}%
\pgfpathlineto{\pgfqpoint{1.160175in}{1.084313in}}%
\pgfpathlineto{\pgfqpoint{1.157465in}{1.098156in}}%
\pgfpathlineto{\pgfqpoint{1.157914in}{1.111338in}}%
\pgfpathlineto{\pgfqpoint{1.161712in}{1.123846in}}%
\pgfpathlineto{\pgfqpoint{1.167759in}{1.135656in}}%
\pgfpathlineto{\pgfqpoint{1.175832in}{1.146766in}}%
\pgfpathlineto{\pgfqpoint{1.185830in}{1.157172in}}%
\pgfpathlineto{\pgfqpoint{1.197687in}{1.166872in}}%
\pgfpathlineto{\pgfqpoint{1.211378in}{1.175866in}}%
\pgfpathlineto{\pgfqpoint{1.226913in}{1.184156in}}%
\pgfpathlineto{\pgfqpoint{1.253795in}{1.195271in}}%
\pgfpathlineto{\pgfqpoint{1.285258in}{1.204815in}}%
\pgfpathlineto{\pgfqpoint{1.321367in}{1.212780in}}%
\pgfpathlineto{\pgfqpoint{1.362264in}{1.219151in}}%
\pgfpathlineto{\pgfqpoint{1.408282in}{1.223904in}}%
\pgfpathlineto{\pgfqpoint{1.459834in}{1.227011in}}%
\pgfpathlineto{\pgfqpoint{1.517409in}{1.228434in}}%
\pgfpathlineto{\pgfqpoint{1.581575in}{1.228127in}}%
\pgfpathlineto{\pgfqpoint{1.678511in}{1.224936in}}%
\pgfpathlineto{\pgfqpoint{1.789973in}{1.218419in}}%
\pgfpathlineto{\pgfqpoint{1.917728in}{1.208332in}}%
\pgfpathlineto{\pgfqpoint{2.062992in}{1.194414in}}%
\pgfpathlineto{\pgfqpoint{2.225218in}{1.176437in}}%
\pgfpathlineto{\pgfqpoint{2.402041in}{1.154216in}}%
\pgfpathlineto{\pgfqpoint{2.540506in}{1.134748in}}%
\pgfpathlineto{\pgfqpoint{2.679093in}{1.112999in}}%
\pgfpathlineto{\pgfqpoint{2.813334in}{1.089188in}}%
\pgfpathlineto{\pgfqpoint{2.898231in}{1.072321in}}%
\pgfpathlineto{\pgfqpoint{2.977939in}{1.054796in}}%
\pgfpathlineto{\pgfqpoint{3.051179in}{1.036752in}}%
\pgfpathlineto{\pgfqpoint{3.117226in}{1.018337in}}%
\pgfpathlineto{\pgfqpoint{3.176156in}{0.999699in}}%
\pgfpathlineto{\pgfqpoint{3.228160in}{0.980978in}}%
\pgfpathlineto{\pgfqpoint{3.273490in}{0.962299in}}%
\pgfpathlineto{\pgfqpoint{3.312455in}{0.943776in}}%
\pgfpathlineto{\pgfqpoint{3.345422in}{0.925511in}}%
\pgfpathlineto{\pgfqpoint{3.372820in}{0.907596in}}%
\pgfpathlineto{\pgfqpoint{3.395135in}{0.890109in}}%
\pgfpathlineto{\pgfqpoint{3.412852in}{0.873121in}}%
\pgfpathlineto{\pgfqpoint{3.426585in}{0.856681in}}%
\pgfpathlineto{\pgfqpoint{3.436935in}{0.840822in}}%
\pgfpathlineto{\pgfqpoint{3.444359in}{0.825574in}}%
\pgfpathlineto{\pgfqpoint{3.449169in}{0.810959in}}%
\pgfpathlineto{\pgfqpoint{3.451537in}{0.796996in}}%
\pgfpathlineto{\pgfqpoint{3.451489in}{0.783699in}}%
\pgfpathlineto{\pgfqpoint{3.448908in}{0.771077in}}%
\pgfpathlineto{\pgfqpoint{3.443686in}{0.759136in}}%
\pgfpathlineto{\pgfqpoint{3.436283in}{0.747888in}}%
\pgfpathlineto{\pgfqpoint{3.426876in}{0.737337in}}%
\pgfpathlineto{\pgfqpoint{3.415559in}{0.727486in}}%
\pgfpathlineto{\pgfqpoint{3.402388in}{0.718337in}}%
\pgfpathlineto{\pgfqpoint{3.387385in}{0.709891in}}%
\pgfpathlineto{\pgfqpoint{3.361408in}{0.698539in}}%
\pgfpathlineto{\pgfqpoint{3.331074in}{0.688765in}}%
\pgfpathlineto{\pgfqpoint{3.296011in}{0.680559in}}%
\pgfpathlineto{\pgfqpoint{3.256091in}{0.673932in}}%
\pgfpathlineto{\pgfqpoint{3.211130in}{0.668907in}}%
\pgfpathlineto{\pgfqpoint{3.160693in}{0.665515in}}%
\pgfpathlineto{\pgfqpoint{3.104289in}{0.663792in}}%
\pgfpathlineto{\pgfqpoint{3.041367in}{0.663787in}}%
\pgfpathlineto{\pgfqpoint{2.971317in}{0.665556in}}%
\pgfpathlineto{\pgfqpoint{2.865666in}{0.670792in}}%
\pgfpathlineto{\pgfqpoint{2.744443in}{0.679489in}}%
\pgfpathlineto{\pgfqpoint{2.606172in}{0.691903in}}%
\pgfpathlineto{\pgfqpoint{2.450445in}{0.708290in}}%
\pgfpathlineto{\pgfqpoint{2.278933in}{0.728857in}}%
\pgfpathlineto{\pgfqpoint{2.142670in}{0.747087in}}%
\pgfpathlineto{\pgfqpoint{2.003779in}{0.767669in}}%
\pgfpathlineto{\pgfqpoint{1.866602in}{0.790443in}}%
\pgfpathlineto{\pgfqpoint{1.778540in}{0.806715in}}%
\pgfpathlineto{\pgfqpoint{1.695067in}{0.823758in}}%
\pgfpathlineto{\pgfqpoint{1.617267in}{0.841436in}}%
\pgfpathlineto{\pgfqpoint{1.545793in}{0.859584in}}%
\pgfpathlineto{\pgfqpoint{1.481098in}{0.878049in}}%
\pgfpathlineto{\pgfqpoint{1.423433in}{0.896688in}}%
\pgfpathlineto{\pgfqpoint{1.372847in}{0.915371in}}%
\pgfpathlineto{\pgfqpoint{1.329189in}{0.933980in}}%
\pgfpathlineto{\pgfqpoint{1.292105in}{0.952407in}}%
\pgfpathlineto{\pgfqpoint{1.261040in}{0.970555in}}%
\pgfpathlineto{\pgfqpoint{1.235238in}{0.988342in}}%
\pgfpathlineto{\pgfqpoint{1.214206in}{1.005682in}}%
\pgfpathlineto{\pgfqpoint{1.197714in}{1.022506in}}%
\pgfpathlineto{\pgfqpoint{1.184913in}{1.038778in}}%
\pgfpathlineto{\pgfqpoint{1.175137in}{1.054470in}}%
\pgfpathlineto{\pgfqpoint{1.167923in}{1.069556in}}%
\pgfpathlineto{\pgfqpoint{1.163006in}{1.084014in}}%
\pgfpathlineto{\pgfqpoint{1.160321in}{1.097827in}}%
\pgfpathlineto{\pgfqpoint{1.160002in}{1.110982in}}%
\pgfpathlineto{\pgfqpoint{1.162384in}{1.123471in}}%
\pgfpathlineto{\pgfqpoint{1.167924in}{1.135287in}}%
\pgfpathlineto{\pgfqpoint{1.175896in}{1.146410in}}%
\pgfpathlineto{\pgfqpoint{1.185857in}{1.156832in}}%
\pgfpathlineto{\pgfqpoint{1.197726in}{1.166551in}}%
\pgfpathlineto{\pgfqpoint{1.211451in}{1.175565in}}%
\pgfpathlineto{\pgfqpoint{1.227016in}{1.183873in}}%
\pgfpathlineto{\pgfqpoint{1.253850in}{1.195013in}}%
\pgfpathlineto{\pgfqpoint{1.285043in}{1.204569in}}%
\pgfpathlineto{\pgfqpoint{1.320936in}{1.212547in}}%
\pgfpathlineto{\pgfqpoint{1.361698in}{1.218941in}}%
\pgfpathlineto{\pgfqpoint{1.407543in}{1.223726in}}%
\pgfpathlineto{\pgfqpoint{1.458933in}{1.226871in}}%
\pgfpathlineto{\pgfqpoint{1.516382in}{1.228339in}}%
\pgfpathlineto{\pgfqpoint{1.580454in}{1.228079in}}%
\pgfpathlineto{\pgfqpoint{1.677261in}{1.224940in}}%
\pgfpathlineto{\pgfqpoint{1.788549in}{1.218456in}}%
\pgfpathlineto{\pgfqpoint{1.916009in}{1.208420in}}%
\pgfpathlineto{\pgfqpoint{2.060873in}{1.194567in}}%
\pgfpathlineto{\pgfqpoint{2.222890in}{1.176651in}}%
\pgfpathlineto{\pgfqpoint{2.399403in}{1.154497in}}%
\pgfpathlineto{\pgfqpoint{2.537625in}{1.135086in}}%
\pgfpathlineto{\pgfqpoint{2.676388in}{1.113385in}}%
\pgfpathlineto{\pgfqpoint{2.811052in}{1.089608in}}%
\pgfpathlineto{\pgfqpoint{2.896003in}{1.072753in}}%
\pgfpathlineto{\pgfqpoint{2.975426in}{1.055216in}}%
\pgfpathlineto{\pgfqpoint{3.048617in}{1.037161in}}%
\pgfpathlineto{\pgfqpoint{3.115110in}{1.018748in}}%
\pgfpathlineto{\pgfqpoint{3.174629in}{1.000123in}}%
\pgfpathlineto{\pgfqpoint{3.227085in}{0.981422in}}%
\pgfpathlineto{\pgfqpoint{3.272578in}{0.962767in}}%
\pgfpathlineto{\pgfqpoint{3.311400in}{0.944270in}}%
\pgfpathlineto{\pgfqpoint{3.344028in}{0.926028in}}%
\pgfpathlineto{\pgfqpoint{3.371131in}{0.908129in}}%
\pgfpathlineto{\pgfqpoint{3.393465in}{0.890650in}}%
\pgfpathlineto{\pgfqpoint{3.411106in}{0.873674in}}%
\pgfpathlineto{\pgfqpoint{3.424764in}{0.857244in}}%
\pgfpathlineto{\pgfqpoint{3.435185in}{0.841388in}}%
\pgfpathlineto{\pgfqpoint{3.442917in}{0.826135in}}%
\pgfpathlineto{\pgfqpoint{3.448314in}{0.811504in}}%
\pgfpathlineto{\pgfqpoint{3.451529in}{0.797516in}}%
\pgfpathlineto{\pgfqpoint{3.452523in}{0.784182in}}%
\pgfpathlineto{\pgfqpoint{3.451055in}{0.771514in}}%
\pgfpathlineto{\pgfqpoint{3.446692in}{0.759515in}}%
\pgfpathlineto{\pgfqpoint{3.439323in}{0.748197in}}%
\pgfpathlineto{\pgfqpoint{3.429882in}{0.737579in}}%
\pgfpathlineto{\pgfqpoint{3.418505in}{0.727664in}}%
\pgfpathlineto{\pgfqpoint{3.405253in}{0.718454in}}%
\pgfpathlineto{\pgfqpoint{3.390156in}{0.709949in}}%
\pgfpathlineto{\pgfqpoint{3.364042in}{0.698516in}}%
\pgfpathlineto{\pgfqpoint{3.333634in}{0.688670in}}%
\pgfpathlineto{\pgfqpoint{3.298642in}{0.680406in}}%
\pgfpathlineto{\pgfqpoint{3.258751in}{0.673723in}}%
\pgfpathlineto{\pgfqpoint{3.213859in}{0.668644in}}%
\pgfpathlineto{\pgfqpoint{3.163516in}{0.665197in}}%
\pgfpathlineto{\pgfqpoint{3.107205in}{0.663419in}}%
\pgfpathlineto{\pgfqpoint{3.044364in}{0.663359in}}%
\pgfpathlineto{\pgfqpoint{2.974385in}{0.665075in}}%
\pgfpathlineto{\pgfqpoint{2.868846in}{0.670245in}}%
\pgfpathlineto{\pgfqpoint{2.747793in}{0.678886in}}%
\pgfpathlineto{\pgfqpoint{2.609727in}{0.691233in}}%
\pgfpathlineto{\pgfqpoint{2.454118in}{0.707559in}}%
\pgfpathlineto{\pgfqpoint{2.282688in}{0.728067in}}%
\pgfpathlineto{\pgfqpoint{2.146348in}{0.746255in}}%
\pgfpathlineto{\pgfqpoint{2.007255in}{0.766801in}}%
\pgfpathlineto{\pgfqpoint{1.869906in}{0.789547in}}%
\pgfpathlineto{\pgfqpoint{1.739036in}{0.814224in}}%
\pgfpathlineto{\pgfqpoint{1.657887in}{0.831574in}}%
\pgfpathlineto{\pgfqpoint{1.583103in}{0.849500in}}%
\pgfpathlineto{\pgfqpoint{1.515024in}{0.867836in}}%
\pgfpathlineto{\pgfqpoint{1.453811in}{0.886427in}}%
\pgfpathlineto{\pgfqpoint{1.399503in}{0.905131in}}%
\pgfpathlineto{\pgfqpoint{1.352010in}{0.923819in}}%
\pgfpathlineto{\pgfqpoint{1.311119in}{0.942374in}}%
\pgfpathlineto{\pgfqpoint{1.276492in}{0.960696in}}%
\pgfpathlineto{\pgfqpoint{1.247664in}{0.978694in}}%
\pgfpathlineto{\pgfqpoint{1.224045in}{0.996293in}}%
\pgfpathlineto{\pgfqpoint{1.204941in}{1.013429in}}%
\pgfpathlineto{\pgfqpoint{1.190197in}{1.030029in}}%
\pgfpathlineto{\pgfqpoint{1.179259in}{1.046050in}}%
\pgfpathlineto{\pgfqpoint{1.171390in}{1.061468in}}%
\pgfpathlineto{\pgfqpoint{1.166039in}{1.076262in}}%
\pgfpathlineto{\pgfqpoint{1.162831in}{1.090414in}}%
\pgfpathlineto{\pgfqpoint{1.161573in}{1.103912in}}%
\pgfpathlineto{\pgfqpoint{1.162255in}{1.116745in}}%
\pgfpathlineto{\pgfqpoint{1.165044in}{1.128907in}}%
\pgfpathlineto{\pgfqpoint{1.170291in}{1.140396in}}%
\pgfpathlineto{\pgfqpoint{1.178523in}{1.151214in}}%
\pgfpathlineto{\pgfqpoint{1.189381in}{1.161344in}}%
\pgfpathlineto{\pgfqpoint{1.202174in}{1.170770in}}%
\pgfpathlineto{\pgfqpoint{1.216853in}{1.179491in}}%
\pgfpathlineto{\pgfqpoint{1.242370in}{1.191247in}}%
\pgfpathlineto{\pgfqpoint{1.272129in}{1.201412in}}%
\pgfpathlineto{\pgfqpoint{1.306298in}{1.209983in}}%
\pgfpathlineto{\pgfqpoint{1.345182in}{1.216959in}}%
\pgfpathlineto{\pgfqpoint{1.389171in}{1.222339in}}%
\pgfpathlineto{\pgfqpoint{1.438411in}{1.226100in}}%
\pgfpathlineto{\pgfqpoint{1.493466in}{1.228202in}}%
\pgfpathlineto{\pgfqpoint{1.554996in}{1.228599in}}%
\pgfpathlineto{\pgfqpoint{1.623649in}{1.227233in}}%
\pgfpathlineto{\pgfqpoint{1.727357in}{1.222547in}}%
\pgfpathlineto{\pgfqpoint{1.846306in}{1.214410in}}%
\pgfpathlineto{\pgfqpoint{1.981908in}{1.202592in}}%
\pgfpathlineto{\pgfqpoint{2.135239in}{1.186857in}}%
\pgfpathlineto{\pgfqpoint{2.304888in}{1.166942in}}%
\pgfpathlineto{\pgfqpoint{2.440147in}{1.149177in}}%
\pgfpathlineto{\pgfqpoint{2.578925in}{1.129038in}}%
\pgfpathlineto{\pgfqpoint{2.717151in}{1.106672in}}%
\pgfpathlineto{\pgfqpoint{2.849779in}{1.082320in}}%
\pgfpathlineto{\pgfqpoint{2.932420in}{1.065129in}}%
\pgfpathlineto{\pgfqpoint{3.009243in}{1.047335in}}%
\pgfpathlineto{\pgfqpoint{3.079597in}{1.029099in}}%
\pgfpathlineto{\pgfqpoint{3.143053in}{1.010571in}}%
\pgfpathlineto{\pgfqpoint{3.199395in}{0.991889in}}%
\pgfpathlineto{\pgfqpoint{3.248627in}{0.973182in}}%
\pgfpathlineto{\pgfqpoint{3.290969in}{0.954568in}}%
\pgfpathlineto{\pgfqpoint{3.326856in}{0.936154in}}%
\pgfpathlineto{\pgfqpoint{3.356944in}{0.918035in}}%
\pgfpathlineto{\pgfqpoint{3.381849in}{0.900305in}}%
\pgfpathlineto{\pgfqpoint{3.401742in}{0.883051in}}%
\pgfpathlineto{\pgfqpoint{3.417531in}{0.866313in}}%
\pgfpathlineto{\pgfqpoint{3.429986in}{0.850126in}}%
\pgfpathlineto{\pgfqpoint{3.439654in}{0.834520in}}%
\pgfpathlineto{\pgfqpoint{3.446867in}{0.819522in}}%
\pgfpathlineto{\pgfqpoint{3.451736in}{0.805152in}}%
\pgfpathlineto{\pgfqpoint{3.454151in}{0.791429in}}%
\pgfpathlineto{\pgfqpoint{3.453787in}{0.778365in}}%
\pgfpathlineto{\pgfqpoint{3.450121in}{0.765971in}}%
\pgfpathlineto{\pgfqpoint{3.443777in}{0.754266in}}%
\pgfpathlineto{\pgfqpoint{3.435390in}{0.743262in}}%
\pgfpathlineto{\pgfqpoint{3.425065in}{0.732961in}}%
\pgfpathlineto{\pgfqpoint{3.412871in}{0.723367in}}%
\pgfpathlineto{\pgfqpoint{3.398840in}{0.714478in}}%
\pgfpathlineto{\pgfqpoint{3.382972in}{0.706295in}}%
\pgfpathlineto{\pgfqpoint{3.355637in}{0.695341in}}%
\pgfpathlineto{\pgfqpoint{3.323814in}{0.685967in}}%
\pgfpathlineto{\pgfqpoint{3.287234in}{0.678167in}}%
\pgfpathlineto{\pgfqpoint{3.245848in}{0.671961in}}%
\pgfpathlineto{\pgfqpoint{3.199301in}{0.667371in}}%
\pgfpathlineto{\pgfqpoint{3.147158in}{0.664428in}}%
\pgfpathlineto{\pgfqpoint{3.088915in}{0.663173in}}%
\pgfpathlineto{\pgfqpoint{3.024005in}{0.663653in}}%
\pgfpathlineto{\pgfqpoint{2.925976in}{0.667094in}}%
\pgfpathlineto{\pgfqpoint{2.813269in}{0.673887in}}%
\pgfpathlineto{\pgfqpoint{2.684232in}{0.684258in}}%
\pgfpathlineto{\pgfqpoint{2.537688in}{0.698476in}}%
\pgfpathlineto{\pgfqpoint{2.374194in}{0.716774in}}%
\pgfpathlineto{\pgfqpoint{2.196502in}{0.739323in}}%
\pgfpathlineto{\pgfqpoint{2.057978in}{0.759025in}}%
\pgfpathlineto{\pgfqpoint{1.919470in}{0.780997in}}%
\pgfpathlineto{\pgfqpoint{1.785612in}{0.805011in}}%
\pgfpathlineto{\pgfqpoint{1.701495in}{0.821996in}}%
\pgfpathlineto{\pgfqpoint{1.623189in}{0.839638in}}%
\pgfpathlineto{\pgfqpoint{1.551302in}{0.857775in}}%
\pgfpathlineto{\pgfqpoint{1.486177in}{0.876246in}}%
\pgfpathlineto{\pgfqpoint{1.428001in}{0.894904in}}%
\pgfpathlineto{\pgfqpoint{1.376801in}{0.913613in}}%
\pgfpathlineto{\pgfqpoint{1.332443in}{0.932251in}}%
\pgfpathlineto{\pgfqpoint{1.294638in}{0.950710in}}%
\pgfpathlineto{\pgfqpoint{1.262936in}{0.968893in}}%
\pgfpathlineto{\pgfqpoint{1.236728in}{0.986718in}}%
\pgfpathlineto{\pgfqpoint{1.215248in}{1.004114in}}%
\pgfpathlineto{\pgfqpoint{1.198231in}{1.021006in}}%
\pgfpathlineto{\pgfqpoint{1.185285in}{1.037339in}}%
\pgfpathlineto{\pgfqpoint{1.175637in}{1.053087in}}%
\pgfpathlineto{\pgfqpoint{1.168696in}{1.068225in}}%
\pgfpathlineto{\pgfqpoint{1.164061in}{1.082733in}}%
\pgfpathlineto{\pgfqpoint{1.161513in}{1.096595in}}%
\pgfpathlineto{\pgfqpoint{1.161019in}{1.109798in}}%
\pgfpathlineto{\pgfqpoint{1.162732in}{1.122334in}}%
\pgfpathlineto{\pgfqpoint{1.166990in}{1.134198in}}%
\pgfpathlineto{\pgfqpoint{1.174289in}{1.145390in}}%
\pgfpathlineto{\pgfqpoint{1.184067in}{1.155890in}}%
\pgfpathlineto{\pgfqpoint{1.195796in}{1.165687in}}%
\pgfpathlineto{\pgfqpoint{1.209416in}{1.174780in}}%
\pgfpathlineto{\pgfqpoint{1.224893in}{1.183167in}}%
\pgfpathlineto{\pgfqpoint{1.251594in}{1.194423in}}%
\pgfpathlineto{\pgfqpoint{1.282581in}{1.204090in}}%
\pgfpathlineto{\pgfqpoint{1.318103in}{1.212167in}}%
\pgfpathlineto{\pgfqpoint{1.358543in}{1.218658in}}%
\pgfpathlineto{\pgfqpoint{1.404049in}{1.223548in}}%
\pgfpathlineto{\pgfqpoint{1.455014in}{1.226804in}}%
\pgfpathlineto{\pgfqpoint{1.512021in}{1.228387in}}%
\pgfpathlineto{\pgfqpoint{1.575669in}{1.228249in}}%
\pgfpathlineto{\pgfqpoint{1.646574in}{1.226327in}}%
\pgfpathlineto{\pgfqpoint{1.753506in}{1.220866in}}%
\pgfpathlineto{\pgfqpoint{1.876027in}{1.211908in}}%
\pgfpathlineto{\pgfqpoint{2.015628in}{1.199221in}}%
\pgfpathlineto{\pgfqpoint{2.172731in}{1.182537in}}%
\pgfpathlineto{\pgfqpoint{2.345483in}{1.161641in}}%
\pgfpathlineto{\pgfqpoint{2.482280in}{1.143173in}}%
\pgfpathlineto{\pgfqpoint{2.621507in}{1.122356in}}%
\pgfpathlineto{\pgfqpoint{2.758533in}{1.099346in}}%
\pgfpathlineto{\pgfqpoint{2.846059in}{1.082942in}}%
\pgfpathlineto{\pgfqpoint{2.928987in}{1.065816in}}%
\pgfpathlineto{\pgfqpoint{3.006265in}{1.048088in}}%
\pgfpathlineto{\pgfqpoint{3.077066in}{1.029892in}}%
\pgfpathlineto{\pgfqpoint{3.140799in}{1.011370in}}%
\pgfpathlineto{\pgfqpoint{3.197030in}{0.992680in}}%
\pgfpathlineto{\pgfqpoint{3.245950in}{0.973968in}}%
\pgfpathlineto{\pgfqpoint{3.288324in}{0.955345in}}%
\pgfpathlineto{\pgfqpoint{3.324802in}{0.936916in}}%
\pgfpathlineto{\pgfqpoint{3.355914in}{0.918773in}}%
\pgfpathlineto{\pgfqpoint{3.382061in}{0.900999in}}%
\pgfpathlineto{\pgfqpoint{3.403519in}{0.883667in}}%
\pgfpathlineto{\pgfqpoint{3.420442in}{0.866842in}}%
\pgfpathlineto{\pgfqpoint{3.432964in}{0.850578in}}%
\pgfpathlineto{\pgfqpoint{3.441910in}{0.834918in}}%
\pgfpathlineto{\pgfqpoint{3.447829in}{0.819887in}}%
\pgfpathlineto{\pgfqpoint{3.451083in}{0.805504in}}%
\pgfpathlineto{\pgfqpoint{3.451939in}{0.791785in}}%
\pgfpathlineto{\pgfqpoint{3.450568in}{0.778743in}}%
\pgfpathlineto{\pgfqpoint{3.447041in}{0.766385in}}%
\pgfpathlineto{\pgfqpoint{3.441336in}{0.754716in}}%
\pgfpathlineto{\pgfqpoint{3.433342in}{0.743736in}}%
\pgfpathlineto{\pgfqpoint{3.423197in}{0.733450in}}%
\pgfpathlineto{\pgfqpoint{3.411089in}{0.723862in}}%
\pgfpathlineto{\pgfqpoint{3.397084in}{0.714974in}}%
\pgfpathlineto{\pgfqpoint{3.381215in}{0.706789in}}%
\pgfpathlineto{\pgfqpoint{3.353923in}{0.695832in}}%
\pgfpathlineto{\pgfqpoint{3.322342in}{0.686461in}}%
\pgfpathlineto{\pgfqpoint{3.286224in}{0.678677in}}%
\pgfpathlineto{\pgfqpoint{3.245169in}{0.672480in}}%
\pgfpathlineto{\pgfqpoint{3.198852in}{0.667876in}}%
\pgfpathlineto{\pgfqpoint{3.147070in}{0.664905in}}%
\pgfpathlineto{\pgfqpoint{3.089168in}{0.663608in}}%
\pgfpathlineto{\pgfqpoint{3.024488in}{0.664038in}}%
\pgfpathlineto{\pgfqpoint{2.926603in}{0.667410in}}%
\pgfpathlineto{\pgfqpoint{2.814064in}{0.674147in}}%
\pgfpathlineto{\pgfqpoint{2.685447in}{0.684462in}}%
\pgfpathlineto{\pgfqpoint{2.539424in}{0.698597in}}%
\pgfpathlineto{\pgfqpoint{2.376138in}{0.716796in}}%
\pgfpathlineto{\pgfqpoint{2.199003in}{0.739277in}}%
\pgfpathlineto{\pgfqpoint{2.060852in}{0.758930in}}%
\pgfpathlineto{\pgfqpoint{1.922362in}{0.780842in}}%
\pgfpathlineto{\pgfqpoint{1.788365in}{0.804792in}}%
\pgfpathlineto{\pgfqpoint{1.704252in}{0.821749in}}%
\pgfpathlineto{\pgfqpoint{1.625712in}{0.839357in}}%
\pgfpathlineto{\pgfqpoint{1.553452in}{0.857455in}}%
\pgfpathlineto{\pgfqpoint{1.487967in}{0.875889in}}%
\pgfpathlineto{\pgfqpoint{1.429533in}{0.894516in}}%
\pgfpathlineto{\pgfqpoint{1.378214in}{0.913206in}}%
\pgfpathlineto{\pgfqpoint{1.333859in}{0.931838in}}%
\pgfpathlineto{\pgfqpoint{1.296103in}{0.950302in}}%
\pgfpathlineto{\pgfqpoint{1.264365in}{0.968498in}}%
\pgfpathlineto{\pgfqpoint{1.237905in}{0.986336in}}%
\pgfpathlineto{\pgfqpoint{1.216579in}{1.003721in}}%
\pgfpathlineto{\pgfqpoint{1.199652in}{1.020601in}}%
\pgfpathlineto{\pgfqpoint{1.186285in}{1.036940in}}%
\pgfpathlineto{\pgfqpoint{1.175858in}{1.052706in}}%
\pgfpathlineto{\pgfqpoint{1.167966in}{1.067873in}}%
\pgfpathlineto{\pgfqpoint{1.162424in}{1.082418in}}%
\pgfpathlineto{\pgfqpoint{1.159262in}{1.096322in}}%
\pgfpathlineto{\pgfqpoint{1.158728in}{1.109570in}}%
\pgfpathlineto{\pgfqpoint{1.161286in}{1.122152in}}%
\pgfpathlineto{\pgfqpoint{1.166961in}{1.134055in}}%
\pgfpathlineto{\pgfqpoint{1.174748in}{1.145258in}}%
\pgfpathlineto{\pgfqpoint{1.184504in}{1.155759in}}%
\pgfpathlineto{\pgfqpoint{1.196151in}{1.165556in}}%
\pgfpathlineto{\pgfqpoint{1.209642in}{1.174647in}}%
\pgfpathlineto{\pgfqpoint{1.224968in}{1.183033in}}%
\pgfpathlineto{\pgfqpoint{1.251457in}{1.194289in}}%
\pgfpathlineto{\pgfqpoint{1.282360in}{1.203966in}}%
\pgfpathlineto{\pgfqpoint{1.318001in}{1.212069in}}%
\pgfpathlineto{\pgfqpoint{1.358418in}{1.218583in}}%
\pgfpathlineto{\pgfqpoint{1.403919in}{1.223487in}}%
\pgfpathlineto{\pgfqpoint{1.454928in}{1.226751in}}%
\pgfpathlineto{\pgfqpoint{1.511935in}{1.228337in}}%
\pgfpathlineto{\pgfqpoint{1.575497in}{1.228198in}}%
\pgfpathlineto{\pgfqpoint{1.646236in}{1.226278in}}%
\pgfpathlineto{\pgfqpoint{1.752916in}{1.220831in}}%
\pgfpathlineto{\pgfqpoint{1.875291in}{1.211908in}}%
\pgfpathlineto{\pgfqpoint{2.014843in}{1.199243in}}%
\pgfpathlineto{\pgfqpoint{2.171806in}{1.182585in}}%
\pgfpathlineto{\pgfqpoint{2.344405in}{1.161732in}}%
\pgfpathlineto{\pgfqpoint{2.481218in}{1.143283in}}%
\pgfpathlineto{\pgfqpoint{2.620268in}{1.122489in}}%
\pgfpathlineto{\pgfqpoint{2.757188in}{1.099520in}}%
\pgfpathlineto{\pgfqpoint{2.844876in}{1.083132in}}%
\pgfpathlineto{\pgfqpoint{2.927825in}{1.065997in}}%
\pgfpathlineto{\pgfqpoint{3.004840in}{1.048237in}}%
\pgfpathlineto{\pgfqpoint{3.075387in}{1.030020in}}%
\pgfpathlineto{\pgfqpoint{3.139117in}{1.011502in}}%
\pgfpathlineto{\pgfqpoint{3.195855in}{0.992827in}}%
\pgfpathlineto{\pgfqpoint{3.245594in}{0.974127in}}%
\pgfpathlineto{\pgfqpoint{3.288502in}{0.955520in}}%
\pgfpathlineto{\pgfqpoint{3.324918in}{0.937113in}}%
\pgfpathlineto{\pgfqpoint{3.355351in}{0.918999in}}%
\pgfpathlineto{\pgfqpoint{3.380483in}{0.901258in}}%
\pgfpathlineto{\pgfqpoint{3.401045in}{0.883963in}}%
\pgfpathlineto{\pgfqpoint{3.417091in}{0.867191in}}%
\pgfpathlineto{\pgfqpoint{3.429327in}{0.850980in}}%
\pgfpathlineto{\pgfqpoint{3.438464in}{0.835359in}}%
\pgfpathlineto{\pgfqpoint{3.445023in}{0.820350in}}%
\pgfpathlineto{\pgfqpoint{3.449337in}{0.805974in}}%
\pgfpathlineto{\pgfqpoint{3.451550in}{0.792247in}}%
\pgfpathlineto{\pgfqpoint{3.451617in}{0.779180in}}%
\pgfpathlineto{\pgfqpoint{3.449305in}{0.766781in}}%
\pgfpathlineto{\pgfqpoint{3.444190in}{0.755054in}}%
\pgfpathlineto{\pgfqpoint{3.436075in}{0.744006in}}%
\pgfpathlineto{\pgfqpoint{3.425878in}{0.733657in}}%
\pgfpathlineto{\pgfqpoint{3.413756in}{0.724012in}}%
\pgfpathlineto{\pgfqpoint{3.399761in}{0.715072in}}%
\pgfpathlineto{\pgfqpoint{3.383916in}{0.706837in}}%
\pgfpathlineto{\pgfqpoint{3.356660in}{0.695810in}}%
\pgfpathlineto{\pgfqpoint{3.325080in}{0.686371in}}%
\pgfpathlineto{\pgfqpoint{3.288885in}{0.678517in}}%
\pgfpathlineto{\pgfqpoint{3.247727in}{0.672248in}}%
\pgfpathlineto{\pgfqpoint{3.201490in}{0.667588in}}%
\pgfpathlineto{\pgfqpoint{3.149691in}{0.664566in}}%
\pgfpathlineto{\pgfqpoint{3.091770in}{0.663224in}}%
\pgfpathlineto{\pgfqpoint{3.027141in}{0.663612in}}%
\pgfpathlineto{\pgfqpoint{2.929460in}{0.666930in}}%
\pgfpathlineto{\pgfqpoint{2.817208in}{0.673612in}}%
\pgfpathlineto{\pgfqpoint{2.688756in}{0.683867in}}%
\pgfpathlineto{\pgfqpoint{2.542901in}{0.697945in}}%
\pgfpathlineto{\pgfqpoint{2.379859in}{0.716110in}}%
\pgfpathlineto{\pgfqpoint{2.202683in}{0.738514in}}%
\pgfpathlineto{\pgfqpoint{2.064163in}{0.758111in}}%
\pgfpathlineto{\pgfqpoint{1.925381in}{0.779992in}}%
\pgfpathlineto{\pgfqpoint{1.791401in}{0.803920in}}%
\pgfpathlineto{\pgfqpoint{1.706993in}{0.820849in}}%
\pgfpathlineto{\pgfqpoint{1.627904in}{0.838423in}}%
\pgfpathlineto{\pgfqpoint{1.555185in}{0.856506in}}%
\pgfpathlineto{\pgfqpoint{1.489822in}{0.874942in}}%
\pgfpathlineto{\pgfqpoint{1.431769in}{0.893584in}}%
\pgfpathlineto{\pgfqpoint{1.380588in}{0.912298in}}%
\pgfpathlineto{\pgfqpoint{1.335867in}{0.930963in}}%
\pgfpathlineto{\pgfqpoint{1.297221in}{0.949466in}}%
\pgfpathlineto{\pgfqpoint{1.264292in}{0.967710in}}%
\pgfpathlineto{\pgfqpoint{1.236751in}{0.985603in}}%
\pgfpathlineto{\pgfqpoint{1.214293in}{1.003070in}}%
\pgfpathlineto{\pgfqpoint{1.196643in}{1.020044in}}%
\pgfpathlineto{\pgfqpoint{1.183305in}{1.036462in}}%
\pgfpathlineto{\pgfqpoint{1.173539in}{1.052287in}}%
\pgfpathlineto{\pgfqpoint{1.166794in}{1.067492in}}%
\pgfpathlineto{\pgfqpoint{1.162654in}{1.082058in}}%
\pgfpathlineto{\pgfqpoint{1.160830in}{1.095966in}}%
\pgfpathlineto{\pgfqpoint{1.161169in}{1.109206in}}%
\pgfpathlineto{\pgfqpoint{1.163645in}{1.121766in}}%
\pgfpathlineto{\pgfqpoint{1.168368in}{1.133644in}}%
\pgfpathlineto{\pgfqpoint{1.175575in}{1.144839in}}%
\pgfpathlineto{\pgfqpoint{1.185178in}{1.155344in}}%
\pgfpathlineto{\pgfqpoint{1.196756in}{1.165149in}}%
\pgfpathlineto{\pgfqpoint{1.210240in}{1.174253in}}%
\pgfpathlineto{\pgfqpoint{1.225592in}{1.182654in}}%
\pgfpathlineto{\pgfqpoint{1.252106in}{1.193934in}}%
\pgfpathlineto{\pgfqpoint{1.282891in}{1.203626in}}%
\pgfpathlineto{\pgfqpoint{1.318177in}{1.211731in}}%
\pgfpathlineto{\pgfqpoint{1.358337in}{1.218249in}}%
\pgfpathlineto{\pgfqpoint{1.403662in}{1.223174in}}%
\pgfpathlineto{\pgfqpoint{1.454379in}{1.226469in}}%
\pgfpathlineto{\pgfqpoint{1.511109in}{1.228097in}}%
\pgfpathlineto{\pgfqpoint{1.574481in}{1.228008in}}%
\pgfpathlineto{\pgfqpoint{1.645121in}{1.226139in}}%
\pgfpathlineto{\pgfqpoint{1.751702in}{1.220755in}}%
\pgfpathlineto{\pgfqpoint{1.873809in}{1.211880in}}%
\pgfpathlineto{\pgfqpoint{2.012909in}{1.199280in}}%
\pgfpathlineto{\pgfqpoint{2.169531in}{1.182707in}}%
\pgfpathlineto{\pgfqpoint{2.341837in}{1.161913in}}%
\pgfpathlineto{\pgfqpoint{2.478385in}{1.143513in}}%
\pgfpathlineto{\pgfqpoint{2.617447in}{1.122781in}}%
\pgfpathlineto{\pgfqpoint{2.754454in}{1.099861in}}%
\pgfpathlineto{\pgfqpoint{2.842125in}{1.083480in}}%
\pgfpathlineto{\pgfqpoint{2.925257in}{1.066369in}}%
\pgfpathlineto{\pgfqpoint{3.002763in}{1.048665in}}%
\pgfpathlineto{\pgfqpoint{3.073833in}{1.030508in}}%
\pgfpathlineto{\pgfqpoint{3.137938in}{1.012031in}}%
\pgfpathlineto{\pgfqpoint{3.194827in}{0.993369in}}%
\pgfpathlineto{\pgfqpoint{3.244529in}{0.974652in}}%
\pgfpathlineto{\pgfqpoint{3.287350in}{0.956010in}}%
\pgfpathlineto{\pgfqpoint{3.323416in}{0.937589in}}%
\pgfpathlineto{\pgfqpoint{3.353508in}{0.919483in}}%
\pgfpathlineto{\pgfqpoint{3.378719in}{0.901752in}}%
\pgfpathlineto{\pgfqpoint{3.399876in}{0.884453in}}%
\pgfpathlineto{\pgfqpoint{3.417542in}{0.867634in}}%
\pgfpathlineto{\pgfqpoint{3.432019in}{0.851343in}}%
\pgfpathlineto{\pgfqpoint{3.443343in}{0.835621in}}%
\pgfpathlineto{\pgfqpoint{3.451286in}{0.820504in}}%
\pgfpathlineto{\pgfqpoint{3.455357in}{0.806023in}}%
\pgfpathlineto{\pgfqpoint{3.455836in}{0.792210in}}%
\pgfpathlineto{\pgfqpoint{3.453917in}{0.779082in}}%
\pgfpathlineto{\pgfqpoint{3.449807in}{0.766648in}}%
\pgfpathlineto{\pgfqpoint{3.443658in}{0.754912in}}%
\pgfpathlineto{\pgfqpoint{3.435571in}{0.743878in}}%
\pgfpathlineto{\pgfqpoint{3.425594in}{0.733548in}}%
\pgfpathlineto{\pgfqpoint{3.413728in}{0.723921in}}%
\pgfpathlineto{\pgfqpoint{3.399921in}{0.714995in}}%
\pgfpathlineto{\pgfqpoint{3.384078in}{0.706767in}}%
\pgfpathlineto{\pgfqpoint{3.356603in}{0.695734in}}%
\pgfpathlineto{\pgfqpoint{3.324727in}{0.686283in}}%
\pgfpathlineto{\pgfqpoint{3.288318in}{0.678424in}}%
\pgfpathlineto{\pgfqpoint{3.247146in}{0.672172in}}%
\pgfpathlineto{\pgfqpoint{3.200874in}{0.667547in}}%
\pgfpathlineto{\pgfqpoint{3.149063in}{0.664570in}}%
\pgfpathlineto{\pgfqpoint{3.091172in}{0.663266in}}%
\pgfpathlineto{\pgfqpoint{3.026715in}{0.663658in}}%
\pgfpathlineto{\pgfqpoint{2.929961in}{0.666889in}}%
\pgfpathlineto{\pgfqpoint{2.817842in}{0.673513in}}%
\pgfpathlineto{\pgfqpoint{2.688348in}{0.683818in}}%
\pgfpathlineto{\pgfqpoint{2.541222in}{0.698026in}}%
\pgfpathlineto{\pgfqpoint{2.377964in}{0.716294in}}%
\pgfpathlineto{\pgfqpoint{2.201825in}{0.738714in}}%
\pgfpathlineto{\pgfqpoint{2.064215in}{0.758275in}}%
\pgfpathlineto{\pgfqpoint{1.925574in}{0.780171in}}%
\pgfpathlineto{\pgfqpoint{1.791608in}{0.804155in}}%
\pgfpathlineto{\pgfqpoint{1.707232in}{0.821096in}}%
\pgfpathlineto{\pgfqpoint{1.628098in}{0.838650in}}%
\pgfpathlineto{\pgfqpoint{1.555058in}{0.856691in}}%
\pgfpathlineto{\pgfqpoint{1.488737in}{0.875090in}}%
\pgfpathlineto{\pgfqpoint{1.429533in}{0.893719in}}%
\pgfpathlineto{\pgfqpoint{1.377614in}{0.912447in}}%
\pgfpathlineto{\pgfqpoint{1.332920in}{0.931145in}}%
\pgfpathlineto{\pgfqpoint{1.295238in}{0.949674in}}%
\pgfpathlineto{\pgfqpoint{1.264059in}{0.967888in}}%
\pgfpathlineto{\pgfqpoint{1.238152in}{0.985727in}}%
\pgfpathlineto{\pgfqpoint{1.216484in}{1.003142in}}%
\pgfpathlineto{\pgfqpoint{1.198313in}{1.020088in}}%
\pgfpathlineto{\pgfqpoint{1.183178in}{1.036519in}}%
\pgfpathlineto{\pgfqpoint{1.170904in}{1.052399in}}%
\pgfpathlineto{\pgfqpoint{1.161603in}{1.067688in}}%
\pgfpathlineto{\pgfqpoint{1.155671in}{1.082356in}}%
\pgfpathlineto{\pgfqpoint{1.153787in}{1.096370in}}%
\pgfpathlineto{\pgfqpoint{1.155331in}{1.109702in}}%
\pgfpathlineto{\pgfqpoint{1.159125in}{1.122337in}}%
\pgfpathlineto{\pgfqpoint{1.165004in}{1.134270in}}%
\pgfpathlineto{\pgfqpoint{1.172846in}{1.145498in}}%
\pgfpathlineto{\pgfqpoint{1.182577in}{1.156018in}}%
\pgfpathlineto{\pgfqpoint{1.194165in}{1.165830in}}%
\pgfpathlineto{\pgfqpoint{1.207624in}{1.174936in}}%
\pgfpathlineto{\pgfqpoint{1.223013in}{1.183338in}}%
\pgfpathlineto{\pgfqpoint{1.240426in}{1.191040in}}%
\pgfpathlineto{\pgfqpoint{1.270251in}{1.201279in}}%
\pgfpathlineto{\pgfqpoint{1.304541in}{1.209930in}}%
\pgfpathlineto{\pgfqpoint{1.343483in}{1.216980in}}%
\pgfpathlineto{\pgfqpoint{1.387356in}{1.222413in}}%
\pgfpathlineto{\pgfqpoint{1.436539in}{1.226204in}}%
\pgfpathlineto{\pgfqpoint{1.491510in}{1.228327in}}%
\pgfpathlineto{\pgfqpoint{1.552840in}{1.228749in}}%
\pgfpathlineto{\pgfqpoint{1.621104in}{1.227438in}}%
\pgfpathlineto{\pgfqpoint{1.723657in}{1.222901in}}%
\pgfpathlineto{\pgfqpoint{1.842171in}{1.214885in}}%
\pgfpathlineto{\pgfqpoint{1.978340in}{1.203112in}}%
\pgfpathlineto{\pgfqpoint{2.131973in}{1.187367in}}%
\pgfpathlineto{\pgfqpoint{2.301003in}{1.167501in}}%
\pgfpathlineto{\pgfqpoint{2.435576in}{1.149846in}}%
\pgfpathlineto{\pgfqpoint{2.574236in}{1.129812in}}%
\pgfpathlineto{\pgfqpoint{2.712615in}{1.107449in}}%
\pgfpathlineto{\pgfqpoint{2.845294in}{1.083098in}}%
\pgfpathlineto{\pgfqpoint{2.928403in}{1.065961in}}%
\pgfpathlineto{\pgfqpoint{3.006028in}{1.048245in}}%
\pgfpathlineto{\pgfqpoint{3.077379in}{1.030075in}}%
\pgfpathlineto{\pgfqpoint{3.141881in}{1.011579in}}%
\pgfpathlineto{\pgfqpoint{3.199176in}{0.992885in}}%
\pgfpathlineto{\pgfqpoint{3.249124in}{0.974126in}}%
\pgfpathlineto{\pgfqpoint{3.291802in}{0.955433in}}%
\pgfpathlineto{\pgfqpoint{3.327555in}{0.936954in}}%
\pgfpathlineto{\pgfqpoint{3.357345in}{0.918799in}}%
\pgfpathlineto{\pgfqpoint{3.382191in}{0.901031in}}%
\pgfpathlineto{\pgfqpoint{3.402877in}{0.883709in}}%
\pgfpathlineto{\pgfqpoint{3.419947in}{0.866883in}}%
\pgfpathlineto{\pgfqpoint{3.433709in}{0.850599in}}%
\pgfpathlineto{\pgfqpoint{3.444231in}{0.834897in}}%
\pgfpathlineto{\pgfqpoint{3.451345in}{0.819812in}}%
\pgfpathlineto{\pgfqpoint{3.454704in}{0.805371in}}%
\pgfpathlineto{\pgfqpoint{3.455091in}{0.791602in}}%
\pgfpathlineto{\pgfqpoint{3.453111in}{0.778518in}}%
\pgfpathlineto{\pgfqpoint{3.448961in}{0.766125in}}%
\pgfpathlineto{\pgfqpoint{3.442787in}{0.754431in}}%
\pgfpathlineto{\pgfqpoint{3.434678in}{0.743437in}}%
\pgfpathlineto{\pgfqpoint{3.424674in}{0.733146in}}%
\pgfpathlineto{\pgfqpoint{3.412757in}{0.723556in}}%
\pgfpathlineto{\pgfqpoint{3.398859in}{0.714666in}}%
\pgfpathlineto{\pgfqpoint{3.382904in}{0.706471in}}%
\pgfpathlineto{\pgfqpoint{3.355315in}{0.695491in}}%
\pgfpathlineto{\pgfqpoint{3.323331in}{0.686093in}}%
\pgfpathlineto{\pgfqpoint{3.286819in}{0.678287in}}%
\pgfpathlineto{\pgfqpoint{3.245538in}{0.672090in}}%
\pgfpathlineto{\pgfqpoint{3.199145in}{0.667518in}}%
\pgfpathlineto{\pgfqpoint{3.147190in}{0.664592in}}%
\pgfpathlineto{\pgfqpoint{3.089116in}{0.663336in}}%
\pgfpathlineto{\pgfqpoint{3.024618in}{0.663766in}}%
\pgfpathlineto{\pgfqpoint{2.927519in}{0.667086in}}%
\pgfpathlineto{\pgfqpoint{2.814873in}{0.673815in}}%
\pgfpathlineto{\pgfqpoint{2.684973in}{0.684211in}}%
\pgfpathlineto{\pgfqpoint{2.537671in}{0.698484in}}%
\pgfpathlineto{\pgfqpoint{2.374380in}{0.716801in}}%
\pgfpathlineto{\pgfqpoint{2.198075in}{0.739280in}}%
\pgfpathlineto{\pgfqpoint{2.060004in}{0.758921in}}%
\pgfpathlineto{\pgfqpoint{1.921245in}{0.780900in}}%
\pgfpathlineto{\pgfqpoint{1.787457in}{0.804910in}}%
\pgfpathlineto{\pgfqpoint{1.703335in}{0.821861in}}%
\pgfpathlineto{\pgfqpoint{1.624556in}{0.839428in}}%
\pgfpathlineto{\pgfqpoint{1.551965in}{0.857487in}}%
\pgfpathlineto{\pgfqpoint{1.486182in}{0.875910in}}%
\pgfpathlineto{\pgfqpoint{1.427601in}{0.894561in}}%
\pgfpathlineto{\pgfqpoint{1.376386in}{0.913301in}}%
\pgfpathlineto{\pgfqpoint{1.332492in}{0.931978in}}%
\pgfpathlineto{\pgfqpoint{1.295287in}{0.950459in}}%
\pgfpathlineto{\pgfqpoint{1.263772in}{0.968659in}}%
\pgfpathlineto{\pgfqpoint{1.237146in}{0.986506in}}%
\pgfpathlineto{\pgfqpoint{1.214820in}{1.003930in}}%
\pgfpathlineto{\pgfqpoint{1.196419in}{1.020872in}}%
\pgfpathlineto{\pgfqpoint{1.181779in}{1.037281in}}%
\pgfpathlineto{\pgfqpoint{1.170948in}{1.053110in}}%
\pgfpathlineto{\pgfqpoint{1.164101in}{1.068321in}}%
\pgfpathlineto{\pgfqpoint{1.160402in}{1.082884in}}%
\pgfpathlineto{\pgfqpoint{1.159300in}{1.096780in}}%
\pgfpathlineto{\pgfqpoint{1.160544in}{1.109998in}}%
\pgfpathlineto{\pgfqpoint{1.163944in}{1.122530in}}%
\pgfpathlineto{\pgfqpoint{1.169382in}{1.134369in}}%
\pgfpathlineto{\pgfqpoint{1.176804in}{1.145511in}}%
\pgfpathlineto{\pgfqpoint{1.186223in}{1.155956in}}%
\pgfpathlineto{\pgfqpoint{1.197717in}{1.165706in}}%
\pgfpathlineto{\pgfqpoint{1.211281in}{1.174760in}}%
\pgfpathlineto{\pgfqpoint{1.226778in}{1.183117in}}%
\pgfpathlineto{\pgfqpoint{1.253601in}{1.194337in}}%
\pgfpathlineto{\pgfqpoint{1.284740in}{1.203972in}}%
\pgfpathlineto{\pgfqpoint{1.320334in}{1.212013in}}%
\pgfpathlineto{\pgfqpoint{1.360641in}{1.218451in}}%
\pgfpathlineto{\pgfqpoint{1.406042in}{1.223277in}}%
\pgfpathlineto{\pgfqpoint{1.457038in}{1.226479in}}%
\pgfpathlineto{\pgfqpoint{1.513873in}{1.228038in}}%
\pgfpathlineto{\pgfqpoint{1.577090in}{1.227891in}}%
\pgfpathlineto{\pgfqpoint{1.647741in}{1.225962in}}%
\pgfpathlineto{\pgfqpoint{1.754917in}{1.220479in}}%
\pgfpathlineto{\pgfqpoint{1.878139in}{1.211480in}}%
\pgfpathlineto{\pgfqpoint{2.018083in}{1.198748in}}%
\pgfpathlineto{\pgfqpoint{2.174728in}{1.182052in}}%
\pgfpathlineto{\pgfqpoint{2.347500in}{1.161137in}}%
\pgfpathlineto{\pgfqpoint{2.485442in}{1.142560in}}%
\pgfpathlineto{\pgfqpoint{2.624866in}{1.121655in}}%
\pgfpathlineto{\pgfqpoint{2.760850in}{1.098649in}}%
\pgfpathlineto{\pgfqpoint{2.889219in}{1.073828in}}%
\pgfpathlineto{\pgfqpoint{2.968845in}{1.056439in}}%
\pgfpathlineto{\pgfqpoint{3.042720in}{1.038512in}}%
\pgfpathlineto{\pgfqpoint{3.110161in}{1.020174in}}%
\pgfpathlineto{\pgfqpoint{3.170633in}{1.001564in}}%
\pgfpathlineto{\pgfqpoint{3.223750in}{0.982833in}}%
\pgfpathlineto{\pgfqpoint{3.269411in}{0.964139in}}%
\pgfpathlineto{\pgfqpoint{3.308378in}{0.945606in}}%
\pgfpathlineto{\pgfqpoint{3.341471in}{0.927327in}}%
\pgfpathlineto{\pgfqpoint{3.369357in}{0.909388in}}%
\pgfpathlineto{\pgfqpoint{3.392557in}{0.891861in}}%
\pgfpathlineto{\pgfqpoint{3.411450in}{0.874813in}}%
\pgfpathlineto{\pgfqpoint{3.426268in}{0.858297in}}%
\pgfpathlineto{\pgfqpoint{3.437133in}{0.842360in}}%
\pgfpathlineto{\pgfqpoint{3.444618in}{0.827039in}}%
\pgfpathlineto{\pgfqpoint{3.449202in}{0.812356in}}%
\pgfpathlineto{\pgfqpoint{3.451178in}{0.798329in}}%
\pgfpathlineto{\pgfqpoint{3.450766in}{0.784972in}}%
\pgfpathlineto{\pgfqpoint{3.448113in}{0.772297in}}%
\pgfpathlineto{\pgfqpoint{3.443294in}{0.760308in}}%
\pgfpathlineto{\pgfqpoint{3.436312in}{0.749011in}}%
\pgfpathlineto{\pgfqpoint{3.427184in}{0.738406in}}%
\pgfpathlineto{\pgfqpoint{3.416049in}{0.728495in}}%
\pgfpathlineto{\pgfqpoint{3.402992in}{0.719283in}}%
\pgfpathlineto{\pgfqpoint{3.388065in}{0.710771in}}%
\pgfpathlineto{\pgfqpoint{3.362210in}{0.699320in}}%
\pgfpathlineto{\pgfqpoint{3.332137in}{0.689454in}}%
\pgfpathlineto{\pgfqpoint{3.297633in}{0.681173in}}%
\pgfpathlineto{\pgfqpoint{3.258331in}{0.674475in}}%
\pgfpathlineto{\pgfqpoint{3.213770in}{0.669360in}}%
\pgfpathlineto{\pgfqpoint{3.163861in}{0.665860in}}%
\pgfpathlineto{\pgfqpoint{3.108053in}{0.664019in}}%
\pgfpathlineto{\pgfqpoint{3.045682in}{0.663886in}}%
\pgfpathlineto{\pgfqpoint{2.976101in}{0.665520in}}%
\pgfpathlineto{\pgfqpoint{2.871026in}{0.670572in}}%
\pgfpathlineto{\pgfqpoint{2.750572in}{0.679091in}}%
\pgfpathlineto{\pgfqpoint{2.613351in}{0.691309in}}%
\pgfpathlineto{\pgfqpoint{2.458426in}{0.707461in}}%
\pgfpathlineto{\pgfqpoint{2.287583in}{0.727809in}}%
\pgfpathlineto{\pgfqpoint{2.151900in}{0.745892in}}%
\pgfpathlineto{\pgfqpoint{2.013236in}{0.766329in}}%
\pgfpathlineto{\pgfqpoint{1.875764in}{0.788959in}}%
\pgfpathlineto{\pgfqpoint{1.744611in}{0.813529in}}%
\pgfpathlineto{\pgfqpoint{1.663287in}{0.830832in}}%
\pgfpathlineto{\pgfqpoint{1.587936in}{0.848704in}}%
\pgfpathlineto{\pgfqpoint{1.519141in}{0.866987in}}%
\pgfpathlineto{\pgfqpoint{1.457275in}{0.885531in}}%
\pgfpathlineto{\pgfqpoint{1.402502in}{0.904199in}}%
\pgfpathlineto{\pgfqpoint{1.354778in}{0.922867in}}%
\pgfpathlineto{\pgfqpoint{1.313849in}{0.941417in}}%
\pgfpathlineto{\pgfqpoint{1.279253in}{0.959747in}}%
\pgfpathlineto{\pgfqpoint{1.250321in}{0.977763in}}%
\pgfpathlineto{\pgfqpoint{1.226452in}{0.995376in}}%
\pgfpathlineto{\pgfqpoint{1.207477in}{1.012503in}}%
\pgfpathlineto{\pgfqpoint{1.192502in}{1.029104in}}%
\pgfpathlineto{\pgfqpoint{1.180782in}{1.045146in}}%
\pgfpathlineto{\pgfqpoint{1.171785in}{1.060601in}}%
\pgfpathlineto{\pgfqpoint{1.165194in}{1.075444in}}%
\pgfpathlineto{\pgfqpoint{1.160903in}{1.089655in}}%
\pgfpathlineto{\pgfqpoint{1.159024in}{1.103216in}}%
\pgfpathlineto{\pgfqpoint{1.159878in}{1.116117in}}%
\pgfpathlineto{\pgfqpoint{1.163982in}{1.128348in}}%
\pgfpathlineto{\pgfqpoint{1.170759in}{1.139891in}}%
\pgfpathlineto{\pgfqpoint{1.179565in}{1.150734in}}%
\pgfpathlineto{\pgfqpoint{1.190304in}{1.160873in}}%
\pgfpathlineto{\pgfqpoint{1.202911in}{1.170307in}}%
\pgfpathlineto{\pgfqpoint{1.217355in}{1.179035in}}%
\pgfpathlineto{\pgfqpoint{1.233640in}{1.187058in}}%
\pgfpathlineto{\pgfqpoint{1.261610in}{1.197772in}}%
\pgfpathlineto{\pgfqpoint{1.294079in}{1.206907in}}%
\pgfpathlineto{\pgfqpoint{1.331339in}{1.214466in}}%
\pgfpathlineto{\pgfqpoint{1.373446in}{1.220431in}}%
\pgfpathlineto{\pgfqpoint{1.420771in}{1.224777in}}%
\pgfpathlineto{\pgfqpoint{1.473769in}{1.227472in}}%
\pgfpathlineto{\pgfqpoint{1.532956in}{1.228475in}}%
\pgfpathlineto{\pgfqpoint{1.598908in}{1.227734in}}%
\pgfpathlineto{\pgfqpoint{1.698484in}{1.223929in}}%
\pgfpathlineto{\pgfqpoint{1.812920in}{1.216747in}}%
\pgfpathlineto{\pgfqpoint{1.943843in}{1.205961in}}%
\pgfpathlineto{\pgfqpoint{2.092302in}{1.191302in}}%
\pgfpathlineto{\pgfqpoint{2.257545in}{1.172539in}}%
\pgfpathlineto{\pgfqpoint{2.436414in}{1.149516in}}%
\pgfpathlineto{\pgfqpoint{2.575192in}{1.129470in}}%
\pgfpathlineto{\pgfqpoint{2.713341in}{1.107182in}}%
\pgfpathlineto{\pgfqpoint{2.846053in}{1.082894in}}%
\pgfpathlineto{\pgfqpoint{2.928868in}{1.065741in}}%
\pgfpathlineto{\pgfqpoint{3.005841in}{1.047971in}}%
\pgfpathlineto{\pgfqpoint{3.076382in}{1.029751in}}%
\pgfpathlineto{\pgfqpoint{3.140094in}{1.011234in}}%
\pgfpathlineto{\pgfqpoint{3.196768in}{0.992563in}}%
\pgfpathlineto{\pgfqpoint{3.246391in}{0.973867in}}%
\pgfpathlineto{\pgfqpoint{3.289136in}{0.955263in}}%
\pgfpathlineto{\pgfqpoint{3.325372in}{0.936858in}}%
\pgfpathlineto{\pgfqpoint{3.355657in}{0.918746in}}%
\pgfpathlineto{\pgfqpoint{3.380740in}{0.901007in}}%
\pgfpathlineto{\pgfqpoint{3.401171in}{0.883723in}}%
\pgfpathlineto{\pgfqpoint{3.417107in}{0.866962in}}%
\pgfpathlineto{\pgfqpoint{3.429377in}{0.850760in}}%
\pgfpathlineto{\pgfqpoint{3.438647in}{0.835144in}}%
\pgfpathlineto{\pgfqpoint{3.445386in}{0.820140in}}%
\pgfpathlineto{\pgfqpoint{3.449868in}{0.805768in}}%
\pgfpathlineto{\pgfqpoint{3.452173in}{0.792043in}}%
\pgfpathlineto{\pgfqpoint{3.452184in}{0.778979in}}%
\pgfpathlineto{\pgfqpoint{3.449589in}{0.766582in}}%
\pgfpathlineto{\pgfqpoint{3.443904in}{0.754857in}}%
\pgfpathlineto{\pgfqpoint{3.435662in}{0.743823in}}%
\pgfpathlineto{\pgfqpoint{3.425433in}{0.733490in}}%
\pgfpathlineto{\pgfqpoint{3.413295in}{0.723861in}}%
\pgfpathlineto{\pgfqpoint{3.399298in}{0.714937in}}%
\pgfpathlineto{\pgfqpoint{3.383460in}{0.706719in}}%
\pgfpathlineto{\pgfqpoint{3.356212in}{0.695715in}}%
\pgfpathlineto{\pgfqpoint{3.324607in}{0.686297in}}%
\pgfpathlineto{\pgfqpoint{3.288315in}{0.678459in}}%
\pgfpathlineto{\pgfqpoint{3.247104in}{0.672207in}}%
\pgfpathlineto{\pgfqpoint{3.200793in}{0.667565in}}%
\pgfpathlineto{\pgfqpoint{3.148897in}{0.664564in}}%
\pgfpathlineto{\pgfqpoint{3.090886in}{0.663245in}}%
\pgfpathlineto{\pgfqpoint{3.026184in}{0.663658in}}%
\pgfpathlineto{\pgfqpoint{2.928433in}{0.667009in}}%
\pgfpathlineto{\pgfqpoint{2.816096in}{0.673723in}}%
\pgfpathlineto{\pgfqpoint{2.687500in}{0.684006in}}%
\pgfpathlineto{\pgfqpoint{2.541466in}{0.698122in}}%
\pgfpathlineto{\pgfqpoint{2.378333in}{0.716319in}}%
\pgfpathlineto{\pgfqpoint{2.201041in}{0.738758in}}%
\pgfpathlineto{\pgfqpoint{2.062521in}{0.758380in}}%
\pgfpathlineto{\pgfqpoint{1.923931in}{0.780277in}}%
\pgfpathlineto{\pgfqpoint{1.789912in}{0.804226in}}%
\pgfpathlineto{\pgfqpoint{1.705503in}{0.821171in}}%
\pgfpathlineto{\pgfqpoint{1.626854in}{0.838772in}}%
\pgfpathlineto{\pgfqpoint{1.554761in}{0.856883in}}%
\pgfpathlineto{\pgfqpoint{1.489455in}{0.875339in}}%
\pgfpathlineto{\pgfqpoint{1.431044in}{0.893992in}}%
\pgfpathlineto{\pgfqpoint{1.379510in}{0.912704in}}%
\pgfpathlineto{\pgfqpoint{1.334716in}{0.931351in}}%
\pgfpathlineto{\pgfqpoint{1.296396in}{0.949823in}}%
\pgfpathlineto{\pgfqpoint{1.264164in}{0.968023in}}%
\pgfpathlineto{\pgfqpoint{1.237509in}{0.985868in}}%
\pgfpathlineto{\pgfqpoint{1.215796in}{1.003286in}}%
\pgfpathlineto{\pgfqpoint{1.198443in}{1.020215in}}%
\pgfpathlineto{\pgfqpoint{1.185336in}{1.036584in}}%
\pgfpathlineto{\pgfqpoint{1.175725in}{1.052364in}}%
\pgfpathlineto{\pgfqpoint{1.168945in}{1.067532in}}%
\pgfpathlineto{\pgfqpoint{1.164513in}{1.082068in}}%
\pgfpathlineto{\pgfqpoint{1.162123in}{1.095956in}}%
\pgfpathlineto{\pgfqpoint{1.161652in}{1.109185in}}%
\pgfpathlineto{\pgfqpoint{1.163157in}{1.121747in}}%
\pgfpathlineto{\pgfqpoint{1.166874in}{1.133636in}}%
\pgfpathlineto{\pgfqpoint{1.173221in}{1.144853in}}%
\pgfpathlineto{\pgfqpoint{1.182643in}{1.155396in}}%
\pgfpathlineto{\pgfqpoint{1.194291in}{1.165242in}}%
\pgfpathlineto{\pgfqpoint{1.207852in}{1.174383in}}%
\pgfpathlineto{\pgfqpoint{1.223289in}{1.182819in}}%
\pgfpathlineto{\pgfqpoint{1.249945in}{1.194148in}}%
\pgfpathlineto{\pgfqpoint{1.280874in}{1.203884in}}%
\pgfpathlineto{\pgfqpoint{1.316282in}{1.212025in}}%
\pgfpathlineto{\pgfqpoint{1.356505in}{1.218571in}}%
\pgfpathlineto{\pgfqpoint{1.401893in}{1.223518in}}%
\pgfpathlineto{\pgfqpoint{1.452634in}{1.226836in}}%
\pgfpathlineto{\pgfqpoint{1.509367in}{1.228484in}}%
\pgfpathlineto{\pgfqpoint{1.572752in}{1.228412in}}%
\pgfpathlineto{\pgfqpoint{1.643435in}{1.226559in}}%
\pgfpathlineto{\pgfqpoint{1.750114in}{1.221193in}}%
\pgfpathlineto{\pgfqpoint{1.872326in}{1.212331in}}%
\pgfpathlineto{\pgfqpoint{2.011459in}{1.199739in}}%
\pgfpathlineto{\pgfqpoint{2.168348in}{1.183182in}}%
\pgfpathlineto{\pgfqpoint{2.340854in}{1.162400in}}%
\pgfpathlineto{\pgfqpoint{2.477386in}{1.143984in}}%
\pgfpathlineto{\pgfqpoint{2.616416in}{1.123223in}}%
\pgfpathlineto{\pgfqpoint{2.753657in}{1.100292in}}%
\pgfpathlineto{\pgfqpoint{2.841615in}{1.083927in}}%
\pgfpathlineto{\pgfqpoint{2.924802in}{1.066808in}}%
\pgfpathlineto{\pgfqpoint{3.002170in}{1.049064in}}%
\pgfpathlineto{\pgfqpoint{3.073103in}{1.030860in}}%
\pgfpathlineto{\pgfqpoint{3.137186in}{1.012349in}}%
\pgfpathlineto{\pgfqpoint{3.194204in}{0.993673in}}%
\pgfpathlineto{\pgfqpoint{3.244141in}{0.974963in}}%
\pgfpathlineto{\pgfqpoint{3.287182in}{0.956337in}}%
\pgfpathlineto{\pgfqpoint{3.323709in}{0.937903in}}%
\pgfpathlineto{\pgfqpoint{3.354308in}{0.919757in}}%
\pgfpathlineto{\pgfqpoint{3.379744in}{0.901984in}}%
\pgfpathlineto{\pgfqpoint{3.400206in}{0.884676in}}%
\pgfpathlineto{\pgfqpoint{3.416294in}{0.867886in}}%
\pgfpathlineto{\pgfqpoint{3.428850in}{0.851648in}}%
\pgfpathlineto{\pgfqpoint{3.438506in}{0.835992in}}%
\pgfpathlineto{\pgfqpoint{3.445682in}{0.820944in}}%
\pgfpathlineto{\pgfqpoint{3.450589in}{0.806523in}}%
\pgfpathlineto{\pgfqpoint{3.453225in}{0.792747in}}%
\pgfpathlineto{\pgfqpoint{3.453380in}{0.779630in}}%
\pgfpathlineto{\pgfqpoint{3.450633in}{0.767179in}}%
\pgfpathlineto{\pgfqpoint{3.444667in}{0.755405in}}%
\pgfpathlineto{\pgfqpoint{3.436484in}{0.744328in}}%
\pgfpathlineto{\pgfqpoint{3.426332in}{0.733955in}}%
\pgfpathlineto{\pgfqpoint{3.414289in}{0.724287in}}%
\pgfpathlineto{\pgfqpoint{3.400398in}{0.715324in}}%
\pgfpathlineto{\pgfqpoint{3.384672in}{0.707068in}}%
\pgfpathlineto{\pgfqpoint{3.357586in}{0.696008in}}%
\pgfpathlineto{\pgfqpoint{3.326107in}{0.686530in}}%
\pgfpathlineto{\pgfqpoint{3.289887in}{0.678630in}}%
\pgfpathlineto{\pgfqpoint{3.248834in}{0.672318in}}%
\pgfpathlineto{\pgfqpoint{3.202663in}{0.667619in}}%
\pgfpathlineto{\pgfqpoint{3.150920in}{0.664563in}}%
\pgfpathlineto{\pgfqpoint{3.093097in}{0.663188in}}%
\pgfpathlineto{\pgfqpoint{3.028626in}{0.663546in}}%
\pgfpathlineto{\pgfqpoint{2.931242in}{0.666820in}}%
\pgfpathlineto{\pgfqpoint{2.819294in}{0.673447in}}%
\pgfpathlineto{\pgfqpoint{2.691097in}{0.683635in}}%
\pgfpathlineto{\pgfqpoint{2.545449in}{0.697655in}}%
\pgfpathlineto{\pgfqpoint{2.382727in}{0.715746in}}%
\pgfpathlineto{\pgfqpoint{2.205649in}{0.738082in}}%
\pgfpathlineto{\pgfqpoint{2.067264in}{0.757628in}}%
\pgfpathlineto{\pgfqpoint{1.928551in}{0.779456in}}%
\pgfpathlineto{\pgfqpoint{1.794211in}{0.803346in}}%
\pgfpathlineto{\pgfqpoint{1.709699in}{0.820273in}}%
\pgfpathlineto{\pgfqpoint{1.630741in}{0.837867in}}%
\pgfpathlineto{\pgfqpoint{1.558028in}{0.855961in}}%
\pgfpathlineto{\pgfqpoint{1.492042in}{0.874399in}}%
\pgfpathlineto{\pgfqpoint{1.433068in}{0.893036in}}%
\pgfpathlineto{\pgfqpoint{1.381187in}{0.911740in}}%
\pgfpathlineto{\pgfqpoint{1.336283in}{0.930389in}}%
\pgfpathlineto{\pgfqpoint{1.298037in}{0.948873in}}%
\pgfpathlineto{\pgfqpoint{1.265931in}{0.967095in}}%
\pgfpathlineto{\pgfqpoint{1.239246in}{0.984968in}}%
\pgfpathlineto{\pgfqpoint{1.217340in}{1.002410in}}%
\pgfpathlineto{\pgfqpoint{1.200100in}{1.019344in}}%
\pgfpathlineto{\pgfqpoint{1.186697in}{1.035731in}}%
\pgfpathlineto{\pgfqpoint{1.176427in}{1.051543in}}%
\pgfpathlineto{\pgfqpoint{1.168785in}{1.066751in}}%
\pgfpathlineto{\pgfqpoint{1.163468in}{1.081335in}}%
\pgfpathlineto{\pgfqpoint{1.160372in}{1.095276in}}%
\pgfpathlineto{\pgfqpoint{1.159595in}{1.108561in}}%
\pgfpathlineto{\pgfqpoint{1.161433in}{1.121180in}}%
\pgfpathlineto{\pgfqpoint{1.166375in}{1.133128in}}%
\pgfpathlineto{\pgfqpoint{1.173976in}{1.144387in}}%
\pgfpathlineto{\pgfqpoint{1.183584in}{1.154946in}}%
\pgfpathlineto{\pgfqpoint{1.195111in}{1.164800in}}%
\pgfpathlineto{\pgfqpoint{1.208502in}{1.173950in}}%
\pgfpathlineto{\pgfqpoint{1.223732in}{1.182393in}}%
\pgfpathlineto{\pgfqpoint{1.250051in}{1.193736in}}%
\pgfpathlineto{\pgfqpoint{1.280695in}{1.203493in}}%
\pgfpathlineto{\pgfqpoint{1.315986in}{1.211671in}}%
\pgfpathlineto{\pgfqpoint{1.356150in}{1.218266in}}%
\pgfpathlineto{\pgfqpoint{1.401342in}{1.223256in}}%
\pgfpathlineto{\pgfqpoint{1.452019in}{1.226610in}}%
\pgfpathlineto{\pgfqpoint{1.508687in}{1.228291in}}%
\pgfpathlineto{\pgfqpoint{1.571906in}{1.228251in}}%
\pgfpathlineto{\pgfqpoint{1.642286in}{1.226432in}}%
\pgfpathlineto{\pgfqpoint{1.748416in}{1.221119in}}%
\pgfpathlineto{\pgfqpoint{1.870154in}{1.212331in}}%
\pgfpathlineto{\pgfqpoint{2.008965in}{1.199822in}}%
\pgfpathlineto{\pgfqpoint{2.165276in}{1.183325in}}%
\pgfpathlineto{\pgfqpoint{2.337296in}{1.162640in}}%
\pgfpathlineto{\pgfqpoint{2.473882in}{1.144319in}}%
\pgfpathlineto{\pgfqpoint{2.612959in}{1.123646in}}%
\pgfpathlineto{\pgfqpoint{2.750127in}{1.100785in}}%
\pgfpathlineto{\pgfqpoint{2.838103in}{1.084459in}}%
\pgfpathlineto{\pgfqpoint{2.921450in}{1.067375in}}%
\pgfpathlineto{\pgfqpoint{2.998993in}{1.049657in}}%
\pgfpathlineto{\pgfqpoint{3.070134in}{1.031472in}}%
\pgfpathlineto{\pgfqpoint{3.134469in}{1.012975in}}%
\pgfpathlineto{\pgfqpoint{3.191784in}{0.994311in}}%
\pgfpathlineto{\pgfqpoint{3.242048in}{0.975610in}}%
\pgfpathlineto{\pgfqpoint{3.285423in}{0.956991in}}%
\pgfpathlineto{\pgfqpoint{3.322256in}{0.938562in}}%
\pgfpathlineto{\pgfqpoint{3.353082in}{0.920417in}}%
\pgfpathlineto{\pgfqpoint{3.378624in}{0.902640in}}%
\pgfpathlineto{\pgfqpoint{3.399491in}{0.885308in}}%
\pgfpathlineto{\pgfqpoint{3.415797in}{0.868498in}}%
\pgfpathlineto{\pgfqpoint{3.428367in}{0.852242in}}%
\pgfpathlineto{\pgfqpoint{3.437887in}{0.836571in}}%
\pgfpathlineto{\pgfqpoint{3.444849in}{0.821510in}}%
\pgfpathlineto{\pgfqpoint{3.449545in}{0.807077in}}%
\pgfpathlineto{\pgfqpoint{3.452075in}{0.793291in}}%
\pgfpathlineto{\pgfqpoint{3.452338in}{0.780164in}}%
\pgfpathlineto{\pgfqpoint{3.450041in}{0.767704in}}%
\pgfpathlineto{\pgfqpoint{3.444692in}{0.755915in}}%
\pgfpathlineto{\pgfqpoint{3.436635in}{0.744814in}}%
\pgfpathlineto{\pgfqpoint{3.426581in}{0.734414in}}%
\pgfpathlineto{\pgfqpoint{3.414613in}{0.724718in}}%
\pgfpathlineto{\pgfqpoint{3.400782in}{0.715727in}}%
\pgfpathlineto{\pgfqpoint{3.385109in}{0.707441in}}%
\pgfpathlineto{\pgfqpoint{3.358114in}{0.696337in}}%
\pgfpathlineto{\pgfqpoint{3.326781in}{0.686820in}}%
\pgfpathlineto{\pgfqpoint{3.290790in}{0.678884in}}%
\pgfpathlineto{\pgfqpoint{3.249878in}{0.672531in}}%
\pgfpathlineto{\pgfqpoint{3.203895in}{0.667788in}}%
\pgfpathlineto{\pgfqpoint{3.152358in}{0.664684in}}%
\pgfpathlineto{\pgfqpoint{3.094739in}{0.663258in}}%
\pgfpathlineto{\pgfqpoint{3.030465in}{0.663561in}}%
\pgfpathlineto{\pgfqpoint{2.933345in}{0.666761in}}%
\pgfpathlineto{\pgfqpoint{2.821717in}{0.673314in}}%
\pgfpathlineto{\pgfqpoint{2.693903in}{0.683426in}}%
\pgfpathlineto{\pgfqpoint{2.548685in}{0.697358in}}%
\pgfpathlineto{\pgfqpoint{2.386292in}{0.715361in}}%
\pgfpathlineto{\pgfqpoint{2.209547in}{0.737602in}}%
\pgfpathlineto{\pgfqpoint{2.071164in}{0.757079in}}%
\pgfpathlineto{\pgfqpoint{1.932437in}{0.778842in}}%
\pgfpathlineto{\pgfqpoint{1.797997in}{0.802673in}}%
\pgfpathlineto{\pgfqpoint{1.713129in}{0.819553in}}%
\pgfpathlineto{\pgfqpoint{1.633868in}{0.837096in}}%
\pgfpathlineto{\pgfqpoint{1.561180in}{0.855166in}}%
\pgfpathlineto{\pgfqpoint{1.495280in}{0.873598in}}%
\pgfpathlineto{\pgfqpoint{1.436259in}{0.892240in}}%
\pgfpathlineto{\pgfqpoint{1.384097in}{0.910953in}}%
\pgfpathlineto{\pgfqpoint{1.338659in}{0.929612in}}%
\pgfpathlineto{\pgfqpoint{1.299694in}{0.948107in}}%
\pgfpathlineto{\pgfqpoint{1.266840in}{0.966337in}}%
\pgfpathlineto{\pgfqpoint{1.239620in}{0.984219in}}%
\pgfpathlineto{\pgfqpoint{1.217443in}{1.001681in}}%
\pgfpathlineto{\pgfqpoint{1.199667in}{1.018662in}}%
\pgfpathlineto{\pgfqpoint{1.186176in}{1.035089in}}%
\pgfpathlineto{\pgfqpoint{1.176309in}{1.050928in}}%
\pgfpathlineto{\pgfqpoint{1.169368in}{1.066154in}}%
\pgfpathlineto{\pgfqpoint{1.164831in}{1.080750in}}%
\pgfpathlineto{\pgfqpoint{1.162356in}{1.094699in}}%
\pgfpathlineto{\pgfqpoint{1.161778in}{1.107989in}}%
\pgfpathlineto{\pgfqpoint{1.163112in}{1.120611in}}%
\pgfpathlineto{\pgfqpoint{1.166551in}{1.132561in}}%
\pgfpathlineto{\pgfqpoint{1.172464in}{1.143838in}}%
\pgfpathlineto{\pgfqpoint{1.181391in}{1.154443in}}%
\pgfpathlineto{\pgfqpoint{1.192858in}{1.164358in}}%
\pgfpathlineto{\pgfqpoint{1.206249in}{1.173570in}}%
\pgfpathlineto{\pgfqpoint{1.221523in}{1.182075in}}%
\pgfpathlineto{\pgfqpoint{1.247940in}{1.193509in}}%
\pgfpathlineto{\pgfqpoint{1.278625in}{1.203349in}}%
\pgfpathlineto{\pgfqpoint{1.313763in}{1.211594in}}%
\pgfpathlineto{\pgfqpoint{1.353672in}{1.218242in}}%
\pgfpathlineto{\pgfqpoint{1.398743in}{1.223290in}}%
\pgfpathlineto{\pgfqpoint{1.449132in}{1.226712in}}%
\pgfpathlineto{\pgfqpoint{1.505448in}{1.228468in}}%
\pgfpathlineto{\pgfqpoint{1.568381in}{1.228507in}}%
\pgfpathlineto{\pgfqpoint{1.638591in}{1.226770in}}%
\pgfpathlineto{\pgfqpoint{1.744613in}{1.221564in}}%
\pgfpathlineto{\pgfqpoint{1.866110in}{1.212873in}}%
\pgfpathlineto{\pgfqpoint{2.004409in}{1.200462in}}%
\pgfpathlineto{\pgfqpoint{2.160682in}{1.184101in}}%
\pgfpathlineto{\pgfqpoint{2.332894in}{1.163541in}}%
\pgfpathlineto{\pgfqpoint{2.469073in}{1.145277in}}%
\pgfpathlineto{\pgfqpoint{2.607705in}{1.124644in}}%
\pgfpathlineto{\pgfqpoint{2.744855in}{1.101815in}}%
\pgfpathlineto{\pgfqpoint{2.875914in}{1.077068in}}%
\pgfpathlineto{\pgfqpoint{2.957324in}{1.059690in}}%
\pgfpathlineto{\pgfqpoint{3.032023in}{1.041760in}}%
\pgfpathlineto{\pgfqpoint{3.099482in}{1.023421in}}%
\pgfpathlineto{\pgfqpoint{3.159952in}{1.004827in}}%
\pgfpathlineto{\pgfqpoint{3.213701in}{0.986116in}}%
\pgfpathlineto{\pgfqpoint{3.261004in}{0.967418in}}%
\pgfpathlineto{\pgfqpoint{3.302149in}{0.948848in}}%
\pgfpathlineto{\pgfqpoint{3.337431in}{0.930508in}}%
\pgfpathlineto{\pgfqpoint{3.367156in}{0.912488in}}%
\pgfpathlineto{\pgfqpoint{3.391639in}{0.894866in}}%
\pgfpathlineto{\pgfqpoint{3.411205in}{0.877708in}}%
\pgfpathlineto{\pgfqpoint{3.426187in}{0.861065in}}%
\pgfpathlineto{\pgfqpoint{3.436941in}{0.845001in}}%
\pgfpathlineto{\pgfqpoint{3.444168in}{0.829559in}}%
\pgfpathlineto{\pgfqpoint{3.448571in}{0.814755in}}%
\pgfpathlineto{\pgfqpoint{3.450693in}{0.800606in}}%
\pgfpathlineto{\pgfqpoint{3.450914in}{0.787121in}}%
\pgfpathlineto{\pgfqpoint{3.449449in}{0.774310in}}%
\pgfpathlineto{\pgfqpoint{3.446353in}{0.762178in}}%
\pgfpathlineto{\pgfqpoint{3.441517in}{0.750726in}}%
\pgfpathlineto{\pgfqpoint{3.434670in}{0.739953in}}%
\pgfpathlineto{\pgfqpoint{3.425378in}{0.729854in}}%
\pgfpathlineto{\pgfqpoint{3.413057in}{0.720424in}}%
\pgfpathlineto{\pgfqpoint{3.398227in}{0.711682in}}%
\pgfpathlineto{\pgfqpoint{3.381496in}{0.703647in}}%
\pgfpathlineto{\pgfqpoint{3.352837in}{0.692924in}}%
\pgfpathlineto{\pgfqpoint{3.319822in}{0.683799in}}%
\pgfpathlineto{\pgfqpoint{3.282261in}{0.676280in}}%
\pgfpathlineto{\pgfqpoint{3.239852in}{0.670376in}}%
\pgfpathlineto{\pgfqpoint{3.192182in}{0.666097in}}%
\pgfpathlineto{\pgfqpoint{3.138881in}{0.663456in}}%
\pgfpathlineto{\pgfqpoint{3.079671in}{0.662492in}}%
\pgfpathlineto{\pgfqpoint{3.013563in}{0.663272in}}%
\pgfpathlineto{\pgfqpoint{2.913197in}{0.667152in}}%
\pgfpathlineto{\pgfqpoint{2.797439in}{0.674454in}}%
\pgfpathlineto{\pgfqpoint{2.665263in}{0.685390in}}%
\pgfpathlineto{\pgfqpoint{2.516207in}{0.700191in}}%
\pgfpathlineto{\pgfqpoint{2.350322in}{0.719110in}}%
\pgfpathlineto{\pgfqpoint{2.169139in}{0.742409in}}%
\pgfpathlineto{\pgfqpoint{2.029467in}{0.762666in}}%
\pgfpathlineto{\pgfqpoint{1.891992in}{0.785103in}}%
\pgfpathlineto{\pgfqpoint{1.760922in}{0.809462in}}%
\pgfpathlineto{\pgfqpoint{1.678891in}{0.826612in}}%
\pgfpathlineto{\pgfqpoint{1.602209in}{0.844362in}}%
\pgfpathlineto{\pgfqpoint{1.531652in}{0.862584in}}%
\pgfpathlineto{\pgfqpoint{1.467870in}{0.881140in}}%
\pgfpathlineto{\pgfqpoint{1.411389in}{0.899877in}}%
\pgfpathlineto{\pgfqpoint{1.362495in}{0.918634in}}%
\pgfpathlineto{\pgfqpoint{1.320524in}{0.937284in}}%
\pgfpathlineto{\pgfqpoint{1.284722in}{0.955722in}}%
\pgfpathlineto{\pgfqpoint{1.254472in}{0.973858in}}%
\pgfpathlineto{\pgfqpoint{1.229260in}{0.991608in}}%
\pgfpathlineto{\pgfqpoint{1.208669in}{1.008902in}}%
\pgfpathlineto{\pgfqpoint{1.192383in}{1.025681in}}%
\pgfpathlineto{\pgfqpoint{1.180050in}{1.041895in}}%
\pgfpathlineto{\pgfqpoint{1.171094in}{1.057507in}}%
\pgfpathlineto{\pgfqpoint{1.165138in}{1.072493in}}%
\pgfpathlineto{\pgfqpoint{1.161892in}{1.086831in}}%
\pgfpathlineto{\pgfqpoint{1.161141in}{1.100504in}}%
\pgfpathlineto{\pgfqpoint{1.162752in}{1.113502in}}%
\pgfpathlineto{\pgfqpoint{1.166668in}{1.125815in}}%
\pgfpathlineto{\pgfqpoint{1.172835in}{1.137439in}}%
\pgfpathlineto{\pgfqpoint{1.181097in}{1.148372in}}%
\pgfpathlineto{\pgfqpoint{1.191323in}{1.158607in}}%
\pgfpathlineto{\pgfqpoint{1.203425in}{1.168143in}}%
\pgfpathlineto{\pgfqpoint{1.217349in}{1.176977in}}%
\pgfpathlineto{\pgfqpoint{1.233082in}{1.185108in}}%
\pgfpathlineto{\pgfqpoint{1.260138in}{1.195989in}}%
\pgfpathlineto{\pgfqpoint{1.291573in}{1.205294in}}%
\pgfpathlineto{\pgfqpoint{1.327831in}{1.213035in}}%
\pgfpathlineto{\pgfqpoint{1.369125in}{1.219202in}}%
\pgfpathlineto{\pgfqpoint{1.415548in}{1.223761in}}%
\pgfpathlineto{\pgfqpoint{1.467579in}{1.226682in}}%
\pgfpathlineto{\pgfqpoint{1.525742in}{1.227924in}}%
\pgfpathlineto{\pgfqpoint{1.590610in}{1.227434in}}%
\pgfpathlineto{\pgfqpoint{1.688606in}{1.223983in}}%
\pgfpathlineto{\pgfqpoint{1.801230in}{1.217170in}}%
\pgfpathlineto{\pgfqpoint{1.930145in}{1.206787in}}%
\pgfpathlineto{\pgfqpoint{2.076496in}{1.192568in}}%
\pgfpathlineto{\pgfqpoint{2.239838in}{1.174270in}}%
\pgfpathlineto{\pgfqpoint{2.417222in}{1.151733in}}%
\pgfpathlineto{\pgfqpoint{2.555577in}{1.132044in}}%
\pgfpathlineto{\pgfqpoint{2.693894in}{1.110091in}}%
\pgfpathlineto{\pgfqpoint{2.827507in}{1.086099in}}%
\pgfpathlineto{\pgfqpoint{2.911440in}{1.069128in}}%
\pgfpathlineto{\pgfqpoint{2.989634in}{1.051504in}}%
\pgfpathlineto{\pgfqpoint{3.061469in}{1.033389in}}%
\pgfpathlineto{\pgfqpoint{3.126555in}{1.014942in}}%
\pgfpathlineto{\pgfqpoint{3.184672in}{0.996310in}}%
\pgfpathlineto{\pgfqpoint{3.235781in}{0.977626in}}%
\pgfpathlineto{\pgfqpoint{3.280015in}{0.959010in}}%
\pgfpathlineto{\pgfqpoint{3.317680in}{0.940572in}}%
\pgfpathlineto{\pgfqpoint{3.349262in}{0.922407in}}%
\pgfpathlineto{\pgfqpoint{3.375418in}{0.904600in}}%
\pgfpathlineto{\pgfqpoint{3.396896in}{0.887224in}}%
\pgfpathlineto{\pgfqpoint{3.413773in}{0.870360in}}%
\pgfpathlineto{\pgfqpoint{3.426728in}{0.854049in}}%
\pgfpathlineto{\pgfqpoint{3.436498in}{0.838320in}}%
\pgfpathlineto{\pgfqpoint{3.443627in}{0.823197in}}%
\pgfpathlineto{\pgfqpoint{3.448468in}{0.808703in}}%
\pgfpathlineto{\pgfqpoint{3.451182in}{0.794852in}}%
\pgfpathlineto{\pgfqpoint{3.451736in}{0.781659in}}%
\pgfpathlineto{\pgfqpoint{3.449906in}{0.769132in}}%
\pgfpathlineto{\pgfqpoint{3.445277in}{0.757275in}}%
\pgfpathlineto{\pgfqpoint{3.437590in}{0.746095in}}%
\pgfpathlineto{\pgfqpoint{3.427780in}{0.735614in}}%
\pgfpathlineto{\pgfqpoint{3.416035in}{0.725836in}}%
\pgfpathlineto{\pgfqpoint{3.402413in}{0.716763in}}%
\pgfpathlineto{\pgfqpoint{3.386942in}{0.708395in}}%
\pgfpathlineto{\pgfqpoint{3.360254in}{0.697166in}}%
\pgfpathlineto{\pgfqpoint{3.329262in}{0.687525in}}%
\pgfpathlineto{\pgfqpoint{3.293683in}{0.679469in}}%
\pgfpathlineto{\pgfqpoint{3.253169in}{0.672996in}}%
\pgfpathlineto{\pgfqpoint{3.207614in}{0.668127in}}%
\pgfpathlineto{\pgfqpoint{3.156557in}{0.664893in}}%
\pgfpathlineto{\pgfqpoint{3.099455in}{0.663334in}}%
\pgfpathlineto{\pgfqpoint{3.035729in}{0.663497in}}%
\pgfpathlineto{\pgfqpoint{2.964766in}{0.665445in}}%
\pgfpathlineto{\pgfqpoint{2.857771in}{0.670940in}}%
\pgfpathlineto{\pgfqpoint{2.735133in}{0.679930in}}%
\pgfpathlineto{\pgfqpoint{2.595385in}{0.692651in}}%
\pgfpathlineto{\pgfqpoint{2.438128in}{0.709375in}}%
\pgfpathlineto{\pgfqpoint{2.265341in}{0.730297in}}%
\pgfpathlineto{\pgfqpoint{2.128464in}{0.748792in}}%
\pgfpathlineto{\pgfqpoint{1.989219in}{0.769637in}}%
\pgfpathlineto{\pgfqpoint{1.852481in}{0.792649in}}%
\pgfpathlineto{\pgfqpoint{1.765037in}{0.809060in}}%
\pgfpathlineto{\pgfqpoint{1.682094in}{0.826202in}}%
\pgfpathlineto{\pgfqpoint{1.604890in}{0.843948in}}%
\pgfpathlineto{\pgfqpoint{1.534919in}{0.862146in}}%
\pgfpathlineto{\pgfqpoint{1.472360in}{0.880644in}}%
\pgfpathlineto{\pgfqpoint{1.416706in}{0.899304in}}%
\pgfpathlineto{\pgfqpoint{1.367498in}{0.917999in}}%
\pgfpathlineto{\pgfqpoint{1.324335in}{0.936614in}}%
\pgfpathlineto{\pgfqpoint{1.286866in}{0.955043in}}%
\pgfpathlineto{\pgfqpoint{1.254795in}{0.973192in}}%
\pgfpathlineto{\pgfqpoint{1.227880in}{0.990977in}}%
\pgfpathlineto{\pgfqpoint{1.205931in}{1.008326in}}%
\pgfpathlineto{\pgfqpoint{1.188813in}{1.025176in}}%
\pgfpathlineto{\pgfqpoint{1.176417in}{1.041477in}}%
\pgfpathlineto{\pgfqpoint{1.167957in}{1.057169in}}%
\pgfpathlineto{\pgfqpoint{1.162623in}{1.072222in}}%
\pgfpathlineto{\pgfqpoint{1.159899in}{1.086620in}}%
\pgfpathlineto{\pgfqpoint{1.159396in}{1.100349in}}%
\pgfpathlineto{\pgfqpoint{1.160856in}{1.113398in}}%
\pgfpathlineto{\pgfqpoint{1.164144in}{1.125762in}}%
\pgfpathlineto{\pgfqpoint{1.169256in}{1.137437in}}%
\pgfpathlineto{\pgfqpoint{1.176316in}{1.148422in}}%
\pgfpathlineto{\pgfqpoint{1.185574in}{1.158719in}}%
\pgfpathlineto{\pgfqpoint{1.197407in}{1.168336in}}%
\pgfpathlineto{\pgfqpoint{1.211592in}{1.177262in}}%
\pgfpathlineto{\pgfqpoint{1.227687in}{1.185484in}}%
\pgfpathlineto{\pgfqpoint{1.255393in}{1.196493in}}%
\pgfpathlineto{\pgfqpoint{1.287430in}{1.205908in}}%
\pgfpathlineto{\pgfqpoint{1.323967in}{1.213722in}}%
\pgfpathlineto{\pgfqpoint{1.365288in}{1.219927in}}%
\pgfpathlineto{\pgfqpoint{1.411790in}{1.224514in}}%
\pgfpathlineto{\pgfqpoint{1.463899in}{1.227471in}}%
\pgfpathlineto{\pgfqpoint{1.521800in}{1.228764in}}%
\pgfpathlineto{\pgfqpoint{1.586411in}{1.228329in}}%
\pgfpathlineto{\pgfqpoint{1.684526in}{1.224936in}}%
\pgfpathlineto{\pgfqpoint{1.797801in}{1.218155in}}%
\pgfpathlineto{\pgfqpoint{1.927346in}{1.207778in}}%
\pgfpathlineto{\pgfqpoint{2.073728in}{1.193576in}}%
\pgfpathlineto{\pgfqpoint{2.236979in}{1.175298in}}%
\pgfpathlineto{\pgfqpoint{2.416275in}{1.152668in}}%
\pgfpathlineto{\pgfqpoint{2.555638in}{1.132887in}}%
\pgfpathlineto{\pgfqpoint{2.693696in}{1.110886in}}%
\pgfpathlineto{\pgfqpoint{2.826165in}{1.086907in}}%
\pgfpathlineto{\pgfqpoint{2.909533in}{1.069971in}}%
\pgfpathlineto{\pgfqpoint{2.987834in}{1.052401in}}%
\pgfpathlineto{\pgfqpoint{3.060252in}{1.034319in}}%
\pgfpathlineto{\pgfqpoint{3.126091in}{1.015861in}}%
\pgfpathlineto{\pgfqpoint{3.184777in}{0.997175in}}%
\pgfpathlineto{\pgfqpoint{3.235883in}{0.978424in}}%
\pgfpathlineto{\pgfqpoint{3.279826in}{0.959746in}}%
\pgfpathlineto{\pgfqpoint{3.317410in}{0.941248in}}%
\pgfpathlineto{\pgfqpoint{3.349268in}{0.923027in}}%
\pgfpathlineto{\pgfqpoint{3.375939in}{0.905166in}}%
\pgfpathlineto{\pgfqpoint{3.397867in}{0.887741in}}%
\pgfpathlineto{\pgfqpoint{3.415398in}{0.870815in}}%
\pgfpathlineto{\pgfqpoint{3.428844in}{0.854440in}}%
\pgfpathlineto{\pgfqpoint{3.438774in}{0.838656in}}%
\pgfpathlineto{\pgfqpoint{3.445603in}{0.823490in}}%
\pgfpathlineto{\pgfqpoint{3.449645in}{0.808966in}}%
\pgfpathlineto{\pgfqpoint{3.451132in}{0.795101in}}%
\pgfpathlineto{\pgfqpoint{3.450219in}{0.781909in}}%
\pgfpathlineto{\pgfqpoint{3.446983in}{0.769399in}}%
\pgfpathlineto{\pgfqpoint{3.441463in}{0.757576in}}%
\pgfpathlineto{\pgfqpoint{3.433808in}{0.746443in}}%
\pgfpathlineto{\pgfqpoint{3.424159in}{0.736006in}}%
\pgfpathlineto{\pgfqpoint{3.412618in}{0.726268in}}%
\pgfpathlineto{\pgfqpoint{3.399245in}{0.717231in}}%
\pgfpathlineto{\pgfqpoint{3.384065in}{0.708896in}}%
\pgfpathlineto{\pgfqpoint{3.357859in}{0.697709in}}%
\pgfpathlineto{\pgfqpoint{3.327324in}{0.688098in}}%
\pgfpathlineto{\pgfqpoint{3.292041in}{0.680052in}}%
\pgfpathlineto{\pgfqpoint{3.251720in}{0.673574in}}%
\pgfpathlineto{\pgfqpoint{3.206340in}{0.668699in}}%
\pgfpathlineto{\pgfqpoint{3.155447in}{0.665456in}}%
\pgfpathlineto{\pgfqpoint{3.098532in}{0.663883in}}%
\pgfpathlineto{\pgfqpoint{3.035032in}{0.664030in}}%
\pgfpathlineto{\pgfqpoint{2.964337in}{0.665957in}}%
\pgfpathlineto{\pgfqpoint{2.857736in}{0.671413in}}%
\pgfpathlineto{\pgfqpoint{2.735475in}{0.680349in}}%
\pgfpathlineto{\pgfqpoint{2.596124in}{0.693013in}}%
\pgfpathlineto{\pgfqpoint{2.439336in}{0.709666in}}%
\pgfpathlineto{\pgfqpoint{2.266999in}{0.730508in}}%
\pgfpathlineto{\pgfqpoint{2.130388in}{0.748941in}}%
\pgfpathlineto{\pgfqpoint{1.991517in}{0.769716in}}%
\pgfpathlineto{\pgfqpoint{1.854798in}{0.792662in}}%
\pgfpathlineto{\pgfqpoint{1.767270in}{0.809033in}}%
\pgfpathlineto{\pgfqpoint{1.684491in}{0.826156in}}%
\pgfpathlineto{\pgfqpoint{1.607540in}{0.843896in}}%
\pgfpathlineto{\pgfqpoint{1.537008in}{0.862089in}}%
\pgfpathlineto{\pgfqpoint{1.473290in}{0.880581in}}%
\pgfpathlineto{\pgfqpoint{1.416596in}{0.899230in}}%
\pgfpathlineto{\pgfqpoint{1.366940in}{0.917906in}}%
\pgfpathlineto{\pgfqpoint{1.324147in}{0.936492in}}%
\pgfpathlineto{\pgfqpoint{1.287854in}{0.954881in}}%
\pgfpathlineto{\pgfqpoint{1.257504in}{0.972979in}}%
\pgfpathlineto{\pgfqpoint{1.232349in}{0.990705in}}%
\pgfpathlineto{\pgfqpoint{1.211888in}{1.007976in}}%
\pgfpathlineto{\pgfqpoint{1.195918in}{1.024725in}}%
\pgfpathlineto{\pgfqpoint{1.183599in}{1.040918in}}%
\pgfpathlineto{\pgfqpoint{1.174269in}{1.056525in}}%
\pgfpathlineto{\pgfqpoint{1.167465in}{1.071523in}}%
\pgfpathlineto{\pgfqpoint{1.162918in}{1.085890in}}%
\pgfpathlineto{\pgfqpoint{1.160558in}{1.099610in}}%
\pgfpathlineto{\pgfqpoint{1.160511in}{1.112671in}}%
\pgfpathlineto{\pgfqpoint{1.163101in}{1.125065in}}%
\pgfpathlineto{\pgfqpoint{1.168812in}{1.136788in}}%
\pgfpathlineto{\pgfqpoint{1.177042in}{1.147820in}}%
\pgfpathlineto{\pgfqpoint{1.187259in}{1.158150in}}%
\pgfpathlineto{\pgfqpoint{1.199383in}{1.167777in}}%
\pgfpathlineto{\pgfqpoint{1.213367in}{1.176699in}}%
\pgfpathlineto{\pgfqpoint{1.229192in}{1.184916in}}%
\pgfpathlineto{\pgfqpoint{1.256423in}{1.195918in}}%
\pgfpathlineto{\pgfqpoint{1.288014in}{1.205335in}}%
\pgfpathlineto{\pgfqpoint{1.324302in}{1.213172in}}%
\pgfpathlineto{\pgfqpoint{1.365500in}{1.219425in}}%
\pgfpathlineto{\pgfqpoint{1.411801in}{1.224067in}}%
\pgfpathlineto{\pgfqpoint{1.463688in}{1.227068in}}%
\pgfpathlineto{\pgfqpoint{1.521690in}{1.228387in}}%
\pgfpathlineto{\pgfqpoint{1.586378in}{1.227975in}}%
\pgfpathlineto{\pgfqpoint{1.684108in}{1.224624in}}%
\pgfpathlineto{\pgfqpoint{1.796422in}{1.217912in}}%
\pgfpathlineto{\pgfqpoint{1.924997in}{1.207632in}}%
\pgfpathlineto{\pgfqpoint{2.071010in}{1.193519in}}%
\pgfpathlineto{\pgfqpoint{2.234113in}{1.175327in}}%
\pgfpathlineto{\pgfqpoint{2.411386in}{1.152892in}}%
\pgfpathlineto{\pgfqpoint{2.549879in}{1.133274in}}%
\pgfpathlineto{\pgfqpoint{2.688458in}{1.111380in}}%
\pgfpathlineto{\pgfqpoint{2.822483in}{1.087434in}}%
\pgfpathlineto{\pgfqpoint{2.906879in}{1.070491in}}%
\pgfpathlineto{\pgfqpoint{2.985510in}{1.052889in}}%
\pgfpathlineto{\pgfqpoint{3.057660in}{1.034779in}}%
\pgfpathlineto{\pgfqpoint{3.123042in}{1.016324in}}%
\pgfpathlineto{\pgfqpoint{3.181509in}{0.997674in}}%
\pgfpathlineto{\pgfqpoint{3.233052in}{0.978966in}}%
\pgfpathlineto{\pgfqpoint{3.277804in}{0.960321in}}%
\pgfpathlineto{\pgfqpoint{3.316037in}{0.941851in}}%
\pgfpathlineto{\pgfqpoint{3.348160in}{0.923652in}}%
\pgfpathlineto{\pgfqpoint{3.374725in}{0.905808in}}%
\pgfpathlineto{\pgfqpoint{3.396422in}{0.888391in}}%
\pgfpathlineto{\pgfqpoint{3.413729in}{0.871468in}}%
\pgfpathlineto{\pgfqpoint{3.426833in}{0.855104in}}%
\pgfpathlineto{\pgfqpoint{3.436530in}{0.839327in}}%
\pgfpathlineto{\pgfqpoint{3.443454in}{0.824160in}}%
\pgfpathlineto{\pgfqpoint{3.448054in}{0.809623in}}%
\pgfpathlineto{\pgfqpoint{3.450597in}{0.795732in}}%
\pgfpathlineto{\pgfqpoint{3.451165in}{0.782500in}}%
\pgfpathlineto{\pgfqpoint{3.449658in}{0.769934in}}%
\pgfpathlineto{\pgfqpoint{3.445790in}{0.758040in}}%
\pgfpathlineto{\pgfqpoint{3.439092in}{0.746818in}}%
\pgfpathlineto{\pgfqpoint{3.429456in}{0.736276in}}%
\pgfpathlineto{\pgfqpoint{3.417812in}{0.726437in}}%
\pgfpathlineto{\pgfqpoint{3.404265in}{0.717301in}}%
\pgfpathlineto{\pgfqpoint{3.388850in}{0.708871in}}%
\pgfpathlineto{\pgfqpoint{3.362237in}{0.697552in}}%
\pgfpathlineto{\pgfqpoint{3.331350in}{0.687824in}}%
\pgfpathlineto{\pgfqpoint{3.295965in}{0.679688in}}%
\pgfpathlineto{\pgfqpoint{3.255723in}{0.673143in}}%
\pgfpathlineto{\pgfqpoint{3.210359in}{0.668197in}}%
\pgfpathlineto{\pgfqpoint{3.159604in}{0.664882in}}%
\pgfpathlineto{\pgfqpoint{3.102836in}{0.663238in}}%
\pgfpathlineto{\pgfqpoint{3.039429in}{0.663314in}}%
\pgfpathlineto{\pgfqpoint{2.968756in}{0.665170in}}%
\pgfpathlineto{\pgfqpoint{2.862135in}{0.670541in}}%
\pgfpathlineto{\pgfqpoint{2.739980in}{0.679406in}}%
\pgfpathlineto{\pgfqpoint{2.600817in}{0.691998in}}%
\pgfpathlineto{\pgfqpoint{2.444114in}{0.708567in}}%
\pgfpathlineto{\pgfqpoint{2.271688in}{0.729361in}}%
\pgfpathlineto{\pgfqpoint{2.135020in}{0.747763in}}%
\pgfpathlineto{\pgfqpoint{1.995823in}{0.768500in}}%
\pgfpathlineto{\pgfqpoint{1.858673in}{0.791431in}}%
\pgfpathlineto{\pgfqpoint{1.770894in}{0.807820in}}%
\pgfpathlineto{\pgfqpoint{1.687656in}{0.824938in}}%
\pgfpathlineto{\pgfqpoint{1.610054in}{0.842650in}}%
\pgfpathlineto{\pgfqpoint{1.538899in}{0.860818in}}%
\pgfpathlineto{\pgfqpoint{1.474723in}{0.879308in}}%
\pgfpathlineto{\pgfqpoint{1.417776in}{0.897986in}}%
\pgfpathlineto{\pgfqpoint{1.368027in}{0.916719in}}%
\pgfpathlineto{\pgfqpoint{1.325201in}{0.935374in}}%
\pgfpathlineto{\pgfqpoint{1.289201in}{0.953802in}}%
\pgfpathlineto{\pgfqpoint{1.259045in}{0.971920in}}%
\pgfpathlineto{\pgfqpoint{1.233686in}{0.989666in}}%
\pgfpathlineto{\pgfqpoint{1.212339in}{1.006985in}}%
\pgfpathlineto{\pgfqpoint{1.194488in}{1.023823in}}%
\pgfpathlineto{\pgfqpoint{1.179880in}{1.040135in}}%
\pgfpathlineto{\pgfqpoint{1.168528in}{1.055877in}}%
\pgfpathlineto{\pgfqpoint{1.160711in}{1.071012in}}%
\pgfpathlineto{\pgfqpoint{1.156955in}{1.085507in}}%
\pgfpathlineto{\pgfqpoint{1.156488in}{1.099330in}}%
\pgfpathlineto{\pgfqpoint{1.158398in}{1.112467in}}%
\pgfpathlineto{\pgfqpoint{1.162486in}{1.124911in}}%
\pgfpathlineto{\pgfqpoint{1.168605in}{1.136656in}}%
\pgfpathlineto{\pgfqpoint{1.176661in}{1.147699in}}%
\pgfpathlineto{\pgfqpoint{1.186611in}{1.158038in}}%
\pgfpathlineto{\pgfqpoint{1.198466in}{1.167674in}}%
\pgfpathlineto{\pgfqpoint{1.212287in}{1.176609in}}%
\pgfpathlineto{\pgfqpoint{1.228156in}{1.184847in}}%
\pgfpathlineto{\pgfqpoint{1.255622in}{1.195892in}}%
\pgfpathlineto{\pgfqpoint{1.287478in}{1.205354in}}%
\pgfpathlineto{\pgfqpoint{1.323853in}{1.213223in}}%
\pgfpathlineto{\pgfqpoint{1.364983in}{1.219483in}}%
\pgfpathlineto{\pgfqpoint{1.411210in}{1.224117in}}%
\pgfpathlineto{\pgfqpoint{1.462978in}{1.227105in}}%
\pgfpathlineto{\pgfqpoint{1.520841in}{1.228422in}}%
\pgfpathlineto{\pgfqpoint{1.585153in}{1.228052in}}%
\pgfpathlineto{\pgfqpoint{1.681893in}{1.224823in}}%
\pgfpathlineto{\pgfqpoint{1.794102in}{1.218191in}}%
\pgfpathlineto{\pgfqpoint{1.923564in}{1.207896in}}%
\pgfpathlineto{\pgfqpoint{2.070478in}{1.193724in}}%
\pgfpathlineto{\pgfqpoint{2.233456in}{1.175509in}}%
\pgfpathlineto{\pgfqpoint{2.409521in}{1.153133in}}%
\pgfpathlineto{\pgfqpoint{2.547462in}{1.133576in}}%
\pgfpathlineto{\pgfqpoint{2.686322in}{1.111672in}}%
\pgfpathlineto{\pgfqpoint{2.820414in}{1.087718in}}%
\pgfpathlineto{\pgfqpoint{2.904820in}{1.070798in}}%
\pgfpathlineto{\pgfqpoint{2.983933in}{1.053256in}}%
\pgfpathlineto{\pgfqpoint{3.056895in}{1.035216in}}%
\pgfpathlineto{\pgfqpoint{3.123074in}{1.016806in}}%
\pgfpathlineto{\pgfqpoint{3.182070in}{0.998161in}}%
\pgfpathlineto{\pgfqpoint{3.233710in}{0.979418in}}%
\pgfpathlineto{\pgfqpoint{3.278033in}{0.960727in}}%
\pgfpathlineto{\pgfqpoint{3.315563in}{0.942228in}}%
\pgfpathlineto{\pgfqpoint{3.347314in}{0.924007in}}%
\pgfpathlineto{\pgfqpoint{3.374125in}{0.906138in}}%
\pgfpathlineto{\pgfqpoint{3.396617in}{0.888688in}}%
\pgfpathlineto{\pgfqpoint{3.415197in}{0.871716in}}%
\pgfpathlineto{\pgfqpoint{3.430057in}{0.855276in}}%
\pgfpathlineto{\pgfqpoint{3.441170in}{0.839411in}}%
\pgfpathlineto{\pgfqpoint{3.448320in}{0.824162in}}%
\pgfpathlineto{\pgfqpoint{3.452163in}{0.809560in}}%
\pgfpathlineto{\pgfqpoint{3.453383in}{0.795624in}}%
\pgfpathlineto{\pgfqpoint{3.452241in}{0.782365in}}%
\pgfpathlineto{\pgfqpoint{3.448929in}{0.769794in}}%
\pgfpathlineto{\pgfqpoint{3.443576in}{0.757915in}}%
\pgfpathlineto{\pgfqpoint{3.436244in}{0.746733in}}%
\pgfpathlineto{\pgfqpoint{3.426931in}{0.736248in}}%
\pgfpathlineto{\pgfqpoint{3.415570in}{0.726461in}}%
\pgfpathlineto{\pgfqpoint{3.402120in}{0.717367in}}%
\pgfpathlineto{\pgfqpoint{3.386725in}{0.708972in}}%
\pgfpathlineto{\pgfqpoint{3.360048in}{0.697693in}}%
\pgfpathlineto{\pgfqpoint{3.329051in}{0.687999in}}%
\pgfpathlineto{\pgfqpoint{3.293604in}{0.679898in}}%
\pgfpathlineto{\pgfqpoint{3.253458in}{0.673399in}}%
\pgfpathlineto{\pgfqpoint{3.208242in}{0.668515in}}%
\pgfpathlineto{\pgfqpoint{3.157466in}{0.665256in}}%
\pgfpathlineto{\pgfqpoint{3.100806in}{0.663639in}}%
\pgfpathlineto{\pgfqpoint{3.037861in}{0.663721in}}%
\pgfpathlineto{\pgfqpoint{2.967534in}{0.665582in}}%
\pgfpathlineto{\pgfqpoint{2.860797in}{0.670972in}}%
\pgfpathlineto{\pgfqpoint{2.737977in}{0.679876in}}%
\pgfpathlineto{\pgfqpoint{2.598407in}{0.692508in}}%
\pgfpathlineto{\pgfqpoint{2.442178in}{0.709096in}}%
\pgfpathlineto{\pgfqpoint{2.270070in}{0.729882in}}%
\pgfpathlineto{\pgfqpoint{2.132376in}{0.748368in}}%
\pgfpathlineto{\pgfqpoint{1.992800in}{0.769201in}}%
\pgfpathlineto{\pgfqpoint{1.856431in}{0.792146in}}%
\pgfpathlineto{\pgfqpoint{1.727558in}{0.816914in}}%
\pgfpathlineto{\pgfqpoint{1.647569in}{0.834271in}}%
\pgfpathlineto{\pgfqpoint{1.573321in}{0.852172in}}%
\pgfpathlineto{\pgfqpoint{1.505499in}{0.870490in}}%
\pgfpathlineto{\pgfqpoint{1.444628in}{0.889088in}}%
\pgfpathlineto{\pgfqpoint{1.391076in}{0.907820in}}%
\pgfpathlineto{\pgfqpoint{1.345000in}{0.926527in}}%
\pgfpathlineto{\pgfqpoint{1.305767in}{0.945075in}}%
\pgfpathlineto{\pgfqpoint{1.272474in}{0.963372in}}%
\pgfpathlineto{\pgfqpoint{1.244390in}{0.981333in}}%
\pgfpathlineto{\pgfqpoint{1.220947in}{0.998886in}}%
\pgfpathlineto{\pgfqpoint{1.201743in}{1.015965in}}%
\pgfpathlineto{\pgfqpoint{1.186537in}{1.032517in}}%
\pgfpathlineto{\pgfqpoint{1.175255in}{1.048495in}}%
\pgfpathlineto{\pgfqpoint{1.167635in}{1.063860in}}%
\pgfpathlineto{\pgfqpoint{1.162965in}{1.078585in}}%
\pgfpathlineto{\pgfqpoint{1.160932in}{1.092654in}}%
\pgfpathlineto{\pgfqpoint{1.161296in}{1.106051in}}%
\pgfpathlineto{\pgfqpoint{1.163887in}{1.118768in}}%
\pgfpathlineto{\pgfqpoint{1.168607in}{1.130796in}}%
\pgfpathlineto{\pgfqpoint{1.175426in}{1.142132in}}%
\pgfpathlineto{\pgfqpoint{1.184381in}{1.152775in}}%
\pgfpathlineto{\pgfqpoint{1.195403in}{1.162724in}}%
\pgfpathlineto{\pgfqpoint{1.208381in}{1.171976in}}%
\pgfpathlineto{\pgfqpoint{1.223256in}{1.180529in}}%
\pgfpathlineto{\pgfqpoint{1.249069in}{1.192041in}}%
\pgfpathlineto{\pgfqpoint{1.279114in}{1.201969in}}%
\pgfpathlineto{\pgfqpoint{1.313558in}{1.210309in}}%
\pgfpathlineto{\pgfqpoint{1.352710in}{1.217057in}}%
\pgfpathlineto{\pgfqpoint{1.397026in}{1.222213in}}%
\pgfpathlineto{\pgfqpoint{1.446807in}{1.225764in}}%
\pgfpathlineto{\pgfqpoint{1.502347in}{1.227661in}}%
\pgfpathlineto{\pgfqpoint{1.564417in}{1.227854in}}%
\pgfpathlineto{\pgfqpoint{1.633733in}{1.226280in}}%
\pgfpathlineto{\pgfqpoint{1.738547in}{1.221311in}}%
\pgfpathlineto{\pgfqpoint{1.858773in}{1.212878in}}%
\pgfpathlineto{\pgfqpoint{1.995592in}{1.200753in}}%
\pgfpathlineto{\pgfqpoint{2.150633in}{1.184661in}}%
\pgfpathlineto{\pgfqpoint{2.325523in}{1.164280in}}%
\pgfpathlineto{\pgfqpoint{2.463429in}{1.146183in}}%
\pgfpathlineto{\pgfqpoint{2.601863in}{1.125786in}}%
\pgfpathlineto{\pgfqpoint{2.736736in}{1.103259in}}%
\pgfpathlineto{\pgfqpoint{2.864541in}{1.078842in}}%
\pgfpathlineto{\pgfqpoint{2.944351in}{1.061662in}}%
\pgfpathlineto{\pgfqpoint{3.018986in}{1.043885in}}%
\pgfpathlineto{\pgfqpoint{3.087841in}{1.025631in}}%
\pgfpathlineto{\pgfqpoint{3.150425in}{1.007034in}}%
\pgfpathlineto{\pgfqpoint{3.206361in}{0.988242in}}%
\pgfpathlineto{\pgfqpoint{3.255389in}{0.969417in}}%
\pgfpathlineto{\pgfqpoint{3.297365in}{0.950732in}}%
\pgfpathlineto{\pgfqpoint{3.332380in}{0.932357in}}%
\pgfpathlineto{\pgfqpoint{3.361600in}{0.914326in}}%
\pgfpathlineto{\pgfqpoint{3.385758in}{0.896695in}}%
\pgfpathlineto{\pgfqpoint{3.405369in}{0.879532in}}%
\pgfpathlineto{\pgfqpoint{3.420889in}{0.862892in}}%
\pgfpathlineto{\pgfqpoint{3.432735in}{0.846819in}}%
\pgfpathlineto{\pgfqpoint{3.441284in}{0.831348in}}%
\pgfpathlineto{\pgfqpoint{3.446847in}{0.816505in}}%
\pgfpathlineto{\pgfqpoint{3.449674in}{0.802312in}}%
\pgfpathlineto{\pgfqpoint{3.449901in}{0.788782in}}%
\pgfpathlineto{\pgfqpoint{3.447646in}{0.775928in}}%
\pgfpathlineto{\pgfqpoint{3.443168in}{0.763757in}}%
\pgfpathlineto{\pgfqpoint{3.436671in}{0.752277in}}%
\pgfpathlineto{\pgfqpoint{3.428300in}{0.741491in}}%
\pgfpathlineto{\pgfqpoint{3.418137in}{0.731404in}}%
\pgfpathlineto{\pgfqpoint{3.406205in}{0.722014in}}%
\pgfpathlineto{\pgfqpoint{3.392464in}{0.713323in}}%
\pgfpathlineto{\pgfqpoint{3.376816in}{0.705325in}}%
\pgfpathlineto{\pgfqpoint{3.359099in}{0.698016in}}%
\pgfpathlineto{\pgfqpoint{3.328318in}{0.688334in}}%
\pgfpathlineto{\pgfqpoint{3.292873in}{0.680223in}}%
\pgfpathlineto{\pgfqpoint{3.252654in}{0.673702in}}%
\pgfpathlineto{\pgfqpoint{3.207352in}{0.668791in}}%
\pgfpathlineto{\pgfqpoint{3.156572in}{0.665521in}}%
\pgfpathlineto{\pgfqpoint{3.099835in}{0.663924in}}%
\pgfpathlineto{\pgfqpoint{3.036576in}{0.664044in}}%
\pgfpathlineto{\pgfqpoint{2.940965in}{0.666957in}}%
\pgfpathlineto{\pgfqpoint{2.831129in}{0.673169in}}%
\pgfpathlineto{\pgfqpoint{2.704905in}{0.682941in}}%
\pgfpathlineto{\pgfqpoint{2.561206in}{0.696523in}}%
\pgfpathlineto{\pgfqpoint{2.400629in}{0.714131in}}%
\pgfpathlineto{\pgfqpoint{2.225436in}{0.735951in}}%
\pgfpathlineto{\pgfqpoint{2.087167in}{0.755156in}}%
\pgfpathlineto{\pgfqpoint{1.948267in}{0.776659in}}%
\pgfpathlineto{\pgfqpoint{1.813698in}{0.800219in}}%
\pgfpathlineto{\pgfqpoint{1.728513in}{0.816918in}}%
\pgfpathlineto{\pgfqpoint{1.648227in}{0.834287in}}%
\pgfpathlineto{\pgfqpoint{1.573772in}{0.852204in}}%
\pgfpathlineto{\pgfqpoint{1.505937in}{0.870535in}}%
\pgfpathlineto{\pgfqpoint{1.445368in}{0.889133in}}%
\pgfpathlineto{\pgfqpoint{1.392425in}{0.907834in}}%
\pgfpathlineto{\pgfqpoint{1.346569in}{0.926504in}}%
\pgfpathlineto{\pgfqpoint{1.307053in}{0.945032in}}%
\pgfpathlineto{\pgfqpoint{1.273248in}{0.963321in}}%
\pgfpathlineto{\pgfqpoint{1.244648in}{0.981282in}}%
\pgfpathlineto{\pgfqpoint{1.220867in}{0.998838in}}%
\pgfpathlineto{\pgfqpoint{1.201637in}{1.015919in}}%
\pgfpathlineto{\pgfqpoint{1.186812in}{1.032468in}}%
\pgfpathlineto{\pgfqpoint{1.175956in}{1.048436in}}%
\pgfpathlineto{\pgfqpoint{1.168349in}{1.063791in}}%
\pgfpathlineto{\pgfqpoint{1.163601in}{1.078509in}}%
\pgfpathlineto{\pgfqpoint{1.161414in}{1.092573in}}%
\pgfpathlineto{\pgfqpoint{1.161577in}{1.105968in}}%
\pgfpathlineto{\pgfqpoint{1.163973in}{1.118683in}}%
\pgfpathlineto{\pgfqpoint{1.168573in}{1.130713in}}%
\pgfpathlineto{\pgfqpoint{1.175437in}{1.142054in}}%
\pgfpathlineto{\pgfqpoint{1.184479in}{1.152703in}}%
\pgfpathlineto{\pgfqpoint{1.195515in}{1.162657in}}%
\pgfpathlineto{\pgfqpoint{1.208464in}{1.171911in}}%
\pgfpathlineto{\pgfqpoint{1.223278in}{1.180464in}}%
\pgfpathlineto{\pgfqpoint{1.248961in}{1.191976in}}%
\pgfpathlineto{\pgfqpoint{1.278872in}{1.201905in}}%
\pgfpathlineto{\pgfqpoint{1.313237in}{1.210250in}}%
\pgfpathlineto{\pgfqpoint{1.352440in}{1.217014in}}%
\pgfpathlineto{\pgfqpoint{1.396842in}{1.222196in}}%
\pgfpathlineto{\pgfqpoint{1.446568in}{1.225756in}}%
\pgfpathlineto{\pgfqpoint{1.502204in}{1.227659in}}%
\pgfpathlineto{\pgfqpoint{1.564372in}{1.227854in}}%
\pgfpathlineto{\pgfqpoint{1.633694in}{1.226283in}}%
\pgfpathlineto{\pgfqpoint{1.738338in}{1.221320in}}%
\pgfpathlineto{\pgfqpoint{1.858314in}{1.212896in}}%
\pgfpathlineto{\pgfqpoint{1.995121in}{1.200781in}}%
\pgfpathlineto{\pgfqpoint{2.149450in}{1.184727in}}%
\pgfpathlineto{\pgfqpoint{2.319863in}{1.164480in}}%
\pgfpathlineto{\pgfqpoint{2.455523in}{1.146494in}}%
\pgfpathlineto{\pgfqpoint{2.594346in}{1.126161in}}%
\pgfpathlineto{\pgfqpoint{2.731940in}{1.103603in}}%
\pgfpathlineto{\pgfqpoint{2.863284in}{1.079065in}}%
\pgfpathlineto{\pgfqpoint{2.945067in}{1.061814in}}%
\pgfpathlineto{\pgfqpoint{3.020995in}{1.044004in}}%
\pgfpathlineto{\pgfqpoint{3.090328in}{1.025767in}}%
\pgfpathlineto{\pgfqpoint{3.152605in}{1.007239in}}%
\pgfpathlineto{\pgfqpoint{3.207645in}{0.988553in}}%
\pgfpathlineto{\pgfqpoint{3.255547in}{0.969848in}}%
\pgfpathlineto{\pgfqpoint{3.296298in}{0.951281in}}%
\pgfpathlineto{\pgfqpoint{3.330657in}{0.932958in}}%
\pgfpathlineto{\pgfqpoint{3.359742in}{0.914949in}}%
\pgfpathlineto{\pgfqpoint{3.384407in}{0.897316in}}%
\pgfpathlineto{\pgfqpoint{3.405247in}{0.880119in}}%
\pgfpathlineto{\pgfqpoint{3.422590in}{0.863411in}}%
\pgfpathlineto{\pgfqpoint{3.436506in}{0.847241in}}%
\pgfpathlineto{\pgfqpoint{3.446799in}{0.831653in}}%
\pgfpathlineto{\pgfqpoint{3.453011in}{0.816687in}}%
\pgfpathlineto{\pgfqpoint{3.455378in}{0.802379in}}%
\pgfpathlineto{\pgfqpoint{3.455206in}{0.788748in}}%
\pgfpathlineto{\pgfqpoint{3.452743in}{0.775805in}}%
\pgfpathlineto{\pgfqpoint{3.448166in}{0.763555in}}%
\pgfpathlineto{\pgfqpoint{3.441596in}{0.752005in}}%
\pgfpathlineto{\pgfqpoint{3.433097in}{0.741157in}}%
\pgfpathlineto{\pgfqpoint{3.422676in}{0.731010in}}%
\pgfpathlineto{\pgfqpoint{3.410287in}{0.721565in}}%
\pgfpathlineto{\pgfqpoint{3.395842in}{0.712816in}}%
\pgfpathlineto{\pgfqpoint{3.379426in}{0.704767in}}%
\pgfpathlineto{\pgfqpoint{3.351190in}{0.694007in}}%
\pgfpathlineto{\pgfqpoint{3.318571in}{0.684835in}}%
\pgfpathlineto{\pgfqpoint{3.281416in}{0.677260in}}%
\pgfpathlineto{\pgfqpoint{3.239460in}{0.671296in}}%
\pgfpathlineto{\pgfqpoint{3.192327in}{0.666959in}}%
\pgfpathlineto{\pgfqpoint{3.139529in}{0.664267in}}%
\pgfpathlineto{\pgfqpoint{3.080556in}{0.663237in}}%
\pgfpathlineto{\pgfqpoint{3.015283in}{0.663903in}}%
\pgfpathlineto{\pgfqpoint{2.916374in}{0.667604in}}%
\pgfpathlineto{\pgfqpoint{2.801682in}{0.674740in}}%
\pgfpathlineto{\pgfqpoint{2.669963in}{0.685535in}}%
\pgfpathlineto{\pgfqpoint{2.521212in}{0.700196in}}%
\pgfpathlineto{\pgfqpoint{2.356662in}{0.718908in}}%
\pgfpathlineto{\pgfqpoint{2.178786in}{0.741840in}}%
\pgfpathlineto{\pgfqpoint{2.039642in}{0.761879in}}%
\pgfpathlineto{\pgfqpoint{1.901236in}{0.784177in}}%
\pgfpathlineto{\pgfqpoint{1.768724in}{0.808430in}}%
\pgfpathlineto{\pgfqpoint{1.685759in}{0.825515in}}%
\pgfpathlineto{\pgfqpoint{1.608289in}{0.843198in}}%
\pgfpathlineto{\pgfqpoint{1.537117in}{0.861357in}}%
\pgfpathlineto{\pgfqpoint{1.472857in}{0.879856in}}%
\pgfpathlineto{\pgfqpoint{1.415928in}{0.898553in}}%
\pgfpathlineto{\pgfqpoint{1.366546in}{0.917293in}}%
\pgfpathlineto{\pgfqpoint{1.324282in}{0.935928in}}%
\pgfpathlineto{\pgfqpoint{1.288196in}{0.954359in}}%
\pgfpathlineto{\pgfqpoint{1.257492in}{0.972500in}}%
\pgfpathlineto{\pgfqpoint{1.231565in}{0.990273in}}%
\pgfpathlineto{\pgfqpoint{1.209990in}{1.007608in}}%
\pgfpathlineto{\pgfqpoint{1.192527in}{1.024445in}}%
\pgfpathlineto{\pgfqpoint{1.179123in}{1.040732in}}%
\pgfpathlineto{\pgfqpoint{1.169835in}{1.056423in}}%
\pgfpathlineto{\pgfqpoint{1.163885in}{1.071483in}}%
\pgfpathlineto{\pgfqpoint{1.160711in}{1.085891in}}%
\pgfpathlineto{\pgfqpoint{1.160022in}{1.099632in}}%
\pgfpathlineto{\pgfqpoint{1.161598in}{1.112696in}}%
\pgfpathlineto{\pgfqpoint{1.165298in}{1.125072in}}%
\pgfpathlineto{\pgfqpoint{1.171057in}{1.136757in}}%
\pgfpathlineto{\pgfqpoint{1.178883in}{1.147747in}}%
\pgfpathlineto{\pgfqpoint{1.188858in}{1.158044in}}%
\pgfpathlineto{\pgfqpoint{1.200906in}{1.167646in}}%
\pgfpathlineto{\pgfqpoint{1.214894in}{1.176549in}}%
\pgfpathlineto{\pgfqpoint{1.230778in}{1.184752in}}%
\pgfpathlineto{\pgfqpoint{1.258131in}{1.195738in}}%
\pgfpathlineto{\pgfqpoint{1.289777in}{1.205137in}}%
\pgfpathlineto{\pgfqpoint{1.325903in}{1.212942in}}%
\pgfpathlineto{\pgfqpoint{1.366834in}{1.219150in}}%
\pgfpathlineto{\pgfqpoint{1.413026in}{1.223755in}}%
\pgfpathlineto{\pgfqpoint{1.464776in}{1.226744in}}%
\pgfpathlineto{\pgfqpoint{1.522439in}{1.228065in}}%
\pgfpathlineto{\pgfqpoint{1.586871in}{1.227662in}}%
\pgfpathlineto{\pgfqpoint{1.684609in}{1.224325in}}%
\pgfpathlineto{\pgfqpoint{1.797236in}{1.217620in}}%
\pgfpathlineto{\pgfqpoint{1.925945in}{1.207336in}}%
\pgfpathlineto{\pgfqpoint{2.071601in}{1.193235in}}%
\pgfpathlineto{\pgfqpoint{2.235341in}{1.175034in}}%
\pgfpathlineto{\pgfqpoint{2.415067in}{1.152500in}}%
\pgfpathlineto{\pgfqpoint{2.553735in}{1.132829in}}%
\pgfpathlineto{\pgfqpoint{2.690798in}{1.110945in}}%
\pgfpathlineto{\pgfqpoint{2.822364in}{1.087072in}}%
\pgfpathlineto{\pgfqpoint{2.905323in}{1.070197in}}%
\pgfpathlineto{\pgfqpoint{2.983436in}{1.052675in}}%
\pgfpathlineto{\pgfqpoint{3.055920in}{1.034629in}}%
\pgfpathlineto{\pgfqpoint{3.122086in}{1.016195in}}%
\pgfpathlineto{\pgfqpoint{3.181347in}{0.997522in}}%
\pgfpathlineto{\pgfqpoint{3.233212in}{0.978776in}}%
\pgfpathlineto{\pgfqpoint{3.277582in}{0.960118in}}%
\pgfpathlineto{\pgfqpoint{3.315463in}{0.941637in}}%
\pgfpathlineto{\pgfqpoint{3.347494in}{0.923430in}}%
\pgfpathlineto{\pgfqpoint{3.374214in}{0.905583in}}%
\pgfpathlineto{\pgfqpoint{3.396115in}{0.888172in}}%
\pgfpathlineto{\pgfqpoint{3.413642in}{0.871258in}}%
\pgfpathlineto{\pgfqpoint{3.427273in}{0.854892in}}%
\pgfpathlineto{\pgfqpoint{3.437440in}{0.839110in}}%
\pgfpathlineto{\pgfqpoint{3.444489in}{0.823944in}}%
\pgfpathlineto{\pgfqpoint{3.448693in}{0.809417in}}%
\pgfpathlineto{\pgfqpoint{3.450255in}{0.795548in}}%
\pgfpathlineto{\pgfqpoint{3.449268in}{0.782349in}}%
\pgfpathlineto{\pgfqpoint{3.445914in}{0.769829in}}%
\pgfpathlineto{\pgfqpoint{3.440419in}{0.757997in}}%
\pgfpathlineto{\pgfqpoint{3.432954in}{0.746857in}}%
\pgfpathlineto{\pgfqpoint{3.423631in}{0.736414in}}%
\pgfpathlineto{\pgfqpoint{3.412510in}{0.726669in}}%
\pgfpathlineto{\pgfqpoint{3.399592in}{0.717623in}}%
\pgfpathlineto{\pgfqpoint{3.384825in}{0.709274in}}%
\pgfpathlineto{\pgfqpoint{3.368097in}{0.701619in}}%
\pgfpathlineto{\pgfqpoint{3.338966in}{0.691424in}}%
\pgfpathlineto{\pgfqpoint{3.305072in}{0.682787in}}%
\pgfpathlineto{\pgfqpoint{3.266488in}{0.675734in}}%
\pgfpathlineto{\pgfqpoint{3.222932in}{0.670284in}}%
\pgfpathlineto{\pgfqpoint{3.174039in}{0.666463in}}%
\pgfpathlineto{\pgfqpoint{3.119357in}{0.664303in}}%
\pgfpathlineto{\pgfqpoint{3.058353in}{0.663845in}}%
\pgfpathlineto{\pgfqpoint{2.990405in}{0.665135in}}%
\pgfpathlineto{\pgfqpoint{2.887897in}{0.669671in}}%
\pgfpathlineto{\pgfqpoint{2.770132in}{0.677628in}}%
\pgfpathlineto{\pgfqpoint{2.635256in}{0.689271in}}%
\pgfpathlineto{\pgfqpoint{2.482987in}{0.704836in}}%
\pgfpathlineto{\pgfqpoint{2.314731in}{0.724524in}}%
\pgfpathlineto{\pgfqpoint{2.133643in}{0.748505in}}%
\pgfpathlineto{\pgfqpoint{1.994467in}{0.769251in}}%
\pgfpathlineto{\pgfqpoint{1.858017in}{0.792146in}}%
\pgfpathlineto{\pgfqpoint{1.728716in}{0.816907in}}%
\pgfpathlineto{\pgfqpoint{1.648383in}{0.834279in}}%
\pgfpathlineto{\pgfqpoint{1.573890in}{0.852201in}}%
\pgfpathlineto{\pgfqpoint{1.506079in}{0.870536in}}%
\pgfpathlineto{\pgfqpoint{1.445661in}{0.889129in}}%
\pgfpathlineto{\pgfqpoint{1.392794in}{0.907822in}}%
\pgfpathlineto{\pgfqpoint{1.346841in}{0.926489in}}%
\pgfpathlineto{\pgfqpoint{1.307170in}{0.945018in}}%
\pgfpathlineto{\pgfqpoint{1.273241in}{0.963308in}}%
\pgfpathlineto{\pgfqpoint{1.244603in}{0.981270in}}%
\pgfpathlineto{\pgfqpoint{1.220897in}{0.998825in}}%
\pgfpathlineto{\pgfqpoint{1.201856in}{1.015902in}}%
\pgfpathlineto{\pgfqpoint{1.187250in}{1.032445in}}%
\pgfpathlineto{\pgfqpoint{1.176392in}{1.048405in}}%
\pgfpathlineto{\pgfqpoint{1.168739in}{1.063755in}}%
\pgfpathlineto{\pgfqpoint{1.163909in}{1.078470in}}%
\pgfpathlineto{\pgfqpoint{1.161622in}{1.092531in}}%
\pgfpathlineto{\pgfqpoint{1.161693in}{1.105924in}}%
\pgfpathlineto{\pgfqpoint{1.164036in}{1.118639in}}%
\pgfpathlineto{\pgfqpoint{1.168661in}{1.130669in}}%
\pgfpathlineto{\pgfqpoint{1.175613in}{1.142012in}}%
\pgfpathlineto{\pgfqpoint{1.184667in}{1.152662in}}%
\pgfpathlineto{\pgfqpoint{1.195691in}{1.162615in}}%
\pgfpathlineto{\pgfqpoint{1.208609in}{1.171869in}}%
\pgfpathlineto{\pgfqpoint{1.223374in}{1.180420in}}%
\pgfpathlineto{\pgfqpoint{1.248971in}{1.191930in}}%
\pgfpathlineto{\pgfqpoint{1.278808in}{1.201859in}}%
\pgfpathlineto{\pgfqpoint{1.313150in}{1.210208in}}%
\pgfpathlineto{\pgfqpoint{1.352428in}{1.216984in}}%
\pgfpathlineto{\pgfqpoint{1.396796in}{1.222171in}}%
\pgfpathlineto{\pgfqpoint{1.446536in}{1.225736in}}%
\pgfpathlineto{\pgfqpoint{1.502203in}{1.227640in}}%
\pgfpathlineto{\pgfqpoint{1.564377in}{1.227837in}}%
\pgfpathlineto{\pgfqpoint{1.633665in}{1.226267in}}%
\pgfpathlineto{\pgfqpoint{1.738214in}{1.221306in}}%
\pgfpathlineto{\pgfqpoint{1.858112in}{1.212886in}}%
\pgfpathlineto{\pgfqpoint{1.994898in}{1.200782in}}%
\pgfpathlineto{\pgfqpoint{2.149206in}{1.184724in}}%
\pgfpathlineto{\pgfqpoint{2.319571in}{1.164492in}}%
\pgfpathlineto{\pgfqpoint{2.455195in}{1.146526in}}%
\pgfpathlineto{\pgfqpoint{2.594132in}{1.126189in}}%
\pgfpathlineto{\pgfqpoint{2.731705in}{1.103636in}}%
\pgfpathlineto{\pgfqpoint{2.862928in}{1.079140in}}%
\pgfpathlineto{\pgfqpoint{2.944751in}{1.061903in}}%
\pgfpathlineto{\pgfqpoint{3.020765in}{1.044086in}}%
\pgfpathlineto{\pgfqpoint{3.090017in}{1.025832in}}%
\pgfpathlineto{\pgfqpoint{3.151771in}{1.007299in}}%
\pgfpathlineto{\pgfqpoint{3.206343in}{0.988628in}}%
\pgfpathlineto{\pgfqpoint{3.254213in}{0.969945in}}%
\pgfpathlineto{\pgfqpoint{3.295823in}{0.951368in}}%
\pgfpathlineto{\pgfqpoint{3.331570in}{0.933000in}}%
\pgfpathlineto{\pgfqpoint{3.361812in}{0.914936in}}%
\pgfpathlineto{\pgfqpoint{3.386862in}{0.897260in}}%
\pgfpathlineto{\pgfqpoint{3.406995in}{0.880044in}}%
\pgfpathlineto{\pgfqpoint{3.422489in}{0.863350in}}%
\pgfpathlineto{\pgfqpoint{3.434053in}{0.847233in}}%
\pgfpathlineto{\pgfqpoint{3.442323in}{0.831722in}}%
\pgfpathlineto{\pgfqpoint{3.447765in}{0.816839in}}%
\pgfpathlineto{\pgfqpoint{3.450726in}{0.802606in}}%
\pgfpathlineto{\pgfqpoint{3.451423in}{0.789036in}}%
\pgfpathlineto{\pgfqpoint{3.449951in}{0.776140in}}%
\pgfpathlineto{\pgfqpoint{3.446282in}{0.763926in}}%
\pgfpathlineto{\pgfqpoint{3.440261in}{0.752395in}}%
\pgfpathlineto{\pgfqpoint{3.431754in}{0.741551in}}%
\pgfpathlineto{\pgfqpoint{3.421182in}{0.731402in}}%
\pgfpathlineto{\pgfqpoint{3.408674in}{0.721954in}}%
\pgfpathlineto{\pgfqpoint{3.394286in}{0.713207in}}%
\pgfpathlineto{\pgfqpoint{3.378042in}{0.705164in}}%
\pgfpathlineto{\pgfqpoint{3.350175in}{0.694420in}}%
\pgfpathlineto{\pgfqpoint{3.317975in}{0.685262in}}%
\pgfpathlineto{\pgfqpoint{3.281154in}{0.677690in}}%
\pgfpathlineto{\pgfqpoint{3.239299in}{0.671701in}}%
\pgfpathlineto{\pgfqpoint{3.192276in}{0.667315in}}%
\pgfpathlineto{\pgfqpoint{3.139647in}{0.664570in}}%
\pgfpathlineto{\pgfqpoint{3.080796in}{0.663508in}}%
\pgfpathlineto{\pgfqpoint{3.015103in}{0.664182in}}%
\pgfpathlineto{\pgfqpoint{2.915785in}{0.667894in}}%
\pgfpathlineto{\pgfqpoint{2.801696in}{0.674992in}}%
\pgfpathlineto{\pgfqpoint{2.671327in}{0.685692in}}%
\pgfpathlineto{\pgfqpoint{2.523478in}{0.700236in}}%
\pgfpathlineto{\pgfqpoint{2.358666in}{0.718892in}}%
\pgfpathlineto{\pgfqpoint{2.180288in}{0.741813in}}%
\pgfpathlineto{\pgfqpoint{2.041643in}{0.761767in}}%
\pgfpathlineto{\pgfqpoint{1.903429in}{0.783972in}}%
\pgfpathlineto{\pgfqpoint{1.770621in}{0.808216in}}%
\pgfpathlineto{\pgfqpoint{1.687477in}{0.825312in}}%
\pgfpathlineto{\pgfqpoint{1.609962in}{0.843002in}}%
\pgfpathlineto{\pgfqpoint{1.538890in}{0.861152in}}%
\pgfpathlineto{\pgfqpoint{1.474790in}{0.879627in}}%
\pgfpathlineto{\pgfqpoint{1.417912in}{0.898293in}}%
\pgfpathlineto{\pgfqpoint{1.368228in}{0.917014in}}%
\pgfpathlineto{\pgfqpoint{1.325661in}{0.935638in}}%
\pgfpathlineto{\pgfqpoint{1.289805in}{0.954037in}}%
\pgfpathlineto{\pgfqpoint{1.259496in}{0.972137in}}%
\pgfpathlineto{\pgfqpoint{1.233805in}{0.989876in}}%
\pgfpathlineto{\pgfqpoint{1.212067in}{1.007194in}}%
\pgfpathlineto{\pgfqpoint{1.193877in}{1.024035in}}%
\pgfpathlineto{\pgfqpoint{1.179097in}{1.040349in}}%
\pgfpathlineto{\pgfqpoint{1.167851in}{1.056092in}}%
\pgfpathlineto{\pgfqpoint{1.160524in}{1.071222in}}%
\pgfpathlineto{\pgfqpoint{1.157339in}{1.085703in}}%
\pgfpathlineto{\pgfqpoint{1.156900in}{1.099510in}}%
\pgfpathlineto{\pgfqpoint{1.158807in}{1.112632in}}%
\pgfpathlineto{\pgfqpoint{1.162868in}{1.125060in}}%
\pgfpathlineto{\pgfqpoint{1.168947in}{1.136791in}}%
\pgfpathlineto{\pgfqpoint{1.176965in}{1.147820in}}%
\pgfpathlineto{\pgfqpoint{1.186900in}{1.158146in}}%
\pgfpathlineto{\pgfqpoint{1.198784in}{1.167771in}}%
\pgfpathlineto{\pgfqpoint{1.212704in}{1.176698in}}%
\pgfpathlineto{\pgfqpoint{1.228631in}{1.184927in}}%
\pgfpathlineto{\pgfqpoint{1.256135in}{1.195955in}}%
\pgfpathlineto{\pgfqpoint{1.288007in}{1.205399in}}%
\pgfpathlineto{\pgfqpoint{1.324381in}{1.213247in}}%
\pgfpathlineto{\pgfqpoint{1.365504in}{1.219486in}}%
\pgfpathlineto{\pgfqpoint{1.411731in}{1.224101in}}%
\pgfpathlineto{\pgfqpoint{1.463530in}{1.227075in}}%
\pgfpathlineto{\pgfqpoint{1.521457in}{1.228388in}}%
\pgfpathlineto{\pgfqpoint{1.585602in}{1.228015in}}%
\pgfpathlineto{\pgfqpoint{1.682648in}{1.224736in}}%
\pgfpathlineto{\pgfqpoint{1.795230in}{1.218051in}}%
\pgfpathlineto{\pgfqpoint{1.924764in}{1.207733in}}%
\pgfpathlineto{\pgfqpoint{2.071410in}{1.193574in}}%
\pgfpathlineto{\pgfqpoint{2.234079in}{1.175386in}}%
\pgfpathlineto{\pgfqpoint{2.410428in}{1.152999in}}%
\pgfpathlineto{\pgfqpoint{2.549100in}{1.133371in}}%
\pgfpathlineto{\pgfqpoint{2.688027in}{1.111436in}}%
\pgfpathlineto{\pgfqpoint{2.821876in}{1.087490in}}%
\pgfpathlineto{\pgfqpoint{2.906096in}{1.070576in}}%
\pgfpathlineto{\pgfqpoint{2.985055in}{1.053032in}}%
\pgfpathlineto{\pgfqpoint{3.057898in}{1.034982in}}%
\pgfpathlineto{\pgfqpoint{3.123963in}{1.016556in}}%
\pgfpathlineto{\pgfqpoint{3.182786in}{0.997895in}}%
\pgfpathlineto{\pgfqpoint{3.234091in}{0.979149in}}%
\pgfpathlineto{\pgfqpoint{3.278075in}{0.960476in}}%
\pgfpathlineto{\pgfqpoint{3.315651in}{0.941985in}}%
\pgfpathlineto{\pgfqpoint{3.347661in}{0.923762in}}%
\pgfpathlineto{\pgfqpoint{3.374759in}{0.905888in}}%
\pgfpathlineto{\pgfqpoint{3.397418in}{0.888435in}}%
\pgfpathlineto{\pgfqpoint{3.415922in}{0.871465in}}%
\pgfpathlineto{\pgfqpoint{3.430373in}{0.855033in}}%
\pgfpathlineto{\pgfqpoint{3.440691in}{0.839186in}}%
\pgfpathlineto{\pgfqpoint{3.447404in}{0.823962in}}%
\pgfpathlineto{\pgfqpoint{3.451252in}{0.809386in}}%
\pgfpathlineto{\pgfqpoint{3.452548in}{0.795473in}}%
\pgfpathlineto{\pgfqpoint{3.451530in}{0.782235in}}%
\pgfpathlineto{\pgfqpoint{3.448360in}{0.769682in}}%
\pgfpathlineto{\pgfqpoint{3.443123in}{0.757821in}}%
\pgfpathlineto{\pgfqpoint{3.435830in}{0.746653in}}%
\pgfpathlineto{\pgfqpoint{3.426415in}{0.736180in}}%
\pgfpathlineto{\pgfqpoint{3.414872in}{0.726400in}}%
\pgfpathlineto{\pgfqpoint{3.401373in}{0.717318in}}%
\pgfpathlineto{\pgfqpoint{3.385969in}{0.708936in}}%
\pgfpathlineto{\pgfqpoint{3.359334in}{0.697681in}}%
\pgfpathlineto{\pgfqpoint{3.328425in}{0.688012in}}%
\pgfpathlineto{\pgfqpoint{3.293076in}{0.679935in}}%
\pgfpathlineto{\pgfqpoint{3.252988in}{0.673456in}}%
\pgfpathlineto{\pgfqpoint{3.207728in}{0.668578in}}%
\pgfpathlineto{\pgfqpoint{3.156918in}{0.665312in}}%
\pgfpathlineto{\pgfqpoint{3.100322in}{0.663703in}}%
\pgfpathlineto{\pgfqpoint{3.037099in}{0.663809in}}%
\pgfpathlineto{\pgfqpoint{2.966468in}{0.665694in}}%
\pgfpathlineto{\pgfqpoint{2.859616in}{0.671104in}}%
\pgfpathlineto{\pgfqpoint{2.737090in}{0.680011in}}%
\pgfpathlineto{\pgfqpoint{2.597907in}{0.692641in}}%
\pgfpathlineto{\pgfqpoint{2.441333in}{0.709254in}}%
\pgfpathlineto{\pgfqpoint{2.266534in}{0.730150in}}%
\pgfpathlineto{\pgfqpoint{2.128702in}{0.748644in}}%
\pgfpathlineto{\pgfqpoint{1.990303in}{0.769437in}}%
\pgfpathlineto{\pgfqpoint{1.855485in}{0.792343in}}%
\pgfpathlineto{\pgfqpoint{1.727901in}{0.817098in}}%
\pgfpathlineto{\pgfqpoint{1.648447in}{0.834467in}}%
\pgfpathlineto{\pgfqpoint{1.574437in}{0.852393in}}%
\pgfpathlineto{\pgfqpoint{1.506589in}{0.870743in}}%
\pgfpathlineto{\pgfqpoint{1.445522in}{0.889369in}}%
\pgfpathlineto{\pgfqpoint{1.391758in}{0.908111in}}%
\pgfpathlineto{\pgfqpoint{1.345585in}{0.926800in}}%
\pgfpathlineto{\pgfqpoint{1.306120in}{0.945336in}}%
\pgfpathlineto{\pgfqpoint{1.272648in}{0.963622in}}%
\pgfpathlineto{\pgfqpoint{1.244617in}{0.981568in}}%
\pgfpathlineto{\pgfqpoint{1.221526in}{0.999097in}}%
\pgfpathlineto{\pgfqpoint{1.202924in}{1.016143in}}%
\pgfpathlineto{\pgfqpoint{1.188347in}{1.032654in}}%
\pgfpathlineto{\pgfqpoint{1.177334in}{1.048588in}}%
\pgfpathlineto{\pgfqpoint{1.169520in}{1.063914in}}%
\pgfpathlineto{\pgfqpoint{1.164610in}{1.078606in}}%
\pgfpathlineto{\pgfqpoint{1.162387in}{1.092645in}}%
\pgfpathlineto{\pgfqpoint{1.162722in}{1.106016in}}%
\pgfpathlineto{\pgfqpoint{1.165491in}{1.118710in}}%
\pgfpathlineto{\pgfqpoint{1.170460in}{1.130718in}}%
\pgfpathlineto{\pgfqpoint{1.177444in}{1.142036in}}%
\pgfpathlineto{\pgfqpoint{1.186314in}{1.152657in}}%
\pgfpathlineto{\pgfqpoint{1.196994in}{1.162580in}}%
\pgfpathlineto{\pgfqpoint{1.209465in}{1.171804in}}%
\pgfpathlineto{\pgfqpoint{1.223763in}{1.180331in}}%
\pgfpathlineto{\pgfqpoint{1.239979in}{1.188162in}}%
\pgfpathlineto{\pgfqpoint{1.268233in}{1.198617in}}%
\pgfpathlineto{\pgfqpoint{1.301350in}{1.207521in}}%
\pgfpathlineto{\pgfqpoint{1.339118in}{1.214844in}}%
\pgfpathlineto{\pgfqpoint{1.381804in}{1.220568in}}%
\pgfpathlineto{\pgfqpoint{1.429761in}{1.224669in}}%
\pgfpathlineto{\pgfqpoint{1.483425in}{1.227114in}}%
\pgfpathlineto{\pgfqpoint{1.543316in}{1.227865in}}%
\pgfpathlineto{\pgfqpoint{1.610037in}{1.226875in}}%
\pgfpathlineto{\pgfqpoint{1.710764in}{1.222755in}}%
\pgfpathlineto{\pgfqpoint{1.826457in}{1.215249in}}%
\pgfpathlineto{\pgfqpoint{1.959083in}{1.204090in}}%
\pgfpathlineto{\pgfqpoint{2.109158in}{1.189039in}}%
\pgfpathlineto{\pgfqpoint{2.275520in}{1.169886in}}%
\pgfpathlineto{\pgfqpoint{2.455320in}{1.146457in}}%
\pgfpathlineto{\pgfqpoint{2.594408in}{1.126104in}}%
\pgfpathlineto{\pgfqpoint{2.731603in}{1.103562in}}%
\pgfpathlineto{\pgfqpoint{2.862428in}{1.079094in}}%
\pgfpathlineto{\pgfqpoint{2.944144in}{1.061878in}}%
\pgfpathlineto{\pgfqpoint{3.020246in}{1.044079in}}%
\pgfpathlineto{\pgfqpoint{3.089814in}{1.025832in}}%
\pgfpathlineto{\pgfqpoint{3.152048in}{1.007292in}}%
\pgfpathlineto{\pgfqpoint{3.206743in}{0.988614in}}%
\pgfpathlineto{\pgfqpoint{3.254467in}{0.969930in}}%
\pgfpathlineto{\pgfqpoint{3.295786in}{0.951355in}}%
\pgfpathlineto{\pgfqpoint{3.331205in}{0.932993in}}%
\pgfpathlineto{\pgfqpoint{3.361157in}{0.914938in}}%
\pgfpathlineto{\pgfqpoint{3.386013in}{0.897274in}}%
\pgfpathlineto{\pgfqpoint{3.406074in}{0.880071in}}%
\pgfpathlineto{\pgfqpoint{3.421636in}{0.863392in}}%
\pgfpathlineto{\pgfqpoint{3.433380in}{0.847285in}}%
\pgfpathlineto{\pgfqpoint{3.441845in}{0.831781in}}%
\pgfpathlineto{\pgfqpoint{3.447426in}{0.816905in}}%
\pgfpathlineto{\pgfqpoint{3.450415in}{0.802677in}}%
\pgfpathlineto{\pgfqpoint{3.451000in}{0.789113in}}%
\pgfpathlineto{\pgfqpoint{3.449266in}{0.776225in}}%
\pgfpathlineto{\pgfqpoint{3.445194in}{0.764019in}}%
\pgfpathlineto{\pgfqpoint{3.438758in}{0.752499in}}%
\pgfpathlineto{\pgfqpoint{3.430213in}{0.741672in}}%
\pgfpathlineto{\pgfqpoint{3.419692in}{0.731542in}}%
\pgfpathlineto{\pgfqpoint{3.407276in}{0.722111in}}%
\pgfpathlineto{\pgfqpoint{3.393015in}{0.713382in}}%
\pgfpathlineto{\pgfqpoint{3.368186in}{0.701605in}}%
\pgfpathlineto{\pgfqpoint{3.339132in}{0.691408in}}%
\pgfpathlineto{\pgfqpoint{3.305581in}{0.682788in}}%
\pgfpathlineto{\pgfqpoint{3.267109in}{0.675737in}}%
\pgfpathlineto{\pgfqpoint{3.223623in}{0.670273in}}%
\pgfpathlineto{\pgfqpoint{3.174820in}{0.666429in}}%
\pgfpathlineto{\pgfqpoint{3.120181in}{0.664239in}}%
\pgfpathlineto{\pgfqpoint{3.059153in}{0.663749in}}%
\pgfpathlineto{\pgfqpoint{2.991138in}{0.665015in}}%
\pgfpathlineto{\pgfqpoint{2.888485in}{0.669551in}}%
\pgfpathlineto{\pgfqpoint{2.770669in}{0.677515in}}%
\pgfpathlineto{\pgfqpoint{2.636105in}{0.689130in}}%
\pgfpathlineto{\pgfqpoint{2.484001in}{0.704669in}}%
\pgfpathlineto{\pgfqpoint{2.315558in}{0.724352in}}%
\pgfpathlineto{\pgfqpoint{2.180803in}{0.741909in}}%
\pgfpathlineto{\pgfqpoint{2.042161in}{0.761846in}}%
\pgfpathlineto{\pgfqpoint{1.904140in}{0.784028in}}%
\pgfpathlineto{\pgfqpoint{1.771400in}{0.808215in}}%
\pgfpathlineto{\pgfqpoint{1.688157in}{0.825288in}}%
\pgfpathlineto{\pgfqpoint{1.610974in}{0.842978in}}%
\pgfpathlineto{\pgfqpoint{1.540925in}{0.861144in}}%
\pgfpathlineto{\pgfqpoint{1.477817in}{0.879626in}}%
\pgfpathlineto{\pgfqpoint{1.421413in}{0.898277in}}%
\pgfpathlineto{\pgfqpoint{1.371464in}{0.916962in}}%
\pgfpathlineto{\pgfqpoint{1.327706in}{0.935560in}}%
\pgfpathlineto{\pgfqpoint{1.289868in}{0.953966in}}%
\pgfpathlineto{\pgfqpoint{1.257664in}{0.972084in}}%
\pgfpathlineto{\pgfqpoint{1.230796in}{0.989834in}}%
\pgfpathlineto{\pgfqpoint{1.208956in}{1.007150in}}%
\pgfpathlineto{\pgfqpoint{1.191823in}{1.023978in}}%
\pgfpathlineto{\pgfqpoint{1.179063in}{1.040276in}}%
\pgfpathlineto{\pgfqpoint{1.170243in}{1.055982in}}%
\pgfpathlineto{\pgfqpoint{1.164693in}{1.071054in}}%
\pgfpathlineto{\pgfqpoint{1.161815in}{1.085474in}}%
\pgfpathlineto{\pgfqpoint{1.161153in}{1.099230in}}%
\pgfpathlineto{\pgfqpoint{1.162387in}{1.112312in}}%
\pgfpathlineto{\pgfqpoint{1.165342in}{1.124711in}}%
\pgfpathlineto{\pgfqpoint{1.169980in}{1.136425in}}%
\pgfpathlineto{\pgfqpoint{1.176401in}{1.147452in}}%
\pgfpathlineto{\pgfqpoint{1.184849in}{1.157796in}}%
\pgfpathlineto{\pgfqpoint{1.195704in}{1.167462in}}%
\pgfpathlineto{\pgfqpoint{1.209409in}{1.176456in}}%
\pgfpathlineto{\pgfqpoint{1.225325in}{1.184756in}}%
\pgfpathlineto{\pgfqpoint{1.252780in}{1.195882in}}%
\pgfpathlineto{\pgfqpoint{1.284581in}{1.205416in}}%
\pgfpathlineto{\pgfqpoint{1.320878in}{1.213349in}}%
\pgfpathlineto{\pgfqpoint{1.361931in}{1.219672in}}%
\pgfpathlineto{\pgfqpoint{1.408107in}{1.224371in}}%
\pgfpathlineto{\pgfqpoint{1.459879in}{1.227434in}}%
\pgfpathlineto{\pgfqpoint{1.517450in}{1.228840in}}%
\pgfpathlineto{\pgfqpoint{1.581474in}{1.228531in}}%
\pgfpathlineto{\pgfqpoint{1.678721in}{1.225312in}}%
\pgfpathlineto{\pgfqpoint{1.791257in}{1.218705in}}%
\pgfpathlineto{\pgfqpoint{1.920230in}{1.208501in}}%
\pgfpathlineto{\pgfqpoint{2.066031in}{1.194481in}}%
\pgfpathlineto{\pgfqpoint{2.228292in}{1.176416in}}%
\pgfpathlineto{\pgfqpoint{2.405934in}{1.154046in}}%
\pgfpathlineto{\pgfqpoint{2.545487in}{1.134396in}}%
\pgfpathlineto{\pgfqpoint{2.684448in}{1.112496in}}%
\pgfpathlineto{\pgfqpoint{2.818060in}{1.088608in}}%
\pgfpathlineto{\pgfqpoint{2.902180in}{1.071734in}}%
\pgfpathlineto{\pgfqpoint{2.981183in}{1.054224in}}%
\pgfpathlineto{\pgfqpoint{3.054258in}{1.036195in}}%
\pgfpathlineto{\pgfqpoint{3.120755in}{1.017777in}}%
\pgfpathlineto{\pgfqpoint{3.180184in}{0.999108in}}%
\pgfpathlineto{\pgfqpoint{3.232214in}{0.980338in}}%
\pgfpathlineto{\pgfqpoint{3.276797in}{0.961625in}}%
\pgfpathlineto{\pgfqpoint{3.314717in}{0.943094in}}%
\pgfpathlineto{\pgfqpoint{3.346855in}{0.924834in}}%
\pgfpathlineto{\pgfqpoint{3.373917in}{0.906927in}}%
\pgfpathlineto{\pgfqpoint{3.396441in}{0.889444in}}%
\pgfpathlineto{\pgfqpoint{3.414799in}{0.872447in}}%
\pgfpathlineto{\pgfqpoint{3.429194in}{0.855989in}}%
\pgfpathlineto{\pgfqpoint{3.439663in}{0.840115in}}%
\pgfpathlineto{\pgfqpoint{3.446618in}{0.824862in}}%
\pgfpathlineto{\pgfqpoint{3.450699in}{0.810255in}}%
\pgfpathlineto{\pgfqpoint{3.452200in}{0.796308in}}%
\pgfpathlineto{\pgfqpoint{3.451348in}{0.783035in}}%
\pgfpathlineto{\pgfqpoint{3.448299in}{0.770446in}}%
\pgfpathlineto{\pgfqpoint{3.443137in}{0.758548in}}%
\pgfpathlineto{\pgfqpoint{3.435880in}{0.747343in}}%
\pgfpathlineto{\pgfqpoint{3.426488in}{0.736832in}}%
\pgfpathlineto{\pgfqpoint{3.415054in}{0.727015in}}%
\pgfpathlineto{\pgfqpoint{3.401679in}{0.717897in}}%
\pgfpathlineto{\pgfqpoint{3.386415in}{0.709479in}}%
\pgfpathlineto{\pgfqpoint{3.360021in}{0.698170in}}%
\pgfpathlineto{\pgfqpoint{3.329383in}{0.688447in}}%
\pgfpathlineto{\pgfqpoint{3.294316in}{0.680314in}}%
\pgfpathlineto{\pgfqpoint{3.254493in}{0.673774in}}%
\pgfpathlineto{\pgfqpoint{3.209443in}{0.668827in}}%
\pgfpathlineto{\pgfqpoint{3.158923in}{0.665490in}}%
\pgfpathlineto{\pgfqpoint{3.102552in}{0.663814in}}%
\pgfpathlineto{\pgfqpoint{3.039555in}{0.663850in}}%
\pgfpathlineto{\pgfqpoint{2.969217in}{0.665662in}}%
\pgfpathlineto{\pgfqpoint{2.862907in}{0.670966in}}%
\pgfpathlineto{\pgfqpoint{2.741050in}{0.679756in}}%
\pgfpathlineto{\pgfqpoint{2.602497in}{0.692263in}}%
\pgfpathlineto{\pgfqpoint{2.445466in}{0.708772in}}%
\pgfpathlineto{\pgfqpoint{2.269130in}{0.729602in}}%
\pgfpathlineto{\pgfqpoint{2.130836in}{0.748029in}}%
\pgfpathlineto{\pgfqpoint{1.992575in}{0.768734in}}%
\pgfpathlineto{\pgfqpoint{1.858391in}{0.791532in}}%
\pgfpathlineto{\pgfqpoint{1.731732in}{0.816173in}}%
\pgfpathlineto{\pgfqpoint{1.652905in}{0.833473in}}%
\pgfpathlineto{\pgfqpoint{1.579398in}{0.851343in}}%
\pgfpathlineto{\pgfqpoint{1.511790in}{0.869662in}}%
\pgfpathlineto{\pgfqpoint{1.450543in}{0.888295in}}%
\pgfpathlineto{\pgfqpoint{1.395998in}{0.907093in}}%
\pgfpathlineto{\pgfqpoint{1.348382in}{0.925895in}}%
\pgfpathlineto{\pgfqpoint{1.307802in}{0.944528in}}%
\pgfpathlineto{\pgfqpoint{1.274011in}{0.962836in}}%
\pgfpathlineto{\pgfqpoint{1.245814in}{0.980796in}}%
\pgfpathlineto{\pgfqpoint{1.222584in}{0.998342in}}%
\pgfpathlineto{\pgfqpoint{1.203816in}{1.015407in}}%
\pgfpathlineto{\pgfqpoint{1.189048in}{1.031939in}}%
\pgfpathlineto{\pgfqpoint{1.177864in}{1.047898in}}%
\pgfpathlineto{\pgfqpoint{1.169914in}{1.063249in}}%
\pgfpathlineto{\pgfqpoint{1.164932in}{1.077967in}}%
\pgfpathlineto{\pgfqpoint{1.162800in}{1.092031in}}%
\pgfpathlineto{\pgfqpoint{1.163198in}{1.105429in}}%
\pgfpathlineto{\pgfqpoint{1.165834in}{1.118149in}}%
\pgfpathlineto{\pgfqpoint{1.170491in}{1.130183in}}%
\pgfpathlineto{\pgfqpoint{1.177031in}{1.141526in}}%
\pgfpathlineto{\pgfqpoint{1.185391in}{1.152175in}}%
\pgfpathlineto{\pgfqpoint{1.195586in}{1.162130in}}%
\pgfpathlineto{\pgfqpoint{1.207707in}{1.171392in}}%
\pgfpathlineto{\pgfqpoint{1.221922in}{1.179967in}}%
\pgfpathlineto{\pgfqpoint{1.238431in}{1.187859in}}%
\pgfpathlineto{\pgfqpoint{1.266995in}{1.198396in}}%
\pgfpathlineto{\pgfqpoint{1.300014in}{1.207357in}}%
\pgfpathlineto{\pgfqpoint{1.337649in}{1.214728in}}%
\pgfpathlineto{\pgfqpoint{1.380160in}{1.220494in}}%
\pgfpathlineto{\pgfqpoint{1.427905in}{1.224632in}}%
\pgfpathlineto{\pgfqpoint{1.481342in}{1.227120in}}%
\pgfpathlineto{\pgfqpoint{1.541027in}{1.227929in}}%
\pgfpathlineto{\pgfqpoint{1.630756in}{1.226370in}}%
\pgfpathlineto{\pgfqpoint{1.734203in}{1.221589in}}%
\pgfpathlineto{\pgfqpoint{1.854181in}{1.213276in}}%
\pgfpathlineto{\pgfqpoint{1.991807in}{1.201180in}}%
\pgfpathlineto{\pgfqpoint{2.146495in}{1.185112in}}%
\pgfpathlineto{\pgfqpoint{2.315957in}{1.164943in}}%
\pgfpathlineto{\pgfqpoint{2.450413in}{1.147082in}}%
\pgfpathlineto{\pgfqpoint{2.588643in}{1.126868in}}%
\pgfpathlineto{\pgfqpoint{2.726322in}{1.104356in}}%
\pgfpathlineto{\pgfqpoint{2.857873in}{1.079886in}}%
\pgfpathlineto{\pgfqpoint{2.940002in}{1.062689in}}%
\pgfpathlineto{\pgfqpoint{3.016493in}{1.044933in}}%
\pgfpathlineto{\pgfqpoint{3.086589in}{1.026744in}}%
\pgfpathlineto{\pgfqpoint{3.149766in}{1.008250in}}%
\pgfpathlineto{\pgfqpoint{3.205729in}{0.989582in}}%
\pgfpathlineto{\pgfqpoint{3.254412in}{0.970871in}}%
\pgfpathlineto{\pgfqpoint{3.295968in}{0.952249in}}%
\pgfpathlineto{\pgfqpoint{3.330608in}{0.933878in}}%
\pgfpathlineto{\pgfqpoint{3.359416in}{0.915845in}}%
\pgfpathlineto{\pgfqpoint{3.383536in}{0.898202in}}%
\pgfpathlineto{\pgfqpoint{3.403827in}{0.880997in}}%
\pgfpathlineto{\pgfqpoint{3.420862in}{0.864279in}}%
\pgfpathlineto{\pgfqpoint{3.434928in}{0.848089in}}%
\pgfpathlineto{\pgfqpoint{3.446027in}{0.832467in}}%
\pgfpathlineto{\pgfqpoint{3.453875in}{0.817449in}}%
\pgfpathlineto{\pgfqpoint{3.457901in}{0.803069in}}%
\pgfpathlineto{\pgfqpoint{3.457861in}{0.789358in}}%
\pgfpathlineto{\pgfqpoint{3.455330in}{0.776339in}}%
\pgfpathlineto{\pgfqpoint{3.450634in}{0.764019in}}%
\pgfpathlineto{\pgfqpoint{3.443917in}{0.752402in}}%
\pgfpathlineto{\pgfqpoint{3.435276in}{0.741490in}}%
\pgfpathlineto{\pgfqpoint{3.424763in}{0.731286in}}%
\pgfpathlineto{\pgfqpoint{3.412383in}{0.721788in}}%
\pgfpathlineto{\pgfqpoint{3.398094in}{0.712995in}}%
\pgfpathlineto{\pgfqpoint{3.381810in}{0.704902in}}%
\pgfpathlineto{\pgfqpoint{3.353639in}{0.694076in}}%
\pgfpathlineto{\pgfqpoint{3.321049in}{0.684833in}}%
\pgfpathlineto{\pgfqpoint{3.283894in}{0.677186in}}%
\pgfpathlineto{\pgfqpoint{3.241929in}{0.671150in}}%
\pgfpathlineto{\pgfqpoint{3.194808in}{0.666746in}}%
\pgfpathlineto{\pgfqpoint{3.142089in}{0.663998in}}%
\pgfpathlineto{\pgfqpoint{3.083227in}{0.662936in}}%
\pgfpathlineto{\pgfqpoint{3.017622in}{0.663588in}}%
\pgfpathlineto{\pgfqpoint{2.919244in}{0.667146in}}%
\pgfpathlineto{\pgfqpoint{2.805476in}{0.674098in}}%
\pgfpathlineto{\pgfqpoint{2.674072in}{0.684787in}}%
\pgfpathlineto{\pgfqpoint{2.524808in}{0.699458in}}%
\pgfpathlineto{\pgfqpoint{2.359485in}{0.718253in}}%
\pgfpathlineto{\pgfqpoint{2.181925in}{0.741215in}}%
\pgfpathlineto{\pgfqpoint{2.044163in}{0.761139in}}%
\pgfpathlineto{\pgfqpoint{1.906088in}{0.783316in}}%
\pgfpathlineto{\pgfqpoint{1.773136in}{0.807565in}}%
\pgfpathlineto{\pgfqpoint{1.689825in}{0.824667in}}%
\pgfpathlineto{\pgfqpoint{1.612045in}{0.842359in}}%
\pgfpathlineto{\pgfqpoint{1.540599in}{0.860504in}}%
\pgfpathlineto{\pgfqpoint{1.476038in}{0.878971in}}%
\pgfpathlineto{\pgfqpoint{1.418660in}{0.897629in}}%
\pgfpathlineto{\pgfqpoint{1.368506in}{0.916351in}}%
\pgfpathlineto{\pgfqpoint{1.325367in}{0.935010in}}%
\pgfpathlineto{\pgfqpoint{1.288888in}{0.953476in}}%
\pgfpathlineto{\pgfqpoint{1.258702in}{0.971617in}}%
\pgfpathlineto{\pgfqpoint{1.233683in}{0.989370in}}%
\pgfpathlineto{\pgfqpoint{1.212863in}{1.006685in}}%
\pgfpathlineto{\pgfqpoint{1.195540in}{1.023515in}}%
\pgfpathlineto{\pgfqpoint{1.181283in}{1.039817in}}%
\pgfpathlineto{\pgfqpoint{1.169926in}{1.055554in}}%
\pgfpathlineto{\pgfqpoint{1.161571in}{1.070691in}}%
\pgfpathlineto{\pgfqpoint{1.156588in}{1.085198in}}%
\pgfpathlineto{\pgfqpoint{1.155541in}{1.099048in}}%
\pgfpathlineto{\pgfqpoint{1.157439in}{1.112216in}}%
\pgfpathlineto{\pgfqpoint{1.161564in}{1.124688in}}%
\pgfpathlineto{\pgfqpoint{1.167755in}{1.136460in}}%
\pgfpathlineto{\pgfqpoint{1.175897in}{1.147528in}}%
\pgfpathlineto{\pgfqpoint{1.185920in}{1.157892in}}%
\pgfpathlineto{\pgfqpoint{1.197802in}{1.167549in}}%
\pgfpathlineto{\pgfqpoint{1.211565in}{1.176503in}}%
\pgfpathlineto{\pgfqpoint{1.227277in}{1.184754in}}%
\pgfpathlineto{\pgfqpoint{1.254653in}{1.195823in}}%
\pgfpathlineto{\pgfqpoint{1.286470in}{1.205314in}}%
\pgfpathlineto{\pgfqpoint{1.322835in}{1.213214in}}%
\pgfpathlineto{\pgfqpoint{1.363976in}{1.219509in}}%
\pgfpathlineto{\pgfqpoint{1.410217in}{1.224178in}}%
\pgfpathlineto{\pgfqpoint{1.461979in}{1.227195in}}%
\pgfpathlineto{\pgfqpoint{1.519782in}{1.228530in}}%
\pgfpathlineto{\pgfqpoint{1.584238in}{1.228149in}}%
\pgfpathlineto{\pgfqpoint{1.681240in}{1.224912in}}%
\pgfpathlineto{\pgfqpoint{1.792960in}{1.218331in}}%
\pgfpathlineto{\pgfqpoint{1.921834in}{1.208103in}}%
\pgfpathlineto{\pgfqpoint{2.068446in}{1.193989in}}%
\pgfpathlineto{\pgfqpoint{2.231516in}{1.175811in}}%
\pgfpathlineto{\pgfqpoint{2.407899in}{1.153455in}}%
\pgfpathlineto{\pgfqpoint{2.545972in}{1.133912in}}%
\pgfpathlineto{\pgfqpoint{2.684846in}{1.112008in}}%
\pgfpathlineto{\pgfqpoint{2.819022in}{1.088045in}}%
\pgfpathlineto{\pgfqpoint{2.903574in}{1.071121in}}%
\pgfpathlineto{\pgfqpoint{2.982922in}{1.053580in}}%
\pgfpathlineto{\pgfqpoint{3.056207in}{1.035544in}}%
\pgfpathlineto{\pgfqpoint{3.122789in}{1.017141in}}%
\pgfpathlineto{\pgfqpoint{3.182243in}{0.998500in}}%
\pgfpathlineto{\pgfqpoint{3.234361in}{0.979752in}}%
\pgfpathlineto{\pgfqpoint{3.279151in}{0.961032in}}%
\pgfpathlineto{\pgfqpoint{3.316871in}{0.942487in}}%
\pgfpathlineto{\pgfqpoint{3.348388in}{0.924237in}}%
\pgfpathlineto{\pgfqpoint{3.374747in}{0.906353in}}%
\pgfpathlineto{\pgfqpoint{3.396760in}{0.888894in}}%
\pgfpathlineto{\pgfqpoint{3.415003in}{0.871915in}}%
\pgfpathlineto{\pgfqpoint{3.429819in}{0.855465in}}%
\pgfpathlineto{\pgfqpoint{3.441316in}{0.839586in}}%
\pgfpathlineto{\pgfqpoint{3.449368in}{0.824315in}}%
\pgfpathlineto{\pgfqpoint{3.453640in}{0.809681in}}%
\pgfpathlineto{\pgfqpoint{3.454808in}{0.795714in}}%
\pgfpathlineto{\pgfqpoint{3.453551in}{0.782428in}}%
\pgfpathlineto{\pgfqpoint{3.450080in}{0.769831in}}%
\pgfpathlineto{\pgfqpoint{3.444553in}{0.757929in}}%
\pgfpathlineto{\pgfqpoint{3.437070in}{0.746727in}}%
\pgfpathlineto{\pgfqpoint{3.427677in}{0.736227in}}%
\pgfpathlineto{\pgfqpoint{3.416363in}{0.726427in}}%
\pgfpathlineto{\pgfqpoint{3.403063in}{0.717327in}}%
\pgfpathlineto{\pgfqpoint{3.387702in}{0.708922in}}%
\pgfpathlineto{\pgfqpoint{3.361005in}{0.697625in}}%
\pgfpathlineto{\pgfqpoint{3.329936in}{0.687909in}}%
\pgfpathlineto{\pgfqpoint{3.294383in}{0.679784in}}%
\pgfpathlineto{\pgfqpoint{3.254124in}{0.673263in}}%
\pgfpathlineto{\pgfqpoint{3.208830in}{0.668364in}}%
\pgfpathlineto{\pgfqpoint{3.158064in}{0.665104in}}%
\pgfpathlineto{\pgfqpoint{3.101281in}{0.663505in}}%
\pgfpathlineto{\pgfqpoint{3.038189in}{0.663583in}}%
\pgfpathlineto{\pgfqpoint{2.968239in}{0.665406in}}%
\pgfpathlineto{\pgfqpoint{2.861995in}{0.670749in}}%
\pgfpathlineto{\pgfqpoint{2.739090in}{0.679650in}}%
\pgfpathlineto{\pgfqpoint{2.598837in}{0.692329in}}%
\pgfpathlineto{\pgfqpoint{2.441945in}{0.708974in}}%
\pgfpathlineto{\pgfqpoint{2.270520in}{0.729744in}}%
\pgfpathlineto{\pgfqpoint{2.134469in}{0.748106in}}%
\pgfpathlineto{\pgfqpoint{1.995132in}{0.768876in}}%
\pgfpathlineto{\pgfqpoint{1.858069in}{0.791833in}}%
\pgfpathlineto{\pgfqpoint{1.728307in}{0.816644in}}%
\pgfpathlineto{\pgfqpoint{1.647853in}{0.834031in}}%
\pgfpathlineto{\pgfqpoint{1.573337in}{0.851955in}}%
\pgfpathlineto{\pgfqpoint{1.505460in}{0.870287in}}%
\pgfpathlineto{\pgfqpoint{1.444713in}{0.888891in}}%
\pgfpathlineto{\pgfqpoint{1.391376in}{0.907622in}}%
\pgfpathlineto{\pgfqpoint{1.345488in}{0.926323in}}%
\pgfpathlineto{\pgfqpoint{1.306339in}{0.944865in}}%
\pgfpathlineto{\pgfqpoint{1.272991in}{0.963160in}}%
\pgfpathlineto{\pgfqpoint{1.244699in}{0.981126in}}%
\pgfpathlineto{\pgfqpoint{1.220912in}{0.998691in}}%
\pgfpathlineto{\pgfqpoint{1.201275in}{1.015790in}}%
\pgfpathlineto{\pgfqpoint{1.185627in}{1.032366in}}%
\pgfpathlineto{\pgfqpoint{1.173999in}{1.048373in}}%
\pgfpathlineto{\pgfqpoint{1.166378in}{1.063767in}}%
\pgfpathlineto{\pgfqpoint{1.161821in}{1.078518in}}%
\pgfpathlineto{\pgfqpoint{1.159909in}{1.092609in}}%
\pgfpathlineto{\pgfqpoint{1.160375in}{1.106028in}}%
\pgfpathlineto{\pgfqpoint{1.163031in}{1.118763in}}%
\pgfpathlineto{\pgfqpoint{1.167758in}{1.130808in}}%
\pgfpathlineto{\pgfqpoint{1.174511in}{1.142158in}}%
\pgfpathlineto{\pgfqpoint{1.183322in}{1.152814in}}%
\pgfpathlineto{\pgfqpoint{1.194276in}{1.162776in}}%
\pgfpathlineto{\pgfqpoint{1.207260in}{1.172042in}}%
\pgfpathlineto{\pgfqpoint{1.222171in}{1.180609in}}%
\pgfpathlineto{\pgfqpoint{1.248087in}{1.192144in}}%
\pgfpathlineto{\pgfqpoint{1.278275in}{1.202095in}}%
\pgfpathlineto{\pgfqpoint{1.312868in}{1.210454in}}%
\pgfpathlineto{\pgfqpoint{1.352125in}{1.217214in}}%
\pgfpathlineto{\pgfqpoint{1.396436in}{1.222371in}}%
\pgfpathlineto{\pgfqpoint{1.446288in}{1.225916in}}%
\pgfpathlineto{\pgfqpoint{1.501816in}{1.227818in}}%
\pgfpathlineto{\pgfqpoint{1.563759in}{1.228019in}}%
\pgfpathlineto{\pgfqpoint{1.632989in}{1.226452in}}%
\pgfpathlineto{\pgfqpoint{1.737883in}{1.221484in}}%
\pgfpathlineto{\pgfqpoint{1.858390in}{1.213045in}}%
\pgfpathlineto{\pgfqpoint{1.995446in}{1.200914in}}%
\pgfpathlineto{\pgfqpoint{2.149555in}{1.184843in}}%
\pgfpathlineto{\pgfqpoint{2.321259in}{1.164545in}}%
\pgfpathlineto{\pgfqpoint{2.458387in}{1.146472in}}%
\pgfpathlineto{\pgfqpoint{2.597241in}{1.126066in}}%
\pgfpathlineto{\pgfqpoint{2.733415in}{1.103504in}}%
\pgfpathlineto{\pgfqpoint{2.863004in}{1.079041in}}%
\pgfpathlineto{\pgfqpoint{2.944039in}{1.061835in}}%
\pgfpathlineto{\pgfqpoint{3.019745in}{1.044044in}}%
\pgfpathlineto{\pgfqpoint{3.089338in}{1.025796in}}%
\pgfpathlineto{\pgfqpoint{3.152137in}{1.007238in}}%
\pgfpathlineto{\pgfqpoint{3.207559in}{0.988527in}}%
\pgfpathlineto{\pgfqpoint{3.255401in}{0.969825in}}%
\pgfpathlineto{\pgfqpoint{3.296509in}{0.951241in}}%
\pgfpathlineto{\pgfqpoint{3.331534in}{0.932879in}}%
\pgfpathlineto{\pgfqpoint{3.361035in}{0.914832in}}%
\pgfpathlineto{\pgfqpoint{3.385499in}{0.897181in}}%
\pgfpathlineto{\pgfqpoint{3.405345in}{0.879995in}}%
\pgfpathlineto{\pgfqpoint{3.420944in}{0.863333in}}%
\pgfpathlineto{\pgfqpoint{3.432819in}{0.847238in}}%
\pgfpathlineto{\pgfqpoint{3.441405in}{0.831745in}}%
\pgfpathlineto{\pgfqpoint{3.447039in}{0.816879in}}%
\pgfpathlineto{\pgfqpoint{3.449976in}{0.802662in}}%
\pgfpathlineto{\pgfqpoint{3.450394in}{0.789110in}}%
\pgfpathlineto{\pgfqpoint{3.448390in}{0.776234in}}%
\pgfpathlineto{\pgfqpoint{3.444044in}{0.764040in}}%
\pgfpathlineto{\pgfqpoint{3.437532in}{0.752534in}}%
\pgfpathlineto{\pgfqpoint{3.429014in}{0.741723in}}%
\pgfpathlineto{\pgfqpoint{3.418606in}{0.731609in}}%
\pgfpathlineto{\pgfqpoint{3.406375in}{0.722194in}}%
\pgfpathlineto{\pgfqpoint{3.392348in}{0.713481in}}%
\pgfpathlineto{\pgfqpoint{3.376504in}{0.705468in}}%
\pgfpathlineto{\pgfqpoint{3.349178in}{0.694757in}}%
\pgfpathlineto{\pgfqpoint{3.317200in}{0.685606in}}%
\pgfpathlineto{\pgfqpoint{3.280276in}{0.678012in}}%
\pgfpathlineto{\pgfqpoint{3.238497in}{0.672008in}}%
\pgfpathlineto{\pgfqpoint{3.191506in}{0.667617in}}%
\pgfpathlineto{\pgfqpoint{3.138864in}{0.664872in}}%
\pgfpathlineto{\pgfqpoint{3.080067in}{0.663813in}}%
\pgfpathlineto{\pgfqpoint{3.014541in}{0.664491in}}%
\pgfpathlineto{\pgfqpoint{2.915585in}{0.668198in}}%
\pgfpathlineto{\pgfqpoint{2.801837in}{0.675264in}}%
\pgfpathlineto{\pgfqpoint{2.671647in}{0.685926in}}%
\pgfpathlineto{\pgfqpoint{2.523967in}{0.700447in}}%
\pgfpathlineto{\pgfqpoint{2.359520in}{0.719055in}}%
\pgfpathlineto{\pgfqpoint{2.181264in}{0.741913in}}%
\pgfpathlineto{\pgfqpoint{2.042813in}{0.761833in}}%
\pgfpathlineto{\pgfqpoint{1.904841in}{0.783999in}}%
\pgfpathlineto{\pgfqpoint{1.771998in}{0.808172in}}%
\pgfpathlineto{\pgfqpoint{1.688832in}{0.825241in}}%
\pgfpathlineto{\pgfqpoint{1.611629in}{0.842943in}}%
\pgfpathlineto{\pgfqpoint{1.540915in}{0.861115in}}%
\pgfpathlineto{\pgfqpoint{1.476983in}{0.879598in}}%
\pgfpathlineto{\pgfqpoint{1.419976in}{0.898246in}}%
\pgfpathlineto{\pgfqpoint{1.369890in}{0.916926in}}%
\pgfpathlineto{\pgfqpoint{1.326570in}{0.935517in}}%
\pgfpathlineto{\pgfqpoint{1.289713in}{0.953913in}}%
\pgfpathlineto{\pgfqpoint{1.258865in}{0.972021in}}%
\pgfpathlineto{\pgfqpoint{1.233424in}{0.989759in}}%
\pgfpathlineto{\pgfqpoint{1.212639in}{1.007059in}}%
\pgfpathlineto{\pgfqpoint{1.196254in}{1.023848in}}%
\pgfpathlineto{\pgfqpoint{1.183876in}{1.040074in}}%
\pgfpathlineto{\pgfqpoint{1.174745in}{1.055710in}}%
\pgfpathlineto{\pgfqpoint{1.168283in}{1.070732in}}%
\pgfpathlineto{\pgfqpoint{1.164093in}{1.085122in}}%
\pgfpathlineto{\pgfqpoint{1.161960in}{1.098862in}}%
\pgfpathlineto{\pgfqpoint{1.161848in}{1.111943in}}%
\pgfpathlineto{\pgfqpoint{1.163905in}{1.124356in}}%
\pgfpathlineto{\pgfqpoint{1.168458in}{1.136097in}}%
\pgfpathlineto{\pgfqpoint{1.176006in}{1.147165in}}%
\pgfpathlineto{\pgfqpoint{1.186091in}{1.157545in}}%
\pgfpathlineto{\pgfqpoint{1.198125in}{1.167222in}}%
\pgfpathlineto{\pgfqpoint{1.212050in}{1.176196in}}%
\pgfpathlineto{\pgfqpoint{1.227838in}{1.184464in}}%
\pgfpathlineto{\pgfqpoint{1.255016in}{1.195542in}}%
\pgfpathlineto{\pgfqpoint{1.286497in}{1.205031in}}%
\pgfpathlineto{\pgfqpoint{1.322531in}{1.212929in}}%
\pgfpathlineto{\pgfqpoint{1.363502in}{1.219239in}}%
\pgfpathlineto{\pgfqpoint{1.409582in}{1.223947in}}%
\pgfpathlineto{\pgfqpoint{1.461158in}{1.227019in}}%
\pgfpathlineto{\pgfqpoint{1.518841in}{1.228414in}}%
\pgfpathlineto{\pgfqpoint{1.583246in}{1.228081in}}%
\pgfpathlineto{\pgfqpoint{1.680656in}{1.224841in}}%
\pgfpathlineto{\pgfqpoint{1.792626in}{1.218241in}}%
\pgfpathlineto{\pgfqpoint{1.920679in}{1.208068in}}%
\pgfpathlineto{\pgfqpoint{2.066129in}{1.194083in}}%
\pgfpathlineto{\pgfqpoint{2.228785in}{1.176018in}}%
\pgfpathlineto{\pgfqpoint{2.405736in}{1.153695in}}%
\pgfpathlineto{\pgfqpoint{2.544081in}{1.134173in}}%
\pgfpathlineto{\pgfqpoint{2.682879in}{1.112367in}}%
\pgfpathlineto{\pgfqpoint{2.817244in}{1.088465in}}%
\pgfpathlineto{\pgfqpoint{2.901875in}{1.071554in}}%
\pgfpathlineto{\pgfqpoint{2.981146in}{1.054011in}}%
\pgfpathlineto{\pgfqpoint{3.054166in}{1.035966in}}%
\pgfpathlineto{\pgfqpoint{3.120322in}{1.017553in}}%
\pgfpathlineto{\pgfqpoint{3.179282in}{0.998909in}}%
\pgfpathlineto{\pgfqpoint{3.230989in}{0.980174in}}%
\pgfpathlineto{\pgfqpoint{3.275388in}{0.961506in}}%
\pgfpathlineto{\pgfqpoint{3.312985in}{0.943032in}}%
\pgfpathlineto{\pgfqpoint{3.344938in}{0.924827in}}%
\pgfpathlineto{\pgfqpoint{3.372155in}{0.906960in}}%
\pgfpathlineto{\pgfqpoint{3.395285in}{0.889493in}}%
\pgfpathlineto{\pgfqpoint{3.414716in}{0.872487in}}%
\pgfpathlineto{\pgfqpoint{3.430576in}{0.855994in}}%
\pgfpathlineto{\pgfqpoint{3.442735in}{0.840065in}}%
\pgfpathlineto{\pgfqpoint{3.450802in}{0.824742in}}%
\pgfpathlineto{\pgfqpoint{3.454683in}{0.810066in}}%
\pgfpathlineto{\pgfqpoint{3.455816in}{0.796061in}}%
\pgfpathlineto{\pgfqpoint{3.454565in}{0.782740in}}%
\pgfpathlineto{\pgfqpoint{3.451132in}{0.770109in}}%
\pgfpathlineto{\pgfqpoint{3.445657in}{0.758176in}}%
\pgfpathlineto{\pgfqpoint{3.438222in}{0.746943in}}%
\pgfpathlineto{\pgfqpoint{3.428850in}{0.736412in}}%
\pgfpathlineto{\pgfqpoint{3.417505in}{0.726583in}}%
\pgfpathlineto{\pgfqpoint{3.404100in}{0.717451in}}%
\pgfpathlineto{\pgfqpoint{3.388707in}{0.709018in}}%
\pgfpathlineto{\pgfqpoint{3.362020in}{0.697684in}}%
\pgfpathlineto{\pgfqpoint{3.331002in}{0.687936in}}%
\pgfpathlineto{\pgfqpoint{3.295533in}{0.679784in}}%
\pgfpathlineto{\pgfqpoint{3.255378in}{0.673238in}}%
\pgfpathlineto{\pgfqpoint{3.210187in}{0.668312in}}%
\pgfpathlineto{\pgfqpoint{3.159494in}{0.665019in}}%
\pgfpathlineto{\pgfqpoint{3.102792in}{0.663375in}}%
\pgfpathlineto{\pgfqpoint{3.039956in}{0.663416in}}%
\pgfpathlineto{\pgfqpoint{2.969891in}{0.665228in}}%
\pgfpathlineto{\pgfqpoint{2.863446in}{0.670560in}}%
\pgfpathlineto{\pgfqpoint{2.740679in}{0.679420in}}%
\pgfpathlineto{\pgfqpoint{2.600963in}{0.692022in}}%
\pgfpathlineto{\pgfqpoint{2.444666in}{0.708576in}}%
\pgfpathlineto{\pgfqpoint{2.273161in}{0.729289in}}%
\pgfpathlineto{\pgfqpoint{2.136162in}{0.747684in}}%
\pgfpathlineto{\pgfqpoint{1.996457in}{0.768474in}}%
\pgfpathlineto{\pgfqpoint{1.859501in}{0.791408in}}%
\pgfpathlineto{\pgfqpoint{1.729886in}{0.816181in}}%
\pgfpathlineto{\pgfqpoint{1.649424in}{0.833546in}}%
\pgfpathlineto{\pgfqpoint{1.574775in}{0.851455in}}%
\pgfpathlineto{\pgfqpoint{1.506648in}{0.869783in}}%
\pgfpathlineto{\pgfqpoint{1.445577in}{0.888392in}}%
\pgfpathlineto{\pgfqpoint{1.391922in}{0.907136in}}%
\pgfpathlineto{\pgfqpoint{1.345791in}{0.925858in}}%
\pgfpathlineto{\pgfqpoint{1.306484in}{0.944422in}}%
\pgfpathlineto{\pgfqpoint{1.273069in}{0.962736in}}%
\pgfpathlineto{\pgfqpoint{1.244795in}{0.980720in}}%
\pgfpathlineto{\pgfqpoint{1.221094in}{0.998300in}}%
\pgfpathlineto{\pgfqpoint{1.201582in}{1.015411in}}%
\pgfpathlineto{\pgfqpoint{1.186058in}{1.031998in}}%
\pgfpathlineto{\pgfqpoint{1.174503in}{1.048013in}}%
\pgfpathlineto{\pgfqpoint{1.166824in}{1.063416in}}%
\pgfpathlineto{\pgfqpoint{1.162166in}{1.078178in}}%
\pgfpathlineto{\pgfqpoint{1.160154in}{1.092280in}}%
\pgfpathlineto{\pgfqpoint{1.160532in}{1.105710in}}%
\pgfpathlineto{\pgfqpoint{1.163117in}{1.118458in}}%
\pgfpathlineto{\pgfqpoint{1.167798in}{1.130516in}}%
\pgfpathlineto{\pgfqpoint{1.174535in}{1.141881in}}%
\pgfpathlineto{\pgfqpoint{1.183359in}{1.152550in}}%
\pgfpathlineto{\pgfqpoint{1.194316in}{1.162526in}}%
\pgfpathlineto{\pgfqpoint{1.207257in}{1.171806in}}%
\pgfpathlineto{\pgfqpoint{1.222114in}{1.180386in}}%
\pgfpathlineto{\pgfqpoint{1.247932in}{1.191940in}}%
\pgfpathlineto{\pgfqpoint{1.278003in}{1.201910in}}%
\pgfpathlineto{\pgfqpoint{1.312468in}{1.210288in}}%
\pgfpathlineto{\pgfqpoint{1.351603in}{1.217070in}}%
\pgfpathlineto{\pgfqpoint{1.395820in}{1.222252in}}%
\pgfpathlineto{\pgfqpoint{1.445564in}{1.225828in}}%
\pgfpathlineto{\pgfqpoint{1.500986in}{1.227756in}}%
\pgfpathlineto{\pgfqpoint{1.562874in}{1.227982in}}%
\pgfpathlineto{\pgfqpoint{1.632027in}{1.226442in}}%
\pgfpathlineto{\pgfqpoint{1.736723in}{1.221516in}}%
\pgfpathlineto{\pgfqpoint{1.856922in}{1.213124in}}%
\pgfpathlineto{\pgfqpoint{1.993658in}{1.201043in}}%
\pgfpathlineto{\pgfqpoint{2.147635in}{1.185020in}}%
\pgfpathlineto{\pgfqpoint{2.320129in}{1.164749in}}%
\pgfpathlineto{\pgfqpoint{2.457334in}{1.146710in}}%
\pgfpathlineto{\pgfqpoint{2.595947in}{1.126345in}}%
\pgfpathlineto{\pgfqpoint{2.731727in}{1.103827in}}%
\pgfpathlineto{\pgfqpoint{2.860928in}{1.079404in}}%
\pgfpathlineto{\pgfqpoint{2.941783in}{1.062218in}}%
\pgfpathlineto{\pgfqpoint{3.017418in}{1.044441in}}%
\pgfpathlineto{\pgfqpoint{3.087091in}{1.026200in}}%
\pgfpathlineto{\pgfqpoint{3.150157in}{1.007638in}}%
\pgfpathlineto{\pgfqpoint{3.206067in}{0.988913in}}%
\pgfpathlineto{\pgfqpoint{3.254391in}{0.970195in}}%
\pgfpathlineto{\pgfqpoint{3.295683in}{0.951608in}}%
\pgfpathlineto{\pgfqpoint{3.330814in}{0.933242in}}%
\pgfpathlineto{\pgfqpoint{3.360350in}{0.915191in}}%
\pgfpathlineto{\pgfqpoint{3.384807in}{0.897536in}}%
\pgfpathlineto{\pgfqpoint{3.404652in}{0.880346in}}%
\pgfpathlineto{\pgfqpoint{3.420332in}{0.863679in}}%
\pgfpathlineto{\pgfqpoint{3.432327in}{0.847577in}}%
\pgfpathlineto{\pgfqpoint{3.441026in}{0.832074in}}%
\pgfpathlineto{\pgfqpoint{3.446742in}{0.817199in}}%
\pgfpathlineto{\pgfqpoint{3.449717in}{0.802972in}}%
\pgfpathlineto{\pgfqpoint{3.450108in}{0.789409in}}%
\pgfpathlineto{\pgfqpoint{3.448010in}{0.776521in}}%
\pgfpathlineto{\pgfqpoint{3.443650in}{0.764316in}}%
\pgfpathlineto{\pgfqpoint{3.437230in}{0.752800in}}%
\pgfpathlineto{\pgfqpoint{3.428894in}{0.741980in}}%
\pgfpathlineto{\pgfqpoint{3.418733in}{0.731857in}}%
\pgfpathlineto{\pgfqpoint{3.406781in}{0.722433in}}%
\pgfpathlineto{\pgfqpoint{3.393017in}{0.713708in}}%
\pgfpathlineto{\pgfqpoint{3.377365in}{0.705678in}}%
\pgfpathlineto{\pgfqpoint{3.359693in}{0.698339in}}%
\pgfpathlineto{\pgfqpoint{3.329068in}{0.688618in}}%
\pgfpathlineto{\pgfqpoint{3.293765in}{0.680465in}}%
\pgfpathlineto{\pgfqpoint{3.253696in}{0.673900in}}%
\pgfpathlineto{\pgfqpoint{3.208551in}{0.668945in}}%
\pgfpathlineto{\pgfqpoint{3.157939in}{0.665628in}}%
\pgfpathlineto{\pgfqpoint{3.101383in}{0.663984in}}%
\pgfpathlineto{\pgfqpoint{3.038324in}{0.664055in}}%
\pgfpathlineto{\pgfqpoint{2.943004in}{0.666904in}}%
\pgfpathlineto{\pgfqpoint{2.833464in}{0.673047in}}%
\pgfpathlineto{\pgfqpoint{2.707647in}{0.682740in}}%
\pgfpathlineto{\pgfqpoint{2.564381in}{0.696234in}}%
\pgfpathlineto{\pgfqpoint{2.404176in}{0.713752in}}%
\pgfpathlineto{\pgfqpoint{2.229184in}{0.735482in}}%
\pgfpathlineto{\pgfqpoint{2.091069in}{0.754610in}}%
\pgfpathlineto{\pgfqpoint{1.952270in}{0.776034in}}%
\pgfpathlineto{\pgfqpoint{1.817580in}{0.799530in}}%
\pgfpathlineto{\pgfqpoint{1.732175in}{0.816200in}}%
\pgfpathlineto{\pgfqpoint{1.651590in}{0.833550in}}%
\pgfpathlineto{\pgfqpoint{1.576819in}{0.851456in}}%
\pgfpathlineto{\pgfqpoint{1.508743in}{0.869778in}}%
\pgfpathlineto{\pgfqpoint{1.448107in}{0.888361in}}%
\pgfpathlineto{\pgfqpoint{1.394953in}{0.907051in}}%
\pgfpathlineto{\pgfqpoint{1.348669in}{0.925721in}}%
\pgfpathlineto{\pgfqpoint{1.308677in}{0.944259in}}%
\pgfpathlineto{\pgfqpoint{1.274475in}{0.962562in}}%
\pgfpathlineto{\pgfqpoint{1.245633in}{0.980539in}}%
\pgfpathlineto{\pgfqpoint{1.221796in}{0.998111in}}%
\pgfpathlineto{\pgfqpoint{1.202682in}{1.015207in}}%
\pgfpathlineto{\pgfqpoint{1.187979in}{1.031769in}}%
\pgfpathlineto{\pgfqpoint{1.176977in}{1.047751in}}%
\pgfpathlineto{\pgfqpoint{1.169171in}{1.063123in}}%
\pgfpathlineto{\pgfqpoint{1.164184in}{1.077863in}}%
\pgfpathlineto{\pgfqpoint{1.161742in}{1.091950in}}%
\pgfpathlineto{\pgfqpoint{1.161673in}{1.105371in}}%
\pgfpathlineto{\pgfqpoint{1.163906in}{1.118113in}}%
\pgfpathlineto{\pgfqpoint{1.168471in}{1.130173in}}%
\pgfpathlineto{\pgfqpoint{1.175368in}{1.141545in}}%
\pgfpathlineto{\pgfqpoint{1.184343in}{1.152223in}}%
\pgfpathlineto{\pgfqpoint{1.195279in}{1.162204in}}%
\pgfpathlineto{\pgfqpoint{1.208100in}{1.171486in}}%
\pgfpathlineto{\pgfqpoint{1.222763in}{1.180065in}}%
\pgfpathlineto{\pgfqpoint{1.248201in}{1.191618in}}%
\pgfpathlineto{\pgfqpoint{1.277886in}{1.201589in}}%
\pgfpathlineto{\pgfqpoint{1.312105in}{1.209984in}}%
\pgfpathlineto{\pgfqpoint{1.351283in}{1.216809in}}%
\pgfpathlineto{\pgfqpoint{1.395498in}{1.222043in}}%
\pgfpathlineto{\pgfqpoint{1.445096in}{1.225654in}}%
\pgfpathlineto{\pgfqpoint{1.500606in}{1.227604in}}%
\pgfpathlineto{\pgfqpoint{1.562594in}{1.227848in}}%
\pgfpathlineto{\pgfqpoint{1.631661in}{1.226327in}}%
\pgfpathlineto{\pgfqpoint{1.735870in}{1.221434in}}%
\pgfpathlineto{\pgfqpoint{1.855413in}{1.213091in}}%
\pgfpathlineto{\pgfqpoint{1.991835in}{1.201068in}}%
\pgfpathlineto{\pgfqpoint{2.145793in}{1.185095in}}%
\pgfpathlineto{\pgfqpoint{2.315812in}{1.164958in}}%
\pgfpathlineto{\pgfqpoint{2.451333in}{1.147057in}}%
\pgfpathlineto{\pgfqpoint{2.590213in}{1.126788in}}%
\pgfpathlineto{\pgfqpoint{2.727773in}{1.104302in}}%
\pgfpathlineto{\pgfqpoint{2.859328in}{1.079855in}}%
\pgfpathlineto{\pgfqpoint{2.941455in}{1.062640in}}%
\pgfpathlineto{\pgfqpoint{3.017458in}{1.044843in}}%
\pgfpathlineto{\pgfqpoint{3.086101in}{1.026610in}}%
\pgfpathlineto{\pgfqpoint{3.147652in}{1.008094in}}%
\pgfpathlineto{\pgfqpoint{3.202484in}{0.989434in}}%
\pgfpathlineto{\pgfqpoint{3.250950in}{0.970760in}}%
\pgfpathlineto{\pgfqpoint{3.293376in}{0.952187in}}%
\pgfpathlineto{\pgfqpoint{3.330068in}{0.933819in}}%
\pgfpathlineto{\pgfqpoint{3.361306in}{0.915750in}}%
\pgfpathlineto{\pgfqpoint{3.387348in}{0.898058in}}%
\pgfpathlineto{\pgfqpoint{3.408429in}{0.880812in}}%
\pgfpathlineto{\pgfqpoint{3.424759in}{0.864068in}}%
\pgfpathlineto{\pgfqpoint{3.436558in}{0.847873in}}%
\pgfpathlineto{\pgfqpoint{3.444508in}{0.832291in}}%
\pgfpathlineto{\pgfqpoint{3.449353in}{0.817349in}}%
\pgfpathlineto{\pgfqpoint{3.451659in}{0.803063in}}%
\pgfpathlineto{\pgfqpoint{3.451848in}{0.789445in}}%
\pgfpathlineto{\pgfqpoint{3.450204in}{0.776506in}}%
\pgfpathlineto{\pgfqpoint{3.446868in}{0.764250in}}%
\pgfpathlineto{\pgfqpoint{3.441839in}{0.752681in}}%
\pgfpathlineto{\pgfqpoint{3.434978in}{0.741798in}}%
\pgfpathlineto{\pgfqpoint{3.426002in}{0.731598in}}%
\pgfpathlineto{\pgfqpoint{3.414488in}{0.722075in}}%
\pgfpathlineto{\pgfqpoint{3.400149in}{0.713227in}}%
\pgfpathlineto{\pgfqpoint{3.383782in}{0.705080in}}%
\pgfpathlineto{\pgfqpoint{3.355653in}{0.694185in}}%
\pgfpathlineto{\pgfqpoint{3.323168in}{0.684885in}}%
\pgfpathlineto{\pgfqpoint{3.286158in}{0.677188in}}%
\pgfpathlineto{\pgfqpoint{3.244345in}{0.671104in}}%
\pgfpathlineto{\pgfqpoint{3.197345in}{0.666645in}}%
\pgfpathlineto{\pgfqpoint{3.144678in}{0.663826in}}%
\pgfpathlineto{\pgfqpoint{3.086207in}{0.662670in}}%
\pgfpathlineto{\pgfqpoint{3.021086in}{0.663243in}}%
\pgfpathlineto{\pgfqpoint{2.922158in}{0.666833in}}%
\pgfpathlineto{\pgfqpoint{2.807790in}{0.673834in}}%
\pgfpathlineto{\pgfqpoint{2.676919in}{0.684456in}}%
\pgfpathlineto{\pgfqpoint{2.529219in}{0.698920in}}%
\pgfpathlineto{\pgfqpoint{2.365105in}{0.717458in}}%
\pgfpathlineto{\pgfqpoint{2.185797in}{0.740338in}}%
\pgfpathlineto{\pgfqpoint{2.045965in}{0.760344in}}%
\pgfpathlineto{\pgfqpoint{1.907536in}{0.782563in}}%
\pgfpathlineto{\pgfqpoint{1.775114in}{0.806725in}}%
\pgfpathlineto{\pgfqpoint{1.692081in}{0.823755in}}%
\pgfpathlineto{\pgfqpoint{1.614357in}{0.841397in}}%
\pgfpathlineto{\pgfqpoint{1.542715in}{0.859529in}}%
\pgfpathlineto{\pgfqpoint{1.477772in}{0.878020in}}%
\pgfpathlineto{\pgfqpoint{1.419985in}{0.896729in}}%
\pgfpathlineto{\pgfqpoint{1.369657in}{0.915502in}}%
\pgfpathlineto{\pgfqpoint{1.326656in}{0.934184in}}%
\pgfpathlineto{\pgfqpoint{1.290081in}{0.952665in}}%
\pgfpathlineto{\pgfqpoint{1.259116in}{0.970854in}}%
\pgfpathlineto{\pgfqpoint{1.233104in}{0.988675in}}%
\pgfpathlineto{\pgfqpoint{1.211554in}{1.006056in}}%
\pgfpathlineto{\pgfqpoint{1.194134in}{1.022937in}}%
\pgfpathlineto{\pgfqpoint{1.180678in}{1.039267in}}%
\pgfpathlineto{\pgfqpoint{1.171123in}{1.055004in}}%
\pgfpathlineto{\pgfqpoint{1.164832in}{1.070113in}}%
\pgfpathlineto{\pgfqpoint{1.161336in}{1.084573in}}%
\pgfpathlineto{\pgfqpoint{1.160359in}{1.098370in}}%
\pgfpathlineto{\pgfqpoint{1.161696in}{1.111490in}}%
\pgfpathlineto{\pgfqpoint{1.165212in}{1.123925in}}%
\pgfpathlineto{\pgfqpoint{1.170842in}{1.135668in}}%
\pgfpathlineto{\pgfqpoint{1.178591in}{1.146718in}}%
\pgfpathlineto{\pgfqpoint{1.188470in}{1.157075in}}%
\pgfpathlineto{\pgfqpoint{1.200346in}{1.166735in}}%
\pgfpathlineto{\pgfqpoint{1.214142in}{1.175697in}}%
\pgfpathlineto{\pgfqpoint{1.229812in}{1.183958in}}%
\pgfpathlineto{\pgfqpoint{1.256805in}{1.195030in}}%
\pgfpathlineto{\pgfqpoint{1.288057in}{1.204515in}}%
\pgfpathlineto{\pgfqpoint{1.323784in}{1.212410in}}%
\pgfpathlineto{\pgfqpoint{1.364347in}{1.218715in}}%
\pgfpathlineto{\pgfqpoint{1.410228in}{1.223430in}}%
\pgfpathlineto{\pgfqpoint{1.461552in}{1.226523in}}%
\pgfpathlineto{\pgfqpoint{1.518881in}{1.227949in}}%
\pgfpathlineto{\pgfqpoint{1.582947in}{1.227653in}}%
\pgfpathlineto{\pgfqpoint{1.680023in}{1.224469in}}%
\pgfpathlineto{\pgfqpoint{1.791764in}{1.217931in}}%
\pgfpathlineto{\pgfqpoint{1.919484in}{1.207828in}}%
\pgfpathlineto{\pgfqpoint{2.064335in}{1.193915in}}%
\pgfpathlineto{\pgfqpoint{2.227247in}{1.175967in}}%
\pgfpathlineto{\pgfqpoint{2.359053in}{1.159720in}}%
\pgfpathlineto{\pgfqpoint{2.495748in}{1.141046in}}%
\pgfpathlineto{\pgfqpoint{2.633866in}{1.120028in}}%
\pgfpathlineto{\pgfqpoint{2.769531in}{1.096854in}}%
\pgfpathlineto{\pgfqpoint{2.898460in}{1.071819in}}%
\pgfpathlineto{\pgfqpoint{2.978371in}{1.054289in}}%
\pgfpathlineto{\pgfqpoint{3.051724in}{1.036253in}}%
\pgfpathlineto{\pgfqpoint{3.117591in}{1.017858in}}%
\pgfpathlineto{\pgfqpoint{3.176218in}{0.999244in}}%
\pgfpathlineto{\pgfqpoint{3.227970in}{0.980547in}}%
\pgfpathlineto{\pgfqpoint{3.273204in}{0.961891in}}%
\pgfpathlineto{\pgfqpoint{3.312276in}{0.943389in}}%
\pgfpathlineto{\pgfqpoint{3.345537in}{0.925140in}}%
\pgfpathlineto{\pgfqpoint{3.373331in}{0.907237in}}%
\pgfpathlineto{\pgfqpoint{3.396001in}{0.889755in}}%
\pgfpathlineto{\pgfqpoint{3.413886in}{0.872763in}}%
\pgfpathlineto{\pgfqpoint{3.427520in}{0.856322in}}%
\pgfpathlineto{\pgfqpoint{3.437591in}{0.840470in}}%
\pgfpathlineto{\pgfqpoint{3.444642in}{0.825233in}}%
\pgfpathlineto{\pgfqpoint{3.449083in}{0.810632in}}%
\pgfpathlineto{\pgfqpoint{3.451190in}{0.796685in}}%
\pgfpathlineto{\pgfqpoint{3.451101in}{0.783405in}}%
\pgfpathlineto{\pgfqpoint{3.448821in}{0.770801in}}%
\pgfpathlineto{\pgfqpoint{3.444217in}{0.758878in}}%
\pgfpathlineto{\pgfqpoint{3.437042in}{0.747638in}}%
\pgfpathlineto{\pgfqpoint{3.427590in}{0.737089in}}%
\pgfpathlineto{\pgfqpoint{3.416169in}{0.727239in}}%
\pgfpathlineto{\pgfqpoint{3.402848in}{0.718091in}}%
\pgfpathlineto{\pgfqpoint{3.387665in}{0.709646in}}%
\pgfpathlineto{\pgfqpoint{3.361411in}{0.698298in}}%
\pgfpathlineto{\pgfqpoint{3.330888in}{0.688536in}}%
\pgfpathlineto{\pgfqpoint{3.295853in}{0.680359in}}%
\pgfpathlineto{\pgfqpoint{3.255914in}{0.673764in}}%
\pgfpathlineto{\pgfqpoint{3.210872in}{0.668762in}}%
\pgfpathlineto{\pgfqpoint{3.160424in}{0.665389in}}%
\pgfpathlineto{\pgfqpoint{3.103982in}{0.663682in}}%
\pgfpathlineto{\pgfqpoint{3.040944in}{0.663690in}}%
\pgfpathlineto{\pgfqpoint{2.970693in}{0.665473in}}%
\pgfpathlineto{\pgfqpoint{2.864712in}{0.670737in}}%
\pgfpathlineto{\pgfqpoint{2.743250in}{0.679482in}}%
\pgfpathlineto{\pgfqpoint{2.604802in}{0.691939in}}%
\pgfpathlineto{\pgfqpoint{2.448855in}{0.708370in}}%
\pgfpathlineto{\pgfqpoint{2.277074in}{0.729003in}}%
\pgfpathlineto{\pgfqpoint{2.140757in}{0.747273in}}%
\pgfpathlineto{\pgfqpoint{2.001738in}{0.767890in}}%
\pgfpathlineto{\pgfqpoint{1.864465in}{0.790723in}}%
\pgfpathlineto{\pgfqpoint{1.776531in}{0.807025in}}%
\pgfpathlineto{\pgfqpoint{1.693058in}{0.824061in}}%
\pgfpathlineto{\pgfqpoint{1.615133in}{0.841708in}}%
\pgfpathlineto{\pgfqpoint{1.543591in}{0.859838in}}%
\pgfpathlineto{\pgfqpoint{1.479015in}{0.878312in}}%
\pgfpathlineto{\pgfqpoint{1.421748in}{0.896981in}}%
\pgfpathlineto{\pgfqpoint{1.371970in}{0.915689in}}%
\pgfpathlineto{\pgfqpoint{1.329008in}{0.934312in}}%
\pgfpathlineto{\pgfqpoint{1.292068in}{0.952751in}}%
\pgfpathlineto{\pgfqpoint{1.260511in}{0.970914in}}%
\pgfpathlineto{\pgfqpoint{1.233850in}{0.988719in}}%
\pgfpathlineto{\pgfqpoint{1.211750in}{1.006093in}}%
\pgfpathlineto{\pgfqpoint{1.194033in}{1.022972in}}%
\pgfpathlineto{\pgfqpoint{1.180672in}{1.039301in}}%
\pgfpathlineto{\pgfqpoint{1.171306in}{1.055033in}}%
\pgfpathlineto{\pgfqpoint{1.165072in}{1.070137in}}%
\pgfpathlineto{\pgfqpoint{1.161587in}{1.084593in}}%
\pgfpathlineto{\pgfqpoint{1.160560in}{1.098386in}}%
\pgfpathlineto{\pgfqpoint{1.161785in}{1.111504in}}%
\pgfpathlineto{\pgfqpoint{1.165146in}{1.123937in}}%
\pgfpathlineto{\pgfqpoint{1.170615in}{1.135681in}}%
\pgfpathlineto{\pgfqpoint{1.178249in}{1.146733in}}%
\pgfpathlineto{\pgfqpoint{1.188114in}{1.157094in}}%
\pgfpathlineto{\pgfqpoint{1.199991in}{1.166759in}}%
\pgfpathlineto{\pgfqpoint{1.213791in}{1.175724in}}%
\pgfpathlineto{\pgfqpoint{1.229471in}{1.183989in}}%
\pgfpathlineto{\pgfqpoint{1.256496in}{1.195066in}}%
\pgfpathlineto{\pgfqpoint{1.287796in}{1.204557in}}%
\pgfpathlineto{\pgfqpoint{1.323584in}{1.212457in}}%
\pgfpathlineto{\pgfqpoint{1.364210in}{1.218766in}}%
\pgfpathlineto{\pgfqpoint{1.410127in}{1.223482in}}%
\pgfpathlineto{\pgfqpoint{1.461463in}{1.226575in}}%
\pgfpathlineto{\pgfqpoint{1.518815in}{1.227999in}}%
\pgfpathlineto{\pgfqpoint{1.582911in}{1.227700in}}%
\pgfpathlineto{\pgfqpoint{1.680023in}{1.224512in}}%
\pgfpathlineto{\pgfqpoint{1.791795in}{1.217970in}}%
\pgfpathlineto{\pgfqpoint{1.919546in}{1.207861in}}%
\pgfpathlineto{\pgfqpoint{2.064445in}{1.193942in}}%
\pgfpathlineto{\pgfqpoint{2.227359in}{1.175988in}}%
\pgfpathlineto{\pgfqpoint{2.404306in}{1.153774in}}%
\pgfpathlineto{\pgfqpoint{2.541912in}{1.134297in}}%
\pgfpathlineto{\pgfqpoint{2.679758in}{1.112527in}}%
\pgfpathlineto{\pgfqpoint{2.813785in}{1.088690in}}%
\pgfpathlineto{\pgfqpoint{2.898803in}{1.071808in}}%
\pgfpathlineto{\pgfqpoint{2.978673in}{1.054278in}}%
\pgfpathlineto{\pgfqpoint{3.051874in}{1.036242in}}%
\pgfpathlineto{\pgfqpoint{3.117632in}{1.017845in}}%
\pgfpathlineto{\pgfqpoint{3.176222in}{0.999230in}}%
\pgfpathlineto{\pgfqpoint{3.227986in}{0.980532in}}%
\pgfpathlineto{\pgfqpoint{3.273265in}{0.961876in}}%
\pgfpathlineto{\pgfqpoint{3.312398in}{0.943372in}}%
\pgfpathlineto{\pgfqpoint{3.345725in}{0.925124in}}%
\pgfpathlineto{\pgfqpoint{3.373581in}{0.907219in}}%
\pgfpathlineto{\pgfqpoint{3.396302in}{0.889735in}}%
\pgfpathlineto{\pgfqpoint{3.414223in}{0.872739in}}%
\pgfpathlineto{\pgfqpoint{3.427825in}{0.856293in}}%
\pgfpathlineto{\pgfqpoint{3.437810in}{0.840439in}}%
\pgfpathlineto{\pgfqpoint{3.444761in}{0.825200in}}%
\pgfpathlineto{\pgfqpoint{3.449121in}{0.810598in}}%
\pgfpathlineto{\pgfqpoint{3.451192in}{0.796651in}}%
\pgfpathlineto{\pgfqpoint{3.451136in}{0.783370in}}%
\pgfpathlineto{\pgfqpoint{3.448974in}{0.770765in}}%
\pgfpathlineto{\pgfqpoint{3.444589in}{0.758841in}}%
\pgfpathlineto{\pgfqpoint{3.437721in}{0.747599in}}%
\pgfpathlineto{\pgfqpoint{3.428287in}{0.737041in}}%
\pgfpathlineto{\pgfqpoint{3.416845in}{0.727183in}}%
\pgfpathlineto{\pgfqpoint{3.403488in}{0.718026in}}%
\pgfpathlineto{\pgfqpoint{3.388258in}{0.709572in}}%
\pgfpathlineto{\pgfqpoint{3.361920in}{0.698213in}}%
\pgfpathlineto{\pgfqpoint{3.331316in}{0.688441in}}%
\pgfpathlineto{\pgfqpoint{3.296230in}{0.680259in}}%
\pgfpathlineto{\pgfqpoint{3.256305in}{0.673664in}}%
\pgfpathlineto{\pgfqpoint{3.211216in}{0.668663in}}%
\pgfpathlineto{\pgfqpoint{3.160769in}{0.665288in}}%
\pgfpathlineto{\pgfqpoint{3.104343in}{0.663579in}}%
\pgfpathlineto{\pgfqpoint{3.041301in}{0.663585in}}%
\pgfpathlineto{\pgfqpoint{2.971011in}{0.665367in}}%
\pgfpathlineto{\pgfqpoint{2.864934in}{0.670630in}}%
\pgfpathlineto{\pgfqpoint{2.743383in}{0.679377in}}%
\pgfpathlineto{\pgfqpoint{2.604905in}{0.691841in}}%
\pgfpathlineto{\pgfqpoint{2.448851in}{0.708266in}}%
\pgfpathlineto{\pgfqpoint{2.277021in}{0.728908in}}%
\pgfpathlineto{\pgfqpoint{2.140702in}{0.747200in}}%
\pgfpathlineto{\pgfqpoint{2.001671in}{0.767831in}}%
\pgfpathlineto{\pgfqpoint{1.864408in}{0.790646in}}%
\pgfpathlineto{\pgfqpoint{1.776472in}{0.806965in}}%
\pgfpathlineto{\pgfqpoint{1.693009in}{0.824038in}}%
\pgfpathlineto{\pgfqpoint{1.615102in}{0.841715in}}%
\pgfpathlineto{\pgfqpoint{1.543564in}{0.859853in}}%
\pgfpathlineto{\pgfqpoint{1.478941in}{0.878314in}}%
\pgfpathlineto{\pgfqpoint{1.421506in}{0.896962in}}%
\pgfpathlineto{\pgfqpoint{1.371266in}{0.915670in}}%
\pgfpathlineto{\pgfqpoint{1.327957in}{0.934315in}}%
\pgfpathlineto{\pgfqpoint{1.291084in}{0.952776in}}%
\pgfpathlineto{\pgfqpoint{1.260468in}{0.970922in}}%
\pgfpathlineto{\pgfqpoint{1.235195in}{0.988681in}}%
\pgfpathlineto{\pgfqpoint{1.214270in}{1.006001in}}%
\pgfpathlineto{\pgfqpoint{1.196952in}{1.022836in}}%
\pgfpathlineto{\pgfqpoint{1.182758in}{1.039142in}}%
\pgfpathlineto{\pgfqpoint{1.171459in}{1.054882in}}%
\pgfpathlineto{\pgfqpoint{1.163085in}{1.070023in}}%
\pgfpathlineto{\pgfqpoint{1.157919in}{1.084536in}}%
\pgfpathlineto{\pgfqpoint{1.156493in}{1.098398in}}%
\pgfpathlineto{\pgfqpoint{1.158208in}{1.111581in}}%
\pgfpathlineto{\pgfqpoint{1.162176in}{1.124072in}}%
\pgfpathlineto{\pgfqpoint{1.168228in}{1.135864in}}%
\pgfpathlineto{\pgfqpoint{1.176244in}{1.146955in}}%
\pgfpathlineto{\pgfqpoint{1.186146in}{1.157342in}}%
\pgfpathlineto{\pgfqpoint{1.197906in}{1.167024in}}%
\pgfpathlineto{\pgfqpoint{1.211537in}{1.176003in}}%
\pgfpathlineto{\pgfqpoint{1.227101in}{1.184280in}}%
\pgfpathlineto{\pgfqpoint{1.244699in}{1.191860in}}%
\pgfpathlineto{\pgfqpoint{1.274848in}{1.201920in}}%
\pgfpathlineto{\pgfqpoint{1.309493in}{1.210397in}}%
\pgfpathlineto{\pgfqpoint{1.348827in}{1.217277in}}%
\pgfpathlineto{\pgfqpoint{1.393138in}{1.222541in}}%
\pgfpathlineto{\pgfqpoint{1.442809in}{1.226164in}}%
\pgfpathlineto{\pgfqpoint{1.498318in}{1.228117in}}%
\pgfpathlineto{\pgfqpoint{1.560238in}{1.228366in}}%
\pgfpathlineto{\pgfqpoint{1.653756in}{1.225981in}}%
\pgfpathlineto{\pgfqpoint{1.761134in}{1.220337in}}%
\pgfpathlineto{\pgfqpoint{1.885042in}{1.211137in}}%
\pgfpathlineto{\pgfqpoint{2.026636in}{1.198126in}}%
\pgfpathlineto{\pgfqpoint{2.185244in}{1.181102in}}%
\pgfpathlineto{\pgfqpoint{2.358360in}{1.159920in}}%
\pgfpathlineto{\pgfqpoint{2.495176in}{1.141248in}}%
\pgfpathlineto{\pgfqpoint{2.634542in}{1.120174in}}%
\pgfpathlineto{\pgfqpoint{2.771024in}{1.096936in}}%
\pgfpathlineto{\pgfqpoint{2.857944in}{1.080419in}}%
\pgfpathlineto{\pgfqpoint{2.940194in}{1.063216in}}%
\pgfpathlineto{\pgfqpoint{3.016800in}{1.045444in}}%
\pgfpathlineto{\pgfqpoint{3.087002in}{1.027229in}}%
\pgfpathlineto{\pgfqpoint{3.150253in}{1.008699in}}%
\pgfpathlineto{\pgfqpoint{3.206217in}{0.989989in}}%
\pgfpathlineto{\pgfqpoint{3.254774in}{0.971238in}}%
\pgfpathlineto{\pgfqpoint{3.296053in}{0.952599in}}%
\pgfpathlineto{\pgfqpoint{3.330853in}{0.934199in}}%
\pgfpathlineto{\pgfqpoint{3.360189in}{0.916115in}}%
\pgfpathlineto{\pgfqpoint{3.384861in}{0.898416in}}%
\pgfpathlineto{\pgfqpoint{3.405445in}{0.881166in}}%
\pgfpathlineto{\pgfqpoint{3.422298in}{0.864418in}}%
\pgfpathlineto{\pgfqpoint{3.435557in}{0.848223in}}%
\pgfpathlineto{\pgfqpoint{3.445133in}{0.832619in}}%
\pgfpathlineto{\pgfqpoint{3.450816in}{0.817644in}}%
\pgfpathlineto{\pgfqpoint{3.453447in}{0.803326in}}%
\pgfpathlineto{\pgfqpoint{3.453561in}{0.789680in}}%
\pgfpathlineto{\pgfqpoint{3.451393in}{0.776718in}}%
\pgfpathlineto{\pgfqpoint{3.447112in}{0.764446in}}%
\pgfpathlineto{\pgfqpoint{3.440829in}{0.752869in}}%
\pgfpathlineto{\pgfqpoint{3.432589in}{0.741992in}}%
\pgfpathlineto{\pgfqpoint{3.422379in}{0.731813in}}%
\pgfpathlineto{\pgfqpoint{3.410119in}{0.722330in}}%
\pgfpathlineto{\pgfqpoint{3.395804in}{0.713544in}}%
\pgfpathlineto{\pgfqpoint{3.379556in}{0.705455in}}%
\pgfpathlineto{\pgfqpoint{3.351592in}{0.694638in}}%
\pgfpathlineto{\pgfqpoint{3.319271in}{0.685408in}}%
\pgfpathlineto{\pgfqpoint{3.282434in}{0.677773in}}%
\pgfpathlineto{\pgfqpoint{3.240807in}{0.671746in}}%
\pgfpathlineto{\pgfqpoint{3.193996in}{0.667339in}}%
\pgfpathlineto{\pgfqpoint{3.141493in}{0.664566in}}%
\pgfpathlineto{\pgfqpoint{3.083007in}{0.663445in}}%
\pgfpathlineto{\pgfqpoint{3.018052in}{0.664038in}}%
\pgfpathlineto{\pgfqpoint{2.919409in}{0.667646in}}%
\pgfpathlineto{\pgfqpoint{2.805229in}{0.674665in}}%
\pgfpathlineto{\pgfqpoint{2.674411in}{0.685307in}}%
\pgfpathlineto{\pgfqpoint{2.526755in}{0.699786in}}%
\pgfpathlineto{\pgfqpoint{2.362964in}{0.718320in}}%
\pgfpathlineto{\pgfqpoint{2.184659in}{0.741143in}}%
\pgfpathlineto{\pgfqpoint{2.045215in}{0.761119in}}%
\pgfpathlineto{\pgfqpoint{1.906874in}{0.783319in}}%
\pgfpathlineto{\pgfqpoint{1.774423in}{0.807462in}}%
\pgfpathlineto{\pgfqpoint{1.691371in}{0.824475in}}%
\pgfpathlineto{\pgfqpoint{1.613655in}{0.842095in}}%
\pgfpathlineto{\pgfqpoint{1.542060in}{0.860202in}}%
\pgfpathlineto{\pgfqpoint{1.477205in}{0.878665in}}%
\pgfpathlineto{\pgfqpoint{1.419537in}{0.897345in}}%
\pgfpathlineto{\pgfqpoint{1.369336in}{0.916092in}}%
\pgfpathlineto{\pgfqpoint{1.326485in}{0.934748in}}%
\pgfpathlineto{\pgfqpoint{1.290077in}{0.953200in}}%
\pgfpathlineto{\pgfqpoint{1.259237in}{0.971360in}}%
\pgfpathlineto{\pgfqpoint{1.233271in}{0.989153in}}%
\pgfpathlineto{\pgfqpoint{1.211669in}{1.006510in}}%
\pgfpathlineto{\pgfqpoint{1.194104in}{1.023371in}}%
\pgfpathlineto{\pgfqpoint{1.180431in}{1.039685in}}%
\pgfpathlineto{\pgfqpoint{1.170690in}{1.055408in}}%
\pgfpathlineto{\pgfqpoint{1.164483in}{1.070505in}}%
\pgfpathlineto{\pgfqpoint{1.161095in}{1.084951in}}%
\pgfpathlineto{\pgfqpoint{1.160234in}{1.098732in}}%
\pgfpathlineto{\pgfqpoint{1.161679in}{1.111836in}}%
\pgfpathlineto{\pgfqpoint{1.165277in}{1.124255in}}%
\pgfpathlineto{\pgfqpoint{1.170945in}{1.135981in}}%
\pgfpathlineto{\pgfqpoint{1.178666in}{1.147013in}}%
\pgfpathlineto{\pgfqpoint{1.188495in}{1.157351in}}%
\pgfpathlineto{\pgfqpoint{1.200395in}{1.166994in}}%
\pgfpathlineto{\pgfqpoint{1.214242in}{1.175940in}}%
\pgfpathlineto{\pgfqpoint{1.229989in}{1.184186in}}%
\pgfpathlineto{\pgfqpoint{1.257139in}{1.195238in}}%
\pgfpathlineto{\pgfqpoint{1.288573in}{1.204703in}}%
\pgfpathlineto{\pgfqpoint{1.324473in}{1.212575in}}%
\pgfpathlineto{\pgfqpoint{1.365153in}{1.218850in}}%
\pgfpathlineto{\pgfqpoint{1.411062in}{1.223523in}}%
\pgfpathlineto{\pgfqpoint{1.462563in}{1.226582in}}%
\pgfpathlineto{\pgfqpoint{1.519918in}{1.227979in}}%
\pgfpathlineto{\pgfqpoint{1.583994in}{1.227655in}}%
\pgfpathlineto{\pgfqpoint{1.681220in}{1.224428in}}%
\pgfpathlineto{\pgfqpoint{1.793318in}{1.217840in}}%
\pgfpathlineto{\pgfqpoint{1.921477in}{1.207681in}}%
\pgfpathlineto{\pgfqpoint{2.066529in}{1.193715in}}%
\pgfpathlineto{\pgfqpoint{2.229338in}{1.175667in}}%
\pgfpathlineto{\pgfqpoint{2.408427in}{1.153293in}}%
\pgfpathlineto{\pgfqpoint{2.547042in}{1.133732in}}%
\pgfpathlineto{\pgfqpoint{2.684362in}{1.111946in}}%
\pgfpathlineto{\pgfqpoint{2.816430in}{1.088157in}}%
\pgfpathlineto{\pgfqpoint{2.899820in}{1.071328in}}%
\pgfpathlineto{\pgfqpoint{2.978410in}{1.053846in}}%
\pgfpathlineto{\pgfqpoint{3.051386in}{1.035833in}}%
\pgfpathlineto{\pgfqpoint{3.118027in}{1.017424in}}%
\pgfpathlineto{\pgfqpoint{3.177710in}{0.998770in}}%
\pgfpathlineto{\pgfqpoint{3.229905in}{0.980037in}}%
\pgfpathlineto{\pgfqpoint{3.274671in}{0.961377in}}%
\pgfpathlineto{\pgfqpoint{3.312975in}{0.942884in}}%
\pgfpathlineto{\pgfqpoint{3.345411in}{0.924657in}}%
\pgfpathlineto{\pgfqpoint{3.372516in}{0.906784in}}%
\pgfpathlineto{\pgfqpoint{3.394767in}{0.889340in}}%
\pgfpathlineto{\pgfqpoint{3.412592in}{0.872391in}}%
\pgfpathlineto{\pgfqpoint{3.426468in}{0.855986in}}%
\pgfpathlineto{\pgfqpoint{3.436851in}{0.840164in}}%
\pgfpathlineto{\pgfqpoint{3.444102in}{0.824955in}}%
\pgfpathlineto{\pgfqpoint{3.448503in}{0.810383in}}%
\pgfpathlineto{\pgfqpoint{3.450262in}{0.796468in}}%
\pgfpathlineto{\pgfqpoint{3.449505in}{0.783223in}}%
\pgfpathlineto{\pgfqpoint{3.446336in}{0.770657in}}%
\pgfpathlineto{\pgfqpoint{3.440971in}{0.758777in}}%
\pgfpathlineto{\pgfqpoint{3.433588in}{0.747588in}}%
\pgfpathlineto{\pgfqpoint{3.424316in}{0.737096in}}%
\pgfpathlineto{\pgfqpoint{3.413229in}{0.727303in}}%
\pgfpathlineto{\pgfqpoint{3.400351in}{0.718209in}}%
\pgfpathlineto{\pgfqpoint{3.385654in}{0.709814in}}%
\pgfpathlineto{\pgfqpoint{3.369057in}{0.702115in}}%
\pgfpathlineto{\pgfqpoint{3.340297in}{0.691861in}}%
\pgfpathlineto{\pgfqpoint{3.306631in}{0.683158in}}%
\pgfpathlineto{\pgfqpoint{3.268273in}{0.676035in}}%
\pgfpathlineto{\pgfqpoint{3.224954in}{0.670512in}}%
\pgfpathlineto{\pgfqpoint{3.176309in}{0.666615in}}%
\pgfpathlineto{\pgfqpoint{3.121889in}{0.664376in}}%
\pgfpathlineto{\pgfqpoint{3.061168in}{0.663839in}}%
\pgfpathlineto{\pgfqpoint{2.993539in}{0.665051in}}%
\pgfpathlineto{\pgfqpoint{2.891443in}{0.669486in}}%
\pgfpathlineto{\pgfqpoint{2.774224in}{0.677327in}}%
\pgfpathlineto{\pgfqpoint{2.640035in}{0.688838in}}%
\pgfpathlineto{\pgfqpoint{2.488406in}{0.704262in}}%
\pgfpathlineto{\pgfqpoint{2.320586in}{0.723810in}}%
\pgfpathlineto{\pgfqpoint{2.139831in}{0.747637in}}%
\pgfpathlineto{\pgfqpoint{2.000939in}{0.768258in}}%
\pgfpathlineto{\pgfqpoint{1.864321in}{0.791051in}}%
\pgfpathlineto{\pgfqpoint{1.734346in}{0.815748in}}%
\pgfpathlineto{\pgfqpoint{1.653440in}{0.833096in}}%
\pgfpathlineto{\pgfqpoint{1.578500in}{0.851001in}}%
\pgfpathlineto{\pgfqpoint{1.510602in}{0.869313in}}%
\pgfpathlineto{\pgfqpoint{1.450049in}{0.887883in}}%
\pgfpathlineto{\pgfqpoint{1.396553in}{0.906569in}}%
\pgfpathlineto{\pgfqpoint{1.349786in}{0.925244in}}%
\pgfpathlineto{\pgfqpoint{1.309398in}{0.943790in}}%
\pgfpathlineto{\pgfqpoint{1.275018in}{0.962103in}}%
\pgfpathlineto{\pgfqpoint{1.246250in}{0.980088in}}%
\pgfpathlineto{\pgfqpoint{1.222676in}{0.997664in}}%
\pgfpathlineto{\pgfqpoint{1.203839in}{1.014760in}}%
\pgfpathlineto{\pgfqpoint{1.189140in}{1.031319in}}%
\pgfpathlineto{\pgfqpoint{1.177987in}{1.047305in}}%
\pgfpathlineto{\pgfqpoint{1.169919in}{1.062686in}}%
\pgfpathlineto{\pgfqpoint{1.164598in}{1.077439in}}%
\pgfpathlineto{\pgfqpoint{1.161814in}{1.091542in}}%
\pgfpathlineto{\pgfqpoint{1.161483in}{1.104981in}}%
\pgfpathlineto{\pgfqpoint{1.163650in}{1.117746in}}%
\pgfpathlineto{\pgfqpoint{1.168359in}{1.129829in}}%
\pgfpathlineto{\pgfqpoint{1.175255in}{1.141223in}}%
\pgfpathlineto{\pgfqpoint{1.184172in}{1.151921in}}%
\pgfpathlineto{\pgfqpoint{1.195009in}{1.161920in}}%
\pgfpathlineto{\pgfqpoint{1.207700in}{1.171219in}}%
\pgfpathlineto{\pgfqpoint{1.222219in}{1.179815in}}%
\pgfpathlineto{\pgfqpoint{1.247457in}{1.191395in}}%
\pgfpathlineto{\pgfqpoint{1.277034in}{1.201400in}}%
\pgfpathlineto{\pgfqpoint{1.311330in}{1.209840in}}%
\pgfpathlineto{\pgfqpoint{1.350477in}{1.216707in}}%
\pgfpathlineto{\pgfqpoint{1.394617in}{1.221974in}}%
\pgfpathlineto{\pgfqpoint{1.444163in}{1.225615in}}%
\pgfpathlineto{\pgfqpoint{1.499593in}{1.227593in}}%
\pgfpathlineto{\pgfqpoint{1.561446in}{1.227862in}}%
\pgfpathlineto{\pgfqpoint{1.630323in}{1.226367in}}%
\pgfpathlineto{\pgfqpoint{1.734243in}{1.221517in}}%
\pgfpathlineto{\pgfqpoint{1.853541in}{1.213233in}}%
\pgfpathlineto{\pgfqpoint{1.989754in}{1.201264in}}%
\pgfpathlineto{\pgfqpoint{2.143454in}{1.185351in}}%
\pgfpathlineto{\pgfqpoint{2.313245in}{1.165279in}}%
\pgfpathlineto{\pgfqpoint{2.448713in}{1.147423in}}%
\pgfpathlineto{\pgfqpoint{2.587404in}{1.127202in}}%
\pgfpathlineto{\pgfqpoint{2.725094in}{1.104758in}}%
\pgfpathlineto{\pgfqpoint{2.856918in}{1.080342in}}%
\pgfpathlineto{\pgfqpoint{2.938922in}{1.063120in}}%
\pgfpathlineto{\pgfqpoint{3.015032in}{1.045304in}}%
\pgfpathlineto{\pgfqpoint{3.084671in}{1.027060in}}%
\pgfpathlineto{\pgfqpoint{3.147457in}{1.008538in}}%
\pgfpathlineto{\pgfqpoint{3.203200in}{0.989879in}}%
\pgfpathlineto{\pgfqpoint{3.251907in}{0.971209in}}%
\pgfpathlineto{\pgfqpoint{3.293776in}{0.952645in}}%
\pgfpathlineto{\pgfqpoint{3.329203in}{0.934291in}}%
\pgfpathlineto{\pgfqpoint{3.358774in}{0.916239in}}%
\pgfpathlineto{\pgfqpoint{3.383273in}{0.898570in}}%
\pgfpathlineto{\pgfqpoint{3.403080in}{0.881366in}}%
\pgfpathlineto{\pgfqpoint{3.418511in}{0.864691in}}%
\pgfpathlineto{\pgfqpoint{3.430390in}{0.848576in}}%
\pgfpathlineto{\pgfqpoint{3.439349in}{0.833050in}}%
\pgfpathlineto{\pgfqpoint{3.445823in}{0.818137in}}%
\pgfpathlineto{\pgfqpoint{3.450052in}{0.803858in}}%
\pgfpathlineto{\pgfqpoint{3.452080in}{0.790227in}}%
\pgfpathlineto{\pgfqpoint{3.451758in}{0.777258in}}%
\pgfpathlineto{\pgfqpoint{3.448737in}{0.764956in}}%
\pgfpathlineto{\pgfqpoint{3.442587in}{0.753328in}}%
\pgfpathlineto{\pgfqpoint{3.434082in}{0.742394in}}%
\pgfpathlineto{\pgfqpoint{3.423603in}{0.732161in}}%
\pgfpathlineto{\pgfqpoint{3.411226in}{0.722632in}}%
\pgfpathlineto{\pgfqpoint{3.396995in}{0.713807in}}%
\pgfpathlineto{\pgfqpoint{3.380921in}{0.705688in}}%
\pgfpathlineto{\pgfqpoint{3.353308in}{0.694833in}}%
\pgfpathlineto{\pgfqpoint{3.321309in}{0.685562in}}%
\pgfpathlineto{\pgfqpoint{3.284573in}{0.677869in}}%
\pgfpathlineto{\pgfqpoint{3.242931in}{0.671763in}}%
\pgfpathlineto{\pgfqpoint{3.196141in}{0.667270in}}%
\pgfpathlineto{\pgfqpoint{3.143720in}{0.664422in}}%
\pgfpathlineto{\pgfqpoint{3.085134in}{0.663259in}}%
\pgfpathlineto{\pgfqpoint{3.019807in}{0.663832in}}%
\pgfpathlineto{\pgfqpoint{2.921135in}{0.667408in}}%
\pgfpathlineto{\pgfqpoint{2.807764in}{0.674357in}}%
\pgfpathlineto{\pgfqpoint{2.678027in}{0.684891in}}%
\pgfpathlineto{\pgfqpoint{2.530808in}{0.699276in}}%
\pgfpathlineto{\pgfqpoint{2.366619in}{0.717754in}}%
\pgfpathlineto{\pgfqpoint{2.188562in}{0.740482in}}%
\pgfpathlineto{\pgfqpoint{2.049887in}{0.760314in}}%
\pgfpathlineto{\pgfqpoint{1.911531in}{0.782406in}}%
\pgfpathlineto{\pgfqpoint{1.778164in}{0.806523in}}%
\pgfpathlineto{\pgfqpoint{1.694485in}{0.823563in}}%
\pgfpathlineto{\pgfqpoint{1.616722in}{0.841246in}}%
\pgfpathlineto{\pgfqpoint{1.545467in}{0.859412in}}%
\pgfpathlineto{\pgfqpoint{1.481006in}{0.877901in}}%
\pgfpathlineto{\pgfqpoint{1.423478in}{0.896565in}}%
\pgfpathlineto{\pgfqpoint{1.372880in}{0.915271in}}%
\pgfpathlineto{\pgfqpoint{1.329061in}{0.933897in}}%
\pgfpathlineto{\pgfqpoint{1.291726in}{0.952335in}}%
\pgfpathlineto{\pgfqpoint{1.260437in}{0.970489in}}%
\pgfpathlineto{\pgfqpoint{1.234607in}{0.988278in}}%
\pgfpathlineto{\pgfqpoint{1.213508in}{1.005632in}}%
\pgfpathlineto{\pgfqpoint{1.196845in}{1.022478in}}%
\pgfpathlineto{\pgfqpoint{1.184246in}{1.038761in}}%
\pgfpathlineto{\pgfqpoint{1.174928in}{1.054455in}}%
\pgfpathlineto{\pgfqpoint{1.168294in}{1.069537in}}%
\pgfpathlineto{\pgfqpoint{1.163932in}{1.083986in}}%
\pgfpathlineto{\pgfqpoint{1.161614in}{1.097788in}}%
\pgfpathlineto{\pgfqpoint{1.161297in}{1.110930in}}%
\pgfpathlineto{\pgfqpoint{1.163124in}{1.123405in}}%
\pgfpathlineto{\pgfqpoint{1.167422in}{1.135209in}}%
\pgfpathlineto{\pgfqpoint{1.174705in}{1.146340in}}%
\pgfpathlineto{\pgfqpoint{1.184660in}{1.156784in}}%
\pgfpathlineto{\pgfqpoint{1.196566in}{1.166525in}}%
\pgfpathlineto{\pgfqpoint{1.210366in}{1.175562in}}%
\pgfpathlineto{\pgfqpoint{1.226028in}{1.183892in}}%
\pgfpathlineto{\pgfqpoint{1.253012in}{1.195063in}}%
\pgfpathlineto{\pgfqpoint{1.284286in}{1.204643in}}%
\pgfpathlineto{\pgfqpoint{1.320095in}{1.212631in}}%
\pgfpathlineto{\pgfqpoint{1.360816in}{1.219030in}}%
\pgfpathlineto{\pgfqpoint{1.406641in}{1.223828in}}%
\pgfpathlineto{\pgfqpoint{1.457929in}{1.226990in}}%
\pgfpathlineto{\pgfqpoint{1.515291in}{1.228477in}}%
\pgfpathlineto{\pgfqpoint{1.579342in}{1.228239in}}%
\pgfpathlineto{\pgfqpoint{1.650707in}{1.226215in}}%
\pgfpathlineto{\pgfqpoint{1.758329in}{1.220609in}}%
\pgfpathlineto{\pgfqpoint{1.881594in}{1.211495in}}%
\pgfpathlineto{\pgfqpoint{2.021970in}{1.198638in}}%
\pgfpathlineto{\pgfqpoint{2.179828in}{1.181779in}}%
\pgfpathlineto{\pgfqpoint{2.353190in}{1.160692in}}%
\pgfpathlineto{\pgfqpoint{2.490259in}{1.142077in}}%
\pgfpathlineto{\pgfqpoint{2.629464in}{1.121134in}}%
\pgfpathlineto{\pgfqpoint{2.766192in}{1.098000in}}%
\pgfpathlineto{\pgfqpoint{2.853434in}{1.081511in}}%
\pgfpathlineto{\pgfqpoint{2.935955in}{1.064314in}}%
\pgfpathlineto{\pgfqpoint{3.012704in}{1.046540in}}%
\pgfpathlineto{\pgfqpoint{3.082907in}{1.028320in}}%
\pgfpathlineto{\pgfqpoint{3.146069in}{1.009789in}}%
\pgfpathlineto{\pgfqpoint{3.201974in}{0.991085in}}%
\pgfpathlineto{\pgfqpoint{3.250645in}{0.972353in}}%
\pgfpathlineto{\pgfqpoint{3.292031in}{0.953753in}}%
\pgfpathlineto{\pgfqpoint{3.327126in}{0.935380in}}%
\pgfpathlineto{\pgfqpoint{3.356989in}{0.917306in}}%
\pgfpathlineto{\pgfqpoint{3.382421in}{0.899597in}}%
\pgfpathlineto{\pgfqpoint{3.403960in}{0.882314in}}%
\pgfpathlineto{\pgfqpoint{3.421887in}{0.865515in}}%
\pgfpathlineto{\pgfqpoint{3.436221in}{0.849250in}}%
\pgfpathlineto{\pgfqpoint{3.446722in}{0.833568in}}%
\pgfpathlineto{\pgfqpoint{3.452899in}{0.818511in}}%
\pgfpathlineto{\pgfqpoint{3.455527in}{0.804113in}}%
\pgfpathlineto{\pgfqpoint{3.455629in}{0.790393in}}%
\pgfpathlineto{\pgfqpoint{3.453438in}{0.777359in}}%
\pgfpathlineto{\pgfqpoint{3.449132in}{0.765019in}}%
\pgfpathlineto{\pgfqpoint{3.442825in}{0.753378in}}%
\pgfpathlineto{\pgfqpoint{3.434576in}{0.742439in}}%
\pgfpathlineto{\pgfqpoint{3.424382in}{0.732202in}}%
\pgfpathlineto{\pgfqpoint{3.412182in}{0.722665in}}%
\pgfpathlineto{\pgfqpoint{3.397921in}{0.713825in}}%
\pgfpathlineto{\pgfqpoint{3.381719in}{0.705686in}}%
\pgfpathlineto{\pgfqpoint{3.353824in}{0.694794in}}%
\pgfpathlineto{\pgfqpoint{3.321577in}{0.685490in}}%
\pgfpathlineto{\pgfqpoint{3.284825in}{0.677785in}}%
\pgfpathlineto{\pgfqpoint{3.243302in}{0.671690in}}%
\pgfpathlineto{\pgfqpoint{3.196623in}{0.667219in}}%
\pgfpathlineto{\pgfqpoint{3.144291in}{0.664385in}}%
\pgfpathlineto{\pgfqpoint{3.085909in}{0.663205in}}%
\pgfpathlineto{\pgfqpoint{3.021174in}{0.663732in}}%
\pgfpathlineto{\pgfqpoint{2.922931in}{0.667246in}}%
\pgfpathlineto{\pgfqpoint{2.809123in}{0.674173in}}%
\pgfpathlineto{\pgfqpoint{2.678582in}{0.684727in}}%
\pgfpathlineto{\pgfqpoint{2.531136in}{0.699120in}}%
\pgfpathlineto{\pgfqpoint{2.367615in}{0.717560in}}%
\pgfpathlineto{\pgfqpoint{2.189847in}{0.740260in}}%
\pgfpathlineto{\pgfqpoint{2.050494in}{0.760150in}}%
\pgfpathlineto{\pgfqpoint{1.911879in}{0.782288in}}%
\pgfpathlineto{\pgfqpoint{1.778933in}{0.806388in}}%
\pgfpathlineto{\pgfqpoint{1.695489in}{0.823383in}}%
\pgfpathlineto{\pgfqpoint{1.617366in}{0.840991in}}%
\pgfpathlineto{\pgfqpoint{1.545375in}{0.859092in}}%
\pgfpathlineto{\pgfqpoint{1.480150in}{0.877555in}}%
\pgfpathlineto{\pgfqpoint{1.422154in}{0.896240in}}%
\pgfpathlineto{\pgfqpoint{1.371672in}{0.914995in}}%
\pgfpathlineto{\pgfqpoint{1.328526in}{0.933663in}}%
\pgfpathlineto{\pgfqpoint{1.291793in}{0.952134in}}%
\pgfpathlineto{\pgfqpoint{1.260616in}{0.970319in}}%
\pgfpathlineto{\pgfqpoint{1.234320in}{0.988142in}}%
\pgfpathlineto{\pgfqpoint{1.212414in}{1.005533in}}%
\pgfpathlineto{\pgfqpoint{1.194588in}{1.022431in}}%
\pgfpathlineto{\pgfqpoint{1.180720in}{1.038784in}}%
\pgfpathlineto{\pgfqpoint{1.170864in}{1.054547in}}%
\pgfpathlineto{\pgfqpoint{1.164525in}{1.069684in}}%
\pgfpathlineto{\pgfqpoint{1.161012in}{1.084171in}}%
\pgfpathlineto{\pgfqpoint{1.160029in}{1.097993in}}%
\pgfpathlineto{\pgfqpoint{1.161351in}{1.111138in}}%
\pgfpathlineto{\pgfqpoint{1.164825in}{1.123597in}}%
\pgfpathlineto{\pgfqpoint{1.170369in}{1.135365in}}%
\pgfpathlineto{\pgfqpoint{1.177973in}{1.146438in}}%
\pgfpathlineto{\pgfqpoint{1.187698in}{1.156817in}}%
\pgfpathlineto{\pgfqpoint{1.199502in}{1.166502in}}%
\pgfpathlineto{\pgfqpoint{1.213253in}{1.175488in}}%
\pgfpathlineto{\pgfqpoint{1.228903in}{1.183775in}}%
\pgfpathlineto{\pgfqpoint{1.255903in}{1.194887in}}%
\pgfpathlineto{\pgfqpoint{1.287182in}{1.204412in}}%
\pgfpathlineto{\pgfqpoint{1.322919in}{1.212345in}}%
\pgfpathlineto{\pgfqpoint{1.363427in}{1.218680in}}%
\pgfpathlineto{\pgfqpoint{1.409155in}{1.223414in}}%
\pgfpathlineto{\pgfqpoint{1.460451in}{1.226535in}}%
\pgfpathlineto{\pgfqpoint{1.517594in}{1.227994in}}%
\pgfpathlineto{\pgfqpoint{1.581436in}{1.227733in}}%
\pgfpathlineto{\pgfqpoint{1.678299in}{1.224595in}}%
\pgfpathlineto{\pgfqpoint{1.789971in}{1.218102in}}%
\pgfpathlineto{\pgfqpoint{1.917663in}{1.208044in}}%
\pgfpathlineto{\pgfqpoint{2.062254in}{1.194185in}}%
\pgfpathlineto{\pgfqpoint{2.224783in}{1.176245in}}%
\pgfpathlineto{\pgfqpoint{2.403790in}{1.153978in}}%
\pgfpathlineto{\pgfqpoint{2.542388in}{1.134497in}}%
\pgfpathlineto{\pgfqpoint{2.679737in}{1.112788in}}%
\pgfpathlineto{\pgfqpoint{2.811900in}{1.089068in}}%
\pgfpathlineto{\pgfqpoint{2.895408in}{1.072279in}}%
\pgfpathlineto{\pgfqpoint{2.974172in}{1.054831in}}%
\pgfpathlineto{\pgfqpoint{3.047394in}{1.036843in}}%
\pgfpathlineto{\pgfqpoint{3.114371in}{1.018449in}}%
\pgfpathlineto{\pgfqpoint{3.174497in}{0.999799in}}%
\pgfpathlineto{\pgfqpoint{3.227268in}{0.981055in}}%
\pgfpathlineto{\pgfqpoint{3.272490in}{0.962383in}}%
\pgfpathlineto{\pgfqpoint{3.311127in}{0.943876in}}%
\pgfpathlineto{\pgfqpoint{3.343846in}{0.925631in}}%
\pgfpathlineto{\pgfqpoint{3.371193in}{0.907737in}}%
\pgfpathlineto{\pgfqpoint{3.393664in}{0.890270in}}%
\pgfpathlineto{\pgfqpoint{3.411703in}{0.873293in}}%
\pgfpathlineto{\pgfqpoint{3.425786in}{0.856858in}}%
\pgfpathlineto{\pgfqpoint{3.436356in}{0.841003in}}%
\pgfpathlineto{\pgfqpoint{3.443770in}{0.825761in}}%
\pgfpathlineto{\pgfqpoint{3.448312in}{0.811155in}}%
\pgfpathlineto{\pgfqpoint{3.450192in}{0.797205in}}%
\pgfpathlineto{\pgfqpoint{3.449520in}{0.783924in}}%
\pgfpathlineto{\pgfqpoint{3.446446in}{0.771321in}}%
\pgfpathlineto{\pgfqpoint{3.441201in}{0.759404in}}%
\pgfpathlineto{\pgfqpoint{3.433963in}{0.748180in}}%
\pgfpathlineto{\pgfqpoint{3.424853in}{0.737651in}}%
\pgfpathlineto{\pgfqpoint{3.413940in}{0.727821in}}%
\pgfpathlineto{\pgfqpoint{3.401233in}{0.718690in}}%
\pgfpathlineto{\pgfqpoint{3.386690in}{0.710257in}}%
\pgfpathlineto{\pgfqpoint{3.370210in}{0.702517in}}%
\pgfpathlineto{\pgfqpoint{3.341507in}{0.692198in}}%
\pgfpathlineto{\pgfqpoint{3.307985in}{0.683434in}}%
\pgfpathlineto{\pgfqpoint{3.269794in}{0.676252in}}%
\pgfpathlineto{\pgfqpoint{3.226656in}{0.670671in}}%
\pgfpathlineto{\pgfqpoint{3.178213in}{0.666716in}}%
\pgfpathlineto{\pgfqpoint{3.124020in}{0.664419in}}%
\pgfpathlineto{\pgfqpoint{3.063550in}{0.663821in}}%
\pgfpathlineto{\pgfqpoint{2.996193in}{0.664967in}}%
\pgfpathlineto{\pgfqpoint{2.894536in}{0.669304in}}%
\pgfpathlineto{\pgfqpoint{2.777768in}{0.677045in}}%
\pgfpathlineto{\pgfqpoint{2.643976in}{0.688456in}}%
\pgfpathlineto{\pgfqpoint{2.492761in}{0.703774in}}%
\pgfpathlineto{\pgfqpoint{2.325402in}{0.723205in}}%
\pgfpathlineto{\pgfqpoint{2.144907in}{0.746922in}}%
\pgfpathlineto{\pgfqpoint{2.005777in}{0.767478in}}%
\pgfpathlineto{\pgfqpoint{1.868966in}{0.790204in}}%
\pgfpathlineto{\pgfqpoint{1.738912in}{0.814826in}}%
\pgfpathlineto{\pgfqpoint{1.657897in}{0.832126in}}%
\pgfpathlineto{\pgfqpoint{1.582620in}{0.849992in}}%
\pgfpathlineto{\pgfqpoint{1.513975in}{0.868286in}}%
\pgfpathlineto{\pgfqpoint{1.452726in}{0.886854in}}%
\pgfpathlineto{\pgfqpoint{1.398991in}{0.905540in}}%
\pgfpathlineto{\pgfqpoint{1.352170in}{0.924216in}}%
\pgfpathlineto{\pgfqpoint{1.311685in}{0.942768in}}%
\pgfpathlineto{\pgfqpoint{1.277030in}{0.961094in}}%
\pgfpathlineto{\pgfqpoint{1.247770in}{0.979101in}}%
\pgfpathlineto{\pgfqpoint{1.223544in}{0.996708in}}%
\pgfpathlineto{\pgfqpoint{1.204064in}{1.013845in}}%
\pgfpathlineto{\pgfqpoint{1.189034in}{1.030452in}}%
\pgfpathlineto{\pgfqpoint{1.177754in}{1.046481in}}%
\pgfpathlineto{\pgfqpoint{1.169703in}{1.061904in}}%
\pgfpathlineto{\pgfqpoint{1.164497in}{1.076695in}}%
\pgfpathlineto{\pgfqpoint{1.161854in}{1.090836in}}%
\pgfpathlineto{\pgfqpoint{1.161594in}{1.104311in}}%
\pgfpathlineto{\pgfqpoint{1.163640in}{1.117108in}}%
\pgfpathlineto{\pgfqpoint{1.168016in}{1.129223in}}%
\pgfpathlineto{\pgfqpoint{1.174737in}{1.140651in}}%
\pgfpathlineto{\pgfqpoint{1.183549in}{1.151385in}}%
\pgfpathlineto{\pgfqpoint{1.194330in}{1.161423in}}%
\pgfpathlineto{\pgfqpoint{1.206999in}{1.170761in}}%
\pgfpathlineto{\pgfqpoint{1.221512in}{1.179397in}}%
\pgfpathlineto{\pgfqpoint{1.246724in}{1.191035in}}%
\pgfpathlineto{\pgfqpoint{1.276172in}{1.201091in}}%
\pgfpathlineto{\pgfqpoint{1.310137in}{1.209571in}}%
\pgfpathlineto{\pgfqpoint{1.349045in}{1.216481in}}%
\pgfpathlineto{\pgfqpoint{1.392984in}{1.221802in}}%
\pgfpathlineto{\pgfqpoint{1.442279in}{1.225501in}}%
\pgfpathlineto{\pgfqpoint{1.497458in}{1.227542in}}%
\pgfpathlineto{\pgfqpoint{1.559081in}{1.227879in}}%
\pgfpathlineto{\pgfqpoint{1.627750in}{1.226455in}}%
\pgfpathlineto{\pgfqpoint{1.731372in}{1.221697in}}%
\pgfpathlineto{\pgfqpoint{1.850262in}{1.213497in}}%
\pgfpathlineto{\pgfqpoint{1.986010in}{1.201626in}}%
\pgfpathlineto{\pgfqpoint{2.099313in}{1.190154in}}%
\pgfpathlineto{\pgfqpoint{2.099313in}{1.190154in}}%
\pgfusepath{stroke}%
\end{pgfscope}%
\begin{pgfscope}%
\pgfsetrectcap%
\pgfsetmiterjoin%
\pgfsetlinewidth{0.803000pt}%
\definecolor{currentstroke}{rgb}{0.000000,0.000000,0.000000}%
\pgfsetstrokecolor{currentstroke}%
\pgfsetdash{}{0pt}%
\pgfpathmoveto{\pgfqpoint{0.562500in}{0.275000in}}%
\pgfpathlineto{\pgfqpoint{0.562500in}{2.200000in}}%
\pgfusepath{stroke}%
\end{pgfscope}%
\begin{pgfscope}%
\pgfsetrectcap%
\pgfsetmiterjoin%
\pgfsetlinewidth{0.803000pt}%
\definecolor{currentstroke}{rgb}{0.000000,0.000000,0.000000}%
\pgfsetstrokecolor{currentstroke}%
\pgfsetdash{}{0pt}%
\pgfpathmoveto{\pgfqpoint{4.050000in}{0.275000in}}%
\pgfpathlineto{\pgfqpoint{4.050000in}{2.200000in}}%
\pgfusepath{stroke}%
\end{pgfscope}%
\begin{pgfscope}%
\pgfsetrectcap%
\pgfsetmiterjoin%
\pgfsetlinewidth{0.803000pt}%
\definecolor{currentstroke}{rgb}{0.000000,0.000000,0.000000}%
\pgfsetstrokecolor{currentstroke}%
\pgfsetdash{}{0pt}%
\pgfpathmoveto{\pgfqpoint{0.562500in}{0.275000in}}%
\pgfpathlineto{\pgfqpoint{4.050000in}{0.275000in}}%
\pgfusepath{stroke}%
\end{pgfscope}%
\begin{pgfscope}%
\pgfsetrectcap%
\pgfsetmiterjoin%
\pgfsetlinewidth{0.803000pt}%
\definecolor{currentstroke}{rgb}{0.000000,0.000000,0.000000}%
\pgfsetstrokecolor{currentstroke}%
\pgfsetdash{}{0pt}%
\pgfpathmoveto{\pgfqpoint{0.562500in}{2.200000in}}%
\pgfpathlineto{\pgfqpoint{4.050000in}{2.200000in}}%
\pgfusepath{stroke}%
\end{pgfscope}%
\end{pgfpicture}%
\makeatother%
\endgroup%

        \end{center}
    \end{frame}

    \begin{frame}
    \frametitle{Lösungen Recharge Oszillator bei variierendem $\alpha$}
        \begin{center}
            %% Creator: Matplotlib, PGF backend
%%
%% To include the figure in your LaTeX document, write
%%   \input{<filename>.pgf}
%%
%% Make sure the required packages are loaded in your preamble
%%   \usepackage{pgf}
%%
%% Also ensure that all the required font packages are loaded; for instance,
%% the lmodern package is sometimes necessary when using math font.
%%   \usepackage{lmodern}
%%
%% Figures using additional raster images can only be included by \input if
%% they are in the same directory as the main LaTeX file. For loading figures
%% from other directories you can use the `import` package
%%   \usepackage{import}
%%
%% and then include the figures with
%%   \import{<path to file>}{<filename>.pgf}
%%
%% Matplotlib used the following preamble
%%   \usepackage{bm}
%%   \usepackage{amsmath}
%%   \usepackage{xcolor}
%%   \usepackage{tgtermes}
%%   \makeatletter\@ifpackageloaded{underscore}{}{\usepackage[strings]{underscore}}\makeatother
%%
\begingroup%
\makeatletter%
\begin{pgfpicture}%
\pgfpathrectangle{\pgfpointorigin}{\pgfqpoint{4.500000in}{2.500000in}}%
\pgfusepath{use as bounding box, clip}%
\begin{pgfscope}%
\pgfsetbuttcap%
\pgfsetmiterjoin%
\definecolor{currentfill}{rgb}{1.000000,1.000000,1.000000}%
\pgfsetfillcolor{currentfill}%
\pgfsetlinewidth{0.000000pt}%
\definecolor{currentstroke}{rgb}{1.000000,1.000000,1.000000}%
\pgfsetstrokecolor{currentstroke}%
\pgfsetdash{}{0pt}%
\pgfpathmoveto{\pgfqpoint{0.000000in}{0.000000in}}%
\pgfpathlineto{\pgfqpoint{4.500000in}{0.000000in}}%
\pgfpathlineto{\pgfqpoint{4.500000in}{2.500000in}}%
\pgfpathlineto{\pgfqpoint{0.000000in}{2.500000in}}%
\pgfpathlineto{\pgfqpoint{0.000000in}{0.000000in}}%
\pgfpathclose%
\pgfusepath{fill}%
\end{pgfscope}%
\begin{pgfscope}%
\pgfsetbuttcap%
\pgfsetmiterjoin%
\definecolor{currentfill}{rgb}{1.000000,1.000000,1.000000}%
\pgfsetfillcolor{currentfill}%
\pgfsetlinewidth{0.000000pt}%
\definecolor{currentstroke}{rgb}{0.000000,0.000000,0.000000}%
\pgfsetstrokecolor{currentstroke}%
\pgfsetstrokeopacity{0.000000}%
\pgfsetdash{}{0pt}%
\pgfpathmoveto{\pgfqpoint{0.562500in}{0.275000in}}%
\pgfpathlineto{\pgfqpoint{4.050000in}{0.275000in}}%
\pgfpathlineto{\pgfqpoint{4.050000in}{2.200000in}}%
\pgfpathlineto{\pgfqpoint{0.562500in}{2.200000in}}%
\pgfpathlineto{\pgfqpoint{0.562500in}{0.275000in}}%
\pgfpathclose%
\pgfusepath{fill}%
\end{pgfscope}%
\begin{pgfscope}%
\pgfpathrectangle{\pgfqpoint{0.562500in}{0.275000in}}{\pgfqpoint{3.487500in}{1.925000in}}%
\pgfusepath{clip}%
\pgfsetrectcap%
\pgfsetroundjoin%
\pgfsetlinewidth{0.803000pt}%
\definecolor{currentstroke}{rgb}{0.690196,0.690196,0.690196}%
\pgfsetstrokecolor{currentstroke}%
\pgfsetdash{}{0pt}%
\pgfpathmoveto{\pgfqpoint{0.765198in}{0.275000in}}%
\pgfpathlineto{\pgfqpoint{0.765198in}{2.200000in}}%
\pgfusepath{stroke}%
\end{pgfscope}%
\begin{pgfscope}%
\pgfsetbuttcap%
\pgfsetroundjoin%
\definecolor{currentfill}{rgb}{0.000000,0.000000,0.000000}%
\pgfsetfillcolor{currentfill}%
\pgfsetlinewidth{0.803000pt}%
\definecolor{currentstroke}{rgb}{0.000000,0.000000,0.000000}%
\pgfsetstrokecolor{currentstroke}%
\pgfsetdash{}{0pt}%
\pgfsys@defobject{currentmarker}{\pgfqpoint{0.000000in}{-0.048611in}}{\pgfqpoint{0.000000in}{0.000000in}}{%
\pgfpathmoveto{\pgfqpoint{0.000000in}{0.000000in}}%
\pgfpathlineto{\pgfqpoint{0.000000in}{-0.048611in}}%
\pgfusepath{stroke,fill}%
}%
\begin{pgfscope}%
\pgfsys@transformshift{0.765198in}{0.275000in}%
\pgfsys@useobject{currentmarker}{}%
\end{pgfscope}%
\end{pgfscope}%
\begin{pgfscope}%
\definecolor{textcolor}{rgb}{0.000000,0.000000,0.000000}%
\pgfsetstrokecolor{textcolor}%
\pgfsetfillcolor{textcolor}%
\pgftext[x=0.765198in,y=0.177778in,,top]{\color{textcolor}\rmfamily\fontsize{10.000000}{12.000000}\selectfont \(\displaystyle {-0.75}\)}%
\end{pgfscope}%
\begin{pgfscope}%
\pgfpathrectangle{\pgfqpoint{0.562500in}{0.275000in}}{\pgfqpoint{3.487500in}{1.925000in}}%
\pgfusepath{clip}%
\pgfsetrectcap%
\pgfsetroundjoin%
\pgfsetlinewidth{0.803000pt}%
\definecolor{currentstroke}{rgb}{0.690196,0.690196,0.690196}%
\pgfsetstrokecolor{currentstroke}%
\pgfsetdash{}{0pt}%
\pgfpathmoveto{\pgfqpoint{1.211810in}{0.275000in}}%
\pgfpathlineto{\pgfqpoint{1.211810in}{2.200000in}}%
\pgfusepath{stroke}%
\end{pgfscope}%
\begin{pgfscope}%
\pgfsetbuttcap%
\pgfsetroundjoin%
\definecolor{currentfill}{rgb}{0.000000,0.000000,0.000000}%
\pgfsetfillcolor{currentfill}%
\pgfsetlinewidth{0.803000pt}%
\definecolor{currentstroke}{rgb}{0.000000,0.000000,0.000000}%
\pgfsetstrokecolor{currentstroke}%
\pgfsetdash{}{0pt}%
\pgfsys@defobject{currentmarker}{\pgfqpoint{0.000000in}{-0.048611in}}{\pgfqpoint{0.000000in}{0.000000in}}{%
\pgfpathmoveto{\pgfqpoint{0.000000in}{0.000000in}}%
\pgfpathlineto{\pgfqpoint{0.000000in}{-0.048611in}}%
\pgfusepath{stroke,fill}%
}%
\begin{pgfscope}%
\pgfsys@transformshift{1.211810in}{0.275000in}%
\pgfsys@useobject{currentmarker}{}%
\end{pgfscope}%
\end{pgfscope}%
\begin{pgfscope}%
\definecolor{textcolor}{rgb}{0.000000,0.000000,0.000000}%
\pgfsetstrokecolor{textcolor}%
\pgfsetfillcolor{textcolor}%
\pgftext[x=1.211810in,y=0.177778in,,top]{\color{textcolor}\rmfamily\fontsize{10.000000}{12.000000}\selectfont \(\displaystyle {-0.50}\)}%
\end{pgfscope}%
\begin{pgfscope}%
\pgfpathrectangle{\pgfqpoint{0.562500in}{0.275000in}}{\pgfqpoint{3.487500in}{1.925000in}}%
\pgfusepath{clip}%
\pgfsetrectcap%
\pgfsetroundjoin%
\pgfsetlinewidth{0.803000pt}%
\definecolor{currentstroke}{rgb}{0.690196,0.690196,0.690196}%
\pgfsetstrokecolor{currentstroke}%
\pgfsetdash{}{0pt}%
\pgfpathmoveto{\pgfqpoint{1.658421in}{0.275000in}}%
\pgfpathlineto{\pgfqpoint{1.658421in}{2.200000in}}%
\pgfusepath{stroke}%
\end{pgfscope}%
\begin{pgfscope}%
\pgfsetbuttcap%
\pgfsetroundjoin%
\definecolor{currentfill}{rgb}{0.000000,0.000000,0.000000}%
\pgfsetfillcolor{currentfill}%
\pgfsetlinewidth{0.803000pt}%
\definecolor{currentstroke}{rgb}{0.000000,0.000000,0.000000}%
\pgfsetstrokecolor{currentstroke}%
\pgfsetdash{}{0pt}%
\pgfsys@defobject{currentmarker}{\pgfqpoint{0.000000in}{-0.048611in}}{\pgfqpoint{0.000000in}{0.000000in}}{%
\pgfpathmoveto{\pgfqpoint{0.000000in}{0.000000in}}%
\pgfpathlineto{\pgfqpoint{0.000000in}{-0.048611in}}%
\pgfusepath{stroke,fill}%
}%
\begin{pgfscope}%
\pgfsys@transformshift{1.658421in}{0.275000in}%
\pgfsys@useobject{currentmarker}{}%
\end{pgfscope}%
\end{pgfscope}%
\begin{pgfscope}%
\definecolor{textcolor}{rgb}{0.000000,0.000000,0.000000}%
\pgfsetstrokecolor{textcolor}%
\pgfsetfillcolor{textcolor}%
\pgftext[x=1.658421in,y=0.177778in,,top]{\color{textcolor}\rmfamily\fontsize{10.000000}{12.000000}\selectfont \(\displaystyle {-0.25}\)}%
\end{pgfscope}%
\begin{pgfscope}%
\pgfpathrectangle{\pgfqpoint{0.562500in}{0.275000in}}{\pgfqpoint{3.487500in}{1.925000in}}%
\pgfusepath{clip}%
\pgfsetrectcap%
\pgfsetroundjoin%
\pgfsetlinewidth{0.803000pt}%
\definecolor{currentstroke}{rgb}{0.690196,0.690196,0.690196}%
\pgfsetstrokecolor{currentstroke}%
\pgfsetdash{}{0pt}%
\pgfpathmoveto{\pgfqpoint{2.105032in}{0.275000in}}%
\pgfpathlineto{\pgfqpoint{2.105032in}{2.200000in}}%
\pgfusepath{stroke}%
\end{pgfscope}%
\begin{pgfscope}%
\pgfsetbuttcap%
\pgfsetroundjoin%
\definecolor{currentfill}{rgb}{0.000000,0.000000,0.000000}%
\pgfsetfillcolor{currentfill}%
\pgfsetlinewidth{0.803000pt}%
\definecolor{currentstroke}{rgb}{0.000000,0.000000,0.000000}%
\pgfsetstrokecolor{currentstroke}%
\pgfsetdash{}{0pt}%
\pgfsys@defobject{currentmarker}{\pgfqpoint{0.000000in}{-0.048611in}}{\pgfqpoint{0.000000in}{0.000000in}}{%
\pgfpathmoveto{\pgfqpoint{0.000000in}{0.000000in}}%
\pgfpathlineto{\pgfqpoint{0.000000in}{-0.048611in}}%
\pgfusepath{stroke,fill}%
}%
\begin{pgfscope}%
\pgfsys@transformshift{2.105032in}{0.275000in}%
\pgfsys@useobject{currentmarker}{}%
\end{pgfscope}%
\end{pgfscope}%
\begin{pgfscope}%
\definecolor{textcolor}{rgb}{0.000000,0.000000,0.000000}%
\pgfsetstrokecolor{textcolor}%
\pgfsetfillcolor{textcolor}%
\pgftext[x=2.105032in,y=0.177778in,,top]{\color{textcolor}\rmfamily\fontsize{10.000000}{12.000000}\selectfont \(\displaystyle {0.00}\)}%
\end{pgfscope}%
\begin{pgfscope}%
\pgfpathrectangle{\pgfqpoint{0.562500in}{0.275000in}}{\pgfqpoint{3.487500in}{1.925000in}}%
\pgfusepath{clip}%
\pgfsetrectcap%
\pgfsetroundjoin%
\pgfsetlinewidth{0.803000pt}%
\definecolor{currentstroke}{rgb}{0.690196,0.690196,0.690196}%
\pgfsetstrokecolor{currentstroke}%
\pgfsetdash{}{0pt}%
\pgfpathmoveto{\pgfqpoint{2.551644in}{0.275000in}}%
\pgfpathlineto{\pgfqpoint{2.551644in}{2.200000in}}%
\pgfusepath{stroke}%
\end{pgfscope}%
\begin{pgfscope}%
\pgfsetbuttcap%
\pgfsetroundjoin%
\definecolor{currentfill}{rgb}{0.000000,0.000000,0.000000}%
\pgfsetfillcolor{currentfill}%
\pgfsetlinewidth{0.803000pt}%
\definecolor{currentstroke}{rgb}{0.000000,0.000000,0.000000}%
\pgfsetstrokecolor{currentstroke}%
\pgfsetdash{}{0pt}%
\pgfsys@defobject{currentmarker}{\pgfqpoint{0.000000in}{-0.048611in}}{\pgfqpoint{0.000000in}{0.000000in}}{%
\pgfpathmoveto{\pgfqpoint{0.000000in}{0.000000in}}%
\pgfpathlineto{\pgfqpoint{0.000000in}{-0.048611in}}%
\pgfusepath{stroke,fill}%
}%
\begin{pgfscope}%
\pgfsys@transformshift{2.551644in}{0.275000in}%
\pgfsys@useobject{currentmarker}{}%
\end{pgfscope}%
\end{pgfscope}%
\begin{pgfscope}%
\definecolor{textcolor}{rgb}{0.000000,0.000000,0.000000}%
\pgfsetstrokecolor{textcolor}%
\pgfsetfillcolor{textcolor}%
\pgftext[x=2.551644in,y=0.177778in,,top]{\color{textcolor}\rmfamily\fontsize{10.000000}{12.000000}\selectfont \(\displaystyle {0.25}\)}%
\end{pgfscope}%
\begin{pgfscope}%
\pgfpathrectangle{\pgfqpoint{0.562500in}{0.275000in}}{\pgfqpoint{3.487500in}{1.925000in}}%
\pgfusepath{clip}%
\pgfsetrectcap%
\pgfsetroundjoin%
\pgfsetlinewidth{0.803000pt}%
\definecolor{currentstroke}{rgb}{0.690196,0.690196,0.690196}%
\pgfsetstrokecolor{currentstroke}%
\pgfsetdash{}{0pt}%
\pgfpathmoveto{\pgfqpoint{2.998255in}{0.275000in}}%
\pgfpathlineto{\pgfqpoint{2.998255in}{2.200000in}}%
\pgfusepath{stroke}%
\end{pgfscope}%
\begin{pgfscope}%
\pgfsetbuttcap%
\pgfsetroundjoin%
\definecolor{currentfill}{rgb}{0.000000,0.000000,0.000000}%
\pgfsetfillcolor{currentfill}%
\pgfsetlinewidth{0.803000pt}%
\definecolor{currentstroke}{rgb}{0.000000,0.000000,0.000000}%
\pgfsetstrokecolor{currentstroke}%
\pgfsetdash{}{0pt}%
\pgfsys@defobject{currentmarker}{\pgfqpoint{0.000000in}{-0.048611in}}{\pgfqpoint{0.000000in}{0.000000in}}{%
\pgfpathmoveto{\pgfqpoint{0.000000in}{0.000000in}}%
\pgfpathlineto{\pgfqpoint{0.000000in}{-0.048611in}}%
\pgfusepath{stroke,fill}%
}%
\begin{pgfscope}%
\pgfsys@transformshift{2.998255in}{0.275000in}%
\pgfsys@useobject{currentmarker}{}%
\end{pgfscope}%
\end{pgfscope}%
\begin{pgfscope}%
\definecolor{textcolor}{rgb}{0.000000,0.000000,0.000000}%
\pgfsetstrokecolor{textcolor}%
\pgfsetfillcolor{textcolor}%
\pgftext[x=2.998255in,y=0.177778in,,top]{\color{textcolor}\rmfamily\fontsize{10.000000}{12.000000}\selectfont \(\displaystyle {0.50}\)}%
\end{pgfscope}%
\begin{pgfscope}%
\pgfpathrectangle{\pgfqpoint{0.562500in}{0.275000in}}{\pgfqpoint{3.487500in}{1.925000in}}%
\pgfusepath{clip}%
\pgfsetrectcap%
\pgfsetroundjoin%
\pgfsetlinewidth{0.803000pt}%
\definecolor{currentstroke}{rgb}{0.690196,0.690196,0.690196}%
\pgfsetstrokecolor{currentstroke}%
\pgfsetdash{}{0pt}%
\pgfpathmoveto{\pgfqpoint{3.444866in}{0.275000in}}%
\pgfpathlineto{\pgfqpoint{3.444866in}{2.200000in}}%
\pgfusepath{stroke}%
\end{pgfscope}%
\begin{pgfscope}%
\pgfsetbuttcap%
\pgfsetroundjoin%
\definecolor{currentfill}{rgb}{0.000000,0.000000,0.000000}%
\pgfsetfillcolor{currentfill}%
\pgfsetlinewidth{0.803000pt}%
\definecolor{currentstroke}{rgb}{0.000000,0.000000,0.000000}%
\pgfsetstrokecolor{currentstroke}%
\pgfsetdash{}{0pt}%
\pgfsys@defobject{currentmarker}{\pgfqpoint{0.000000in}{-0.048611in}}{\pgfqpoint{0.000000in}{0.000000in}}{%
\pgfpathmoveto{\pgfqpoint{0.000000in}{0.000000in}}%
\pgfpathlineto{\pgfqpoint{0.000000in}{-0.048611in}}%
\pgfusepath{stroke,fill}%
}%
\begin{pgfscope}%
\pgfsys@transformshift{3.444866in}{0.275000in}%
\pgfsys@useobject{currentmarker}{}%
\end{pgfscope}%
\end{pgfscope}%
\begin{pgfscope}%
\definecolor{textcolor}{rgb}{0.000000,0.000000,0.000000}%
\pgfsetstrokecolor{textcolor}%
\pgfsetfillcolor{textcolor}%
\pgftext[x=3.444866in,y=0.177778in,,top]{\color{textcolor}\rmfamily\fontsize{10.000000}{12.000000}\selectfont \(\displaystyle {0.75}\)}%
\end{pgfscope}%
\begin{pgfscope}%
\pgfpathrectangle{\pgfqpoint{0.562500in}{0.275000in}}{\pgfqpoint{3.487500in}{1.925000in}}%
\pgfusepath{clip}%
\pgfsetrectcap%
\pgfsetroundjoin%
\pgfsetlinewidth{0.803000pt}%
\definecolor{currentstroke}{rgb}{0.690196,0.690196,0.690196}%
\pgfsetstrokecolor{currentstroke}%
\pgfsetdash{}{0pt}%
\pgfpathmoveto{\pgfqpoint{3.891477in}{0.275000in}}%
\pgfpathlineto{\pgfqpoint{3.891477in}{2.200000in}}%
\pgfusepath{stroke}%
\end{pgfscope}%
\begin{pgfscope}%
\pgfsetbuttcap%
\pgfsetroundjoin%
\definecolor{currentfill}{rgb}{0.000000,0.000000,0.000000}%
\pgfsetfillcolor{currentfill}%
\pgfsetlinewidth{0.803000pt}%
\definecolor{currentstroke}{rgb}{0.000000,0.000000,0.000000}%
\pgfsetstrokecolor{currentstroke}%
\pgfsetdash{}{0pt}%
\pgfsys@defobject{currentmarker}{\pgfqpoint{0.000000in}{-0.048611in}}{\pgfqpoint{0.000000in}{0.000000in}}{%
\pgfpathmoveto{\pgfqpoint{0.000000in}{0.000000in}}%
\pgfpathlineto{\pgfqpoint{0.000000in}{-0.048611in}}%
\pgfusepath{stroke,fill}%
}%
\begin{pgfscope}%
\pgfsys@transformshift{3.891477in}{0.275000in}%
\pgfsys@useobject{currentmarker}{}%
\end{pgfscope}%
\end{pgfscope}%
\begin{pgfscope}%
\definecolor{textcolor}{rgb}{0.000000,0.000000,0.000000}%
\pgfsetstrokecolor{textcolor}%
\pgfsetfillcolor{textcolor}%
\pgftext[x=3.891477in,y=0.177778in,,top]{\color{textcolor}\rmfamily\fontsize{10.000000}{12.000000}\selectfont \(\displaystyle {1.00}\)}%
\end{pgfscope}%
\begin{pgfscope}%
\pgfpathrectangle{\pgfqpoint{0.562500in}{0.275000in}}{\pgfqpoint{3.487500in}{1.925000in}}%
\pgfusepath{clip}%
\pgfsetrectcap%
\pgfsetroundjoin%
\pgfsetlinewidth{0.803000pt}%
\definecolor{currentstroke}{rgb}{0.690196,0.690196,0.690196}%
\pgfsetstrokecolor{currentstroke}%
\pgfsetdash{}{0pt}%
\pgfpathmoveto{\pgfqpoint{0.562500in}{0.364857in}}%
\pgfpathlineto{\pgfqpoint{4.050000in}{0.364857in}}%
\pgfusepath{stroke}%
\end{pgfscope}%
\begin{pgfscope}%
\pgfsetbuttcap%
\pgfsetroundjoin%
\definecolor{currentfill}{rgb}{0.000000,0.000000,0.000000}%
\pgfsetfillcolor{currentfill}%
\pgfsetlinewidth{0.803000pt}%
\definecolor{currentstroke}{rgb}{0.000000,0.000000,0.000000}%
\pgfsetstrokecolor{currentstroke}%
\pgfsetdash{}{0pt}%
\pgfsys@defobject{currentmarker}{\pgfqpoint{-0.048611in}{0.000000in}}{\pgfqpoint{-0.000000in}{0.000000in}}{%
\pgfpathmoveto{\pgfqpoint{-0.000000in}{0.000000in}}%
\pgfpathlineto{\pgfqpoint{-0.048611in}{0.000000in}}%
\pgfusepath{stroke,fill}%
}%
\begin{pgfscope}%
\pgfsys@transformshift{0.562500in}{0.364857in}%
\pgfsys@useobject{currentmarker}{}%
\end{pgfscope}%
\end{pgfscope}%
\begin{pgfscope}%
\definecolor{textcolor}{rgb}{0.000000,0.000000,0.000000}%
\pgfsetstrokecolor{textcolor}%
\pgfsetfillcolor{textcolor}%
\pgftext[x=0.179783in, y=0.318156in, left, base]{\color{textcolor}\rmfamily\fontsize{10.000000}{12.000000}\selectfont \(\displaystyle {-1.0}\)}%
\end{pgfscope}%
\begin{pgfscope}%
\pgfpathrectangle{\pgfqpoint{0.562500in}{0.275000in}}{\pgfqpoint{3.487500in}{1.925000in}}%
\pgfusepath{clip}%
\pgfsetrectcap%
\pgfsetroundjoin%
\pgfsetlinewidth{0.803000pt}%
\definecolor{currentstroke}{rgb}{0.690196,0.690196,0.690196}%
\pgfsetstrokecolor{currentstroke}%
\pgfsetdash{}{0pt}%
\pgfpathmoveto{\pgfqpoint{0.562500in}{0.940002in}}%
\pgfpathlineto{\pgfqpoint{4.050000in}{0.940002in}}%
\pgfusepath{stroke}%
\end{pgfscope}%
\begin{pgfscope}%
\pgfsetbuttcap%
\pgfsetroundjoin%
\definecolor{currentfill}{rgb}{0.000000,0.000000,0.000000}%
\pgfsetfillcolor{currentfill}%
\pgfsetlinewidth{0.803000pt}%
\definecolor{currentstroke}{rgb}{0.000000,0.000000,0.000000}%
\pgfsetstrokecolor{currentstroke}%
\pgfsetdash{}{0pt}%
\pgfsys@defobject{currentmarker}{\pgfqpoint{-0.048611in}{0.000000in}}{\pgfqpoint{-0.000000in}{0.000000in}}{%
\pgfpathmoveto{\pgfqpoint{-0.000000in}{0.000000in}}%
\pgfpathlineto{\pgfqpoint{-0.048611in}{0.000000in}}%
\pgfusepath{stroke,fill}%
}%
\begin{pgfscope}%
\pgfsys@transformshift{0.562500in}{0.940002in}%
\pgfsys@useobject{currentmarker}{}%
\end{pgfscope}%
\end{pgfscope}%
\begin{pgfscope}%
\definecolor{textcolor}{rgb}{0.000000,0.000000,0.000000}%
\pgfsetstrokecolor{textcolor}%
\pgfsetfillcolor{textcolor}%
\pgftext[x=0.179783in, y=0.893301in, left, base]{\color{textcolor}\rmfamily\fontsize{10.000000}{12.000000}\selectfont \(\displaystyle {-0.5}\)}%
\end{pgfscope}%
\begin{pgfscope}%
\pgfpathrectangle{\pgfqpoint{0.562500in}{0.275000in}}{\pgfqpoint{3.487500in}{1.925000in}}%
\pgfusepath{clip}%
\pgfsetrectcap%
\pgfsetroundjoin%
\pgfsetlinewidth{0.803000pt}%
\definecolor{currentstroke}{rgb}{0.690196,0.690196,0.690196}%
\pgfsetstrokecolor{currentstroke}%
\pgfsetdash{}{0pt}%
\pgfpathmoveto{\pgfqpoint{0.562500in}{1.515147in}}%
\pgfpathlineto{\pgfqpoint{4.050000in}{1.515147in}}%
\pgfusepath{stroke}%
\end{pgfscope}%
\begin{pgfscope}%
\pgfsetbuttcap%
\pgfsetroundjoin%
\definecolor{currentfill}{rgb}{0.000000,0.000000,0.000000}%
\pgfsetfillcolor{currentfill}%
\pgfsetlinewidth{0.803000pt}%
\definecolor{currentstroke}{rgb}{0.000000,0.000000,0.000000}%
\pgfsetstrokecolor{currentstroke}%
\pgfsetdash{}{0pt}%
\pgfsys@defobject{currentmarker}{\pgfqpoint{-0.048611in}{0.000000in}}{\pgfqpoint{-0.000000in}{0.000000in}}{%
\pgfpathmoveto{\pgfqpoint{-0.000000in}{0.000000in}}%
\pgfpathlineto{\pgfqpoint{-0.048611in}{0.000000in}}%
\pgfusepath{stroke,fill}%
}%
\begin{pgfscope}%
\pgfsys@transformshift{0.562500in}{1.515147in}%
\pgfsys@useobject{currentmarker}{}%
\end{pgfscope}%
\end{pgfscope}%
\begin{pgfscope}%
\definecolor{textcolor}{rgb}{0.000000,0.000000,0.000000}%
\pgfsetstrokecolor{textcolor}%
\pgfsetfillcolor{textcolor}%
\pgftext[x=0.287808in, y=1.468445in, left, base]{\color{textcolor}\rmfamily\fontsize{10.000000}{12.000000}\selectfont \(\displaystyle {0.0}\)}%
\end{pgfscope}%
\begin{pgfscope}%
\pgfpathrectangle{\pgfqpoint{0.562500in}{0.275000in}}{\pgfqpoint{3.487500in}{1.925000in}}%
\pgfusepath{clip}%
\pgfsetrectcap%
\pgfsetroundjoin%
\pgfsetlinewidth{0.803000pt}%
\definecolor{currentstroke}{rgb}{0.690196,0.690196,0.690196}%
\pgfsetstrokecolor{currentstroke}%
\pgfsetdash{}{0pt}%
\pgfpathmoveto{\pgfqpoint{0.562500in}{2.090291in}}%
\pgfpathlineto{\pgfqpoint{4.050000in}{2.090291in}}%
\pgfusepath{stroke}%
\end{pgfscope}%
\begin{pgfscope}%
\pgfsetbuttcap%
\pgfsetroundjoin%
\definecolor{currentfill}{rgb}{0.000000,0.000000,0.000000}%
\pgfsetfillcolor{currentfill}%
\pgfsetlinewidth{0.803000pt}%
\definecolor{currentstroke}{rgb}{0.000000,0.000000,0.000000}%
\pgfsetstrokecolor{currentstroke}%
\pgfsetdash{}{0pt}%
\pgfsys@defobject{currentmarker}{\pgfqpoint{-0.048611in}{0.000000in}}{\pgfqpoint{-0.000000in}{0.000000in}}{%
\pgfpathmoveto{\pgfqpoint{-0.000000in}{0.000000in}}%
\pgfpathlineto{\pgfqpoint{-0.048611in}{0.000000in}}%
\pgfusepath{stroke,fill}%
}%
\begin{pgfscope}%
\pgfsys@transformshift{0.562500in}{2.090291in}%
\pgfsys@useobject{currentmarker}{}%
\end{pgfscope}%
\end{pgfscope}%
\begin{pgfscope}%
\definecolor{textcolor}{rgb}{0.000000,0.000000,0.000000}%
\pgfsetstrokecolor{textcolor}%
\pgfsetfillcolor{textcolor}%
\pgftext[x=0.287808in, y=2.043590in, left, base]{\color{textcolor}\rmfamily\fontsize{10.000000}{12.000000}\selectfont \(\displaystyle {0.5}\)}%
\end{pgfscope}%
\begin{pgfscope}%
\pgfpathrectangle{\pgfqpoint{0.562500in}{0.275000in}}{\pgfqpoint{3.487500in}{1.925000in}}%
\pgfusepath{clip}%
\pgfsetrectcap%
\pgfsetroundjoin%
\pgfsetlinewidth{1.505625pt}%
\definecolor{currentstroke}{rgb}{0.121569,0.466667,0.705882}%
\pgfsetstrokecolor{currentstroke}%
\pgfsetdash{}{0pt}%
\pgfpathmoveto{\pgfqpoint{3.891477in}{0.364857in}}%
\pgfpathlineto{\pgfqpoint{3.766980in}{0.366831in}}%
\pgfpathlineto{\pgfqpoint{3.641847in}{0.372619in}}%
\pgfpathlineto{\pgfqpoint{3.513546in}{0.382113in}}%
\pgfpathlineto{\pgfqpoint{3.379924in}{0.395264in}}%
\pgfpathlineto{\pgfqpoint{3.239187in}{0.412079in}}%
\pgfpathlineto{\pgfqpoint{3.089893in}{0.432615in}}%
\pgfpathlineto{\pgfqpoint{2.930959in}{0.456984in}}%
\pgfpathlineto{\pgfqpoint{2.759670in}{0.485406in}}%
\pgfpathlineto{\pgfqpoint{2.574395in}{0.518119in}}%
\pgfpathlineto{\pgfqpoint{2.381738in}{0.555065in}}%
\pgfpathlineto{\pgfqpoint{2.187831in}{0.596100in}}%
\pgfpathlineto{\pgfqpoint{2.092098in}{0.618084in}}%
\pgfpathlineto{\pgfqpoint{1.997974in}{0.640999in}}%
\pgfpathlineto{\pgfqpoint{1.905994in}{0.664807in}}%
\pgfpathlineto{\pgfqpoint{1.816635in}{0.689462in}}%
\pgfpathlineto{\pgfqpoint{1.730328in}{0.714913in}}%
\pgfpathlineto{\pgfqpoint{1.647449in}{0.741106in}}%
\pgfpathlineto{\pgfqpoint{1.568322in}{0.767981in}}%
\pgfpathlineto{\pgfqpoint{1.493219in}{0.795473in}}%
\pgfpathlineto{\pgfqpoint{1.422360in}{0.823513in}}%
\pgfpathlineto{\pgfqpoint{1.355915in}{0.852026in}}%
\pgfpathlineto{\pgfqpoint{1.293999in}{0.880933in}}%
\pgfpathlineto{\pgfqpoint{1.236676in}{0.910148in}}%
\pgfpathlineto{\pgfqpoint{1.183960in}{0.939584in}}%
\pgfpathlineto{\pgfqpoint{1.135810in}{0.969145in}}%
\pgfpathlineto{\pgfqpoint{1.092134in}{0.998732in}}%
\pgfpathlineto{\pgfqpoint{1.052789in}{1.028243in}}%
\pgfpathlineto{\pgfqpoint{1.017579in}{1.057567in}}%
\pgfpathlineto{\pgfqpoint{0.986256in}{1.086590in}}%
\pgfpathlineto{\pgfqpoint{0.958427in}{1.115230in}}%
\pgfpathlineto{\pgfqpoint{0.933338in}{1.143563in}}%
\pgfpathlineto{\pgfqpoint{0.910658in}{1.171577in}}%
\pgfpathlineto{\pgfqpoint{0.890139in}{1.199242in}}%
\pgfpathlineto{\pgfqpoint{0.871557in}{1.226534in}}%
\pgfpathlineto{\pgfqpoint{0.839435in}{1.279922in}}%
\pgfpathlineto{\pgfqpoint{0.812985in}{1.331621in}}%
\pgfpathlineto{\pgfqpoint{0.791220in}{1.381544in}}%
\pgfpathlineto{\pgfqpoint{0.773412in}{1.429623in}}%
\pgfpathlineto{\pgfqpoint{0.758894in}{1.475868in}}%
\pgfpathlineto{\pgfqpoint{0.747166in}{1.520291in}}%
\pgfpathlineto{\pgfqpoint{0.737912in}{1.562899in}}%
\pgfpathlineto{\pgfqpoint{0.730997in}{1.603691in}}%
\pgfpathlineto{\pgfqpoint{0.726471in}{1.642660in}}%
\pgfpathlineto{\pgfqpoint{0.724350in}{1.679807in}}%
\pgfpathlineto{\pgfqpoint{0.724221in}{1.715172in}}%
\pgfpathlineto{\pgfqpoint{0.725959in}{1.748786in}}%
\pgfpathlineto{\pgfqpoint{0.729493in}{1.780679in}}%
\pgfpathlineto{\pgfqpoint{0.734787in}{1.810878in}}%
\pgfpathlineto{\pgfqpoint{0.741846in}{1.839409in}}%
\pgfpathlineto{\pgfqpoint{0.750712in}{1.866296in}}%
\pgfpathlineto{\pgfqpoint{0.761472in}{1.891563in}}%
\pgfpathlineto{\pgfqpoint{0.774138in}{1.915238in}}%
\pgfpathlineto{\pgfqpoint{0.788607in}{1.937338in}}%
\pgfpathlineto{\pgfqpoint{0.804811in}{1.957879in}}%
\pgfpathlineto{\pgfqpoint{0.822725in}{1.976877in}}%
\pgfpathlineto{\pgfqpoint{0.842365in}{1.994347in}}%
\pgfpathlineto{\pgfqpoint{0.863788in}{2.010306in}}%
\pgfpathlineto{\pgfqpoint{0.887094in}{2.024766in}}%
\pgfpathlineto{\pgfqpoint{0.912421in}{2.037743in}}%
\pgfpathlineto{\pgfqpoint{0.939953in}{2.049251in}}%
\pgfpathlineto{\pgfqpoint{0.969911in}{2.059303in}}%
\pgfpathlineto{\pgfqpoint{1.002485in}{2.067907in}}%
\pgfpathlineto{\pgfqpoint{1.037541in}{2.075038in}}%
\pgfpathlineto{\pgfqpoint{1.075215in}{2.080680in}}%
\pgfpathlineto{\pgfqpoint{1.115701in}{2.084819in}}%
\pgfpathlineto{\pgfqpoint{1.159200in}{2.087434in}}%
\pgfpathlineto{\pgfqpoint{1.205922in}{2.088500in}}%
\pgfpathlineto{\pgfqpoint{1.256083in}{2.087986in}}%
\pgfpathlineto{\pgfqpoint{1.309909in}{2.085859in}}%
\pgfpathlineto{\pgfqpoint{1.398030in}{2.079556in}}%
\pgfpathlineto{\pgfqpoint{1.495735in}{2.069379in}}%
\pgfpathlineto{\pgfqpoint{1.603880in}{2.055152in}}%
\pgfpathlineto{\pgfqpoint{1.723208in}{2.036669in}}%
\pgfpathlineto{\pgfqpoint{1.854058in}{2.013687in}}%
\pgfpathlineto{\pgfqpoint{1.995995in}{1.986007in}}%
\pgfpathlineto{\pgfqpoint{2.147468in}{1.953495in}}%
\pgfpathlineto{\pgfqpoint{2.252487in}{1.929104in}}%
\pgfpathlineto{\pgfqpoint{2.359338in}{1.902567in}}%
\pgfpathlineto{\pgfqpoint{2.466389in}{1.873961in}}%
\pgfpathlineto{\pgfqpoint{2.571992in}{1.843408in}}%
\pgfpathlineto{\pgfqpoint{2.674499in}{1.811072in}}%
\pgfpathlineto{\pgfqpoint{2.772260in}{1.777162in}}%
\pgfpathlineto{\pgfqpoint{2.863485in}{1.741925in}}%
\pgfpathlineto{\pgfqpoint{2.946481in}{1.705640in}}%
\pgfpathlineto{\pgfqpoint{3.021308in}{1.668631in}}%
\pgfpathlineto{\pgfqpoint{3.088234in}{1.631203in}}%
\pgfpathlineto{\pgfqpoint{3.147543in}{1.593635in}}%
\pgfpathlineto{\pgfqpoint{3.199539in}{1.556176in}}%
\pgfpathlineto{\pgfqpoint{3.244545in}{1.519052in}}%
\pgfpathlineto{\pgfqpoint{3.282898in}{1.482460in}}%
\pgfpathlineto{\pgfqpoint{3.314959in}{1.446571in}}%
\pgfpathlineto{\pgfqpoint{3.341102in}{1.411529in}}%
\pgfpathlineto{\pgfqpoint{3.361723in}{1.377452in}}%
\pgfpathlineto{\pgfqpoint{3.377232in}{1.344433in}}%
\pgfpathlineto{\pgfqpoint{3.388136in}{1.312605in}}%
\pgfpathlineto{\pgfqpoint{3.395267in}{1.282044in}}%
\pgfpathlineto{\pgfqpoint{3.399368in}{1.252779in}}%
\pgfpathlineto{\pgfqpoint{3.401005in}{1.224835in}}%
\pgfpathlineto{\pgfqpoint{3.400561in}{1.198232in}}%
\pgfpathlineto{\pgfqpoint{3.398240in}{1.172982in}}%
\pgfpathlineto{\pgfqpoint{3.394063in}{1.149092in}}%
\pgfpathlineto{\pgfqpoint{3.387874in}{1.126564in}}%
\pgfpathlineto{\pgfqpoint{3.379335in}{1.105392in}}%
\pgfpathlineto{\pgfqpoint{3.367927in}{1.085567in}}%
\pgfpathlineto{\pgfqpoint{3.353125in}{1.067079in}}%
\pgfpathlineto{\pgfqpoint{3.335882in}{1.049982in}}%
\pgfpathlineto{\pgfqpoint{3.316513in}{1.034290in}}%
\pgfpathlineto{\pgfqpoint{3.295015in}{1.020007in}}%
\pgfpathlineto{\pgfqpoint{3.271361in}{1.007133in}}%
\pgfpathlineto{\pgfqpoint{3.245499in}{0.995671in}}%
\pgfpathlineto{\pgfqpoint{3.217352in}{0.985626in}}%
\pgfpathlineto{\pgfqpoint{3.186818in}{0.977001in}}%
\pgfpathlineto{\pgfqpoint{3.153770in}{0.969802in}}%
\pgfpathlineto{\pgfqpoint{3.118057in}{0.964033in}}%
\pgfpathlineto{\pgfqpoint{3.079513in}{0.959702in}}%
\pgfpathlineto{\pgfqpoint{3.038178in}{0.956823in}}%
\pgfpathlineto{\pgfqpoint{2.993816in}{0.955427in}}%
\pgfpathlineto{\pgfqpoint{2.946083in}{0.955553in}}%
\pgfpathlineto{\pgfqpoint{2.894672in}{0.957241in}}%
\pgfpathlineto{\pgfqpoint{2.810079in}{0.962799in}}%
\pgfpathlineto{\pgfqpoint{2.715855in}{0.972127in}}%
\pgfpathlineto{\pgfqpoint{2.611429in}{0.985399in}}%
\pgfpathlineto{\pgfqpoint{2.496416in}{1.002799in}}%
\pgfpathlineto{\pgfqpoint{2.370618in}{1.024527in}}%
\pgfpathlineto{\pgfqpoint{2.234021in}{1.050796in}}%
\pgfpathlineto{\pgfqpoint{2.086464in}{1.081865in}}%
\pgfpathlineto{\pgfqpoint{1.983291in}{1.105300in}}%
\pgfpathlineto{\pgfqpoint{1.878357in}{1.130844in}}%
\pgfpathlineto{\pgfqpoint{1.773318in}{1.158403in}}%
\pgfpathlineto{\pgfqpoint{1.669700in}{1.187859in}}%
\pgfpathlineto{\pgfqpoint{1.568897in}{1.219062in}}%
\pgfpathlineto{\pgfqpoint{1.472170in}{1.251839in}}%
\pgfpathlineto{\pgfqpoint{1.380650in}{1.285987in}}%
\pgfpathlineto{\pgfqpoint{1.295334in}{1.321277in}}%
\pgfpathlineto{\pgfqpoint{1.217087in}{1.357450in}}%
\pgfpathlineto{\pgfqpoint{1.146645in}{1.394223in}}%
\pgfpathlineto{\pgfqpoint{1.084609in}{1.431283in}}%
\pgfpathlineto{\pgfqpoint{1.031097in}{1.468318in}}%
\pgfpathlineto{\pgfqpoint{0.985172in}{1.505099in}}%
\pgfpathlineto{\pgfqpoint{0.946069in}{1.541425in}}%
\pgfpathlineto{\pgfqpoint{0.913137in}{1.577114in}}%
\pgfpathlineto{\pgfqpoint{0.885823in}{1.612010in}}%
\pgfpathlineto{\pgfqpoint{0.863680in}{1.645975in}}%
\pgfpathlineto{\pgfqpoint{0.846349in}{1.678897in}}%
\pgfpathlineto{\pgfqpoint{0.833265in}{1.710683in}}%
\pgfpathlineto{\pgfqpoint{0.823876in}{1.741270in}}%
\pgfpathlineto{\pgfqpoint{0.817792in}{1.770610in}}%
\pgfpathlineto{\pgfqpoint{0.814712in}{1.798661in}}%
\pgfpathlineto{\pgfqpoint{0.814427in}{1.825392in}}%
\pgfpathlineto{\pgfqpoint{0.816815in}{1.850783in}}%
\pgfpathlineto{\pgfqpoint{0.821828in}{1.874820in}}%
\pgfpathlineto{\pgfqpoint{0.829322in}{1.897494in}}%
\pgfpathlineto{\pgfqpoint{0.839115in}{1.918796in}}%
\pgfpathlineto{\pgfqpoint{0.851075in}{1.938719in}}%
\pgfpathlineto{\pgfqpoint{0.865114in}{1.957256in}}%
\pgfpathlineto{\pgfqpoint{0.881190in}{1.974407in}}%
\pgfpathlineto{\pgfqpoint{0.899309in}{1.990169in}}%
\pgfpathlineto{\pgfqpoint{0.919521in}{2.004546in}}%
\pgfpathlineto{\pgfqpoint{0.941922in}{2.017542in}}%
\pgfpathlineto{\pgfqpoint{0.966655in}{2.029162in}}%
\pgfpathlineto{\pgfqpoint{0.993905in}{2.039416in}}%
\pgfpathlineto{\pgfqpoint{1.023634in}{2.048289in}}%
\pgfpathlineto{\pgfqpoint{1.055840in}{2.055759in}}%
\pgfpathlineto{\pgfqpoint{1.090662in}{2.061814in}}%
\pgfpathlineto{\pgfqpoint{1.128250in}{2.066435in}}%
\pgfpathlineto{\pgfqpoint{1.168770in}{2.069600in}}%
\pgfpathlineto{\pgfqpoint{1.212399in}{2.071284in}}%
\pgfpathlineto{\pgfqpoint{1.259325in}{2.071457in}}%
\pgfpathlineto{\pgfqpoint{1.309752in}{2.070085in}}%
\pgfpathlineto{\pgfqpoint{1.392430in}{2.065051in}}%
\pgfpathlineto{\pgfqpoint{1.484252in}{2.056317in}}%
\pgfpathlineto{\pgfqpoint{1.586048in}{2.043722in}}%
\pgfpathlineto{\pgfqpoint{1.698534in}{2.027054in}}%
\pgfpathlineto{\pgfqpoint{1.822183in}{2.006084in}}%
\pgfpathlineto{\pgfqpoint{1.956773in}{1.980617in}}%
\pgfpathlineto{\pgfqpoint{2.101303in}{1.950487in}}%
\pgfpathlineto{\pgfqpoint{2.253655in}{1.915606in}}%
\pgfpathlineto{\pgfqpoint{2.357720in}{1.889737in}}%
\pgfpathlineto{\pgfqpoint{2.462213in}{1.861849in}}%
\pgfpathlineto{\pgfqpoint{2.565618in}{1.832052in}}%
\pgfpathlineto{\pgfqpoint{2.666270in}{1.800501in}}%
\pgfpathlineto{\pgfqpoint{2.762339in}{1.767364in}}%
\pgfpathlineto{\pgfqpoint{2.852360in}{1.732867in}}%
\pgfpathlineto{\pgfqpoint{2.935485in}{1.697337in}}%
\pgfpathlineto{\pgfqpoint{3.011108in}{1.661083in}}%
\pgfpathlineto{\pgfqpoint{3.078859in}{1.624391in}}%
\pgfpathlineto{\pgfqpoint{3.138600in}{1.587526in}}%
\pgfpathlineto{\pgfqpoint{3.190426in}{1.550731in}}%
\pgfpathlineto{\pgfqpoint{3.234667in}{1.514224in}}%
\pgfpathlineto{\pgfqpoint{3.271884in}{1.478203in}}%
\pgfpathlineto{\pgfqpoint{3.302873in}{1.442843in}}%
\pgfpathlineto{\pgfqpoint{3.328382in}{1.408312in}}%
\pgfpathlineto{\pgfqpoint{3.348540in}{1.374767in}}%
\pgfpathlineto{\pgfqpoint{3.364286in}{1.342287in}}%
\pgfpathlineto{\pgfqpoint{3.376431in}{1.310933in}}%
\pgfpathlineto{\pgfqpoint{3.385554in}{1.280760in}}%
\pgfpathlineto{\pgfqpoint{3.392007in}{1.251813in}}%
\pgfpathlineto{\pgfqpoint{3.395914in}{1.224129in}}%
\pgfpathlineto{\pgfqpoint{3.397169in}{1.197737in}}%
\pgfpathlineto{\pgfqpoint{3.395438in}{1.172657in}}%
\pgfpathlineto{\pgfqpoint{3.390173in}{1.148901in}}%
\pgfpathlineto{\pgfqpoint{3.381921in}{1.126505in}}%
\pgfpathlineto{\pgfqpoint{3.371392in}{1.105493in}}%
\pgfpathlineto{\pgfqpoint{3.358691in}{1.085870in}}%
\pgfpathlineto{\pgfqpoint{3.343884in}{1.067638in}}%
\pgfpathlineto{\pgfqpoint{3.327000in}{1.050800in}}%
\pgfpathlineto{\pgfqpoint{3.308033in}{1.035356in}}%
\pgfpathlineto{\pgfqpoint{3.286938in}{1.021304in}}%
\pgfpathlineto{\pgfqpoint{3.263636in}{1.008641in}}%
\pgfpathlineto{\pgfqpoint{3.238008in}{0.997363in}}%
\pgfpathlineto{\pgfqpoint{3.209930in}{0.987467in}}%
\pgfpathlineto{\pgfqpoint{3.179440in}{0.978966in}}%
\pgfpathlineto{\pgfqpoint{3.146440in}{0.971874in}}%
\pgfpathlineto{\pgfqpoint{3.110785in}{0.966206in}}%
\pgfpathlineto{\pgfqpoint{3.072317in}{0.961980in}}%
\pgfpathlineto{\pgfqpoint{3.030868in}{0.959220in}}%
\pgfpathlineto{\pgfqpoint{2.986255in}{0.957953in}}%
\pgfpathlineto{\pgfqpoint{2.938285in}{0.958208in}}%
\pgfpathlineto{\pgfqpoint{2.859580in}{0.961523in}}%
\pgfpathlineto{\pgfqpoint{2.772106in}{0.968474in}}%
\pgfpathlineto{\pgfqpoint{2.675054in}{0.979213in}}%
\pgfpathlineto{\pgfqpoint{2.567617in}{0.993919in}}%
\pgfpathlineto{\pgfqpoint{2.449176in}{1.012825in}}%
\pgfpathlineto{\pgfqpoint{2.319561in}{1.036146in}}%
\pgfpathlineto{\pgfqpoint{2.179414in}{1.064062in}}%
\pgfpathlineto{\pgfqpoint{2.030225in}{1.096709in}}%
\pgfpathlineto{\pgfqpoint{1.927167in}{1.121115in}}%
\pgfpathlineto{\pgfqpoint{1.822698in}{1.147598in}}%
\pgfpathlineto{\pgfqpoint{1.718235in}{1.176074in}}%
\pgfpathlineto{\pgfqpoint{1.615332in}{1.206422in}}%
\pgfpathlineto{\pgfqpoint{1.515680in}{1.238475in}}%
\pgfpathlineto{\pgfqpoint{1.421109in}{1.272057in}}%
\pgfpathlineto{\pgfqpoint{1.332933in}{1.306916in}}%
\pgfpathlineto{\pgfqpoint{1.251889in}{1.342722in}}%
\pgfpathlineto{\pgfqpoint{1.178492in}{1.379171in}}%
\pgfpathlineto{\pgfqpoint{1.113028in}{1.415979in}}%
\pgfpathlineto{\pgfqpoint{1.055561in}{1.452885in}}%
\pgfpathlineto{\pgfqpoint{1.005928in}{1.489652in}}%
\pgfpathlineto{\pgfqpoint{0.963743in}{1.526068in}}%
\pgfpathlineto{\pgfqpoint{0.928393in}{1.561942in}}%
\pgfpathlineto{\pgfqpoint{0.899043in}{1.597105in}}%
\pgfpathlineto{\pgfqpoint{0.875063in}{1.631395in}}%
\pgfpathlineto{\pgfqpoint{0.856259in}{1.664667in}}%
\pgfpathlineto{\pgfqpoint{0.841673in}{1.696852in}}%
\pgfpathlineto{\pgfqpoint{0.830538in}{1.727890in}}%
\pgfpathlineto{\pgfqpoint{0.822311in}{1.757731in}}%
\pgfpathlineto{\pgfqpoint{0.816677in}{1.786333in}}%
\pgfpathlineto{\pgfqpoint{0.813545in}{1.813662in}}%
\pgfpathlineto{\pgfqpoint{0.813054in}{1.839691in}}%
\pgfpathlineto{\pgfqpoint{0.815563in}{1.864404in}}%
\pgfpathlineto{\pgfqpoint{0.821620in}{1.887789in}}%
\pgfpathlineto{\pgfqpoint{0.830526in}{1.909808in}}%
\pgfpathlineto{\pgfqpoint{0.841684in}{1.930442in}}%
\pgfpathlineto{\pgfqpoint{0.854999in}{1.949686in}}%
\pgfpathlineto{\pgfqpoint{0.870412in}{1.967536in}}%
\pgfpathlineto{\pgfqpoint{0.887902in}{1.983993in}}%
\pgfpathlineto{\pgfqpoint{0.907481in}{1.999055in}}%
\pgfpathlineto{\pgfqpoint{0.929200in}{2.012725in}}%
\pgfpathlineto{\pgfqpoint{0.953143in}{2.025004in}}%
\pgfpathlineto{\pgfqpoint{0.979432in}{2.035897in}}%
\pgfpathlineto{\pgfqpoint{1.008186in}{2.045406in}}%
\pgfpathlineto{\pgfqpoint{1.039370in}{2.053517in}}%
\pgfpathlineto{\pgfqpoint{1.073096in}{2.060216in}}%
\pgfpathlineto{\pgfqpoint{1.109517in}{2.065487in}}%
\pgfpathlineto{\pgfqpoint{1.148798in}{2.069309in}}%
\pgfpathlineto{\pgfqpoint{1.191113in}{2.071658in}}%
\pgfpathlineto{\pgfqpoint{1.236648in}{2.072508in}}%
\pgfpathlineto{\pgfqpoint{1.285601in}{2.071825in}}%
\pgfpathlineto{\pgfqpoint{1.365899in}{2.067849in}}%
\pgfpathlineto{\pgfqpoint{1.455113in}{2.060208in}}%
\pgfpathlineto{\pgfqpoint{1.554051in}{2.048744in}}%
\pgfpathlineto{\pgfqpoint{1.663505in}{2.033273in}}%
\pgfpathlineto{\pgfqpoint{1.784036in}{2.013561in}}%
\pgfpathlineto{\pgfqpoint{1.915704in}{1.989396in}}%
\pgfpathlineto{\pgfqpoint{2.057697in}{1.960605in}}%
\pgfpathlineto{\pgfqpoint{2.208300in}{1.927070in}}%
\pgfpathlineto{\pgfqpoint{2.311919in}{1.902076in}}%
\pgfpathlineto{\pgfqpoint{2.416557in}{1.875025in}}%
\pgfpathlineto{\pgfqpoint{2.520747in}{1.846009in}}%
\pgfpathlineto{\pgfqpoint{2.622913in}{1.815164in}}%
\pgfpathlineto{\pgfqpoint{2.721367in}{1.782667in}}%
\pgfpathlineto{\pgfqpoint{2.814297in}{1.748709in}}%
\pgfpathlineto{\pgfqpoint{2.900510in}{1.713553in}}%
\pgfpathlineto{\pgfqpoint{2.979390in}{1.677530in}}%
\pgfpathlineto{\pgfqpoint{3.050532in}{1.640943in}}%
\pgfpathlineto{\pgfqpoint{3.113736in}{1.604072in}}%
\pgfpathlineto{\pgfqpoint{3.169012in}{1.567173in}}%
\pgfpathlineto{\pgfqpoint{3.216575in}{1.530479in}}%
\pgfpathlineto{\pgfqpoint{3.256851in}{1.494196in}}%
\pgfpathlineto{\pgfqpoint{3.290472in}{1.458507in}}%
\pgfpathlineto{\pgfqpoint{3.318277in}{1.423572in}}%
\pgfpathlineto{\pgfqpoint{3.340846in}{1.389546in}}%
\pgfpathlineto{\pgfqpoint{3.358388in}{1.356566in}}%
\pgfpathlineto{\pgfqpoint{3.371845in}{1.324695in}}%
\pgfpathlineto{\pgfqpoint{3.381961in}{1.293990in}}%
\pgfpathlineto{\pgfqpoint{3.389263in}{1.264497in}}%
\pgfpathlineto{\pgfqpoint{3.394052in}{1.236255in}}%
\pgfpathlineto{\pgfqpoint{3.396414in}{1.209295in}}%
\pgfpathlineto{\pgfqpoint{3.396209in}{1.183640in}}%
\pgfpathlineto{\pgfqpoint{3.393078in}{1.159306in}}%
\pgfpathlineto{\pgfqpoint{3.386462in}{1.136299in}}%
\pgfpathlineto{\pgfqpoint{3.376931in}{1.114654in}}%
\pgfpathlineto{\pgfqpoint{3.365163in}{1.094397in}}%
\pgfpathlineto{\pgfqpoint{3.351243in}{1.075530in}}%
\pgfpathlineto{\pgfqpoint{3.335225in}{1.058057in}}%
\pgfpathlineto{\pgfqpoint{3.317126in}{1.041978in}}%
\pgfpathlineto{\pgfqpoint{3.296930in}{1.027294in}}%
\pgfpathlineto{\pgfqpoint{3.274586in}{1.014004in}}%
\pgfpathlineto{\pgfqpoint{3.250009in}{1.002106in}}%
\pgfpathlineto{\pgfqpoint{3.223079in}{0.991597in}}%
\pgfpathlineto{\pgfqpoint{3.193662in}{0.982474in}}%
\pgfpathlineto{\pgfqpoint{3.161786in}{0.974751in}}%
\pgfpathlineto{\pgfqpoint{3.127341in}{0.968443in}}%
\pgfpathlineto{\pgfqpoint{3.090162in}{0.963568in}}%
\pgfpathlineto{\pgfqpoint{3.050075in}{0.960146in}}%
\pgfpathlineto{\pgfqpoint{3.006898in}{0.958203in}}%
\pgfpathlineto{\pgfqpoint{2.960439in}{0.957768in}}%
\pgfpathlineto{\pgfqpoint{2.910495in}{0.958874in}}%
\pgfpathlineto{\pgfqpoint{2.828586in}{0.963506in}}%
\pgfpathlineto{\pgfqpoint{2.737616in}{0.971834in}}%
\pgfpathlineto{\pgfqpoint{2.636792in}{0.984023in}}%
\pgfpathlineto{\pgfqpoint{2.525333in}{1.000257in}}%
\pgfpathlineto{\pgfqpoint{2.402751in}{1.020770in}}%
\pgfpathlineto{\pgfqpoint{2.269074in}{1.045777in}}%
\pgfpathlineto{\pgfqpoint{2.125307in}{1.075438in}}%
\pgfpathlineto{\pgfqpoint{1.973463in}{1.109848in}}%
\pgfpathlineto{\pgfqpoint{1.869344in}{1.135419in}}%
\pgfpathlineto{\pgfqpoint{1.764625in}{1.163026in}}%
\pgfpathlineto{\pgfqpoint{1.660861in}{1.192564in}}%
\pgfpathlineto{\pgfqpoint{1.559619in}{1.223885in}}%
\pgfpathlineto{\pgfqpoint{1.462477in}{1.256799in}}%
\pgfpathlineto{\pgfqpoint{1.371062in}{1.291077in}}%
\pgfpathlineto{\pgfqpoint{1.287323in}{1.326472in}}%
\pgfpathlineto{\pgfqpoint{1.211548in}{1.362674in}}%
\pgfpathlineto{\pgfqpoint{1.143495in}{1.399373in}}%
\pgfpathlineto{\pgfqpoint{1.082906in}{1.436288in}}%
\pgfpathlineto{\pgfqpoint{1.029504in}{1.473165in}}%
\pgfpathlineto{\pgfqpoint{0.982993in}{1.509775in}}%
\pgfpathlineto{\pgfqpoint{0.943059in}{1.545918in}}%
\pgfpathlineto{\pgfqpoint{0.909369in}{1.581420in}}%
\pgfpathlineto{\pgfqpoint{0.881573in}{1.616133in}}%
\pgfpathlineto{\pgfqpoint{0.859302in}{1.649937in}}%
\pgfpathlineto{\pgfqpoint{0.842167in}{1.682738in}}%
\pgfpathlineto{\pgfqpoint{0.829760in}{1.714451in}}%
\pgfpathlineto{\pgfqpoint{0.821475in}{1.744943in}}%
\pgfpathlineto{\pgfqpoint{0.816565in}{1.774166in}}%
\pgfpathlineto{\pgfqpoint{0.814428in}{1.802089in}}%
\pgfpathlineto{\pgfqpoint{0.814619in}{1.828689in}}%
\pgfpathlineto{\pgfqpoint{0.816850in}{1.853948in}}%
\pgfpathlineto{\pgfqpoint{0.820994in}{1.877856in}}%
\pgfpathlineto{\pgfqpoint{0.827079in}{1.900408in}}%
\pgfpathlineto{\pgfqpoint{0.835291in}{1.921604in}}%
\pgfpathlineto{\pgfqpoint{0.845976in}{1.941453in}}%
\pgfpathlineto{\pgfqpoint{0.859635in}{1.959970in}}%
\pgfpathlineto{\pgfqpoint{0.876333in}{1.977145in}}%
\pgfpathlineto{\pgfqpoint{0.895218in}{1.992929in}}%
\pgfpathlineto{\pgfqpoint{0.916257in}{2.007317in}}%
\pgfpathlineto{\pgfqpoint{0.939471in}{2.020308in}}%
\pgfpathlineto{\pgfqpoint{0.964902in}{2.031896in}}%
\pgfpathlineto{\pgfqpoint{0.992618in}{2.042076in}}%
\pgfpathlineto{\pgfqpoint{1.022711in}{2.050842in}}%
\pgfpathlineto{\pgfqpoint{1.055297in}{2.058188in}}%
\pgfpathlineto{\pgfqpoint{1.090515in}{2.064106in}}%
\pgfpathlineto{\pgfqpoint{1.128531in}{2.068587in}}%
\pgfpathlineto{\pgfqpoint{1.169477in}{2.071624in}}%
\pgfpathlineto{\pgfqpoint{1.213303in}{2.073198in}}%
\pgfpathlineto{\pgfqpoint{1.260362in}{2.073268in}}%
\pgfpathlineto{\pgfqpoint{1.311026in}{2.071787in}}%
\pgfpathlineto{\pgfqpoint{1.394471in}{2.066557in}}%
\pgfpathlineto{\pgfqpoint{1.487601in}{2.057570in}}%
\pgfpathlineto{\pgfqpoint{1.591017in}{2.044652in}}%
\pgfpathlineto{\pgfqpoint{1.705054in}{2.027626in}}%
\pgfpathlineto{\pgfqpoint{1.829787in}{2.006305in}}%
\pgfpathlineto{\pgfqpoint{1.965025in}{1.980493in}}%
\pgfpathlineto{\pgfqpoint{2.110335in}{1.949987in}}%
\pgfpathlineto{\pgfqpoint{2.264635in}{1.914548in}}%
\pgfpathlineto{\pgfqpoint{2.369840in}{1.888267in}}%
\pgfpathlineto{\pgfqpoint{2.474781in}{1.860004in}}%
\pgfpathlineto{\pgfqpoint{2.577825in}{1.829906in}}%
\pgfpathlineto{\pgfqpoint{2.677520in}{1.798140in}}%
\pgfpathlineto{\pgfqpoint{2.772589in}{1.764896in}}%
\pgfpathlineto{\pgfqpoint{2.861934in}{1.730384in}}%
\pgfpathlineto{\pgfqpoint{2.944633in}{1.694837in}}%
\pgfpathlineto{\pgfqpoint{3.019942in}{1.658509in}}%
\pgfpathlineto{\pgfqpoint{3.087295in}{1.621675in}}%
\pgfpathlineto{\pgfqpoint{3.146303in}{1.584632in}}%
\pgfpathlineto{\pgfqpoint{3.172611in}{1.566131in}}%
\pgfpathlineto{\pgfqpoint{3.219303in}{1.529341in}}%
\pgfpathlineto{\pgfqpoint{3.258979in}{1.492994in}}%
\pgfpathlineto{\pgfqpoint{3.292515in}{1.457259in}}%
\pgfpathlineto{\pgfqpoint{3.320606in}{1.422289in}}%
\pgfpathlineto{\pgfqpoint{3.343763in}{1.388216in}}%
\pgfpathlineto{\pgfqpoint{3.362315in}{1.355157in}}%
\pgfpathlineto{\pgfqpoint{3.376409in}{1.323208in}}%
\pgfpathlineto{\pgfqpoint{3.386199in}{1.292449in}}%
\pgfpathlineto{\pgfqpoint{3.392495in}{1.262943in}}%
\pgfpathlineto{\pgfqpoint{3.395720in}{1.234725in}}%
\pgfpathlineto{\pgfqpoint{3.396165in}{1.207825in}}%
\pgfpathlineto{\pgfqpoint{3.394042in}{1.182267in}}%
\pgfpathlineto{\pgfqpoint{3.389483in}{1.158065in}}%
\pgfpathlineto{\pgfqpoint{3.382542in}{1.135230in}}%
\pgfpathlineto{\pgfqpoint{3.373190in}{1.113765in}}%
\pgfpathlineto{\pgfqpoint{3.361469in}{1.093671in}}%
\pgfpathlineto{\pgfqpoint{3.347522in}{1.074953in}}%
\pgfpathlineto{\pgfqpoint{3.331425in}{1.057617in}}%
\pgfpathlineto{\pgfqpoint{3.313220in}{1.041667in}}%
\pgfpathlineto{\pgfqpoint{3.292916in}{1.027107in}}%
\pgfpathlineto{\pgfqpoint{3.270490in}{1.013939in}}%
\pgfpathlineto{\pgfqpoint{3.245885in}{1.002166in}}%
\pgfpathlineto{\pgfqpoint{3.219011in}{0.991789in}}%
\pgfpathlineto{\pgfqpoint{3.189747in}{0.982808in}}%
\pgfpathlineto{\pgfqpoint{3.157937in}{0.975223in}}%
\pgfpathlineto{\pgfqpoint{3.123400in}{0.969033in}}%
\pgfpathlineto{\pgfqpoint{3.086168in}{0.964259in}}%
\pgfpathlineto{\pgfqpoint{3.046110in}{0.960930in}}%
\pgfpathlineto{\pgfqpoint{3.002970in}{0.959073in}}%
\pgfpathlineto{\pgfqpoint{2.956505in}{0.958719in}}%
\pgfpathlineto{\pgfqpoint{2.906480in}{0.959903in}}%
\pgfpathlineto{\pgfqpoint{2.824286in}{0.964650in}}%
\pgfpathlineto{\pgfqpoint{2.732882in}{0.973098in}}%
\pgfpathlineto{\pgfqpoint{2.631624in}{0.985416in}}%
\pgfpathlineto{\pgfqpoint{2.519919in}{1.001795in}}%
\pgfpathlineto{\pgfqpoint{2.397234in}{1.022446in}}%
\pgfpathlineto{\pgfqpoint{2.262961in}{1.047554in}}%
\pgfpathlineto{\pgfqpoint{2.118288in}{1.077310in}}%
\pgfpathlineto{\pgfqpoint{2.017588in}{1.099811in}}%
\pgfpathlineto{\pgfqpoint{1.914707in}{1.124440in}}%
\pgfpathlineto{\pgfqpoint{1.810803in}{1.151149in}}%
\pgfpathlineto{\pgfqpoint{1.707133in}{1.179845in}}%
\pgfpathlineto{\pgfqpoint{1.605053in}{1.210397in}}%
\pgfpathlineto{\pgfqpoint{1.506020in}{1.242630in}}%
\pgfpathlineto{\pgfqpoint{1.411589in}{1.276328in}}%
\pgfpathlineto{\pgfqpoint{1.323415in}{1.311233in}}%
\pgfpathlineto{\pgfqpoint{1.243155in}{1.347054in}}%
\pgfpathlineto{\pgfqpoint{1.171275in}{1.383509in}}%
\pgfpathlineto{\pgfqpoint{1.107398in}{1.420318in}}%
\pgfpathlineto{\pgfqpoint{1.051131in}{1.457219in}}%
\pgfpathlineto{\pgfqpoint{1.002085in}{1.493973in}}%
\pgfpathlineto{\pgfqpoint{0.959873in}{1.530365in}}%
\pgfpathlineto{\pgfqpoint{0.924110in}{1.566203in}}%
\pgfpathlineto{\pgfqpoint{0.894413in}{1.601314in}}%
\pgfpathlineto{\pgfqpoint{0.870400in}{1.635553in}}%
\pgfpathlineto{\pgfqpoint{0.851678in}{1.668791in}}%
\pgfpathlineto{\pgfqpoint{0.837578in}{1.700912in}}%
\pgfpathlineto{\pgfqpoint{0.827337in}{1.731852in}}%
\pgfpathlineto{\pgfqpoint{0.820357in}{1.761564in}}%
\pgfpathlineto{\pgfqpoint{0.816194in}{1.790006in}}%
\pgfpathlineto{\pgfqpoint{0.814563in}{1.817149in}}%
\pgfpathlineto{\pgfqpoint{0.815335in}{1.842968in}}%
\pgfpathlineto{\pgfqpoint{0.818537in}{1.867448in}}%
\pgfpathlineto{\pgfqpoint{0.824357in}{1.890582in}}%
\pgfpathlineto{\pgfqpoint{0.833116in}{1.912370in}}%
\pgfpathlineto{\pgfqpoint{0.844459in}{1.932790in}}%
\pgfpathlineto{\pgfqpoint{0.858009in}{1.951828in}}%
\pgfpathlineto{\pgfqpoint{0.873699in}{1.969479in}}%
\pgfpathlineto{\pgfqpoint{0.891493in}{1.985740in}}%
\pgfpathlineto{\pgfqpoint{0.911387in}{2.000609in}}%
\pgfpathlineto{\pgfqpoint{0.933409in}{2.014085in}}%
\pgfpathlineto{\pgfqpoint{0.957620in}{2.026167in}}%
\pgfpathlineto{\pgfqpoint{0.984111in}{2.036854in}}%
\pgfpathlineto{\pgfqpoint{1.013008in}{2.046149in}}%
\pgfpathlineto{\pgfqpoint{1.044467in}{2.054052in}}%
\pgfpathlineto{\pgfqpoint{1.078503in}{2.060553in}}%
\pgfpathlineto{\pgfqpoint{1.115190in}{2.065629in}}%
\pgfpathlineto{\pgfqpoint{1.154736in}{2.069260in}}%
\pgfpathlineto{\pgfqpoint{1.197349in}{2.071420in}}%
\pgfpathlineto{\pgfqpoint{1.243240in}{2.072081in}}%
\pgfpathlineto{\pgfqpoint{1.292618in}{2.071209in}}%
\pgfpathlineto{\pgfqpoint{1.373681in}{2.066942in}}%
\pgfpathlineto{\pgfqpoint{1.463771in}{2.058996in}}%
\pgfpathlineto{\pgfqpoint{1.563600in}{2.047201in}}%
\pgfpathlineto{\pgfqpoint{1.673881in}{2.031371in}}%
\pgfpathlineto{\pgfqpoint{1.795256in}{2.011298in}}%
\pgfpathlineto{\pgfqpoint{1.927735in}{1.986759in}}%
\pgfpathlineto{\pgfqpoint{2.070518in}{1.957570in}}%
\pgfpathlineto{\pgfqpoint{2.221582in}{1.923643in}}%
\pgfpathlineto{\pgfqpoint{2.325320in}{1.898401in}}%
\pgfpathlineto{\pgfqpoint{2.325320in}{1.898401in}}%
\pgfusepath{stroke}%
\end{pgfscope}%
\begin{pgfscope}%
\pgfpathrectangle{\pgfqpoint{0.562500in}{0.275000in}}{\pgfqpoint{3.487500in}{1.925000in}}%
\pgfusepath{clip}%
\pgfsetrectcap%
\pgfsetroundjoin%
\pgfsetlinewidth{1.505625pt}%
\definecolor{currentstroke}{rgb}{1.000000,0.498039,0.054902}%
\pgfsetstrokecolor{currentstroke}%
\pgfsetdash{}{0pt}%
\pgfpathmoveto{\pgfqpoint{3.891477in}{0.364857in}}%
\pgfpathlineto{\pgfqpoint{3.829045in}{0.363122in}}%
\pgfpathlineto{\pgfqpoint{3.703873in}{0.362966in}}%
\pgfpathlineto{\pgfqpoint{3.575912in}{0.367110in}}%
\pgfpathlineto{\pgfqpoint{3.442700in}{0.375494in}}%
\pgfpathlineto{\pgfqpoint{3.302193in}{0.388115in}}%
\pgfpathlineto{\pgfqpoint{3.152768in}{0.405027in}}%
\pgfpathlineto{\pgfqpoint{2.993224in}{0.426339in}}%
\pgfpathlineto{\pgfqpoint{2.822749in}{0.452214in}}%
\pgfpathlineto{\pgfqpoint{2.641022in}{0.482777in}}%
\pgfpathlineto{\pgfqpoint{2.450437in}{0.518184in}}%
\pgfpathlineto{\pgfqpoint{2.254703in}{0.558471in}}%
\pgfpathlineto{\pgfqpoint{2.156280in}{0.580406in}}%
\pgfpathlineto{\pgfqpoint{2.058318in}{0.603493in}}%
\pgfpathlineto{\pgfqpoint{1.961502in}{0.627687in}}%
\pgfpathlineto{\pgfqpoint{1.866567in}{0.652930in}}%
\pgfpathlineto{\pgfqpoint{1.774297in}{0.679154in}}%
\pgfpathlineto{\pgfqpoint{1.685526in}{0.706281in}}%
\pgfpathlineto{\pgfqpoint{1.601134in}{0.734220in}}%
\pgfpathlineto{\pgfqpoint{1.521754in}{0.762867in}}%
\pgfpathlineto{\pgfqpoint{1.447510in}{0.792117in}}%
\pgfpathlineto{\pgfqpoint{1.378474in}{0.821872in}}%
\pgfpathlineto{\pgfqpoint{1.314651in}{0.852034in}}%
\pgfpathlineto{\pgfqpoint{1.255986in}{0.882508in}}%
\pgfpathlineto{\pgfqpoint{1.202362in}{0.913205in}}%
\pgfpathlineto{\pgfqpoint{1.153599in}{0.944033in}}%
\pgfpathlineto{\pgfqpoint{1.109454in}{0.974909in}}%
\pgfpathlineto{\pgfqpoint{1.069625in}{1.005747in}}%
\pgfpathlineto{\pgfqpoint{1.033744in}{1.036467in}}%
\pgfpathlineto{\pgfqpoint{1.001397in}{1.066990in}}%
\pgfpathlineto{\pgfqpoint{0.972216in}{1.097256in}}%
\pgfpathlineto{\pgfqpoint{0.945821in}{1.127241in}}%
\pgfpathlineto{\pgfqpoint{0.921876in}{1.156925in}}%
\pgfpathlineto{\pgfqpoint{0.900091in}{1.186286in}}%
\pgfpathlineto{\pgfqpoint{0.880232in}{1.215300in}}%
\pgfpathlineto{\pgfqpoint{0.862113in}{1.243945in}}%
\pgfpathlineto{\pgfqpoint{0.830603in}{1.300020in}}%
\pgfpathlineto{\pgfqpoint{0.817094in}{1.327397in}}%
\pgfpathlineto{\pgfqpoint{0.794396in}{1.380702in}}%
\pgfpathlineto{\pgfqpoint{0.775978in}{1.432073in}}%
\pgfpathlineto{\pgfqpoint{0.761044in}{1.481507in}}%
\pgfpathlineto{\pgfqpoint{0.749156in}{1.528993in}}%
\pgfpathlineto{\pgfqpoint{0.740037in}{1.574524in}}%
\pgfpathlineto{\pgfqpoint{0.733565in}{1.618089in}}%
\pgfpathlineto{\pgfqpoint{0.729617in}{1.659693in}}%
\pgfpathlineto{\pgfqpoint{0.727972in}{1.699360in}}%
\pgfpathlineto{\pgfqpoint{0.728492in}{1.737112in}}%
\pgfpathlineto{\pgfqpoint{0.731098in}{1.772971in}}%
\pgfpathlineto{\pgfqpoint{0.735765in}{1.806957in}}%
\pgfpathlineto{\pgfqpoint{0.742524in}{1.839090in}}%
\pgfpathlineto{\pgfqpoint{0.751461in}{1.869385in}}%
\pgfpathlineto{\pgfqpoint{0.762677in}{1.897865in}}%
\pgfpathlineto{\pgfqpoint{0.776104in}{1.924547in}}%
\pgfpathlineto{\pgfqpoint{0.791660in}{1.949441in}}%
\pgfpathlineto{\pgfqpoint{0.809307in}{1.972554in}}%
\pgfpathlineto{\pgfqpoint{0.829060in}{1.993895in}}%
\pgfpathlineto{\pgfqpoint{0.850978in}{2.013474in}}%
\pgfpathlineto{\pgfqpoint{0.875172in}{2.031296in}}%
\pgfpathlineto{\pgfqpoint{0.901798in}{2.047370in}}%
\pgfpathlineto{\pgfqpoint{0.931062in}{2.061703in}}%
\pgfpathlineto{\pgfqpoint{0.963218in}{2.074303in}}%
\pgfpathlineto{\pgfqpoint{0.998500in}{2.085170in}}%
\pgfpathlineto{\pgfqpoint{1.036801in}{2.094269in}}%
\pgfpathlineto{\pgfqpoint{1.078320in}{2.101573in}}%
\pgfpathlineto{\pgfqpoint{1.123322in}{2.107049in}}%
\pgfpathlineto{\pgfqpoint{1.172070in}{2.110664in}}%
\pgfpathlineto{\pgfqpoint{1.224825in}{2.112374in}}%
\pgfpathlineto{\pgfqpoint{1.281847in}{2.112132in}}%
\pgfpathlineto{\pgfqpoint{1.343395in}{2.109887in}}%
\pgfpathlineto{\pgfqpoint{1.409724in}{2.105579in}}%
\pgfpathlineto{\pgfqpoint{1.481090in}{2.099144in}}%
\pgfpathlineto{\pgfqpoint{1.557745in}{2.090513in}}%
\pgfpathlineto{\pgfqpoint{1.639940in}{2.079611in}}%
\pgfpathlineto{\pgfqpoint{1.727853in}{2.066365in}}%
\pgfpathlineto{\pgfqpoint{1.821454in}{2.050692in}}%
\pgfpathlineto{\pgfqpoint{1.920544in}{2.032496in}}%
\pgfpathlineto{\pgfqpoint{2.024620in}{2.011722in}}%
\pgfpathlineto{\pgfqpoint{2.132875in}{1.988347in}}%
\pgfpathlineto{\pgfqpoint{2.244192in}{1.962388in}}%
\pgfpathlineto{\pgfqpoint{2.357144in}{1.933900in}}%
\pgfpathlineto{\pgfqpoint{2.470002in}{1.902974in}}%
\pgfpathlineto{\pgfqpoint{2.580760in}{1.869729in}}%
\pgfpathlineto{\pgfqpoint{2.687398in}{1.834372in}}%
\pgfpathlineto{\pgfqpoint{2.788140in}{1.797236in}}%
\pgfpathlineto{\pgfqpoint{2.881589in}{1.758654in}}%
\pgfpathlineto{\pgfqpoint{2.966739in}{1.718958in}}%
\pgfpathlineto{\pgfqpoint{3.005992in}{1.698793in}}%
\pgfpathlineto{\pgfqpoint{3.042967in}{1.678473in}}%
\pgfpathlineto{\pgfqpoint{3.077649in}{1.658035in}}%
\pgfpathlineto{\pgfqpoint{3.110043in}{1.637522in}}%
\pgfpathlineto{\pgfqpoint{3.140181in}{1.616972in}}%
\pgfpathlineto{\pgfqpoint{3.168119in}{1.596425in}}%
\pgfpathlineto{\pgfqpoint{3.193936in}{1.575920in}}%
\pgfpathlineto{\pgfqpoint{3.217575in}{1.555511in}}%
\pgfpathlineto{\pgfqpoint{3.239105in}{1.535235in}}%
\pgfpathlineto{\pgfqpoint{3.276689in}{1.495158in}}%
\pgfpathlineto{\pgfqpoint{3.308150in}{1.455838in}}%
\pgfpathlineto{\pgfqpoint{3.334575in}{1.417412in}}%
\pgfpathlineto{\pgfqpoint{3.356675in}{1.380007in}}%
\pgfpathlineto{\pgfqpoint{3.374786in}{1.343736in}}%
\pgfpathlineto{\pgfqpoint{3.388865in}{1.308707in}}%
\pgfpathlineto{\pgfqpoint{3.398495in}{1.275011in}}%
\pgfpathlineto{\pgfqpoint{3.401408in}{1.258690in}}%
\pgfpathlineto{\pgfqpoint{3.403558in}{1.227153in}}%
\pgfpathlineto{\pgfqpoint{3.402480in}{1.197136in}}%
\pgfpathlineto{\pgfqpoint{3.398534in}{1.168668in}}%
\pgfpathlineto{\pgfqpoint{3.391907in}{1.141769in}}%
\pgfpathlineto{\pgfqpoint{3.382714in}{1.116456in}}%
\pgfpathlineto{\pgfqpoint{3.371006in}{1.092741in}}%
\pgfpathlineto{\pgfqpoint{3.356765in}{1.070631in}}%
\pgfpathlineto{\pgfqpoint{3.339906in}{1.050129in}}%
\pgfpathlineto{\pgfqpoint{3.320328in}{1.031234in}}%
\pgfpathlineto{\pgfqpoint{3.298114in}{1.013960in}}%
\pgfpathlineto{\pgfqpoint{3.273247in}{0.998322in}}%
\pgfpathlineto{\pgfqpoint{3.245663in}{0.984337in}}%
\pgfpathlineto{\pgfqpoint{3.215272in}{0.972023in}}%
\pgfpathlineto{\pgfqpoint{3.181958in}{0.961400in}}%
\pgfpathlineto{\pgfqpoint{3.145577in}{0.952491in}}%
\pgfpathlineto{\pgfqpoint{3.105956in}{0.945319in}}%
\pgfpathlineto{\pgfqpoint{3.062897in}{0.939911in}}%
\pgfpathlineto{\pgfqpoint{3.016175in}{0.936293in}}%
\pgfpathlineto{\pgfqpoint{2.965536in}{0.934496in}}%
\pgfpathlineto{\pgfqpoint{2.910832in}{0.934530in}}%
\pgfpathlineto{\pgfqpoint{2.852047in}{0.936406in}}%
\pgfpathlineto{\pgfqpoint{2.788555in}{0.940250in}}%
\pgfpathlineto{\pgfqpoint{2.719843in}{0.946185in}}%
\pgfpathlineto{\pgfqpoint{2.645578in}{0.954316in}}%
\pgfpathlineto{\pgfqpoint{2.565603in}{0.964730in}}%
\pgfpathlineto{\pgfqpoint{2.479941in}{0.977499in}}%
\pgfpathlineto{\pgfqpoint{2.388790in}{0.992677in}}%
\pgfpathlineto{\pgfqpoint{2.292529in}{1.010301in}}%
\pgfpathlineto{\pgfqpoint{2.191711in}{1.030392in}}%
\pgfpathlineto{\pgfqpoint{2.087071in}{1.052954in}}%
\pgfpathlineto{\pgfqpoint{1.979518in}{1.077973in}}%
\pgfpathlineto{\pgfqpoint{1.870141in}{1.105419in}}%
\pgfpathlineto{\pgfqpoint{1.760206in}{1.135246in}}%
\pgfpathlineto{\pgfqpoint{1.651377in}{1.167394in}}%
\pgfpathlineto{\pgfqpoint{1.546017in}{1.201674in}}%
\pgfpathlineto{\pgfqpoint{1.445864in}{1.237760in}}%
\pgfpathlineto{\pgfqpoint{1.352292in}{1.275326in}}%
\pgfpathlineto{\pgfqpoint{1.266351in}{1.314053in}}%
\pgfpathlineto{\pgfqpoint{1.188764in}{1.353628in}}%
\pgfpathlineto{\pgfqpoint{1.153239in}{1.373638in}}%
\pgfpathlineto{\pgfqpoint{1.119925in}{1.393747in}}%
\pgfpathlineto{\pgfqpoint{1.088822in}{1.413916in}}%
\pgfpathlineto{\pgfqpoint{1.059907in}{1.434109in}}%
\pgfpathlineto{\pgfqpoint{1.033138in}{1.454290in}}%
\pgfpathlineto{\pgfqpoint{1.008453in}{1.474422in}}%
\pgfpathlineto{\pgfqpoint{0.985768in}{1.494471in}}%
\pgfpathlineto{\pgfqpoint{0.965070in}{1.514392in}}%
\pgfpathlineto{\pgfqpoint{0.946335in}{1.534144in}}%
\pgfpathlineto{\pgfqpoint{0.913958in}{1.573072in}}%
\pgfpathlineto{\pgfqpoint{0.887244in}{1.611125in}}%
\pgfpathlineto{\pgfqpoint{0.865145in}{1.648185in}}%
\pgfpathlineto{\pgfqpoint{0.846961in}{1.684145in}}%
\pgfpathlineto{\pgfqpoint{0.832338in}{1.718910in}}%
\pgfpathlineto{\pgfqpoint{0.821271in}{1.752394in}}%
\pgfpathlineto{\pgfqpoint{0.814101in}{1.784527in}}%
\pgfpathlineto{\pgfqpoint{0.812179in}{1.800067in}}%
\pgfpathlineto{\pgfqpoint{0.811517in}{1.815247in}}%
\pgfpathlineto{\pgfqpoint{0.813371in}{1.844488in}}%
\pgfpathlineto{\pgfqpoint{0.818160in}{1.872192in}}%
\pgfpathlineto{\pgfqpoint{0.825671in}{1.898337in}}%
\pgfpathlineto{\pgfqpoint{0.835772in}{1.922906in}}%
\pgfpathlineto{\pgfqpoint{0.848385in}{1.945884in}}%
\pgfpathlineto{\pgfqpoint{0.863493in}{1.967261in}}%
\pgfpathlineto{\pgfqpoint{0.881134in}{1.987031in}}%
\pgfpathlineto{\pgfqpoint{0.901403in}{2.005189in}}%
\pgfpathlineto{\pgfqpoint{0.924446in}{2.021736in}}%
\pgfpathlineto{\pgfqpoint{0.950231in}{2.036660in}}%
\pgfpathlineto{\pgfqpoint{0.978783in}{2.049939in}}%
\pgfpathlineto{\pgfqpoint{1.010213in}{2.061555in}}%
\pgfpathlineto{\pgfqpoint{1.044650in}{2.071483in}}%
\pgfpathlineto{\pgfqpoint{1.082245in}{2.079699in}}%
\pgfpathlineto{\pgfqpoint{1.123172in}{2.086171in}}%
\pgfpathlineto{\pgfqpoint{1.167625in}{2.090868in}}%
\pgfpathlineto{\pgfqpoint{1.215820in}{2.093753in}}%
\pgfpathlineto{\pgfqpoint{1.267994in}{2.094789in}}%
\pgfpathlineto{\pgfqpoint{1.324406in}{2.093931in}}%
\pgfpathlineto{\pgfqpoint{1.385335in}{2.091137in}}%
\pgfpathlineto{\pgfqpoint{1.450975in}{2.086350in}}%
\pgfpathlineto{\pgfqpoint{1.521635in}{2.079485in}}%
\pgfpathlineto{\pgfqpoint{1.597669in}{2.070454in}}%
\pgfpathlineto{\pgfqpoint{1.679260in}{2.059175in}}%
\pgfpathlineto{\pgfqpoint{1.766423in}{2.045578in}}%
\pgfpathlineto{\pgfqpoint{1.859003in}{2.029597in}}%
\pgfpathlineto{\pgfqpoint{1.956672in}{2.011178in}}%
\pgfpathlineto{\pgfqpoint{2.058936in}{1.990274in}}%
\pgfpathlineto{\pgfqpoint{2.165130in}{1.966845in}}%
\pgfpathlineto{\pgfqpoint{2.274430in}{1.940859in}}%
\pgfpathlineto{\pgfqpoint{2.385034in}{1.912392in}}%
\pgfpathlineto{\pgfqpoint{2.494924in}{1.881588in}}%
\pgfpathlineto{\pgfqpoint{2.602270in}{1.848626in}}%
\pgfpathlineto{\pgfqpoint{2.705425in}{1.813717in}}%
\pgfpathlineto{\pgfqpoint{2.802928in}{1.777109in}}%
\pgfpathlineto{\pgfqpoint{2.893503in}{1.739084in}}%
\pgfpathlineto{\pgfqpoint{2.976060in}{1.699956in}}%
\pgfpathlineto{\pgfqpoint{3.014042in}{1.680087in}}%
\pgfpathlineto{\pgfqpoint{3.049694in}{1.660077in}}%
\pgfpathlineto{\pgfqpoint{3.082990in}{1.639976in}}%
\pgfpathlineto{\pgfqpoint{3.114037in}{1.619822in}}%
\pgfpathlineto{\pgfqpoint{3.142954in}{1.599652in}}%
\pgfpathlineto{\pgfqpoint{3.169849in}{1.579498in}}%
\pgfpathlineto{\pgfqpoint{3.217967in}{1.539369in}}%
\pgfpathlineto{\pgfqpoint{3.259088in}{1.499672in}}%
\pgfpathlineto{\pgfqpoint{3.293775in}{1.460622in}}%
\pgfpathlineto{\pgfqpoint{3.322452in}{1.422407in}}%
\pgfpathlineto{\pgfqpoint{3.345408in}{1.385191in}}%
\pgfpathlineto{\pgfqpoint{3.362842in}{1.349113in}}%
\pgfpathlineto{\pgfqpoint{3.375599in}{1.314291in}}%
\pgfpathlineto{\pgfqpoint{3.384403in}{1.280800in}}%
\pgfpathlineto{\pgfqpoint{3.389717in}{1.248700in}}%
\pgfpathlineto{\pgfqpoint{3.391879in}{1.218041in}}%
\pgfpathlineto{\pgfqpoint{3.391099in}{1.188862in}}%
\pgfpathlineto{\pgfqpoint{3.387460in}{1.161191in}}%
\pgfpathlineto{\pgfqpoint{3.380918in}{1.135049in}}%
\pgfpathlineto{\pgfqpoint{3.371369in}{1.110447in}}%
\pgfpathlineto{\pgfqpoint{3.359070in}{1.087406in}}%
\pgfpathlineto{\pgfqpoint{3.344162in}{1.065945in}}%
\pgfpathlineto{\pgfqpoint{3.326700in}{1.046080in}}%
\pgfpathlineto{\pgfqpoint{3.306698in}{1.027826in}}%
\pgfpathlineto{\pgfqpoint{3.284132in}{1.011195in}}%
\pgfpathlineto{\pgfqpoint{3.258938in}{0.996201in}}%
\pgfpathlineto{\pgfqpoint{3.231009in}{0.982853in}}%
\pgfpathlineto{\pgfqpoint{3.200202in}{0.971161in}}%
\pgfpathlineto{\pgfqpoint{3.166332in}{0.961133in}}%
\pgfpathlineto{\pgfqpoint{3.129215in}{0.952779in}}%
\pgfpathlineto{\pgfqpoint{3.088898in}{0.946134in}}%
\pgfpathlineto{\pgfqpoint{3.045118in}{0.941239in}}%
\pgfpathlineto{\pgfqpoint{2.997565in}{0.938139in}}%
\pgfpathlineto{\pgfqpoint{2.945951in}{0.936879in}}%
\pgfpathlineto{\pgfqpoint{2.890012in}{0.937513in}}%
\pgfpathlineto{\pgfqpoint{2.829506in}{0.940096in}}%
\pgfpathlineto{\pgfqpoint{2.764216in}{0.944689in}}%
\pgfpathlineto{\pgfqpoint{2.693947in}{0.951354in}}%
\pgfpathlineto{\pgfqpoint{2.618526in}{0.960162in}}%
\pgfpathlineto{\pgfqpoint{2.537805in}{0.971185in}}%
\pgfpathlineto{\pgfqpoint{2.451658in}{0.984500in}}%
\pgfpathlineto{\pgfqpoint{2.311537in}{1.008980in}}%
\pgfpathlineto{\pgfqpoint{2.209884in}{1.028506in}}%
\pgfpathlineto{\pgfqpoint{2.103533in}{1.050606in}}%
\pgfpathlineto{\pgfqpoint{1.994456in}{1.075228in}}%
\pgfpathlineto{\pgfqpoint{1.884468in}{1.102291in}}%
\pgfpathlineto{\pgfqpoint{1.775228in}{1.131682in}}%
\pgfpathlineto{\pgfqpoint{1.668236in}{1.163259in}}%
\pgfpathlineto{\pgfqpoint{1.564835in}{1.196850in}}%
\pgfpathlineto{\pgfqpoint{1.466211in}{1.232254in}}%
\pgfpathlineto{\pgfqpoint{1.373393in}{1.269238in}}%
\pgfpathlineto{\pgfqpoint{1.287252in}{1.307540in}}%
\pgfpathlineto{\pgfqpoint{1.208501in}{1.346869in}}%
\pgfpathlineto{\pgfqpoint{1.137698in}{1.386903in}}%
\pgfpathlineto{\pgfqpoint{1.105408in}{1.407075in}}%
\pgfpathlineto{\pgfqpoint{1.075242in}{1.427289in}}%
\pgfpathlineto{\pgfqpoint{1.047226in}{1.447497in}}%
\pgfpathlineto{\pgfqpoint{1.021374in}{1.467647in}}%
\pgfpathlineto{\pgfqpoint{0.997694in}{1.487687in}}%
\pgfpathlineto{\pgfqpoint{0.976069in}{1.507587in}}%
\pgfpathlineto{\pgfqpoint{0.938113in}{1.546952in}}%
\pgfpathlineto{\pgfqpoint{0.906629in}{1.585571in}}%
\pgfpathlineto{\pgfqpoint{0.880971in}{1.623272in}}%
\pgfpathlineto{\pgfqpoint{0.860543in}{1.659912in}}%
\pgfpathlineto{\pgfqpoint{0.844799in}{1.695376in}}%
\pgfpathlineto{\pgfqpoint{0.833274in}{1.729570in}}%
\pgfpathlineto{\pgfqpoint{0.825591in}{1.762419in}}%
\pgfpathlineto{\pgfqpoint{0.821585in}{1.793864in}}%
\pgfpathlineto{\pgfqpoint{0.820920in}{1.823862in}}%
\pgfpathlineto{\pgfqpoint{0.823224in}{1.852374in}}%
\pgfpathlineto{\pgfqpoint{0.828228in}{1.879371in}}%
\pgfpathlineto{\pgfqpoint{0.835759in}{1.904827in}}%
\pgfpathlineto{\pgfqpoint{0.845743in}{1.928727in}}%
\pgfpathlineto{\pgfqpoint{0.858204in}{1.951059in}}%
\pgfpathlineto{\pgfqpoint{0.873268in}{1.971820in}}%
\pgfpathlineto{\pgfqpoint{0.891156in}{1.991012in}}%
\pgfpathlineto{\pgfqpoint{0.912078in}{2.008637in}}%
\pgfpathlineto{\pgfqpoint{0.935730in}{2.024657in}}%
\pgfpathlineto{\pgfqpoint{0.962108in}{2.039052in}}%
\pgfpathlineto{\pgfqpoint{0.991292in}{2.051801in}}%
\pgfpathlineto{\pgfqpoint{1.023388in}{2.062884in}}%
\pgfpathlineto{\pgfqpoint{1.058526in}{2.072277in}}%
\pgfpathlineto{\pgfqpoint{1.096863in}{2.079955in}}%
\pgfpathlineto{\pgfqpoint{1.138578in}{2.085889in}}%
\pgfpathlineto{\pgfqpoint{1.183877in}{2.090050in}}%
\pgfpathlineto{\pgfqpoint{1.232990in}{2.092405in}}%
\pgfpathlineto{\pgfqpoint{1.286170in}{2.092919in}}%
\pgfpathlineto{\pgfqpoint{1.343492in}{2.091571in}}%
\pgfpathlineto{\pgfqpoint{1.405223in}{2.088294in}}%
\pgfpathlineto{\pgfqpoint{1.471915in}{2.082984in}}%
\pgfpathlineto{\pgfqpoint{1.543959in}{2.075545in}}%
\pgfpathlineto{\pgfqpoint{1.621576in}{2.065893in}}%
\pgfpathlineto{\pgfqpoint{1.704822in}{2.053957in}}%
\pgfpathlineto{\pgfqpoint{1.793588in}{2.039674in}}%
\pgfpathlineto{\pgfqpoint{1.887597in}{2.022995in}}%
\pgfpathlineto{\pgfqpoint{1.986405in}{2.003881in}}%
\pgfpathlineto{\pgfqpoint{2.089404in}{1.982305in}}%
\pgfpathlineto{\pgfqpoint{2.195817in}{1.958249in}}%
\pgfpathlineto{\pgfqpoint{2.304703in}{1.931710in}}%
\pgfpathlineto{\pgfqpoint{2.414803in}{1.902681in}}%
\pgfpathlineto{\pgfqpoint{2.524018in}{1.871264in}}%
\pgfpathlineto{\pgfqpoint{2.630359in}{1.837724in}}%
\pgfpathlineto{\pgfqpoint{2.732125in}{1.802337in}}%
\pgfpathlineto{\pgfqpoint{2.827896in}{1.765387in}}%
\pgfpathlineto{\pgfqpoint{2.916529in}{1.727160in}}%
\pgfpathlineto{\pgfqpoint{2.997160in}{1.687951in}}%
\pgfpathlineto{\pgfqpoint{3.034279in}{1.668072in}}%
\pgfpathlineto{\pgfqpoint{3.069200in}{1.648061in}}%
\pgfpathlineto{\pgfqpoint{3.101893in}{1.627956in}}%
\pgfpathlineto{\pgfqpoint{3.132344in}{1.607796in}}%
\pgfpathlineto{\pgfqpoint{3.160558in}{1.587620in}}%
\pgfpathlineto{\pgfqpoint{3.186559in}{1.567467in}}%
\pgfpathlineto{\pgfqpoint{3.210384in}{1.547380in}}%
\pgfpathlineto{\pgfqpoint{3.232118in}{1.527408in}}%
\pgfpathlineto{\pgfqpoint{3.270027in}{1.487916in}}%
\pgfpathlineto{\pgfqpoint{3.301605in}{1.449161in}}%
\pgfpathlineto{\pgfqpoint{3.327876in}{1.411289in}}%
\pgfpathlineto{\pgfqpoint{3.349561in}{1.374434in}}%
\pgfpathlineto{\pgfqpoint{3.367074in}{1.338712in}}%
\pgfpathlineto{\pgfqpoint{3.380523in}{1.304222in}}%
\pgfpathlineto{\pgfqpoint{3.389709in}{1.271051in}}%
\pgfpathlineto{\pgfqpoint{3.392558in}{1.254982in}}%
\pgfpathlineto{\pgfqpoint{3.394963in}{1.223922in}}%
\pgfpathlineto{\pgfqpoint{3.394138in}{1.194350in}}%
\pgfpathlineto{\pgfqpoint{3.390397in}{1.166295in}}%
\pgfpathlineto{\pgfqpoint{3.383936in}{1.139784in}}%
\pgfpathlineto{\pgfqpoint{3.374883in}{1.114834in}}%
\pgfpathlineto{\pgfqpoint{3.363292in}{1.091462in}}%
\pgfpathlineto{\pgfqpoint{3.349152in}{1.069677in}}%
\pgfpathlineto{\pgfqpoint{3.332377in}{1.049482in}}%
\pgfpathlineto{\pgfqpoint{3.312834in}{1.030878in}}%
\pgfpathlineto{\pgfqpoint{3.290601in}{1.013879in}}%
\pgfpathlineto{\pgfqpoint{3.265687in}{0.998503in}}%
\pgfpathlineto{\pgfqpoint{3.238028in}{0.984768in}}%
\pgfpathlineto{\pgfqpoint{3.207533in}{0.972693in}}%
\pgfpathlineto{\pgfqpoint{3.174085in}{0.962301in}}%
\pgfpathlineto{\pgfqpoint{3.137541in}{0.953614in}}%
\pgfpathlineto{\pgfqpoint{3.097732in}{0.946660in}}%
\pgfpathlineto{\pgfqpoint{3.054463in}{0.941465in}}%
\pgfpathlineto{\pgfqpoint{3.007515in}{0.938060in}}%
\pgfpathlineto{\pgfqpoint{2.956639in}{0.936476in}}%
\pgfpathlineto{\pgfqpoint{2.901609in}{0.936744in}}%
\pgfpathlineto{\pgfqpoint{2.842427in}{0.938887in}}%
\pgfpathlineto{\pgfqpoint{2.778573in}{0.943005in}}%
\pgfpathlineto{\pgfqpoint{2.709556in}{0.949204in}}%
\pgfpathlineto{\pgfqpoint{2.635051in}{0.957576in}}%
\pgfpathlineto{\pgfqpoint{2.554906in}{0.968200in}}%
\pgfpathlineto{\pgfqpoint{2.469137in}{0.981145in}}%
\pgfpathlineto{\pgfqpoint{2.377930in}{0.996466in}}%
\pgfpathlineto{\pgfqpoint{2.281642in}{1.014208in}}%
\pgfpathlineto{\pgfqpoint{2.180798in}{1.034402in}}%
\pgfpathlineto{\pgfqpoint{2.076096in}{1.057068in}}%
\pgfpathlineto{\pgfqpoint{1.968400in}{1.082212in}}%
\pgfpathlineto{\pgfqpoint{1.858745in}{1.109831in}}%
\pgfpathlineto{\pgfqpoint{1.748658in}{1.139920in}}%
\pgfpathlineto{\pgfqpoint{1.640348in}{1.172313in}}%
\pgfpathlineto{\pgfqpoint{1.535675in}{1.206726in}}%
\pgfpathlineto{\pgfqpoint{1.436216in}{1.242869in}}%
\pgfpathlineto{\pgfqpoint{1.343271in}{1.280451in}}%
\pgfpathlineto{\pgfqpoint{1.257859in}{1.319177in}}%
\pgfpathlineto{\pgfqpoint{1.180722in}{1.358750in}}%
\pgfpathlineto{\pgfqpoint{1.145411in}{1.378759in}}%
\pgfpathlineto{\pgfqpoint{1.112322in}{1.398868in}}%
\pgfpathlineto{\pgfqpoint{1.081467in}{1.419037in}}%
\pgfpathlineto{\pgfqpoint{1.052844in}{1.439229in}}%
\pgfpathlineto{\pgfqpoint{1.026431in}{1.459405in}}%
\pgfpathlineto{\pgfqpoint{1.002192in}{1.479527in}}%
\pgfpathlineto{\pgfqpoint{0.980097in}{1.499548in}}%
\pgfpathlineto{\pgfqpoint{0.960093in}{1.519415in}}%
\pgfpathlineto{\pgfqpoint{0.925490in}{1.558614in}}%
\pgfpathlineto{\pgfqpoint{0.896828in}{1.596994in}}%
\pgfpathlineto{\pgfqpoint{0.872922in}{1.634438in}}%
\pgfpathlineto{\pgfqpoint{0.852963in}{1.670838in}}%
\pgfpathlineto{\pgfqpoint{0.836520in}{1.706096in}}%
\pgfpathlineto{\pgfqpoint{0.823542in}{1.740120in}}%
\pgfpathlineto{\pgfqpoint{0.814354in}{1.772831in}}%
\pgfpathlineto{\pgfqpoint{0.811386in}{1.788670in}}%
\pgfpathlineto{\pgfqpoint{0.809659in}{1.804154in}}%
\pgfpathlineto{\pgfqpoint{0.809321in}{1.819275in}}%
\pgfpathlineto{\pgfqpoint{0.811738in}{1.848384in}}%
\pgfpathlineto{\pgfqpoint{0.817025in}{1.875945in}}%
\pgfpathlineto{\pgfqpoint{0.825002in}{1.901936in}}%
\pgfpathlineto{\pgfqpoint{0.835546in}{1.926344in}}%
\pgfpathlineto{\pgfqpoint{0.848593in}{1.949153in}}%
\pgfpathlineto{\pgfqpoint{0.864132in}{1.970355in}}%
\pgfpathlineto{\pgfqpoint{0.882209in}{1.989944in}}%
\pgfpathlineto{\pgfqpoint{0.902929in}{2.007918in}}%
\pgfpathlineto{\pgfqpoint{0.926419in}{2.024276in}}%
\pgfpathlineto{\pgfqpoint{0.952630in}{2.039003in}}%
\pgfpathlineto{\pgfqpoint{0.981618in}{2.052081in}}%
\pgfpathlineto{\pgfqpoint{1.013497in}{2.063488in}}%
\pgfpathlineto{\pgfqpoint{1.048399in}{2.073203in}}%
\pgfpathlineto{\pgfqpoint{1.086479in}{2.081197in}}%
\pgfpathlineto{\pgfqpoint{1.127913in}{2.087442in}}%
\pgfpathlineto{\pgfqpoint{1.172901in}{2.091905in}}%
\pgfpathlineto{\pgfqpoint{1.221661in}{2.094549in}}%
\pgfpathlineto{\pgfqpoint{1.274435in}{2.095336in}}%
\pgfpathlineto{\pgfqpoint{1.331486in}{2.094223in}}%
\pgfpathlineto{\pgfqpoint{1.393095in}{2.091166in}}%
\pgfpathlineto{\pgfqpoint{1.459430in}{2.086105in}}%
\pgfpathlineto{\pgfqpoint{1.530854in}{2.078952in}}%
\pgfpathlineto{\pgfqpoint{1.607695in}{2.069617in}}%
\pgfpathlineto{\pgfqpoint{1.690114in}{2.058023in}}%
\pgfpathlineto{\pgfqpoint{1.778099in}{2.044099in}}%
\pgfpathlineto{\pgfqpoint{1.871470in}{2.027781in}}%
\pgfpathlineto{\pgfqpoint{1.969876in}{2.009016in}}%
\pgfpathlineto{\pgfqpoint{2.072794in}{1.987759in}}%
\pgfpathlineto{\pgfqpoint{2.179538in}{1.963970in}}%
\pgfpathlineto{\pgfqpoint{2.289221in}{1.937623in}}%
\pgfpathlineto{\pgfqpoint{2.399948in}{1.908806in}}%
\pgfpathlineto{\pgfqpoint{2.509701in}{1.877672in}}%
\pgfpathlineto{\pgfqpoint{2.616656in}{1.844406in}}%
\pgfpathlineto{\pgfqpoint{2.719185in}{1.809229in}}%
\pgfpathlineto{\pgfqpoint{2.815857in}{1.772390in}}%
\pgfpathlineto{\pgfqpoint{2.905436in}{1.734176in}}%
\pgfpathlineto{\pgfqpoint{2.986880in}{1.694903in}}%
\pgfpathlineto{\pgfqpoint{3.024280in}{1.674978in}}%
\pgfpathlineto{\pgfqpoint{3.059343in}{1.654923in}}%
\pgfpathlineto{\pgfqpoint{3.092049in}{1.634789in}}%
\pgfpathlineto{\pgfqpoint{3.122510in}{1.614614in}}%
\pgfpathlineto{\pgfqpoint{3.150851in}{1.594432in}}%
\pgfpathlineto{\pgfqpoint{3.201613in}{1.554180in}}%
\pgfpathlineto{\pgfqpoint{3.245135in}{1.514281in}}%
\pgfpathlineto{\pgfqpoint{3.282061in}{1.474954in}}%
\pgfpathlineto{\pgfqpoint{3.312887in}{1.436396in}}%
\pgfpathlineto{\pgfqpoint{3.337956in}{1.398781in}}%
\pgfpathlineto{\pgfqpoint{3.357459in}{1.362255in}}%
\pgfpathlineto{\pgfqpoint{3.371809in}{1.326945in}}%
\pgfpathlineto{\pgfqpoint{3.381978in}{1.292944in}}%
\pgfpathlineto{\pgfqpoint{3.388486in}{1.260315in}}%
\pgfpathlineto{\pgfqpoint{3.391723in}{1.229112in}}%
\pgfpathlineto{\pgfqpoint{3.391953in}{1.199376in}}%
\pgfpathlineto{\pgfqpoint{3.389316in}{1.171142in}}%
\pgfpathlineto{\pgfqpoint{3.383832in}{1.144433in}}%
\pgfpathlineto{\pgfqpoint{3.375395in}{1.119260in}}%
\pgfpathlineto{\pgfqpoint{3.364073in}{1.095640in}}%
\pgfpathlineto{\pgfqpoint{3.350105in}{1.073594in}}%
\pgfpathlineto{\pgfqpoint{3.333563in}{1.053138in}}%
\pgfpathlineto{\pgfqpoint{3.314478in}{1.034287in}}%
\pgfpathlineto{\pgfqpoint{3.292841in}{1.017056in}}%
\pgfpathlineto{\pgfqpoint{3.268603in}{1.001457in}}%
\pgfpathlineto{\pgfqpoint{3.241678in}{0.987501in}}%
\pgfpathlineto{\pgfqpoint{3.211938in}{0.975199in}}%
\pgfpathlineto{\pgfqpoint{3.179216in}{0.964561in}}%
\pgfpathlineto{\pgfqpoint{3.143306in}{0.955594in}}%
\pgfpathlineto{\pgfqpoint{3.104150in}{0.948317in}}%
\pgfpathlineto{\pgfqpoint{3.061676in}{0.942775in}}%
\pgfpathlineto{\pgfqpoint{3.015554in}{0.939011in}}%
\pgfpathlineto{\pgfqpoint{2.965475in}{0.937069in}}%
\pgfpathlineto{\pgfqpoint{2.911159in}{0.937002in}}%
\pgfpathlineto{\pgfqpoint{2.852353in}{0.938863in}}%
\pgfpathlineto{\pgfqpoint{2.788827in}{0.942711in}}%
\pgfpathlineto{\pgfqpoint{2.720381in}{0.948610in}}%
\pgfpathlineto{\pgfqpoint{2.646839in}{0.956625in}}%
\pgfpathlineto{\pgfqpoint{2.568051in}{0.966829in}}%
\pgfpathlineto{\pgfqpoint{2.483894in}{0.979295in}}%
\pgfpathlineto{\pgfqpoint{2.394272in}{0.994103in}}%
\pgfpathlineto{\pgfqpoint{2.248280in}{1.020983in}}%
\pgfpathlineto{\pgfqpoint{2.143469in}{1.042168in}}%
\pgfpathlineto{\pgfqpoint{2.035058in}{1.065914in}}%
\pgfpathlineto{\pgfqpoint{1.924999in}{1.092138in}}%
\pgfpathlineto{\pgfqpoint{1.815073in}{1.120730in}}%
\pgfpathlineto{\pgfqpoint{1.706889in}{1.151552in}}%
\pgfpathlineto{\pgfqpoint{1.601883in}{1.184438in}}%
\pgfpathlineto{\pgfqpoint{1.501322in}{1.219196in}}%
\pgfpathlineto{\pgfqpoint{1.406299in}{1.255604in}}%
\pgfpathlineto{\pgfqpoint{1.317738in}{1.293414in}}%
\pgfpathlineto{\pgfqpoint{1.236388in}{1.332351in}}%
\pgfpathlineto{\pgfqpoint{1.162831in}{1.372109in}}%
\pgfpathlineto{\pgfqpoint{1.129108in}{1.392194in}}%
\pgfpathlineto{\pgfqpoint{1.097473in}{1.412358in}}%
\pgfpathlineto{\pgfqpoint{1.067949in}{1.432555in}}%
\pgfpathlineto{\pgfqpoint{1.040551in}{1.452738in}}%
\pgfpathlineto{\pgfqpoint{1.015280in}{1.472857in}}%
\pgfpathlineto{\pgfqpoint{0.992129in}{1.492862in}}%
\pgfpathlineto{\pgfqpoint{0.951593in}{1.532444in}}%
\pgfpathlineto{\pgfqpoint{0.917780in}{1.571352in}}%
\pgfpathlineto{\pgfqpoint{0.889997in}{1.609407in}}%
\pgfpathlineto{\pgfqpoint{0.867640in}{1.646453in}}%
\pgfpathlineto{\pgfqpoint{0.850192in}{1.682363in}}%
\pgfpathlineto{\pgfqpoint{0.837154in}{1.717034in}}%
\pgfpathlineto{\pgfqpoint{0.828082in}{1.750387in}}%
\pgfpathlineto{\pgfqpoint{0.822625in}{1.782358in}}%
\pgfpathlineto{\pgfqpoint{0.820509in}{1.812894in}}%
\pgfpathlineto{\pgfqpoint{0.821569in}{1.841959in}}%
\pgfpathlineto{\pgfqpoint{0.825716in}{1.869527in}}%
\pgfpathlineto{\pgfqpoint{0.832695in}{1.895570in}}%
\pgfpathlineto{\pgfqpoint{0.842303in}{1.920063in}}%
\pgfpathlineto{\pgfqpoint{0.854403in}{1.942987in}}%
\pgfpathlineto{\pgfqpoint{0.868926in}{1.964326in}}%
\pgfpathlineto{\pgfqpoint{0.885875in}{1.984070in}}%
\pgfpathlineto{\pgfqpoint{0.905317in}{2.002212in}}%
\pgfpathlineto{\pgfqpoint{0.927391in}{2.018749in}}%
\pgfpathlineto{\pgfqpoint{0.952303in}{2.033685in}}%
\pgfpathlineto{\pgfqpoint{0.980303in}{2.047022in}}%
\pgfpathlineto{\pgfqpoint{1.011257in}{2.058725in}}%
\pgfpathlineto{\pgfqpoint{1.045194in}{2.068760in}}%
\pgfpathlineto{\pgfqpoint{1.082285in}{2.077100in}}%
\pgfpathlineto{\pgfqpoint{1.122714in}{2.083716in}}%
\pgfpathlineto{\pgfqpoint{1.166678in}{2.088572in}}%
\pgfpathlineto{\pgfqpoint{1.214387in}{2.091628in}}%
\pgfpathlineto{\pgfqpoint{1.266066in}{2.092842in}}%
\pgfpathlineto{\pgfqpoint{1.321952in}{2.092165in}}%
\pgfpathlineto{\pgfqpoint{1.382296in}{2.089545in}}%
\pgfpathlineto{\pgfqpoint{1.447362in}{2.084926in}}%
\pgfpathlineto{\pgfqpoint{1.517427in}{2.078247in}}%
\pgfpathlineto{\pgfqpoint{1.592705in}{2.069437in}}%
\pgfpathlineto{\pgfqpoint{1.673416in}{2.058406in}}%
\pgfpathlineto{\pgfqpoint{1.759690in}{2.045072in}}%
\pgfpathlineto{\pgfqpoint{1.851445in}{2.029364in}}%
\pgfpathlineto{\pgfqpoint{1.948389in}{2.011222in}}%
\pgfpathlineto{\pgfqpoint{2.050021in}{1.990599in}}%
\pgfpathlineto{\pgfqpoint{2.155634in}{1.967462in}}%
\pgfpathlineto{\pgfqpoint{2.264273in}{1.941804in}}%
\pgfpathlineto{\pgfqpoint{2.374300in}{1.913683in}}%
\pgfpathlineto{\pgfqpoint{2.483937in}{1.883207in}}%
\pgfpathlineto{\pgfqpoint{2.591443in}{1.850532in}}%
\pgfpathlineto{\pgfqpoint{2.695113in}{1.815864in}}%
\pgfpathlineto{\pgfqpoint{2.793279in}{1.779455in}}%
\pgfpathlineto{\pgfqpoint{2.884244in}{1.741607in}}%
\pgfpathlineto{\pgfqpoint{2.926331in}{1.722254in}}%
\pgfpathlineto{\pgfqpoint{2.966110in}{1.702676in}}%
\pgfpathlineto{\pgfqpoint{3.003674in}{1.682915in}}%
\pgfpathlineto{\pgfqpoint{3.072493in}{1.643004in}}%
\pgfpathlineto{\pgfqpoint{3.133410in}{1.602828in}}%
\pgfpathlineto{\pgfqpoint{3.186967in}{1.562667in}}%
\pgfpathlineto{\pgfqpoint{3.233618in}{1.522773in}}%
\pgfpathlineto{\pgfqpoint{3.273736in}{1.483374in}}%
\pgfpathlineto{\pgfqpoint{3.307609in}{1.444669in}}%
\pgfpathlineto{\pgfqpoint{3.335442in}{1.406832in}}%
\pgfpathlineto{\pgfqpoint{3.357356in}{1.370011in}}%
\pgfpathlineto{\pgfqpoint{3.366107in}{1.352020in}}%
\pgfpathlineto{\pgfqpoint{3.373386in}{1.334326in}}%
\pgfpathlineto{\pgfqpoint{3.384003in}{1.299903in}}%
\pgfpathlineto{\pgfqpoint{3.390540in}{1.266863in}}%
\pgfpathlineto{\pgfqpoint{3.393675in}{1.235257in}}%
\pgfpathlineto{\pgfqpoint{3.393913in}{1.205126in}}%
\pgfpathlineto{\pgfqpoint{3.391603in}{1.176503in}}%
\pgfpathlineto{\pgfqpoint{3.386938in}{1.149413in}}%
\pgfpathlineto{\pgfqpoint{3.379954in}{1.123871in}}%
\pgfpathlineto{\pgfqpoint{3.370533in}{1.099885in}}%
\pgfpathlineto{\pgfqpoint{3.358397in}{1.077455in}}%
\pgfpathlineto{\pgfqpoint{3.343115in}{1.056572in}}%
\pgfpathlineto{\pgfqpoint{3.324455in}{1.037242in}}%
\pgfpathlineto{\pgfqpoint{3.303131in}{1.019531in}}%
\pgfpathlineto{\pgfqpoint{3.279163in}{1.003453in}}%
\pgfpathlineto{\pgfqpoint{3.252484in}{0.989021in}}%
\pgfpathlineto{\pgfqpoint{3.223001in}{0.976249in}}%
\pgfpathlineto{\pgfqpoint{3.190599in}{0.965152in}}%
\pgfpathlineto{\pgfqpoint{3.155137in}{0.955748in}}%
\pgfpathlineto{\pgfqpoint{3.155137in}{0.955748in}}%
\pgfusepath{stroke}%
\end{pgfscope}%
\begin{pgfscope}%
\pgfpathrectangle{\pgfqpoint{0.562500in}{0.275000in}}{\pgfqpoint{3.487500in}{1.925000in}}%
\pgfusepath{clip}%
\pgfsetrectcap%
\pgfsetroundjoin%
\pgfsetlinewidth{1.505625pt}%
\definecolor{currentstroke}{rgb}{0.172549,0.627451,0.172549}%
\pgfsetstrokecolor{currentstroke}%
\pgfsetdash{}{0pt}%
\pgfpathmoveto{\pgfqpoint{3.891477in}{0.364857in}}%
\pgfpathlineto{\pgfqpoint{3.767322in}{0.371162in}}%
\pgfpathlineto{\pgfqpoint{3.643286in}{0.380738in}}%
\pgfpathlineto{\pgfqpoint{3.516972in}{0.393469in}}%
\pgfpathlineto{\pgfqpoint{3.386330in}{0.409297in}}%
\pgfpathlineto{\pgfqpoint{3.249607in}{0.428213in}}%
\pgfpathlineto{\pgfqpoint{3.105355in}{0.450255in}}%
\pgfpathlineto{\pgfqpoint{2.952425in}{0.475510in}}%
\pgfpathlineto{\pgfqpoint{2.790033in}{0.504090in}}%
\pgfpathlineto{\pgfqpoint{2.618220in}{0.536110in}}%
\pgfpathlineto{\pgfqpoint{2.438194in}{0.571681in}}%
\pgfpathlineto{\pgfqpoint{2.252447in}{0.610821in}}%
\pgfpathlineto{\pgfqpoint{2.064761in}{0.653447in}}%
\pgfpathlineto{\pgfqpoint{1.971725in}{0.676015in}}%
\pgfpathlineto{\pgfqpoint{1.880206in}{0.699378in}}%
\pgfpathlineto{\pgfqpoint{1.790964in}{0.723504in}}%
\pgfpathlineto{\pgfqpoint{1.704684in}{0.748327in}}%
\pgfpathlineto{\pgfqpoint{1.621968in}{0.773768in}}%
\pgfpathlineto{\pgfqpoint{1.543306in}{0.799752in}}%
\pgfpathlineto{\pgfqpoint{1.469086in}{0.826202in}}%
\pgfpathlineto{\pgfqpoint{1.399590in}{0.853041in}}%
\pgfpathlineto{\pgfqpoint{1.334990in}{0.880195in}}%
\pgfpathlineto{\pgfqpoint{1.275356in}{0.907589in}}%
\pgfpathlineto{\pgfqpoint{1.220650in}{0.935148in}}%
\pgfpathlineto{\pgfqpoint{1.170728in}{0.962799in}}%
\pgfpathlineto{\pgfqpoint{1.125341in}{0.990469in}}%
\pgfpathlineto{\pgfqpoint{1.084175in}{1.018080in}}%
\pgfpathlineto{\pgfqpoint{1.046915in}{1.045571in}}%
\pgfpathlineto{\pgfqpoint{1.013185in}{1.072902in}}%
\pgfpathlineto{\pgfqpoint{0.982643in}{1.100036in}}%
\pgfpathlineto{\pgfqpoint{0.954988in}{1.126936in}}%
\pgfpathlineto{\pgfqpoint{0.929953in}{1.153567in}}%
\pgfpathlineto{\pgfqpoint{0.907310in}{1.179895in}}%
\pgfpathlineto{\pgfqpoint{0.886868in}{1.205888in}}%
\pgfpathlineto{\pgfqpoint{0.868473in}{1.231513in}}%
\pgfpathlineto{\pgfqpoint{0.836906in}{1.281588in}}%
\pgfpathlineto{\pgfqpoint{0.810914in}{1.330046in}}%
\pgfpathlineto{\pgfqpoint{0.789481in}{1.376834in}}%
\pgfpathlineto{\pgfqpoint{0.771856in}{1.421908in}}%
\pgfpathlineto{\pgfqpoint{0.757542in}{1.465230in}}%
\pgfpathlineto{\pgfqpoint{0.746067in}{1.506794in}}%
\pgfpathlineto{\pgfqpoint{0.737040in}{1.546626in}}%
\pgfpathlineto{\pgfqpoint{0.730175in}{1.584754in}}%
\pgfpathlineto{\pgfqpoint{0.725275in}{1.621202in}}%
\pgfpathlineto{\pgfqpoint{0.722231in}{1.655990in}}%
\pgfpathlineto{\pgfqpoint{0.721023in}{1.689137in}}%
\pgfpathlineto{\pgfqpoint{0.721650in}{1.720662in}}%
\pgfpathlineto{\pgfqpoint{0.723930in}{1.750601in}}%
\pgfpathlineto{\pgfqpoint{0.727748in}{1.778991in}}%
\pgfpathlineto{\pgfqpoint{0.733030in}{1.805866in}}%
\pgfpathlineto{\pgfqpoint{0.739735in}{1.831259in}}%
\pgfpathlineto{\pgfqpoint{0.747853in}{1.855199in}}%
\pgfpathlineto{\pgfqpoint{0.757409in}{1.877714in}}%
\pgfpathlineto{\pgfqpoint{0.768459in}{1.898830in}}%
\pgfpathlineto{\pgfqpoint{0.781089in}{1.918573in}}%
\pgfpathlineto{\pgfqpoint{0.795307in}{1.936972in}}%
\pgfpathlineto{\pgfqpoint{0.811031in}{1.954048in}}%
\pgfpathlineto{\pgfqpoint{0.828200in}{1.969818in}}%
\pgfpathlineto{\pgfqpoint{0.846784in}{1.984300in}}%
\pgfpathlineto{\pgfqpoint{0.866783in}{1.997511in}}%
\pgfpathlineto{\pgfqpoint{0.888226in}{2.009468in}}%
\pgfpathlineto{\pgfqpoint{0.911169in}{2.020187in}}%
\pgfpathlineto{\pgfqpoint{0.935703in}{2.029683in}}%
\pgfpathlineto{\pgfqpoint{0.961942in}{2.037971in}}%
\pgfpathlineto{\pgfqpoint{0.990034in}{2.045065in}}%
\pgfpathlineto{\pgfqpoint{1.020156in}{2.050979in}}%
\pgfpathlineto{\pgfqpoint{1.052477in}{2.055723in}}%
\pgfpathlineto{\pgfqpoint{1.104919in}{2.060609in}}%
\pgfpathlineto{\pgfqpoint{1.162500in}{2.062789in}}%
\pgfpathlineto{\pgfqpoint{1.225817in}{2.062218in}}%
\pgfpathlineto{\pgfqpoint{1.295494in}{2.058836in}}%
\pgfpathlineto{\pgfqpoint{1.372184in}{2.052557in}}%
\pgfpathlineto{\pgfqpoint{1.456566in}{2.043278in}}%
\pgfpathlineto{\pgfqpoint{1.549349in}{2.030873in}}%
\pgfpathlineto{\pgfqpoint{1.651265in}{2.015197in}}%
\pgfpathlineto{\pgfqpoint{1.763038in}{1.996083in}}%
\pgfpathlineto{\pgfqpoint{1.885004in}{1.973344in}}%
\pgfpathlineto{\pgfqpoint{2.016961in}{1.946798in}}%
\pgfpathlineto{\pgfqpoint{2.157755in}{1.916327in}}%
\pgfpathlineto{\pgfqpoint{2.305228in}{1.881874in}}%
\pgfpathlineto{\pgfqpoint{2.405637in}{1.856717in}}%
\pgfpathlineto{\pgfqpoint{2.506085in}{1.829882in}}%
\pgfpathlineto{\pgfqpoint{2.605109in}{1.801472in}}%
\pgfpathlineto{\pgfqpoint{2.701264in}{1.771623in}}%
\pgfpathlineto{\pgfqpoint{2.793118in}{1.740510in}}%
\pgfpathlineto{\pgfqpoint{2.879180in}{1.708336in}}%
\pgfpathlineto{\pgfqpoint{2.957777in}{1.675331in}}%
\pgfpathlineto{\pgfqpoint{3.028846in}{1.641762in}}%
\pgfpathlineto{\pgfqpoint{3.092650in}{1.607879in}}%
\pgfpathlineto{\pgfqpoint{3.149461in}{1.573914in}}%
\pgfpathlineto{\pgfqpoint{3.199562in}{1.540077in}}%
\pgfpathlineto{\pgfqpoint{3.243242in}{1.506557in}}%
\pgfpathlineto{\pgfqpoint{3.280802in}{1.473521in}}%
\pgfpathlineto{\pgfqpoint{3.312550in}{1.441116in}}%
\pgfpathlineto{\pgfqpoint{3.338807in}{1.409467in}}%
\pgfpathlineto{\pgfqpoint{3.359899in}{1.378680in}}%
\pgfpathlineto{\pgfqpoint{3.376163in}{1.348837in}}%
\pgfpathlineto{\pgfqpoint{3.387966in}{1.320024in}}%
\pgfpathlineto{\pgfqpoint{3.395960in}{1.292344in}}%
\pgfpathlineto{\pgfqpoint{3.400912in}{1.265819in}}%
\pgfpathlineto{\pgfqpoint{3.403438in}{1.240461in}}%
\pgfpathlineto{\pgfqpoint{3.403997in}{1.216280in}}%
\pgfpathlineto{\pgfqpoint{3.402891in}{1.193281in}}%
\pgfpathlineto{\pgfqpoint{3.400266in}{1.171468in}}%
\pgfpathlineto{\pgfqpoint{3.396109in}{1.150837in}}%
\pgfpathlineto{\pgfqpoint{3.390253in}{1.131385in}}%
\pgfpathlineto{\pgfqpoint{3.382373in}{1.113102in}}%
\pgfpathlineto{\pgfqpoint{3.371985in}{1.095977in}}%
\pgfpathlineto{\pgfqpoint{3.358613in}{1.079998in}}%
\pgfpathlineto{\pgfqpoint{3.343209in}{1.065205in}}%
\pgfpathlineto{\pgfqpoint{3.326102in}{1.051600in}}%
\pgfpathlineto{\pgfqpoint{3.307313in}{1.039177in}}%
\pgfpathlineto{\pgfqpoint{3.286843in}{1.027930in}}%
\pgfpathlineto{\pgfqpoint{3.264674in}{1.017852in}}%
\pgfpathlineto{\pgfqpoint{3.240765in}{1.008939in}}%
\pgfpathlineto{\pgfqpoint{3.215056in}{1.001186in}}%
\pgfpathlineto{\pgfqpoint{3.187465in}{0.994587in}}%
\pgfpathlineto{\pgfqpoint{3.157892in}{0.989138in}}%
\pgfpathlineto{\pgfqpoint{3.109551in}{0.983113in}}%
\pgfpathlineto{\pgfqpoint{3.056322in}{0.979683in}}%
\pgfpathlineto{\pgfqpoint{2.997732in}{0.978895in}}%
\pgfpathlineto{\pgfqpoint{2.933101in}{0.980808in}}%
\pgfpathlineto{\pgfqpoint{2.861765in}{0.985501in}}%
\pgfpathlineto{\pgfqpoint{2.783078in}{0.993073in}}%
\pgfpathlineto{\pgfqpoint{2.696411in}{1.003639in}}%
\pgfpathlineto{\pgfqpoint{2.601155in}{1.017335in}}%
\pgfpathlineto{\pgfqpoint{2.496715in}{1.034313in}}%
\pgfpathlineto{\pgfqpoint{2.382515in}{1.054746in}}%
\pgfpathlineto{\pgfqpoint{2.258043in}{1.078787in}}%
\pgfpathlineto{\pgfqpoint{2.123993in}{1.106606in}}%
\pgfpathlineto{\pgfqpoint{1.982376in}{1.138363in}}%
\pgfpathlineto{\pgfqpoint{1.835946in}{1.174056in}}%
\pgfpathlineto{\pgfqpoint{1.737346in}{1.199961in}}%
\pgfpathlineto{\pgfqpoint{1.639360in}{1.227452in}}%
\pgfpathlineto{\pgfqpoint{1.543350in}{1.256404in}}%
\pgfpathlineto{\pgfqpoint{1.450824in}{1.286659in}}%
\pgfpathlineto{\pgfqpoint{1.363435in}{1.318027in}}%
\pgfpathlineto{\pgfqpoint{1.282760in}{1.350310in}}%
\pgfpathlineto{\pgfqpoint{1.209239in}{1.383264in}}%
\pgfpathlineto{\pgfqpoint{1.142889in}{1.416629in}}%
\pgfpathlineto{\pgfqpoint{1.083650in}{1.450165in}}%
\pgfpathlineto{\pgfqpoint{1.031385in}{1.483656in}}%
\pgfpathlineto{\pgfqpoint{0.985880in}{1.516904in}}%
\pgfpathlineto{\pgfqpoint{0.946845in}{1.549735in}}%
\pgfpathlineto{\pgfqpoint{0.913914in}{1.581994in}}%
\pgfpathlineto{\pgfqpoint{0.886641in}{1.613549in}}%
\pgfpathlineto{\pgfqpoint{0.864507in}{1.644289in}}%
\pgfpathlineto{\pgfqpoint{0.846965in}{1.674117in}}%
\pgfpathlineto{\pgfqpoint{0.833865in}{1.702911in}}%
\pgfpathlineto{\pgfqpoint{0.824458in}{1.730625in}}%
\pgfpathlineto{\pgfqpoint{0.817961in}{1.757235in}}%
\pgfpathlineto{\pgfqpoint{0.813775in}{1.782724in}}%
\pgfpathlineto{\pgfqpoint{0.811486in}{1.807078in}}%
\pgfpathlineto{\pgfqpoint{0.810865in}{1.830286in}}%
\pgfpathlineto{\pgfqpoint{0.811866in}{1.852341in}}%
\pgfpathlineto{\pgfqpoint{0.814629in}{1.873239in}}%
\pgfpathlineto{\pgfqpoint{0.819478in}{1.892982in}}%
\pgfpathlineto{\pgfqpoint{0.826920in}{1.911574in}}%
\pgfpathlineto{\pgfqpoint{0.837593in}{1.929020in}}%
\pgfpathlineto{\pgfqpoint{0.850632in}{1.945287in}}%
\pgfpathlineto{\pgfqpoint{0.865410in}{1.960360in}}%
\pgfpathlineto{\pgfqpoint{0.881882in}{1.974248in}}%
\pgfpathlineto{\pgfqpoint{0.900025in}{1.986956in}}%
\pgfpathlineto{\pgfqpoint{0.919839in}{1.998491in}}%
\pgfpathlineto{\pgfqpoint{0.941345in}{2.008859in}}%
\pgfpathlineto{\pgfqpoint{0.964589in}{2.018068in}}%
\pgfpathlineto{\pgfqpoint{0.989638in}{2.026123in}}%
\pgfpathlineto{\pgfqpoint{1.016581in}{2.033032in}}%
\pgfpathlineto{\pgfqpoint{1.045530in}{2.038800in}}%
\pgfpathlineto{\pgfqpoint{1.092834in}{2.045315in}}%
\pgfpathlineto{\pgfqpoint{1.144973in}{2.049235in}}%
\pgfpathlineto{\pgfqpoint{1.202470in}{2.050524in}}%
\pgfpathlineto{\pgfqpoint{1.265884in}{2.049126in}}%
\pgfpathlineto{\pgfqpoint{1.335803in}{2.044967in}}%
\pgfpathlineto{\pgfqpoint{1.412845in}{2.037955in}}%
\pgfpathlineto{\pgfqpoint{1.497660in}{2.027978in}}%
\pgfpathlineto{\pgfqpoint{1.590927in}{2.014907in}}%
\pgfpathlineto{\pgfqpoint{1.693353in}{1.998594in}}%
\pgfpathlineto{\pgfqpoint{1.805567in}{1.978870in}}%
\pgfpathlineto{\pgfqpoint{1.927763in}{1.955545in}}%
\pgfpathlineto{\pgfqpoint{2.059532in}{1.928457in}}%
\pgfpathlineto{\pgfqpoint{2.199507in}{1.897505in}}%
\pgfpathlineto{\pgfqpoint{2.345367in}{1.862655in}}%
\pgfpathlineto{\pgfqpoint{2.444145in}{1.837297in}}%
\pgfpathlineto{\pgfqpoint{2.542503in}{1.810327in}}%
\pgfpathlineto{\pgfqpoint{2.639027in}{1.781854in}}%
\pgfpathlineto{\pgfqpoint{2.732337in}{1.752024in}}%
\pgfpathlineto{\pgfqpoint{2.821086in}{1.721013in}}%
\pgfpathlineto{\pgfqpoint{2.903883in}{1.689031in}}%
\pgfpathlineto{\pgfqpoint{2.979197in}{1.656313in}}%
\pgfpathlineto{\pgfqpoint{3.047087in}{1.623115in}}%
\pgfpathlineto{\pgfqpoint{3.107897in}{1.589672in}}%
\pgfpathlineto{\pgfqpoint{3.161956in}{1.556203in}}%
\pgfpathlineto{\pgfqpoint{3.209583in}{1.522907in}}%
\pgfpathlineto{\pgfqpoint{3.251085in}{1.489961in}}%
\pgfpathlineto{\pgfqpoint{3.286758in}{1.457526in}}%
\pgfpathlineto{\pgfqpoint{3.316886in}{1.425742in}}%
\pgfpathlineto{\pgfqpoint{3.341741in}{1.394727in}}%
\pgfpathlineto{\pgfqpoint{3.361584in}{1.364582in}}%
\pgfpathlineto{\pgfqpoint{3.376665in}{1.335389in}}%
\pgfpathlineto{\pgfqpoint{3.387341in}{1.307233in}}%
\pgfpathlineto{\pgfqpoint{3.394406in}{1.280203in}}%
\pgfpathlineto{\pgfqpoint{3.398556in}{1.254320in}}%
\pgfpathlineto{\pgfqpoint{3.400342in}{1.229597in}}%
\pgfpathlineto{\pgfqpoint{3.400170in}{1.206047in}}%
\pgfpathlineto{\pgfqpoint{3.398306in}{1.183674in}}%
\pgfpathlineto{\pgfqpoint{3.394872in}{1.162481in}}%
\pgfpathlineto{\pgfqpoint{3.389847in}{1.142466in}}%
\pgfpathlineto{\pgfqpoint{3.383067in}{1.123624in}}%
\pgfpathlineto{\pgfqpoint{3.374227in}{1.105944in}}%
\pgfpathlineto{\pgfqpoint{3.362879in}{1.089413in}}%
\pgfpathlineto{\pgfqpoint{3.348750in}{1.074024in}}%
\pgfpathlineto{\pgfqpoint{3.332759in}{1.059815in}}%
\pgfpathlineto{\pgfqpoint{3.315058in}{1.046783in}}%
\pgfpathlineto{\pgfqpoint{3.295660in}{1.034923in}}%
\pgfpathlineto{\pgfqpoint{3.274564in}{1.024230in}}%
\pgfpathlineto{\pgfqpoint{3.251742in}{1.014699in}}%
\pgfpathlineto{\pgfqpoint{3.227151in}{1.006325in}}%
\pgfpathlineto{\pgfqpoint{3.200724in}{0.999104in}}%
\pgfpathlineto{\pgfqpoint{3.172378in}{0.993032in}}%
\pgfpathlineto{\pgfqpoint{3.142005in}{0.988108in}}%
\pgfpathlineto{\pgfqpoint{3.092382in}{0.982864in}}%
\pgfpathlineto{\pgfqpoint{3.037789in}{0.980216in}}%
\pgfpathlineto{\pgfqpoint{2.977680in}{0.980215in}}%
\pgfpathlineto{\pgfqpoint{2.911338in}{0.982931in}}%
\pgfpathlineto{\pgfqpoint{2.838079in}{0.988447in}}%
\pgfpathlineto{\pgfqpoint{2.757256in}{0.996869in}}%
\pgfpathlineto{\pgfqpoint{2.668254in}{1.008318in}}%
\pgfpathlineto{\pgfqpoint{2.570495in}{1.022934in}}%
\pgfpathlineto{\pgfqpoint{2.463434in}{1.040873in}}%
\pgfpathlineto{\pgfqpoint{2.346560in}{1.062312in}}%
\pgfpathlineto{\pgfqpoint{2.219345in}{1.087405in}}%
\pgfpathlineto{\pgfqpoint{2.082771in}{1.116303in}}%
\pgfpathlineto{\pgfqpoint{1.939491in}{1.149140in}}%
\pgfpathlineto{\pgfqpoint{1.792660in}{1.185882in}}%
\pgfpathlineto{\pgfqpoint{1.694576in}{1.212449in}}%
\pgfpathlineto{\pgfqpoint{1.597719in}{1.240557in}}%
\pgfpathlineto{\pgfqpoint{1.503383in}{1.270068in}}%
\pgfpathlineto{\pgfqpoint{1.412956in}{1.300808in}}%
\pgfpathlineto{\pgfqpoint{1.327924in}{1.332571in}}%
\pgfpathlineto{\pgfqpoint{1.249840in}{1.365119in}}%
\pgfpathlineto{\pgfqpoint{1.179464in}{1.398215in}}%
\pgfpathlineto{\pgfqpoint{1.116519in}{1.431622in}}%
\pgfpathlineto{\pgfqpoint{1.060665in}{1.465117in}}%
\pgfpathlineto{\pgfqpoint{1.011563in}{1.498495in}}%
\pgfpathlineto{\pgfqpoint{0.968875in}{1.531570in}}%
\pgfpathlineto{\pgfqpoint{0.932263in}{1.564175in}}%
\pgfpathlineto{\pgfqpoint{0.901390in}{1.596160in}}%
\pgfpathlineto{\pgfqpoint{0.875920in}{1.627395in}}%
\pgfpathlineto{\pgfqpoint{0.855516in}{1.657766in}}%
\pgfpathlineto{\pgfqpoint{0.839812in}{1.687173in}}%
\pgfpathlineto{\pgfqpoint{0.828145in}{1.715528in}}%
\pgfpathlineto{\pgfqpoint{0.819813in}{1.742793in}}%
\pgfpathlineto{\pgfqpoint{0.814264in}{1.768940in}}%
\pgfpathlineto{\pgfqpoint{0.811082in}{1.793947in}}%
\pgfpathlineto{\pgfqpoint{0.809999in}{1.817795in}}%
\pgfpathlineto{\pgfqpoint{0.810885in}{1.840475in}}%
\pgfpathlineto{\pgfqpoint{0.813753in}{1.861979in}}%
\pgfpathlineto{\pgfqpoint{0.818759in}{1.882305in}}%
\pgfpathlineto{\pgfqpoint{0.826201in}{1.901460in}}%
\pgfpathlineto{\pgfqpoint{0.836218in}{1.919442in}}%
\pgfpathlineto{\pgfqpoint{0.848132in}{1.936234in}}%
\pgfpathlineto{\pgfqpoint{0.861805in}{1.951840in}}%
\pgfpathlineto{\pgfqpoint{0.877181in}{1.966265in}}%
\pgfpathlineto{\pgfqpoint{0.894225in}{1.979515in}}%
\pgfpathlineto{\pgfqpoint{0.912933in}{1.991596in}}%
\pgfpathlineto{\pgfqpoint{0.933321in}{2.002514in}}%
\pgfpathlineto{\pgfqpoint{0.955435in}{2.012277in}}%
\pgfpathlineto{\pgfqpoint{0.979344in}{2.020894in}}%
\pgfpathlineto{\pgfqpoint{1.005144in}{2.028371in}}%
\pgfpathlineto{\pgfqpoint{1.032955in}{2.034717in}}%
\pgfpathlineto{\pgfqpoint{1.062775in}{2.039929in}}%
\pgfpathlineto{\pgfqpoint{1.111387in}{2.045603in}}%
\pgfpathlineto{\pgfqpoint{1.165045in}{2.048678in}}%
\pgfpathlineto{\pgfqpoint{1.224234in}{2.049111in}}%
\pgfpathlineto{\pgfqpoint{1.289489in}{2.046840in}}%
\pgfpathlineto{\pgfqpoint{1.361394in}{2.041787in}}%
\pgfpathlineto{\pgfqpoint{1.440583in}{2.033855in}}%
\pgfpathlineto{\pgfqpoint{1.527739in}{2.022930in}}%
\pgfpathlineto{\pgfqpoint{1.623599in}{2.008880in}}%
\pgfpathlineto{\pgfqpoint{1.728871in}{1.991548in}}%
\pgfpathlineto{\pgfqpoint{1.844010in}{1.970746in}}%
\pgfpathlineto{\pgfqpoint{1.969016in}{1.946301in}}%
\pgfpathlineto{\pgfqpoint{2.103245in}{1.918065in}}%
\pgfpathlineto{\pgfqpoint{2.245091in}{1.885946in}}%
\pgfpathlineto{\pgfqpoint{2.391758in}{1.849955in}}%
\pgfpathlineto{\pgfqpoint{2.490320in}{1.823882in}}%
\pgfpathlineto{\pgfqpoint{2.587881in}{1.796201in}}%
\pgfpathlineto{\pgfqpoint{2.682907in}{1.767037in}}%
\pgfpathlineto{\pgfqpoint{2.773997in}{1.736614in}}%
\pgfpathlineto{\pgfqpoint{2.859981in}{1.705156in}}%
\pgfpathlineto{\pgfqpoint{2.939924in}{1.672881in}}%
\pgfpathlineto{\pgfqpoint{3.013121in}{1.640007in}}%
\pgfpathlineto{\pgfqpoint{3.079101in}{1.606744in}}%
\pgfpathlineto{\pgfqpoint{3.137627in}{1.573304in}}%
\pgfpathlineto{\pgfqpoint{3.188693in}{1.539891in}}%
\pgfpathlineto{\pgfqpoint{3.232525in}{1.506710in}}%
\pgfpathlineto{\pgfqpoint{3.269479in}{1.473977in}}%
\pgfpathlineto{\pgfqpoint{3.300047in}{1.441911in}}%
\pgfpathlineto{\pgfqpoint{3.325395in}{1.410600in}}%
\pgfpathlineto{\pgfqpoint{3.346500in}{1.380116in}}%
\pgfpathlineto{\pgfqpoint{3.364080in}{1.350526in}}%
\pgfpathlineto{\pgfqpoint{3.378594in}{1.321891in}}%
\pgfpathlineto{\pgfqpoint{3.390242in}{1.294269in}}%
\pgfpathlineto{\pgfqpoint{3.398965in}{1.267712in}}%
\pgfpathlineto{\pgfqpoint{3.404444in}{1.242266in}}%
\pgfpathlineto{\pgfqpoint{3.406101in}{1.217975in}}%
\pgfpathlineto{\pgfqpoint{3.404343in}{1.194876in}}%
\pgfpathlineto{\pgfqpoint{3.400385in}{1.172988in}}%
\pgfpathlineto{\pgfqpoint{3.394413in}{1.152309in}}%
\pgfpathlineto{\pgfqpoint{3.386568in}{1.132839in}}%
\pgfpathlineto{\pgfqpoint{3.376947in}{1.114573in}}%
\pgfpathlineto{\pgfqpoint{3.365604in}{1.097508in}}%
\pgfpathlineto{\pgfqpoint{3.352552in}{1.081633in}}%
\pgfpathlineto{\pgfqpoint{3.337759in}{1.066940in}}%
\pgfpathlineto{\pgfqpoint{3.321151in}{1.053415in}}%
\pgfpathlineto{\pgfqpoint{3.302692in}{1.041049in}}%
\pgfpathlineto{\pgfqpoint{3.282476in}{1.029838in}}%
\pgfpathlineto{\pgfqpoint{3.260505in}{1.019780in}}%
\pgfpathlineto{\pgfqpoint{3.236762in}{1.010873in}}%
\pgfpathlineto{\pgfqpoint{3.211209in}{1.003116in}}%
\pgfpathlineto{\pgfqpoint{3.183790in}{0.996509in}}%
\pgfpathlineto{\pgfqpoint{3.154431in}{0.991053in}}%
\pgfpathlineto{\pgfqpoint{3.106549in}{0.985034in}}%
\pgfpathlineto{\pgfqpoint{3.053700in}{0.981625in}}%
\pgfpathlineto{\pgfqpoint{2.995399in}{0.980844in}}%
\pgfpathlineto{\pgfqpoint{2.931177in}{0.982710in}}%
\pgfpathlineto{\pgfqpoint{2.860983in}{0.987253in}}%
\pgfpathlineto{\pgfqpoint{2.783532in}{0.994640in}}%
\pgfpathlineto{\pgfqpoint{2.697621in}{1.005048in}}%
\pgfpathlineto{\pgfqpoint{2.602443in}{1.018641in}}%
\pgfpathlineto{\pgfqpoint{2.497581in}{1.035568in}}%
\pgfpathlineto{\pgfqpoint{2.383011in}{1.055967in}}%
\pgfpathlineto{\pgfqpoint{2.259103in}{1.079960in}}%
\pgfpathlineto{\pgfqpoint{2.126619in}{1.107657in}}%
\pgfpathlineto{\pgfqpoint{1.986713in}{1.139155in}}%
\pgfpathlineto{\pgfqpoint{1.840952in}{1.174537in}}%
\pgfpathlineto{\pgfqpoint{1.742155in}{1.200285in}}%
\pgfpathlineto{\pgfqpoint{1.644084in}{1.227633in}}%
\pgfpathlineto{\pgfqpoint{1.548330in}{1.256428in}}%
\pgfpathlineto{\pgfqpoint{1.456280in}{1.286503in}}%
\pgfpathlineto{\pgfqpoint{1.369120in}{1.317677in}}%
\pgfpathlineto{\pgfqpoint{1.287829in}{1.349755in}}%
\pgfpathlineto{\pgfqpoint{1.213186in}{1.382528in}}%
\pgfpathlineto{\pgfqpoint{1.145767in}{1.415773in}}%
\pgfpathlineto{\pgfqpoint{1.085941in}{1.449251in}}%
\pgfpathlineto{\pgfqpoint{1.033991in}{1.482690in}}%
\pgfpathlineto{\pgfqpoint{0.989498in}{1.515843in}}%
\pgfpathlineto{\pgfqpoint{0.951312in}{1.548580in}}%
\pgfpathlineto{\pgfqpoint{0.918494in}{1.580779in}}%
\pgfpathlineto{\pgfqpoint{0.890338in}{1.612327in}}%
\pgfpathlineto{\pgfqpoint{0.866375in}{1.643117in}}%
\pgfpathlineto{\pgfqpoint{0.846369in}{1.673051in}}%
\pgfpathlineto{\pgfqpoint{0.830319in}{1.702037in}}%
\pgfpathlineto{\pgfqpoint{0.818459in}{1.729988in}}%
\pgfpathlineto{\pgfqpoint{0.811193in}{1.756829in}}%
\pgfpathlineto{\pgfqpoint{0.807396in}{1.782507in}}%
\pgfpathlineto{\pgfqpoint{0.806190in}{1.807004in}}%
\pgfpathlineto{\pgfqpoint{0.807302in}{1.830310in}}%
\pgfpathlineto{\pgfqpoint{0.810524in}{1.852416in}}%
\pgfpathlineto{\pgfqpoint{0.815711in}{1.873320in}}%
\pgfpathlineto{\pgfqpoint{0.822783in}{1.893021in}}%
\pgfpathlineto{\pgfqpoint{0.831723in}{1.911524in}}%
\pgfpathlineto{\pgfqpoint{0.842577in}{1.928837in}}%
\pgfpathlineto{\pgfqpoint{0.855397in}{1.944969in}}%
\pgfpathlineto{\pgfqpoint{0.870025in}{1.959924in}}%
\pgfpathlineto{\pgfqpoint{0.886387in}{1.973706in}}%
\pgfpathlineto{\pgfqpoint{0.904447in}{1.986319in}}%
\pgfpathlineto{\pgfqpoint{0.924188in}{1.997768in}}%
\pgfpathlineto{\pgfqpoint{0.945618in}{2.008056in}}%
\pgfpathlineto{\pgfqpoint{0.968771in}{2.017189in}}%
\pgfpathlineto{\pgfqpoint{0.993701in}{2.025171in}}%
\pgfpathlineto{\pgfqpoint{1.020487in}{2.032006in}}%
\pgfpathlineto{\pgfqpoint{1.049230in}{2.037700in}}%
\pgfpathlineto{\pgfqpoint{1.096294in}{2.044110in}}%
\pgfpathlineto{\pgfqpoint{1.148382in}{2.047961in}}%
\pgfpathlineto{\pgfqpoint{1.205671in}{2.049199in}}%
\pgfpathlineto{\pgfqpoint{1.268854in}{2.047764in}}%
\pgfpathlineto{\pgfqpoint{1.338621in}{2.043576in}}%
\pgfpathlineto{\pgfqpoint{1.415633in}{2.036542in}}%
\pgfpathlineto{\pgfqpoint{1.500518in}{2.026548in}}%
\pgfpathlineto{\pgfqpoint{1.593875in}{2.013463in}}%
\pgfpathlineto{\pgfqpoint{1.696272in}{1.997136in}}%
\pgfpathlineto{\pgfqpoint{1.808247in}{1.977402in}}%
\pgfpathlineto{\pgfqpoint{1.930343in}{1.954086in}}%
\pgfpathlineto{\pgfqpoint{2.062477in}{1.927059in}}%
\pgfpathlineto{\pgfqpoint{2.202582in}{1.896136in}}%
\pgfpathlineto{\pgfqpoint{2.347856in}{1.861278in}}%
\pgfpathlineto{\pgfqpoint{2.445941in}{1.835909in}}%
\pgfpathlineto{\pgfqpoint{2.543709in}{1.808925in}}%
\pgfpathlineto{\pgfqpoint{2.639914in}{1.780442in}}%
\pgfpathlineto{\pgfqpoint{2.733203in}{1.750607in}}%
\pgfpathlineto{\pgfqpoint{2.822110in}{1.719603in}}%
\pgfpathlineto{\pgfqpoint{2.905059in}{1.687646in}}%
\pgfpathlineto{\pgfqpoint{2.980478in}{1.654976in}}%
\pgfpathlineto{\pgfqpoint{3.048228in}{1.621835in}}%
\pgfpathlineto{\pgfqpoint{3.108771in}{1.588453in}}%
\pgfpathlineto{\pgfqpoint{3.162532in}{1.555041in}}%
\pgfpathlineto{\pgfqpoint{3.209897in}{1.521794in}}%
\pgfpathlineto{\pgfqpoint{3.251216in}{1.488889in}}%
\pgfpathlineto{\pgfqpoint{3.286803in}{1.456484in}}%
\pgfpathlineto{\pgfqpoint{3.316932in}{1.424722in}}%
\pgfpathlineto{\pgfqpoint{3.341843in}{1.393725in}}%
\pgfpathlineto{\pgfqpoint{3.361736in}{1.363601in}}%
\pgfpathlineto{\pgfqpoint{3.376775in}{1.334438in}}%
\pgfpathlineto{\pgfqpoint{3.387381in}{1.306326in}}%
\pgfpathlineto{\pgfqpoint{3.394506in}{1.279335in}}%
\pgfpathlineto{\pgfqpoint{3.398767in}{1.253487in}}%
\pgfpathlineto{\pgfqpoint{3.400645in}{1.228801in}}%
\pgfpathlineto{\pgfqpoint{3.400488in}{1.205287in}}%
\pgfpathlineto{\pgfqpoint{3.398519in}{1.182953in}}%
\pgfpathlineto{\pgfqpoint{3.394828in}{1.161804in}}%
\pgfpathlineto{\pgfqpoint{3.389377in}{1.141837in}}%
\pgfpathlineto{\pgfqpoint{3.381998in}{1.123048in}}%
\pgfpathlineto{\pgfqpoint{3.372393in}{1.105425in}}%
\pgfpathlineto{\pgfqpoint{3.360189in}{1.088957in}}%
\pgfpathlineto{\pgfqpoint{3.345928in}{1.073664in}}%
\pgfpathlineto{\pgfqpoint{3.329931in}{1.059551in}}%
\pgfpathlineto{\pgfqpoint{3.312234in}{1.046613in}}%
\pgfpathlineto{\pgfqpoint{3.292850in}{1.034843in}}%
\pgfpathlineto{\pgfqpoint{3.271770in}{1.024237in}}%
\pgfpathlineto{\pgfqpoint{3.248966in}{1.014790in}}%
\pgfpathlineto{\pgfqpoint{3.224385in}{1.006496in}}%
\pgfpathlineto{\pgfqpoint{3.197955in}{0.999352in}}%
\pgfpathlineto{\pgfqpoint{3.169581in}{0.993351in}}%
\pgfpathlineto{\pgfqpoint{3.139147in}{0.988490in}}%
\pgfpathlineto{\pgfqpoint{3.089455in}{0.983334in}}%
\pgfpathlineto{\pgfqpoint{3.034768in}{0.980778in}}%
\pgfpathlineto{\pgfqpoint{2.974470in}{0.980872in}}%
\pgfpathlineto{\pgfqpoint{2.907923in}{0.983681in}}%
\pgfpathlineto{\pgfqpoint{2.834489in}{0.989291in}}%
\pgfpathlineto{\pgfqpoint{2.753538in}{0.997805in}}%
\pgfpathlineto{\pgfqpoint{2.664443in}{1.009345in}}%
\pgfpathlineto{\pgfqpoint{2.566584in}{1.024053in}}%
\pgfpathlineto{\pgfqpoint{2.459343in}{1.042088in}}%
\pgfpathlineto{\pgfqpoint{2.342136in}{1.063622in}}%
\pgfpathlineto{\pgfqpoint{2.214976in}{1.088813in}}%
\pgfpathlineto{\pgfqpoint{2.078845in}{1.117854in}}%
\pgfpathlineto{\pgfqpoint{1.935696in}{1.150818in}}%
\pgfpathlineto{\pgfqpoint{1.788483in}{1.187636in}}%
\pgfpathlineto{\pgfqpoint{1.689995in}{1.214225in}}%
\pgfpathlineto{\pgfqpoint{1.592856in}{1.242338in}}%
\pgfpathlineto{\pgfqpoint{1.498679in}{1.271850in}}%
\pgfpathlineto{\pgfqpoint{1.409001in}{1.302623in}}%
\pgfpathlineto{\pgfqpoint{1.324911in}{1.334419in}}%
\pgfpathlineto{\pgfqpoint{1.247237in}{1.366980in}}%
\pgfpathlineto{\pgfqpoint{1.176565in}{1.400064in}}%
\pgfpathlineto{\pgfqpoint{1.113239in}{1.433443in}}%
\pgfpathlineto{\pgfqpoint{1.057362in}{1.466901in}}%
\pgfpathlineto{\pgfqpoint{1.008795in}{1.500239in}}%
\pgfpathlineto{\pgfqpoint{0.967157in}{1.533266in}}%
\pgfpathlineto{\pgfqpoint{0.931826in}{1.565811in}}%
\pgfpathlineto{\pgfqpoint{0.902123in}{1.597711in}}%
\pgfpathlineto{\pgfqpoint{0.877563in}{1.628837in}}%
\pgfpathlineto{\pgfqpoint{0.857613in}{1.659087in}}%
\pgfpathlineto{\pgfqpoint{0.841799in}{1.688374in}}%
\pgfpathlineto{\pgfqpoint{0.841799in}{1.688374in}}%
\pgfusepath{stroke}%
\end{pgfscope}%
\begin{pgfscope}%
\pgfsetrectcap%
\pgfsetmiterjoin%
\pgfsetlinewidth{0.803000pt}%
\definecolor{currentstroke}{rgb}{0.000000,0.000000,0.000000}%
\pgfsetstrokecolor{currentstroke}%
\pgfsetdash{}{0pt}%
\pgfpathmoveto{\pgfqpoint{0.562500in}{0.275000in}}%
\pgfpathlineto{\pgfqpoint{0.562500in}{2.200000in}}%
\pgfusepath{stroke}%
\end{pgfscope}%
\begin{pgfscope}%
\pgfsetrectcap%
\pgfsetmiterjoin%
\pgfsetlinewidth{0.803000pt}%
\definecolor{currentstroke}{rgb}{0.000000,0.000000,0.000000}%
\pgfsetstrokecolor{currentstroke}%
\pgfsetdash{}{0pt}%
\pgfpathmoveto{\pgfqpoint{4.050000in}{0.275000in}}%
\pgfpathlineto{\pgfqpoint{4.050000in}{2.200000in}}%
\pgfusepath{stroke}%
\end{pgfscope}%
\begin{pgfscope}%
\pgfsetrectcap%
\pgfsetmiterjoin%
\pgfsetlinewidth{0.803000pt}%
\definecolor{currentstroke}{rgb}{0.000000,0.000000,0.000000}%
\pgfsetstrokecolor{currentstroke}%
\pgfsetdash{}{0pt}%
\pgfpathmoveto{\pgfqpoint{0.562500in}{0.275000in}}%
\pgfpathlineto{\pgfqpoint{4.050000in}{0.275000in}}%
\pgfusepath{stroke}%
\end{pgfscope}%
\begin{pgfscope}%
\pgfsetrectcap%
\pgfsetmiterjoin%
\pgfsetlinewidth{0.803000pt}%
\definecolor{currentstroke}{rgb}{0.000000,0.000000,0.000000}%
\pgfsetstrokecolor{currentstroke}%
\pgfsetdash{}{0pt}%
\pgfpathmoveto{\pgfqpoint{0.562500in}{2.200000in}}%
\pgfpathlineto{\pgfqpoint{4.050000in}{2.200000in}}%
\pgfusepath{stroke}%
\end{pgfscope}%
\begin{pgfscope}%
\pgfsetbuttcap%
\pgfsetmiterjoin%
\definecolor{currentfill}{rgb}{1.000000,1.000000,1.000000}%
\pgfsetfillcolor{currentfill}%
\pgfsetfillopacity{0.800000}%
\pgfsetlinewidth{1.003750pt}%
\definecolor{currentstroke}{rgb}{0.800000,0.800000,0.800000}%
\pgfsetstrokecolor{currentstroke}%
\pgfsetstrokeopacity{0.800000}%
\pgfsetdash{}{0pt}%
\pgfpathmoveto{\pgfqpoint{0.659722in}{0.344444in}}%
\pgfpathlineto{\pgfqpoint{1.695071in}{0.344444in}}%
\pgfpathquadraticcurveto{\pgfqpoint{1.722849in}{0.344444in}}{\pgfqpoint{1.722849in}{0.372222in}}%
\pgfpathlineto{\pgfqpoint{1.722849in}{0.948750in}}%
\pgfpathquadraticcurveto{\pgfqpoint{1.722849in}{0.976528in}}{\pgfqpoint{1.695071in}{0.976528in}}%
\pgfpathlineto{\pgfqpoint{0.659722in}{0.976528in}}%
\pgfpathquadraticcurveto{\pgfqpoint{0.631944in}{0.976528in}}{\pgfqpoint{0.631944in}{0.948750in}}%
\pgfpathlineto{\pgfqpoint{0.631944in}{0.372222in}}%
\pgfpathquadraticcurveto{\pgfqpoint{0.631944in}{0.344444in}}{\pgfqpoint{0.659722in}{0.344444in}}%
\pgfpathlineto{\pgfqpoint{0.659722in}{0.344444in}}%
\pgfpathclose%
\pgfusepath{stroke,fill}%
\end{pgfscope}%
\begin{pgfscope}%
\pgfsetrectcap%
\pgfsetroundjoin%
\pgfsetlinewidth{1.505625pt}%
\definecolor{currentstroke}{rgb}{0.121569,0.466667,0.705882}%
\pgfsetstrokecolor{currentstroke}%
\pgfsetdash{}{0pt}%
\pgfpathmoveto{\pgfqpoint{0.687500in}{0.872361in}}%
\pgfpathlineto{\pgfqpoint{0.826389in}{0.872361in}}%
\pgfpathlineto{\pgfqpoint{0.965278in}{0.872361in}}%
\pgfusepath{stroke}%
\end{pgfscope}%
\begin{pgfscope}%
\definecolor{textcolor}{rgb}{0.000000,0.000000,0.000000}%
\pgfsetstrokecolor{textcolor}%
\pgfsetfillcolor{textcolor}%
\pgftext[x=1.076389in,y=0.823750in,left,base]{\color{textcolor}\rmfamily\fontsize{10.000000}{12.000000}\selectfont \(\displaystyle \alpha=0.125\)}%
\end{pgfscope}%
\begin{pgfscope}%
\pgfsetrectcap%
\pgfsetroundjoin%
\pgfsetlinewidth{1.505625pt}%
\definecolor{currentstroke}{rgb}{1.000000,0.498039,0.054902}%
\pgfsetstrokecolor{currentstroke}%
\pgfsetdash{}{0pt}%
\pgfpathmoveto{\pgfqpoint{0.687500in}{0.675556in}}%
\pgfpathlineto{\pgfqpoint{0.826389in}{0.675556in}}%
\pgfpathlineto{\pgfqpoint{0.965278in}{0.675556in}}%
\pgfusepath{stroke}%
\end{pgfscope}%
\begin{pgfscope}%
\definecolor{textcolor}{rgb}{0.000000,0.000000,0.000000}%
\pgfsetstrokecolor{textcolor}%
\pgfsetfillcolor{textcolor}%
\pgftext[x=1.076389in,y=0.626944in,left,base]{\color{textcolor}\rmfamily\fontsize{10.000000}{12.000000}\selectfont \(\displaystyle \alpha=0.135\)}%
\end{pgfscope}%
\begin{pgfscope}%
\pgfsetrectcap%
\pgfsetroundjoin%
\pgfsetlinewidth{1.505625pt}%
\definecolor{currentstroke}{rgb}{0.172549,0.627451,0.172549}%
\pgfsetstrokecolor{currentstroke}%
\pgfsetdash{}{0pt}%
\pgfpathmoveto{\pgfqpoint{0.687500in}{0.478750in}}%
\pgfpathlineto{\pgfqpoint{0.826389in}{0.478750in}}%
\pgfpathlineto{\pgfqpoint{0.965278in}{0.478750in}}%
\pgfusepath{stroke}%
\end{pgfscope}%
\begin{pgfscope}%
\definecolor{textcolor}{rgb}{0.000000,0.000000,0.000000}%
\pgfsetstrokecolor{textcolor}%
\pgfsetfillcolor{textcolor}%
\pgftext[x=1.076389in,y=0.430139in,left,base]{\color{textcolor}\rmfamily\fontsize{10.000000}{12.000000}\selectfont \(\displaystyle \alpha=0.115\)}%
\end{pgfscope}%
\end{pgfpicture}%
\makeatother%
\endgroup%

        \end{center}
    \end{frame}

    \begin{frame}
    \frametitle{Lösungen Recharge Oszillator bei variierendem $\gamma$}
        \begin{center}
            %% Creator: Matplotlib, PGF backend
%%
%% To include the figure in your LaTeX document, write
%%   \input{<filename>.pgf}
%%
%% Make sure the required packages are loaded in your preamble
%%   \usepackage{pgf}
%%
%% Also ensure that all the required font packages are loaded; for instance,
%% the lmodern package is sometimes necessary when using math font.
%%   \usepackage{lmodern}
%%
%% Figures using additional raster images can only be included by \input if
%% they are in the same directory as the main LaTeX file. For loading figures
%% from other directories you can use the `import` package
%%   \usepackage{import}
%%
%% and then include the figures with
%%   \import{<path to file>}{<filename>.pgf}
%%
%% Matplotlib used the following preamble
%%   \usepackage{bm}
%%   \usepackage{amsmath}
%%   \usepackage{xcolor}
%%   \usepackage{tgtermes}
%%   \makeatletter\@ifpackageloaded{underscore}{}{\usepackage[strings]{underscore}}\makeatother
%%
\begingroup%
\makeatletter%
\begin{pgfpicture}%
\pgfpathrectangle{\pgfpointorigin}{\pgfqpoint{4.500000in}{2.500000in}}%
\pgfusepath{use as bounding box, clip}%
\begin{pgfscope}%
\pgfsetbuttcap%
\pgfsetmiterjoin%
\definecolor{currentfill}{rgb}{1.000000,1.000000,1.000000}%
\pgfsetfillcolor{currentfill}%
\pgfsetlinewidth{0.000000pt}%
\definecolor{currentstroke}{rgb}{1.000000,1.000000,1.000000}%
\pgfsetstrokecolor{currentstroke}%
\pgfsetdash{}{0pt}%
\pgfpathmoveto{\pgfqpoint{0.000000in}{0.000000in}}%
\pgfpathlineto{\pgfqpoint{4.500000in}{0.000000in}}%
\pgfpathlineto{\pgfqpoint{4.500000in}{2.500000in}}%
\pgfpathlineto{\pgfqpoint{0.000000in}{2.500000in}}%
\pgfpathlineto{\pgfqpoint{0.000000in}{0.000000in}}%
\pgfpathclose%
\pgfusepath{fill}%
\end{pgfscope}%
\begin{pgfscope}%
\pgfsetbuttcap%
\pgfsetmiterjoin%
\definecolor{currentfill}{rgb}{1.000000,1.000000,1.000000}%
\pgfsetfillcolor{currentfill}%
\pgfsetlinewidth{0.000000pt}%
\definecolor{currentstroke}{rgb}{0.000000,0.000000,0.000000}%
\pgfsetstrokecolor{currentstroke}%
\pgfsetstrokeopacity{0.000000}%
\pgfsetdash{}{0pt}%
\pgfpathmoveto{\pgfqpoint{0.562500in}{0.275000in}}%
\pgfpathlineto{\pgfqpoint{4.050000in}{0.275000in}}%
\pgfpathlineto{\pgfqpoint{4.050000in}{2.200000in}}%
\pgfpathlineto{\pgfqpoint{0.562500in}{2.200000in}}%
\pgfpathlineto{\pgfqpoint{0.562500in}{0.275000in}}%
\pgfpathclose%
\pgfusepath{fill}%
\end{pgfscope}%
\begin{pgfscope}%
\pgfpathrectangle{\pgfqpoint{0.562500in}{0.275000in}}{\pgfqpoint{3.487500in}{1.925000in}}%
\pgfusepath{clip}%
\pgfsetrectcap%
\pgfsetroundjoin%
\pgfsetlinewidth{0.803000pt}%
\definecolor{currentstroke}{rgb}{0.690196,0.690196,0.690196}%
\pgfsetstrokecolor{currentstroke}%
\pgfsetdash{}{0pt}%
\pgfpathmoveto{\pgfqpoint{0.819500in}{0.275000in}}%
\pgfpathlineto{\pgfqpoint{0.819500in}{2.200000in}}%
\pgfusepath{stroke}%
\end{pgfscope}%
\begin{pgfscope}%
\pgfsetbuttcap%
\pgfsetroundjoin%
\definecolor{currentfill}{rgb}{0.000000,0.000000,0.000000}%
\pgfsetfillcolor{currentfill}%
\pgfsetlinewidth{0.803000pt}%
\definecolor{currentstroke}{rgb}{0.000000,0.000000,0.000000}%
\pgfsetstrokecolor{currentstroke}%
\pgfsetdash{}{0pt}%
\pgfsys@defobject{currentmarker}{\pgfqpoint{0.000000in}{-0.048611in}}{\pgfqpoint{0.000000in}{0.000000in}}{%
\pgfpathmoveto{\pgfqpoint{0.000000in}{0.000000in}}%
\pgfpathlineto{\pgfqpoint{0.000000in}{-0.048611in}}%
\pgfusepath{stroke,fill}%
}%
\begin{pgfscope}%
\pgfsys@transformshift{0.819500in}{0.275000in}%
\pgfsys@useobject{currentmarker}{}%
\end{pgfscope}%
\end{pgfscope}%
\begin{pgfscope}%
\definecolor{textcolor}{rgb}{0.000000,0.000000,0.000000}%
\pgfsetstrokecolor{textcolor}%
\pgfsetfillcolor{textcolor}%
\pgftext[x=0.819500in,y=0.177778in,,top]{\color{textcolor}\rmfamily\fontsize{10.000000}{12.000000}\selectfont \(\displaystyle {-0.75}\)}%
\end{pgfscope}%
\begin{pgfscope}%
\pgfpathrectangle{\pgfqpoint{0.562500in}{0.275000in}}{\pgfqpoint{3.487500in}{1.925000in}}%
\pgfusepath{clip}%
\pgfsetrectcap%
\pgfsetroundjoin%
\pgfsetlinewidth{0.803000pt}%
\definecolor{currentstroke}{rgb}{0.690196,0.690196,0.690196}%
\pgfsetstrokecolor{currentstroke}%
\pgfsetdash{}{0pt}%
\pgfpathmoveto{\pgfqpoint{1.258354in}{0.275000in}}%
\pgfpathlineto{\pgfqpoint{1.258354in}{2.200000in}}%
\pgfusepath{stroke}%
\end{pgfscope}%
\begin{pgfscope}%
\pgfsetbuttcap%
\pgfsetroundjoin%
\definecolor{currentfill}{rgb}{0.000000,0.000000,0.000000}%
\pgfsetfillcolor{currentfill}%
\pgfsetlinewidth{0.803000pt}%
\definecolor{currentstroke}{rgb}{0.000000,0.000000,0.000000}%
\pgfsetstrokecolor{currentstroke}%
\pgfsetdash{}{0pt}%
\pgfsys@defobject{currentmarker}{\pgfqpoint{0.000000in}{-0.048611in}}{\pgfqpoint{0.000000in}{0.000000in}}{%
\pgfpathmoveto{\pgfqpoint{0.000000in}{0.000000in}}%
\pgfpathlineto{\pgfqpoint{0.000000in}{-0.048611in}}%
\pgfusepath{stroke,fill}%
}%
\begin{pgfscope}%
\pgfsys@transformshift{1.258354in}{0.275000in}%
\pgfsys@useobject{currentmarker}{}%
\end{pgfscope}%
\end{pgfscope}%
\begin{pgfscope}%
\definecolor{textcolor}{rgb}{0.000000,0.000000,0.000000}%
\pgfsetstrokecolor{textcolor}%
\pgfsetfillcolor{textcolor}%
\pgftext[x=1.258354in,y=0.177778in,,top]{\color{textcolor}\rmfamily\fontsize{10.000000}{12.000000}\selectfont \(\displaystyle {-0.50}\)}%
\end{pgfscope}%
\begin{pgfscope}%
\pgfpathrectangle{\pgfqpoint{0.562500in}{0.275000in}}{\pgfqpoint{3.487500in}{1.925000in}}%
\pgfusepath{clip}%
\pgfsetrectcap%
\pgfsetroundjoin%
\pgfsetlinewidth{0.803000pt}%
\definecolor{currentstroke}{rgb}{0.690196,0.690196,0.690196}%
\pgfsetstrokecolor{currentstroke}%
\pgfsetdash{}{0pt}%
\pgfpathmoveto{\pgfqpoint{1.697208in}{0.275000in}}%
\pgfpathlineto{\pgfqpoint{1.697208in}{2.200000in}}%
\pgfusepath{stroke}%
\end{pgfscope}%
\begin{pgfscope}%
\pgfsetbuttcap%
\pgfsetroundjoin%
\definecolor{currentfill}{rgb}{0.000000,0.000000,0.000000}%
\pgfsetfillcolor{currentfill}%
\pgfsetlinewidth{0.803000pt}%
\definecolor{currentstroke}{rgb}{0.000000,0.000000,0.000000}%
\pgfsetstrokecolor{currentstroke}%
\pgfsetdash{}{0pt}%
\pgfsys@defobject{currentmarker}{\pgfqpoint{0.000000in}{-0.048611in}}{\pgfqpoint{0.000000in}{0.000000in}}{%
\pgfpathmoveto{\pgfqpoint{0.000000in}{0.000000in}}%
\pgfpathlineto{\pgfqpoint{0.000000in}{-0.048611in}}%
\pgfusepath{stroke,fill}%
}%
\begin{pgfscope}%
\pgfsys@transformshift{1.697208in}{0.275000in}%
\pgfsys@useobject{currentmarker}{}%
\end{pgfscope}%
\end{pgfscope}%
\begin{pgfscope}%
\definecolor{textcolor}{rgb}{0.000000,0.000000,0.000000}%
\pgfsetstrokecolor{textcolor}%
\pgfsetfillcolor{textcolor}%
\pgftext[x=1.697208in,y=0.177778in,,top]{\color{textcolor}\rmfamily\fontsize{10.000000}{12.000000}\selectfont \(\displaystyle {-0.25}\)}%
\end{pgfscope}%
\begin{pgfscope}%
\pgfpathrectangle{\pgfqpoint{0.562500in}{0.275000in}}{\pgfqpoint{3.487500in}{1.925000in}}%
\pgfusepath{clip}%
\pgfsetrectcap%
\pgfsetroundjoin%
\pgfsetlinewidth{0.803000pt}%
\definecolor{currentstroke}{rgb}{0.690196,0.690196,0.690196}%
\pgfsetstrokecolor{currentstroke}%
\pgfsetdash{}{0pt}%
\pgfpathmoveto{\pgfqpoint{2.136062in}{0.275000in}}%
\pgfpathlineto{\pgfqpoint{2.136062in}{2.200000in}}%
\pgfusepath{stroke}%
\end{pgfscope}%
\begin{pgfscope}%
\pgfsetbuttcap%
\pgfsetroundjoin%
\definecolor{currentfill}{rgb}{0.000000,0.000000,0.000000}%
\pgfsetfillcolor{currentfill}%
\pgfsetlinewidth{0.803000pt}%
\definecolor{currentstroke}{rgb}{0.000000,0.000000,0.000000}%
\pgfsetstrokecolor{currentstroke}%
\pgfsetdash{}{0pt}%
\pgfsys@defobject{currentmarker}{\pgfqpoint{0.000000in}{-0.048611in}}{\pgfqpoint{0.000000in}{0.000000in}}{%
\pgfpathmoveto{\pgfqpoint{0.000000in}{0.000000in}}%
\pgfpathlineto{\pgfqpoint{0.000000in}{-0.048611in}}%
\pgfusepath{stroke,fill}%
}%
\begin{pgfscope}%
\pgfsys@transformshift{2.136062in}{0.275000in}%
\pgfsys@useobject{currentmarker}{}%
\end{pgfscope}%
\end{pgfscope}%
\begin{pgfscope}%
\definecolor{textcolor}{rgb}{0.000000,0.000000,0.000000}%
\pgfsetstrokecolor{textcolor}%
\pgfsetfillcolor{textcolor}%
\pgftext[x=2.136062in,y=0.177778in,,top]{\color{textcolor}\rmfamily\fontsize{10.000000}{12.000000}\selectfont \(\displaystyle {0.00}\)}%
\end{pgfscope}%
\begin{pgfscope}%
\pgfpathrectangle{\pgfqpoint{0.562500in}{0.275000in}}{\pgfqpoint{3.487500in}{1.925000in}}%
\pgfusepath{clip}%
\pgfsetrectcap%
\pgfsetroundjoin%
\pgfsetlinewidth{0.803000pt}%
\definecolor{currentstroke}{rgb}{0.690196,0.690196,0.690196}%
\pgfsetstrokecolor{currentstroke}%
\pgfsetdash{}{0pt}%
\pgfpathmoveto{\pgfqpoint{2.574916in}{0.275000in}}%
\pgfpathlineto{\pgfqpoint{2.574916in}{2.200000in}}%
\pgfusepath{stroke}%
\end{pgfscope}%
\begin{pgfscope}%
\pgfsetbuttcap%
\pgfsetroundjoin%
\definecolor{currentfill}{rgb}{0.000000,0.000000,0.000000}%
\pgfsetfillcolor{currentfill}%
\pgfsetlinewidth{0.803000pt}%
\definecolor{currentstroke}{rgb}{0.000000,0.000000,0.000000}%
\pgfsetstrokecolor{currentstroke}%
\pgfsetdash{}{0pt}%
\pgfsys@defobject{currentmarker}{\pgfqpoint{0.000000in}{-0.048611in}}{\pgfqpoint{0.000000in}{0.000000in}}{%
\pgfpathmoveto{\pgfqpoint{0.000000in}{0.000000in}}%
\pgfpathlineto{\pgfqpoint{0.000000in}{-0.048611in}}%
\pgfusepath{stroke,fill}%
}%
\begin{pgfscope}%
\pgfsys@transformshift{2.574916in}{0.275000in}%
\pgfsys@useobject{currentmarker}{}%
\end{pgfscope}%
\end{pgfscope}%
\begin{pgfscope}%
\definecolor{textcolor}{rgb}{0.000000,0.000000,0.000000}%
\pgfsetstrokecolor{textcolor}%
\pgfsetfillcolor{textcolor}%
\pgftext[x=2.574916in,y=0.177778in,,top]{\color{textcolor}\rmfamily\fontsize{10.000000}{12.000000}\selectfont \(\displaystyle {0.25}\)}%
\end{pgfscope}%
\begin{pgfscope}%
\pgfpathrectangle{\pgfqpoint{0.562500in}{0.275000in}}{\pgfqpoint{3.487500in}{1.925000in}}%
\pgfusepath{clip}%
\pgfsetrectcap%
\pgfsetroundjoin%
\pgfsetlinewidth{0.803000pt}%
\definecolor{currentstroke}{rgb}{0.690196,0.690196,0.690196}%
\pgfsetstrokecolor{currentstroke}%
\pgfsetdash{}{0pt}%
\pgfpathmoveto{\pgfqpoint{3.013769in}{0.275000in}}%
\pgfpathlineto{\pgfqpoint{3.013769in}{2.200000in}}%
\pgfusepath{stroke}%
\end{pgfscope}%
\begin{pgfscope}%
\pgfsetbuttcap%
\pgfsetroundjoin%
\definecolor{currentfill}{rgb}{0.000000,0.000000,0.000000}%
\pgfsetfillcolor{currentfill}%
\pgfsetlinewidth{0.803000pt}%
\definecolor{currentstroke}{rgb}{0.000000,0.000000,0.000000}%
\pgfsetstrokecolor{currentstroke}%
\pgfsetdash{}{0pt}%
\pgfsys@defobject{currentmarker}{\pgfqpoint{0.000000in}{-0.048611in}}{\pgfqpoint{0.000000in}{0.000000in}}{%
\pgfpathmoveto{\pgfqpoint{0.000000in}{0.000000in}}%
\pgfpathlineto{\pgfqpoint{0.000000in}{-0.048611in}}%
\pgfusepath{stroke,fill}%
}%
\begin{pgfscope}%
\pgfsys@transformshift{3.013769in}{0.275000in}%
\pgfsys@useobject{currentmarker}{}%
\end{pgfscope}%
\end{pgfscope}%
\begin{pgfscope}%
\definecolor{textcolor}{rgb}{0.000000,0.000000,0.000000}%
\pgfsetstrokecolor{textcolor}%
\pgfsetfillcolor{textcolor}%
\pgftext[x=3.013769in,y=0.177778in,,top]{\color{textcolor}\rmfamily\fontsize{10.000000}{12.000000}\selectfont \(\displaystyle {0.50}\)}%
\end{pgfscope}%
\begin{pgfscope}%
\pgfpathrectangle{\pgfqpoint{0.562500in}{0.275000in}}{\pgfqpoint{3.487500in}{1.925000in}}%
\pgfusepath{clip}%
\pgfsetrectcap%
\pgfsetroundjoin%
\pgfsetlinewidth{0.803000pt}%
\definecolor{currentstroke}{rgb}{0.690196,0.690196,0.690196}%
\pgfsetstrokecolor{currentstroke}%
\pgfsetdash{}{0pt}%
\pgfpathmoveto{\pgfqpoint{3.452623in}{0.275000in}}%
\pgfpathlineto{\pgfqpoint{3.452623in}{2.200000in}}%
\pgfusepath{stroke}%
\end{pgfscope}%
\begin{pgfscope}%
\pgfsetbuttcap%
\pgfsetroundjoin%
\definecolor{currentfill}{rgb}{0.000000,0.000000,0.000000}%
\pgfsetfillcolor{currentfill}%
\pgfsetlinewidth{0.803000pt}%
\definecolor{currentstroke}{rgb}{0.000000,0.000000,0.000000}%
\pgfsetstrokecolor{currentstroke}%
\pgfsetdash{}{0pt}%
\pgfsys@defobject{currentmarker}{\pgfqpoint{0.000000in}{-0.048611in}}{\pgfqpoint{0.000000in}{0.000000in}}{%
\pgfpathmoveto{\pgfqpoint{0.000000in}{0.000000in}}%
\pgfpathlineto{\pgfqpoint{0.000000in}{-0.048611in}}%
\pgfusepath{stroke,fill}%
}%
\begin{pgfscope}%
\pgfsys@transformshift{3.452623in}{0.275000in}%
\pgfsys@useobject{currentmarker}{}%
\end{pgfscope}%
\end{pgfscope}%
\begin{pgfscope}%
\definecolor{textcolor}{rgb}{0.000000,0.000000,0.000000}%
\pgfsetstrokecolor{textcolor}%
\pgfsetfillcolor{textcolor}%
\pgftext[x=3.452623in,y=0.177778in,,top]{\color{textcolor}\rmfamily\fontsize{10.000000}{12.000000}\selectfont \(\displaystyle {0.75}\)}%
\end{pgfscope}%
\begin{pgfscope}%
\pgfpathrectangle{\pgfqpoint{0.562500in}{0.275000in}}{\pgfqpoint{3.487500in}{1.925000in}}%
\pgfusepath{clip}%
\pgfsetrectcap%
\pgfsetroundjoin%
\pgfsetlinewidth{0.803000pt}%
\definecolor{currentstroke}{rgb}{0.690196,0.690196,0.690196}%
\pgfsetstrokecolor{currentstroke}%
\pgfsetdash{}{0pt}%
\pgfpathmoveto{\pgfqpoint{3.891477in}{0.275000in}}%
\pgfpathlineto{\pgfqpoint{3.891477in}{2.200000in}}%
\pgfusepath{stroke}%
\end{pgfscope}%
\begin{pgfscope}%
\pgfsetbuttcap%
\pgfsetroundjoin%
\definecolor{currentfill}{rgb}{0.000000,0.000000,0.000000}%
\pgfsetfillcolor{currentfill}%
\pgfsetlinewidth{0.803000pt}%
\definecolor{currentstroke}{rgb}{0.000000,0.000000,0.000000}%
\pgfsetstrokecolor{currentstroke}%
\pgfsetdash{}{0pt}%
\pgfsys@defobject{currentmarker}{\pgfqpoint{0.000000in}{-0.048611in}}{\pgfqpoint{0.000000in}{0.000000in}}{%
\pgfpathmoveto{\pgfqpoint{0.000000in}{0.000000in}}%
\pgfpathlineto{\pgfqpoint{0.000000in}{-0.048611in}}%
\pgfusepath{stroke,fill}%
}%
\begin{pgfscope}%
\pgfsys@transformshift{3.891477in}{0.275000in}%
\pgfsys@useobject{currentmarker}{}%
\end{pgfscope}%
\end{pgfscope}%
\begin{pgfscope}%
\definecolor{textcolor}{rgb}{0.000000,0.000000,0.000000}%
\pgfsetstrokecolor{textcolor}%
\pgfsetfillcolor{textcolor}%
\pgftext[x=3.891477in,y=0.177778in,,top]{\color{textcolor}\rmfamily\fontsize{10.000000}{12.000000}\selectfont \(\displaystyle {1.00}\)}%
\end{pgfscope}%
\begin{pgfscope}%
\pgfpathrectangle{\pgfqpoint{0.562500in}{0.275000in}}{\pgfqpoint{3.487500in}{1.925000in}}%
\pgfusepath{clip}%
\pgfsetrectcap%
\pgfsetroundjoin%
\pgfsetlinewidth{0.803000pt}%
\definecolor{currentstroke}{rgb}{0.690196,0.690196,0.690196}%
\pgfsetstrokecolor{currentstroke}%
\pgfsetdash{}{0pt}%
\pgfpathmoveto{\pgfqpoint{0.562500in}{0.362500in}}%
\pgfpathlineto{\pgfqpoint{4.050000in}{0.362500in}}%
\pgfusepath{stroke}%
\end{pgfscope}%
\begin{pgfscope}%
\pgfsetbuttcap%
\pgfsetroundjoin%
\definecolor{currentfill}{rgb}{0.000000,0.000000,0.000000}%
\pgfsetfillcolor{currentfill}%
\pgfsetlinewidth{0.803000pt}%
\definecolor{currentstroke}{rgb}{0.000000,0.000000,0.000000}%
\pgfsetstrokecolor{currentstroke}%
\pgfsetdash{}{0pt}%
\pgfsys@defobject{currentmarker}{\pgfqpoint{-0.048611in}{0.000000in}}{\pgfqpoint{-0.000000in}{0.000000in}}{%
\pgfpathmoveto{\pgfqpoint{-0.000000in}{0.000000in}}%
\pgfpathlineto{\pgfqpoint{-0.048611in}{0.000000in}}%
\pgfusepath{stroke,fill}%
}%
\begin{pgfscope}%
\pgfsys@transformshift{0.562500in}{0.362500in}%
\pgfsys@useobject{currentmarker}{}%
\end{pgfscope}%
\end{pgfscope}%
\begin{pgfscope}%
\definecolor{textcolor}{rgb}{0.000000,0.000000,0.000000}%
\pgfsetstrokecolor{textcolor}%
\pgfsetfillcolor{textcolor}%
\pgftext[x=0.179783in, y=0.315799in, left, base]{\color{textcolor}\rmfamily\fontsize{10.000000}{12.000000}\selectfont \(\displaystyle {-1.0}\)}%
\end{pgfscope}%
\begin{pgfscope}%
\pgfpathrectangle{\pgfqpoint{0.562500in}{0.275000in}}{\pgfqpoint{3.487500in}{1.925000in}}%
\pgfusepath{clip}%
\pgfsetrectcap%
\pgfsetroundjoin%
\pgfsetlinewidth{0.803000pt}%
\definecolor{currentstroke}{rgb}{0.690196,0.690196,0.690196}%
\pgfsetstrokecolor{currentstroke}%
\pgfsetdash{}{0pt}%
\pgfpathmoveto{\pgfqpoint{0.562500in}{0.932975in}}%
\pgfpathlineto{\pgfqpoint{4.050000in}{0.932975in}}%
\pgfusepath{stroke}%
\end{pgfscope}%
\begin{pgfscope}%
\pgfsetbuttcap%
\pgfsetroundjoin%
\definecolor{currentfill}{rgb}{0.000000,0.000000,0.000000}%
\pgfsetfillcolor{currentfill}%
\pgfsetlinewidth{0.803000pt}%
\definecolor{currentstroke}{rgb}{0.000000,0.000000,0.000000}%
\pgfsetstrokecolor{currentstroke}%
\pgfsetdash{}{0pt}%
\pgfsys@defobject{currentmarker}{\pgfqpoint{-0.048611in}{0.000000in}}{\pgfqpoint{-0.000000in}{0.000000in}}{%
\pgfpathmoveto{\pgfqpoint{-0.000000in}{0.000000in}}%
\pgfpathlineto{\pgfqpoint{-0.048611in}{0.000000in}}%
\pgfusepath{stroke,fill}%
}%
\begin{pgfscope}%
\pgfsys@transformshift{0.562500in}{0.932975in}%
\pgfsys@useobject{currentmarker}{}%
\end{pgfscope}%
\end{pgfscope}%
\begin{pgfscope}%
\definecolor{textcolor}{rgb}{0.000000,0.000000,0.000000}%
\pgfsetstrokecolor{textcolor}%
\pgfsetfillcolor{textcolor}%
\pgftext[x=0.179783in, y=0.886273in, left, base]{\color{textcolor}\rmfamily\fontsize{10.000000}{12.000000}\selectfont \(\displaystyle {-0.5}\)}%
\end{pgfscope}%
\begin{pgfscope}%
\pgfpathrectangle{\pgfqpoint{0.562500in}{0.275000in}}{\pgfqpoint{3.487500in}{1.925000in}}%
\pgfusepath{clip}%
\pgfsetrectcap%
\pgfsetroundjoin%
\pgfsetlinewidth{0.803000pt}%
\definecolor{currentstroke}{rgb}{0.690196,0.690196,0.690196}%
\pgfsetstrokecolor{currentstroke}%
\pgfsetdash{}{0pt}%
\pgfpathmoveto{\pgfqpoint{0.562500in}{1.503449in}}%
\pgfpathlineto{\pgfqpoint{4.050000in}{1.503449in}}%
\pgfusepath{stroke}%
\end{pgfscope}%
\begin{pgfscope}%
\pgfsetbuttcap%
\pgfsetroundjoin%
\definecolor{currentfill}{rgb}{0.000000,0.000000,0.000000}%
\pgfsetfillcolor{currentfill}%
\pgfsetlinewidth{0.803000pt}%
\definecolor{currentstroke}{rgb}{0.000000,0.000000,0.000000}%
\pgfsetstrokecolor{currentstroke}%
\pgfsetdash{}{0pt}%
\pgfsys@defobject{currentmarker}{\pgfqpoint{-0.048611in}{0.000000in}}{\pgfqpoint{-0.000000in}{0.000000in}}{%
\pgfpathmoveto{\pgfqpoint{-0.000000in}{0.000000in}}%
\pgfpathlineto{\pgfqpoint{-0.048611in}{0.000000in}}%
\pgfusepath{stroke,fill}%
}%
\begin{pgfscope}%
\pgfsys@transformshift{0.562500in}{1.503449in}%
\pgfsys@useobject{currentmarker}{}%
\end{pgfscope}%
\end{pgfscope}%
\begin{pgfscope}%
\definecolor{textcolor}{rgb}{0.000000,0.000000,0.000000}%
\pgfsetstrokecolor{textcolor}%
\pgfsetfillcolor{textcolor}%
\pgftext[x=0.287808in, y=1.456748in, left, base]{\color{textcolor}\rmfamily\fontsize{10.000000}{12.000000}\selectfont \(\displaystyle {0.0}\)}%
\end{pgfscope}%
\begin{pgfscope}%
\pgfpathrectangle{\pgfqpoint{0.562500in}{0.275000in}}{\pgfqpoint{3.487500in}{1.925000in}}%
\pgfusepath{clip}%
\pgfsetrectcap%
\pgfsetroundjoin%
\pgfsetlinewidth{0.803000pt}%
\definecolor{currentstroke}{rgb}{0.690196,0.690196,0.690196}%
\pgfsetstrokecolor{currentstroke}%
\pgfsetdash{}{0pt}%
\pgfpathmoveto{\pgfqpoint{0.562500in}{2.073924in}}%
\pgfpathlineto{\pgfqpoint{4.050000in}{2.073924in}}%
\pgfusepath{stroke}%
\end{pgfscope}%
\begin{pgfscope}%
\pgfsetbuttcap%
\pgfsetroundjoin%
\definecolor{currentfill}{rgb}{0.000000,0.000000,0.000000}%
\pgfsetfillcolor{currentfill}%
\pgfsetlinewidth{0.803000pt}%
\definecolor{currentstroke}{rgb}{0.000000,0.000000,0.000000}%
\pgfsetstrokecolor{currentstroke}%
\pgfsetdash{}{0pt}%
\pgfsys@defobject{currentmarker}{\pgfqpoint{-0.048611in}{0.000000in}}{\pgfqpoint{-0.000000in}{0.000000in}}{%
\pgfpathmoveto{\pgfqpoint{-0.000000in}{0.000000in}}%
\pgfpathlineto{\pgfqpoint{-0.048611in}{0.000000in}}%
\pgfusepath{stroke,fill}%
}%
\begin{pgfscope}%
\pgfsys@transformshift{0.562500in}{2.073924in}%
\pgfsys@useobject{currentmarker}{}%
\end{pgfscope}%
\end{pgfscope}%
\begin{pgfscope}%
\definecolor{textcolor}{rgb}{0.000000,0.000000,0.000000}%
\pgfsetstrokecolor{textcolor}%
\pgfsetfillcolor{textcolor}%
\pgftext[x=0.287808in, y=2.027223in, left, base]{\color{textcolor}\rmfamily\fontsize{10.000000}{12.000000}\selectfont \(\displaystyle {0.5}\)}%
\end{pgfscope}%
\begin{pgfscope}%
\pgfpathrectangle{\pgfqpoint{0.562500in}{0.275000in}}{\pgfqpoint{3.487500in}{1.925000in}}%
\pgfusepath{clip}%
\pgfsetrectcap%
\pgfsetroundjoin%
\pgfsetlinewidth{1.505625pt}%
\definecolor{currentstroke}{rgb}{0.121569,0.466667,0.705882}%
\pgfsetstrokecolor{currentstroke}%
\pgfsetdash{}{0pt}%
\pgfpathmoveto{\pgfqpoint{3.891477in}{0.362500in}}%
\pgfpathlineto{\pgfqpoint{3.769142in}{0.364458in}}%
\pgfpathlineto{\pgfqpoint{3.646183in}{0.370199in}}%
\pgfpathlineto{\pgfqpoint{3.520110in}{0.379616in}}%
\pgfpathlineto{\pgfqpoint{3.388810in}{0.392660in}}%
\pgfpathlineto{\pgfqpoint{3.250517in}{0.409339in}}%
\pgfpathlineto{\pgfqpoint{3.103816in}{0.429708in}}%
\pgfpathlineto{\pgfqpoint{2.947643in}{0.453879in}}%
\pgfpathlineto{\pgfqpoint{2.779329in}{0.482071in}}%
\pgfpathlineto{\pgfqpoint{2.597271in}{0.514518in}}%
\pgfpathlineto{\pgfqpoint{2.407961in}{0.551164in}}%
\pgfpathlineto{\pgfqpoint{2.217422in}{0.591865in}}%
\pgfpathlineto{\pgfqpoint{2.123352in}{0.613670in}}%
\pgfpathlineto{\pgfqpoint{2.030863in}{0.636400in}}%
\pgfpathlineto{\pgfqpoint{1.940480in}{0.660015in}}%
\pgfpathlineto{\pgfqpoint{1.852674in}{0.684469in}}%
\pgfpathlineto{\pgfqpoint{1.767866in}{0.709713in}}%
\pgfpathlineto{\pgfqpoint{1.686426in}{0.735694in}}%
\pgfpathlineto{\pgfqpoint{1.608673in}{0.762351in}}%
\pgfpathlineto{\pgfqpoint{1.534875in}{0.789620in}}%
\pgfpathlineto{\pgfqpoint{1.465247in}{0.817432in}}%
\pgfpathlineto{\pgfqpoint{1.399956in}{0.845713in}}%
\pgfpathlineto{\pgfqpoint{1.339115in}{0.874385in}}%
\pgfpathlineto{\pgfqpoint{1.282789in}{0.903364in}}%
\pgfpathlineto{\pgfqpoint{1.230988in}{0.932560in}}%
\pgfpathlineto{\pgfqpoint{1.183674in}{0.961881in}}%
\pgfpathlineto{\pgfqpoint{1.140757in}{0.991228in}}%
\pgfpathlineto{\pgfqpoint{1.102095in}{1.020499in}}%
\pgfpathlineto{\pgfqpoint{1.067496in}{1.049585in}}%
\pgfpathlineto{\pgfqpoint{1.036718in}{1.078373in}}%
\pgfpathlineto{\pgfqpoint{1.009372in}{1.106780in}}%
\pgfpathlineto{\pgfqpoint{0.984719in}{1.134883in}}%
\pgfpathlineto{\pgfqpoint{0.962433in}{1.162669in}}%
\pgfpathlineto{\pgfqpoint{0.942270in}{1.190110in}}%
\pgfpathlineto{\pgfqpoint{0.924011in}{1.217181in}}%
\pgfpathlineto{\pgfqpoint{0.892447in}{1.270135in}}%
\pgfpathlineto{\pgfqpoint{0.866457in}{1.321414in}}%
\pgfpathlineto{\pgfqpoint{0.845069in}{1.370931in}}%
\pgfpathlineto{\pgfqpoint{0.827571in}{1.418620in}}%
\pgfpathlineto{\pgfqpoint{0.813305in}{1.464490in}}%
\pgfpathlineto{\pgfqpoint{0.801781in}{1.508552in}}%
\pgfpathlineto{\pgfqpoint{0.792687in}{1.550814in}}%
\pgfpathlineto{\pgfqpoint{0.785892in}{1.591275in}}%
\pgfpathlineto{\pgfqpoint{0.781445in}{1.629927in}}%
\pgfpathlineto{\pgfqpoint{0.779361in}{1.666773in}}%
\pgfpathlineto{\pgfqpoint{0.779235in}{1.701850in}}%
\pgfpathlineto{\pgfqpoint{0.780942in}{1.735191in}}%
\pgfpathlineto{\pgfqpoint{0.784414in}{1.766825in}}%
\pgfpathlineto{\pgfqpoint{0.789617in}{1.796780in}}%
\pgfpathlineto{\pgfqpoint{0.796553in}{1.825079in}}%
\pgfpathlineto{\pgfqpoint{0.805265in}{1.851747in}}%
\pgfpathlineto{\pgfqpoint{0.815838in}{1.876809in}}%
\pgfpathlineto{\pgfqpoint{0.828284in}{1.900292in}}%
\pgfpathlineto{\pgfqpoint{0.842501in}{1.922212in}}%
\pgfpathlineto{\pgfqpoint{0.858424in}{1.942586in}}%
\pgfpathlineto{\pgfqpoint{0.876027in}{1.961430in}}%
\pgfpathlineto{\pgfqpoint{0.895326in}{1.978759in}}%
\pgfpathlineto{\pgfqpoint{0.916377in}{1.994588in}}%
\pgfpathlineto{\pgfqpoint{0.939278in}{2.008931in}}%
\pgfpathlineto{\pgfqpoint{0.964166in}{2.021802in}}%
\pgfpathlineto{\pgfqpoint{0.991219in}{2.033217in}}%
\pgfpathlineto{\pgfqpoint{1.020656in}{2.043187in}}%
\pgfpathlineto{\pgfqpoint{1.052665in}{2.051722in}}%
\pgfpathlineto{\pgfqpoint{1.087112in}{2.058794in}}%
\pgfpathlineto{\pgfqpoint{1.124132in}{2.064391in}}%
\pgfpathlineto{\pgfqpoint{1.163914in}{2.068496in}}%
\pgfpathlineto{\pgfqpoint{1.206658in}{2.071090in}}%
\pgfpathlineto{\pgfqpoint{1.252568in}{2.072147in}}%
\pgfpathlineto{\pgfqpoint{1.301858in}{2.071638in}}%
\pgfpathlineto{\pgfqpoint{1.354750in}{2.069528in}}%
\pgfpathlineto{\pgfqpoint{1.441339in}{2.063276in}}%
\pgfpathlineto{\pgfqpoint{1.537347in}{2.053182in}}%
\pgfpathlineto{\pgfqpoint{1.643614in}{2.039070in}}%
\pgfpathlineto{\pgfqpoint{1.760870in}{2.020736in}}%
\pgfpathlineto{\pgfqpoint{1.889446in}{1.997942in}}%
\pgfpathlineto{\pgfqpoint{2.028918in}{1.970487in}}%
\pgfpathlineto{\pgfqpoint{2.177761in}{1.938238in}}%
\pgfpathlineto{\pgfqpoint{2.280955in}{1.914045in}}%
\pgfpathlineto{\pgfqpoint{2.385950in}{1.887723in}}%
\pgfpathlineto{\pgfqpoint{2.491142in}{1.859350in}}%
\pgfpathlineto{\pgfqpoint{2.594911in}{1.829045in}}%
\pgfpathlineto{\pgfqpoint{2.695637in}{1.796972in}}%
\pgfpathlineto{\pgfqpoint{2.791700in}{1.763337in}}%
\pgfpathlineto{\pgfqpoint{2.881340in}{1.728386in}}%
\pgfpathlineto{\pgfqpoint{2.962895in}{1.692396in}}%
\pgfpathlineto{\pgfqpoint{3.036422in}{1.655688in}}%
\pgfpathlineto{\pgfqpoint{3.102186in}{1.618564in}}%
\pgfpathlineto{\pgfqpoint{3.160465in}{1.581300in}}%
\pgfpathlineto{\pgfqpoint{3.211558in}{1.544145in}}%
\pgfpathlineto{\pgfqpoint{3.255781in}{1.507323in}}%
\pgfpathlineto{\pgfqpoint{3.293469in}{1.471028in}}%
\pgfpathlineto{\pgfqpoint{3.324973in}{1.435431in}}%
\pgfpathlineto{\pgfqpoint{3.350662in}{1.400673in}}%
\pgfpathlineto{\pgfqpoint{3.370925in}{1.366873in}}%
\pgfpathlineto{\pgfqpoint{3.386164in}{1.334122in}}%
\pgfpathlineto{\pgfqpoint{3.396879in}{1.302553in}}%
\pgfpathlineto{\pgfqpoint{3.403886in}{1.272239in}}%
\pgfpathlineto{\pgfqpoint{3.407916in}{1.243212in}}%
\pgfpathlineto{\pgfqpoint{3.409525in}{1.215495in}}%
\pgfpathlineto{\pgfqpoint{3.409088in}{1.189108in}}%
\pgfpathlineto{\pgfqpoint{3.406807in}{1.164063in}}%
\pgfpathlineto{\pgfqpoint{3.402703in}{1.140367in}}%
\pgfpathlineto{\pgfqpoint{3.396622in}{1.118022in}}%
\pgfpathlineto{\pgfqpoint{3.388231in}{1.097022in}}%
\pgfpathlineto{\pgfqpoint{3.377021in}{1.077358in}}%
\pgfpathlineto{\pgfqpoint{3.362476in}{1.059020in}}%
\pgfpathlineto{\pgfqpoint{3.345533in}{1.042061in}}%
\pgfpathlineto{\pgfqpoint{3.326500in}{1.026498in}}%
\pgfpathlineto{\pgfqpoint{3.305375in}{1.012330in}}%
\pgfpathlineto{\pgfqpoint{3.282132in}{0.999560in}}%
\pgfpathlineto{\pgfqpoint{3.256719in}{0.988192in}}%
\pgfpathlineto{\pgfqpoint{3.229061in}{0.978228in}}%
\pgfpathlineto{\pgfqpoint{3.199057in}{0.969674in}}%
\pgfpathlineto{\pgfqpoint{3.166584in}{0.962533in}}%
\pgfpathlineto{\pgfqpoint{3.131491in}{0.956811in}}%
\pgfpathlineto{\pgfqpoint{3.093616in}{0.952515in}}%
\pgfpathlineto{\pgfqpoint{3.053000in}{0.949659in}}%
\pgfpathlineto{\pgfqpoint{3.009408in}{0.948274in}}%
\pgfpathlineto{\pgfqpoint{2.962504in}{0.948399in}}%
\pgfpathlineto{\pgfqpoint{2.885288in}{0.951506in}}%
\pgfpathlineto{\pgfqpoint{2.799083in}{0.958247in}}%
\pgfpathlineto{\pgfqpoint{2.703207in}{0.968784in}}%
\pgfpathlineto{\pgfqpoint{2.597160in}{0.983291in}}%
\pgfpathlineto{\pgfqpoint{2.480625in}{1.001958in}}%
\pgfpathlineto{\pgfqpoint{2.353467in}{1.024987in}}%
\pgfpathlineto{\pgfqpoint{2.215707in}{1.052596in}}%
\pgfpathlineto{\pgfqpoint{2.067449in}{1.085042in}}%
\pgfpathlineto{\pgfqpoint{1.964991in}{1.109343in}}%
\pgfpathlineto{\pgfqpoint{1.861632in}{1.135692in}}%
\pgfpathlineto{\pgfqpoint{1.758937in}{1.163984in}}%
\pgfpathlineto{\pgfqpoint{1.658339in}{1.194087in}}%
\pgfpathlineto{\pgfqpoint{1.561142in}{1.225841in}}%
\pgfpathlineto{\pgfqpoint{1.468523in}{1.259059in}}%
\pgfpathlineto{\pgfqpoint{1.381526in}{1.293526in}}%
\pgfpathlineto{\pgfqpoint{1.301067in}{1.329001in}}%
\pgfpathlineto{\pgfqpoint{1.227932in}{1.365215in}}%
\pgfpathlineto{\pgfqpoint{1.162777in}{1.401872in}}%
\pgfpathlineto{\pgfqpoint{1.133362in}{1.420267in}}%
\pgfpathlineto{\pgfqpoint{1.080780in}{1.457001in}}%
\pgfpathlineto{\pgfqpoint{1.035653in}{1.493483in}}%
\pgfpathlineto{\pgfqpoint{0.997229in}{1.529514in}}%
\pgfpathlineto{\pgfqpoint{0.964869in}{1.564914in}}%
\pgfpathlineto{\pgfqpoint{0.938030in}{1.599526in}}%
\pgfpathlineto{\pgfqpoint{0.916271in}{1.633216in}}%
\pgfpathlineto{\pgfqpoint{0.899241in}{1.665870in}}%
\pgfpathlineto{\pgfqpoint{0.886384in}{1.697398in}}%
\pgfpathlineto{\pgfqpoint{0.877158in}{1.727737in}}%
\pgfpathlineto{\pgfqpoint{0.871180in}{1.756838in}}%
\pgfpathlineto{\pgfqpoint{0.868154in}{1.784661in}}%
\pgfpathlineto{\pgfqpoint{0.867873in}{1.811176in}}%
\pgfpathlineto{\pgfqpoint{0.870220in}{1.836360in}}%
\pgfpathlineto{\pgfqpoint{0.875146in}{1.860202in}}%
\pgfpathlineto{\pgfqpoint{0.882509in}{1.882692in}}%
\pgfpathlineto{\pgfqpoint{0.892133in}{1.903821in}}%
\pgfpathlineto{\pgfqpoint{0.903885in}{1.923582in}}%
\pgfpathlineto{\pgfqpoint{0.917680in}{1.941969in}}%
\pgfpathlineto{\pgfqpoint{0.933477in}{1.958980in}}%
\pgfpathlineto{\pgfqpoint{0.951281in}{1.974615in}}%
\pgfpathlineto{\pgfqpoint{0.971142in}{1.988875in}}%
\pgfpathlineto{\pgfqpoint{0.993153in}{2.001765in}}%
\pgfpathlineto{\pgfqpoint{1.017457in}{2.013291in}}%
\pgfpathlineto{\pgfqpoint{1.044234in}{2.023461in}}%
\pgfpathlineto{\pgfqpoint{1.073446in}{2.032262in}}%
\pgfpathlineto{\pgfqpoint{1.105093in}{2.039672in}}%
\pgfpathlineto{\pgfqpoint{1.139310in}{2.045678in}}%
\pgfpathlineto{\pgfqpoint{1.176246in}{2.050262in}}%
\pgfpathlineto{\pgfqpoint{1.216062in}{2.053401in}}%
\pgfpathlineto{\pgfqpoint{1.258932in}{2.055071in}}%
\pgfpathlineto{\pgfqpoint{1.305044in}{2.055242in}}%
\pgfpathlineto{\pgfqpoint{1.354595in}{2.053882in}}%
\pgfpathlineto{\pgfqpoint{1.435837in}{2.048889in}}%
\pgfpathlineto{\pgfqpoint{1.526064in}{2.040226in}}%
\pgfpathlineto{\pgfqpoint{1.626092in}{2.027733in}}%
\pgfpathlineto{\pgfqpoint{1.736624in}{2.011200in}}%
\pgfpathlineto{\pgfqpoint{1.858125in}{1.990401in}}%
\pgfpathlineto{\pgfqpoint{1.990377in}{1.965140in}}%
\pgfpathlineto{\pgfqpoint{2.132397in}{1.935254in}}%
\pgfpathlineto{\pgfqpoint{2.282103in}{1.900657in}}%
\pgfpathlineto{\pgfqpoint{2.384360in}{1.874998in}}%
\pgfpathlineto{\pgfqpoint{2.487038in}{1.847336in}}%
\pgfpathlineto{\pgfqpoint{2.588647in}{1.817782in}}%
\pgfpathlineto{\pgfqpoint{2.687551in}{1.786486in}}%
\pgfpathlineto{\pgfqpoint{2.781952in}{1.753618in}}%
\pgfpathlineto{\pgfqpoint{2.870409in}{1.719402in}}%
\pgfpathlineto{\pgfqpoint{2.952090in}{1.684160in}}%
\pgfpathlineto{\pgfqpoint{3.026399in}{1.648200in}}%
\pgfpathlineto{\pgfqpoint{3.092973in}{1.611807in}}%
\pgfpathlineto{\pgfqpoint{3.151677in}{1.575241in}}%
\pgfpathlineto{\pgfqpoint{3.202603in}{1.538744in}}%
\pgfpathlineto{\pgfqpoint{3.246075in}{1.502534in}}%
\pgfpathlineto{\pgfqpoint{3.282646in}{1.466806in}}%
\pgfpathlineto{\pgfqpoint{3.313097in}{1.431733in}}%
\pgfpathlineto{\pgfqpoint{3.338162in}{1.397482in}}%
\pgfpathlineto{\pgfqpoint{3.357970in}{1.364210in}}%
\pgfpathlineto{\pgfqpoint{3.373443in}{1.331993in}}%
\pgfpathlineto{\pgfqpoint{3.385377in}{1.300894in}}%
\pgfpathlineto{\pgfqpoint{3.394341in}{1.270966in}}%
\pgfpathlineto{\pgfqpoint{3.400683in}{1.242254in}}%
\pgfpathlineto{\pgfqpoint{3.404522in}{1.214795in}}%
\pgfpathlineto{\pgfqpoint{3.405755in}{1.188617in}}%
\pgfpathlineto{\pgfqpoint{3.404054in}{1.163741in}}%
\pgfpathlineto{\pgfqpoint{3.398880in}{1.140178in}}%
\pgfpathlineto{\pgfqpoint{3.390771in}{1.117963in}}%
\pgfpathlineto{\pgfqpoint{3.380426in}{1.097122in}}%
\pgfpathlineto{\pgfqpoint{3.367945in}{1.077658in}}%
\pgfpathlineto{\pgfqpoint{3.353395in}{1.059575in}}%
\pgfpathlineto{\pgfqpoint{3.336805in}{1.042873in}}%
\pgfpathlineto{\pgfqpoint{3.318167in}{1.027554in}}%
\pgfpathlineto{\pgfqpoint{3.297439in}{1.013616in}}%
\pgfpathlineto{\pgfqpoint{3.274541in}{1.001056in}}%
\pgfpathlineto{\pgfqpoint{3.249358in}{0.989870in}}%
\pgfpathlineto{\pgfqpoint{3.221768in}{0.980054in}}%
\pgfpathlineto{\pgfqpoint{3.191808in}{0.971623in}}%
\pgfpathlineto{\pgfqpoint{3.159381in}{0.964588in}}%
\pgfpathlineto{\pgfqpoint{3.124345in}{0.958966in}}%
\pgfpathlineto{\pgfqpoint{3.086545in}{0.954774in}}%
\pgfpathlineto{\pgfqpoint{3.045816in}{0.952037in}}%
\pgfpathlineto{\pgfqpoint{3.001978in}{0.950780in}}%
\pgfpathlineto{\pgfqpoint{2.954841in}{0.951033in}}%
\pgfpathlineto{\pgfqpoint{2.877504in}{0.954321in}}%
\pgfpathlineto{\pgfqpoint{2.791549in}{0.961215in}}%
\pgfpathlineto{\pgfqpoint{2.696182in}{0.971867in}}%
\pgfpathlineto{\pgfqpoint{2.590612in}{0.986454in}}%
\pgfpathlineto{\pgfqpoint{2.474228in}{1.005206in}}%
\pgfpathlineto{\pgfqpoint{2.346865in}{1.028338in}}%
\pgfpathlineto{\pgfqpoint{2.209152in}{1.056028in}}%
\pgfpathlineto{\pgfqpoint{2.062553in}{1.088409in}}%
\pgfpathlineto{\pgfqpoint{1.961286in}{1.112618in}}%
\pgfpathlineto{\pgfqpoint{1.858631in}{1.138885in}}%
\pgfpathlineto{\pgfqpoint{1.755982in}{1.167130in}}%
\pgfpathlineto{\pgfqpoint{1.654868in}{1.197232in}}%
\pgfpathlineto{\pgfqpoint{1.556946in}{1.229025in}}%
\pgfpathlineto{\pgfqpoint{1.464018in}{1.262334in}}%
\pgfpathlineto{\pgfqpoint{1.377373in}{1.296909in}}%
\pgfpathlineto{\pgfqpoint{1.297737in}{1.332425in}}%
\pgfpathlineto{\pgfqpoint{1.225615in}{1.368578in}}%
\pgfpathlineto{\pgfqpoint{1.161288in}{1.405087in}}%
\pgfpathlineto{\pgfqpoint{1.104819in}{1.441693in}}%
\pgfpathlineto{\pgfqpoint{1.056048in}{1.478162in}}%
\pgfpathlineto{\pgfqpoint{1.014596in}{1.514282in}}%
\pgfpathlineto{\pgfqpoint{0.979860in}{1.549864in}}%
\pgfpathlineto{\pgfqpoint{0.951020in}{1.584743in}}%
\pgfpathlineto{\pgfqpoint{0.927456in}{1.618754in}}%
\pgfpathlineto{\pgfqpoint{0.908979in}{1.651756in}}%
\pgfpathlineto{\pgfqpoint{0.894646in}{1.683679in}}%
\pgfpathlineto{\pgfqpoint{0.883704in}{1.714465in}}%
\pgfpathlineto{\pgfqpoint{0.875620in}{1.744064in}}%
\pgfpathlineto{\pgfqpoint{0.870084in}{1.772434in}}%
\pgfpathlineto{\pgfqpoint{0.867007in}{1.799541in}}%
\pgfpathlineto{\pgfqpoint{0.866524in}{1.825358in}}%
\pgfpathlineto{\pgfqpoint{0.868990in}{1.849870in}}%
\pgfpathlineto{\pgfqpoint{0.874941in}{1.873066in}}%
\pgfpathlineto{\pgfqpoint{0.883693in}{1.894906in}}%
\pgfpathlineto{\pgfqpoint{0.894657in}{1.915373in}}%
\pgfpathlineto{\pgfqpoint{0.907740in}{1.934460in}}%
\pgfpathlineto{\pgfqpoint{0.922886in}{1.952166in}}%
\pgfpathlineto{\pgfqpoint{0.940072in}{1.968489in}}%
\pgfpathlineto{\pgfqpoint{0.959311in}{1.983429in}}%
\pgfpathlineto{\pgfqpoint{0.980652in}{1.996987in}}%
\pgfpathlineto{\pgfqpoint{1.004180in}{2.009166in}}%
\pgfpathlineto{\pgfqpoint{1.030012in}{2.019971in}}%
\pgfpathlineto{\pgfqpoint{1.058267in}{2.029403in}}%
\pgfpathlineto{\pgfqpoint{1.088909in}{2.037448in}}%
\pgfpathlineto{\pgfqpoint{1.122049in}{2.044093in}}%
\pgfpathlineto{\pgfqpoint{1.157838in}{2.049320in}}%
\pgfpathlineto{\pgfqpoint{1.196436in}{2.053111in}}%
\pgfpathlineto{\pgfqpoint{1.238016in}{2.055442in}}%
\pgfpathlineto{\pgfqpoint{1.282760in}{2.056284in}}%
\pgfpathlineto{\pgfqpoint{1.330863in}{2.055607in}}%
\pgfpathlineto{\pgfqpoint{1.409767in}{2.051664in}}%
\pgfpathlineto{\pgfqpoint{1.497431in}{2.044084in}}%
\pgfpathlineto{\pgfqpoint{1.594651in}{2.032713in}}%
\pgfpathlineto{\pgfqpoint{1.702203in}{2.017368in}}%
\pgfpathlineto{\pgfqpoint{1.820641in}{1.997817in}}%
\pgfpathlineto{\pgfqpoint{1.950022in}{1.973848in}}%
\pgfpathlineto{\pgfqpoint{2.089549in}{1.945290in}}%
\pgfpathlineto{\pgfqpoint{2.237536in}{1.912028in}}%
\pgfpathlineto{\pgfqpoint{2.339355in}{1.887237in}}%
\pgfpathlineto{\pgfqpoint{2.442176in}{1.860406in}}%
\pgfpathlineto{\pgfqpoint{2.544556in}{1.831625in}}%
\pgfpathlineto{\pgfqpoint{2.644947in}{1.801030in}}%
\pgfpathlineto{\pgfqpoint{2.741691in}{1.768798in}}%
\pgfpathlineto{\pgfqpoint{2.833007in}{1.735115in}}%
\pgfpathlineto{\pgfqpoint{2.917722in}{1.700245in}}%
\pgfpathlineto{\pgfqpoint{2.995233in}{1.664514in}}%
\pgfpathlineto{\pgfqpoint{3.065139in}{1.628224in}}%
\pgfpathlineto{\pgfqpoint{3.127245in}{1.591653in}}%
\pgfpathlineto{\pgfqpoint{3.181561in}{1.555054in}}%
\pgfpathlineto{\pgfqpoint{3.228298in}{1.518657in}}%
\pgfpathlineto{\pgfqpoint{3.267874in}{1.482668in}}%
\pgfpathlineto{\pgfqpoint{3.300911in}{1.447269in}}%
\pgfpathlineto{\pgfqpoint{3.328233in}{1.412618in}}%
\pgfpathlineto{\pgfqpoint{3.350410in}{1.378869in}}%
\pgfpathlineto{\pgfqpoint{3.367648in}{1.346156in}}%
\pgfpathlineto{\pgfqpoint{3.380870in}{1.314544in}}%
\pgfpathlineto{\pgfqpoint{3.390811in}{1.284088in}}%
\pgfpathlineto{\pgfqpoint{3.397986in}{1.254835in}}%
\pgfpathlineto{\pgfqpoint{3.402692in}{1.226822in}}%
\pgfpathlineto{\pgfqpoint{3.405013in}{1.200081in}}%
\pgfpathlineto{\pgfqpoint{3.404811in}{1.174635in}}%
\pgfpathlineto{\pgfqpoint{3.401735in}{1.150498in}}%
\pgfpathlineto{\pgfqpoint{3.395233in}{1.127677in}}%
\pgfpathlineto{\pgfqpoint{3.385868in}{1.106209in}}%
\pgfpathlineto{\pgfqpoint{3.374304in}{1.086116in}}%
\pgfpathlineto{\pgfqpoint{3.360627in}{1.067402in}}%
\pgfpathlineto{\pgfqpoint{3.344887in}{1.050071in}}%
\pgfpathlineto{\pgfqpoint{3.327102in}{1.034123in}}%
\pgfpathlineto{\pgfqpoint{3.307257in}{1.019558in}}%
\pgfpathlineto{\pgfqpoint{3.285301in}{1.006376in}}%
\pgfpathlineto{\pgfqpoint{3.261151in}{0.994575in}}%
\pgfpathlineto{\pgfqpoint{3.234689in}{0.984151in}}%
\pgfpathlineto{\pgfqpoint{3.205783in}{0.975102in}}%
\pgfpathlineto{\pgfqpoint{3.174460in}{0.967441in}}%
\pgfpathlineto{\pgfqpoint{3.140613in}{0.961185in}}%
\pgfpathlineto{\pgfqpoint{3.104080in}{0.956349in}}%
\pgfpathlineto{\pgfqpoint{3.064690in}{0.952955in}}%
\pgfpathlineto{\pgfqpoint{3.022263in}{0.951028in}}%
\pgfpathlineto{\pgfqpoint{2.976610in}{0.950597in}}%
\pgfpathlineto{\pgfqpoint{2.927534in}{0.951694in}}%
\pgfpathlineto{\pgfqpoint{2.847048in}{0.956288in}}%
\pgfpathlineto{\pgfqpoint{2.757658in}{0.964549in}}%
\pgfpathlineto{\pgfqpoint{2.658585in}{0.976638in}}%
\pgfpathlineto{\pgfqpoint{2.549062in}{0.992741in}}%
\pgfpathlineto{\pgfqpoint{2.428609in}{1.013087in}}%
\pgfpathlineto{\pgfqpoint{2.297254in}{1.037891in}}%
\pgfpathlineto{\pgfqpoint{2.155985in}{1.067311in}}%
\pgfpathlineto{\pgfqpoint{2.006778in}{1.101442in}}%
\pgfpathlineto{\pgfqpoint{1.904468in}{1.126805in}}%
\pgfpathlineto{\pgfqpoint{1.801567in}{1.154188in}}%
\pgfpathlineto{\pgfqpoint{1.699605in}{1.183486in}}%
\pgfpathlineto{\pgfqpoint{1.600121in}{1.214552in}}%
\pgfpathlineto{\pgfqpoint{1.504667in}{1.247199in}}%
\pgfpathlineto{\pgfqpoint{1.414840in}{1.281199in}}%
\pgfpathlineto{\pgfqpoint{1.332556in}{1.316307in}}%
\pgfpathlineto{\pgfqpoint{1.258096in}{1.352215in}}%
\pgfpathlineto{\pgfqpoint{1.191226in}{1.388616in}}%
\pgfpathlineto{\pgfqpoint{1.131689in}{1.425231in}}%
\pgfpathlineto{\pgfqpoint{1.079215in}{1.461809in}}%
\pgfpathlineto{\pgfqpoint{1.033511in}{1.498122in}}%
\pgfpathlineto{\pgfqpoint{0.994271in}{1.533971in}}%
\pgfpathlineto{\pgfqpoint{0.961166in}{1.569185in}}%
\pgfpathlineto{\pgfqpoint{0.933853in}{1.603616in}}%
\pgfpathlineto{\pgfqpoint{0.911969in}{1.637145in}}%
\pgfpathlineto{\pgfqpoint{0.895132in}{1.669680in}}%
\pgfpathlineto{\pgfqpoint{0.882940in}{1.701136in}}%
\pgfpathlineto{\pgfqpoint{0.874799in}{1.731380in}}%
\pgfpathlineto{\pgfqpoint{0.869975in}{1.760365in}}%
\pgfpathlineto{\pgfqpoint{0.867875in}{1.788062in}}%
\pgfpathlineto{\pgfqpoint{0.868062in}{1.814446in}}%
\pgfpathlineto{\pgfqpoint{0.870255in}{1.839500in}}%
\pgfpathlineto{\pgfqpoint{0.874326in}{1.863214in}}%
\pgfpathlineto{\pgfqpoint{0.880305in}{1.885582in}}%
\pgfpathlineto{\pgfqpoint{0.888375in}{1.906606in}}%
\pgfpathlineto{\pgfqpoint{0.898874in}{1.926294in}}%
\pgfpathlineto{\pgfqpoint{0.912296in}{1.944661in}}%
\pgfpathlineto{\pgfqpoint{0.928704in}{1.961696in}}%
\pgfpathlineto{\pgfqpoint{0.947261in}{1.977352in}}%
\pgfpathlineto{\pgfqpoint{0.967935in}{1.991624in}}%
\pgfpathlineto{\pgfqpoint{0.990745in}{2.004509in}}%
\pgfpathlineto{\pgfqpoint{1.015735in}{2.016002in}}%
\pgfpathlineto{\pgfqpoint{1.042969in}{2.026100in}}%
\pgfpathlineto{\pgfqpoint{1.072540in}{2.034795in}}%
\pgfpathlineto{\pgfqpoint{1.104559in}{2.042081in}}%
\pgfpathlineto{\pgfqpoint{1.139166in}{2.047951in}}%
\pgfpathlineto{\pgfqpoint{1.176521in}{2.052396in}}%
\pgfpathlineto{\pgfqpoint{1.216756in}{2.055408in}}%
\pgfpathlineto{\pgfqpoint{1.259821in}{2.056969in}}%
\pgfpathlineto{\pgfqpoint{1.306063in}{2.057038in}}%
\pgfpathlineto{\pgfqpoint{1.355847in}{2.055570in}}%
\pgfpathlineto{\pgfqpoint{1.437842in}{2.050382in}}%
\pgfpathlineto{\pgfqpoint{1.529355in}{2.041468in}}%
\pgfpathlineto{\pgfqpoint{1.630974in}{2.028655in}}%
\pgfpathlineto{\pgfqpoint{1.743031in}{2.011768in}}%
\pgfpathlineto{\pgfqpoint{1.865597in}{1.990619in}}%
\pgfpathlineto{\pgfqpoint{1.998486in}{1.965017in}}%
\pgfpathlineto{\pgfqpoint{2.141272in}{1.934759in}}%
\pgfpathlineto{\pgfqpoint{2.292892in}{1.899607in}}%
\pgfpathlineto{\pgfqpoint{2.396270in}{1.873540in}}%
\pgfpathlineto{\pgfqpoint{2.499388in}{1.845507in}}%
\pgfpathlineto{\pgfqpoint{2.600642in}{1.815653in}}%
\pgfpathlineto{\pgfqpoint{2.698606in}{1.784145in}}%
\pgfpathlineto{\pgfqpoint{2.792024in}{1.751171in}}%
\pgfpathlineto{\pgfqpoint{2.879816in}{1.716939in}}%
\pgfpathlineto{\pgfqpoint{2.961079in}{1.681681in}}%
\pgfpathlineto{\pgfqpoint{3.035080in}{1.645648in}}%
\pgfpathlineto{\pgfqpoint{3.101263in}{1.609113in}}%
\pgfpathlineto{\pgfqpoint{3.159246in}{1.572370in}}%
\pgfpathlineto{\pgfqpoint{3.208960in}{1.535731in}}%
\pgfpathlineto{\pgfqpoint{3.251275in}{1.499437in}}%
\pgfpathlineto{\pgfqpoint{3.287149in}{1.463669in}}%
\pgfpathlineto{\pgfqpoint{3.317354in}{1.428585in}}%
\pgfpathlineto{\pgfqpoint{3.342481in}{1.394329in}}%
\pgfpathlineto{\pgfqpoint{3.362944in}{1.361022in}}%
\pgfpathlineto{\pgfqpoint{3.378977in}{1.328771in}}%
\pgfpathlineto{\pgfqpoint{3.390639in}{1.297662in}}%
\pgfpathlineto{\pgfqpoint{3.398467in}{1.267769in}}%
\pgfpathlineto{\pgfqpoint{3.403104in}{1.239137in}}%
\pgfpathlineto{\pgfqpoint{3.404876in}{1.211799in}}%
\pgfpathlineto{\pgfqpoint{3.404030in}{1.185780in}}%
\pgfpathlineto{\pgfqpoint{3.400736in}{1.161101in}}%
\pgfpathlineto{\pgfqpoint{3.395085in}{1.137773in}}%
\pgfpathlineto{\pgfqpoint{3.387088in}{1.115803in}}%
\pgfpathlineto{\pgfqpoint{3.376713in}{1.095191in}}%
\pgfpathlineto{\pgfqpoint{3.364091in}{1.075942in}}%
\pgfpathlineto{\pgfqpoint{3.349323in}{1.058061in}}%
\pgfpathlineto{\pgfqpoint{3.332466in}{1.041552in}}%
\pgfpathlineto{\pgfqpoint{3.313547in}{1.026421in}}%
\pgfpathlineto{\pgfqpoint{3.292558in}{1.012669in}}%
\pgfpathlineto{\pgfqpoint{3.269460in}{1.000300in}}%
\pgfpathlineto{\pgfqpoint{3.244180in}{0.989314in}}%
\pgfpathlineto{\pgfqpoint{3.216616in}{0.979714in}}%
\pgfpathlineto{\pgfqpoint{3.186630in}{0.971498in}}%
\pgfpathlineto{\pgfqpoint{3.154055in}{0.964667in}}%
\pgfpathlineto{\pgfqpoint{3.118780in}{0.959225in}}%
\pgfpathlineto{\pgfqpoint{3.080838in}{0.955203in}}%
\pgfpathlineto{\pgfqpoint{3.039992in}{0.952628in}}%
\pgfpathlineto{\pgfqpoint{2.995997in}{0.951527in}}%
\pgfpathlineto{\pgfqpoint{2.948618in}{0.951934in}}%
\pgfpathlineto{\pgfqpoint{2.870717in}{0.955453in}}%
\pgfpathlineto{\pgfqpoint{2.783991in}{0.962591in}}%
\pgfpathlineto{\pgfqpoint{2.687786in}{0.973510in}}%
\pgfpathlineto{\pgfqpoint{2.581504in}{0.988391in}}%
\pgfpathlineto{\pgfqpoint{2.464601in}{1.007437in}}%
\pgfpathlineto{\pgfqpoint{2.336498in}{1.030853in}}%
\pgfpathlineto{\pgfqpoint{2.197430in}{1.058804in}}%
\pgfpathlineto{\pgfqpoint{2.050137in}{1.091486in}}%
\pgfpathlineto{\pgfqpoint{1.949043in}{1.115915in}}%
\pgfpathlineto{\pgfqpoint{1.846943in}{1.142407in}}%
\pgfpathlineto{\pgfqpoint{1.745074in}{1.170871in}}%
\pgfpathlineto{\pgfqpoint{1.644767in}{1.201174in}}%
\pgfpathlineto{\pgfqpoint{1.547453in}{1.233145in}}%
\pgfpathlineto{\pgfqpoint{1.454663in}{1.266570in}}%
\pgfpathlineto{\pgfqpoint{1.368020in}{1.301192in}}%
\pgfpathlineto{\pgfqpoint{1.289155in}{1.336721in}}%
\pgfpathlineto{\pgfqpoint{1.218523in}{1.372881in}}%
\pgfpathlineto{\pgfqpoint{1.155755in}{1.409391in}}%
\pgfpathlineto{\pgfqpoint{1.100466in}{1.445992in}}%
\pgfpathlineto{\pgfqpoint{1.052272in}{1.482448in}}%
\pgfpathlineto{\pgfqpoint{1.010793in}{1.518545in}}%
\pgfpathlineto{\pgfqpoint{0.975651in}{1.554091in}}%
\pgfpathlineto{\pgfqpoint{0.946470in}{1.588917in}}%
\pgfpathlineto{\pgfqpoint{0.922874in}{1.622878in}}%
\pgfpathlineto{\pgfqpoint{0.904478in}{1.655846in}}%
\pgfpathlineto{\pgfqpoint{0.890622in}{1.687707in}}%
\pgfpathlineto{\pgfqpoint{0.880559in}{1.718395in}}%
\pgfpathlineto{\pgfqpoint{0.873700in}{1.747865in}}%
\pgfpathlineto{\pgfqpoint{0.869610in}{1.776077in}}%
\pgfpathlineto{\pgfqpoint{0.868007in}{1.802999in}}%
\pgfpathlineto{\pgfqpoint{0.868765in}{1.828609in}}%
\pgfpathlineto{\pgfqpoint{0.871912in}{1.852890in}}%
\pgfpathlineto{\pgfqpoint{0.877631in}{1.875836in}}%
\pgfpathlineto{\pgfqpoint{0.886238in}{1.897447in}}%
\pgfpathlineto{\pgfqpoint{0.897383in}{1.917702in}}%
\pgfpathlineto{\pgfqpoint{0.910698in}{1.936585in}}%
\pgfpathlineto{\pgfqpoint{0.926116in}{1.954092in}}%
\pgfpathlineto{\pgfqpoint{0.943601in}{1.970222in}}%
\pgfpathlineto{\pgfqpoint{0.963149in}{1.984970in}}%
\pgfpathlineto{\pgfqpoint{0.984788in}{1.998336in}}%
\pgfpathlineto{\pgfqpoint{1.008579in}{2.010320in}}%
\pgfpathlineto{\pgfqpoint{1.034610in}{2.020921in}}%
\pgfpathlineto{\pgfqpoint{1.063006in}{2.030140in}}%
\pgfpathlineto{\pgfqpoint{1.093918in}{2.037979in}}%
\pgfpathlineto{\pgfqpoint{1.127363in}{2.044427in}}%
\pgfpathlineto{\pgfqpoint{1.163412in}{2.049462in}}%
\pgfpathlineto{\pgfqpoint{1.202271in}{2.053063in}}%
\pgfpathlineto{\pgfqpoint{1.244144in}{2.055206in}}%
\pgfpathlineto{\pgfqpoint{1.289238in}{2.055861in}}%
\pgfpathlineto{\pgfqpoint{1.337758in}{2.054996in}}%
\pgfpathlineto{\pgfqpoint{1.417413in}{2.050765in}}%
\pgfpathlineto{\pgfqpoint{1.505939in}{2.042882in}}%
\pgfpathlineto{\pgfqpoint{1.604034in}{2.031184in}}%
\pgfpathlineto{\pgfqpoint{1.712399in}{2.015482in}}%
\pgfpathlineto{\pgfqpoint{1.831666in}{1.995571in}}%
\pgfpathlineto{\pgfqpoint{1.961844in}{1.971233in}}%
\pgfpathlineto{\pgfqpoint{2.102147in}{1.942280in}}%
\pgfpathlineto{\pgfqpoint{2.250587in}{1.908629in}}%
\pgfpathlineto{\pgfqpoint{2.352523in}{1.883592in}}%
\pgfpathlineto{\pgfqpoint{2.352523in}{1.883592in}}%
\pgfusepath{stroke}%
\end{pgfscope}%
\begin{pgfscope}%
\pgfpathrectangle{\pgfqpoint{0.562500in}{0.275000in}}{\pgfqpoint{3.487500in}{1.925000in}}%
\pgfusepath{clip}%
\pgfsetrectcap%
\pgfsetroundjoin%
\pgfsetlinewidth{1.505625pt}%
\definecolor{currentstroke}{rgb}{1.000000,0.498039,0.054902}%
\pgfsetstrokecolor{currentstroke}%
\pgfsetdash{}{0pt}%
\pgfpathmoveto{\pgfqpoint{3.891477in}{0.362500in}}%
\pgfpathlineto{\pgfqpoint{3.775708in}{0.364351in}}%
\pgfpathlineto{\pgfqpoint{3.658609in}{0.369795in}}%
\pgfpathlineto{\pgfqpoint{3.537886in}{0.378749in}}%
\pgfpathlineto{\pgfqpoint{3.411516in}{0.391189in}}%
\pgfpathlineto{\pgfqpoint{3.277730in}{0.407144in}}%
\pgfpathlineto{\pgfqpoint{3.135019in}{0.426694in}}%
\pgfpathlineto{\pgfqpoint{2.982127in}{0.449970in}}%
\pgfpathlineto{\pgfqpoint{2.818127in}{0.477135in}}%
\pgfpathlineto{\pgfqpoint{2.643006in}{0.508355in}}%
\pgfpathlineto{\pgfqpoint{2.458083in}{0.543794in}}%
\pgfpathlineto{\pgfqpoint{2.266212in}{0.583502in}}%
\pgfpathlineto{\pgfqpoint{2.071788in}{0.627397in}}%
\pgfpathlineto{\pgfqpoint{1.975415in}{0.650855in}}%
\pgfpathlineto{\pgfqpoint{1.880744in}{0.675275in}}%
\pgfpathlineto{\pgfqpoint{1.788632in}{0.700611in}}%
\pgfpathlineto{\pgfqpoint{1.699849in}{0.726778in}}%
\pgfpathlineto{\pgfqpoint{1.615047in}{0.753684in}}%
\pgfpathlineto{\pgfqpoint{1.534752in}{0.781238in}}%
\pgfpathlineto{\pgfqpoint{1.459363in}{0.809349in}}%
\pgfpathlineto{\pgfqpoint{1.389154in}{0.837927in}}%
\pgfpathlineto{\pgfqpoint{1.324270in}{0.866882in}}%
\pgfpathlineto{\pgfqpoint{1.264731in}{0.896126in}}%
\pgfpathlineto{\pgfqpoint{1.210431in}{0.925569in}}%
\pgfpathlineto{\pgfqpoint{1.161135in}{0.955123in}}%
\pgfpathlineto{\pgfqpoint{1.116503in}{0.984699in}}%
\pgfpathlineto{\pgfqpoint{1.076185in}{1.014209in}}%
\pgfpathlineto{\pgfqpoint{1.039763in}{1.043604in}}%
\pgfpathlineto{\pgfqpoint{1.006843in}{1.072837in}}%
\pgfpathlineto{\pgfqpoint{0.977078in}{1.101863in}}%
\pgfpathlineto{\pgfqpoint{0.950162in}{1.130638in}}%
\pgfpathlineto{\pgfqpoint{0.925836in}{1.159122in}}%
\pgfpathlineto{\pgfqpoint{0.903883in}{1.187273in}}%
\pgfpathlineto{\pgfqpoint{0.884129in}{1.215054in}}%
\pgfpathlineto{\pgfqpoint{0.866326in}{1.242434in}}%
\pgfpathlineto{\pgfqpoint{0.835464in}{1.295963in}}%
\pgfpathlineto{\pgfqpoint{0.809907in}{1.347781in}}%
\pgfpathlineto{\pgfqpoint{0.788710in}{1.397820in}}%
\pgfpathlineto{\pgfqpoint{0.771235in}{1.446023in}}%
\pgfpathlineto{\pgfqpoint{0.756971in}{1.492354in}}%
\pgfpathlineto{\pgfqpoint{0.745453in}{1.536830in}}%
\pgfpathlineto{\pgfqpoint{0.736350in}{1.579472in}}%
\pgfpathlineto{\pgfqpoint{0.729436in}{1.620299in}}%
\pgfpathlineto{\pgfqpoint{0.724587in}{1.659326in}}%
\pgfpathlineto{\pgfqpoint{0.721780in}{1.696564in}}%
\pgfpathlineto{\pgfqpoint{0.721023in}{1.732028in}}%
\pgfpathlineto{\pgfqpoint{0.722130in}{1.765750in}}%
\pgfpathlineto{\pgfqpoint{0.724981in}{1.797764in}}%
\pgfpathlineto{\pgfqpoint{0.729502in}{1.828102in}}%
\pgfpathlineto{\pgfqpoint{0.735663in}{1.856791in}}%
\pgfpathlineto{\pgfqpoint{0.743471in}{1.883859in}}%
\pgfpathlineto{\pgfqpoint{0.752978in}{1.909329in}}%
\pgfpathlineto{\pgfqpoint{0.764277in}{1.933224in}}%
\pgfpathlineto{\pgfqpoint{0.777479in}{1.955567in}}%
\pgfpathlineto{\pgfqpoint{0.792557in}{1.976384in}}%
\pgfpathlineto{\pgfqpoint{0.809435in}{1.995686in}}%
\pgfpathlineto{\pgfqpoint{0.828076in}{2.013485in}}%
\pgfpathlineto{\pgfqpoint{0.848479in}{2.029792in}}%
\pgfpathlineto{\pgfqpoint{0.870683in}{2.044615in}}%
\pgfpathlineto{\pgfqpoint{0.894761in}{2.057965in}}%
\pgfpathlineto{\pgfqpoint{0.920827in}{2.069850in}}%
\pgfpathlineto{\pgfqpoint{0.949032in}{2.080278in}}%
\pgfpathlineto{\pgfqpoint{0.979564in}{2.089256in}}%
\pgfpathlineto{\pgfqpoint{1.012650in}{2.096791in}}%
\pgfpathlineto{\pgfqpoint{1.048550in}{2.102888in}}%
\pgfpathlineto{\pgfqpoint{1.087251in}{2.107523in}}%
\pgfpathlineto{\pgfqpoint{1.128849in}{2.110662in}}%
\pgfpathlineto{\pgfqpoint{1.173652in}{2.112275in}}%
\pgfpathlineto{\pgfqpoint{1.221966in}{2.112330in}}%
\pgfpathlineto{\pgfqpoint{1.274096in}{2.110786in}}%
\pgfpathlineto{\pgfqpoint{1.360099in}{2.105365in}}%
\pgfpathlineto{\pgfqpoint{1.456362in}{2.096052in}}%
\pgfpathlineto{\pgfqpoint{1.563874in}{2.082626in}}%
\pgfpathlineto{\pgfqpoint{1.683606in}{2.064836in}}%
\pgfpathlineto{\pgfqpoint{1.816516in}{2.042395in}}%
\pgfpathlineto{\pgfqpoint{1.963240in}{2.015037in}}%
\pgfpathlineto{\pgfqpoint{2.122546in}{1.982451in}}%
\pgfpathlineto{\pgfqpoint{2.234326in}{1.957713in}}%
\pgfpathlineto{\pgfqpoint{2.349142in}{1.930564in}}%
\pgfpathlineto{\pgfqpoint{2.465439in}{1.901061in}}%
\pgfpathlineto{\pgfqpoint{2.581362in}{1.869316in}}%
\pgfpathlineto{\pgfqpoint{2.694759in}{1.835491in}}%
\pgfpathlineto{\pgfqpoint{2.803180in}{1.799802in}}%
\pgfpathlineto{\pgfqpoint{2.904305in}{1.762490in}}%
\pgfpathlineto{\pgfqpoint{2.996997in}{1.723912in}}%
\pgfpathlineto{\pgfqpoint{3.039968in}{1.704266in}}%
\pgfpathlineto{\pgfqpoint{3.080612in}{1.684443in}}%
\pgfpathlineto{\pgfqpoint{3.118897in}{1.664485in}}%
\pgfpathlineto{\pgfqpoint{3.154807in}{1.644433in}}%
\pgfpathlineto{\pgfqpoint{3.188348in}{1.624326in}}%
\pgfpathlineto{\pgfqpoint{3.219542in}{1.604201in}}%
\pgfpathlineto{\pgfqpoint{3.248432in}{1.584094in}}%
\pgfpathlineto{\pgfqpoint{3.275079in}{1.564038in}}%
\pgfpathlineto{\pgfqpoint{3.299564in}{1.544064in}}%
\pgfpathlineto{\pgfqpoint{3.321985in}{1.524204in}}%
\pgfpathlineto{\pgfqpoint{3.342460in}{1.504486in}}%
\pgfpathlineto{\pgfqpoint{3.378141in}{1.465576in}}%
\pgfpathlineto{\pgfqpoint{3.393567in}{1.446443in}}%
\pgfpathlineto{\pgfqpoint{3.419583in}{1.408978in}}%
\pgfpathlineto{\pgfqpoint{3.440405in}{1.372614in}}%
\pgfpathlineto{\pgfqpoint{3.457273in}{1.337386in}}%
\pgfpathlineto{\pgfqpoint{3.471063in}{1.303335in}}%
\pgfpathlineto{\pgfqpoint{3.482284in}{1.270501in}}%
\pgfpathlineto{\pgfqpoint{3.491081in}{1.238930in}}%
\pgfpathlineto{\pgfqpoint{3.497232in}{1.208670in}}%
\pgfpathlineto{\pgfqpoint{3.500147in}{1.179771in}}%
\pgfpathlineto{\pgfqpoint{3.500108in}{1.165849in}}%
\pgfpathlineto{\pgfqpoint{3.496708in}{1.139091in}}%
\pgfpathlineto{\pgfqpoint{3.490720in}{1.113794in}}%
\pgfpathlineto{\pgfqpoint{3.482642in}{1.089953in}}%
\pgfpathlineto{\pgfqpoint{3.472572in}{1.067563in}}%
\pgfpathlineto{\pgfqpoint{3.460570in}{1.046619in}}%
\pgfpathlineto{\pgfqpoint{3.446653in}{1.027112in}}%
\pgfpathlineto{\pgfqpoint{3.430796in}{1.009036in}}%
\pgfpathlineto{\pgfqpoint{3.412933in}{0.992379in}}%
\pgfpathlineto{\pgfqpoint{3.392955in}{0.977134in}}%
\pgfpathlineto{\pgfqpoint{3.370801in}{0.963291in}}%
\pgfpathlineto{\pgfqpoint{3.346498in}{0.950851in}}%
\pgfpathlineto{\pgfqpoint{3.319992in}{0.939820in}}%
\pgfpathlineto{\pgfqpoint{3.291202in}{0.930203in}}%
\pgfpathlineto{\pgfqpoint{3.260021in}{0.922010in}}%
\pgfpathlineto{\pgfqpoint{3.226316in}{0.915254in}}%
\pgfpathlineto{\pgfqpoint{3.189930in}{0.909951in}}%
\pgfpathlineto{\pgfqpoint{3.150679in}{0.906118in}}%
\pgfpathlineto{\pgfqpoint{3.108351in}{0.903777in}}%
\pgfpathlineto{\pgfqpoint{3.062711in}{0.902952in}}%
\pgfpathlineto{\pgfqpoint{3.013496in}{0.903670in}}%
\pgfpathlineto{\pgfqpoint{2.932376in}{0.907708in}}%
\pgfpathlineto{\pgfqpoint{2.841820in}{0.915454in}}%
\pgfpathlineto{\pgfqpoint{2.740253in}{0.927174in}}%
\pgfpathlineto{\pgfqpoint{2.626408in}{0.943137in}}%
\pgfpathlineto{\pgfqpoint{2.499613in}{0.963607in}}%
\pgfpathlineto{\pgfqpoint{2.359802in}{0.988844in}}%
\pgfpathlineto{\pgfqpoint{2.207507in}{1.019101in}}%
\pgfpathlineto{\pgfqpoint{2.043764in}{1.054641in}}%
\pgfpathlineto{\pgfqpoint{1.929554in}{1.081381in}}%
\pgfpathlineto{\pgfqpoint{1.813952in}{1.110460in}}%
\pgfpathlineto{\pgfqpoint{1.699233in}{1.141736in}}%
\pgfpathlineto{\pgfqpoint{1.587428in}{1.175031in}}%
\pgfpathlineto{\pgfqpoint{1.480326in}{1.210138in}}%
\pgfpathlineto{\pgfqpoint{1.379472in}{1.246819in}}%
\pgfpathlineto{\pgfqpoint{1.286170in}{1.284803in}}%
\pgfpathlineto{\pgfqpoint{1.242691in}{1.304190in}}%
\pgfpathlineto{\pgfqpoint{1.201478in}{1.323787in}}%
\pgfpathlineto{\pgfqpoint{1.162626in}{1.343551in}}%
\pgfpathlineto{\pgfqpoint{1.126215in}{1.363437in}}%
\pgfpathlineto{\pgfqpoint{1.092307in}{1.383398in}}%
\pgfpathlineto{\pgfqpoint{1.060919in}{1.403387in}}%
\pgfpathlineto{\pgfqpoint{1.031877in}{1.423361in}}%
\pgfpathlineto{\pgfqpoint{1.005033in}{1.443287in}}%
\pgfpathlineto{\pgfqpoint{0.980257in}{1.463137in}}%
\pgfpathlineto{\pgfqpoint{0.936440in}{1.502493in}}%
\pgfpathlineto{\pgfqpoint{0.899581in}{1.541218in}}%
\pgfpathlineto{\pgfqpoint{0.868987in}{1.579128in}}%
\pgfpathlineto{\pgfqpoint{0.844120in}{1.616061in}}%
\pgfpathlineto{\pgfqpoint{0.824491in}{1.651883in}}%
\pgfpathlineto{\pgfqpoint{0.809243in}{1.686501in}}%
\pgfpathlineto{\pgfqpoint{0.797772in}{1.719856in}}%
\pgfpathlineto{\pgfqpoint{0.789628in}{1.751897in}}%
\pgfpathlineto{\pgfqpoint{0.784489in}{1.782585in}}%
\pgfpathlineto{\pgfqpoint{0.782155in}{1.811889in}}%
\pgfpathlineto{\pgfqpoint{0.782545in}{1.839792in}}%
\pgfpathlineto{\pgfqpoint{0.785500in}{1.866281in}}%
\pgfpathlineto{\pgfqpoint{0.790767in}{1.891352in}}%
\pgfpathlineto{\pgfqpoint{0.798149in}{1.915001in}}%
\pgfpathlineto{\pgfqpoint{0.807512in}{1.937229in}}%
\pgfpathlineto{\pgfqpoint{0.818781in}{1.958038in}}%
\pgfpathlineto{\pgfqpoint{0.831945in}{1.977433in}}%
\pgfpathlineto{\pgfqpoint{0.847050in}{1.995422in}}%
\pgfpathlineto{\pgfqpoint{0.864208in}{2.012013in}}%
\pgfpathlineto{\pgfqpoint{0.883589in}{2.027220in}}%
\pgfpathlineto{\pgfqpoint{0.905321in}{2.041050in}}%
\pgfpathlineto{\pgfqpoint{0.929238in}{2.053490in}}%
\pgfpathlineto{\pgfqpoint{0.955374in}{2.064534in}}%
\pgfpathlineto{\pgfqpoint{0.983808in}{2.074174in}}%
\pgfpathlineto{\pgfqpoint{1.014642in}{2.082402in}}%
\pgfpathlineto{\pgfqpoint{1.048006in}{2.089203in}}%
\pgfpathlineto{\pgfqpoint{1.084054in}{2.094562in}}%
\pgfpathlineto{\pgfqpoint{1.122965in}{2.098460in}}%
\pgfpathlineto{\pgfqpoint{1.164944in}{2.100876in}}%
\pgfpathlineto{\pgfqpoint{1.210220in}{2.101784in}}%
\pgfpathlineto{\pgfqpoint{1.259049in}{2.101157in}}%
\pgfpathlineto{\pgfqpoint{1.339512in}{2.097270in}}%
\pgfpathlineto{\pgfqpoint{1.429339in}{2.089693in}}%
\pgfpathlineto{\pgfqpoint{1.530141in}{2.078162in}}%
\pgfpathlineto{\pgfqpoint{1.643161in}{2.062411in}}%
\pgfpathlineto{\pgfqpoint{1.769059in}{2.042177in}}%
\pgfpathlineto{\pgfqpoint{1.907912in}{2.017204in}}%
\pgfpathlineto{\pgfqpoint{2.059213in}{1.987241in}}%
\pgfpathlineto{\pgfqpoint{2.221949in}{1.952026in}}%
\pgfpathlineto{\pgfqpoint{2.335673in}{1.925500in}}%
\pgfpathlineto{\pgfqpoint{2.450994in}{1.896626in}}%
\pgfpathlineto{\pgfqpoint{2.565593in}{1.865553in}}%
\pgfpathlineto{\pgfqpoint{2.677408in}{1.832460in}}%
\pgfpathlineto{\pgfqpoint{2.784632in}{1.797553in}}%
\pgfpathlineto{\pgfqpoint{2.885713in}{1.761068in}}%
\pgfpathlineto{\pgfqpoint{2.979358in}{1.723269in}}%
\pgfpathlineto{\pgfqpoint{3.064526in}{1.684450in}}%
\pgfpathlineto{\pgfqpoint{3.103678in}{1.664758in}}%
\pgfpathlineto{\pgfqpoint{3.140436in}{1.644934in}}%
\pgfpathlineto{\pgfqpoint{3.174745in}{1.625023in}}%
\pgfpathlineto{\pgfqpoint{3.206561in}{1.605072in}}%
\pgfpathlineto{\pgfqpoint{3.235929in}{1.585128in}}%
\pgfpathlineto{\pgfqpoint{3.263047in}{1.565229in}}%
\pgfpathlineto{\pgfqpoint{3.288059in}{1.545404in}}%
\pgfpathlineto{\pgfqpoint{3.332291in}{1.506089in}}%
\pgfpathlineto{\pgfqpoint{3.369557in}{1.467387in}}%
\pgfpathlineto{\pgfqpoint{3.400605in}{1.429482in}}%
\pgfpathlineto{\pgfqpoint{3.426001in}{1.392533in}}%
\pgfpathlineto{\pgfqpoint{3.446132in}{1.356677in}}%
\pgfpathlineto{\pgfqpoint{3.461654in}{1.322019in}}%
\pgfpathlineto{\pgfqpoint{3.473331in}{1.288623in}}%
\pgfpathlineto{\pgfqpoint{3.481637in}{1.256537in}}%
\pgfpathlineto{\pgfqpoint{3.486930in}{1.225799in}}%
\pgfpathlineto{\pgfqpoint{3.489451in}{1.196440in}}%
\pgfpathlineto{\pgfqpoint{3.489324in}{1.168480in}}%
\pgfpathlineto{\pgfqpoint{3.486599in}{1.141930in}}%
\pgfpathlineto{\pgfqpoint{3.481474in}{1.116796in}}%
\pgfpathlineto{\pgfqpoint{3.474162in}{1.093081in}}%
\pgfpathlineto{\pgfqpoint{3.464820in}{1.070786in}}%
\pgfpathlineto{\pgfqpoint{3.453550in}{1.049910in}}%
\pgfpathlineto{\pgfqpoint{3.440395in}{1.030449in}}%
\pgfpathlineto{\pgfqpoint{3.425340in}{1.012398in}}%
\pgfpathlineto{\pgfqpoint{3.408318in}{0.995748in}}%
\pgfpathlineto{\pgfqpoint{3.389199in}{0.980490in}}%
\pgfpathlineto{\pgfqpoint{3.367803in}{0.966611in}}%
\pgfpathlineto{\pgfqpoint{3.344031in}{0.954108in}}%
\pgfpathlineto{\pgfqpoint{3.318028in}{0.942998in}}%
\pgfpathlineto{\pgfqpoint{3.289724in}{0.933286in}}%
\pgfpathlineto{\pgfqpoint{3.259017in}{0.924983in}}%
\pgfpathlineto{\pgfqpoint{3.225781in}{0.918102in}}%
\pgfpathlineto{\pgfqpoint{3.189865in}{0.912660in}}%
\pgfpathlineto{\pgfqpoint{3.151093in}{0.908677in}}%
\pgfpathlineto{\pgfqpoint{3.109264in}{0.906176in}}%
\pgfpathlineto{\pgfqpoint{3.064154in}{0.905185in}}%
\pgfpathlineto{\pgfqpoint{3.015512in}{0.905735in}}%
\pgfpathlineto{\pgfqpoint{2.935323in}{0.909524in}}%
\pgfpathlineto{\pgfqpoint{2.845612in}{0.917001in}}%
\pgfpathlineto{\pgfqpoint{2.745223in}{0.928403in}}%
\pgfpathlineto{\pgfqpoint{2.632859in}{0.943992in}}%
\pgfpathlineto{\pgfqpoint{2.507708in}{0.964038in}}%
\pgfpathlineto{\pgfqpoint{2.369470in}{0.988815in}}%
\pgfpathlineto{\pgfqpoint{2.218342in}{1.018605in}}%
\pgfpathlineto{\pgfqpoint{2.055508in}{1.053629in}}%
\pgfpathlineto{\pgfqpoint{1.942603in}{1.079937in}}%
\pgfpathlineto{\pgfqpoint{1.828130in}{1.108577in}}%
\pgfpathlineto{\pgfqpoint{1.713837in}{1.139459in}}%
\pgfpathlineto{\pgfqpoint{1.601574in}{1.172436in}}%
\pgfpathlineto{\pgfqpoint{1.493286in}{1.207309in}}%
\pgfpathlineto{\pgfqpoint{1.391019in}{1.243822in}}%
\pgfpathlineto{\pgfqpoint{1.296834in}{1.281669in}}%
\pgfpathlineto{\pgfqpoint{1.253132in}{1.300989in}}%
\pgfpathlineto{\pgfqpoint{1.211747in}{1.320513in}}%
\pgfpathlineto{\pgfqpoint{1.172704in}{1.340200in}}%
\pgfpathlineto{\pgfqpoint{1.136011in}{1.360006in}}%
\pgfpathlineto{\pgfqpoint{1.101661in}{1.379893in}}%
\pgfpathlineto{\pgfqpoint{1.069629in}{1.399821in}}%
\pgfpathlineto{\pgfqpoint{1.039875in}{1.419753in}}%
\pgfpathlineto{\pgfqpoint{1.012340in}{1.439654in}}%
\pgfpathlineto{\pgfqpoint{0.986949in}{1.459490in}}%
\pgfpathlineto{\pgfqpoint{0.963612in}{1.479228in}}%
\pgfpathlineto{\pgfqpoint{0.942220in}{1.498838in}}%
\pgfpathlineto{\pgfqpoint{0.904833in}{1.537556in}}%
\pgfpathlineto{\pgfqpoint{0.873961in}{1.575453in}}%
\pgfpathlineto{\pgfqpoint{0.848653in}{1.612397in}}%
\pgfpathlineto{\pgfqpoint{0.828148in}{1.648274in}}%
\pgfpathlineto{\pgfqpoint{0.811892in}{1.682986in}}%
\pgfpathlineto{\pgfqpoint{0.799547in}{1.716453in}}%
\pgfpathlineto{\pgfqpoint{0.790957in}{1.748610in}}%
\pgfpathlineto{\pgfqpoint{0.785684in}{1.779410in}}%
\pgfpathlineto{\pgfqpoint{0.783258in}{1.808830in}}%
\pgfpathlineto{\pgfqpoint{0.783363in}{1.836855in}}%
\pgfpathlineto{\pgfqpoint{0.785771in}{1.863473in}}%
\pgfpathlineto{\pgfqpoint{0.790346in}{1.888677in}}%
\pgfpathlineto{\pgfqpoint{0.797038in}{1.912467in}}%
\pgfpathlineto{\pgfqpoint{0.805885in}{1.934843in}}%
\pgfpathlineto{\pgfqpoint{0.817012in}{1.955813in}}%
\pgfpathlineto{\pgfqpoint{0.830363in}{1.975381in}}%
\pgfpathlineto{\pgfqpoint{0.845777in}{1.993546in}}%
\pgfpathlineto{\pgfqpoint{0.863207in}{2.010306in}}%
\pgfpathlineto{\pgfqpoint{0.882638in}{2.025662in}}%
\pgfpathlineto{\pgfqpoint{0.904086in}{2.039614in}}%
\pgfpathlineto{\pgfqpoint{0.927603in}{2.052161in}}%
\pgfpathlineto{\pgfqpoint{0.953271in}{2.063303in}}%
\pgfpathlineto{\pgfqpoint{0.981206in}{2.073038in}}%
\pgfpathlineto{\pgfqpoint{1.011556in}{2.081367in}}%
\pgfpathlineto{\pgfqpoint{1.044504in}{2.088289in}}%
\pgfpathlineto{\pgfqpoint{1.080250in}{2.093801in}}%
\pgfpathlineto{\pgfqpoint{1.118769in}{2.097878in}}%
\pgfpathlineto{\pgfqpoint{1.160261in}{2.100490in}}%
\pgfpathlineto{\pgfqpoint{1.205041in}{2.101604in}}%
\pgfpathlineto{\pgfqpoint{1.253415in}{2.101181in}}%
\pgfpathlineto{\pgfqpoint{1.305683in}{2.099179in}}%
\pgfpathlineto{\pgfqpoint{1.392023in}{2.093102in}}%
\pgfpathlineto{\pgfqpoint{1.488731in}{2.083162in}}%
\pgfpathlineto{\pgfqpoint{1.596721in}{2.069137in}}%
\pgfpathlineto{\pgfqpoint{1.716866in}{2.050771in}}%
\pgfpathlineto{\pgfqpoint{1.850002in}{2.027783in}}%
\pgfpathlineto{\pgfqpoint{1.996856in}{1.999904in}}%
\pgfpathlineto{\pgfqpoint{2.155954in}{1.966868in}}%
\pgfpathlineto{\pgfqpoint{2.267026in}{1.941870in}}%
\pgfpathlineto{\pgfqpoint{2.380561in}{1.914496in}}%
\pgfpathlineto{\pgfqpoint{2.495018in}{1.884811in}}%
\pgfpathlineto{\pgfqpoint{2.608667in}{1.852936in}}%
\pgfpathlineto{\pgfqpoint{2.719583in}{1.819048in}}%
\pgfpathlineto{\pgfqpoint{2.825651in}{1.783379in}}%
\pgfpathlineto{\pgfqpoint{2.924560in}{1.746218in}}%
\pgfpathlineto{\pgfqpoint{2.970578in}{1.727182in}}%
\pgfpathlineto{\pgfqpoint{3.014180in}{1.707902in}}%
\pgfpathlineto{\pgfqpoint{3.055419in}{1.688419in}}%
\pgfpathlineto{\pgfqpoint{3.094346in}{1.668775in}}%
\pgfpathlineto{\pgfqpoint{3.131013in}{1.649009in}}%
\pgfpathlineto{\pgfqpoint{3.165474in}{1.629158in}}%
\pgfpathlineto{\pgfqpoint{3.197781in}{1.609259in}}%
\pgfpathlineto{\pgfqpoint{3.256148in}{1.569455in}}%
\pgfpathlineto{\pgfqpoint{3.306549in}{1.529858in}}%
\pgfpathlineto{\pgfqpoint{3.349423in}{1.490708in}}%
\pgfpathlineto{\pgfqpoint{3.385217in}{1.452218in}}%
\pgfpathlineto{\pgfqpoint{3.414384in}{1.414582in}}%
\pgfpathlineto{\pgfqpoint{3.437385in}{1.377968in}}%
\pgfpathlineto{\pgfqpoint{3.454827in}{1.342538in}}%
\pgfpathlineto{\pgfqpoint{3.467815in}{1.308377in}}%
\pgfpathlineto{\pgfqpoint{3.477362in}{1.275509in}}%
\pgfpathlineto{\pgfqpoint{3.484227in}{1.243955in}}%
\pgfpathlineto{\pgfqpoint{3.488911in}{1.213733in}}%
\pgfpathlineto{\pgfqpoint{3.491662in}{1.184860in}}%
\pgfpathlineto{\pgfqpoint{3.492469in}{1.157351in}}%
\pgfpathlineto{\pgfqpoint{3.491069in}{1.131216in}}%
\pgfpathlineto{\pgfqpoint{3.486940in}{1.106467in}}%
\pgfpathlineto{\pgfqpoint{3.479308in}{1.083109in}}%
\pgfpathlineto{\pgfqpoint{3.468643in}{1.061170in}}%
\pgfpathlineto{\pgfqpoint{3.455946in}{1.040668in}}%
\pgfpathlineto{\pgfqpoint{3.441282in}{1.021600in}}%
\pgfpathlineto{\pgfqpoint{3.424680in}{1.003959in}}%
\pgfpathlineto{\pgfqpoint{3.406139in}{0.987742in}}%
\pgfpathlineto{\pgfqpoint{3.385620in}{0.972944in}}%
\pgfpathlineto{\pgfqpoint{3.363053in}{0.959561in}}%
\pgfpathlineto{\pgfqpoint{3.338334in}{0.947586in}}%
\pgfpathlineto{\pgfqpoint{3.311325in}{0.937016in}}%
\pgfpathlineto{\pgfqpoint{3.281868in}{0.927846in}}%
\pgfpathlineto{\pgfqpoint{3.249958in}{0.920089in}}%
\pgfpathlineto{\pgfqpoint{3.215483in}{0.913760in}}%
\pgfpathlineto{\pgfqpoint{3.178258in}{0.908874in}}%
\pgfpathlineto{\pgfqpoint{3.138083in}{0.905455in}}%
\pgfpathlineto{\pgfqpoint{3.094735in}{0.903529in}}%
\pgfpathlineto{\pgfqpoint{3.047977in}{0.903130in}}%
\pgfpathlineto{\pgfqpoint{2.997553in}{0.904299in}}%
\pgfpathlineto{\pgfqpoint{2.914438in}{0.909087in}}%
\pgfpathlineto{\pgfqpoint{2.821453in}{0.917678in}}%
\pgfpathlineto{\pgfqpoint{2.717508in}{0.930274in}}%
\pgfpathlineto{\pgfqpoint{2.601451in}{0.947112in}}%
\pgfpathlineto{\pgfqpoint{2.472373in}{0.968492in}}%
\pgfpathlineto{\pgfqpoint{2.329960in}{0.994713in}}%
\pgfpathlineto{\pgfqpoint{2.174923in}{1.026034in}}%
\pgfpathlineto{\pgfqpoint{2.065537in}{1.049846in}}%
\pgfpathlineto{\pgfqpoint{1.952626in}{1.076028in}}%
\pgfpathlineto{\pgfqpoint{1.837706in}{1.104551in}}%
\pgfpathlineto{\pgfqpoint{1.722631in}{1.135333in}}%
\pgfpathlineto{\pgfqpoint{1.609632in}{1.168275in}}%
\pgfpathlineto{\pgfqpoint{1.501003in}{1.203211in}}%
\pgfpathlineto{\pgfqpoint{1.398503in}{1.239785in}}%
\pgfpathlineto{\pgfqpoint{1.303514in}{1.277651in}}%
\pgfpathlineto{\pgfqpoint{1.217054in}{1.316477in}}%
\pgfpathlineto{\pgfqpoint{1.177239in}{1.336151in}}%
\pgfpathlineto{\pgfqpoint{1.139773in}{1.355948in}}%
\pgfpathlineto{\pgfqpoint{1.104679in}{1.375832in}}%
\pgfpathlineto{\pgfqpoint{1.071958in}{1.395766in}}%
\pgfpathlineto{\pgfqpoint{1.041590in}{1.415715in}}%
\pgfpathlineto{\pgfqpoint{1.013531in}{1.435647in}}%
\pgfpathlineto{\pgfqpoint{0.987713in}{1.455527in}}%
\pgfpathlineto{\pgfqpoint{0.964047in}{1.475324in}}%
\pgfpathlineto{\pgfqpoint{0.942421in}{1.495007in}}%
\pgfpathlineto{\pgfqpoint{0.922699in}{1.514546in}}%
\pgfpathlineto{\pgfqpoint{0.904843in}{1.533894in}}%
\pgfpathlineto{\pgfqpoint{0.874286in}{1.571897in}}%
\pgfpathlineto{\pgfqpoint{0.849310in}{1.608925in}}%
\pgfpathlineto{\pgfqpoint{0.828699in}{1.644913in}}%
\pgfpathlineto{\pgfqpoint{0.811616in}{1.679794in}}%
\pgfpathlineto{\pgfqpoint{0.797604in}{1.713504in}}%
\pgfpathlineto{\pgfqpoint{0.786581in}{1.745976in}}%
\pgfpathlineto{\pgfqpoint{0.778848in}{1.777145in}}%
\pgfpathlineto{\pgfqpoint{0.775083in}{1.806944in}}%
\pgfpathlineto{\pgfqpoint{0.775566in}{1.835312in}}%
\pgfpathlineto{\pgfqpoint{0.778582in}{1.862238in}}%
\pgfpathlineto{\pgfqpoint{0.783821in}{1.887719in}}%
\pgfpathlineto{\pgfqpoint{0.791145in}{1.911759in}}%
\pgfpathlineto{\pgfqpoint{0.800470in}{1.934358in}}%
\pgfpathlineto{\pgfqpoint{0.811760in}{1.955523in}}%
\pgfpathlineto{\pgfqpoint{0.825031in}{1.975261in}}%
\pgfpathlineto{\pgfqpoint{0.840350in}{1.993581in}}%
\pgfpathlineto{\pgfqpoint{0.857811in}{2.010493in}}%
\pgfpathlineto{\pgfqpoint{0.877369in}{2.026003in}}%
\pgfpathlineto{\pgfqpoint{0.899010in}{2.040110in}}%
\pgfpathlineto{\pgfqpoint{0.922757in}{2.052811in}}%
\pgfpathlineto{\pgfqpoint{0.948663in}{2.064101in}}%
\pgfpathlineto{\pgfqpoint{0.976810in}{2.073976in}}%
\pgfpathlineto{\pgfqpoint{1.007305in}{2.082427in}}%
\pgfpathlineto{\pgfqpoint{1.040286in}{2.089446in}}%
\pgfpathlineto{\pgfqpoint{1.075919in}{2.095021in}}%
\pgfpathlineto{\pgfqpoint{1.114395in}{2.099140in}}%
\pgfpathlineto{\pgfqpoint{1.155936in}{2.101789in}}%
\pgfpathlineto{\pgfqpoint{1.200792in}{2.102952in}}%
\pgfpathlineto{\pgfqpoint{1.248956in}{2.102626in}}%
\pgfpathlineto{\pgfqpoint{1.327843in}{2.099242in}}%
\pgfpathlineto{\pgfqpoint{1.416602in}{2.092126in}}%
\pgfpathlineto{\pgfqpoint{1.516839in}{2.081001in}}%
\pgfpathlineto{\pgfqpoint{1.629562in}{2.065610in}}%
\pgfpathlineto{\pgfqpoint{1.755177in}{2.045711in}}%
\pgfpathlineto{\pgfqpoint{1.893493in}{2.021084in}}%
\pgfpathlineto{\pgfqpoint{2.043719in}{1.991523in}}%
\pgfpathlineto{\pgfqpoint{2.204465in}{1.956841in}}%
\pgfpathlineto{\pgfqpoint{2.316516in}{1.930792in}}%
\pgfpathlineto{\pgfqpoint{2.431344in}{1.902354in}}%
\pgfpathlineto{\pgfqpoint{2.546486in}{1.871617in}}%
\pgfpathlineto{\pgfqpoint{2.659482in}{1.838775in}}%
\pgfpathlineto{\pgfqpoint{2.768213in}{1.804050in}}%
\pgfpathlineto{\pgfqpoint{2.870889in}{1.767691in}}%
\pgfpathlineto{\pgfqpoint{2.966052in}{1.729974in}}%
\pgfpathlineto{\pgfqpoint{3.010450in}{1.710700in}}%
\pgfpathlineto{\pgfqpoint{3.052570in}{1.691203in}}%
\pgfpathlineto{\pgfqpoint{3.092324in}{1.671524in}}%
\pgfpathlineto{\pgfqpoint{3.129642in}{1.651707in}}%
\pgfpathlineto{\pgfqpoint{3.164474in}{1.631798in}}%
\pgfpathlineto{\pgfqpoint{3.196815in}{1.611850in}}%
\pgfpathlineto{\pgfqpoint{3.226791in}{1.591902in}}%
\pgfpathlineto{\pgfqpoint{3.254552in}{1.571986in}}%
\pgfpathlineto{\pgfqpoint{3.280236in}{1.552130in}}%
\pgfpathlineto{\pgfqpoint{3.325863in}{1.512711in}}%
\pgfpathlineto{\pgfqpoint{3.364528in}{1.473859in}}%
\pgfpathlineto{\pgfqpoint{3.396890in}{1.435764in}}%
\pgfpathlineto{\pgfqpoint{3.423409in}{1.398597in}}%
\pgfpathlineto{\pgfqpoint{3.444351in}{1.362508in}}%
\pgfpathlineto{\pgfqpoint{3.460398in}{1.327615in}}%
\pgfpathlineto{\pgfqpoint{3.472519in}{1.293981in}}%
\pgfpathlineto{\pgfqpoint{3.481247in}{1.261654in}}%
\pgfpathlineto{\pgfqpoint{3.486979in}{1.230670in}}%
\pgfpathlineto{\pgfqpoint{3.489979in}{1.201062in}}%
\pgfpathlineto{\pgfqpoint{3.490374in}{1.172849in}}%
\pgfpathlineto{\pgfqpoint{3.488159in}{1.146046in}}%
\pgfpathlineto{\pgfqpoint{3.483405in}{1.120659in}}%
\pgfpathlineto{\pgfqpoint{3.476387in}{1.096693in}}%
\pgfpathlineto{\pgfqpoint{3.467275in}{1.074150in}}%
\pgfpathlineto{\pgfqpoint{3.456184in}{1.053027in}}%
\pgfpathlineto{\pgfqpoint{3.443183in}{1.033322in}}%
\pgfpathlineto{\pgfqpoint{3.428288in}{1.015032in}}%
\pgfpathlineto{\pgfqpoint{3.411462in}{0.998149in}}%
\pgfpathlineto{\pgfqpoint{3.392621in}{0.982665in}}%
\pgfpathlineto{\pgfqpoint{3.371628in}{0.968571in}}%
\pgfpathlineto{\pgfqpoint{3.348297in}{0.955856in}}%
\pgfpathlineto{\pgfqpoint{3.322636in}{0.944524in}}%
\pgfpathlineto{\pgfqpoint{3.294688in}{0.934590in}}%
\pgfpathlineto{\pgfqpoint{3.264353in}{0.926061in}}%
\pgfpathlineto{\pgfqpoint{3.231508in}{0.918951in}}%
\pgfpathlineto{\pgfqpoint{3.196005in}{0.913274in}}%
\pgfpathlineto{\pgfqpoint{3.157673in}{0.909052in}}%
\pgfpathlineto{\pgfqpoint{3.116314in}{0.906309in}}%
\pgfpathlineto{\pgfqpoint{3.071710in}{0.905073in}}%
\pgfpathlineto{\pgfqpoint{3.023616in}{0.905375in}}%
\pgfpathlineto{\pgfqpoint{2.971765in}{0.907252in}}%
\pgfpathlineto{\pgfqpoint{2.886295in}{0.913108in}}%
\pgfpathlineto{\pgfqpoint{2.790669in}{0.922770in}}%
\pgfpathlineto{\pgfqpoint{2.683688in}{0.936496in}}%
\pgfpathlineto{\pgfqpoint{2.564256in}{0.954555in}}%
\pgfpathlineto{\pgfqpoint{2.431765in}{0.977227in}}%
\pgfpathlineto{\pgfqpoint{2.286108in}{1.004797in}}%
\pgfpathlineto{\pgfqpoint{2.128135in}{1.037514in}}%
\pgfpathlineto{\pgfqpoint{2.017399in}{1.062274in}}%
\pgfpathlineto{\pgfqpoint{1.903809in}{1.089396in}}%
\pgfpathlineto{\pgfqpoint{1.788962in}{1.118822in}}%
\pgfpathlineto{\pgfqpoint{1.674743in}{1.150446in}}%
\pgfpathlineto{\pgfqpoint{1.563332in}{1.184106in}}%
\pgfpathlineto{\pgfqpoint{1.457159in}{1.219622in}}%
\pgfpathlineto{\pgfqpoint{1.358037in}{1.256711in}}%
\pgfpathlineto{\pgfqpoint{1.267049in}{1.294990in}}%
\pgfpathlineto{\pgfqpoint{1.184939in}{1.334098in}}%
\pgfpathlineto{\pgfqpoint{1.147355in}{1.353858in}}%
\pgfpathlineto{\pgfqpoint{1.112117in}{1.373703in}}%
\pgfpathlineto{\pgfqpoint{1.079225in}{1.393596in}}%
\pgfpathlineto{\pgfqpoint{1.048657in}{1.413501in}}%
\pgfpathlineto{\pgfqpoint{1.020370in}{1.433384in}}%
\pgfpathlineto{\pgfqpoint{0.994299in}{1.453213in}}%
\pgfpathlineto{\pgfqpoint{0.970359in}{1.472957in}}%
\pgfpathlineto{\pgfqpoint{0.948446in}{1.492587in}}%
\pgfpathlineto{\pgfqpoint{0.910167in}{1.531402in}}%
\pgfpathlineto{\pgfqpoint{0.878388in}{1.569445in}}%
\pgfpathlineto{\pgfqpoint{0.852834in}{1.606487in}}%
\pgfpathlineto{\pgfqpoint{0.832305in}{1.642452in}}%
\pgfpathlineto{\pgfqpoint{0.815701in}{1.677286in}}%
\pgfpathlineto{\pgfqpoint{0.802255in}{1.710938in}}%
\pgfpathlineto{\pgfqpoint{0.791541in}{1.743359in}}%
\pgfpathlineto{\pgfqpoint{0.783468in}{1.774500in}}%
\pgfpathlineto{\pgfqpoint{0.778284in}{1.804316in}}%
\pgfpathlineto{\pgfqpoint{0.776573in}{1.832761in}}%
\pgfpathlineto{\pgfqpoint{0.778932in}{1.859794in}}%
\pgfpathlineto{\pgfqpoint{0.783928in}{1.885387in}}%
\pgfpathlineto{\pgfqpoint{0.791085in}{1.909539in}}%
\pgfpathlineto{\pgfqpoint{0.800281in}{1.932255in}}%
\pgfpathlineto{\pgfqpoint{0.811441in}{1.953536in}}%
\pgfpathlineto{\pgfqpoint{0.824535in}{1.973389in}}%
\pgfpathlineto{\pgfqpoint{0.839578in}{1.991821in}}%
\pgfpathlineto{\pgfqpoint{0.856632in}{2.008839in}}%
\pgfpathlineto{\pgfqpoint{0.875803in}{2.024454in}}%
\pgfpathlineto{\pgfqpoint{0.897174in}{2.038674in}}%
\pgfpathlineto{\pgfqpoint{0.920693in}{2.051497in}}%
\pgfpathlineto{\pgfqpoint{0.946403in}{2.062919in}}%
\pgfpathlineto{\pgfqpoint{0.974375in}{2.072933in}}%
\pgfpathlineto{\pgfqpoint{1.004708in}{2.081531in}}%
\pgfpathlineto{\pgfqpoint{1.037525in}{2.088700in}}%
\pgfpathlineto{\pgfqpoint{1.072977in}{2.094426in}}%
\pgfpathlineto{\pgfqpoint{1.111240in}{2.098692in}}%
\pgfpathlineto{\pgfqpoint{1.152515in}{2.101476in}}%
\pgfpathlineto{\pgfqpoint{1.197031in}{2.102758in}}%
\pgfpathlineto{\pgfqpoint{1.245040in}{2.102509in}}%
\pgfpathlineto{\pgfqpoint{1.324217in}{2.099204in}}%
\pgfpathlineto{\pgfqpoint{1.412853in}{2.092210in}}%
\pgfpathlineto{\pgfqpoint{1.512045in}{2.081307in}}%
\pgfpathlineto{\pgfqpoint{1.622901in}{2.066257in}}%
\pgfpathlineto{\pgfqpoint{1.622901in}{2.066257in}}%
\pgfusepath{stroke}%
\end{pgfscope}%
\begin{pgfscope}%
\pgfpathrectangle{\pgfqpoint{0.562500in}{0.275000in}}{\pgfqpoint{3.487500in}{1.925000in}}%
\pgfusepath{clip}%
\pgfsetrectcap%
\pgfsetroundjoin%
\pgfsetlinewidth{1.505625pt}%
\definecolor{currentstroke}{rgb}{0.172549,0.627451,0.172549}%
\pgfsetstrokecolor{currentstroke}%
\pgfsetdash{}{0pt}%
\pgfpathmoveto{\pgfqpoint{3.891477in}{0.362500in}}%
\pgfpathlineto{\pgfqpoint{3.762625in}{0.364564in}}%
\pgfpathlineto{\pgfqpoint{3.633934in}{0.370600in}}%
\pgfpathlineto{\pgfqpoint{3.502717in}{0.380471in}}%
\pgfpathlineto{\pgfqpoint{3.366777in}{0.394103in}}%
\pgfpathlineto{\pgfqpoint{3.224358in}{0.411478in}}%
\pgfpathlineto{\pgfqpoint{3.074153in}{0.432629in}}%
\pgfpathlineto{\pgfqpoint{2.915298in}{0.457642in}}%
\pgfpathlineto{\pgfqpoint{2.747135in}{0.486614in}}%
\pgfpathlineto{\pgfqpoint{2.570581in}{0.519624in}}%
\pgfpathlineto{\pgfqpoint{2.388609in}{0.556741in}}%
\pgfpathlineto{\pgfqpoint{2.204749in}{0.597895in}}%
\pgfpathlineto{\pgfqpoint{2.113358in}{0.619926in}}%
\pgfpathlineto{\pgfqpoint{2.023057in}{0.642877in}}%
\pgfpathlineto{\pgfqpoint{1.934436in}{0.666701in}}%
\pgfpathlineto{\pgfqpoint{1.848117in}{0.691340in}}%
\pgfpathlineto{\pgfqpoint{1.764756in}{0.716731in}}%
\pgfpathlineto{\pgfqpoint{1.685041in}{0.742797in}}%
\pgfpathlineto{\pgfqpoint{1.609687in}{0.769458in}}%
\pgfpathlineto{\pgfqpoint{1.539051in}{0.796625in}}%
\pgfpathlineto{\pgfqpoint{1.473138in}{0.824213in}}%
\pgfpathlineto{\pgfqpoint{1.411963in}{0.852137in}}%
\pgfpathlineto{\pgfqpoint{1.355487in}{0.880317in}}%
\pgfpathlineto{\pgfqpoint{1.303620in}{0.908677in}}%
\pgfpathlineto{\pgfqpoint{1.256223in}{0.937139in}}%
\pgfpathlineto{\pgfqpoint{1.213105in}{0.965632in}}%
\pgfpathlineto{\pgfqpoint{1.174023in}{0.994085in}}%
\pgfpathlineto{\pgfqpoint{1.138686in}{1.022431in}}%
\pgfpathlineto{\pgfqpoint{1.106754in}{1.050605in}}%
\pgfpathlineto{\pgfqpoint{1.077904in}{1.078551in}}%
\pgfpathlineto{\pgfqpoint{1.051809in}{1.106241in}}%
\pgfpathlineto{\pgfqpoint{1.028166in}{1.133650in}}%
\pgfpathlineto{\pgfqpoint{1.006710in}{1.160754in}}%
\pgfpathlineto{\pgfqpoint{0.987218in}{1.187528in}}%
\pgfpathlineto{\pgfqpoint{0.969506in}{1.213951in}}%
\pgfpathlineto{\pgfqpoint{0.938881in}{1.265648in}}%
\pgfpathlineto{\pgfqpoint{0.914115in}{1.315670in}}%
\pgfpathlineto{\pgfqpoint{0.894077in}{1.363924in}}%
\pgfpathlineto{\pgfqpoint{0.877797in}{1.410400in}}%
\pgfpathlineto{\pgfqpoint{0.864708in}{1.455087in}}%
\pgfpathlineto{\pgfqpoint{0.854423in}{1.497974in}}%
\pgfpathlineto{\pgfqpoint{0.846734in}{1.539052in}}%
\pgfpathlineto{\pgfqpoint{0.841568in}{1.578315in}}%
\pgfpathlineto{\pgfqpoint{0.838655in}{1.615780in}}%
\pgfpathlineto{\pgfqpoint{0.837782in}{1.651476in}}%
\pgfpathlineto{\pgfqpoint{0.838814in}{1.685432in}}%
\pgfpathlineto{\pgfqpoint{0.841670in}{1.717672in}}%
\pgfpathlineto{\pgfqpoint{0.846319in}{1.748220in}}%
\pgfpathlineto{\pgfqpoint{0.852785in}{1.777100in}}%
\pgfpathlineto{\pgfqpoint{0.861140in}{1.804331in}}%
\pgfpathlineto{\pgfqpoint{0.871417in}{1.829941in}}%
\pgfpathlineto{\pgfqpoint{0.883514in}{1.853951in}}%
\pgfpathlineto{\pgfqpoint{0.897350in}{1.876381in}}%
\pgfpathlineto{\pgfqpoint{0.912880in}{1.897249in}}%
\pgfpathlineto{\pgfqpoint{0.930097in}{1.916573in}}%
\pgfpathlineto{\pgfqpoint{0.949035in}{1.934371in}}%
\pgfpathlineto{\pgfqpoint{0.969762in}{1.950660in}}%
\pgfpathlineto{\pgfqpoint{0.992386in}{1.965457in}}%
\pgfpathlineto{\pgfqpoint{1.017052in}{1.978778in}}%
\pgfpathlineto{\pgfqpoint{1.043944in}{1.990638in}}%
\pgfpathlineto{\pgfqpoint{1.073167in}{2.001048in}}%
\pgfpathlineto{\pgfqpoint{1.104616in}{2.009994in}}%
\pgfpathlineto{\pgfqpoint{1.138404in}{2.017472in}}%
\pgfpathlineto{\pgfqpoint{1.174662in}{2.023473in}}%
\pgfpathlineto{\pgfqpoint{1.213530in}{2.027986in}}%
\pgfpathlineto{\pgfqpoint{1.255155in}{2.030997in}}%
\pgfpathlineto{\pgfqpoint{1.299690in}{2.032487in}}%
\pgfpathlineto{\pgfqpoint{1.347298in}{2.032437in}}%
\pgfpathlineto{\pgfqpoint{1.398146in}{2.030823in}}%
\pgfpathlineto{\pgfqpoint{1.480877in}{2.025408in}}%
\pgfpathlineto{\pgfqpoint{1.571917in}{2.016308in}}%
\pgfpathlineto{\pgfqpoint{1.671803in}{2.003403in}}%
\pgfpathlineto{\pgfqpoint{1.780964in}{1.986516in}}%
\pgfpathlineto{\pgfqpoint{1.899552in}{1.965470in}}%
\pgfpathlineto{\pgfqpoint{2.026927in}{1.940152in}}%
\pgfpathlineto{\pgfqpoint{2.161646in}{1.910514in}}%
\pgfpathlineto{\pgfqpoint{2.301466in}{1.876572in}}%
\pgfpathlineto{\pgfqpoint{2.396067in}{1.851585in}}%
\pgfpathlineto{\pgfqpoint{2.490393in}{1.824789in}}%
\pgfpathlineto{\pgfqpoint{2.582943in}{1.796341in}}%
\pgfpathlineto{\pgfqpoint{2.672382in}{1.766417in}}%
\pgfpathlineto{\pgfqpoint{2.757558in}{1.735205in}}%
\pgfpathlineto{\pgfqpoint{2.837504in}{1.702910in}}%
\pgfpathlineto{\pgfqpoint{2.911435in}{1.669749in}}%
\pgfpathlineto{\pgfqpoint{2.978749in}{1.635953in}}%
\pgfpathlineto{\pgfqpoint{3.039031in}{1.601767in}}%
\pgfpathlineto{\pgfqpoint{3.091986in}{1.567456in}}%
\pgfpathlineto{\pgfqpoint{3.137764in}{1.533271in}}%
\pgfpathlineto{\pgfqpoint{3.177203in}{1.499398in}}%
\pgfpathlineto{\pgfqpoint{3.211032in}{1.466001in}}%
\pgfpathlineto{\pgfqpoint{3.239832in}{1.433231in}}%
\pgfpathlineto{\pgfqpoint{3.264030in}{1.401218in}}%
\pgfpathlineto{\pgfqpoint{3.283903in}{1.370078in}}%
\pgfpathlineto{\pgfqpoint{3.299573in}{1.339908in}}%
\pgfpathlineto{\pgfqpoint{3.311014in}{1.310792in}}%
\pgfpathlineto{\pgfqpoint{3.318149in}{1.282795in}}%
\pgfpathlineto{\pgfqpoint{3.321822in}{1.255991in}}%
\pgfpathlineto{\pgfqpoint{3.322554in}{1.230422in}}%
\pgfpathlineto{\pgfqpoint{3.320629in}{1.206117in}}%
\pgfpathlineto{\pgfqpoint{3.316260in}{1.183099in}}%
\pgfpathlineto{\pgfqpoint{3.309593in}{1.161387in}}%
\pgfpathlineto{\pgfqpoint{3.300702in}{1.140992in}}%
\pgfpathlineto{\pgfqpoint{3.289592in}{1.121919in}}%
\pgfpathlineto{\pgfqpoint{3.276199in}{1.104171in}}%
\pgfpathlineto{\pgfqpoint{3.260397in}{1.087740in}}%
\pgfpathlineto{\pgfqpoint{3.242299in}{1.072637in}}%
\pgfpathlineto{\pgfqpoint{3.222019in}{1.058874in}}%
\pgfpathlineto{\pgfqpoint{3.199550in}{1.046459in}}%
\pgfpathlineto{\pgfqpoint{3.174868in}{1.035399in}}%
\pgfpathlineto{\pgfqpoint{3.147930in}{1.025703in}}%
\pgfpathlineto{\pgfqpoint{3.118673in}{1.017380in}}%
\pgfpathlineto{\pgfqpoint{3.087014in}{1.010438in}}%
\pgfpathlineto{\pgfqpoint{3.052851in}{1.004889in}}%
\pgfpathlineto{\pgfqpoint{3.016064in}{1.000741in}}%
\pgfpathlineto{\pgfqpoint{2.976513in}{0.998005in}}%
\pgfpathlineto{\pgfqpoint{2.934037in}{0.996693in}}%
\pgfpathlineto{\pgfqpoint{2.864913in}{0.997408in}}%
\pgfpathlineto{\pgfqpoint{2.788554in}{1.001481in}}%
\pgfpathlineto{\pgfqpoint{2.703928in}{1.009088in}}%
\pgfpathlineto{\pgfqpoint{2.610420in}{1.020386in}}%
\pgfpathlineto{\pgfqpoint{2.507830in}{1.035507in}}%
\pgfpathlineto{\pgfqpoint{2.396377in}{1.054558in}}%
\pgfpathlineto{\pgfqpoint{2.276693in}{1.077627in}}%
\pgfpathlineto{\pgfqpoint{2.149831in}{1.104774in}}%
\pgfpathlineto{\pgfqpoint{2.017257in}{1.136040in}}%
\pgfpathlineto{\pgfqpoint{1.926622in}{1.159182in}}%
\pgfpathlineto{\pgfqpoint{1.835319in}{1.184157in}}%
\pgfpathlineto{\pgfqpoint{1.744782in}{1.210830in}}%
\pgfpathlineto{\pgfqpoint{1.656304in}{1.239026in}}%
\pgfpathlineto{\pgfqpoint{1.571026in}{1.268562in}}%
\pgfpathlineto{\pgfqpoint{1.489934in}{1.299251in}}%
\pgfpathlineto{\pgfqpoint{1.413863in}{1.330899in}}%
\pgfpathlineto{\pgfqpoint{1.343494in}{1.363305in}}%
\pgfpathlineto{\pgfqpoint{1.279357in}{1.396262in}}%
\pgfpathlineto{\pgfqpoint{1.221829in}{1.429557in}}%
\pgfpathlineto{\pgfqpoint{1.171131in}{1.462969in}}%
\pgfpathlineto{\pgfqpoint{1.127336in}{1.496274in}}%
\pgfpathlineto{\pgfqpoint{1.090455in}{1.529217in}}%
\pgfpathlineto{\pgfqpoint{1.059798in}{1.561605in}}%
\pgfpathlineto{\pgfqpoint{1.034253in}{1.593320in}}%
\pgfpathlineto{\pgfqpoint{1.012931in}{1.624259in}}%
\pgfpathlineto{\pgfqpoint{0.995173in}{1.654332in}}%
\pgfpathlineto{\pgfqpoint{0.980548in}{1.683455in}}%
\pgfpathlineto{\pgfqpoint{0.968853in}{1.711560in}}%
\pgfpathlineto{\pgfqpoint{0.960111in}{1.738587in}}%
\pgfpathlineto{\pgfqpoint{0.954578in}{1.764489in}}%
\pgfpathlineto{\pgfqpoint{0.952732in}{1.789229in}}%
\pgfpathlineto{\pgfqpoint{0.953405in}{1.801155in}}%
\pgfpathlineto{\pgfqpoint{0.957724in}{1.824085in}}%
\pgfpathlineto{\pgfqpoint{0.964614in}{1.845723in}}%
\pgfpathlineto{\pgfqpoint{0.973851in}{1.866052in}}%
\pgfpathlineto{\pgfqpoint{0.985323in}{1.885060in}}%
\pgfpathlineto{\pgfqpoint{0.998952in}{1.902738in}}%
\pgfpathlineto{\pgfqpoint{1.014700in}{1.919080in}}%
\pgfpathlineto{\pgfqpoint{1.032568in}{1.934082in}}%
\pgfpathlineto{\pgfqpoint{1.052592in}{1.947742in}}%
\pgfpathlineto{\pgfqpoint{1.074848in}{1.960063in}}%
\pgfpathlineto{\pgfqpoint{1.099447in}{1.971050in}}%
\pgfpathlineto{\pgfqpoint{1.126371in}{1.980694in}}%
\pgfpathlineto{\pgfqpoint{1.155620in}{1.988982in}}%
\pgfpathlineto{\pgfqpoint{1.187296in}{1.995901in}}%
\pgfpathlineto{\pgfqpoint{1.221505in}{2.001436in}}%
\pgfpathlineto{\pgfqpoint{1.258363in}{2.005569in}}%
\pgfpathlineto{\pgfqpoint{1.297994in}{2.008283in}}%
\pgfpathlineto{\pgfqpoint{1.340526in}{2.009554in}}%
\pgfpathlineto{\pgfqpoint{1.386100in}{2.009360in}}%
\pgfpathlineto{\pgfqpoint{1.460481in}{2.006263in}}%
\pgfpathlineto{\pgfqpoint{1.542558in}{1.999711in}}%
\pgfpathlineto{\pgfqpoint{1.632877in}{1.989597in}}%
\pgfpathlineto{\pgfqpoint{1.731819in}{1.975787in}}%
\pgfpathlineto{\pgfqpoint{1.839789in}{1.958093in}}%
\pgfpathlineto{\pgfqpoint{1.956642in}{1.936370in}}%
\pgfpathlineto{\pgfqpoint{2.081516in}{1.910533in}}%
\pgfpathlineto{\pgfqpoint{2.212835in}{1.880555in}}%
\pgfpathlineto{\pgfqpoint{2.348309in}{1.846471in}}%
\pgfpathlineto{\pgfqpoint{2.439503in}{1.821505in}}%
\pgfpathlineto{\pgfqpoint{2.529936in}{1.794859in}}%
\pgfpathlineto{\pgfqpoint{2.618262in}{1.766691in}}%
\pgfpathlineto{\pgfqpoint{2.703302in}{1.737171in}}%
\pgfpathlineto{\pgfqpoint{2.784038in}{1.706482in}}%
\pgfpathlineto{\pgfqpoint{2.859620in}{1.674822in}}%
\pgfpathlineto{\pgfqpoint{2.929362in}{1.642397in}}%
\pgfpathlineto{\pgfqpoint{2.992741in}{1.609429in}}%
\pgfpathlineto{\pgfqpoint{3.049403in}{1.576152in}}%
\pgfpathlineto{\pgfqpoint{3.099001in}{1.542821in}}%
\pgfpathlineto{\pgfqpoint{3.141602in}{1.509679in}}%
\pgfpathlineto{\pgfqpoint{3.178190in}{1.476888in}}%
\pgfpathlineto{\pgfqpoint{3.209605in}{1.444596in}}%
\pgfpathlineto{\pgfqpoint{3.236502in}{1.412935in}}%
\pgfpathlineto{\pgfqpoint{3.259347in}{1.382023in}}%
\pgfpathlineto{\pgfqpoint{3.278421in}{1.351965in}}%
\pgfpathlineto{\pgfqpoint{3.293817in}{1.322851in}}%
\pgfpathlineto{\pgfqpoint{3.305439in}{1.294755in}}%
\pgfpathlineto{\pgfqpoint{3.313007in}{1.267740in}}%
\pgfpathlineto{\pgfqpoint{3.316111in}{1.241855in}}%
\pgfpathlineto{\pgfqpoint{3.315752in}{1.217178in}}%
\pgfpathlineto{\pgfqpoint{3.312689in}{1.193756in}}%
\pgfpathlineto{\pgfqpoint{3.307138in}{1.171611in}}%
\pgfpathlineto{\pgfqpoint{3.299263in}{1.150760in}}%
\pgfpathlineto{\pgfqpoint{3.289177in}{1.131219in}}%
\pgfpathlineto{\pgfqpoint{3.276938in}{1.112996in}}%
\pgfpathlineto{\pgfqpoint{3.262553in}{1.096096in}}%
\pgfpathlineto{\pgfqpoint{3.245976in}{1.080519in}}%
\pgfpathlineto{\pgfqpoint{3.227108in}{1.066261in}}%
\pgfpathlineto{\pgfqpoint{3.205860in}{1.053320in}}%
\pgfpathlineto{\pgfqpoint{3.182373in}{1.041711in}}%
\pgfpathlineto{\pgfqpoint{3.156626in}{1.031446in}}%
\pgfpathlineto{\pgfqpoint{3.128563in}{1.022534in}}%
\pgfpathlineto{\pgfqpoint{3.098110in}{1.014989in}}%
\pgfpathlineto{\pgfqpoint{3.065184in}{1.008822in}}%
\pgfpathlineto{\pgfqpoint{3.029682in}{1.004049in}}%
\pgfpathlineto{\pgfqpoint{2.991493in}{1.000687in}}%
\pgfpathlineto{\pgfqpoint{2.950487in}{0.998751in}}%
\pgfpathlineto{\pgfqpoint{2.906522in}{0.998263in}}%
\pgfpathlineto{\pgfqpoint{2.834683in}{1.000287in}}%
\pgfpathlineto{\pgfqpoint{2.755377in}{1.005680in}}%
\pgfpathlineto{\pgfqpoint{2.668156in}{1.014562in}}%
\pgfpathlineto{\pgfqpoint{2.572133in}{1.027145in}}%
\pgfpathlineto{\pgfqpoint{2.466937in}{1.043597in}}%
\pgfpathlineto{\pgfqpoint{2.352780in}{1.064044in}}%
\pgfpathlineto{\pgfqpoint{2.230455in}{1.088562in}}%
\pgfpathlineto{\pgfqpoint{2.101337in}{1.117183in}}%
\pgfpathlineto{\pgfqpoint{1.967384in}{1.149892in}}%
\pgfpathlineto{\pgfqpoint{1.876625in}{1.173940in}}%
\pgfpathlineto{\pgfqpoint{1.785883in}{1.199756in}}%
\pgfpathlineto{\pgfqpoint{1.696548in}{1.227204in}}%
\pgfpathlineto{\pgfqpoint{1.609885in}{1.256091in}}%
\pgfpathlineto{\pgfqpoint{1.526988in}{1.286223in}}%
\pgfpathlineto{\pgfqpoint{1.448782in}{1.317401in}}%
\pgfpathlineto{\pgfqpoint{1.376024in}{1.349421in}}%
\pgfpathlineto{\pgfqpoint{1.309297in}{1.382076in}}%
\pgfpathlineto{\pgfqpoint{1.249016in}{1.415156in}}%
\pgfpathlineto{\pgfqpoint{1.195427in}{1.448446in}}%
\pgfpathlineto{\pgfqpoint{1.148604in}{1.481727in}}%
\pgfpathlineto{\pgfqpoint{1.108496in}{1.514773in}}%
\pgfpathlineto{\pgfqpoint{1.074924in}{1.547350in}}%
\pgfpathlineto{\pgfqpoint{1.046909in}{1.579318in}}%
\pgfpathlineto{\pgfqpoint{1.023536in}{1.610563in}}%
\pgfpathlineto{\pgfqpoint{1.004098in}{1.640984in}}%
\pgfpathlineto{\pgfqpoint{0.988098in}{1.670489in}}%
\pgfpathlineto{\pgfqpoint{0.975244in}{1.699004in}}%
\pgfpathlineto{\pgfqpoint{0.965456in}{1.726461in}}%
\pgfpathlineto{\pgfqpoint{0.958859in}{1.752810in}}%
\pgfpathlineto{\pgfqpoint{0.955789in}{1.778009in}}%
\pgfpathlineto{\pgfqpoint{0.956776in}{1.802031in}}%
\pgfpathlineto{\pgfqpoint{0.961133in}{1.824813in}}%
\pgfpathlineto{\pgfqpoint{0.967982in}{1.846310in}}%
\pgfpathlineto{\pgfqpoint{0.977172in}{1.866505in}}%
\pgfpathlineto{\pgfqpoint{0.988591in}{1.885386in}}%
\pgfpathlineto{\pgfqpoint{1.002167in}{1.902944in}}%
\pgfpathlineto{\pgfqpoint{1.017866in}{1.919172in}}%
\pgfpathlineto{\pgfqpoint{1.035697in}{1.934066in}}%
\pgfpathlineto{\pgfqpoint{1.055706in}{1.947626in}}%
\pgfpathlineto{\pgfqpoint{1.077980in}{1.959855in}}%
\pgfpathlineto{\pgfqpoint{1.102622in}{1.970756in}}%
\pgfpathlineto{\pgfqpoint{1.129559in}{1.980316in}}%
\pgfpathlineto{\pgfqpoint{1.158837in}{1.988522in}}%
\pgfpathlineto{\pgfqpoint{1.190551in}{1.995360in}}%
\pgfpathlineto{\pgfqpoint{1.224805in}{2.000817in}}%
\pgfpathlineto{\pgfqpoint{1.261711in}{2.004874in}}%
\pgfpathlineto{\pgfqpoint{1.301390in}{2.007513in}}%
\pgfpathlineto{\pgfqpoint{1.343973in}{2.008711in}}%
\pgfpathlineto{\pgfqpoint{1.389597in}{2.008446in}}%
\pgfpathlineto{\pgfqpoint{1.464060in}{2.005247in}}%
\pgfpathlineto{\pgfqpoint{1.546232in}{1.998599in}}%
\pgfpathlineto{\pgfqpoint{1.636638in}{1.988399in}}%
\pgfpathlineto{\pgfqpoint{1.735681in}{1.974495in}}%
\pgfpathlineto{\pgfqpoint{1.843775in}{1.956704in}}%
\pgfpathlineto{\pgfqpoint{1.960712in}{1.934889in}}%
\pgfpathlineto{\pgfqpoint{2.085595in}{1.908970in}}%
\pgfpathlineto{\pgfqpoint{2.216845in}{1.878919in}}%
\pgfpathlineto{\pgfqpoint{2.352199in}{1.844765in}}%
\pgfpathlineto{\pgfqpoint{2.443281in}{1.819753in}}%
\pgfpathlineto{\pgfqpoint{2.533530in}{1.793072in}}%
\pgfpathlineto{\pgfqpoint{2.621624in}{1.764881in}}%
\pgfpathlineto{\pgfqpoint{2.706400in}{1.735349in}}%
\pgfpathlineto{\pgfqpoint{2.786857in}{1.704658in}}%
\pgfpathlineto{\pgfqpoint{2.862156in}{1.673005in}}%
\pgfpathlineto{\pgfqpoint{2.931617in}{1.640595in}}%
\pgfpathlineto{\pgfqpoint{2.994725in}{1.607651in}}%
\pgfpathlineto{\pgfqpoint{3.051123in}{1.574403in}}%
\pgfpathlineto{\pgfqpoint{3.100486in}{1.541105in}}%
\pgfpathlineto{\pgfqpoint{3.142821in}{1.508005in}}%
\pgfpathlineto{\pgfqpoint{3.179118in}{1.475265in}}%
\pgfpathlineto{\pgfqpoint{3.210250in}{1.443029in}}%
\pgfpathlineto{\pgfqpoint{3.236893in}{1.411428in}}%
\pgfpathlineto{\pgfqpoint{3.259533in}{1.380579in}}%
\pgfpathlineto{\pgfqpoint{3.278463in}{1.350585in}}%
\pgfpathlineto{\pgfqpoint{3.293784in}{1.321535in}}%
\pgfpathlineto{\pgfqpoint{3.305405in}{1.293503in}}%
\pgfpathlineto{\pgfqpoint{3.313041in}{1.266550in}}%
\pgfpathlineto{\pgfqpoint{3.315224in}{1.253494in}}%
\pgfpathlineto{\pgfqpoint{3.316347in}{1.228258in}}%
\pgfpathlineto{\pgfqpoint{3.314492in}{1.204260in}}%
\pgfpathlineto{\pgfqpoint{3.310044in}{1.181529in}}%
\pgfpathlineto{\pgfqpoint{3.303194in}{1.160085in}}%
\pgfpathlineto{\pgfqpoint{3.294079in}{1.139944in}}%
\pgfpathlineto{\pgfqpoint{3.282788in}{1.121118in}}%
\pgfpathlineto{\pgfqpoint{3.269356in}{1.103614in}}%
\pgfpathlineto{\pgfqpoint{3.253771in}{1.087435in}}%
\pgfpathlineto{\pgfqpoint{3.235965in}{1.072579in}}%
\pgfpathlineto{\pgfqpoint{3.215823in}{1.059041in}}%
\pgfpathlineto{\pgfqpoint{3.193348in}{1.046825in}}%
\pgfpathlineto{\pgfqpoint{3.168637in}{1.035948in}}%
\pgfpathlineto{\pgfqpoint{3.141638in}{1.026419in}}%
\pgfpathlineto{\pgfqpoint{3.112283in}{1.018250in}}%
\pgfpathlineto{\pgfqpoint{3.080493in}{1.011453in}}%
\pgfpathlineto{\pgfqpoint{3.046174in}{1.006043in}}%
\pgfpathlineto{\pgfqpoint{3.009218in}{1.002034in}}%
\pgfpathlineto{\pgfqpoint{2.969504in}{0.999446in}}%
\pgfpathlineto{\pgfqpoint{2.926899in}{0.998296in}}%
\pgfpathlineto{\pgfqpoint{2.857240in}{0.999313in}}%
\pgfpathlineto{\pgfqpoint{2.780192in}{1.003686in}}%
\pgfpathlineto{\pgfqpoint{2.695431in}{1.011500in}}%
\pgfpathlineto{\pgfqpoint{2.602152in}{1.022945in}}%
\pgfpathlineto{\pgfqpoint{2.499782in}{1.038206in}}%
\pgfpathlineto{\pgfqpoint{2.388331in}{1.057422in}}%
\pgfpathlineto{\pgfqpoint{2.268397in}{1.080690in}}%
\pgfpathlineto{\pgfqpoint{2.141163in}{1.108057in}}%
\pgfpathlineto{\pgfqpoint{2.008396in}{1.139529in}}%
\pgfpathlineto{\pgfqpoint{1.917949in}{1.162772in}}%
\pgfpathlineto{\pgfqpoint{1.826980in}{1.187807in}}%
\pgfpathlineto{\pgfqpoint{1.736810in}{1.214541in}}%
\pgfpathlineto{\pgfqpoint{1.648775in}{1.242793in}}%
\pgfpathlineto{\pgfqpoint{1.564040in}{1.272376in}}%
\pgfpathlineto{\pgfqpoint{1.483605in}{1.303096in}}%
\pgfpathlineto{\pgfqpoint{1.408301in}{1.334755in}}%
\pgfpathlineto{\pgfqpoint{1.338792in}{1.367150in}}%
\pgfpathlineto{\pgfqpoint{1.275574in}{1.400069in}}%
\pgfpathlineto{\pgfqpoint{1.218973in}{1.433299in}}%
\pgfpathlineto{\pgfqpoint{1.169151in}{1.466619in}}%
\pgfpathlineto{\pgfqpoint{1.126117in}{1.499799in}}%
\pgfpathlineto{\pgfqpoint{1.089853in}{1.532592in}}%
\pgfpathlineto{\pgfqpoint{1.059462in}{1.564840in}}%
\pgfpathlineto{\pgfqpoint{1.033972in}{1.596421in}}%
\pgfpathlineto{\pgfqpoint{1.012622in}{1.627227in}}%
\pgfpathlineto{\pgfqpoint{0.994861in}{1.657163in}}%
\pgfpathlineto{\pgfqpoint{0.980350in}{1.686146in}}%
\pgfpathlineto{\pgfqpoint{0.968962in}{1.714103in}}%
\pgfpathlineto{\pgfqpoint{0.960781in}{1.740977in}}%
\pgfpathlineto{\pgfqpoint{0.956101in}{1.766721in}}%
\pgfpathlineto{\pgfqpoint{0.955429in}{1.791301in}}%
\pgfpathlineto{\pgfqpoint{0.958606in}{1.814667in}}%
\pgfpathlineto{\pgfqpoint{0.964367in}{1.836755in}}%
\pgfpathlineto{\pgfqpoint{0.972521in}{1.857548in}}%
\pgfpathlineto{\pgfqpoint{0.982940in}{1.877031in}}%
\pgfpathlineto{\pgfqpoint{0.995536in}{1.895193in}}%
\pgfpathlineto{\pgfqpoint{1.010262in}{1.912028in}}%
\pgfpathlineto{\pgfqpoint{1.027114in}{1.927530in}}%
\pgfpathlineto{\pgfqpoint{1.046128in}{1.941698in}}%
\pgfpathlineto{\pgfqpoint{1.067379in}{1.954535in}}%
\pgfpathlineto{\pgfqpoint{1.090980in}{1.966045in}}%
\pgfpathlineto{\pgfqpoint{1.116888in}{1.976219in}}%
\pgfpathlineto{\pgfqpoint{1.145103in}{1.985044in}}%
\pgfpathlineto{\pgfqpoint{1.175713in}{1.992506in}}%
\pgfpathlineto{\pgfqpoint{1.208814in}{1.998592in}}%
\pgfpathlineto{\pgfqpoint{1.244514in}{2.003285in}}%
\pgfpathlineto{\pgfqpoint{1.282928in}{2.006569in}}%
\pgfpathlineto{\pgfqpoint{1.324181in}{2.008422in}}%
\pgfpathlineto{\pgfqpoint{1.368409in}{2.008821in}}%
\pgfpathlineto{\pgfqpoint{1.440647in}{2.006643in}}%
\pgfpathlineto{\pgfqpoint{1.520434in}{2.001049in}}%
\pgfpathlineto{\pgfqpoint{1.608309in}{1.991940in}}%
\pgfpathlineto{\pgfqpoint{1.704682in}{1.979178in}}%
\pgfpathlineto{\pgfqpoint{1.810069in}{1.962574in}}%
\pgfpathlineto{\pgfqpoint{1.924432in}{1.941981in}}%
\pgfpathlineto{\pgfqpoint{2.047067in}{1.917304in}}%
\pgfpathlineto{\pgfqpoint{2.176601in}{1.888500in}}%
\pgfpathlineto{\pgfqpoint{2.310997in}{1.855578in}}%
\pgfpathlineto{\pgfqpoint{2.402008in}{1.831362in}}%
\pgfpathlineto{\pgfqpoint{2.492825in}{1.805411in}}%
\pgfpathlineto{\pgfqpoint{2.582061in}{1.777877in}}%
\pgfpathlineto{\pgfqpoint{2.668484in}{1.748926in}}%
\pgfpathlineto{\pgfqpoint{2.751025in}{1.718734in}}%
\pgfpathlineto{\pgfqpoint{2.828773in}{1.687491in}}%
\pgfpathlineto{\pgfqpoint{2.900979in}{1.655399in}}%
\pgfpathlineto{\pgfqpoint{2.967057in}{1.622669in}}%
\pgfpathlineto{\pgfqpoint{3.026578in}{1.589528in}}%
\pgfpathlineto{\pgfqpoint{3.079277in}{1.556210in}}%
\pgfpathlineto{\pgfqpoint{3.124809in}{1.522982in}}%
\pgfpathlineto{\pgfqpoint{3.163693in}{1.490051in}}%
\pgfpathlineto{\pgfqpoint{3.196947in}{1.457567in}}%
\pgfpathlineto{\pgfqpoint{3.225391in}{1.425664in}}%
\pgfpathlineto{\pgfqpoint{3.249641in}{1.394463in}}%
\pgfpathlineto{\pgfqpoint{3.270118in}{1.364072in}}%
\pgfpathlineto{\pgfqpoint{3.287042in}{1.334585in}}%
\pgfpathlineto{\pgfqpoint{3.300432in}{1.306083in}}%
\pgfpathlineto{\pgfqpoint{3.310108in}{1.278634in}}%
\pgfpathlineto{\pgfqpoint{3.315692in}{1.252291in}}%
\pgfpathlineto{\pgfqpoint{3.316828in}{1.227102in}}%
\pgfpathlineto{\pgfqpoint{3.314836in}{1.203148in}}%
\pgfpathlineto{\pgfqpoint{3.310249in}{1.180463in}}%
\pgfpathlineto{\pgfqpoint{3.303255in}{1.159068in}}%
\pgfpathlineto{\pgfqpoint{3.293993in}{1.138977in}}%
\pgfpathlineto{\pgfqpoint{3.282553in}{1.120202in}}%
\pgfpathlineto{\pgfqpoint{3.268975in}{1.102750in}}%
\pgfpathlineto{\pgfqpoint{3.253252in}{1.086626in}}%
\pgfpathlineto{\pgfqpoint{3.235325in}{1.071827in}}%
\pgfpathlineto{\pgfqpoint{3.215089in}{1.058350in}}%
\pgfpathlineto{\pgfqpoint{3.192502in}{1.046195in}}%
\pgfpathlineto{\pgfqpoint{3.167675in}{1.035378in}}%
\pgfpathlineto{\pgfqpoint{3.140559in}{1.025910in}}%
\pgfpathlineto{\pgfqpoint{3.111084in}{1.017801in}}%
\pgfpathlineto{\pgfqpoint{3.079167in}{1.011065in}}%
\pgfpathlineto{\pgfqpoint{3.044715in}{1.005717in}}%
\pgfpathlineto{\pgfqpoint{3.007618in}{1.001772in}}%
\pgfpathlineto{\pgfqpoint{2.967757in}{0.999248in}}%
\pgfpathlineto{\pgfqpoint{2.924996in}{0.998164in}}%
\pgfpathlineto{\pgfqpoint{2.855093in}{0.999286in}}%
\pgfpathlineto{\pgfqpoint{2.777782in}{1.003773in}}%
\pgfpathlineto{\pgfqpoint{2.692694in}{1.011712in}}%
\pgfpathlineto{\pgfqpoint{2.599124in}{1.023279in}}%
\pgfpathlineto{\pgfqpoint{2.496483in}{1.038662in}}%
\pgfpathlineto{\pgfqpoint{2.384777in}{1.058000in}}%
\pgfpathlineto{\pgfqpoint{2.264600in}{1.081390in}}%
\pgfpathlineto{\pgfqpoint{2.137137in}{1.108885in}}%
\pgfpathlineto{\pgfqpoint{2.004164in}{1.140491in}}%
\pgfpathlineto{\pgfqpoint{1.913600in}{1.163831in}}%
\pgfpathlineto{\pgfqpoint{1.822614in}{1.188982in}}%
\pgfpathlineto{\pgfqpoint{1.732553in}{1.215819in}}%
\pgfpathlineto{\pgfqpoint{1.644679in}{1.244160in}}%
\pgfpathlineto{\pgfqpoint{1.560103in}{1.273814in}}%
\pgfpathlineto{\pgfqpoint{1.479787in}{1.304591in}}%
\pgfpathlineto{\pgfqpoint{1.404538in}{1.336292in}}%
\pgfpathlineto{\pgfqpoint{1.335015in}{1.368717in}}%
\pgfpathlineto{\pgfqpoint{1.271725in}{1.401660in}}%
\pgfpathlineto{\pgfqpoint{1.215025in}{1.434910in}}%
\pgfpathlineto{\pgfqpoint{1.165118in}{1.468254in}}%
\pgfpathlineto{\pgfqpoint{1.122060in}{1.501471in}}%
\pgfpathlineto{\pgfqpoint{1.085799in}{1.534328in}}%
\pgfpathlineto{\pgfqpoint{1.055818in}{1.566608in}}%
\pgfpathlineto{\pgfqpoint{1.031025in}{1.598192in}}%
\pgfpathlineto{\pgfqpoint{1.010511in}{1.628979in}}%
\pgfpathlineto{\pgfqpoint{0.993587in}{1.658880in}}%
\pgfpathlineto{\pgfqpoint{0.979788in}{1.687816in}}%
\pgfpathlineto{\pgfqpoint{0.968867in}{1.715719in}}%
\pgfpathlineto{\pgfqpoint{0.960800in}{1.742534in}}%
\pgfpathlineto{\pgfqpoint{0.955782in}{1.768215in}}%
\pgfpathlineto{\pgfqpoint{0.954231in}{1.792728in}}%
\pgfpathlineto{\pgfqpoint{0.954946in}{1.804539in}}%
\pgfpathlineto{\pgfqpoint{0.959445in}{1.827249in}}%
\pgfpathlineto{\pgfqpoint{0.966632in}{1.848672in}}%
\pgfpathlineto{\pgfqpoint{0.976166in}{1.868788in}}%
\pgfpathlineto{\pgfqpoint{0.987937in}{1.887585in}}%
\pgfpathlineto{\pgfqpoint{1.001869in}{1.905053in}}%
\pgfpathlineto{\pgfqpoint{1.017924in}{1.921185in}}%
\pgfpathlineto{\pgfqpoint{1.036099in}{1.935979in}}%
\pgfpathlineto{\pgfqpoint{1.056427in}{1.949431in}}%
\pgfpathlineto{\pgfqpoint{1.078977in}{1.961543in}}%
\pgfpathlineto{\pgfqpoint{1.103855in}{1.972318in}}%
\pgfpathlineto{\pgfqpoint{1.131097in}{1.981754in}}%
\pgfpathlineto{\pgfqpoint{1.160668in}{1.989834in}}%
\pgfpathlineto{\pgfqpoint{1.192673in}{1.996545in}}%
\pgfpathlineto{\pgfqpoint{1.227228in}{2.001872in}}%
\pgfpathlineto{\pgfqpoint{1.264453in}{2.005797in}}%
\pgfpathlineto{\pgfqpoint{1.304475in}{2.008300in}}%
\pgfpathlineto{\pgfqpoint{1.347426in}{2.009360in}}%
\pgfpathlineto{\pgfqpoint{1.393444in}{2.008951in}}%
\pgfpathlineto{\pgfqpoint{1.468540in}{2.005524in}}%
\pgfpathlineto{\pgfqpoint{1.551374in}{1.998631in}}%
\pgfpathlineto{\pgfqpoint{1.642489in}{1.988157in}}%
\pgfpathlineto{\pgfqpoint{1.742263in}{1.973993in}}%
\pgfpathlineto{\pgfqpoint{1.851031in}{1.955943in}}%
\pgfpathlineto{\pgfqpoint{1.968660in}{1.933832in}}%
\pgfpathlineto{\pgfqpoint{2.094235in}{1.907574in}}%
\pgfpathlineto{\pgfqpoint{2.226070in}{1.877165in}}%
\pgfpathlineto{\pgfqpoint{2.361697in}{1.842690in}}%
\pgfpathlineto{\pgfqpoint{2.452652in}{1.817528in}}%
\pgfpathlineto{\pgfqpoint{2.542719in}{1.790708in}}%
\pgfpathlineto{\pgfqpoint{2.630632in}{1.762366in}}%
\pgfpathlineto{\pgfqpoint{2.715149in}{1.732672in}}%
\pgfpathlineto{\pgfqpoint{2.795199in}{1.701816in}}%
\pgfpathlineto{\pgfqpoint{2.869891in}{1.670008in}}%
\pgfpathlineto{\pgfqpoint{2.904999in}{1.653818in}}%
\pgfpathlineto{\pgfqpoint{2.904999in}{1.653818in}}%
\pgfusepath{stroke}%
\end{pgfscope}%
\begin{pgfscope}%
\pgfsetrectcap%
\pgfsetmiterjoin%
\pgfsetlinewidth{0.803000pt}%
\definecolor{currentstroke}{rgb}{0.000000,0.000000,0.000000}%
\pgfsetstrokecolor{currentstroke}%
\pgfsetdash{}{0pt}%
\pgfpathmoveto{\pgfqpoint{0.562500in}{0.275000in}}%
\pgfpathlineto{\pgfqpoint{0.562500in}{2.200000in}}%
\pgfusepath{stroke}%
\end{pgfscope}%
\begin{pgfscope}%
\pgfsetrectcap%
\pgfsetmiterjoin%
\pgfsetlinewidth{0.803000pt}%
\definecolor{currentstroke}{rgb}{0.000000,0.000000,0.000000}%
\pgfsetstrokecolor{currentstroke}%
\pgfsetdash{}{0pt}%
\pgfpathmoveto{\pgfqpoint{4.050000in}{0.275000in}}%
\pgfpathlineto{\pgfqpoint{4.050000in}{2.200000in}}%
\pgfusepath{stroke}%
\end{pgfscope}%
\begin{pgfscope}%
\pgfsetrectcap%
\pgfsetmiterjoin%
\pgfsetlinewidth{0.803000pt}%
\definecolor{currentstroke}{rgb}{0.000000,0.000000,0.000000}%
\pgfsetstrokecolor{currentstroke}%
\pgfsetdash{}{0pt}%
\pgfpathmoveto{\pgfqpoint{0.562500in}{0.275000in}}%
\pgfpathlineto{\pgfqpoint{4.050000in}{0.275000in}}%
\pgfusepath{stroke}%
\end{pgfscope}%
\begin{pgfscope}%
\pgfsetrectcap%
\pgfsetmiterjoin%
\pgfsetlinewidth{0.803000pt}%
\definecolor{currentstroke}{rgb}{0.000000,0.000000,0.000000}%
\pgfsetstrokecolor{currentstroke}%
\pgfsetdash{}{0pt}%
\pgfpathmoveto{\pgfqpoint{0.562500in}{2.200000in}}%
\pgfpathlineto{\pgfqpoint{4.050000in}{2.200000in}}%
\pgfusepath{stroke}%
\end{pgfscope}%
\begin{pgfscope}%
\pgfsetbuttcap%
\pgfsetmiterjoin%
\definecolor{currentfill}{rgb}{1.000000,1.000000,1.000000}%
\pgfsetfillcolor{currentfill}%
\pgfsetfillopacity{0.800000}%
\pgfsetlinewidth{1.003750pt}%
\definecolor{currentstroke}{rgb}{0.800000,0.800000,0.800000}%
\pgfsetstrokecolor{currentstroke}%
\pgfsetstrokeopacity{0.800000}%
\pgfsetdash{}{0pt}%
\pgfpathmoveto{\pgfqpoint{0.659722in}{0.344444in}}%
\pgfpathlineto{\pgfqpoint{1.685332in}{0.344444in}}%
\pgfpathquadraticcurveto{\pgfqpoint{1.713110in}{0.344444in}}{\pgfqpoint{1.713110in}{0.372222in}}%
\pgfpathlineto{\pgfqpoint{1.713110in}{0.948750in}}%
\pgfpathquadraticcurveto{\pgfqpoint{1.713110in}{0.976528in}}{\pgfqpoint{1.685332in}{0.976528in}}%
\pgfpathlineto{\pgfqpoint{0.659722in}{0.976528in}}%
\pgfpathquadraticcurveto{\pgfqpoint{0.631944in}{0.976528in}}{\pgfqpoint{0.631944in}{0.948750in}}%
\pgfpathlineto{\pgfqpoint{0.631944in}{0.372222in}}%
\pgfpathquadraticcurveto{\pgfqpoint{0.631944in}{0.344444in}}{\pgfqpoint{0.659722in}{0.344444in}}%
\pgfpathlineto{\pgfqpoint{0.659722in}{0.344444in}}%
\pgfpathclose%
\pgfusepath{stroke,fill}%
\end{pgfscope}%
\begin{pgfscope}%
\pgfsetrectcap%
\pgfsetroundjoin%
\pgfsetlinewidth{1.505625pt}%
\definecolor{currentstroke}{rgb}{0.121569,0.466667,0.705882}%
\pgfsetstrokecolor{currentstroke}%
\pgfsetdash{}{0pt}%
\pgfpathmoveto{\pgfqpoint{0.687500in}{0.872361in}}%
\pgfpathlineto{\pgfqpoint{0.826389in}{0.872361in}}%
\pgfpathlineto{\pgfqpoint{0.965278in}{0.872361in}}%
\pgfusepath{stroke}%
\end{pgfscope}%
\begin{pgfscope}%
\definecolor{textcolor}{rgb}{0.000000,0.000000,0.000000}%
\pgfsetstrokecolor{textcolor}%
\pgfsetfillcolor{textcolor}%
\pgftext[x=1.076389in,y=0.823750in,left,base]{\color{textcolor}\rmfamily\fontsize{10.000000}{12.000000}\selectfont \(\displaystyle \gamma=0.750\)}%
\end{pgfscope}%
\begin{pgfscope}%
\pgfsetrectcap%
\pgfsetroundjoin%
\pgfsetlinewidth{1.505625pt}%
\definecolor{currentstroke}{rgb}{1.000000,0.498039,0.054902}%
\pgfsetstrokecolor{currentstroke}%
\pgfsetdash{}{0pt}%
\pgfpathmoveto{\pgfqpoint{0.687500in}{0.675556in}}%
\pgfpathlineto{\pgfqpoint{0.826389in}{0.675556in}}%
\pgfpathlineto{\pgfqpoint{0.965278in}{0.675556in}}%
\pgfusepath{stroke}%
\end{pgfscope}%
\begin{pgfscope}%
\definecolor{textcolor}{rgb}{0.000000,0.000000,0.000000}%
\pgfsetstrokecolor{textcolor}%
\pgfsetfillcolor{textcolor}%
\pgftext[x=1.076389in,y=0.626944in,left,base]{\color{textcolor}\rmfamily\fontsize{10.000000}{12.000000}\selectfont \(\displaystyle \gamma=0.770\)}%
\end{pgfscope}%
\begin{pgfscope}%
\pgfsetrectcap%
\pgfsetroundjoin%
\pgfsetlinewidth{1.505625pt}%
\definecolor{currentstroke}{rgb}{0.172549,0.627451,0.172549}%
\pgfsetstrokecolor{currentstroke}%
\pgfsetdash{}{0pt}%
\pgfpathmoveto{\pgfqpoint{0.687500in}{0.478750in}}%
\pgfpathlineto{\pgfqpoint{0.826389in}{0.478750in}}%
\pgfpathlineto{\pgfqpoint{0.965278in}{0.478750in}}%
\pgfusepath{stroke}%
\end{pgfscope}%
\begin{pgfscope}%
\definecolor{textcolor}{rgb}{0.000000,0.000000,0.000000}%
\pgfsetstrokecolor{textcolor}%
\pgfsetfillcolor{textcolor}%
\pgftext[x=1.076389in,y=0.430139in,left,base]{\color{textcolor}\rmfamily\fontsize{10.000000}{12.000000}\selectfont \(\displaystyle \gamma=0.730\)}%
\end{pgfscope}%
\end{pgfpicture}%
\makeatother%
\endgroup%

        \end{center}
    \end{frame}

    \begin{frame}
    \frametitle{Verhalten in 3D: Beispiel Lorenz System}
        \begin{center}
            \includegraphics[width=.7\textwidth]{../images/lorenz_animation_thumbnail.jpg}
        \end{center}
    \end{frame}
    % Plots von ENSO. Lösungen konvergieren.
    % Idee: Plot wo unterwegs die Gleichung (Konstanten) ändert -> bleibt trotzdem auf einem ähnlichen Orbit
    % Link zum Sonnensystem herstellen!
    % Chaos in 3D möglich: Zeigen des Videos von Müller

\end{document}

