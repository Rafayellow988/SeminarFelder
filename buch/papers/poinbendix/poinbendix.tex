\section{Der Satz von Poincaré-Bendixson} \label{poinbendix:section:poinbendix}

%TODO def omega and alpha set
%TODO def 2D Kugeloberfläche S


\begin{satz}[Poincaré-Bendixson]
\label{poinbendix:satz:poinbendix}
$\Phi_t(p) \in \Xi^r(\mathbb{S}^2)$ sei ein $r$-Fach differenzierbares, zweidimensionales dynamisches System mit Startpunkt $p \in \mathbb{S}$.
Dann gilt für das Omega-Limit Set $\omega(p)$ eine der folgenden Optionen:
\begin{enumerate}
\item $\omega(p)$ ist eine Singularität
\item $\omega(p)$ ist ein geschlossener Orbit
\item $\omega(p)$ ist ein geschlossener Orbit welcher Singularitäten verbindet
\end{enumerate}
\end{satz}

In den folgenden Abschnitten wird für jeden Fall jeweils ein Beispiel gezeigt.

\subsection{Fall 1: $\omega(p)$ ist eine Singularität} \label{poinbendix:subsection:fall1}
\subsection{Fall 2: $\omega(p)$ ist ein geschlossener Orbit} \label{poinbendix:subsection:fall2}
\subsection{Fall 3: $\omega(p)$ ist ein geschlossener Orbit welcher Singularitäten verbindet} \label{poinbendix:subsection:fall3}

