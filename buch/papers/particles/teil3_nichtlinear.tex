%
% teil2.tex -- Beispiel-File für teil2 
%
% (c) 2020 Prof Dr Andreas Müller, Hochschule Rapperswil
%
% !TEX root = ../../buch.tex
% !TEX encoding = UTF-8
%
\section{Nichtlineares Medium
\label{particles:section:nichtlinear}}
\kopfrechts{Nichtlineares Medium}
Oftmals verhalten sich linear scheinende Medien in Extremalbereichen nichtlinear.
Dies hat einige Auswirkungen auf das Feld und dessen Verhalten, 
was einige interessante Eigenschaften mit sich bringt. 
% TODO: Allenfalls eine "Definition" für nichtlinearität in diesem Rahmen erläutern

% TODO: Grafik mit gleichem Aufbau wie in "lineares Medium", aber mit nichtlinearem medium (Zwei Lichtwellen)

\subsection{Schwinger Limit}
% TODO: Beim Schwinger Limit wird ein lineares Medium nichtlinear


\subsubsection{Implikationen} % NOTE: Noch nicht sicher, ob es diese subsubsection wirklich braucht...


% TODO: Folgenden Titel anpassen, da noch unsicher, wie man das auf deutsch genau nennen sollte
\subsection{Selbsterhaltende Energie / Beschränkte Energie / Partikel-ähnliches Verhalten} % NOTE: "Confined Energy"
% TODO: Was für Bedingungen müssen erfüllt sein, damit dies geschieht? 
%       Kann man das überhaupt so einfach klassifizieren?
%       Allenfalls Analogie zu Schwingkreisen herstellen?

% TODO: Grafik von "Partikel" einfügen

