%
% teil3.tex -- Beispiel-File für Teil 3
%
% (c) 2020 Prof Dr Andreas Müller, Hochschule Rapperswil
%
% !TEX root = ../../buch.tex
% !TEX encoding = UTF-8
%
\section{Fazit\label{particles:section:fazit}}
\kopfrechts{Fazit}

\mynote{Erkenntnisse: Bei nichtlinearitäten können unter den Richtigen Bedingungen interessante Phänomene auftreten.}

\mynote{Bezug zur Realität: Ein sehr stark vereinfachtes Modell eines Partikels. 
Falls es sich hierbei um ein Soliton handelt: 
Diese Eigenschaft wird in Glasfasern nützlich, 
da diese oft auch nichtlineares Verhalten aufweisen.}

\mynote{Mögliche Erweiterungen: Was könnte man noch vertiefen/konkretisieren?
Wie ähnlich ist es wirklich zum Partikelmodell?}
