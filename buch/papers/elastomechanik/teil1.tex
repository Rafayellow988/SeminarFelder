%
% teil1.tex -- Beispiel-File für das Paper
%
% (c) 2020 Prof Dr Andreas Müller, Hochschule Rapperswil
%
% !TEX root = ../../buch.tex
% !TEX encoding = UTF-8
%
\section{Grundbegriffe der Baustatik und Mechanik}
\label{elastomechanik:section:teil1}
\subsection{Begriffe}
\begin{description}	
	\item[\textbf{Druckkräfte:}] Druckkräfte sind Kräfte, die senkrecht auf die Oberfläche eines Körpers wirken und dabei eine Kompression bzw. Zusammenstauchung des Körpers verursachen. 
	Eine negative Normalkraft entspricht einer Druckkraft.
	
	\item[\textbf{Druckspannungen:}] Druckspannungen sind Spannungen, die im Inneren eines Materials entstehen, wenn Druckkräfte senkrecht auf die Oberfläche eines Körpers ausgeübt werden. 
	Sie bewirken eine Stauchung oder Kompression des Materials.
	Druckspannungen haben ein negatives Vorzeichen, da die Kräfte nach innen gerichtet sind.
	
	\item[\textbf{Elastizitätsmodul ($E$):}] Der Elastizitätsmodul (E-Modul) ist eine Materialkonstante, die die Steifigkeit eines Werkstoffs beschreibt. 
	Er gibt an, wie stark ein Material auf eine mechanische Spannung mit Dehnung reagiert. 
	Je grösser der Elastizitätsmodul, desto weniger verformt sich das Material unter einer gegebenen Belastung.
	Der Elastizitätsmodul wird definiert als
	\begin{equation}
		E=
		\frac{\sigma}{\varepsilon}
	\end{equation}
	$\sigma$ = Spannung
	
	$\varepsilon$ = Dehnung
	
	\item[\textbf{Isotropie:}] Isotropie beschreibt eine Materialeigenschaft, bei der die mechanischen Eigenschaften in allen Raumrichtungen identisch sind. 
	Ein isotropes Material reagiert auf Belastungen unabhängig von der Belastungsrichtung stets gleich.
	In der linearen Elastizitätstheorie bedeutet Isotropie, dass das Materialverhalten vollständig durch zwei unabhängige Materialkonstanten beschrieben werden kann, etwa den Elastizitätsmodul ($E$) und die Poissonzahl ($\nu$).  
	Isotrope Annahmen vereinfachen die mathematische Modellierung deutlich und finden in der technischen Mechanik insbesondere bei homogenen Metallen, Glas und einigen Kunststoffen Anwendung \cite{elastomechanik:Isotropie}.
	
	\item[\textbf{Normalkraft ($N$):}] Die Normalkraft ist eine Kraft, die senkrecht (normal) zur Querschnittsfläche eines Körpers oder Bauteils wirkt. 
	Sie entsteht durch Zug oder Druck entlang der Längsachse des Körpers und ist eine der wichtigsten Grundkräfte in der Technischen Mechanik und Statik.
	
	\item[\textbf{Normalspannungen ($\sigma_N$):}] Die Normalspannung ist eine Form der Spannung, die auftritt, wenn eine Kraft senkrecht (normal) zur Fläche eines Körpers wirkt. 
	Sie beschreibt den inneren Spannungszustand eines Materials bei Zug- oder Druckbelastung.
	
	\item[\textbf{Poissonzahl ($\nu$):}] Die Poissonzahl, auch bekannt als Querkontraktionszahl, ist eine materialabhängige Konstante, die das Verhältnis zwischen der Querdehnung und der Längsdehnung eines Körpers unter Zug- oder Druckbelastung beschreibt.
	Sie ist definiert als
	\begin{equation}
		\nu=
		-\frac{\varepsilon_{quer}}{\varepsilon_{laengs}}
	\end{equation}
	$\varepsilon_quer$ = Dehnung senkrecht zur Belastungsrichtung
	
	$\varepsilon_längs$ = Dehnung in Belastungsrichtung
	
	\item[\textbf{Spannung ($\sigma$):}] Die Spannung beschreibt die innere Kraftverteilung in einem Material, die infolge externer Belastungen (wie Zug, Druck oder Scherung) entsteht. 
	Sie gibt an, welche Kraft pro Flächeneinheit innerhalb eines Körpers wirkt.
		
	\item[\textbf{Verzerrung ($\varepsilon$):}] Die Verzerrung (auch Dehnung genannt) beschreibt die relative Änderung der Form oder Länge eines Körpers infolge einer äußeren Belastung. 
	Sie ist ein maßloser Wert, da sie ein Verhältnis zweier Längen ist.
	Die Normalverrung wird berechnet mit
	\begin{equation}
		\varepsilon 
		= \frac{\Delta L}{L_0}
	\end{equation}
	$\varepsilon$ = Verzerrung
	
	$\Delta L$ = Längenänderung
	
	$\L_0$ = ursprüngliche Länge des Körpers
	
	\item[\textbf{Verzerrungstensor ($\varepsilon_{ij}$):}] Der Verzerrungstensor beschreibt die Verformung (Dehnung und Scherung) eines Körpers infolge äußerer Belastungen in drei Raumrichtungen. 
	Er erfasst nicht nur Längenänderungen (Normalverzerrungen), sondern auch Winkeländerungen (Schubverzerrungen) und stellt somit die vollständige lokale Verformung eines Materials im Raum dar.
	Die Verzerrungstensor wird über die Ableitungen des Verschiebungsfeldes definiert als
	\begin{equation}
		\varepsilon_{ij} = 
		\frac{1}{2} \left( \frac{\partial u_i}{\partial x_j} + \frac{\partial u_j}{\partial x_i} \right)
	\end{equation}
	wobei er in drei Raumdimensionen die Form einer symmetrischen 3×3-Matrix annimmt:
	\begin{equation}
	\boldsymbol{\varepsilon} =
	\begin{bmatrix}
		\varepsilon_{xx} & \varepsilon_{xy} & \varepsilon_{xz} \\
		\varepsilon_{yx} & \varepsilon_{yy} & \varepsilon_{yz} \\
		\varepsilon_{zx} & \varepsilon_{zy} & \varepsilon_{zz}
	\end{bmatrix}
	\end{equation}
		
	\item[\textbf{Zugkraft:}] Die Zugkraft ist eine Kraft, die längs zur Achse eines Körpers wirkt und diesen in Richtung Verlängerung belastet. 
	Sie verursacht Zugverformungen im Material.
	
	\item[\textbf{Zugspannungen:}] Die Zugspannung ist die innere Spannung, die durch eine Zugkraft im Material erzeugt wird. 
	Sie wirkt senkrecht zur Querschnittsfläche des Körpers und beschreibt die Kraft pro Flächeneinheit, mit der das Material dem Zug widersteht.
\end{description}

\subsection{Materialgesetzlichkeiten}

