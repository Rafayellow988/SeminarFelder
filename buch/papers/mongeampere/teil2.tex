%
% teil2.tex -- Beispiel-File für teil2 
%
% (c) 2020 Prof Dr Andreas Müller, Hochschule Rapperswil
%
% !TEX root = ../../buch.tex
% !TEX encoding = UTF-8
%
\section{Prescribed Gaussian Curvature
\label{mongeampere:section:teil2}}
\kopfrechts{Teil 2}
Mit der Definition der Gausschen Krümmung können wir nun eine Differentialgleichung aufstellen,
welche als Lösung eine Fläche mit einer gewünschten Gausschen Krümmung hat.
Dafür nehmen wir die expliyite Form einer Fläche $z = F(x,y)$.
Die Variablen sind nun nicht mehr $u, v$ sonder $x, y$.
Nun brauchen wir die Koeffizienten der Fundamentalformen welche mit dem Radiusvektor und seinen Ableitungen 
beschrieben werden können
\begin{align}
  \vec r &= \begin{pmatrix}
   x \\
   y \\
   F(x, y)
 \end{pmatrix} \\
    \vec r_x &= \begin{pmatrix}
      1 \\
      0 \\
      \pdv{F}{x}
    \end{pmatrix},
      \quad &
    \vec r_y &= \begin{pmatrix}
      0 \\
      1 \\
      \pdv{F}{y}
    \end{pmatrix}\\
      \vec r_{xx} &= \begin{pmatrix}
      0 \\
      0 \\
      \pdv[2]{F}{x}
    \end{pmatrix},
    \quad &
    \vec r_{xy} &= \begin{pmatrix}
      0 \\
      0 \\
      \pdv{F}{x}{y}
    \end{pmatrix},
      \quad &
    \vec r_{yy} &= \begin{pmatrix}
      0 \\
      0 \\
      \pdv[2]{F}{y}
    \end{pmatrix}\\
\end{align}
Damit sind die Koeffizienten der ersten Fundamentalform 
\begin{equation}
  E = 1 + \left(\pdv{F}{x}\right)^2, \quad
  F = \pdv{F}{x} \cdot \pdv{F}{y}, \quad
  G = 1 + \left(\pdv{F}{y}\right)^2
  \label{mongeampere:fund1exp}
\end{equation}
Für die zweite Fundamentalform wird die Flächennormale benötigt, welche aus \eqref{mongeampere:norm} und \eqref{mongeampere:ds} 
\begin{equation}
  \vec m = \frac{\vec r_x \times \vec r_y}{\sqrt{EG-F^2}} = \begin{pmatrix}
    \pdv{F}{x} \\
    \pdv{F}{y} \\
    1
  \end{pmatrix}
  \frac{1}{\sqrt{EG-F^2}}
  \label{mongeampere:norm2}
\end{equation}
ist.
Somit sind die Koeffizienten der zweiten Fundamentalform
\begin{equation}
  L = \frac{\pdv[2]{F}{x}}{\sqrt{EG-F}}, \quad
  M = \frac{\pdv{F}{x}{y}}{\sqrt{EG-F}}, \quad
  N = \frac{\pdv[2]{F}{y}}{\sqrt{EG-F}}
  \label{mongeampere:2fund22}
\end{equation}
Setzen wir die Koeffizienten auf \eqref{mongeampere:fund1exp} und \eqref{mongeampere:2fund22} in die Formel der Gausschen Krümmung \eqref{mongeampere:gausskrumm}
ein, erhalten wir
\begin{equation}
  K = \frac{
    \pdv[2]{F}{x} \cdot \pdv[2]{F}{y} - \left(\pdv{F}{x}{y} \right)^2}
    {\left[1 + 
    \left(\pdv{f}{x}\right)^2 +
    \left(\pdv{f}{y}\right)^2\right]^2}
  \label{mongeampere:pd}
\end{equation}
Wie wir sehen können, ist der Zähler gleich der Determinante der Hessschen Matrix von $F(x,y)$.
Der Nenner ist eine nichtlineare Funktion der ersten Ableitungen.
Somit konnten wir zeigen, dass das Prescribed Gaussian Curvature Problem einer explizit definierten Fläche in einer 
Monge-Ampèreschen Gleichung resultiert.

