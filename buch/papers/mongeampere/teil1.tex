%
% teil1.tex -- Beispiel-File für das Paper
%
% (c) 2020 Prof Dr Andreas Müller, Hochschule Rapperswil
%
% !TEX root = ../../buch.tex
% !TEX encoding = UTF-8
%
\section{Gaussche Krümmung
\label{mongeampere:section:teil1}}
\kopfrechts{Problemstellung}
Damit wir verstehen wie die Monge Ampere Gleichung mit der 
Krümmung einer Fläche zusammenhängt, brauchen wir zuerst eine Methode die 
Krümmug einer Fläche yu beschreiben.
Wir betrachten dabei die Fläche
\begin{equation}
  y = F(x, y)
  \label{mongeampere:area}
\end{equation}
oder in Parameterform
\begin{equation}
  x = \varphi(u,v), \quad y = \psi(u,v), \quad z = \omega(u,v)
  \label{mongeampere:areaparam}
\end{equation}
Die Fläche kann in der Parameterform auch als Radiusvektor $\vec r (u, v)$
schreiben.
Leitet man diesen Radiusvektor nach den Prametern $u$ und $v$ ab bekommnt man zwei Tangenten
$\vec r_u$ und $\vec r_v$,
mit welchen die Flächennormale 
\begin{equation}
  \vec m = \frac{\vec r_u \times \vec r_v}{\sqrt{(\vec r_u \times \vec r_v)^2}} 
  \label{mongeampere:norm}
\end{equation}
wobei $(\dot)^2$ für das Skalarprodukt steht.



\subsection{Erste Fundamentalform
\label{mongeampere:subsection:finibus}}
\ref{mongeampere:section:teil2}.
Die erste Fundamentalform beschreibt die innere Geometrie einer Fläche.
Betrachtet man das Quadrate des Bogendifferential einer auf der Fläche 
beschriebenen Kurve $s(u,v)$ so ist 
\begin{equation}
  \begin{split}
    \dd s^2 &= \dd x^2 + \dd y^2 + \dd z^2 \\
          &= \left(\pdv{x}{u}\dd u + \pdv{x}{v}\dd v  \right)^2
          + \left(\pdv{z}{u}\dd u + \pdv{z}{v}\dd v  \right)^2
          + \left(\pdv{z}{u}\dd u + \pdv{z}{v}\dd v  \right)^2.
  \end{split}
  \label{mongeampere:bogdiff}
\end{equation}
Rechnet man die Klammern aus erhält man 
\begin{equation}
  \begin{split}
    \dd s^2 &=  \left(\pdv{x}{u}\right)^2\\
          &= \left(\pdv{x}{u}\dd u + \pdv{x}{v}\dd v  \right)^2
          + \left(\pdv{z}{u}\dd u + \pdv{z}{v}\dd v  \right)^2
          + \left(\pdv{z}{u}\dd u + \pdv{z}{v}\dd v  \right)^2
  \end{split}
  \label{mongeampere:bogdiff}
\end{equation}


