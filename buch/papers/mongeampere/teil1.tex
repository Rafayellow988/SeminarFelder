%
% teil1.tex -- Beispiel-File für das Paper
%
% (c) 2020 Prof Dr Andreas Müller, Hochschule Rapperswil
%
% !TEX root = ../../buch.tex
% !TEX encoding = UTF-8
%
\section{Gaussche Krümmung
\label{mongeampere:section:teil1}}
\kopfrechts{Problemstellung}
Damit wir verstehen wie die Monge Ampere Gleichung mit der 
Krümmung einer Fläche zusammenhängt, brauchen wir zuerst eine Methode die 
Krümmug einer Fläche yu beschreiben.
Wir betrachten dabei die Fläche
\begin{equation}
  y = F(x, y)
  \label{mongeampere:area}
\end{equation}
oder in Parameterform
\begin{equation}
  x = \varphi(u,v), \quad y = \psi(u,v), \quad z = \omega(u,v)
  \label{mongeampere:areaparam}
\end{equation}
Die Fläche kann in der Parameterform auch als Radiusvektor $\vec r (u, v)$
schreiben.
Leitet man diesen Radiusvektor nach den Prametern $u$ und $v$ ab bekommnt man zwei Tangenten
$\vec r_u$ und $\vec r_v$,
mit welchen die Flächennormale 
\begin{equation}
  \vec m = \frac{\vec r_u \times \vec r_v}{\sqrt{(\vec r_u \times \vec r_v)^2}} 
  \label{mongeampere:norm}
\end{equation}
beschrieben werden kann, wobei $(\cdot)^2$ für das Skalarprodukt steht.



\subsection{Erste Fundamentalform
\label{mongeampere:subsection:finibus}}
\ref{mongeampere:section:teil2}.
Die erste Fundamentalform beschreibt die innere Geometrie einer Fläche.
Betrachtet man das Quadrate des Bogendifferential einer auf der Fläche 
beschriebenen Kurve $S(u,v)$ so ist 
\begin{equation}
  \begin{split}
    \dd s^2 &= \dd x^2 + \dd y^2 + \dd z^2 \\
          &= \left(\pdv{x}{u}\dd u + \pdv{x}{v}\dd v  \right)^2
          + \left(\pdv{z}{u}\dd u + \pdv{z}{v}\dd v  \right)^2
          + \left(\pdv{z}{u}\dd u + \pdv{z}{v}\dd v  \right)^2.
  \end{split}
  \label{mongeampere:bogdiff}
\end{equation}
Rechnet man die Klammern aus erhält man 
\begin{equation}
    \dd s^2 = E(u,v) \dd u^2 + 2F(u,v) \dd u \, \dd v + G()
    \label{mongeampere:1fundform}
\end{equation}
mit fuck
\begin{align}
     E(u,v) &= \left(\pdv{x}{u} \right)^2 +
     \left(\pdv{y}{u} \right)^2 +
     \left(\pdv{z}{u} \right)^2 
            &= \vec r_u^2\\
     F(u,v) &= shit
     \pdv{x}{u} \cdot \pdv{x}{v} +
     \pdv{y}{u} \cdot \pdv{y}{v} +
     \pdv{z}{u} \cdot \pdv{z}{v}
            &= \vec r_u \cdot \vec r_v \\
      G(u,v) &= \left(\pdv{x}{v} \right)^2 +
     \left(\pdv{y}{v} \right)^2 +
     \left(\pdv{z}{v} \right)^2 
             &= \vec r_v ^2\\
  \label{mongeampere:1fundbed}
\end{align}
Diese Forulierung vom Bogendifferential ist als erste Gausssche Fundamentalform bekannt.
Man kann sie auch in Form des Metrischen Tensors 
\begin{equation}
  g_{ij} = \begin{pmatrix}
    E & F \\
    F & G \\
  \end{pmatrix}
  \label{mongeampere:erstmettens}
\end{equation}
beschreiben.

\subsection{Zweite Fundamentalform}
Die zweite Gaussche Fundamentalform befasst sich mit der äusseren Geometrie einer 
Fläche und beschreibt wie sich die Tangente entlang einer Flächenkurve verändert.
Dafür betrachten wir wieder die Kurve $S(u,v)$ mit einem Tangeteneinheitsvektor 
$\vec t$;
Da die Kurve auf der Fläche liegt ist $\vec t$ in jedem Punkt senkrecht auf der 
Flächennormale $\vec m$, womit $\vec t \cdot \vec m = 0$ gilt. 
Berechnen wir nun 
\begin{equation}
  \dv{\vec t \cdot \vec m}{s}
  \label{mongeampere:2fund0}
\end{equation}
erhalten wir 
\begin{equation}
  \dv{\vec t}{s} \cdot \vec m + \vec t \cdot \dv{\vec m}{s} = 0. 
  \label{mongeampere:2fund1}
\end{equation}
Der Term $\dd \vec t / \dd s = \varrho^{-1} \vec n$ beschreibt einen Vektor in Richtung der Kurvennormalen 
$\vec n$ mit der Länge $\varrho^{-1}$, welche den inversen Krümmungsradius der Kurve 
darstellt.
Somit können wir \eqref{mongeampere:2fund1} umformen nach
\begin{equation}
  \frac{\vec n \cdot \vec m}{\varrho} = - \frac{\dd \vec r \cdot \dd \vec m }{\dd s^2} 
  \label{mongeampere:2fund2}
\end{equation}
Rechnet man den Zähler als Funktion der Parameter $u, v$ aus, erhält man die zweite 
Gausssche Fundamentalform
\begin{equation}
  \mathrm{I\!I} = L(u, v) \dd u^2 + 2 M (u,v) \dd u \dd v + N(u,v) \dd v^2.
  \label{mongeampere:2fund}
\end{equation}
mit den Koeffizienten
\begin{align*}
  L &= \vec r_{uu} \cdot m \\ 
  M &= \vec r_{uv} \cdot m \\
  N &= \vec r_{vv} \cdot m.
  \label{mongeampere:2fundkoef}
\end{align*}
Die zweite Gausssche Fundamentalform beschreibt wie start sich die Fläche an einem Punkt
von der Tangentialebene entfernt, wenn sich die Parameter etwas ändern.
Wie stark sich die Fläche von der Tangentialebene entfernt hat einen Zusammenhang mit der Krümmung der Fläche.

In \eqref{mongeampere:2fund2} haben wir die zweite Fundamentalform in Richtung einer Flächenkruve mit
dem Quadrat ihres Bigendifferentiales normalisiert.
Damit erhielten wir den Anteil der Krümmung der Flächenkruve, die von der Krümmung der Fläche kommt.
