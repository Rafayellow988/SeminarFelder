%
% teil1.tex -- Beispiel-File für das Paper
%
% (c) 2020 Prof Dr Andreas Müller, Hochschule Rapperswil
%
% !TEX root = ../../buch.tex
% !TEX encoding = UTF-8
%
\section{Differentialgeometrie
\label{mongeampere:section:teil1}}
\kopfrechts{Problemstellung}
Damit wir verstehen wie die Monge Ampere Gleichung mit der 
Krümmung einer Fläche zusammenhängt, brauchen wir zuerst eine Methode die 
Krümmug einer Fläche yu beschreiben.
Wir betrachten dabei die Fläche
\begin{equation}
  F(x, y, z) = 0
  \label{mongeampere:area}
\end{equation}
oder in parameterform
\begin{equation}
  x = \varphi(u,v), \quad y = \psi(u,v), \quad z = \omega(u,v)
  \label{mongeampere:areaparam}
\end{equation}

\subsection{Erste Fundamentalform
\label{mongeampere:subsection:finibus}}
\ref{mongeampere:section:teil2}.


