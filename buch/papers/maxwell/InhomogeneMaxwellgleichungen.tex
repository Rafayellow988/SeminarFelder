%
% InhomogeneMaxwellgleichungen.tex -- Beispiel-File für das Paper
%
% (c) 2020 Prof Dr Andreas Müller, Hochschule Rapperswil
%
% !TEX root = ../../buch.tex
% !TEX encoding = UTF-8
%
\section{Inhomogene Maxwellgleichungen
	\label{maxwell:section:InhomogeneMaxwellgleichungen}}
\kopfrechts{Inhomogene Maxwellgleichungen}
(Teil dF = 0 kommt dann vorher und auch die "Herleitung" davon)

Den Faraday-Tensor haben wir definiert als

\begin{equation}
	F = \begin{pmatrix}
		0 & -E_x & -E_y & -E_z \\ E_x & 0 & B_z & -B_y \\ E_y & -B_z & 0 & B_x \\ E_z & B_y & -B_x & 0 
	\end{pmatrix}.
%	\label{maxwell:section:teil1:metrik}
\end{equation}
Wir haben dabei festgestellt, dass die äussere Ableitung von F verschwindet, da F eine geschlossene 2-Form ist.
Um nun die inhomogenen Maxwellgleichungen bestimmen zu können müssen wir den Hodgeoperator ($\star$) auf F anwenden.
Wir erinnern uns, dass F ausgeschrieben mit Wedgeprodukten definiert ist als

\begin{align*}
	F = 
	&- E_x \, dt \wedge dx - E_y \, dt \wedge dy - E_z \, dt \wedge dz \\
	&+ B_z \, dx \wedge dy - B_y \, dx \wedge dz + B_x \, dy \wedge dz.\\
\end{align*}

Der Hodgeoperator bildet eine k-Form auf eine (n-k)-Form ab und somit wird aus $\star F$ wieder eine 2-Form

\begin{align*}
	\star F =
	& - E_{x} \, \star(dt \wedge dx) - E_{y} \, \star(dt \wedge dy) - E_{z} \, \star(dt \wedge dz) \\
	& + B_z \, \star(dx \wedge dy) - B_y \, \star(dx \wedge dz) + B_x \, \star(dy \wedge dz)\\
	= 
	& \, E_{x} \, dy \wedge dz - E_{y} \, dx \wedge dz + E_{z} \, dx \wedge dy \\
	& + B_z \, dt \wedge dz + B_y \, dt \wedge dy + B_x \, dt \wedge dx.\\
\end{align*}
Die erhaltene 2-form kann nun wieder als Matrix
\begin{equation}
	\star F = \begin{pmatrix}
		0 & B_x & B_y & B_z \\ -B_x & 0 & E_z & -E_y \\ -B_y & -E_z & 0 & E_x \\ -B_z & E_y & -E_x & 0 
	\end{pmatrix}
	%	\label{maxwell:section:teil1:metrik}
\end{equation}
geschrieben werden.

Nun wenden wir die äussere Ableiung darauf an und erhalten
\begin{align*}
	d(\star F) = \,
	& d (E_{x} \, dy \wedge dz - E_{y} \, dx \wedge dz + E_{z} \, dx \wedge dy + B_z \, dt \wedge dz + B_y \, dt \wedge dy + B_x \, dt \wedge dx)\\
	=
	& \frac{\partial E_x}{\partial t} dt \wedge dy \wedge dz + \frac{\partial E_x}{\partial x} dx \wedge dy \wedge dz -
	\frac{\partial E_y}{\partial t} dt \wedge dx \wedge dz -
	\frac{\partial E_y}{\partial y} dy \wedge dx \wedge dz +\\
	& \frac{\partial E_z}{\partial t} dt \wedge dx \wedge dy +
	\frac{\partial E_z}{\partial z} dz \wedge dx \wedge dy +
	\frac{\partial B_z}{\partial x} dx \wedge dt \wedge dz +
	\frac{\partial B_z}{\partial y} dy \wedge dt \wedge dz +\\
	& \frac{\partial B_y}{\partial x} dx \wedge dt \wedge dy +
	\frac{\partial B_y}{\partial z} dz \wedge dt \wedge dy +
	\frac{\partial B_x}{\partial y} dy \wedge dt \wedge dx +
	\frac{\partial B_x}{\partial z} dz \wedge dt \wedge dx\\
	=
	&\left( \frac{\partial E_x}{\partial x} + \frac{\partial E_y}{\partial y} + \frac{\partial E_z}{\partial z} \right) dx \wedge dy \wedge dz +
	\left(\frac{\partial E_x}{\partial t} + \frac{\partial B_y}{\partial z} - \frac{\partial B_z}{\partial y} \right) dt \wedge dy \wedge dz\\
	&\left( -\frac{\partial E_y}{\partial t} + \frac{\partial B_x}{\partial z} - \frac{\partial B_z}{\partial x} \right) dt \wedge dx \wedge dz +
	\left( \frac{\partial E_z}{\partial t} + \frac{\partial B_x}{\partial y} - \frac{\partial B_y}{\partial x} \right) dt \wedge dx \wedge dy. 
\end{align*}
Wenden wir nun noch mal den Hodge-Operator auf die erhaltene 3-Form an, erhalten wir
\begin{align*}
	\star d \star F = 
	&\left( \frac{\partial E_x}{\partial x} + \frac{\partial E_y}{\partial y} + \frac{\partial E_z}{\partial z} \right) (-dt) +
	\left(\frac{\partial E_x}{\partial t} + \frac{\partial B_y}{\partial z} - \frac{\partial B_z}{\partial y} \right) (-dx)\\
	&\left( -\frac{\partial E_y}{\partial t} + \frac{\partial B_x}{\partial z} - \frac{\partial B_z}{\partial x} \right) dy +
	\left( \frac{\partial E_z}{\partial t} + \frac{\partial B_x}{\partial y} - \frac{\partial B_y}{\partial x} \right) (-dz).\\
	=
	&\left( -\frac{\partial E_x}{\partial x} -\frac{\partial E_y}{\partial y} - \frac{\partial E_z}{\partial z} \right) dt +
	\left(-\frac{\partial E_x}{\partial t} - \frac{\partial B_y}{\partial z} + \frac{\partial B_z}{\partial y} \right) dx\\
	&\left( -\frac{\partial E_y}{\partial t} + \frac{\partial B_x}{\partial z} - \frac{\partial B_z}{\partial x} \right) dy +
	\left( -\frac{\partial E_z}{\partial t} - \frac{\partial B_x}{\partial y} + \frac{\partial B_y}{\partial x} \right) dz.
\end{align*}

Weiter vereinfachen und nachvollziehen...
