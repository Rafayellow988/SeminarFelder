%
% teil1.tex -- Beispiel-File für das Paper
%
% (c) 2020 Prof Dr Andreas Müller, Hochschule Rapperswil
%
% !TEX root = ../../buch.tex
% !TEX encoding = UTF-8
%
\section{Hodge Theorie provisorisch
\label{maxwell:section:teil1}}
\kopfrechts{Hodge Theorie provisorisch}

Für den Teil der inhomogenen Maxwell Gleichungen benötigen wir die Hodge-Duale auf 1-Formen, 2-Formen und 3-Formen im vierdimensionalen Minkowski-Raum.
Um dabei die korrekten Vorzeichen zu erhalten, muss die Hodge-Dualität mit Hilfe des metrischen Tensors $g^{ik}$  der Minkowski-Metrik verwendet werden.



\subsection{Minkowski Metrik}
In der speziellen Relativitätstheorie (SRT) wird die Minkowski-Metrik verwendet.
Da es in der SRT keine Krümmung und Gravitation gibt, sind alle Elemente ausserhalb der Diagonale des metrischen Tensors null und somit ist die Raum-Zeit flach.
Zwei Signaturen sind üblich.
Einerseits gibt es die $(-+++)$-Signatur, bei welcher die Zeitkomponente negativ und die Raumkomponenten positiv gezählt werden.
Andererseits gibt es die $(+---)$-Signatur, bei welcher die Zeitkomponente positiv und die Raumkomponenten negativ gezählt werden.
Beide Signaturen sind gleichwertig, solange man sich auf eine Metrik festlegt und diese konsequent beibehält.
Im Folgenden werden wir uns an die $(-+++)$-Signatur halten.
Daher definieren wir den metrischen Tensor als
\begin{equation}
	g^{ik} = \begin{pmatrix}
		-1 & 0 & 0 & 0 \\ 0 & 1 & 0 & 0 \\ 0 & 0 & 1 & 0 \\ 0 & 0 & 0 & 1 
	\end{pmatrix}.
	\label{maxwell:section:teil1:metrik}
\end{equation}
Der Ausdruck für ein Linienelement in dieser Metrik ist definiert als
\begin{equation}
	dl^2 = -c^2dt^2 +dx^2+dy^2+dz^2.
\end{equation}
Eine Konsequenz dieser Signatur ist, dass zeitartige Abstände $ds^2 < 0$ und raumartige Abstände $ds^2 > 0$ sind.


\subsection{Herleitung Hodge-Duale (Verweis auf Abschnitt im Buch)}
Wir verwenden die Definition
$$\alpha \wedge \ast \beta = \langle \alpha, \beta \rangle \operatorname{vol}(M).$$
Folgend alle Berechungen der Hodge-Duale auf 1-, 2- und 3-Formen mit $(-+++)$ -Metrik.
Es gilt für $g^{ik}$ gemäss \eqref{maxwell:section:teil1:metrik} und für $ \operatorname{vol}(M) = dx^0 \wedge dx^1 \wedge dx^2 \wedge dx^3$.

\subsubsection{Auf 1-Formen}
\begin{align*}
	\ast dx^0 
	&=
	s \, dx^1 \wedge dx^2 \wedge dx^3
	\\
	dx^0 \wedge \ast dx^0 
	&=
	dx^0 \wedge s \, dx^1 \wedge dx^2 \wedge dx^3 
	\\
	&=
	s \, dx^0 \wedge dx^1 \wedge dx^2 \wedge dx^3
	\\
	&=
	\underbrace{\langle dx^0, dx^0 \rangle}_{g^{00}} \, \operatorname{vol}(M) 
	\\
	&= - dx^0 \wedge dx^1 \wedge dx^2 \wedge dx^3 \Rightarrow s = -1
	\\
	\Rightarrow \ast dx^0 
	&=
	- dx^1 \wedge dx^2 \wedge dx^3
\\
\\
	\ast dx^1 
	&=
	s \, dx^0 \wedge dx^2 \wedge dx^3
	\\
	dx^1 \wedge \ast dx^1 
	&=
	dx^1 \wedge s \, dx^0 \wedge dx^2 \wedge dx^3 
	\\
	&=
	-s \, dx^0 \wedge dx^1 \wedge dx^2 \wedge dx^3
	\\
	&=
	\underbrace{\langle dx^1, dx^1 \rangle}_{g^{11}} \, \operatorname{vol}(M) 
	\\
	&=
	- dx^0 \wedge dx^1 \wedge dx^2 \wedge dx^3 \Rightarrow s = -1
	\\
	\Rightarrow \ast dx^1 
	&=
	- dx^0 \wedge dx^2 \wedge dx^3
\\
\\
	\ast dx^2 
	&=
	s \, dx^0 \wedge dx^1 \wedge dx^3
	\\
	dx^2 \wedge \ast dx^2 
	&=
	dx^2 \wedge s \, dx^0 \wedge dx^1 \wedge dx^3 
	\\
	&=
	s \, dx^0 \wedge dx^1 \wedge dx^2 \wedge dx^3
	\\
	&=
	\underbrace{\langle dx^2, dx^2 \rangle}_{g^{22}} \, \operatorname{vol}(M) 
	\\
	&=
	dx^0 \wedge dx^1 \wedge dx^2 \wedge dx^3 \Rightarrow s = 1
	\\
	\Rightarrow \ast dx^2 
	&=
	dx^0 \wedge dx^1 \wedge dx^3
\\
\\
	\ast dx^3 
	&=
	s \, dx^0 \wedge dx^1 \wedge dx^2
	\\
	dx^3 \wedge \ast dx^3 
	&=
	dx^3 \wedge s \, dx^0 \wedge dx^1 \wedge dx^2 
	\\
	&= -s \, dx^0 \wedge dx^1 \wedge dx^2 \wedge dx^3
	\\
	&=
	\underbrace{\langle dx^3, dx^3 \rangle}_{g^{33}} \, \operatorname{vol}(M) 
	\\
	&=
	dx^0 \wedge dx^1 \wedge dx^2 \wedge dx^3 \Rightarrow s = -1
	\\
	\Rightarrow \ast dx^3 
	&=
	-dx^0 \wedge dx^1 \wedge dx^2
\end{align*}

\subsubsection{Auf 2-Formen}
\begin{align*}
	\ast \underbrace{(dx^0 \wedge dx^1)}_{\omega}
	&=
	s \, dx^2 \wedge dx^3
	\\
	\omega \wedge \ast \omega
	&=
	dx^0 \wedge dx^1 \wedge s \, dx^2 \wedge dx^3
	\\
	&=
	s \, dx^0 \wedge dx^1 \wedge dx^2 \wedge dx^3
	\\
	\omega \wedge \ast \omega
	&=
	\langle \omega, \omega \rangle \, \operatorname{vol}(M)
	\\
	&=
	\langle dx^0 \wedge dx^1, dx^0 \wedge dx^1 \rangle \, \operatorname{vol}(M)
	\\
	&=
	g^{00} \cdot g^{11} \cdot \operatorname{vol}(M)
	\\
	&=
	-1 \cdot 1 \cdot \operatorname{vol}(M)
	\\
	&=
	-dx^0 \wedge dx^1 \wedge dx^2 \wedge dx^3 \Rightarrow s = -1
	\\
	\Rightarrow \ast(dx^0 \wedge dx^1)
	&=
	-dx^2 \wedge dx^3
\\
\\
	\ast \underbrace{(dx^1 \wedge dx^2)}_{\omega}
	&=
	s \, dx^0 \wedge dx^3
	\\
	\omega \wedge \ast \omega
	&=
	dx^1 \wedge dx^2 \wedge s \, dx^0 \wedge dx^3
	\\
	&=
	s \, dx^0 \wedge dx^1 \wedge dx^2 \wedge dx^3
	\\
	\omega \wedge \ast \omega
	&=
	\langle \omega, \omega \rangle \, \operatorname{vol}(M)
	\\
	&=
	\langle dx^1 \wedge dx^2, dx^1 \wedge dx^2 \rangle \, \operatorname{vol}(M)
	\\
	&=
	g^{11} \cdot g^{22} \cdot \operatorname{vol}(M)
	\\
	&=
	1 \cdot 1 \cdot \operatorname{vol}(M)
	\\
	&=
	dx^0 \wedge dx^1 \wedge dx^2 \wedge dx^3 \Rightarrow s = 1
	\\
	\Rightarrow \ast(dx^1 \wedge dx^2)
	&=
	dx^0 \wedge dx^3
\\
\\
	\ast \underbrace{(dx^2 \wedge dx^3)}_{\omega}
	&=
	s \, dx^0 \wedge dx^1
	\\
	\omega \wedge \ast \omega
	&=
	dx^2 \wedge dx^3 \wedge s \, dx^0 \wedge dx^1
	\\
	&=
	s \, dx^0 \wedge dx^1 \wedge dx^2 \wedge dx^3
	\\
	\omega \wedge \ast \omega
	&=
	\langle \omega, \omega \rangle \, \operatorname{vol}(M)
	\\
	&=
	\langle dx^2 \wedge dx^3, dx^2 \wedge dx^3 \rangle \, \operatorname{vol}(M)
	\\
	&=
	g^{22} \cdot g^{33} \cdot \operatorname{vol}(M)
	\\
	&=
	1 \cdot 1 \cdot \operatorname{vol}(M)
	\\
	&=
	dx^0 \wedge dx^1 \wedge dx^2 \wedge dx^3 \Rightarrow s = 1
	\\
	\Rightarrow \ast(dx^2 \wedge dx^3)
	&=
	dx^0 \wedge dx^1
\\
\\
	\ast \underbrace{(dx^0 \wedge dx^2)}_{\omega}
	&=
	s \, dx^1 \wedge dx^3
	\\
	\omega \wedge \ast \omega
	&=
	dx^0 \wedge dx^2 \wedge s \, dx^1 \wedge dx^3
	\\
	&=
	-s \, dx^0 \wedge dx^1 \wedge dx^2 \wedge dx^3
	\\
	\omega \wedge \ast \omega
	&=
	\langle \omega, \omega \rangle \, \operatorname{vol}(M)
	\\
	&=
	\langle dx^0 \wedge dx^2, dx^0 \wedge dx^2 \rangle \, \operatorname{vol}(M)
	\\
	&=
	g^{00} \cdot g^{22} \cdot \operatorname{vol}(M)
	\\
	&=
	-1 \cdot 1 \cdot \operatorname{vol}(M)
	\\
	&=
	-dx^0 \wedge dx^1 \wedge dx^2 \wedge dx^3 \Rightarrow s = 1
	\\
	\Rightarrow \ast(dx^0 \wedge dx^2)
	&=
	dx^1 \wedge dx^3
\\
\\
	\ast \underbrace{(dx^0 \wedge dx^3)}_{\omega}
	&=
	s \, dx^1 \wedge dx^2
	\\
	\omega \wedge \ast \omega
	&=
	dx^0 \wedge dx^3 \wedge s \, dx^1 \wedge dx^2
	\\
	&=
	s \, dx^0 \wedge dx^1 \wedge dx^2 \wedge dx^3
	\\
	\omega \wedge \ast \omega
	&=
	\langle \omega, \omega \rangle \, \operatorname{vol}(M)
	\\
	&=
	\langle dx^0 \wedge dx^3, dx^0 \wedge dx^3 \rangle \, \operatorname{vol}(M)
	\\
	&=
	g^{00} \cdot g^{33} \cdot \operatorname{vol}(M)
	\\
	&=
	-1 \cdot 1 \cdot \operatorname{vol}(M)
	\\
	&=
	-dx^0 \wedge dx^1 \wedge dx^2 \wedge dx^3 \Rightarrow s = -1
	\\
	\Rightarrow \ast(dx^0 \wedge dx^3)
	&=
	-dx^1 \wedge dx^2
\\
\\
	\ast \underbrace{(dx^1 \wedge dx^3)}_{\omega}
	&=
	s \, dx^0 \wedge dx^2
	\\
	\omega \wedge \ast \omega
	&=
	dx^1 \wedge dx^3 \wedge s \, dx^0 \wedge dx^2
	\\
	&=
	-s \, dx^0 \wedge dx^1 \wedge dx^2 \wedge dx^3
	\\
	\omega \wedge \ast \omega
	&=
	\langle \omega, \omega \rangle \, \operatorname{vol}(M)
	\\
	&=
	\langle dx^1 \wedge dx^3, dx^1 \wedge dx^3 \rangle \, \operatorname{vol}(M)
	\\
	&=
	g^{11} \cdot g^{33} \cdot \operatorname{vol}(M)
	\\
	&=
	1 \cdot 1 \cdot \operatorname{vol}(M)
	\\
	&=
	dx^0 \wedge dx^1 \wedge dx^2 \wedge dx^3 \Rightarrow s = -1
	\\
	\Rightarrow \ast(dx^1 \wedge dx^3)
	&=
	-dx^0 \wedge dx^2
\end{align*}
\subsubsection{Auf 3-Formen}
\begin{align*}
	\ast \underbrace{dx^0 \wedge dx^1 \wedge dx^2}_{\omega}
	&=
	s \, dx^3
	\\
	\omega \wedge \ast \omega 
	&=
	dx^0 \wedge dx^1 \wedge dx^2 \wedge s \, dx^3
	\\
	&=
	s \, dx^0 \wedge dx^1 \wedge dx^2 \wedge dx^3
	\\
	\omega \wedge \ast \omega
	&=
	\langle dx^0 \wedge dx^1 \wedge dx^2 , dx^0 \wedge dx^1 \wedge dx^2 \rangle \, \operatorname{vol}(M)
	\\
	&=
	g^{00} \cdot g^{11} \cdot g^{22} \cdot \operatorname{vol}(M)
	\\
	&=
	-1 \cdot 1 \cdot 1 \cdot \operatorname{vol}(M) \Rightarrow s = -1
	\\
	\Rightarrow \ast (dx^0 \wedge dx^1 \wedge dx^2) 
	&= - dx^3
\\
\\
	\ast \underbrace{dx^0 \wedge dx^1 \wedge dx^3}_{\omega}
	&=
	s \, dx^2
	\\
	\omega \wedge \ast \omega 
	&=
	dx^0 \wedge dx^1 \wedge dx^3 \wedge s \, dx^2
	\\
	&=
	-s \, dx^0 \wedge dx^1 \wedge dx^2 \wedge dx^3
	\\
	\omega \wedge \ast \omega
	&=
	\langle dx^0 \wedge dx^1 \wedge dx^3 , dx^0 \wedge dx^1 \wedge dx^3 \rangle \, \operatorname{vol}(M)
	\\
	&=
	g^{00} \cdot g^{11} \cdot g^{33} \cdot \operatorname{vol}(M)
	\\
	&=
	-1 \cdot 1 \cdot 1 \cdot \operatorname{vol}(M) \Rightarrow s = 1
	\\
	\Rightarrow \ast (dx^0 \wedge dx^1 \wedge dx^3) 
	&= dx^2
\\
\\
	\ast \underbrace{dx^0 \wedge dx^2 \wedge dx^3}_{\omega}
	&=
	s \, dx^1
	\\
	\omega \wedge \ast \omega 
	&=
	dx^0 \wedge dx^2 \wedge dx^3 \wedge s \, dx^1
	\\
	&=
	s \, dx^0 \wedge dx^1 \wedge dx^2 \wedge dx^3
	\\
	\omega \wedge \ast \omega
	&=
	\langle dx^0 \wedge dx^2 \wedge dx^3 , dx^0 \wedge dx^2 \wedge dx^3 \rangle \, \operatorname{vol}(M)
	\\
	&=
	g^{00} \cdot g^{22} \cdot g^{33} \cdot \operatorname{vol}(M)
	\\
	&=
	-1 \cdot 1 \cdot 1 \cdot \operatorname{vol}(M) \Rightarrow s = -1
	\\
	\Rightarrow \ast (dx^0 \wedge dx^2 \wedge dx^3) 
	&= - dx^1
\\
\\
	\ast \underbrace{dx^1 \wedge dx^2 \wedge dx^3}_{\omega}
	&=
	s \, dx^0
	\\
	\omega \wedge \ast \omega 
	&=
	dx^1 \wedge dx^2 \wedge dx^3 \wedge s \, dx^0
	\\
	&=
	-s \, dx^0 \wedge dx^1 \wedge dx^2 \wedge dx^3
	\\
	\omega \wedge \ast \omega
	&=
	\langle dx^1 \wedge dx^2 \wedge dx^3 , dx^1 \wedge dx^2 \wedge dx^3 \rangle \, \operatorname{vol}(M)
	\\
	&=
	g^{11} \cdot g^{22} \cdot g^{33} \cdot \operatorname{vol}(M)
	\\
	&=
	1 \cdot 1 \cdot 1 \cdot \operatorname{vol}(M) \Rightarrow s = -1
	\\
	\Rightarrow \ast (dx^1 \wedge dx^2 \wedge dx^3) 
	&= - dx^0
\end{align*}
%$\ast dx^0 = - dx^1 \wedge dx^2 \wedge dx^3$\\
%$\ast dx^1 = - dx^0 \wedge dx^2 \wedge dx^3$\\
%$\ast dx^2 = dx^0 \wedge dx^1 \wedge dx^3$\\
%$\ast dx^3 = -dx^0 \wedge dx^1 \wedge dx^2$\\\\
%$\ast(dx^0 \wedge dx^1) = -dx^2 \wedge dx^3$\\
%$\ast(dx^1 \wedge dx^2) = dx^0 \wedge dx^3$\\
%$\ast(dx^2 \wedge dx^3) = dx^0 \wedge dx^1$\\
%$\ast(dx^0 \wedge dx^2) = dx^1 \wedge dx^3$\\
%$\ast(dx^0 \wedge dx^3) = -dx^1 \wedge dx^2$\\
%$\ast(dx^1 \wedge dx^3) = -dx^0 \wedge dx^2$\\\\
%$\ast (dx^0 \wedge dx^1 \wedge dx^2) = -dx^3$\\
%$\ast (dx^0 \wedge dx^1 \wedge dx^3) = dx^2$\\
%$\ast (dx^0 \wedge dx^2 \wedge dx^3) = -dx^1$\\
%$\ast (dx^1 \wedge dx^2 \wedge dx^3) = -dx^0$\\
In der folgenden Tabelle sind alle berechneten Hodge-Duale noch einmal zusammengefasst:

%\begin{table}[htbp]
\begin{center}
	\begin{tabularx}{\textwidth}{ 
			| >{\centering\arraybackslash}X 
			| >{\centering\arraybackslash}X 
			| >{\centering\arraybackslash}X | }
		\hline
		\textbf{1-Form} & \textbf{2-Form} & \textbf{3-Form} \\
		\hline
		\( *dx^0 = -dx^1 \wedge dx^2 \wedge dx^3 \) \newline
		\( *dx^1 = -dx^0 \wedge dx^2 \wedge dx^3 \) \newline
		\( *dx^2 =  dx^0 \wedge dx^1 \wedge dx^3 \) \newline
		\( *dx^3 = -dx^0 \wedge dx^1 \wedge dx^2 \) 
		&
		\( *(dx^0 \wedge dx^1) = -dx^2 \wedge dx^3 \) \newline
		\( *(dx^1 \wedge dx^2) = dx^0 \wedge dx^3 \) \newline
		\( *(dx^2 \wedge dx^3) = dx^0 \wedge dx^1 \) \newline
		\( *(dx^0 \wedge dx^2) = dx^1 \wedge dx^3 \) \newline
		\( *(dx^0 \wedge dx^3) = -dx^1 \wedge dx^2 \) \newline
		\( *(dx^1 \wedge dx^3) = -dx^0 \wedge dx^2 \)
		&
		\( *(dx^0 \wedge dx^1 \wedge dx^2) = -dx^3 \) \newline
		\( *(dx^0 \wedge dx^1 \wedge dx^3) = dx^2 \) \newline
		\( *(dx^0 \wedge dx^2 \wedge dx^3) = -dx^1 \) \newline
		\( *(dx^1 \wedge dx^2 \wedge dx^3) = -dx^0 \)
		\\
		\hline
	\end{tabularx}
\end{center}
%\caption{Hodge-Duale von 1-, 2-, und 3-Formen}
%\label{tab:Tabelle Hodge-Duale}
%\end{table}





