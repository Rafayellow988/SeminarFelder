%
% teil2.tex -- Beispiel-File für teil2 
%
% (c) 2020 Prof Dr Andreas Müller, Hochschule Rapperswil
%
% !TEX root = ../../buch.tex
% !TEX encoding = UTF-8
%
\section{Richardsons Ansatz \label{geostrophisch:section:richardsonAnsatz}}
\kopfrechts{Teil 2}

Richardson wählte einen innovativen, aber im Grunde sehr einfachen Ansatz:  
Statt vereinfachter Modelle verwendete er ein vollständiges System bestehend aus vier Differentialgleichungen.
Die erste Gleichung:
\begin{equation}
	\frac{D\vec{v}}{Dt} + 2\vec{\Omega} \times \vec{v} = -\frac{1}{\rho}\nabla p + \vec{g} \\,
	\label{eq:navstok}
\end{equation}
ist eine Form der Navier-Stokes-Gleichung für rotierende Bezugssysteme.
Wobei $\frac{D\vec{v}}{Dt}$ die zeitliche Ableitung aller Geschwindigkeitskomponenten in Nord-, Ost- und Vertikalrichtungen entspricht. 
Sie beschreibt die Beschleunigung von Luftmassen und ausserdem werden darin die Corioliskraft und der Druckgradient berücksichtigt.
Als nächstes verwendete er die Kontinuitätsgleichung:
\begin{equation}
	\frac{\partial \rho}{\partial t} + \nabla \cdot (\rho \vec{v}) = 0 \\,
	\label{eq:kont}
\end{equation}
welche die Massenerhaltung beschreibt.
Also in anderen Worten die Luft kann nicht verschwinden oder aus dem nichts auftauchen. 
Zudem brauchte er noch eine Gleichung, welche die Energie beschreibt, dazu diente 
\begin{equation}
	\frac{Ds}{Dt} = Q \\.
	\label{eq:enrgy}
\end{equation}
Damit werden Temperaturänderungen durch Transport sowie Expansion und Kompression der Luftmassen ausgedrückt.
Als letztes verwendete er noch das ideale Gasgesetz:
\begin{equation}
	p = \rho R T,
	\label{eq:gasgesetz}
\end{equation}
dieses ist die physikalische Verbindung von Druck, Temperatur und Dichte eines Gases.  

Er konstruierte eine schachbrettartiges Gitter und legte es über die Karte von Europa (Abbildung~\ref{bild:karteEuropa}).
Nun wollte er für jeden dieser Gitterpunkte das beschriebene Gleichungssystem numerisch von Hand lösen.
Für einen alleine dauerten diese Berechnungen sehr lange und somit brauchte er für die Vorhersage viel zu lange.
Aufgrund dessen hatte Richardson eine Traumvorstellung von einer Einrichtung zur Berechnung der Wettervorhersagen für die ganze Welt.
Diese besteht aus einer grossen Kuppel, welche im Innern eine Weltkarte an der Wand hat. 
Darin sitzen rundherum etwa 64000 Leute, die je zu ihrem zugewiesenen Bereich die Daten der Atmosphäre erhalten und damit die Berechnungen machen müssen. 
Ihre Resultate geben sie an die Benachbarten Mitarbeiter weiter und das grosse Rechnen beginnt von vorn. 
In der Mitte auf dem Podest steht eine Art \glqq Dirigent\grqq, welcher den Leuten die zu langsam oder schnell sind mittels einer Lampe Signale gibt.
Siehe Abbildung ~\ref{bild:richardsonsTraum}, eine Illustration von Richardsons Traumvorstellung.  

\begin{figure}
	\centering
	\includegraphics{eingeteilte_Karte.jpg}
	\caption{Gitter zur numerischen Approximation der Felder, welche das Wetter beeinflussen.
		Wie man sehen kann ist das Gitter sehr grob und somit ungenau.}
	\label{bild:karteEuropa}
\end{figure}

\begin{figure}
	\centering
	\includegraphics{Richardsons_Traum.jpg}
	\caption{Traumvorstellung von Richardson einer Station zur Wettervorhersage. 
	Rundherum hängt eine Weltkarte und Leute berechnen ständig die Atmosphäre zu ihrem Bereich.}
	\label{bild:richardsonsTraum}
\end{figure}

\subsection{Falsche erste Vorhersage}

Wie schon erwähnt war seine aller erste Vorhersage aus dem Buch \cite{geostrophisch:wpbnp} weit von dem realen Wetter entfernt.
Weshalb war seine erste Vorhersage so falsch? 
Sein scheitern lag bestimmt nicht an mangelndem physikalischen Verständnis oder fehlerhafter Idee. 
Sondern viel mehr an praktischen Problemen.
Seine Anfangsdaten waren nicht im geostrophischen Gleichgewicht. 
% Hier Weiter
Sein Gitter war viel zu Grob aufgebaut. 
Für die ganze Schweiz war nur ein Gitterpunkt eingebaut, also soll in der ganzen Schweiz das gleiche Wetter herrschen? 
Das ist offensichtlich nicht der Fall, durch die Alpen gibt es oft starke Unterschiede der Witterung im Vergleich zum Flachland.
 



