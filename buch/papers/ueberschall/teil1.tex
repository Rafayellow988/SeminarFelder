%
% teil2.tex -- Beispiel-File für teil2 
%
% (c) 2020 Prof Dr Andreas Müller, Hochschule Rapperswil
%
% !TEX root = ../../buch.tex
% !TEX encoding = UTF-8
%
\section{Potentialströmung\label{ueberschall:section:potentialstroemung}}
\kopfrechts{Potentialströmung}
Bei der Potentialströmung handelt sich um eine stationäre, 
inkompressible und wirbelfreie Strömung.
Anders formuliert
\begin{align}
    \frac{\partial \vec{v}}{\partial t} &= 0 \\
    \mathrm{div}\,\vec{v} &= 0 \\
    \mathrm{rot}\,\vec{v} &= 0 \\
    \nabla \cdot \vec{v} &= 0 \\
    \nabla \times \vec{v} &= 0\label{equation:inkompressibel}
\end{align}
wobei $\vec{v}$ die Geschwindigkeit der Strömung ist.
Das Skalarprodukt der Geschwindigkeit mit dem Nabla-Operator $\nabla \cdot \vec{v}$ 
wird auch bezeichnet als Divergenz der Geschwindigkeit
\begin{equation}
    \mathrm{div}\,(\vec{v})
    =
    \frac{\partial v_x}{\partial x} +
    \frac{\partial v_y}{\partial y} +
    \frac{\partial v_z}{\partial z}.
\end{equation}
Diese gibt an ob das Feld wie eine Quelle oder eine Senke verhält.
Bei $\mathrm{div}\,\vec{v} = 0$ ist sie weder Quelle noch Senke.
Das Kreuzprodukt mit dem Operator entspricht
der Rotation der Geschwindigkeit
\begin{align}
    \mathrm{rot}\,(\vec{v})
    =
    \begin{pmatrix}
        \frac{\partial v_z}{\partial y} - \frac{\partial v_y}{\partial z} \\
        \frac{\partial v_x}{\partial z} - \frac{\partial v_z}{\partial x} \\
        \frac{\partial v_y}{\partial x} - \frac{\partial v_x}{\partial y}
    \end{pmatrix},
\end{align}
welcher wie das Wort schon sagt die Rotation eines Feldes angibt.
Was für ein wirbelfreies Feld $\mathrm{rot}\,(\vec{v}) = 0$ bedeutet.

Zudem gilt für eine Potentialströmung mit dem Potential $\varphi$
\begin{equation}
    \vec{v} = \nabla\varphi,
\end{equation}
wobei $\nabla\varphi$ auch als Gradient des Potential bezeichnet wird
und bedeutet nichts geringeres als die Ableitung in jede Richtung
\begin{align}
    \nabla\varphi
    =
    \begin{pmatrix}
        \frac{\partial \varphi_x}{\partial x}\\
        \frac{\partial \varphi_y}{\partial y}\\
        \frac{\partial \varphi_z}{\partial z}
    \end{pmatrix}.
\end{align} 
Eingesetzt in die Bedingung der Inkompressibilität~\ref{equation:inkompressibel}
führt das zu
\begin{equation}
    \nabla \cdot \nabla \varphi
    = 
    \nabla^2 \varphi
    =
    \Delta\varphi,
\end{equation}
auch bezeichnet als Laplace Gleichung mit dem Laplace-Operator $\Delta$.



\subsection{Unterschallgebiet\label{ueberschall:subsection:unterschallgebiet}}
\begin{equation}
    \Phi
    =
    Ux + A \frac{l}{2\pi}
    \sin(\frac{2\pi x}{l})
    e^{-\frac{2\pi y}{l}}
\end{equation}
wobei $U$ die ungestörte Geschwindigkeit im Unendlichen bedeutet.
test