

Schon im frühen 20. Jahrhundert begannen Meteorolog:innen zu erkennen, dass grossräumige Strömungen in der Atmosphäre nicht nur durch Temperaturunterschiede, sondern auch durch die Erdrotation beeinflusst werden \cite{rossby:https://doi.org/10.1002/wcc.95}.
\index{Erdrotation}%
Eine Schlüsselfigur in diesem Zusammenhang war Carl-Gustaf Rossby, ein schwedisch-amerikanischer Meteorologe.
\index{Rossby, Carl-Gustaf}%
In seiner Arbeit von 1939 untersuchte er den Zusammenhang zwischen der Intensität der zonalen Zirkulation und der Verschiebung der sogenannten permanenten Aktionszentren in der Atmosphäre \cite{rossby:1939relation}.
\index{Zirkulation}%
\index{Aktionszentren, permanente}%
Ein Jahr später veröffentlichte er seine grundlegende Theorie der planetaren Wellenmuster in der Atmosphäre, die heute als \emph{Rossby-Wellen} bekannt sind \cite{rossby:1940planetary}.
\index{planetare Wellen}%
\index{Rossby-Wellen}%
Diese Wellen sind heute ein zentrales Konzept in der Dynamik der Atmosphäre und tragen wesentlich zum Verständnis des Wetters in den mittleren Breiten bei.
\index{Dynamik}%

Rossby erkannte, dass die scheinbaren Kräfte, die durch die Rotation der Erde entstehen, insbesondere der \emph{Coriolis-Effekt}, eine wellenartige Bewegung grossräumiger atmosphärischer Strömungen erzeugen können.
\index{Coriolis-Effekt}%
Diese wellenförmigen Muster bewegen sich typischerweise von West nach Ost und beeinflussen massgeblich Hoch- und Tiefdrucksysteme sowie den Verlauf von Wetterfronten über Tage hinweg.
\index{Hochdrucksystem}%
\index{Tiefdrucksystem}%
\index{Wetterfront}%

Die Untersuchung dieser Prozesse erfordert ein Verständnis physikalischer Konzepte wie der \emph{Vorticity} (Wirbelstärke), der \emph{absoluten} und \emph{potenziellen Vorticity} sowie deren Erhaltung unter bestimmten Bedingungen.
\index{Vorticity}%
\index{Wirbelstärke}%
\index{Vorticity!absolut}%
\index{Vorticity!potenziell}%
Diese Begriffe ermöglichen es, die Bewegung von Luftpaketen auf einem rotierenden Planeten mathematisch zu beschreiben und zu erklären, warum Rossby-Wellen entstehen und wie sie sich ausbreiten.

Ziel dieser Arbeit ist es, die Grundlagen der atmosphärischen Dynamik zu skizzieren, die zur Entstehung von Rossby-Wellen führen.
Beginnend mit der Erddrehung und dem Coriolis-Effekt, werden die Begriffe der Vorticity eingeführt und schrittweise zur potenziellen Vorticity erweitert.
Die Erhaltung dieser Grösse stellt den Ausgangspunkt für die mathematische Herleitung und das physikalische Verständnis von Rossby-Wellen dar.
Abschliessend wird ihre Rolle in typischen Wetterphänomenen beleuchtet.


\subsection{Rossby-Wellen im Wettergeschehen}

Zur Einführung betrachten wir zunächst ein einzelnes Kartenbeispiel
(Abbildung~\ref{fig:rossby_atlantic_single}), das als Referenz für die Interpretation der
folgenden Abbildungen dient.
Die Darstellung basiert auf ERA5-Reanalysedaten des ECMWF (abgerufen über die CDS~API)
\index{ERA5}%
\index{ECMWF}%
und zeigt die atmosphärische Situation am 1.\ Mai~2025 um 00:00~UTC auf dem Druckniveau
von 500\,hPa, das sich in der oberen Troposphäre im Bereich des Jetstreams befindet.
\index{Troposphäre}%


Die Karte nutzt eine Lambert-Conformal-Kartenprojektion mit Mittelmeridian \(-30^\circ\), zentriert auf den Atlantischen Ozean.
\index{Lambert-Conformal}%
Der dargestellte Ausschnitt reicht von Nordamerika über den Atlantik bis nach Europa (\(-90^\circ\) bis \(30^\circ\) Länge, \(20^\circ\) bis \(75^\circ\) Breite).
\index{Nordamerika}%
\index{Atlantik}%
\index{Europa}%
\index{Afrika}%

Inhaltlich zeigt die Karte:
\begin{itemize}
	\item \emph{Geopotentielle Höhenlinien} (in Metern, schwarze Linien): verbinden Punkte gleichen Druckniveaus, geben Auskunft über die grossräumige Druck- und Temperaturverteilung.
\index{geopotentielle Höhe}%
	\item \emph{Farbflächen}: Windgeschwindigkeit in m/s.
\index{Windgeschwindigkeit}%
	\item \emph{Pfeile}: Windvektoren, deren Richtung und Länge Strömungsrichtung und -ge\-schwin\-dig\-keit wiedergeben.
\end{itemize}

Das wellenförmige Muster der Höhenlinien und Windfelder ist charakteristisch für Rossby-Wellen, die sich von West nach Ost ausbreiten und mit Hoch- und Tiefdrucksystemen verknüpft sind.

\begin{figure}
	\centering
	\includegraphics[width=\textwidth, trim=2cm 0cm 3cm 0cm, clip]{papers/rossby/images/data_2025_5_1_00_00_500.jpg}
	\caption{Beispielkarte der Rossby-Wellenstruktur in 500\,hPa Höhe am 1.\ Mai 2025, 00:00~UTC.
		Projektion, Ausschnitt und inhaltliche Elemente wie im Text beschrieben.}
	\label{fig:rossby_atlantic_single}
\end{figure}

Nachdem das Beispielbild in Abbildung~\ref{fig:rossby_atlantic_single} die
Darstellungselemente erläutert, betrachten wir nun eine Abfolge vom 1.\ bis 5.\ Mai~2025
(Abbildung~\ref{fig:rossby_grid}), um die zeitliche Entwicklung der Rossby-Wellen
nachzuvollziehen. Die Karten sind in 12-Stunden-Schritten angeordnet, sodass die
wellenförmige Verlagerung und Verstärkung der Systeme deutlich erkennbar wird.


\begin{figure}
	\centering
	\includegraphics[width=\textwidth]{papers/rossby/images/rossby_2025.pdf}
	\caption{Abfolge der Rossby-Wellenstruktur in 500\,hPa Höhe vom 1.\ bis 5.\ Mai 2025 in 12-Stunden-Schritten.
		Die Teilabbildungen sind zeilenweise von links nach rechts zu lesen, beginnend mit dem 1.\ Mai 00:00~UTC (oben links)
		bis zum 5.\ Mai 00:00~UTC (unten rechts).}
	\label{fig:rossby_grid}
\end{figure}

