%
% main.tex -- Paper zum Thema <rossby>
%
% (c) 2020 Autor, OST Ostschweizer Fachhochschule
%
% !TEX root = papers/rossby/papers/rossby/buch.tex
% !TEX encoding = UTF-8
% !TeX spellcheck = de-CH

\chapter{Rossby-Wellen\label{chapter:rossby}}
\kopflinks{Rossby-Wellen}
\begin{refsection}
	\chapterauthor{Michael Schmid}
\index{Michael Schmid}%
\index{Schmid, Michael}%

    \section{Einleitung}
    %
% einleitung.tex -- Einleitung und Motivation
%
% (c) 2020 Prof Dr Andreas Müller, Hochschule Rapperswil
%
% !TEX root = ../../buch.tex
% !TEX encoding = UTF-8
%

\section{Einleitung\label{fourier:section:einleitung}}
%\kopfrechts{Grundlagen der Fourier-Analyse}


Felder kann man sich als eine unendliche Ansammlung von Feldlinien vorstellen, die sich kreuz und quer durch Raum-Zeit schlängeln. Die Fourier-Analyse hilft dabei, dieses scheinbare Chaos in einfache Sinus- und Cosinus-Schwingungen zu zerlegen. 
In der Quantenfeldtheorie werden diese Schwingungen anders interpretiert, was letztlich dazu führt, dass das Feld als Summe einer beschränkten Anzahl von Teilen verstanden wird.










    \section{Grundlagen}
    
\subsection{Idealisiertes zonales und meridionales Windfeld}

Abbildung~\ref{fig:zonal_wind} zeigt ein idealisiertes zonales Windfeld, bei dem ausschliesslich eine Ost–West-Strömung vorliegt (\(v = 0\)).
Die Geschwindigkeit hängt von der geographischen Breite \(\theta\) ab und folgt der Beziehung
\begin{equation}
	u = U_0 \cdot \sin^2(\theta).
\label{eq:zonal_wind}
\end{equation}
Ein Druckgradient oder eine vertikale Struktur sind nicht berücksichtigt – es handelt sich um ein rein theoretisches, horizontal homogenes Modell.

\begin{figure}
	\centering
	\includegraphics[width=0.6\linewidth]{papers/rossby/images/zonal_wind_plot.pdf}
	\caption{Idealisiertes zonales Windfeld (\(v=0\)), Geschwindigkeit \(u = U_0 \cdot \sin^2(\theta)\).}
	\label{fig:zonal_wind}
\end{figure}

\noindent
Abbildung~\ref{fig:meridional_wind} zeigt dagegen ein idealisiertes meridionales Windfeld, bei dem nur eine Nord–Süd-Strömung existiert (\(u = 0\)).
Die Geschwindigkeit hängt hier von der Breite ab nach
\begin{equation}
	v = V_0 \cdot \cos(\theta).
\label{eq:meridional_wind}
\end{equation}
Auch dieses Feld ist frei von Druckgradienten und vertikaler Struktur und dient der isolierten Betrachtung meridionaler Strömungskomponenten.

\begin{figure}
	\centering
	\includegraphics[width=0.6\linewidth]{papers/rossby/images/meridional_wind_plot.pdf}
	\caption{Idealisiertes meridionales Windfeld (\(u=0\)), Geschwindigkeit \(v = V_0 \cdot \cos(\theta)\).}
	\label{fig:meridional_wind}
\end{figure}

\subsection{Die Drehung der Erde und die Corioliskraft}

Die Erde rotiert einmal pro Tag um ihre eigene Achse. In einem rotierenden Bezugssystem wie der Erde treten dabei Scheinkräfte auf, die in der klassischen Mechanik berücksichtigt werden müssen. Eine davon ist die {Corioliskraft}, welche bewegte Luft- und Wassermassen auf der {Nordhalbkugel} nach rechts und auf der {Südhalbkugel} nach links ablenkt. Ihre Wirkung ist maximal an den Polen und verschwindet am Äquator. Die Corioliskraft ist keine reale Kraft, sondern eine Folge der Drehung des Koordinatensystems.

Mathematisch wird sie beschrieben durch
\begin{equation}
	\vec{F}_C = -2m(\vec{\Omega} \times \vec{v}),
\label{eq:coriolis_force}
\end{equation}
wobei \(m\) die Masse, \(\vec{\Omega}\) der Rotationsvektor der Erde und \(\vec{v}\) die Geschwindigkeit relativ zur Erdoberfläche ist. In atmosphärischen Anwendungen wird häufig nur die Vertikalkomponente betrachtet, was zur Definition des {Coriolis-Parameters}
\begin{equation}
	f = 2\Omega \sin(\varphi)
\label{eq:coriolis_parameter}
\end{equation}
führt. Dessen Breitenänderung ist durch den \(\beta\)-Parameter gegeben:
\begin{equation}
	\beta = \frac{\partial f}{\partial y} = \frac{2 \Omega \cos(\varphi)}{a},
\label{eq:beta_parameter}
\end{equation}
mit dem Erdradius \(a\) und der Nord–Süd-Koordinate \(y\).

\paragraph{Rechenbeispiel: Velofahren in Zürich}

Wir berechnen die Corioliskraft auf einen Radfahrer in Zürich.

\begin{description}
	\item[Gegeben:] Geschwindigkeit \(v = 8.33\,\mathrm{m/s}\) (30\,km/h), Masse \(m = 80\,\mathrm{kg}\), geographische Breite \(\varphi = 47^\circ\) (Zürich), Erdrotationsrate \(\Omega \approx 7.292 \times 10^{-5}\,\mathrm{rad/s}\).
	\item[Formel:]
		\[
			F_C = 2\,m\,v\,\Omega\,\sin(\varphi)
		\]
	\item[Einsetzen:]
		\[
			F_C \approx 2 \cdot 80 \cdot 8.33 \cdot 7.292 \times 10^{-5} \cdot \sin(47^\circ)
			\approx 0.070\,\mathrm{N}
		\]
\end{description}

\noindent
Die resultierende Kraft von rund \(0.07\,\mathrm{N}\) ist für uns im Alltag kaum spürbar, prägt jedoch die grossräumige Atmosphärendynamik.
Sie bewirkt, dass Luftmassen nicht direkt vom Hoch- zum Tiefdruckgebiet strömen, sondern entlang der Isobaren gelenkt werden.
Dieser Effekt ist massgeblich für die Westwinddrift in den mittleren Breiten und bildet die physikalische Grundlage für Rossby-Wellen.

\subsection{Die \texorpdfstring{$\beta$}{β}-Ebene Approximation}

Die $\beta$-Ebene ist eine lokale Approximation der Erdkugel in der Umgebung einer bestimmten geographischen Breite~$\phi_0$
(Abbildung~\ref{fig:beta_plane}).
Ziel dieser Vereinfachung ist es, die Breitenabhängigkeit der Corioliskraft in mathematischen Modellen handhabbar zu machen.
\begin{equation}
	f(y) = f_0 + \beta y,
\label{eq:beta_plane}
\end{equation}
wobei $y$ die meridionale (Nord–Süd-)Entfernung vom Referenzbreitenkreis ist. Der Term
\begin{equation}
	f_0 = 2\Omega \sin(\phi_0)
\label{eq:coriolis_parameter_ref}
\end{equation}
bezeichnet den Coriolisparameter an der Referenzbreite, und
\begin{equation}
	\beta = \left.\frac{\partial f}{\partial y}\right|_{\phi_0} = \frac{2\Omega \cos(\phi_0)}{a}
\label{eq:beta_parameter_ref}
\end{equation}

beschreibt die Änderung von $f$ mit der geographischen Breite; $a$ ist der Erdradius. Diese Darstellung erlaubt es, lokale Phänomene wie Rossby-Wellen auf einer idealisierten Ebene zu analysieren, ohne die volle Kugelgeometrie der Erde berücksichtigen zu müssen.

\begin{figure}
	\centering
	\begin{tikzpicture}[scale=2]
		% Tilted group (Earth and axes)
		\begin{scope}[rotate around={23.5:(0,0)}]
			% Axes
			\draw (-2,0) -- (2,0);
			\draw[->] (0,-2) -- (0,2) node[above] {Pole};

			% Earth (circle)
			\draw[thick] (1,0) arc[start angle=0, end angle=360, radius=1];

			% 3D-style spinning arrow (elliptical path)
			\draw[->, thick] ({0.3*cos(10)},{1.5 + 0.1*sin(10)})
			arc[start angle=10, end angle=320, x radius=0.3, y radius=0.1];

			% Point on the circle
			\fill[red] (30:1) circle[radius=0.05] node[right] {$P$};
			\draw[blue, thick] ($(30:1) + (120:0.5)$) -- ($(30:1) + (-60:0.5)$);
		\end{scope}
	\end{tikzpicture}
	\caption{Schematische Darstellung der $\beta$-Ebene-Approximation um einen Referenzbreitenkreis.
		Der rote Punkt \(P\) markiert die Referenzbreite, in deren Umgebung die Erdoberfläche als tangentiale Ebene angenähert wird.
		Die blaue Linie zeigt den Breitenkreis durch \(P\), der Pfeil die Rotationsrichtung der Erde.}
	\label{fig:beta_plane}
\end{figure}
    \section{Vorticity}
    
\subsection{Vorticity – Wirbelstärke in der Atmosphäre}

Die grossräumige Bewegung der Luft in der Atmosphäre lässt sich nicht nur durch
Geschwindigkeit beschreiben, sondern auch durch ihre Rotationseigenschaften.
Diese Eigenschaft wird durch die sogenannte \emph{Vorticity} (Wirbelstärke)
quantifiziert. Physikalisch beschreibt sie die Tendenz eines Luftpakets, sich
um seine eigene vertikale Achse zu drehen.

Mathematisch ist die Vorticity definiert als Rotation des
Geschwindigkeitsfeldes:
\begin{equation}
	\vec{\zeta} = \nabla \times \vec{u}.
	\label{rossby:eq:vorticity}
\end{equation}
Für grossräumige Strömungen in der Atmosphäre betrachtet man hauptsächlich die vertikale Komponente dieser Rotation, die sogenannte \emph{relative Vorticity}.
In kartesischen Koordinaten ergibt sich diese im einfachsten Fall zu:
\begin{equation}
	\zeta = \frac{\partial v}{\partial x} - \frac{\partial u}{\partial y},
	\label{rossby:eq:relative_vorticity}
\end{equation}
wobei \(u\) und \(v\) die zonale bzw. meridionale Windkomponente bezeichnen.
Positive Vorticity (\(\zeta > 0\)) steht für eine zyklonale Rotation (auf der Nordhalbkugel gegen den Uhrzeigersinn), negative Vorticity (\(\zeta < 0\)) für eine antizyklonale Rotation (im Uhrzeigersinn).
\begin{figure}
	\centering
	% Erste Reihe
	\begin{minipage}{0.32\linewidth}
		\centering
		\includegraphics[width=\linewidth]{papers/rossby/images/vorticity_plot0.pdf}\\
		{\small \( \vec{u} = (2,0),\ \zeta = 0\)}
	\end{minipage}
	\begin{minipage}{0.32\linewidth}
		\centering
		\includegraphics[width=\linewidth]{papers/rossby/images/vorticity_plot1.pdf}\\
		{\small \( \vec{u} = (y,0),\ \zeta = -1\)}
	\end{minipage}
	\begin{minipage}{0.32\linewidth}
		\centering
		\includegraphics[width=\linewidth]{papers/rossby/images/vorticity_plot2.pdf}\\
		{\small \( \vec{u} = (y^2,0),\ \zeta = -2y\)}
	\end{minipage}

	% Zweite Reihe
	\begin{minipage}{0.32\linewidth}
		\centering
		\includegraphics[width=\linewidth]{papers/rossby/images/vorticity_plot3.pdf}\\
		{\small \( \vec{u} = (-y,x),\ \zeta = 2\)}
	\end{minipage}
	\begin{minipage}{0.32\linewidth}
		\centering
		\includegraphics[width=\linewidth]{papers/rossby/images/vorticity_plot4.pdf}\\
		{\small \( \vec{u} = (y,-x),\ \zeta = -2\)}
	\end{minipage}

	\caption{Beispiele für verschiedene Vorticity-Fälle in idealisierten Strömungsfeldern.}
	\label{fig:vorticity_examples}
\end{figure}

\subsection{Absolute Vorticity}

Die bisher betrachtete relative Vorticity beschreibt die Rotation eines
Luftpakets relativ zum Erdboden. Da sich jedoch auch die Erde selbst dreht,
muss diese planetare Rotation bei der Beschreibung der Gesamtdrehung
berücksichtigt werden. Daraus ergibt sich der Begriff der \emph{absoluten
	Vorticity}, welche die Summe aus relativer und planetarer Vorticity darstellt.

Die planetare Vorticity wird durch den \emph{Coriolis-Parameter} \(
f = 2 \Omega \sin \varphi \) beschrieben, wobei \( \Omega \) die
Erdrotationsrate und \( \varphi \) die geographische Breite ist. Dieser
Ausdruck ergibt sich aus der Projektion der Erdrotation auf die lokale
Vertikale. Die absolute Vorticity ergibt sich somit zu:
\begin{equation}
	\eta = \zeta + f
	\label{rossby:eq:absolute_vorticity}
\end{equation}
wobei \( \zeta \) die relative und \( f \) die planetare Vorticity ist.

% Die absolute Vorticity ist eine Erhaltungsgrösse in der reibungsfreien, barotropen Atmosphäre, wenn keine vertikalen Bewegungen stattfinden.
% Sie spielt daher eine zentrale Rolle in der dynamischen Meteorologie und bildet die Grundlage für das Konzept der \emph{potenziellen Vorticity}, welches zusätzlich die Schichtung der Atmosphäre berücksichtigt.

\subsection{Potenzielle Vorticity und ihre Erhaltung}

Die \emph{potenzielle Vorticity} erweitert das Konzept der \emph{absoluten
	Vorticity} um die vertikale Struktur der Atmosphäre. In ihrer Erhaltungsform
ist sie ein zentrales Werkzeug zur Beschreibung grossskaliger Strömungen und
Wellenprozesse.

Unter vereinfachten Bedingungen, barotrope, reibungsfreie Atmosphäre mit
konstantem Luftvolumen, lautet die Definition:
\begin{equation}
	q = \frac{\zeta + f}{H},
	\label{rossby:eq:potential_vorticity}
\end{equation}
wobei \(\zeta\) die relative Vorticity, \(f\) den Coriolis-Parameter und \(H\) die effektive Schichthöhe der betrachteten Luftsäule bezeichnet.
Dieser Ausdruck berücksichtigt sowohl die horizontale Rotation als auch die vertikale Ausdehnung eines Luftpakets.



\begin{figure}
	\centering
	\begin{tikzpicture}[scale=1.05, >=stealth]
		% params
		\def\R{0.35} \def\r{0.12}
		\def\H{3.0}  \def\dH{0.8}
		\def\DX{5.2}

		% guide rays (dashed perspective)
		\draw[dashed] (-0.2,\H) -- (\DX+0.2,\H+\dH);
		\draw[dashed] (-0.2,0)  -- (\DX+0.2,\dH);

		% center cylinder (height H)
		\draw (0,0) ellipse ({\R} and {\r});
		\draw (-\R,0) -- (-\R,\H);
		\draw (\R,0) -- (\R,\H);
		\draw (0,\H) ellipse ({\R} and {\r});
		\node[above] at (0,\H+0.35) {$\zeta$};
		\draw[->] (1.4,1.7) -- (0.7,1.7);

		% right cylinder (H+dH)
		\begin{scope}[xshift=\DX cm]
			\draw (0,\dH) ellipse ({\R} and {\r});
			\draw (-\R,\dH) -- (-\R,\H+\dH);
			\draw (\R,\dH) -- (\R,\H+\dH);
			\draw (0,\H+\dH) ellipse ({\R} and {\r});
			\node[above] at (0,\H+\dH+0.35) {$\zeta+\mathrm{d}\zeta$};
			\node[right] at (\R+0.25,\H/2+\dH) {$H+\mathrm{d}H$};
			\draw[->] (-1.4,1.7+\dH) -- (-0.7,1.7+\dH);
		\end{scope}

		% left cylinder (H-dH)
		\begin{scope}[xshift={- \DX cm}]
			\draw (0,-\dH) ellipse ({\R} and {\r});
			\draw (-\R,-\dH) -- (-\R,\H-\dH);
			\draw (\R,-\dH) -- (\R,\H-\dH);
			\draw (0,\H-\dH) ellipse ({\R} and {\r});
			\node[above] at (0,\H-\dH+0.35) {$\zeta-\mathrm{d}\zeta$};
			\node[left]  at (-\R-0.25,\H/2-\dH) {$H-\mathrm{d}H$};
		\end{scope}

		% spin symbols on tops
		\foreach \x/\y in {- \DX/\H-\dH, 0/\H, \DX/\H+\dH}{
				\begin{scope}[xshift=\x cm]
					\draw (0,\y+0.2) circle (0.15);
					\draw[->] (0.15,\y+0.2) arc (0:300:0.15);
				\end{scope}
			}

		% captions
		\node[align=center] at (-\DX,-1.2)
		{$H\downarrow \Rightarrow |\zeta|\downarrow$\\[2pt]\emph{vortex squashing}\\ horizontal divergence};
		\node[align=center] at (\DX,-1.2)
		{$H\uparrow \Rightarrow |\zeta|\uparrow$\\[2pt]\emph{vortex stretching}\\ horizontal convergence};
	\end{tikzpicture}


	\caption{Schematische Darstellung der Erhaltung der potenziellen Vorticity entlang der Trajektorie eines Luftpakets.}
	\label{fig:pv_conservation}
\end{figure}

Ein zentrales Ergebnis der quasi-geostrophischen Theorie ist die Erhaltung der
potenziellen Vorticity entlang der Trajektorie eines Luftpakets:
\begin{equation}
	\frac{Dq}{Dt} = 0.
	\label{rossby:eq:pv_conservation}
\end{equation}
Unter den genannten Bedingungen treten weder Reibung noch diabatische Wärmeflüsse auf, und die Masse der Luftsäule bleibt erhalten.
Damit heben sich Änderungen in \(\zeta\), \(f\) und \(H\) entlang der Bewegung eines Luftpakets gegenseitig so auf, dass das Verhältnis \((\zeta + f)/H\) konstant bleibt.

Die Erhaltung der potenziellen Vorticity erklärt viele grossskalige atmosphärische Phänomene: Wird ein Luftpaket nach Norden transportiert (höheres \(f\)), so muss es zyklonale
Vorticity (\(\zeta > 0\)) abbauen oder sich ausdehnen (\(H \uparrow\)), um
\(q\) konstant zu halten. Bei einer südlichen Verlagerung gilt der umgekehrte
Mechanismus. Diese Dynamik ist eine der physikalischen Grundlagen für das
Entstehen und die Ausbreitung von Rossby-Wellen, die im folgenden Kapitel
behandelt werden.

    \section{Rossby-Wellen}
    
% % \section{Rossby-Wellen}
% \begin{frame}{Rossby-Wellen auf der Kugel}
%     \begin{columns}
%       \column{0.55\textwidth}
%       \begin{itemize}
%         \item Idealisierte Strömung basierend auf der \textbf{Stromfunktion}:
%         \[
%           \psi(\theta, \phi) = \hat{\psi} \cos(k \phi) \sin(\theta)
%         \]
%         \item Daraus ergibt sich das Geschwindigkeitsfeld über:
%         \[
%           u = -\frac{1}{a} \frac{\partial \psi}{\partial \theta}, \quad
%           v = \frac{1}{a \sin \theta} \frac{\partial \psi}{\partial \phi}
%         \]
%         \item Wellenzahl \( k \), mittlerer Wind \( U_0 \), \\
%               Erdradius \( a \), Beta-Effekt implizit enthalten
%         \item Die Westwärts-Drift ist charakteristisch für Rossby-Wellen
%       \end{itemize}

%       \column{0.45\textwidth}
%       \centering
%       \includegraphics[width=\linewidth]{../images/rossby_wave_plot.png}
%     \end{columns}
%   \end{frame}

%   \begin{frame}{Rossby-Wellen auf der  $\beta$-Ebene}
%     \begin{columns}
%       \column{0.55\textwidth}
%       \begin{itemize}
%         \item Lineare Lösung der barotropen Vorticity-Gleichung auf der $\beta$-Ebene:
%         \[
%           \frac{\partial}{\partial t} \left( \nabla^2 \psi \right) + \beta \frac{\partial \psi}{\partial x} = 0
%         \]
%         \item Wellenansatz für die Stromfunktion:
%         \[
%           \psi(x, y, t) = \hat{\psi} \cos(kx + ly - \omega t)
%         \]
%         \item Dispersionsrelation:
%         \[
%           \omega = -\beta \frac{k}{k^2 + l^2}
%         \]
%         \item Geschwindigkeit:
%         \[
%           u = -\frac{\partial \psi}{\partial y}, \quad
%           v = \frac{\partial \psi}{\partial x}
%         \]
%         \item Charakteristisch: westwärts laufende Phasengeschwindigkeit bei \( \beta > 0 \)
%       \end{itemize}

%       \column{0.45\textwidth}
%       \centering
%       \includegraphics[width=\linewidth]{../images/rossby_wave_beta.png}
%       \scriptsize Darstellung: Windvektoren (Pfeile), Stromfunktion (schwarz), Windgeschwindigkeit (Farbflächen)
%     \end{columns}
%   \end{frame}


\begin{frame}{Rossby-Wellen}
	\begin{itemize}
		\item In Äquatornähe dominiert eine mittlere Ost-West-Strömung \( U \)
		\item Wir betrachten kleine Abweichungen davon:
		      \[
			      u' = U + u, \qquad v' = v \qquad \text{mit } u, v \ll U
		      \]
		\item Die Strömung ist quellenfrei → Stromfunktion \( \psi \) existiert:
		      \[
			      u = -\frac{\partial \psi}{\partial y}, \qquad v = \frac{\partial \psi}{\partial x}
		      \]
	\end{itemize}
\end{frame}

\begin{frame}{Zirkulation und Drehimpuls}
	\begin{itemize}
		\item Relative Vorticity (Zirkulation):
		      \[
			      \zeta = \frac{\partial v}{\partial x} - \frac{\partial u}{\partial y} = \Delta \psi
		      \]
		\item Absolute Vorticity:
		      \[
			      \zeta + f \qquad \text{(mit Coriolisparameter } f = f(y) \text{)}
		      \]
		\item Annahme: Erhaltung der absoluten Vorticity:
		      \[
			      \frac{d}{dt} (\zeta + f) = 0
		      \]
	\end{itemize}
\end{frame}

\begin{frame}{Bewegungsgleichung - Herleitung}
	\begin{itemize}
		\item Kettenregel für totale Ableitung:
		      \[
			      \frac{d}{dt} (\zeta + f)
			      =
			      \frac{\partial \zeta}{\partial t}
			      + (U+u) \frac{\partial \zeta}{\partial x}
			      + v \left( \frac{\partial \zeta}{\partial y} + \frac{\partial f}{\partial y} \right)
		      \]
		\item Näherungen:
		      \begin{itemize}
			      \item \( u \ll U \) → vernachlässigbar
			      \item \( \partial \zeta / \partial y \ll \partial f / \partial y \)
			      \item \( \partial f / \partial y = \beta \)
			      \item \( v = \frac{\partial \psi}{\partial x} \)
		      \end{itemize}
		\item Daraus ergibt sich:
		      \[
			      \frac{\partial \zeta}{\partial t}
			      + U \frac{\partial \zeta}{\partial x}
			      + \beta \frac{\partial \psi}{\partial x} = 0
		      \]
		\item Mit \( \zeta = \Delta \psi \):
		      \[
			      \frac{\partial \Delta \psi}{\partial t}
			      + U \frac{\partial \Delta \psi}{\partial x}
			      + \beta \frac{\partial \psi}{\partial x} = 0
		      \]
	\end{itemize}
\end{frame}

\begin{frame}{Wellenlösung der Gleichung}
	\begin{itemize}
		\item Ansatz: ebene Wellen
		      \[
			      \psi(x, y, t) = \cos(kx + ly - \omega t)
		      \]
		\item Einsetzen in Gleichung ergibt Dispersionsrelation:
		      \[
			      \omega = Uk - \frac{\beta k}{k^2 + l^2}
		      \]
		\item Phasengeschwindigkeit:
		      \[
			      c = \frac{\omega}{k} = U - \frac{\beta}{k^2 + l^2}
		      \]
		\item Interpretation: westwärts laufende Wellen mit geringer Geschwindigkeit als \( U \)
	\end{itemize}
\end{frame}

% \begin{frame}{Physikalisches Feld am Beispiel der Vorticity}
%   \begin{itemize}
%       \item Die \textbf{Vorticity} \( \zeta(x, y, t) \) beschreibt die Rotation eines Luftpakets.
      
%       \item Sie ist ein \textbf{physikalisches Skalarfeld}, das jedem Punkt eine Wirbelstärke zuordnet:
%       \[
%           \zeta = \frac{\partial v}{\partial x} - \frac{\partial u}{\partial y}
%       \]
      
%       \item In der Atmosphäre entsteht Vorticity durch:
%       \begin{itemize}
%           \item Wind-Scherung (Änderung der Windrichtung oder -geschwindigkeit)
%           \item Bewegung entlang der Breitenkreise (\( \beta \)-Effekt)
%       \end{itemize}
      
%       \item Die \textbf{Vorticity-Gleichung} beschreibt, wie sich dieses Feld entwickelt:
%       \[
%           \frac{\partial \zeta}{\partial t} + \vec{v} \cdot \nabla \zeta + \beta v = 0
%       \]
      
%       \item → Diese Gleichung ist eine typische \textbf{Feldgleichung}, weil sie die Änderung eines Feldes durch lokale und advektive Prozesse beschreibt.
%   \end{itemize}
%   \end{frame}
  


\section{Fazit}

Von den grundlegenden Kräften in einem rotierenden Bezugssystem über das Konzept der relativen und potenziellen Vorticity bis hin zur Dynamik von Rossby-Wellen spannt sich ein klarer Bogen:
Die Coriolis-Kraft und die Erhaltung der potenziellen Vorticity liefern die physikalische Grundlage, um grossräumige atmosphärische Strömungen zu verstehen.
Rossby-Wellen sind ein direktes Resultat dieser Prinzipien und prägen massgeblich die Wetter- und Klimadynamik der mittleren Breiten.

Ihre Fähigkeit, über Tage bis Wochen stabile Muster zu bilden, macht sie zu einem zentralen Faktor für die Entwicklung von Hoch- und Tiefdruckgebieten - und damit auch für das Auftreten von Extremereignissen, wie das Beispiel des Sommers 2010 eindrücklich zeigt.
Ein vertieftes Verständnis dieser Prozesse ist nicht nur für die Wettervorhersage, sondern auch für die Einschätzung künftiger Klimarisiken von Bedeutung.


\printbibliography[heading=subbibliography]
\end{refsection}
