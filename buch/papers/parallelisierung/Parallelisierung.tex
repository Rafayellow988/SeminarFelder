%
% teil2.tex -- Beispiel-File für teil2 
%
% (c) 2020 Prof Dr Andreas Müller, Hochschule Rapperswil
%
% !TEX root = ../../buch.tex
% !TEX encoding = UTF-8
%
\section{Parallelisierung
\label{parallelisierung:sec:Parallelisierung}}
\kopfrechts{Parallelisierung}
Wir kennen nun numerische Verfahren, welche es einer Maschine erlaubt unsere Feldgleichungen zu berechnen.
Wir wissen auch, wie wir unser Feld in Teilgebiete unterteilen können.
Das gibt uns alle Werkzeuge, welche wir benötigen um die Berechnungen unserer Gleichungen auf mehrere Maschinen aufzuteilen.

Bei der Parallelisierung von Feldgleichungen in einem Multiprozessorsystem wird jedem Prozessor eines oder mehrere zusammenhängende Teilgebiete zugewiesen.
Die Teilgebiete müssen theoretisch nicht zusammenhängend sein aber um die Kommunikation zwischen Prozessoren zu minimieren ist dieses Vorgehen sinnvoll und üblich.
Da die Randzellen jedes Teilgebietes abhängig von den angrenzenden Zellen sind, siehe \ref{parallelisierung:sec:Gebietsunterteilung}, müssen die vorhandenen Informationen dieser Zellen zwischen Gebieten auf unterschiedlichen Prozessoren ausgetauscht werden.
Dieser Austausch ist mit erheblichem zeitlichen Aufwand verbunden.
Die Frage, wie fein ein Gitter in verschiedene Teilgebiete aufgeteilt werden soll, ist also immer eine optimierungs-Frage.
Man muss ein Optimum finden, wo der zeitliche Gewinn der parallelen Berechnung die zeitlichen Verluste der Interprozess-Kommunikation überwiegen.

%
% einleitung.tex -- Beispiel-File für die Einleitung
%
% (c) 2020 Prof Dr Andreas Müller, Hochschule Rapperswil
%
% !TEX root = ../../buch.tex
% !TEX encoding = UTF-8
%
\subsection{Kommunikation zwischen Gebieten
\label{parallelisierung:sub:Interprozess}}
Zuerst eine kurze Rekapitulation zu einigen Begrifflichkeiten, welche wichtig sind für die Welt der parallelen Programmierung.
Ein Prozessor ist eine Physische Recheneinheit wie zum Beispiel eine CPU oder GPU.
Prozessoren sind meist aus mehreren Kernen aufgebaut.
Jeder Kern ist eine eigenständige Recheneinheit und ist mit privatem wie auch mit anderen Kernen geteiltem Speicher verbunden.
Ein Prozess ist ein eigenständiges Programm, welches vom Betriebssystem alle nötigen Resourcen wie Speicher, Rechenzeit und Daten zugeteilt bekommt.
Threads sind Teilaufgaben eines Prozesses, die gleichzeitig auf den verschiedenen Kernen des Prozessors arbeiten und sich die Daten und den Speicher dieses Prozesses teilen.

Für das Austauschen von Daten zwischen Teilgebieten gibt es zwei häufig verwendete Parallele Programmier- und Kommunikationsschnittstellen, OpenMP und MPI.
Im folgen werden beide kurz vorgestellt.

\subsubsection{OpenMP}
OpenMP ist eine API, welche sehr einfach das erzeugen von Threads und die Kommunikation zwischen diesen ermöglicht.
Jedes Teilgebiet wird hier einem Thread zugeteilt. 
Da sich diese Threads den Speicher des Prozesses teilen, nutzt OpenMP Shared-Memory für den Austausch von Daten.
Bei OpenMP gibt es keinen Thread-eigenen Speicher.
Jeder Thread kann also auf die Randzellen anderer Threads zugreifen.
Dadurch benötigt es keine Kommunikation zwischen den Teilgebieten im engeren Sinn.

Für kleine Systeme ist OpenMP eine einfache und gute Lösung.
Da es aber auf Shared-Memory basiert ist es schlecht skalierbar.
Werden für grosse Berechnungen mehrere Prozessoren verwendet ist Shared-Memory oft nicht verfügbar.
Der einfache Aufbau für den Programmierer bedeutet außerdem weniger Kontrolle über die Datenverteilung.

\subsubsection{MPI}

\input{papers/parallelisierung/Synchronisation.tex}

\subsection{Beispiel an der allgemeinen Wärmeleitungsgleichung
\label{parallelisierung:sub:BeispielParallelisierung}}
Zum besseren Verständnis wollen wir mit OpenMP ein Konkretes Beispiel für die Parallelisierung der allgemeinen Wärmeleitungsgleichung realisieren.
Zum Vergleich wird auch die Serielle Vorgehensweise präsentiert.
Als Ausgangslage soll für ein Gitter von AxB Zellen C Updates durchgeführt werden.
Für jedes Update wird die Formel \ref{parallelisierung:eq:update_formel} aus Abschnitt \ref{parallelisierung:sec:update_formel} verwendet.
Links und Rechts der Zellen gibt es Reihen mit unveränderlicher Temperatur, welche einen Heiz- respektive Kühlkörper darstellen.
Ober- und unterhalb der Zellen gibt es Zeilen mit unveränderlicher Temperatur, welche die Umgebungstemperatur darstellen.
Diese Konstanten, wie auch $\lambda$ als Konstante für die Updateformel können alle im Main-File des Programms definiert werden und sind unveränderlich.

Beide Lösungen verwenden Arrays zur Zwischenspeicherung der Zellen.
Für Arrays werden gerne Vektoren aus der C++ Standar Library (STL) verwendet.
Für ein zweidimensionales Array wird zum Beispiel
\begin{lstlisting}
	std::vector<std::vector<double>> grid;
\end{lstlisting}
benützt.
Man muss sich allerdings bewusst sein, dass diese Variante keinen kontinuierlichen Bereich im Speicher reserviert.
Diese Variante speichert zuerst ein äußeres Array (Zeilen), und jede Zeile allokiert separat ihren eigenen Speicher.
Das verschlechtert enorm die Cache-Lokalität des Programms und damit die Performanz.
Als elegantere Lösung verwendet man bei numerischen Verfahren für partielle Differentialgleichungen eindimensionale Arrays und rechnet die Indizes mittels inline-Funktion um.
\begin{lstlisting}
	std::vector<double> grid(nx * ny);
	
	inline double& at(int i, int j) {
		return grid[i * nx + j];  // Row-major
	}
\end{lstlisting}
Damit garantiert man eine kontinuierliche Speicherung der Daten.

\subsubsection{Serielle Lösung}
\label{parallelisierung:sub:serLoesung}}
Zur Vorbereitung werden Zwei Arrays der Grösse 
\begin{equation}
	A+2 * B+2 
\end{equation}
angelegt. 

\begin{lstlisting}
	std::vector<double> grid(A * B);
	std::vector<double> gridNew(A * B);
\end{lstlisting}
Die Addition von zwei ist notwendig, um die Umgebenden Temperaturen speichern zu können.
In beiden Arrays wird nun die Umgebungstemperatur in die erste und letzte Zeile geschrieben.
Die jeweilige Temperatur der Heiz-/Kühlkörper wird in die erste und letzte Spalte geschrieben.
Das erste Array wird für die aktuellen Temperaturen der Zellen verwendet.
Als Ausgangspunkt kann in dieses die Umgebungstemperatur eingetragen werden.
Mit einer Schlaufe kann nun für alle A*B Zellen der neue Wert mit der Updateformel berechnet und in die entsprechende Zelle im zweiten Array geschrieben werden.
\begin{lstlisting}[caption={Update-Schritt (seriell)},label={parallelisierung:code:updateSeriel}]
	for (int i = 1; i < B - 1; ++i) {     // ueberspringt erste und letzte Zeile
		const int row = i * A;
		for (int j = 1; j < A - 1; ++j) { // ueberspringt erste und letzte Spalte
			const int idx   = row + j;
			const int up    = idx - A;
			const int down  = idx + A;
			const int left  = idx - 1;
			const int right = idx + 1;
			
			nextTemp[idx] = currentTemp[idx]
			+ lambda * (src[up] + src[down] + src[left] + src[right]
			- 4.0 * src[idx]);
		}
	}
\end{lstlisting}
Im Beispiel ist src ein Pointer auf das Array mit den aktuellen Werten und dst ist ein Pointer auf das Array mit den neuen Werten.
Bei der Verwendung von Vektoren erhält man die Adresse des ersten Eintrags mit der Funktion .data().
\begin{lstlisting}
	double* src = grid.data();
	double* dst = gridNew.data();
\end{lstlisting}

Ist die For-Schleife durchgerechnet, kann man die beiden Array elegant austauschen, indem man die beiden Pointer tauscht.
\begin{lstlisting}
	std::swap(src, dst);
\end{lstlisting}
Auf diese weise müssen keine Daten Verschoben werden was viel Zeit einspart.
Dieses Verfahren nennt man passend einen Ping-Pong-Ansatz.

In einer Schlaufe die C mal durchlaufen wird, und den Updatecode sowie den Swap aufruft kann das Array C mal aktualisiert und damit das fortschreiten der Zeit simuliert werden.

\subsubsection{parallele Lösung 
\label{parallelisierung:sub:parLoesung}}
Die Vorgehensweise bei der parallelen Lösung ist sehr ähnlich.
Es werden zwei Arrays verwendet, welche im Shared-Memory des Prozesses liegen.
Jeder Thread kann auf das gesamte Array mit den aktuellen Werten zugreifen und kann daher auch die Randzellen problemlos berechnen.
Jedem Thread wird ein Teilgebiet des gesamten Arrays zugewiesen.
Der Thread berechnet darin jede Zelle mit den vorhandenen aktuellen Werten und schreibt das Ergebnis an der entsprechenden Stelle in das neue Array.
Da auf dem alten Array keine Schreibaktionen getätigt werden gibt es keine Möglichkeit für Race conditions.
Zur Synchronisation wird am ende der Updateschlaufe eine Barriere gesetzt, welche den Thread am weitermachen hindert, bis jeder andere Thread seine Arbeit verrichtet hat.
Wird diese Barriere überschritten, hat ein Thread die Aufgabe, die Pointer der Arrays auszutauschen.
Eine zweite Barriere sorgt dafür, dass die anderen Threads warten, bis dieser Pointer-Swap durchgeführt wurde.

