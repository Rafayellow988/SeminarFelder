%
% einleitung.tex -- Beispiel-File für die Einleitung
%
% (c) 2020 Prof Dr Andreas Müller, Hochschule Rapperswil
%
% !TEX root = ../../buch.tex
% !TEX encoding = UTF-8
%
\section{Grundlagen Fourier\label{fourier:section:teil0}}
\kopfrechts{Teil 0}


Grundlagen der Fourier-Analyse später Titel

Die Fourier-Analyse ist ein sehr mächtiges Mittel in der Signal-Analyse. 
Mit der Fourier-Reihe lassen sich periodisch wiederholende Signale, wie ein Rechteck- oder Dreiecksignal, mit skalierten Sinus- und Kosinus-Schwingungen darstellen.
Um die Reihe aufzustellen, braucht man nur 3 Koeffizienten zu bestimmen.

\[
f(x) = a_0 + \sum_{n=1}^{\infty} \left( a_n \cos\left( \frac{2\pi n}{T} x \right) + b_n \sin\left( \frac{2\pi n}{T} x \right) \right)
\]

Eine grafische Darstellung der Reihe...

\subsection{Fourierreihe\label{fourier:subsection:fourierreihe}}


\subsection{Fouriertransformation\label{fourier:subsection:fouriertransformation}}

