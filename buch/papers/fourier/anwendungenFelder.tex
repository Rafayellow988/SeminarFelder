%
% anwendungenFelder.tex -- 3.	Anwendungen auf Felder, PDEs in ODE umwandeln (Grundlage, um Feldgleichungen als eine Familie von Oszillator-Gleichungen zu interpretieren)
%
% (c) 2020 Prof Dr Andreas Müller, Hochschule Rapperswil
%
% !TEX root = ../../buch.tex
% !TEX encoding = UTF-8
%

%    Anwendungen auf Felder, PDEs in ODE umwandeln (Grundlage, um Feldgleichungen als eine Familie von Oszillator-Gleichungen zu interpretieren) 


\section{Anwendungen auf Felder\label{fourier:section:teil0}}

\kopfrechts{Anwendungen auf Felder}
%Einleitung, warum man Fourier Transformation zur Vereinfachung von PDEs braucht.
%Verbindung zur Feldtheorie. Feldgleichungen sind ansammlungen von Oszillator-Gleichungen

\subsection{PDEs zu ODEs\label{fourier:subsection:anwendungenFelder}}

%wie man diese umwandlung macht. über den Frequenzraum
%Wellengleichung als Beispiel

ODE steht für ordinary differential equation, diese Gleichungen sind berühmt berüchtigt schwierig zu lösen.
Schlimmer geht es immer.
PDE steht für partial differential equantion. Sie hat enthaltet ableitungen nach mehreren Variabeln.
In diesem Abschnitt vereinfachen wir eine PDE in eine ODE, mithilfe der Fourierreihe.
Alle Ableitungen sollten schlussendlich den identischen Nenner besitzen.
Als PDE verwenden wir die Wellengleichung.


\begin{equation}
	\frac{\partial^2 u(x, t)}{\partial t^2} = c^2 \cdot \frac{\partial^2 u(x, t)}{\partial x^2}
\end{equation}

Wir nehmen an $u(x, t)$ ist eine sich periodisch wiederholende Funktion. 
Wir führen eine Fourierreihen-Entwicklung durch. 

\begin{equation}
	u(x,t) = \frac{a_0}{2} + \sum_{n=1}^{\infty} \left( a_n(t) \cos(n \omega t) + b_n(t) \sin(n \omega t) \right)
\end{equation}

Die Fourier Koeffizienten $a_n(t)$ und $b_n(t)$ sind neu von der Zeit abhängig.
Nun wird $u(x,t)$ in seiner Summenform in die Wellengleichung eingebaut.
Die zweite Ableitungen von $u(x,t)$ nach der Zeit und nach x lauten:

\begin{equation}
	\frac{\partial^2 u(x,t)}{\partial t^2} = \sum_{n=1}^{\infty} \left( -a_n n^2 \omega^2 \cos(n \omega t) - b_n n^2 \omega^2 \sin(n \omega t) \right)
\end{equation}

\begin{equation}
	\frac{\partial^2 u(x,t)}{\partial x^2} = \sum_{n=1}^{\infty} \left( a_n''(x) \cos(n \omega t) + b_n''(x) \sin(n \omega t) \right)
\end{equation}

Die Resultate schlussendlich einsetzen.

\begin{equation}
	\sum_{n=1}^{\infty} \left( -a_n n^2 \omega^2 \cos(n \omega t) - b_n n^2 \omega^2 \sin(n \omega t) \right) = c^2 \sum_{n=1}^{\infty} \left( a_n''(x) \cos(n \omega t) + b_n''(x) \sin(n \omega t) \right)
\end{equation}

Die Beiden Summenzeichen kürzen sich raus. Mithilfe dem Koeffizientenvergleich, findet man zwei lösungen:

\begin{equation}
	-a_n n^2 \omega^2 = c^2 a_n''(x) \quad \text{und} \quad -b_n n^2 \omega^2 = c^2 b_n''(x)
\end{equation}


