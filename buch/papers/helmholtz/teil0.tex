%
% einleitung.tex -- Beispiel-File für die Einleitung
%
% (c) 2020 Prof Dr Andreas Müller, Hochschule Rapperswil
%
% !TEX root = ../../buch.tex
% !TEX encoding = UTF-8
%
\section{Einleitung\label{helmholtz:section:teil0}}
\kopfrechts{Teil 0}

\subsection{Mathematische Tools}

Referenzen auf Buchkapitel von Hr. Müller. einfügen.

Der Gradient eines Skalarfeldes $a$ lautet:
\begin{equation}
\nabla a(\mathbf{r}) = \frac{\partial a}{\partial x}\mathbf{e}_x + \frac{\partial a}{\partial y}\mathbf{e}_y + \frac{\partial a}{\partial z}\mathbf{e}_z
\end{equation}

Das Gradient (wie beschrieben in  \eqref{buch:kurvenintegral:differential:eqn:ricthungsableitung} ) wirkt auf ein Skalarfeld und liefert ein Vektorfeld, das in Richtung des steilsten Anstiegs von $a$ zeigt. Der Gradient an einem Punkt ergibt die Richtung maximaler Zunahme des Skalars an. Der Betrag entspricht der Steigung in der entsprechenden Richtung. \newline

% https://www.elektroniktutor.de/fachmathematik/nabla.html#:~:text=Richtung%20ihr%20Maximum,die%20Gr%C3%B6%C3%9Fe%20der%20Steigung%20an

Die Divergenz eines Vektorfeldes $\mathbf{A}$:
\begin{equation}
\nabla \cdot \mathbf{A}(\mathbf{a}) = \frac{\partial A_x}{\partial x} + \frac{\partial A_y}{\partial y} + \frac{\partial A_z}{\partial z}
\end{equation}

Die Divergenzmiss für ein bestimmtes Vektorfeld $\mathbf{A}(\mathbf{r})$ die ?QUellendichte? , was bedeutet wie stark die Feldlinien auseinanderstreben (positive DIvergenz) oder zusammenlaufen (negative Divergenz). \newline

%https://www.elektroniktutor.de/fachmathematik/nabla.html#:~:text=Wird%20der%20Differenzialoperator%20Nabla%20skalar,Ist

Die Rotation eines Vektorfeldes:
\begin{equation}
\nabla \times \mathbf{A}(\mathbf{r}) = \begin{vmatrix}
	\mathbf{e}_x & \mathbf{e}_y & \mathbf{e}_z \\
	\frac{\partial}{\partial x} & \frac{\partial}{\partial y} & \frac{\partial}{\partial z}\\
	A_x & A_y & A_z
\end{vmatrix}
\end{equation}


Die Rotation misst die Wirbeldichte eines Vektorfeldes $A$. Gleichtbedeutend mit wie stark sich die Feldlinien, um eine Achse drehen. Ein nichtverschwindendes $\nabla\times \mathbf{A}$ bedeutet, dass im Feld, Wirbel vorhanden sind: lokale Umlaufströmung. Insbesondere gibt $\nabla\times \mathbf{A}$ die Achsen-Ortientierung und Stärke eines solchen Wirbels an. Ist $\nabla\times \mathbf{A}=0$, so ist das Feld wirbelfrei (irrotational) \newline
% https://www.elektroniktutor.de/fachmathematik/nabla.html#:~:text=Kr%C3%A4ftefeld%2C%20so%20wird%20entlang%20eines,gilt%20das%20Feld%20als%20wirbelfrei


Laplace-Operator eines Skalarfeldes:
\begin{equation}
\nabla^2 a(\mathbf{r}) = \frac{\partial^2 a}{\partial x^2} + \frac{\partial^2 a}{\partial y^2} + \frac{\partial^2 a}{\partial z^2}
\end{equation}


\subsection{Physikalische Grundlage}

Die Schallintensität oder auch Akustische Intensität ist die Multiplikation von des Schalldrucks $P$ mit momentanen Schallschnelle (Teilchengeschwindigkeit) $v$ und erhält den Leistungsfluss pro Fläche $=$ momentane Intensität $\mathbf{I_i~(t)}$. Oder kurz:

\begin{equation}
I_i ~(t) = P(t) \cdot v(t)
\label{helmholtz:equationIntensitaetDef}
\end{equation}



\begin{itemize}
	\item komplexe akustische Intensität: \\
	\begin{equation}
	\mathbf{I}_c ~(\mathbf{r}) = \mathbf{I}(\mathbf{r}) + j\,\mathbf{Q}(\mathbf{r}),
	\label{helmholtz:equationIntensitaetComplex}
	\end{equation}	
	
	\item Momentan-Intensität \\
	\begin{equation}
	\mathbf{I}_i ~(\mathbf{r},t) = p(\mathbf{r},t)~\mathbf{u}(\mathbf{r},t)
	\label{helmholtz:equationIntensitaetMomentan}
	\end{equation}
			
	\item Zeitlich gemittelte Intensität (aktive Intensität) \\
	\begin{equation}
	\mathbf{I}~(\mathbf{r}) = \frac{1}{T}\int_0^T \mathbf{I}_i(\mathbf{r},t)\,~\mathrm{d}t = \frac{1}{2}\Re\left\{P(\mathbf{r})~\mathbf{U}^*(\mathbf{r})\right\}
	\end{equation}
	
	\item aktive Intensität: \\
	\begin{equation}
	\mathbf{I} = \frac{P^2}{2P_0\omega}
	\label{helmholtz:equationAktiveIntensitaet}
	\end{equation}
	
	\item Reaktive Intensität: \\
	\begin{equation}
	\mathbf{Q}(\mathbf{r}) = \frac{1}{2}\Im\left\{P(\mathbf{r})~\mathbf{U}^*(\mathbf{r})\right\}
	\label{helmholtz:equationReaktiveIntensitaet}
	\end{equation}
	
\end{itemize}
