%
% kugel.tex -- Kugelkoordinaten
%
% (c) 2021 Prof Dr Andreas Müller, OST Ostschweizer Fachhochschule
%
\bgroup
\begin{frame}[t]
\setlength{\abovedisplayskip}{5pt}
\setlength{\belowdisplayskip}{5pt}
\frametitle{Kugelkoordinaten $(\vartheta,\varphi)$}
\vspace{-20pt}
\begin{columns}[t,onlytextwidth]
\begin{column}{0.58\textwidth}
\begin{enumerate}
\item<2->
Metrik: $\displaystyle
g_{ik}
=
\begin{pmatrix}
1&0\\
0&\sin^2\vartheta
\end{pmatrix}
$
\item<3->
Inverse: $\displaystyle
g^{ik}
=
\begin{pmatrix}
1&0\\
0&\sin^{-2}\vartheta
\end{pmatrix}
$
\item<4->
Ableitungen: $\displaystyle
\frac{\partial g_{22}}{\partial x^1}
=
\frac{\partial g_{22}}{\partial\vartheta}
=
2\sin\vartheta\cos\vartheta.
$
\item<5->
Christoffel-Symbole erster Art:
\[
\begin{aligned}
\Gamma_{1,11}
&=
0,
&
\Gamma_{1,12} = \Gamma_{1,21}
&=
0,
&
\Gamma_{1,22}
&=
-\sin\vartheta\cos\vartheta,
\\
\Gamma_{2,11}
&=
0,
&
\Gamma_{2,12} = \Gamma_{2,21}
&=
\sin\vartheta\cos\vartheta,
&
\Gamma_{2,22}
&=
0.
\end{aligned}
\]
\item<6->
Christoffel-Symbole zweiter Art:
\[
\begin{aligned}
\Gamma^1_{11}
&=
0,
&
\Gamma^1_{12}
=
\Gamma^1_{21}
&=
0,
&
\Gamma^1_{22}
&=
-\sin\vartheta\cos\vartheta,
\\
\Gamma^2_{11}
&=
0,
&
\Gamma^2_{12}
=
\Gamma^2_{21}
&=
\cot\vartheta,
&
\Gamma^2_{22}
&=
0.
\end{aligned}
\]

\end{enumerate}
\end{column}
\begin{column}{0.38\textwidth}
\uncover<7->{%
\begin{block}{Geodätengleichung}
\begin{align*}
\uncover<8->{
\ddot{\vartheta}
&=
\sin\vartheta\cos\vartheta \cdot \dot{\varphi}^2
}
\\
\uncover<9->{
\ddot{\varphi}
&=
-\cot\vartheta \cdot \dot{\vartheta}\dot{\varphi}.
}
\end{align*}
\uncover<10->{Lösungen sind Grosskreise.}
\end{block}}
\end{column}
\end{columns}
\end{frame}
\egroup
