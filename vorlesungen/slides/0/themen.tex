%
% themen.tex -- slide template
%
% (c) 2021 Prof Dr Andreas Müller, OST Ostschweizer Fachhochschule
%
\bgroup
\begin{frame}[t]
\setlength{\abovedisplayskip}{5pt}
\setlength{\belowdisplayskip}{5pt}
\frametitle{Themen}
\vspace{-20pt}
\begin{columns}[t,onlytextwidth]
\only<1-21>{
\begin{column}{0.48\textwidth}
\begin{enumerate}
\item<2-> Phasenportraits in 2D
\item<3-> Geometrische Algebra
\item<4-> Elastizitätstheorie
\item<5-> Platten und Schalen
\item<6-> Geostrophische Näherung
\item<7-> Rossby-Wellen
\item<8-> Kelvin-Wellen
\item<9-> Reynolds-averaging
\item<10-> Wirbelringe
\item<11-> Überschallströmung
\item<12-> Joukowski-Formel
\end{enumerate}
\end{column}
}
\only<13-28>{
\begin{column}{0.48\textwidth}
\begin{enumerate}
\setcounter{enumi}{11}
\item<13-> 4D Maxwell
\item<14-> Eichtransformationen
\item<15-> Fourier-Transformation und Felder
\item<16-> DiffOp in orthogonalen Koordinaten
\item<17-> Turing-Muster
\item<18-> Nervenfasern
\item<19-> Burgers-Gleichung
\item<20-> OpenFOAM
\item<21-> Gebiet und Parallelisierung
\end{enumerate}
\end{column}
}
\only<22->{%
\begin{column}{0.48\textwidth}
\begin{enumerate}
\setcounter{enumi}{21}
\item<22-> MSE: Poincaré-Bendixson
\item<23-> MSE: Nullklinen in 3D
\item<24-> MSE: Reaktion-Diffusion
\item<25-> MSE: Monge-Ampère
\item<26-> MSE: Monge-Kantorovitch-Transport
\item<27-> MSE: Helmholtz-Zerlegung
\item<28-> MSE: Poincaré-Lemma für Wirbelfelder
\item<29-> MSE: Kotangentialbündel
\item<30-> MSE: Schallfeld
\end{enumerate}
\end{column}
}
%\only<29->{%
%\begin{column}{0.48\textwidth}
%\begin{enumerate}
%\setcounter{enumi}{27}
%\end{enumerate}
%\end{column}
%}
\end{columns}
\end{frame}
\egroup
